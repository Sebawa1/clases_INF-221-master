\documentclass[english, spanish, fleqn, 10pt]{article}
\usepackage[top = 2.5cm, bottom = 2cm, left = 2.5cm, right = 2.5cm]{geometry}
\usepackage[es-noindentfirst]{babel}
\usepackage{amsthm, amsmath, amssymb, amsfonts, environ}
\usepackage{caption}
\usepackage{subcaption} % para las subfigures
\usepackage{mathrsfs}
\usepackage[colorlinks, urlcolor=blue,
pdfauthor={Aldo Berrios Valenzuela}]{hyperref}
\usepackage{fourier}
\usepackage{colortbl}
\usepackage{color}
\usepackage{graphicx}
\usepackage{enumitem}
%\renewcommand{\familydefault}{\sfdefault}
%\setlength{\parindent}{0pt}					%Espacio de sangria
\setlength{\parskip}{0.5mm}						%Interlinado entre párrafos
\usepackage{fancyhdr, blindtext}		%Paquete de encabezado
\usepackage{listings}
\usepackage[framemethod=tikz]{mdframed}

\definecolor{webred}{rgb}{0.75,0,0}
\definecolor{mblue}{rgb}{0.0705,0.345,0.7529}
\definecolor{newmblue}{HTML}{004D99}
\usepackage{longtable, colortbl, array}

\definecolor{superGreenNew}{HTML}{169F01}
\hypersetup{
        colorlinks   = true,
        citecolor    = superGreenNew
}


%==============================
%Modificador de Titulos de secciones, subsecciones, etc
\usepackage[explicit, noindentafter]{titlesec}

\titlespacing\subsubsection{0pt}{8pt plus 4pt minus 2pt}{4pt plus 2pt minus 2pt}


%==================================
%Diseno para insertar codigos de programacion
\definecolor{newGray}{HTML}{F8F8F8}
\definecolor{anotherGray}{HTML}{DDDFFF}
\lstdefinestyle{customc}{
        belowcaptionskip=1\baselineskip,
        breaklines=true,
        backgroundcolor=\color{newGray},
        frame=single,
        rulecolor=\color{anotherGray},
        xleftmargin=\parindent,
        language=bash,
        showstringspaces=false,
        basicstyle=\small\ttfamily,
        keywordstyle=\bfseries\color{green!40!black},
        commentstyle=\itshape\color{purple!40!black},
        identifierstyle=\color{black},
        stringstyle=\color{mblue},
        tabsize=2,
        literate={\ \ }{{\ }}1
}
\lstset{style=customc}

%==================================
%Macros útiles
\newcommand{\comillas}[1]{``#1''}
\newcommand{\comentarioc}[1]{\texttt{\textcolor{webred}{/* #1 */}}}

\numberwithin{equation}{section}

%=================================
%comandos matemáticos personalizados de rápido acceso
\newcommand{\nparentesis}[1]{\left( #1 \right)}
\newcommand{\llaves}[1]{\left \{ #1 \right \}}
\newcommand{\nabsoluto}[1]{\left| #1 \right|}
\newcommand{\ncorchetes}[1]{\left[ #1 \right]}
%=================================

%cambia el espaciado entre el parrafo y antes del itemize
\setlist[itemize]{itemsep=1mm, topsep=1.5mm}
\setlist[enumerate]{itemsep=1mm, topsep=1.5mm}
%===========================

\theoremstyle{definition}
\newtheorem{teorema}{Teorema}[section]
\newtheorem{definition}{Definición}[section]
\newtheorem{beforeExample}{Ejemplo}[section]
\newtheorem{preposicion}{Preposición}[section]
\newtheorem{corolario}{Corolario}[teorema]

\captionsetup{justification=centering, margin=2.3cm}

\newenvironment{ejemplo}[1][]{\begin{beforeExample}[#1]\renewcommand{\qedsymbol}{$\blacksquare$}}{\qed\end{beforeExample}}

\captionsetup{justification=centering, margin=2.3cm}

% Para poder hacer "quote" con estilo github

\newenvironment{newquote}{\begin{mdframed}[topline=false,
                rightline=false, bottomline=false, linewidth=.3em,backgroundcolor=black!5, linecolor=black!60]\color{black!70}}{\end{mdframed}}
%================================

\renewcommand*{\arraystretch}{1.2}

\usepackage{fancyhdr}

\pagestyle{fancy}
\lhead{<inserte nombre del ramo>}
\rhead{<inserte contenido>}

\title{INF221 -- Algoritmos y Complejidad\\[.4\baselineskip]Ayudantía 6\\Programación dinámica}
\author{Aldo Berrios Valenzuela}
\date{Jueves 22 de septiembre de 2016}

\begin{document}
\maketitle
\section{Ayudantía}
\begin{center}
        \comentarioc{Dibujo de un grafo}
\end{center}
Demostración de algoritmo de programación dinámica:
\begin{itemize}
        \item \textbf{Complex choice:} Sea $\hat p_i$ $\forall i$ con $i=1, \ldots, n$, las posibles primeras elecciones del algoritmo para el problema $P$, entonces la solución de $P$ incluye a algún elemento de $\hat p_i$. Para el ejemplo del grafo, se tiene que $\hat p_i=\llaves{1, 5, 100}$.

        \item \textbf{Inductive substructure:} Sea $\Pi_i'$ la solución del problema $P_i'$ que se genera al escoger $\hat p_i$. Se debe cumplir que $\hat p_i\cup \Pi'_i$ es solución factible de $P$.

        \item \textbf{Optima substructure:} Sean $\Pi'_i$ las soluciones óptimas de los problemas $P_i'$ al escoger $\hat p_i$ dado, entonces algún $\Pi_i'\cup\hat p_i$ es óptima de $P$.
\end{itemize}

\begin{ejemplo}
        \comentarioc{Descripcion de algoritmo}
        \begin{center}
                \comentarioc{Dibujo parecido al grafo de pierola}
        \end{center}
        Demuestre que es algoritmo de programación dinámica.
\end{ejemplo}
\begin{proof}
        Demostramos las cosas:
        \begin{itemize}
                \item Complex choice: Sea $C$ el conjunto de los $\hat p _i$ para el problema. Sea $n_i$ el nodo inicial. Entonces $n_i$ y todo $n_j$  conecta $n_i\in C$.

                Por contradicción: supongamos que existe una solución $\Pi^*$ tal que
                \begin{equation*}
                C\cap \Pi^* = \emptyset
                \end{equation*}
                Entonces $\nexists n_k \in \Pi ^* $ que esté conectado a $n$. Entonces $n_i\cup \Pi ^*$ es solución factible de $P$. Luego
                \begin{equation*}
                \nabsoluto{n_i\cup \Pi^*} > \nabsoluto{\Pi ^*}
                \end{equation*}
                por lo tanto, $\Pi ^*$ no es óptimo. $\Rightarrow\Leftarrow$. Por lo tanto, algún $n_j\in C$ forma parte de la solución óptima.

                \item \emph{Subestructura inductiva:} Sea $\Pi_i$ una solución de $p_i'$ que es el subproblema de $P$ que se genera al escoger $\hat p_i$. Como $\hat p_i$ es un nodo escogido para la solución y el algoritmo quita todos los nodos conectados a $\hat p_i$, entonces el problema $P'_i$  es un grafo no conectado a $\hat p_i$, por lo tanto ningún nodo $n_k\in \Pi'_i$ está conectado a $\hat p_i$, por lo que siempre es posible unir $\hat p_i$ con $\Pi'_i$ para generar una solución para $P$. Entonces cumple para todo $\hat p_i$.

                \item Optimal substructure: Sea $\Pi_i'$ la solución óptima para el subproblema $P_i'$ que se genera al escoger un $\hat p_i$. Como $\Pi_i'$ es la mejor solución y al unirla a $\hat p_i$ genera una solución aún mejor, entonces, algún $\Pi_i'\cup \hat p_i$ debe ser la mejor solución para $P$, dado que los $\Pi_i'\cup \hat p_i$ representa las mejores soluciones para $\hat p_i$ dado.
        \end{itemize}
\end{proof}


\end{document}
% LocalWords:  Complex choice Inductive substructure Optima pierola
% LocalWords:  Descripcion Optimal
