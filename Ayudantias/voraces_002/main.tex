\documentclass[spanish, fleqn]{article}
\usepackage{babel}
\usepackage{fourier}
\usepackage{amsmath, amsfonts, amsthm, fourier}
\usepackage{float}

\usepackage{graphicx}

\usepackage[colorlinks, urlcolor=blue]{hyperref}

\usepackage{mathtools}
\DeclarePairedDelimiter{\ceil}{\lceil}{\rceil}
\DeclarePairedDelimiter{\floor}{\lfloor}{\rfloor}

\setlength{\parindent}{0pt}

\title{Ayudantía 4 - Algoritmos y Complejidad\\
Algoritmos Voraces II}
\author{Exponential Complexers}
\date{}

\begin{document}
\maketitle

\thispagestyle{empty}
\section{Ejercicios}
\begin{enumerate}
\item Una flota de barcos se reparten en la costa del mar, cada barco, en el eje $x$ ocupa un intervalo de la forma $[a,b]$, se tiene un arma que dispara un rayo en un punto $x$ que destruye todos los barcos cuyo intervalo contenga dicho $x$.
\begin{itemize}
\item Describa un algoritmo voraz que destruya toda la flota con la cantidad mínima de disparos.
\item Demuestre su optimalidad.
\end{itemize}
\item Ahora se tiene una serie de globos representados como círculos de diferente radio en todo el plano, y un arma fija que puede disparar rayos hacia cualquier ángulo.
\begin{itemize}
\item Describa como podría modificar el algoritmo del ejercicio anterior para resolver 
\item Demuestre que este nuevo algoritmo no gasta más de un rayo más que la solución óptima.
\end{itemize}
\item Una string de paréntesis balanceada puede encontrarse en alguna de las siguientes formas:
\begin{itemize}
\item Una string vacía.
\item Una string \textbf{(}$x$\textbf{)} donde $x$ es una string balanceada.
\item Una string $xy$ donde $x$ e $y$ son strings balanceadas.
\end{itemize}
\begin{enumerate}
\item Describir un algoritmo voraz para encontrar la substring balanceada de mayor largo en una string de paréntesis como:
\begin{center}
\textbf{((())()()()))()))(())()()}
\end{center}
\item Demostrar su optimalidad.
\end{enumerate}
\end{enumerate}
\end{document}
