\documentclass[english,spanish,fleqn]{article}

\usepackage{fourier}
\usepackage{amsmath}
\usepackage{babel}
\usepackage[colorlinks, urlcolor = blue]{hyperref}
\usepackage[ruled, noline]{algorithm2e}

%%%
%%% For algorithm2e
%%%

\SetAlgorithmName{Algoritmo}{Algoritmo}{Índice de algoritmos}

\SetAlCapSty{mdseries}
\SetKwProg{Function}{function}{}{end}
\SetKwProg{Procedure}{procedure}{}{end}

\title{Cálculo de raíces cuadradas módulo \(p\)}
\author{\href{mailto:vonbrand@inf.utfsm.cl}{Horst H. von Brand}}

\begin{document}

\maketitle
\thispagestyle{empty}

  En lo que sigue,
  \(p\) es un primo impar
  (el caso  \(p = 2\) es trivial).
  Casi todas las equivalencias serán módulo \(p\),
  omitiremos el módulo cuando así sea.

\section{Preliminares}
\label{sec:preliminares}

  Si \(p\) es primo,q
  el conjunto de residuos módulo \(p\) forma un campo,
  \(\mathbb{Z}_p\).
  Sabemos también que si \(x^2 = 1\) en un campo debe ser \(x = \pm 1\).

  El grupo de unidades \(\mathbb{Z}_p^\times\)
  (los elementos relativamente primos a \(p\),
   residuos \(1\) a \(p - 1\),
   con la multiplicación módulo \(p\))
  es cíclico,
  vale decir,
  hay un generador \(g\)
  (llamado \emph{raíz primitiva de \(p\)})
  tal que todo elemento \(a \in \mathbb{Z}_p^\times\) puede representarse como:
  \begin{equation}
    \label{eq:units-powers}
    a
      \equiv g^r
  \end{equation}
  Las raíces primitivas son relativamente frecuentes,
  es simple ver que si \(\gcd(r, p - 1) = 1\)
  entonces \(g^r\) es raíz primitiva;
  hay \(\phi(p - 1)\) raíces primitivas módulo \(p)\).

  Sabemos del pequeño teorema de Fermat que para \(a \in \mathbb{Z}_p^\times\)
  es:
  \begin{equation}
    \label{eq:Fermat}
    a^{p - 1}
      \equiv 1
  \end{equation}
  Es claro que si y solo si \(a \equiv g^{2 k}\) es un cuadrado
  (les llaman \emph{residuos cuadráticos}),
  exactamente la mitad de los elementos de \(\mathbb{Z}_p^\times\)
  es un cuadrado.
  Resulta el \emph{criterio de Euler}:
  \begin{equation}
    \label{eq:Euler-criterion}
    a^{(p - 1) / 2}
      \equiv \begin{cases}
                1 & \text{\(a\) es residuo cuadrático} \\
               -1 & \text{\(a\) no es residuo cuadrático}
             \end{cases}
  \end{equation}
  Para referencia futura,
  al multiplicar residuos cuadráticos y no cuadráticos
  da un residuo cuadrático
  si y solo si hay un número par de residuos no cuadráticos en el producto.

\section{Raíz cuadrada módulo \(p\)}
\label{sec:sqrt-mod-p}

  Si \(a = 0\),
  su raíz cuadrada es \(0\).

  En caso contrario,
  sea \(a\) un residuo cuadrático módulo \(p\),
  buscamos resolver \(x^2 \equiv a\).

  Debemos considerar varios casos.

\subsection{Caso \(p = 2\)}
\label{sec:p=2}

  Este caso es trivial,
  simplemente es \(\operatorname{sqrt}(a, 2) = a\).

\subsection{Caso \(p \equiv 3 \pmod{4}\)}
\label{sec:p-equiv-3}

  En este caso \((p - 1) / 2\) es impar,
  con lo que por el criterio de Euler:
  \begin{align*}
    a^{(p - 1) / 2}
      &\equiv 1 \\
    a \cdot a^{(p - 1) / 2}
      &\equiv a^{(p + 1) / 2} \\
      &\equiv a
  \end{align*}
  Como \(4 \mid (p + 1)\) en este caso,
  vemos que una raíz cuadrada es \(a^{(p + 1) / 4}\).

\subsection{Caso \(p \equiv 1 \pmod{4}\)}
\label{sec:p-equiv-3}

  Ahora \((p - 1) / 2\) es par.
  Sabemos que \(a^{(p - 1) / 2} \equiv 1\) por el criterio de Euler.
  Extrayendo raíz cuadrada sucesivamente
  (o sea,
   dividiendo el exponente por \(2\))
  obtendremos una secuencia de \(1\).
  Si llegamos a \(a^s \equiv 1\) con \(s\) impar,
  estamos listos:
  la raíz buscada es \(a^{(s + 1) / 2}\).

  Si no,
  en algún punto será \(a^s \equiv -1\) con \(s\) par.
  Ahora,
  si \(b\) es un residuo no cuadrático
  por el criterio de Euler es \(b^{(p - 1) / 2} \equiv -1\)
  y \(a^s b^{(p - 1) / 2} \equiv 1\).
  El punto es que conocemos su raíz cuadrada
  \(a^{s / 2} b^{(p - 1) / 4}\),
  podemos reiniciar el proceso de disminuir el exponente de \(a\).
  Como hemos llegado a \(s\)
  por sucesivas divisiones de \((p - 1) / 2\) por \(2\),
  no llegaremos a un exponente impar de \(b\) antes que de \(a\).
  Si tropezamos con \(-1\)
  nuevamente ajustamos multiplicando por \(b^{(p - 1) / 2}\).

  Un problema es que no se conoce un algoritmo determinista
  que dé un residuo no cuadrático
  en corto tiempo.
  Pero una búsqueda al azar rápidamente dará con uno
  (más de la mitad de los elementos entre \(2\) y \((p - 1) / 2\)
   son residuos no cuadráticos).

  Resulta el algoritmo~\ref{alg:sqrt-mod-p},
  donde hemos hecho un caso especial de \(p \equiv 5 \pmod{8}\).
  En este caso además de simplificar algo el código,
  aprovechamos que \(2\) es residuo no cuadrático
  para ahorrar la búsqueda.
  \begin{algorithm}
    \selectlanguage{english}
    \DontPrintSemicolon

    \Function{\(\operatorname{sqrt}(a, p)\)}{
      \uIf{\(p \equiv 3	 \pmod{4}\)}{
        \Return \(a^{(p + 1) / 4} \bmod p\) \;
      }
      \uElseIf{\(p \equiv 5  \pmod{8}\)}{
        \eIf{\(a^{(p - 1) / 4} \equiv 1 \pmod{p}\)}{
          \Return \(a^{(p + 3) / 8} \bmod p\) \;
        }
        {
          \Return \(2^{(p - 1) / 4} \cdot a^{(p + 3) / 8} \bmod p\) \;
        }
      }
      \Else{
        \(s \leftarrow (p - 1) / 4\) \;
        \While{\(a^s \equiv 1 \pmod{p}\)}{
          \eIf{\(\operatorname{odd}(s)\)}{
            \Return \(a^{(s + 1) / 2} \bmod p\) \;
          }
          {
            \(s \leftarrow s / 2\) \;
          }
        }
        \Repeat{\(b^{(p - 1) / 2} \equiv -1 \pmod{p}\)}{
          Select \(b\) at random from \([2, (p - 1) / 2]\) \;
        }
        \(t \leftarrow (p - 1) / 2\) \;
        \While{\(\neg \operatorname{odd}(s)\)}{
          \(s \leftarrow s / 2\); \(t \leftarrow t / 2\) \;
          \If{\(a^s \cdot b^t \equiv -1 \pmod{p}\)}{
            \(t \leftarrow t + (p - 1) / 2\) \;
          }
        }
        \Return \(a^{(s + 1) / 2} \cdot b^{t / 2} \bmod p\) \;
      }
    }
    \selectlanguage{spanish}
    \caption{Raíz cuadrada módulo \(p\)
             (casos no triviales)}
    \label{alg:sqrt-mod-p}
  \end{algorithm}
\end{document}

%%% Local Variables:
%%% mode: latex
%%% TeX-master: t
%%% ispell-local-dictionary: "spanish"
%%% ispell-local-dictionary: "spanish"
%%% End:

% LocalWords:  empty Select at random from
