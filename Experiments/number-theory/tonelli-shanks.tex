\documentclass[english, german, spanish, fleqn]{article}

\usepackage{fourier}
\usepackage{amsmath}
\usepackage{babel, babelbib}
\usepackage[colorlinks, urlcolor = blue]{hyperref}
\usepackage[ruled, noline, nofillcomment]{algorithm2e}

%%%
%%% For algorithm2e
%%%

\SetAlgorithmName{Algoritmo}{Algoritmo}{Índice de algoritmos}

\SetAlCapSty{mdseries}
\SetKwProg{Function}{function}{}{end}
\SetKwProg{Procedure}{procedure}{}{end}

\title{Cálculo de raíces cuadradas módulo \(p\)}
\author{\href{mailto:vonbrand@inf.utfsm.cl}{Horst H. von Brand}}

\begin{document}
\bibliographystyle{babplain-fl}

\maketitle
\thispagestyle{empty}

  En lo que sigue,
  \(p\) es un primo impar
  (el caso  \(p = 2\) es trivial).
  Casi todas las equivalencias serán módulo \(p\),
  omitiremos el módulo cuando así sea.

\section{Preliminares}
\label{sec:preliminares}

  Si \(p\) es primo,
  el conjunto de residuos módulo \(p\) forma un campo,
  \(\mathbb{Z}_p\).
  Sabemos también que si \(x^2 = 1\) en un campo debe ser \(x = \pm 1\).

  El grupo de unidades \(\mathbb{Z}_p^\times\)
  (los elementos relativamente primos a \(p\),
   residuos \(1\) a \(p - 1\),
   con la multiplicación módulo \(p\))
  es cíclico,
  vale decir,
  hay un generador \(g\)
  (llamado \emph{raíz primitiva de \(p\)})
  tal que todo elemento \(a \in \mathbb{Z}_p^\times\) puede representarse como:
  \begin{equation}
    \label{eq:units-powers}
    a
      \equiv g^r
  \end{equation}
  Las raíces primitivas son relativamente frecuentes,
  es simple ver que si \(\gcd(r, p - 1) = 1\)
  entonces \(g^r\) es raíz primitiva;
  hay \(\phi(p - 1)\) raíces primitivas módulo \(p)\).

  Sabemos del pequeño teorema de Fermat que para \(a \in \mathbb{Z}_p^\times\)
  es:
  \begin{equation}
    \label{eq:Fermat}
    a^{p - 1}
      \equiv 1
  \end{equation}
  Es claro que si y solo si \(a \equiv g^{2 k}\) es un cuadrado
  (les llaman \emph{residuos cuadráticos}),
  exactamente la mitad de los elementos de \(\mathbb{Z}_p^\times\)
  es un cuadrado.
  Resulta el \emph{criterio de Euler}:
  \begin{equation}
    \label{eq:Euler-criterion}
    a^{(p - 1) / 2}
      \equiv \begin{cases}
                1 & \text{\(a\) es residuo cuadrático} \\
               -1 & \text{\(a\) es no-residuo cuadrático}
             \end{cases}
  \end{equation}
  Para referencia futura,
  al multiplicar residuos cuadráticos y no\nobreakdash-residuos cuadráticos
  da un residuo cuadrático
  si y solo si hay un número par de no\nobreakdash-residuos cuadráticos
  en el producto.

\section{Raíz cuadrada módulo \(p\)}
\label{sec:sqrt-mod-p}

  Si \(a = 0\),
  su raíz cuadrada es \(0\).

  En caso contrario,
  sea \(a\) un residuo cuadrático módulo \(p\),
  buscamos resolver \(x^2 \equiv a\).

  Debemos considerar varios casos.

\subsection{Caso \(p = 2\)}
\label{sec:p=2}

  Este caso es trivial,
  simplemente es \(\operatorname{sqrt}(a, 2) = a\).

\subsection{Caso \(p \equiv 3 \pmod{4}\)}
\label{sec:p-equiv-3}

  En este caso \((p - 1) / 2\) es impar,
  con lo que por el criterio de Euler:
  \begin{align*}
    a^{(p - 1) / 2}
      &\equiv 1 \\
    a \cdot a^{(p - 1) / 2}
      &\equiv a^{(p + 1) / 2} \\
      &\equiv a
  \end{align*}
  Como \(4 \mid (p + 1)\) en este caso,
  vemos que una raíz cuadrada es \(a^{(p + 1) / 4}\).

\subsection{Caso \(p \equiv 1 \pmod{4}\)}
\label{sec:p-equiv-1}

  Ahora \((p - 1) / 2\) es par.
  Escribamos \(p - 1 = q \cdot 2^s\) con \(q\) impar.
  Note que:
  \begin{equation*}
    \left( a^{(q + 1) / 2} \right)^2
      \equiv a^q \cdot a
  \end{equation*}
  Si \(a^q \equiv 1\),
  estamos listos.
  En caso contrario,
  tenemos \(r_0\) y \(t_0\) que satisfacen:
  \begin{equation}
    \label{eq:rt}
    r_0^2
      \equiv a t_0
  \end{equation}
  donde \(t_0 \equiv a^q\) es una raíz \(2^{s - 1}\) de \(1\),
  ya que \(t_0^{2^{s - 1}} \equiv a^{q \cdot 2^{s - 1}} \equiv a^{(p - 1) / 2}\).
  Si dados \(r_k\), \(s_k\) y \(t_k\) como en~\eqref{eq:rt}
  podemos construir un nuevo juego \(r_{k + 1}, s_{k + 1}, t_{k + 1}\)
  pero para \(s_{k + 1} = s_k - 1\),
  podemos ir reduciendo el exponente \(s\) hasta \(0\) y estamos listos.

  Si \(t_k^{2^{s_k - 2}} \equiv 1\),
  no hay nada que hacer,
  los valores anteriores sirven:
  \(r_{k + 1} = r_k\) y \(t_{k + 1} = t_k\).
  Si no,
  debe ser \(t_k^{2^{s_k - 2}} \equiv -1\),
  ya que su cuadrado es congruente con \(1\).
  Para hallar un nuevo juego \(r_{k + 1}, t_{k + 1}\)
  multiplicamos \(r_k\) por un valor \(u\) a determinar
  tal que:
  \begin{align}
    (r_k u)^2
      &\equiv a t_k u^2 \\
      &\equiv 1 \label{eq:(ru)^2=1}
  \end{align}
  O sea:
  \begin{align*}
    r_{k + 1}
      &\equiv u r_k \\
    t_{k + 1}
      &\equiv u^2 t_k
  \end{align*}
  De~\eqref{eq:(ru)^2=1}
  vemos que \(u^2\) debe ser una raíz \(2^{s_k - 2}\) de \(-1\).
  El truco es usar un no\nobreakdash-residuo cuadrático \(b\),
  del que por el criterio de Euler sabemos:
  \begin{equation*}
    b^{q \cdot 2^{s_0 - 1}}
      \equiv -1
  \end{equation*}
  con lo que elevando \(b^q\) al cuadrado repetidas veces
  tenemos una secuencia de raíces \(2^k\) de \(-1\),
  elegimos la correcta.

  El algoritmo~\ref{alg:sqrt-mod-p} resultante es de Tonelli~%
     \cite{tonelli91:_bemerkung_aufloesung_congruenzen},
  redescubierto y mejorado casi un siglo después por Shanks~%
    \cite{shanks72:_five_number_theoretic_algorithms}.
  Es lo que provee por ejemplo la biblioteca LibTomMath~%
    \cite{teamtom19:_libtommath_1.2.0},
  descrita en detalle por St~Denis y Rose~%
    \cite{st_denis06:_bignum_math}.
  \begin{algorithm}
    \selectlanguage{english}
    \DontPrintSemicolon

    \Function{\(\operatorname{sqrt}(a, p)\)}{
      \eIf{\(p \equiv 3 \pmod{4}\)}{
        \Return \(a^{(p + 1) / 4} \bmod p\) \;
      }{
        \Repeat{\(b^{p - 1) / 2} \equiv -1 \pmod{p}\)}{
          Select \(b\) at random from \([2, (p - 1) / 2]\) \;
        }
        \tcc{\(b\) is a quadratic non-residue}
        \(q \leftarrow (p - 1) / 4\) \;
        \(s \leftarrow 2\) \;
        \While{\(\neg \operatorname{odd}(q)\)}{
          \(q \leftarrow q / 2\) \;
          \(s \leftarrow s + 1\) \;
        }
        \tcc{Now \(p - 1 = q \cdot 2^s\) with \(q\) odd}
        \(c \leftarrow b^q \bmod p\) \;
        \(t \leftarrow a^q \bmod p\) \;
        \(r \leftarrow a^{(q + 1) / 2} \bmod p\) \;
        \While{\(t \not\equiv 1 \pmod{p}\)}{
          \(i \leftarrow 1\) \;
          \(u \leftarrow t\) \;
          \While{\(u \not\equiv 1 \pmod{p}\)}{
            \(u \leftarrow u^2 \bmod p\) \;
            \(i \leftarrow i + 1\) \;
          }
          \(b \leftarrow c^{2^{s - i - 1}} \bmod p\) \;
          \(s \leftarrow i\) \;
          \(c \leftarrow b^2 \bmod p\) \;
          \(t \leftarrow t b^2 \bmod p\) \;
          \(r \leftarrow r b \bmod p\) \;
        }
        \Return \(r\) \;
      }
    }
    \selectlanguage{spanish}
    \caption{Raíz cuadrada módulo \(p\)
             (casos no triviales)}
    \label{alg:sqrt-mod-p}
  \end{algorithm}
  \bibliography{referencias}
\end{document}

%%% Local Variables:
%%% mode: latex
%%% TeX-master: t
%%% ispell-local-dictionary: "spanish"
%%% ispell-local-dictionary: "spanish"
%%% End:

% LocalWords:  empty Select at random from LibTomMath quadratic Now
% LocalWords:  residue with odd eq rt ru St is
