\documentclass[english, german, spanish, fleqn]{article}

\usepackage{fourier}
\usepackage{amsmath}
\usepackage{babel, babelbib}
\usepackage[colorlinks, urlcolor = blue]{hyperref}
\usepackage[ruled, noline, nofillcomment]{algorithm2e}

%%%
%%% For algorithm2e
%%%

\SetAlgorithmName{Algoritmo}{Algoritmo}{Índice de algoritmos}

\SetAlCapSty{mdseries}
\SetKwProg{Function}{function}{}{end}
\SetKwProg{Procedure}{procedure}{}{end}

\title{Prueba de primalidad}
\author{\href{mailto:vonbrand@inf.utfsm.cl}{Horst H. von Brand}}

\begin{document}
\bibliographystyle{babplain-fl}

\maketitle
\thispagestyle{empty}

  Discutiremos la prueba de primalidad de Miller-Rabin~%
    \cite{miller76:_Riemann_hypot_tests_primality,
          rabin80:_probab_algor_test_primality},
  un algoritmo aleatorizado para detectar números compuestos.

\section{Preliminares}
\label{sec:preliminares}

  Si \(p\) es primo,
  las únicas raíces cuadradas de \(1\)
  (soluciones de \(x^2 - 1 \equiv 0 \pmod{p}\))
  son \(\pm 1\).
  Esto porque \(p \mid x^2 - 1\) es lo mismo que \(p \mid (x + 1) (x - 1)\),
  con lo que por el lema de Euclides \(p \mid x + 1\) o \(p \mid x - 1\).
  Por el pequeño teorema de Fermat sabemos que \(a^{p - 1} \equiv 1 \pmod{p}\)
  si \(p\) es primo y \(p \not\mid a\).

  Si \(p\) es impar,
  \(p - 1 = q \cdot 2^s\) con \(q\) impar y \(s \ge 1\).
  Vale decir,
  \(a^{q \cdot 2^{s - 1}}\) es raíz cuadrada de \(1\),
  si es \(1\) lo es su raíz cuadrada \(a^{q \cdot 2^{s - 2}}\),
  \ldots
  Descendiendo de esta forma llegaremos a que \(a^q \equiv 1 \pmod{p}\)
  o que para algún \(0 \le k < s\) es \(a^{q \cdot 2^k} \equiv -1 \pmod{p}\).
  El contrapositivo de la observación precedente
  es que si hallamos \(a\) y \(0 \le k < s\)
  tales que \(a^{q \cdot 2^k} \equiv 1 \pmod{p}\)
  pero \(a^{q \cdot 2^{k - 1}} \not\equiv \pm 1 \pmod{p}\)
  entonces \(p\) es compuesto.
  Resulta que en tal caso a lo más una cuarta parte de los \(a\)
  mienten.

  El algoritmo~\ref{alg:is-prime} es la modificación aleatorizada de Rabin~%
     \cite{rabin80:_probab_algor_test_primality}
  al algoritmo determinista de Miller~%
     \cite{miller76:_Riemann_hypot_tests_primality}.
  Supone \(p > 3\) impar.
  Generalmente se repite la prueba cierto número de veces
  para obtener mayor confianza en el resultado.
  Esta es la prueba de primalidad en más extenso uso hoy día,
  la provee por ejemplo la biblioteca LibTomMath~%
    \cite{teamtom19:_libtommath_1.2.0},
  descrita en detalle por St~Denis y Rose~%
    \cite{st_denis06:_bignum_math}.
  \begin{algorithm}
    \selectlanguage{english}
    \DontPrintSemicolon

    \Function{\(\operatorname{is\_prime}(p)\)}{
      Write \(p = q \cdot 2^s + 1\) with \(q\) odd \;
      Pick \(a\) at random from \([2, n - 2]\) \;
      \(x \leftarrow a^q \bmod p\) \;
      \eIf{\(x = 1 \vee x = p - 1\)} {
        \Return Probably prime \;
      }{
        \For{\(k \leftarrow 1 \KwTo s - 1\)}{
          \(x \leftarrow x^2 \bmod p\) \;
          \If{\(x = p - 1\)}{
            \Return Probably prime \;
          }
          \Return Composite \;
        }
      }
    }
    \selectlanguage{spanish}
    \caption{Prueba de primalidad}
    \label{alg:miller-rabin}
  \end{algorithm}
  \bibliography{referencias}
\end{document}

%%% Local Variables:
%%% mode: latex
%%% TeX-master: t
%%% ispell-local-dictionary: "spanish"
%%% End:

% LocalWords:  empty Select at random from primalidad aleatorizado St
% LocalWords:  contrapositivo aleatorizada LibTomMath Denis Write odd
% LocalWords:  with Pick Probably Composite
