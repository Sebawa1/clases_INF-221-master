\documentclass[english, spanish, fleqn]{article}
\usepackage{amsmath, amssymb, fourier, mathrsfs}
\usepackage{enumitem}
\usepackage{dcolumn}
\usepackage{cancel}
\usepackage{babel}
\usepackage[noline]{algorithm2e}
\usepackage{listings}
\usepackage{csquotes}
\usepackage[colorlinks, urlcolor = blue]{hyperref}

\title{Cambio de monedas:\\
       Óptimo por programación dinámica}
\author{\href{mailto:vonbrand@inf.utfsm.cl}{Horst H. von Brand}}

%%%
%%% For listings
%%%

\lstloadlanguages{[ANSI]C}

\lstset{basicstyle = \sffamily, commentstyle = \slshape,
        frame = lines, numbers = none,
        showstringspaces = false,
        float, floatplacement = ht, captionpos = b,
        xleftmargin = 1em, xrightmargin = 1em
      }

%%%
%%% For algorithm2e
%%%

\SetAlgorithmName{Algoritmo}{Algoritmo}{Índice de algoritmos}

\SetAlCapSty{mdseries}
\SetKwProg{Function}{function}{}{end}
\SetKwProg{Procedure}{procedure}{}{end}

\begin{document}
\bibliographystyle{babplain-fl}

\maketitle
\thispagestyle{empty}

  En la perdida colonia antártica
  de \href{http://moneyart.biz/dd}{Nadiria}
  la moneda venía en denominaciones
  de \$1, \$4, \$7, \$13, \$28, \$52, \$91 y \$365.
  Como el papel era extremadamente escaso,
  por ley cualquier transacción en dinero
  debía usar el mínimo número de billetes.

\section{Por valor}
\label{sec:by-value}

  Usamos el esquema descrito en clase para obtener el número mínimo
  de billetes para completar la cantidad \(C\).
  Llamaremos \(b_0 = 1 < b_1 < \dotsb < b_{n - 1}\)
  a las denominaciones de los \(n\)~billetes
  (si \(b_0 \ne 1\),
   hay cantidades imposibles de entregar).
  \begin{enumerate}[label = {(\alph*)}]
  \item \textbf{Plantear la recurrencia:}
    En este caso,
    es bastante obvio que el mínimo de monedas requeridas
    para dar la cantidad \(C\)
    es uno más que la cantidad mínima de monedas
    requeridas para entregar
    las cantidades \(C - b_0\), \(C - b_1\), \ldots \(C - b_{n - 1}\)
    (omitiendo monedas de valor mayor que \(C\),
     claro está).
    Llamemos \(M(C)\) al mínimo número de monedas
    requerido para completar \(C\):
    \begin{equation}
      \label{eq:recurrence}
      M(C)
        = 1 + \min_{\substack{
                      0 \le k < n \\
                      b_k \ge C}}
                \{M(C - b_k)\}
      \qquad
      M(0)
        = 0
    \end{equation}
  \item \textbf{Escribir un programa recursivo:}
    El algoritmo recursivo~\ref{alg:recursive}
    expresa la recurrencia~\eqref{eq:recurrence}.
    \begin{algorithm}
      \DontPrintSemicolon\Indp

      \Function{\(M(C)\)}{
        \eIf{\(C = 0\)}{
          \Return \(0\) \;
        }
        {
          \(m \leftarrow 0\) \;
          \(k \leftarrow 0\) \;
          \While{\(k < n \wedge b_k \ge C\)}{
            \If{\(m > M(C - b_k)\)}{
              \(m \leftarrow M(C - b_k)\) \;
            }
            \(k \leftarrow k + 1\) \;
          }
          \Return \(m + 1\) \;
        }
      }

      \caption{Entregar vuelto --- recursivo por valor}
      \label{alg:recursive}
    \end{algorithm}
  \item \textbf{Identificar subproblemas:}
    Subproblemas relevantes son entregar la cantidad \(C - b_k\),
    con \(0 \le k < n\) tales que \(b_k \le C\).
  \item \textbf{Defina una estructura de datos:}
    La estructura obvia es un arreglo que contiene \(M(c)\) para
    \(0 \le c \le C\).
  \item \textbf{Identifique dependencias:}
    El valor de \(M(c)\) depende de los valores de algunos \(c\) anteriores.
    Lo más simple es tenerlos todos a mano.
  \item \textbf{Determine un buen orden de cálculo:}
    Por lo discutido en el punto anterior,
    un orden de cálculo es llenar el arreglo para \(C\) creciente.
  \item \textbf{Analice requerimientos de tiempo y espacio:}
    El tiempo es básicamente \(O(C n)\)
    (debemos llenar \(C + 1\) elementos del arreglo,
     cada uno toma \(O(n)\) operaciones).
    El espacio requerido es \(O(C)\).
  \item \textbf{Escriba el algoritmo:}
    Resulta el algoritmo~\ref{alg:dp}.
    \begin{algorithm}
      \DontPrintSemicolon\Indp

      \Function{\(M(C)\)}{
        \(M_0 \leftarrow 0\) \;
        \For{\(c \leftarrow 0\) \KwTo \(C\)}{
          \(m \leftarrow 0\) \;
          \(k \leftarrow 0\) \;
          \While{\(k < n \wedge b_k \ge C\)}{
            \If{\(m > M_{C - b_k}\)}{
              \(m \leftarrow M_{C - b_k}\) \;
            }
            \(k \leftarrow k + 1\) \;
          }
        }
        \Return \(M_C\) \;
      }

      \caption{Entregar vuelto --- programación dinámica por valor}
      \label{alg:dp}
    \end{algorithm}
  \end{enumerate}
  Lo anterior solo obtiene el número mínimo de monedas,
  lo que interesa realmente es saber cuánto entregar de cada denominación.
  Para ello debemos
  registrar el valor de \(k\) que da el mínimo para cada cantidad,
  recorriendo estos valores desde \(C\) hacia atrás,
  contabilizando cuántas veces aparece cada moneda.
  El programa~\ref{prog:dp} da detalles.
  \lstinputlisting[language = C,
                   firstline = 8,
                   tabsize = 4,
                   caption = {Cambio de monedas por valor},
                   label = prog:dp]{dp.c}

\section{Por moneda}
\label{sec:by-bill}

  Otra manera de plantear una recurrencia
  es considerar agregar una denominación.
  \begin{enumerate}[label = {(\alph*)}]
  \item \textbf{Plantear la recurrencia:}
    Es claro que el mínimo de billetes para entregar \(C\)
    usando las denominaciones \(b_0, \dotsc, b_{k + 1}\)
    es el mínimo entre entregar \(C\) con \(b_0, \dotsc, b_k\)
    y entregar \(C - b_{k + 1}\) más uno.
    Llamemos \(M(C, k)\) al mínimo número de monedas
    requerido para completar \(C\) usando \(b_0, \dotsc, b_k\):
    \begin{equation}
      \label{eq:recurrence-2}
      M(C, k)
        = \min\{
                 M(C, k - 1) + 1, M(C - b_k, k)
              \}
      \qquad
      M(C, 0)
        = C
    \end{equation}
    Es claro que debemos usar~\eqref{eq:recurrence-2} solo si \(C \le b_k\).
  \item \textbf{Escribir un programa recursivo:}
    El algoritmo recursivo~\ref{alg:recursive-2}
    expresa la recurrencia~\eqref{eq:recurrence-2}.
    \begin{algorithm}
      \DontPrintSemicolon\Indp

      \Function{\(M(C, k)\)}{
        \If{\(k = 0\)}{
          \Return \(C\) \;
        }
        {
          \If{\(C \ge b_k\)}{
            \Return \(\min\{M(C, k - 1), 1 + M(C - b_k, k - 1)\}\) \;
            }
            {
              \Return \(M(C, k - 1)\) \;
            }
        }
      }

      \caption{Entregar vuelto --- recursivo por valor y denominación}
      \label{alg:recursive-2}
    \end{algorithm}
  \item \textbf{Identificar subproblemas:}
    Subproblemas relevantes son entregar la cantidad \(C - b_k\),
    con \(0 \le k < n\) tales que \(b_k \le C\).
  \item \textbf{Defina una estructura de datos:}
    La estructura obvia es un arreglo que contiene \(M(c, k)\) para
    \(0 \le c \le C\) y \(0 \le k < n\).
  \item \textbf{Identifique dependencias:}
    El valor de \(M(c, k)\) depende de los valores
    de \(M(c, k - 1)\) y posiblemente \(M(c - b_k, k)\).
  \item \textbf{Determine un buen orden de cálculo:}
    Podemos llenar un arreglo que contenga \(M(c, k)\)
    una fila a la vez,
    para \(c\) creciente.
    Incluso,
    como \(M(c, k)\) solo depende de \(M(c, k - 1)\) y \(M(c', k)\)
    con \(c' < c \)
    podemos usar solo un vector.
  \item \textbf{Analice requerimientos de tiempo y espacio:}
    El tiempo es básicamente \(O(C n)\)
    (debemos llenar \(C + 1\) elementos del arreglo,
     cada uno toma \(O(n)\) operaciones).
    El espacio requerido es \(O(C)\)
    (suponiendo que solo almacenamos un vector).
  \item \textbf{Escriba el algoritmo:}
    Resulta el algoritmo~\ref{alg:dp-2}.
    \begin{algorithm}
      \DontPrintSemicolon\Indp

      \Function{\(M(C)\)}{
        \For{\(c \leftarrow 0\) \KwTo \(C\)}{
          \(M_c \leftarrow c\) \;
        }
        \For{\(k \leftarrow 1\) \KwTo \(n - 1\)}{
          \For{\(c \leftarrow b_k\) \KwTo \(C\)}{
            \If{\(M_{c - b_k} + 1 < M_k\)}{
              \(M_c \leftarrow M_{c - b_k} + 1\) \;
            }
          }
        }
        \Return \(M_C\) \;
      }

      \caption{Entregar vuelto --- programación dinámica por denominación}
      \label{alg:dp-2}
    \end{algorithm}
  \end{enumerate}
  Nuevamente,
  lo anterior solo obtiene el número mínimo de monedas,
  lo que interesa realmente es saber cuánto entregar de cada denominación.
  Para ello debemos
  registrar el valor de \(k\) que da el mínimo para cada cantidad,
  recorriendo estos valores desde \(C\) hacia atrás,
  contabilizando cuántas veces aparece cada moneda.
  El programa~\ref{prog:dp-2} da detalles.
  \lstinputlisting[language = C,
                   firstline = 8,
                   tabsize = 4,
                   caption = {Cambio de monedas por denominación},
                   label = prog:dp-2]{dp-2.c}

\end{document}

%%% Local Variables:
%%% mode: latex
%%% TeX-master: t
%%% ispell-local-dictionary: "spanish"
%%% End:

% LocalWords:  empty Nadiria eq recurrence
