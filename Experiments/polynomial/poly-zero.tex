\documentclass[english, spanish, fleqn]{article}

\usepackage{fourier}
\usepackage{amsmath}
\usepackage{babel}
\usepackage{listings}
\usepackage{pgf}
\usepackage[colorlinks, urlcolor = blue]{hyperref}

\lstloadlanguages{[ANSI]C}

\title{Cálculo de un polinomio cerca de un cero}
\author{\href{mailto:vonbrand@inf.utfsm.cl}{Horst H. von Brand}}

\begin{document}

\lstset{language = [ANSI]C,
        basicstyle = \sffamily, commentstyle = \slshape,
        frame = lines, numbers = none,
        showstringspaces = false,
        float, floatplacement = ht, captionpos = b,
        xleftmargin = 1em, xrightmargin = 1em
       }

\renewcommand{\lstlistingname}{Listado}

\maketitle
\thispagestyle{empty}

  Consideremos el polinomio:
  \begin{equation*}
    p(x)
      = x^4 - 4 x^3 + 6 x^2 - 4 x + 1
  \end{equation*}
  Con su ojo de buen sansano,
  reconocen inmediatamente que \(p(x) = (x - 1)^4\),
  con lo que el único cero es obvio.
  Pero supongamos que no somos tan pillos,
  y calculamos el valor de \(p(x)\) con precisión simple
  (para que quede más claro el efecto de los errores de redondeo).
  Los coeficientes se pueden representar exactamente en punto flotante,
  el valor de \(p\) en los alrededores del cero
  es casi únicamente error.
  El programa es el del listado~\ref{lst:poly}.
  \lstinputlisting[caption = {Cálculo del polynomio por Horner},
                   label = lst:poly]
                  {poly.c}
  Usa el método de Horner para calcular el polinomio,
  vale decir escribir:
  \begin{equation*}
    a_n x^n + a_{n - 1} x^{n - 1} + \dotsb + a_0
      = (\dotsc((a_n x + a_{n - 1}) x + a_{n - 2}) x + \dotsc + a_1) x + a_0
  \end{equation*}
  Esto significa una multiplicación y una suma por cada grado del polinomio,
  un gran ahorro frente al cálculo ingenuo.
  Graficamos los resultados en la figura~\ref{fig:poly}.
  \begin{figure}[ht]
    \centering
    \pgfimage{plot}
    \caption{Valor calculado del polinomio}
    \label{fig:poly}
  \end{figure}
  Vemos que en vez de la curva suave que esperamos
  hay un comportamiento bastante caótico.
\end{document}

%%% Local Variables:
%%% mode: latex
%%% TeX-master: t
%%% ispell-local-dictionary: "spanish"
%%% End:

% LocalWords:  empty sansano Graficamos plot
