\documentclass[english, spanish, fleqn%
, hyperref = {colorlinks, urlcolor = blue}%
% , handout%
]{beamer}

\usepackage{beamerthemesplit}

\usepackage{fourier}
\usepackage{amsmath}
\usepackage{amsthm}
\usepackage{babel}
\usepackage{pgfplots}
\pgfplotsset{compat = newest}

\title{Asintóticas}
\subtitle{INF-221}

\author[Horst H. von Brand]{Horst H. von Brand\\
  \href{mailto:vonbrand@inf.utfsm.cl}{vonbrand@inf.utfsm.cl}}

\institute[DI UTFSM]{Departamento de Informática\\
                     Universidad Técnica Federico Santa María}
\date{}

\begin{document}
\beamerdefaultoverlayspecification{<+->}
\frame{\maketitle}

\begin{frame}
  \setcounter{beamerpauses}{2}
  \frametitle{Funciones ejemplo}

  \uncover<+->{
    Consideremos las funciones:
    \begin{align*}
      f(x)
        &= 3 x \cos 20 x \\
      g(x)
        &= \sin 5 x
    \end{align*}
  }
  \uncover<+->{
    Es simple demostrar que alrededor de \(0\)
    (en realidad,
     valen en todo \(\mathbb{R}\)):
    \begin{align*}
      f(x)
        &= O(x)	 \\
      g(x)
        &= O(x) \\
      f(x) + g(x)
        &= O(x) \\
      f(x) - g(x)
        &= O(x)
    \end{align*}
  }
\end{frame}

\begin{frame}
  \setcounter{beamerpauses}{2}
  \frametitle{Gráficos}
  \framesubtitle{La función \(f(x)\) y cotas \(\pm x\)}

  \begin{center}
    \begin{tikzpicture}
      \begin{axis}[axis lines = middle,
        xtick = \empty,
        ytick = \empty]
        \addplot [domain = -3.5:3.5, samples = 200, smooth, blue]
           {x * cos(deg(10 * x))};
        \addplot [domain = -3.5:3.5, red]
           {x};
        \addplot [domain = -3.5:3.5, red]
           {-x};
      \end{axis}
    \end{tikzpicture}
  \end{center}
\end{frame}

\begin{frame}
  \setcounter{beamerpauses}{2}
  \frametitle{Gráficos}
  \framesubtitle{La función \(g(x)\) y cotas \(\pm 4 x\)}

  \begin{center}
    \begin{tikzpicture}
      \begin{axis}[axis lines = middle,
        xtick = \empty,
        ytick = \empty]
        \addplot [domain = -3.5:3.5, samples = 200, smooth, blue]
           {sin(deg(4 * x))};
        \addplot [domain = -0.875:0.875, red]
           {4 * x};
        \addplot [domain = -0.875:0.875, red]
           {-4 * x};
      \end{axis}
    \end{tikzpicture}
  \end{center}
\end{frame}

\begin{frame}
  \setcounter{beamerpauses}{2}
  \frametitle{Gráficos}
  \framesubtitle{Las funciones \(f(x) \pm \: g(x)\) y cotas \(\pm 5 x\)}

  \begin{center}
    \begin{tikzpicture}
      \begin{axis}[axis lines = middle,
        xtick = \empty,
        ytick = \empty]
        \addplot [domain = -3.5:3.5, samples = 200, smooth, blue]
           {x * cos(deg(20 * x)) + sin(deg(4 * x))};
        \addplot [domain = -3.5:3.5, samples = 200, smooth, green]
           {x * cos(deg(20 * x)) - sin(deg(4 * x))};
        \addplot [domain = -0.8:0.8, red]
           {5 * x};
        \addplot [domain = -0.8:0.8, red]
           {-5 * x};
      \end{axis}
    \end{tikzpicture}
  \end{center}
\end{frame}

\begin{frame}
  \setcounter{beamerpauses}{2}
  \frametitle{Sumas y restas}

  \uncover<+->{
    Este último ejemplo muestra que una suma o resta de funciones
    debe considerar la suma de las cotas
    (en valor absoluto).
  }
\end{frame}

\begin{frame}
  \setcounter{beamerpauses}{2}
  \frametitle{Varios \(O(\cdot)\)}

  \uncover<+->{
    Note que al decir:
    \begin{align*}
      f(x) - g(x)
      &= O(x) + O(x) \\
      &= O(x)
    \end{align*}
    cada una de las veces que aparece \(O(\cdot)\)
    es una función \emph{diferente} la representada.
  }
\end{frame}

\begin{frame}
  \setcounter{beamerpauses}{2}
  \frametitle{Simplificando}

  \uncover<+->{
    Consideremos la expresión:
    \begin{equation*}
      \frac{1}{1 + O(x)}
    \end{equation*}
    Esto representa alguna expresión de la forma:
    \begin{equation*}
      \frac{1}{1 + h(x)} \qquad h(x) = O(x)
    \end{equation*}
  }
\end{frame}

\begin{frame}
  \setcounter{beamerpauses}{2}
  \frametitle{Simplificando}

  \uncover<+->{
    Sabemos que la  fracción puede expandirse,
    siempre que \(\lvert h(x) \rvert < 1\):
  }
  \begin{align*}
    \uncover<+->{
      \frac{1}{1 + h(x)}
        &=   \sum_{n \ge 0} (-1)^k h^k(x) \\
    }
    \uncover<+->{
        &=   1 - h(x) + \sum_{k \ge 2} (-1)^{k - 2} h^k(x) \\
    }
    \uncover<+->{
        &=   1 - h(x) + h^2(x) \sum_{k \ge 2} (-1)^{k - 2} h^{k - 2}(x) \\
    }
    \uncover<+->{
        &=   1 - h(x) + h^2(x) \sum_{k \ge 0}(-1)^k h^k(x) \\
    }
    \uncover<+->{
        &\le 1 - h(x) + h^2(x) \sum_{k \ge 0} \lvert h(x) \rvert^k
    }
  \end{align*}
\end{frame}

\begin{frame}
  \setcounter{beamerpauses}{2}
  \frametitle{Simplificando}

  \uncover<+->{
    Fijemos algún rango \(\lvert x \rvert \le \delta\)
    tal que en él \(\lvert h(x) \rvert \le 1 / 2\).
  }
  \uncover<+->{
    Con esto tenemos:
  }
  \begin{align*}
    \uncover<+->{
      \frac{1}{1 + h(x)}
        &\le 1 - h(x) + h^2(x) \sum_{k \ge 0} \lvert h(x) \rvert^k \\
    }
    \uncover<+->{
        &\le 1 - h(x) + h^2(x) \sum_{k \ge 0} 2^{-k} \\
    }
    \uncover<+->{
        &=   1 - h(x) + h^2(x) \frac{1}{1 - 1/2} \\
    }
    \uncover<+->{
        &=   1 - h(x) + 2 h^2(x) \\
    }
    \uncover<+->{
        &=   1 - h(x) + O(h^2(x))
    }
  \end{align*}
  \uncover<+->{
    Como tenemos \(h(x) = O(x)\),
    por la advertencia anterior concluimos:
    \begin{align*}
      \uncover<.->{
        \frac{1}{1 + O(x)}
          = 1 + O(x) + O(x^2)
      }
      \uncover<+->{
          = 1 + O(x)
      }
    \end{align*}
  }
\end{frame}

\begin{frame}
  \setcounter{beamerpauses}{2}
  \frametitle{Simplificando}

  \uncover<+->{
    Si buscamos simplificar una expresión más compleja,
    como ser:
    \begin{equation*}
      \frac{1 + O(x)}{1 + O(x)}
    \end{equation*}
  }
  \uncover<+->{
    Nuevamente,
    debe tenerse presente que ambos \(O(\cdot)\)
    representan funciones posiblemente diferentes.
  }
  \begin{align*}
    \uncover<+->{
      \frac{1 + O(x)}{1 + O(x)}
         &= (1 + O(x)) \cdot (1 + O(x))
             && \text{Por lo anterior} \\
    }
    \uncover<+->{
         &= 1 + (O(x) + O(x)) + O(x) \cdot O(x)
             && \text{Multiplicar expresiones} \\
    }
    \uncover<+->{
         &= 1 + O(x) + O(x^2)
             && \text{Agrupar, simplificar} \\
    }
    \uncover<+->{
         &= 1 + O(x)
             && \text{Absorber}
    }
  \end{align*}
\end{frame}

\begin{frame}
  \setcounter{beamerpauses}{2}
  \frametitle{Simplificando}

  \uncover<+->{
    Un ejemplo algo más complicado es:
  }
  \begin{align*}
    \uncover<.->{
      \frac{1 + 3 x + O(x^2)}{1 + 5 x + O(x^3)}
         &= (1 + 3 x + O(x^2)) \cdot {} \\
         &\qquad (1 - (5 x + O(x^3)) + (5 x + O(x^3)^2 - \dotsm) \\
    }
    \uncover<+->{
         &= (1 + 3 x + O(x^2)) \cdot {} \\
         &\qquad (1 - 5 x + O(x^3) + 25 x^2 + 10 x O(x^3) + O(x^6) - \dotsm) \\
    }
    \uncover<+->{
         &= (1 + 3 x + O(x^2)) \cdot {} \\
         &\qquad (1 - 5 x + 25 x^2 + O(x^3) + O(x^4) + O(x^6) - \dotsm) \\
    }
    \uncover<+->{
         &= (1 + 3 x + O(x^2))
               (1 - 5 x + 25 x^2 + O(x^3)) \\
    }
    \uncover<+->{
         &= 1 - 2 x + 10 x^2 + O(x^2) + \dotsm \\
    }
    \uncover<+->{
         &= 1 - 2 x + O(x^2)
    }
  \end{align*}
\end{frame}

\begin{frame}
  \setcounter{beamerpauses}{2}
  \frametitle{Simplificando}

  \uncover<+->{
    Esto último ilustra que en general no tiene sentido
    darse el trabajo de combinar aproximaciones de orden distinto,
    términos extra de la expresión más precisa terminan absorbidos.
  }
\end{frame}
\end{document}

%%% Local Variables:
%%% mode: latex
%%% TeX-master: t
%%% End:
