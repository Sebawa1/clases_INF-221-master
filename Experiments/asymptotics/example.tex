\documentclass[spanish, fleqn]{article}

\usepackage{fourier}
\usepackage{amsmath}
\usepackage[es-noshorthands]{babel}
\usepackage{babelbib}
\usepackage{tikz, pgfplots}
\usepackage[colorlinks, urlcolor = blue]{hyperref}
\pgfplotsset{compat = newest}

\title{Un ejemplo detallado de manipulación asintótica}
\author{\href{mailto:vonbrand@inf.utfsm.cl}{Horst H. von Brand}}

\begin{document}
\bibliographystyle{babplain-fl}

\maketitle
\thispagestyle{empty}

  Consideremos las funciones:
  \begin{align*}
    f(x)
      &= 3 x \cos 20 x \\
    g(x)
      &= \sin 5 x
  \end{align*}
  Es simple demostrar que alrededor de \(0\)
  (en realidad,
   valen en todo \(\mathbb{R}\)):
  \begin{align*}
    f(x)
      &= O(x)  \\
    g(x)
      &= O(x) \\
    f(x) + g(x)
      &= O(x) \\
    f(x) - g(x)
      &= O(x)
  \end{align*}
  Gráficamente se ilustran en las figuras~\ref{fig:function-f},
  \ref{fig:function-g}
  y~\ref{fig:function-f+g}.
  \begin{figure}[ht]
    \centering
    \begin{tikzpicture}
      \begin{axis}[axis lines = middle,
        xtick = \empty,
        ytick = \empty]
        \addplot [domain = -3.5:3.5, samples = 200, smooth, blue]
           {x * cos(deg(10 * x))};
        \addplot [domain = -3.5:3.5, red]
           {x};
        \addplot [domain = -3.5:3.5, red]
           {-x};
      \end{axis}
    \end{tikzpicture}
    \caption{La función \(f(x)\) y cotas \(\pm x\)}
    \label{fig:function-f}
  \end{figure}
  \begin{figure}[ht]
    \centering
    \begin{tikzpicture}
      \begin{axis}[axis lines = middle,
        xtick = \empty,
        ytick = \empty]
        \addplot [domain = -3.5:3.5, samples = 200, smooth, blue]
           {sin(deg(4 * x))};
        \addplot [domain = -0.875:0.875, red]
           {4 * x};
        \addplot [domain = -0.875:0.875, red]
           {-4 * x};
      \end{axis}
    \end{tikzpicture}
    \caption{La función \(g(x)\) y cotas \(\pm 4 x\)}
    \label{fig:function-g}
  \end{figure}
  \begin{figure}[ht]
    \centering
    \begin{tikzpicture}
      \begin{axis}[axis lines = middle,
        xtick = \empty,
        ytick = \empty]
        \addplot [domain = -3.5:3.5, samples = 200, smooth, blue]
           {x * cos(deg(20 * x)) + sin(deg(4 * x))};
        \addplot [domain = -3.5:3.5, samples = 200, smooth, green]
           {x * cos(deg(20 * x)) - sin(deg(4 * x))};
        \addplot [domain = -0.8:0.8, red]
           {5 * x};
        \addplot [domain = -0.8:0.8, red]
           {-5 * x};
      \end{axis}
    \end{tikzpicture}
    \caption{Las funciones \(f(x) \pm \: g(x)\) y cotas \(\pm 5 x\)}
    \label{fig:function-f+g}
  \end{figure}
  Este último ejemplo muestra que una suma o resta de funciones
  debe considerar la suma de las cotas
  (en valor absoluto).

  Note que al decir:
  \begin{align*}
    f(x) - g(x)
    &= O(x) + O(x) \\
    &= O(x)
  \end{align*}
  cada una de las veces que aparece \(O(\cdot)\)
  es una función \emph{diferente} la representada.

  Consideremos una expresión como:
  \begin{equation*}
    \frac{1}{1 + O(x)}
  \end{equation*}
  Esto representa alguna expresión de la forma:
  \begin{equation*}
    \frac{1}{1 + h(x)} \qquad h(x) = O(x)
  \end{equation*}
  Sabemos que la  fracción puede expandirse,
  siempre que \(\lvert h(x) \rvert < 1\):
  \begin{align*}
    \frac{1}{1 + h(x)}
      &=   \sum_{n \ge 0} (-1)^k h^k(x) \\
      &=   1 - h(x) + \sum_{k \ge 2} (-1)^{k + 2} h^k(x) \\
      &=   1 - h(x) + h^2(x) \sum_{k \ge 0}(-1)^k h^k(x) \\
      &\le 1 - h(x) + h^2(x) \sum_{k \ge 0} \lvert h(x) \rvert^k \\
  \end{align*}
  Fijemos algún rango \(\lvert x \rvert \le \delta\)
  tal que en él \(\lvert h(x) \rvert \le 1 / 2\).
  Con esto tenemos:
  \begin{align*}
    \frac{1}{1 + h(x)}
      &\le 1 - h(x) + h^2(x) \sum_{k \ge 0} \lvert h(x) \rvert^k \\
      &\le 1 - h(x) + h^2(x) \sum_{k \ge 0} 2^{-k} \\
      &=   1 - h(x) + h^2(x) \frac{1}{1 - 1/2} \\
      &=   1 - h(x) + 2 h^2(x) \\
      &=   1 - h(x) + O(h^2(x))
  \end{align*}
  Como tenemos \(h(x) = O(x)\),
  por la advertencia anterior concluimos:
  \begin{align*}
    \frac{1}{1 + O(x)}
      &= 1 + O(x) + O(x^2) \\
      &= 1 + O(x)
  \end{align*}

  Si buscamos simplificar una expresión más compleja,
  como ser:
  \begin{equation*}
    \frac{1 + O(x)}{1 + O(x)}
  \end{equation*}
  Nuevamente,
  debe tenerse presente que ambos \(O(\cdot)\)
  representan funciones posiblemente diferentes.
  En cámara lenta:
  \begin{align*}
    \frac{1 + O(x)}{1 + O(x)}
       &= (1 + O(x)) \cdot (1 + O(x))
           && \text{Por lo anterior} \\
       &= 1 + (O(x) + O(x)) + O(x) \cdot O(x)
           && \text{Multiplicando las dos expresiones} \\
       &= 1 + O(x) + O(x^2)
           && \text{Agrupando términos y simplificando} \\
       &= 1 + O(x)
  \end{align*}

  Un ejemplo algo más complicado es:
  \begin{align*}
    \frac{1 + 3 x + O(x^2)}{1 + 5 x + O(x^3)}
       &= (1 + 3 x + O(x^2))
            (1 - (5 x + O(x^3)) + (5 x + O(x^3)^2 - \dotsm) \\
       &= (1 + 3 x + O(x^2))
             (1 - 5 x + O(x^3) + 25 x^2 + 10 x O(x^3) + O(x^6) - \dotsm) \\
       &= (1 + 3 x + O(x^2))
             (1 - 5 x + 25 x^2 + O(x^3) + O(x^4) + O(x^6) - \dotsm) \\
       &= (1 + 3 x + O(x^2))
             (1 - 5 x + 25 x^2 + O(x^3)) \\
       &= 1 - 2 x + 10 x^2 + O(x^2) + \dotsm \\
       &= 1 - 2 x + O(x^2)
  \end{align*}
  Esto ilustra que en general no tiene sentido
  darse el trabajo de combinar aproximaciones de orden distinto,
  términos extra de la expresión más precisa terminan absorbidos.
\end{document}

%%% Local Variables:
%%% mode: latex
%%% TeX-master: t
%%% End:

% LocalWords:  empty
