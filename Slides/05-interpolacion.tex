\input glyphtounicode%
\pdfgentounicode=1
\documentclass[english, spanish, fleqn,%
hyperref = {colorlinks, urlcolor = blue}%
%, handout%
]{beamer}

%% ]{article}
%% \usepackage{beamerarticle}

\usepackage{beamerthemesplit}

\usepackage{fourier}
\usepackage{amsmath}
\usepackage{amsthm}
\usepackage[es-noquoting]{babel}
\usepackage{csquotes}
\usepackage{pgfplots}

\pgfplotsset{compat=newest}

\ifdefined\HCode
  \def\pgfsysdriver{pgfsys-dvisvgm4ht.def}
\fi

\usefonttheme{professionalfonts}

\beamerdefaultoverlayspecification{<+->}

\title{Interpolación}

\author[Horst H. von Brand]{Horst H. von Brand\\
  \href{mailto:vonbrand@inf.utfsm.cl}{vonbrand@inf.utfsm.cl}}

\institute[DI UTFSM]{Departamento de Informática\\
                     Universidad Técnica Federico Santa María}
\date{}

\begin{document}
\frame{\maketitle}

\begin{frame}<handout>
  \frametitle{Contenido}

  \tableofcontents
\end{frame}

\section{El problema}

\begin{frame}
  \setcounter{beamerpauses}{2}
  \frametitle{El problema de interpolación}

  \uncover<+->{
    Es común conocer los valores de alguna función solo en algunos puntos:
    son mediciones,
    resultados de experimentos,
    valores históricos,
    incluso resultados de cálculos aproximados
    que solo dan valores en puntos discretos.
  }

  \uncover<+->{
    Pero queremos alguna representación aproximada que sea continua,
    por ejemplo tener valores en puntos intermedios
    para poder graficar los valores.
  }

  \uncover<+->{
    Solía ser una habilidad esencial interpolar cuando funciones trigonométricas
    y logaritmos venían en libros,
    las famosas \emph{tablas de logaritmos}\ldots
  }
\end{frame}

\begin{frame}
  \frametitle{El problema de interpolación}

  \uncover<+->{
    Si buscamos una función que aproxime los puntos dados
    (posiblemente con errores)
    estamos hablando del problema de \emph{ajuste de curvas};
    si nos contentamos con funciones que aproximen tramos,
    típicamente polinomios,
    donde imponemos alguna condición de continuidad
    en los puntos de encuentro,
    es \emph{interpolación por splines}
    (en inglés \emph{\foreignlanguage{english}{spline}}
     es una regla flexible ---cercha--- que se usa para dibujar curvas
     pasando por puntos dados,
     la curva que sigue se puede describir por un polinomio por tramos).
  }
  \uncover<+->{
    Para describir curvas suaves
    se usan curvas de Bezier
    (inicialmente carrocerías de auto;
     hoy describen las formas de letras,
     movimientos de brazos de robot industriales,
     objetos en juegos de vídeo,
     entre muchas otras aplicaciones).
  }
\end{frame}

\section{Interpolar por polinomios}

\begin{frame}
  \setcounter{beamerpauses}{2}
  \frametitle{Funciones continuas simples}

  Posiblemente las funciones continuas más simples de manipular
  son los polinomios.

  \uncover<+->{
    Así,
    podemos plantear nuestro problema formalmente:

    \emph{Dados valores de la función \(f\) en puntos discretos
          \(x_0, x_1, \dotsc, x_n\),
          construir un polinomio que pase por los puntos \((x_i, f(x_i))\).
         }
  }
  \uncover<+->{
    Note que estamos suponiendo que los valores \(f(x_i)\) son exactos
    y que buscamos un polinomio para todo el rango.
  }
\end{frame}

\begin{frame}
  \setcounter{beamerpauses}{2}
  \frametitle{Interpolación por polinomios}

  \uncover<+->{
    Formalmente,
    nuestro problema es:
    dados puntos \((x_i, y_i)\) para \(0 \le i \le n\),
    donde suponemos que los \(x_i\) son todos diferentes,
    hallar un polinomio que pase por ellos.
  }
  \uncover<+->{
    Son \(n + 1\) puntos,
    podemos aspirar a hallar \(n + 1\)~coeficientes,
    vale decir,
    un polinomio de grado \(n\):
    \begin{equation*}
      p(x)
        = a_n x^n + a_{n - 1} x^{n - 1} + \dotsb + a_0
    \end{equation*}
  }
\end{frame}

\begin{frame}
  \frametitle{Sistema de ecuaciones}

  Esto sugiere el sistema de ecuaciones:
  \begin{equation*}
    \begin{pmatrix}
      1	     & x_0    & x_0^2  & \cdots & x_0^n	 \\
      1	     & x_1    & x_1^2  & \cdots & x_1^n	 \\
      \vdots & \vdots & \vdots & \ddots & \vdots \\
      1	     & x_n    & x_n^2  & \cdots & x_n^n	 \\
    \end{pmatrix}
      \cdot \begin{pmatrix}
               a_0    \\
               a_1    \\
               \vdots \\
               a_n
            \end{pmatrix}
      =	 \begin{pmatrix}
            y_0	   \\
            y_1	   \\
            \vdots \\
            y_n
          \end{pmatrix}
  \end{equation*}
\end{frame}

\begin{frame}
  \frametitle{Preguntas obvias}

  \uncover<+->{
    ¿Cuándo tiene solución este sistema?
  }

  \uncover<+->{
    Si hay solución,
    ¿es única?
  }
  \\ \bigskip
  \uncover<+->{
    ¿Hay maneras mejores
    (más simples/directas)
    de expresar el polinomio \(p(x)\)?
  }
\end{frame}

\subsection{Cuándo existe y es único el polinomio interpolador}

\begin{frame}
  \frametitle{Determinante de Vandermonde}

  Sabemos que el sistema de ecuaciones:
  \begin{equation*}
    \mathbf{A} \cdot \mathbf{x}
      = \mathbf{y}
  \end{equation*}
  (con \(\mathbf{A}\) una matriz cuadrada,
   \(\mathbf{x}\) e \(\mathbf{y}\) vectores columna)
  tiene solución única para \(\mathbf{x}\) si y solo si
  el determinante \(\lvert \mathbf{A} \rvert \ne 0\).
\end{frame}

\begin{frame}
  \frametitle{Determinante de Vandermonde}

  \begin{theorem}
      El determinante de Vandermonde vale:
    \begin{equation*}
      \begin{vmatrix}
        1 & a_1 & a_1^2 & \cdots & a_1^{n - 1} \\
        1 & a_2 & a_2^2 & \cdots & a_2^{n - 1} \\
        \vdots & \vdots & \vdots & \ddots & \vdots \\
        1 & a_n & a_n^2 & \cdots & a_n^{n - 1}
      \end{vmatrix}
        = \prod_{1 \le i < j \le n} (a_j - a_i)
    \end{equation*}
  \end{theorem}
\end{frame}

\begin{frame}
  \frametitle{Determinante de Vandermonde}

  \textbf{\large\textcolor{blue}{Demostración}}

  Por inducción sobre \(n\).
  Llamaremos \(V_n\) al determinante con \(a_i\) de \(1 \le i \le n\).
  \begin{description}
  \item[Base:]
    El caso \(n = 1\) es trivial.
  \end{description}
\end{frame}

\begin{frame}
  \frametitle{Determinante de Vandermonde (cont)}

  \begin{description}
  \item[Inducción:]
    Supongamos que vale para \(k\),
    consideremos el determinante de \((k + 1) \times (k + 1)\):
    \begin{equation*}
      \begin{vmatrix}
        1 & a_1 & a_1^2 & \cdots & a_1^k \\
        \vdots & \vdots & \vdots & \ddots & \vdots \\
        1 & a_k & a_k^2 & \cdots & a_k^k \\
        1 & x	  & x^2	  & \cdots & x^k \\
      \end{vmatrix}
    \end{equation*}
    Expandiendo por la última fila
    obtenemos un polinomio \(f(x)\),
    \(\deg(f) \le k\).
    Determinantes con filas iguales son cero,
    concluimos \(f(a_i) = 0\):
    \begin{equation*}
      f(x)
        = c \prod_{1 \le i \le k} (x - a_i)
    \end{equation*}
  \end{description}
\end{frame}

\begin{frame}
  \frametitle{Determinante de Vandermonde (cont)}

  \begin{description}
  \item[]
    La expansión muestra que el coeficiente de \(x^k\) es \(V_k\).
    Con \(x \mapsto a_{k + 1}\) obtenemos:
    \begin{align*}
      V_{k + 1}
        &= V_k \cdot \prod_{1 \le i \le k} (a_{k + 1} - a_i) \\
        &= \prod_{1 \le i < j \le k} (a_j - a_i)
             \cdot \prod_{1 \le i \le k} (a_{k + 1} - a_i) \\
        &= \prod_{1 \le i < j \le k + 1} (a_j - a_i)
    \end{align*}
  \end{description}
  \qed
\end{frame}

\begin{frame}
  \frametitle{Cuándo hay polinomio interpolador único}

  En resumen,
  hay un único polinomio de grado a lo más~\(n\)
  que interpola los \(n + 1\)~puntos
  \(x_i, y_i)\) para \(0 \le i \le n\)
  siempre y cuando los \(x_i\) sean todos diferentes.
\end{frame}

\subsection{Forma de Lagrange}

\begin{frame}
  \frametitle{Forma de Lagrange}

  \uncover<+->{
    Los \emph{polinomios base de Lagrange}:
    \begin{equation*}
      \ell_i(x)
      = \frac{(x - x_n) \dotsm (x - x_{i + 1})
                 (x - x_{i - 1}) \dotsm (x - x_0)}
             {(x_i - x_n) \dotsm (x_i - x_{i + 1})
                 (x_i - x_{i - 1}) \dotsm (x_i - x_0)}
   \end{equation*}
   cumplen:
   \begin{equation*}
     \deg(\ell_i)
       = n
     \qquad
     \ell_i(x_k)
       = [i = k]
    \end{equation*}
  }
  \uncover<+->{
    El polinomio:
    \begin{equation*}
      p(x)
        = \sum_{0 \le k \le n} y_k \ell_k(x)
    \end{equation*}
    es de grado a lo más \(n\)
    y toma los valores \(p(x_k) = y_k\) para \(0 \le k \le n\).
  }
\end{frame}

\subsection{Fenómeno de Runge}

\begin{frame}
  \frametitle{El fenómeno de Runge}

  Consideremos la
  (aparentemente mansa)
  función:
  \begin{equation*}
    f(x)
      = \frac{1}{1 + 25 x^2}
  \end{equation*}
  \\ \medskip
  \begin{center}
    \begin{tikzpicture}[scale = 0.70]
      \begin{axis}[xmin = -1, xmax = 1]
        \addplot[samples = 201, smooth, red] {1 / (1 + 25 * x^2};
      \end{axis}
    \end{tikzpicture}
  \end{center}
\end{frame}

\begin{frame}
  \frametitle{El fenómeno de Runge}

  Interpolando con 9 puntos igualmente espaciados:
  \\ \medskip
  \begin{center}
    \begin{tikzpicture}[scale = 0.85]
      \begin{axis}[xmin = -1, xmax = 1]
        \addplot[samples = 201, smooth, red] {1 / (1 + 25 * x^2};
        \addplot[smooth, blue] file {runge.dat};
           % See runge.py
      \end{axis}
    \end{tikzpicture}
  \end{center}
\end{frame}

\begin{frame}
  \setcounter{beamerpauses}{2}
  \frametitle{El fenómeno de Runge}

  \uncover<+->{
    Aumentar el número de puntos aumenta el error.
  }
  \\ \medskip
  \uncover<+->{
    Hay que tener cuidado al usar polinomios de alto grado.
    Discutiremos el fenómeno en mayor detalle
    al discutir el error de interpolación.
  }
\end{frame}

\section{Error de interpolación}

\begin{frame}
  \setcounter{beamerpauses}{2}
  \frametitle{Error de interpolación}

  \uncover<+->{
    Llamaremos \(Q_n(x)\) al polinomio que interpola los \(n + 1\) puntos
    \((x_i, f(x_i))\) para \(0 \le i \le n\).

    El error de la interpolación en el punto \(x\) es:
    \begin{equation*}
      E
        = f(x) - Q_n(x)
    \end{equation*}
  }
\end{frame}

\begin{frame}
  \frametitle{Error de interpolación}

  \begin{theorem}
    Sea \(f \in C^{n+1}[a, b]\),
    y sea \(Q_n(x)\) el polinomio de grado \(n\)
    que interpola \(f\)
    en los puntos distintos \(x_0, \dotsc, x_n \in [a, b]\).
    Entonces para todo \(x \in [a, b]\)
    hay \(\zeta \in [a, b]\) tal que
    \begin{equation*}
      f(x) - Q_n(x)
        = \frac{1}{(n + 1)!} f^{(n+1)}(\zeta)
            \prod_{0 \le j \le n}(x - x_j)
    \end{equation*}
  \end{theorem}
\end{frame}

\begin{frame}
  \frametitle{Error de interpolación}

  \textbf{\large\textcolor{blue}{Demostración}}

  Fijemos \(x \in [a, b]\),
  diferente de todos los \(x_i\).
  Sea:
  \begin{equation*}
    \omega(x)
      = \prod_{0 \le j \le n}(x - x_j)
  \end{equation*}
  Note que \(\omega\) es mónico de grado \(n + 1\),
  \(\omega^{(n + 1)}(x) = (n + 1)!\).
  Definimos \(F(y) = f(y) - Q_n(y) - \lambda \omega(y)\),
  con \(\lambda\) tal que \(F(x) = 0\).
  La función \(F(y)\) tiene \(n + 2\) ceros diferentes en \([a, b]\).
  Por el teorema de Rolle extendido,
  \(F^{(n + 1)}(y)\) tiene un cero en \([a, b]\),
  llamémosle \(\zeta\):
  \begin{align*}
    F^{(n + 1)}(\zeta)
      &= f^{(n + 1)}(\zeta) - Q_n^{(n + 1)}(\zeta)
           - \lambda \omega^{(n + 1)}(\zeta) \\
    0
      &= f^{(n + 1)}(\zeta) - \lambda (n + 1)! \\
    F(y)
      &= f(y) - Q_n(y) - \frac{f^{(n + 1)}(\zeta)}{(n + 1)!} \omega(y)
  \end{align*}
\end{frame}

\begin{frame}
  \frametitle{Error de interpolación}

  Despejamos el error de \(F(x) = 0\):
  \begin{equation*}
    E
      = \frac{f^{(n + 1)}(\zeta)}{(n + 1)!} \omega(x)
  \end{equation*}
  \qed
\end{frame}

\begin{frame}
  \frametitle{Comportamiento de \(\omega\)}

  Para 9 puntos igualmente espaciados en \([-1, 1]\):
  \begin{center}
    \begin{tikzpicture}
      \begin{axis}[xmin = -1, xmax = 1]
        \addplot[samples = 201, smooth, color = red] file {omega.dat};
        \addlegendentry {\(100 \omega\)};
        \addplot[] {0};
      \end{axis}
    \end{tikzpicture}
  \end{center}
\end{frame}

\begin{frame}
  \frametitle{El fenómeno de Runge}

  El fenómeno de Runge se explica porque la función
  tiene derivadas muy grandes.
\end{frame}

\begin{frame}
  \frametitle{Nodos de Chebyshev}

  \uncover<+->{
    Nos interesaría minimizar el máximo error en \emph{todo} el rango.
    Para ello podemos elegir los puntos \(x_i\).
    Pero la \(n + 1\)\nobreakdash-ésima derivada de \(f\)
    evaluada en el punto \(\zeta\) que varía con \(x\)
    y depende de los \(x_i\) complica esto enormemente.
  }

  \uncover<+->{
    Podemos aproximar esto minimizando únicamente el factor \(\omega(x)\).
  }
\end{frame}

\begin{frame}
  \setcounter{beamerpauses}{2}
  \frametitle{Polinomios de Chebyshev}

  \uncover<+->{
    Los \emph{polinomios de Chebyshev} se definen por:
    \begin{align*}
      T_0(x)
        &= 1 \\
      T_1(x)
        &= x \\
      T_{n + 1}(x)
        &= 2 x T_n(x) - T_{n - 1}(x)
          \qquad n \ge 1
    \end{align*}
  }
  \uncover<+->{
    Resulta ser,
    para \(-1 \le x \le 1\):
    \begin{equation*}
      T_n(x)
        = \cos(n \, \arccos x)
    \end{equation*}
  }
\end{frame}

\begin{frame}
  \frametitle{Comportamiento de \(T_9(x)\)}

  \begin{center}
    \begin{tikzpicture}
      \begin{axis}[xmin = -1, xmax = 1]
        \addplot[samples = 201, smooth, color = red] file {chebyshev.dat};
        \addplot[] {0};
      \end{axis}
    \end{tikzpicture}
  \end{center}
\end{frame}

\begin{frame}
  \setcounter{beamerpauses}{2}
  \frametitle{Nodos de Chebyshev}
  \uncover<+->{
    Conviene usar nodos en los ceros de \(T_n\):
    \begin{equation*}
      x_i
        = \cos \frac{2 i + 1}{2 n + 2} \pi
        \quad 0 \le i \le n
    \end{equation*}
  }
  \uncover<+->{
    Resulta que con estos puntos
    en el rango \([-1, 1]\) es \(\omega(x) = 2^{-n} T_{n + 1}(x)\).
  }
\end{frame}

\begin{frame}
  \frametitle{Comparación igualmente espaciados con Chebyshev}

  \begin{center}
    \begin{tikzpicture}
      \begin{axis}[xmin = -1, xmax = 1]
        \addplot[samples = 201, smooth, color = red] file {chebyshev-s.dat};
        \addlegendentry {\(100 \cdot 2^{-8} T_9\)};
        \addplot[samples = 201, smooth, color = blue] file {omega.dat};
        \addlegendentry {\(100 \cdot \omega\)};
        \addplot[] {0};
      \end{axis}
    \end{tikzpicture}
  \end{center}
\end{frame}

\section{Resumen}

\begin{frame}
  \setcounter{beamerpauses}{2}
  \frametitle{Resumen}

  \begin{itemize}
  \item
    Planteamos el problema de interpolar por polinomios.
  \item
    Demostramos cuándo tiene solución única.
  \item
    Mostramos una forma simple de expresar el polinomio interpolante.
    \uncover<+->{
      El apunte describe formas adicionales,
      con ventajas en ciertas situaciones.
    }
  \item
    Ilustramos el fenómeno de Runge.
  \item
    Derivamos una fórmula para el error de interpolación
  \item
    Hallamos una explicación para el fenómeno de Runge
  \item
    Aprovechando la libertad de elegir los nodos,
    obtenemos una interpolación con error más uniforme
    al usar los nodos de Chebyshev
  \end{itemize}
\end{frame}
\end{document}

%%% Local Variables:
%%% mode: latex
%%% TeX-master: t
%%% ispell-local-dictionary: "spanish"
%%% End:

% LocalWords:  glyphtounicode graficar english cercha blue cont ésima
% LocalWords:  interpolante
