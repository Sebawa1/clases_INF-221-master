\input glyphtounicode%
\pdfgentounicode=1
\documentclass[english, spanish, fleqn,%
hyperref = {colorlinks, urlcolor = blue}%
%, handout%
]{beamer}

%% ]{article}
%% \usepackage{beamerarticle}

\usepackage{beamerthemesplit}

\usepackage[es-noquoting]{babel}
\usepackage{csquotes}
\usepackage{fourier}
\usepackage{amsmath}
\usepackage{amsthm}
\usepackage{dcolumn}
\usepackage[noline, noend]{algorithm2e}

\usefonttheme{professionalfonts}

\beamerdefaultoverlayspecification{<+->}

\title{Raíces cuadradas módulo un primo}

\author[Horst H. von Brand]{Horst H. von Brand\\
  \href{mailto:vonbrand@inf.utfsm.cl}{vonbrand@inf.utfsm.cl}}

\institute[DI UTFSM]{Departamento de Informática\\
                     Universidad Técnica Federico Santa María}
\date{}

\begin{document}
%%%
%%% For algorithm2e
%%%

\SetAlgorithmName{Algoritmo}{Algoritmo}{Índice de algoritmos}

\SetAlCapSty{mdseries}
\SetKwProg{Function}{function}{}{end}
\SetKwProg{Procedure}{procedure}{}{end}
\SetKw{Variables}{variables}
\SetKw{Downto}{downto}
\SetKwBlock{Loop}{loop}{end}
\SetKw{Continue}{continue}
\SetKw{Break}{break}
\SetKw{KwStep}{step}

\frame{\maketitle}

\begin{frame}<handout>
  \frametitle{Contenido}

  \tableofcontents
\end{frame}

\section{El problema}

\begin{frame}
  \setcounter{beamerpauses}{2}
  \frametitle{El problema}

  \uncover<+->{
    Los residuos módulo un primo \(p\) forman el campo \(\mathbb{Z}_p\).
  }
  \uncover<+->{
    En un campo,
    si \(x^2 = 1\) es \(x = \pm 1\).
  }

  \uncover<+->{
    Los elementos no cero de \(\mathbb{Z}_p\)
    forman su \emph{grupo de unidades}
    \(\mathbb{Z}_p^\times\).
    Este grupo es cíclico,
    con lo que si \(p\) es un primo impar
    módulo \(p\) exactamente la mitad de los elementos son cuadrados
    (\emph{residuos cuadráticos}).
  }
\end{frame}

\begin{frame}
  \setcounter{beamerpauses}{2}
  \frametitle{El problema}

  \uncover<+->{
    Interesa calcular raíces cuadradas en \(\mathbb{Z}_p\),
    o sea,
    resolver la ecuación \(x^2 = a\).
  }
  \\ \bigskip
  \uncover<+->{
    En adelante,
    \(p\) es un primo impar.
    Como la mayoría de las equivalencias son módulo \(p\),
    omitiremos mencionarlo en esos casos en aras de la brevedad.
  }
\end{frame}

\section{Teoría}

\begin{frame}
  \setcounter{beamerpauses}{2}
  \frametitle{Teoría}

  \uncover<+->{
    Sabemos que los elementos de \(\mathbb{Z}_p^\times\)
    cumplen la ecuación \(x^{p - 1} = 1\),
    con lo que para \(x \ne 0\)
    (criterio de Euler):
    \begin{equation*}
      x^{(p - 1) / 2}
        = \begin{cases}
              1 & \text{\(x\) es residuo cuadrático} \\
             -1 & \text{\(x\) no es residuo cuadrático}
          \end{cases}
    \end{equation*}
  }
  \medskip
  \uncover<+->{
    Como \(\mathbb{Z}_p^\times\) es cíclico,
    podemos expresar sus elementos como potencias de una \emph{raíz primitiva};
    son cuadrados exactamente las potencias pares.
    El producto de dos cuadrados o dos no\nobreakdash-cuadrados
    es un cuadrado,
    el producto de un cuadrado y un no\nobreakdash-cuadrado
    es un no\nobreakdash-cuadrado.
  }
\end{frame}

\begin{frame}
  \setcounter{beamerpauses}{2}
  \frametitle{El caso \(p \equiv 3 \pmod{4}\)}

  \uncover<+->{
    Si \(p \equiv 3 \pmod{4}\) hay una solución simple:
    escriba \(p = 4 k + 3\),
    y considere \(x = a^{k + 1}\).
  }
  \uncover<+->{
    Por el criterio de Euler:
  }
  \begin{align*}
    \uncover<+->{
      x^2
        &\equiv a^{2 k + 2} \\
    }
    \uncover<+->{
        &\equiv a^{2 k + 1} \cdot a^k \\
    }
    \uncover<+->{
        &\equiv a^{(p - 1) / 2} \cdot a	 \\
    }
    \uncover<+->{
        &\equiv a
    }
  \end{align*}
  \uncover<+->{
    Tenemos una fórmula para \(p \equiv 3 \pmod{4}\).
  }
\end{frame}

\begin{frame}
  \setcounter{beamerpauses}{2}
  \frametitle{El caso \(p \equiv 1 \pmod{4}\)}

  \uncover<+->{
    Si \(p \equiv 1 \pmod{4}\),
    \((p - 1) / 2\) es par.
    El truco anterior no funciona.
  }
\end{frame}

\begin{frame}
  \setcounter{beamerpauses}{2}
  \frametitle{El caso \(p \equiv 1 \pmod{4}\)}

  \uncover<+->{
    Sabemos \(a^{(p - 1) / 2} \equiv 1\),
    extrayendo raíz cuadrada sucesivamente
    (o sea,
     dividiendo el exponente por \(2\))
    obtendremos una secuencia de \(1\)
    hasta finalmente llegar ya sea a \(a^s \equiv 1\) con \(s\) impar,
    estamos listos,
    raíz es \(a^{(s + 1) / 2}\);
    o a \(a^s \equiv -1\) con \(s\) par.
  }
  \uncover<+->{
    En el segundo caso,
    si \(b\) es un residuo no cuadrático
    por el criterio de Euler es \(b^{(p - 1) / 2} \equiv -1\)
    y \(a^s b^{(p - 1) / 2} \equiv 1\).
  }
  \uncover<+->{
    El punto es que conocemos su raíz cuadrada
    \(a^{s / 2} b^{(p - 1) / 4}\),
    podemos reiniciar el proceso de disminuir el exponente de \(a\).
  }
  \uncover<+->{
    Como hemos llegado a \(s\)
    por sucesivas divisiones de \((p - 1) / 2\) por \(2\),
    no llegaremos a un exponente impar de \(b\) antes que de \(a\).
    Si tropezamos con \(-1\)
    nuevamente ajustamos multiplicando por \(b^{(p - 1) / 2}\).
  }
\end{frame}

\begin{frame}
  \setcounter{beamerpauses}{2}
  \frametitle{El caso \(p \equiv 5 \pmod{8}\)}

  \uncover<+->{
    En este caso \(2\) es residuo no cuadrático,
    los ahorramos la búsqueda de \(b\)
    y algo se simplifica el resto.
  }
\end{frame}

\section{El algoritmo}

% Filched from https://users.encs.concordia.ca/~chvatal/notes/sqrt.html;
% use b = 2 for p \equiv 5 \pmod{4}
% Check https://empslocal.ex.ac.uk/people/staff/rjchapma/courses/nt13/sqrt.pdf
% for the theory.

\begin{frame}
  \frametitle{Caso \(p \equiv 3 \pmod{4}\)}

  \small
  \begin{algorithm}[H]
    \Function{\(\mathrm{sqrt}(a, p)\)}{
      \uIf{\(p \equiv 3 \pmod{4}\)}{
        \Return \(a^{(p + 1) / 4} \bmod{p}\) \;
      }
      \uElseIf{\(p \equiv 5 \pmod{8}\)}{
        \Return \(\mathrm{sqrt\_5}(a, p)\) \;
      }
      \Else{
        \Return \(\mathrm{sqrt\_1}(a, p)\) \;
      }
    }
  \end{algorithm}
\end{frame}

\begin{frame}
  \frametitle{Caso \(p \equiv 5 \pmod{8}\)}

  \small
  \begin{algorithm}[H]
    \Function{\(\mathrm{sqrt\_5}(a, p)\)}{
      \eIf{\(a^{(p + 3) / 8} = 1\)}{
        \Return \(a^{(p + 3) / 8} \bmod{p}\) \;
      }
      {
        \Return \(2^{(p - 1) / 4} \cdot a^{(p + 3) / 8} \bmod{p}\) \;
      }
    }
  \end{algorithm}
\end{frame}

\begin{frame}
  \frametitle{Caso \(p \equiv 1 \pmod{8}\)}

  \small
  \begin{algorithm}[H]
    \DontPrintSemicolon

    \Function{\(\mathrm{sqrt\_1}(a, p)\)}{
      \(s \leftarrow (p - 1) / 4\); \(t \leftarrow (p - 1) / 2\) \;
      \While{\(a^s = 1\)}{
        \leIf{\(\mathrm{odd}(s)\)}{
          \Return \(a^{(s + 1) / 2} \bmod{p}\) \;
        }
        {
          \(s \leftarrow s / 2\) \;
        }
      }
      \(b \leftarrow \mathrm{select\_b}(p)\) \;
      \While{\(\neg \mathrm{odd}(s)\)}{
        \(s \leftarrow s / 2\); \(t \leftarrow t / 2\) \;
        \If{\(a^s \cdot b^t \equiv -1\)}{
          \(t \leftarrow t + (p - 1) / 2\) \;
        }
      }
      \Return \(a^{(s + 1) / 2} \cdot b^{t / 2} \bmod{p}\) \;
    }
  \end{algorithm}
\end{frame}

\begin{frame}
  \frametitle{Función auxiliar}

  \small
  \begin{algorithm}[H]
    \Function{\(\mathrm{select\_b}(p)\)}{
      \Repeat{\(b^{(p - 1) / 2} = -1\)}{
        Select \(b\) at random from \(1, \dotsc, p - 1\) \;
      }
      \Return \(b\) \;
    }
  \end{algorithm}
\end{frame}

\begin{frame}
  \setcounter{beamerpauses}{2}
  \frametitle{Análisis}

  \uncover<+->{
    Requerimos un no\nobreakdash-cuadrado,
    que buscamos al azar.
  }
  \uncover<+->{
    Como la mitad de los elementos de \(\mathbb{Z}_p^\times\) son
    no\nobreakdash-cuadrados,
    en promedio hallaremos uno al segundo intento.
  }
\end{frame}

\section{Resumen}

\begin{frame}
  \setcounter{beamerpauses}{2}
  \frametitle{Resumen}

  \begin{itemize}
  \item
    Este algoritmo es un ejemplo de uso de elementos elegidos al azar.
  \item
    Solo usa conceptos básicos de teoría de números,
    y el algoritmo en sí es simple.
  \item
    Hay formas más eficientes
    de determinar si un elemento es residuo cuadrático,
    se sabe por ejemplo que si \(p \not\equiv 1, 7 \pmod{8}\)
    \(2\) no es residuo cuadrático.
  \end{itemize}
\end{frame}
\end{document}

%%% Local Variables:
%%% mode: latex
%%% TeX-master: t
%%% ispell-local-dictionary: "spanish"
%%% End:

% LocalWords:  glyphtounicode mdseries Function function end Downto
% LocalWords:  Procedure procedure downto Loop loop Continue continue
% LocalWords:  Break break KwStep step Select at random from
