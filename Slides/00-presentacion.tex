\documentclass[english, spanish, fleqn%
hyperref = {colorlinks, urlcolor = blue}%
%, handout%
]{beamer}

\usepackage{ifpdf}

\usepackage{beamerthemesplit}

\usepackage{fourier}
\usepackage[group-digits = false, copy-decimal-marker = true]{siunitx}
\usepackage{amsmath}
\usepackage{amsthm}
\usepackage{babel}
\mode<article>{\usepackage[colorlinks, urlcolor = blue]{hyperref}}

\ifdefined\HCode
  \def\pgfsysdriver{pgfsys-dvisvgm4ht.def}
\fi

\usefonttheme{professionalfonts}

\beamerdefaultoverlayspecification{<+->}

\title{Algoritmos y Complejidad}
\subtitle{INF-221}

\author[Horst H. von Brand]{Horst H. von Brand\\
  \href{mailto:vonbrand@inf.utfsm.cl}{vonbrand@inf.utfsm.cl}}

\institute[DI UTFSM]{Departamento de Informática\\
                     Universidad Técnica Federico Santa María}
\date{}

\begin{document}
\frame{\maketitle}

\begin{frame}<handout>
  \frametitle{Contenido}

  \tableofcontents
\end{frame}

\section{Preliminares}

\begin{frame}
  \setcounter{beamerpauses}{2}
  \frametitle{Preliminares}

  \begin{itemize}
  \item
    Medio de comunicación es \href{http://aula.usm.cl}{Aula}.
    Anuncios importantes se harán en el foro del caso,
    hay espacio para discusión y aportes de alumnos.
  \item
    En el aula está publicado el apunte del ramo
    junto a las transparencias,
    en formato PDF.
    El apunte cubre bastante más que la materia del semestre,
    y profundiza mucho más ciertos aspectos.
    Los fuentes \LaTeX{} están en un repositorio
    \href{https://gitlab.labcomp.cl/vonbrand/clases_INF-221}{git},
    al que tienen acceso con su cuenta de Informática.
  \end{itemize}
\end{frame}

\section{Evaluación}

\begin{frame}
  \setcounter{beamerpauses}{2}
  \frametitle{Evaluación}

  \uncover<+->{
    La evaluación del ramo se compone de dos certámenes,
    quizzes y tareas.
    No hay examen.
  }

  \uncover<+->{
    El primer certamen cubre la materia hasta la primera prueba,
    con peso \SI{30}{\percent},
    el segundo cubre toda la materia
    (la segunda parte construye sobre la primera),
    con peso \SI{35}{\percent}.
  }

  \uncover<+->{
    El \SI{35}{\percent} restante es nota de ayudantía,
   \SI{10}{\percent} por quizzes/evaluación de ayudantías
    y \SI{25}{\percent} de tareas.
  }

  \uncover<+->{
    Únicamente para quienes hayan faltado a una evaluación
    hay un certamen recuperativo final,
    sobre toda la materia,
    que substituye la nota faltante.
  }
\end{frame}

\begin{frame}
  \setcounter{beamerpauses}{2}
  \frametitle{Evaluación}

  \uncover<+->{
    Certámenes generalmente tienen 125 puntos sobre 100,
    habrá temas más allá de lo exigido en el ramo
    que dan opción de nota extra.
  }
  \uncover<+->{
    (Si alguien termina con nota final mayor a 100,
     puede abonar el sobrante la próxima vez que tome el ramo.)
  }
\end{frame}

\begin{frame}
  \setcounter{beamerpauses}{2}
  \frametitle{Nota extra}

  \uncover<+->{
    El apunte y demás material del ramo es trabajo en curso,
    seguramente contiene errores,
    partes incomprensibles,
    áreas susceptibles de mejora.
    Aportes en forma de reportes de errores
    (de tipeo,
     gramática,
     matemáticos,
     algoritmos o programas,
     \ldots)
    tienen un premio de 5~puntos en la nota final
    (divididos entre los autores).
  }
  \uncover<+->{
    Asimismo,
    soluciones particularmente ingeniosas o elegantes en certámenes o tareas
    pueden reportar puntos adicionales.
  }
\end{frame}

\begin{frame}
  \setcounter{beamerpauses}{2}
  \frametitle{Nota extra}

  \uncover<+->{
    A veces se reescribe un capítulo del apunte,
    aparece un tema nuevo de gran interés,
    se profundiza algún tema ya tratado.
    Irregularmente se publicarán \emph{actividades extra},
    voluntarias.
    Participando en actividades extra
    opta a 10 puntos adicionales en la nota final.
  }
\end{frame}

\section{Temas a tratar}

\begin{frame}
  \setcounter{beamerpauses}{2}
  \frametitle{Advertencia previa}

  \uncover<+->{
    Haremos uso libremente de resultados de matemáticas
    (manipulaciones algebraicas y trigonométricas,
     series de potencias).
  }

  \uncover<+->{
    Usaremos específicamente resultados sobre solución de recurrencias,
    muchos algoritmos ameritan demostración de que son correctos
    (refiérase a \emph{Estructuras Discretas}, INF-152)
    y complejidad de algoritmos
    (vea \emph{Informática Teórica}, INF-155).
    Obviamente escribiremos programas,
    usando una diversidad de lenguajes
    (relevantes son \emph{Programación}, IWI-131
     y \emph{Lenguajes de Programación}, INF-253).
  }
\end{frame}

\begin{frame}
  \setcounter{beamerpauses}{2}
  \frametitle{Temática}

  \begin{description}
  \item[Revisión de asintóticas:]
    Hablaremos de tasas de crecimiento de tiempos de ejecución
    al crecer el tamaño del problema,
    la rapidez con la cual un método converge
    y otras cantidades difíciles de precisar.
    Análisis asintótico permite expresar y manipular
    las anteriores en forma simple.
  \item[Análisis numérico:]
    Una corta introducción a la temática básica.
    Cubriremos nociones preliminares,
    técnicas de búsqueda de ceros de funciones,
    interpolación y cuadratura.
  \item[Algoritmos combinatorios:]
    El fuerte del curso.
    Discutiremos técnicas para diseñar algoritmos eficientes
    en una variedad de áreas,
    mostrando cómo podemos estimar su rendimiento
    y compararlos.
  \end{description}
\end{frame}
\end{document}

%%% Local Variables:
%%% mode: latex
%%% TeX-master: t
%%% ispell-local-dictionary: "spanish"
%%% End:

% LocalWords:  glyphtounicode presentation PDF git quizzes faltante
% LocalWords:  tipeo INF IWI
