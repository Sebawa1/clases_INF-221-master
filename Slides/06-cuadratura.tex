\input glyphtounicode%
\pdfgentounicode=1
\documentclass[english, spanish, fleqn,%
hyperref = {colorlinks, urlcolor = blue}%
%, handout%
]{beamer}

%% ]{article}
%% \usepackage{beamerarticle}

\usepackage{beamerthemesplit}

\usepackage{fourier}
\usepackage{amsmath}
\usepackage{amsthm}
\usepackage[es-noquoting]{babel}
\usepackage{csquotes}
\usepackage{pgfplots}

\pgfplotsset{compat = newest}

\ifdefined\HCode
  \def\pgfsysdriver{pgfsys-dvisvgm4ht.def}
\fi

\usefonttheme{professionalfonts}

\beamerdefaultoverlayspecification{<+->}

\title{Cuadratura}

\author[Horst H. von Brand]{Horst H. von Brand\\
  \href{mailto:vonbrand@inf.utfsm.cl}{vonbrand@inf.utfsm.cl}}

\institute[DI UTFSM]{Departamento de Informática\\
                     Universidad Técnica Federico Santa María}
\date{}

\begin{document}
\frame{\maketitle}

\begin{frame}<handout>
  \frametitle{Contenido}

  \tableofcontents
\end{frame}

\section{Problema a resolver}

\begin{frame}
  \setcounter{beamerpauses}{2}
  \frametitle{El problema}

  \uncover<+->{
    En ciencia e ingeniería
    una de las tareas comunes
    (después de resolver ecuaciones de todo tipo)
    es calcular el valor de integrales definidas.
  }

  \uncover<+->{
    Calcular la integral es \emph{hallar el área bajo la curva},
    en geometría clásica
    esto se expresaba como \emph{hallar un cuadrado de la misma área},
    de allí el nombre de \emph{cuadratura}.
  }

  \uncover<+->{
    La triste realidad es que muy pocas integrales de interés
    pueden expresarse en fórmulas cerradas,
    hay que recurrir a aproximaciones.
  }
  \uncover<+->{
    Discutiremos técnicas basadas en interpolar
    e integrar	el polinomio resultante.
    Hay técnicas adicionales,
    con aplicaciones especializadas.
  }
\end{frame}

\section{Memorabilia}

\begin{frame}
  \frametitle{Memorabilia}

  \uncover<+->{
    \textbf{Teorema del valor intermedio:}

    Si \(f\) es continua en \([a, b]\),
    \selectlanguage{english}
    \(m = \min_{a \le x \le b}\{ f(x) \}\),
    \(M = \max_{a \le x \le b}\{ f(x) \}\),
    \selectlanguage{spanish}
    para todo \(m \le u \le M\) hay \(a \le \xi \le b\)
    tal que \(f(\xi) = u\).
  }
\end{frame}

\begin{frame}
  \frametitle{Memorabilia}

  \uncover<+->{
    \textbf{Teorema del valor intermedio de la derivada:}

    Si \(f'\) es continua en \([a, b]\),
    hay \(a < \xi < b \)
    tal que:
    \begin{equation*}
      f'(\xi)
        = \frac{f(b) - f(a)}{b - a}
    \end{equation*}
  }
\end{frame}

\begin{frame}
  \frametitle{Memorabilia}

  \uncover<+->{
    \textbf{Teorema del valor intermedio de la integral:}

    Si \(f\) es continua
    y \(g\) es integrable y no cambia de signo
    en \([a, b]\),
    hay \(a < \xi < b\) tal que:
    \begin{equation*}
      \int_a^b f(x) g(x) \, \mathrm{d} x
         = f(\xi) \int_a^b g(x) \,  \mathrm{d} x
    \end{equation*}
  }
  \uncover<+->{
    El caso \(g(x) = 1\) da:
    \begin{equation*}
      \int_a^b f(x) \, \mathrm{d} x
        = (b - a) f(\xi)
    \end{equation*}
  }
\end{frame}

\begin{frame}
  \frametitle{Memorabilia}

  \uncover<+->{
    \textbf{Teorema de Taylor,
            residuo en la forma de Lagrange:}

    \begin{align*}
      f(a + h)
        &= \sum_{0 \le k \le n} \frac{f^{(k)}(a)}{k!} h^k + R_{n + 1}(h) \\
      R_{n + 1}(h)
        &= \frac{f^{(n + 1)}(\xi)}{(n + 1)!} h^{n + 1}
            \qquad a < \xi < a + h
    \end{align*}
  }
\end{frame}

\begin{frame}
  \frametitle{Memorabilia}

  \uncover<+->{
    \textbf{Integración por partes:}

    \begin{equation*}
      \int u \mathrm{d} v
         = u v - \int v \mathrm{d} u
    \end{equation*}
  }
\end{frame}

\begin{frame}
  \setcounter{beamerpauses}{2}
  \frametitle{Planteo del problema}

  \uncover<+->{
    Supondremos dada una función \(f(x)\),
    queremos aproximar \(\int_a^b f(x) \, \mathrm{d} x\)
    dentro de alguna tolerancia \(\epsilon\).
  }
  \uncover<+->{
    Podemos elegir los puntos en los cuales evaluar \(f\) a conveniencia.
  }
  \uncover<+->{
    Inicialmente supondremos puntos igualmente espaciados.
    Llamaremos \(h\)~al espaciado,
    \(n\)~al número de intervalos
    (son \(n + 1\)~puntos).
  }
  \\ \medskip
  \uncover<+->{
    Obtendremos fórmulas de la forma general:
    \begin{equation*}
      \int_{x_0}^{x_n} f(x) \, \mathrm{d} x
        \approx \sum_{0 \le i \le n} A_i f(x_i)
    \end{equation*}
  }
\end{frame}

\section{Fórmulas simples}

\subsection{Rectángulos}

\begin{frame}
  \setcounter{beamerpauses}{2}
  \frametitle{Rectángulos}

  La idea más simple es aproximar el área
  mediante una \textquote{escalera}:
  \begin{equation*}
    \int_a^b f(x) \, \mathrm{d} x
      \approx h \sum_{0 \le k \le n} f(a + k h)
  \end{equation*}
  \uncover<+->{
    Esto corresponde a aproximar la función mediante un polinomio de grado~0
    (constante)
    en cada subintervalo,
    tomando el valor de \(f\) al principio del intervalo.
  }
  \\ \medskip
  \uncover<+->{
    Esta técnica simplista es sencilla de analizar,
    y ayudará como muestra de las técnicas del caso.
    No la presentamos para uso práctico.
  }
\end{frame}

\begin{frame}
  \setcounter{beamerpauses}{2}
  \frametitle{Rectángulos}

  \begin{center}
    \pgfimage[width = 0.85\textwidth]{rectangles}
  \end{center}
\end{frame}

\begin{frame}
  \setcounter{beamerpauses}{2}
  \frametitle{Análisis del error}

  \uncover<+->{
    Consideremos el intervalo \([x_i, x_{i + 1}]\),
    con \(x_{i + 1} - x_i = h\).
    Expresamos la integral \(F\) mediante serie de Taylor
    con residuo en la forma de Lagrange,
    usando el teorema fundamental del cálculo:
    \begin{align*}
      F(x)
        &= \int_{x_i}^x f(t) \, \mathrm{d} t \\
        &= F(x_i) + F'(x_i) (x - x_i) + \frac{1}{2} F''(\xi_i) (x - x_i)^2 \\
        &= 0 + f(x_i) (x - x_i) + \frac{1}{2} f'(\xi_i) (x - x_i)^2
    \end{align*}
    Acá \(x_i < \xi_i < x_{i + 1}\).
  }
  \uncover<+->{
    El error del intervalo es:
    \begin{align*}
      E_i
        &= F(x_{i + 1}) - f(x_i) h \\
        &= \frac{1}{2} f'(\xi_i) h^2
    \end{align*}
  }
\end{frame}

\begin{frame}
  \setcounter{beamerpauses}{2}
  \frametitle{Análisis del error}

  \uncover<+->{
    Debemos sumar sobre los \(n\) intervalos:
    \begin{align*}
      E
        &= \sum_{0 \le i < n} E_i \\
        &= \frac{h^2}{2} \sum_{0 \le i < n}f'(\xi_i)
    \end{align*}
  }
  \uncover<+->{
    Si \(m \le f'(x) \le M\) en \([a, b]\) y son \(n\) intervalos:
    \begin{alignat*}{3}
      &\frac{h^2}{2} \cdot n m
        &{} \le {}& \sum_{0 \le i < n} f'(\xi_i)
        &{} \le {}& \frac{h^2}{2} \cdot n M \\
      &\frac{(b - a) m}{2} h
        &{} \le {}& \sum_{0 \le i < n} f'(\xi_i)
        &{} \le {}& \frac{(b - a) M}{2} h
    \end{alignat*}
  }
\end{frame}

\begin{frame}
  \setcounter{beamerpauses}{2}
  \frametitle{Análisis del error}

  \uncover<+->{
    Si \(f'\) es continua en el intervalo \([a, b]\),
    vemos que hay \(\xi \in [a, b]\) tal que:
    \begin{equation*}
      \sum_{0 \le i < n} f'(\xi_i)
        = n f'(\xi)
    \end{equation*}
  }
  \uncover<+->{
    Concluimos que el error cumple:
    \begin{equation*}
      E
        = \frac{f'(\xi) (b - a) h}{2}
    \end{equation*}
  }
\end{frame}

\begin{frame}
  \setcounter{beamerpauses}{2}
  \frametitle{Variante: Punto medio}

  \uncover<+->{
    Un problema de la técnica anterior es patente
    si la función es monótona
    (creciente o decreciente):
    el valor en el punto inicial sub- o sobre-estima el valor en el intervalo.
  }
  \uncover<+->{
    Intuitivamente,
    dará mejor resultado en general usar el punto medio:
    \begin{equation*}
      \int_a^b f(x) \, \mathrm{d} x
        \approx \frac{b - a}{n}
                   \sum_{0 \le i < n} f\left( \frac{x_i + x_{i + 1}}{2} \right)
    \end{equation*}
    (el \textquote{triángulo} que sobra en un extremo del subintervalo
     se compensa aproximadamente con el que falta en el otro).
  }
\end{frame}

\begin{frame}
  \setcounter{beamerpauses}{2}
  \frametitle{Punto medio}

  \begin{center}
    \pgfimage[width = 0.85\textwidth]{midpoints}
  \end{center}
\end{frame}

\begin{frame}
  \setcounter{beamerpauses}{2}
  \frametitle{Análisis del error}

  \uncover<+->{
    La misma idea general que usamos para analizar rectángulos.
    Solo que ahora deberemos combinar varias estimaciones,
    por lo que describiremos errores mediante cotas asintóticas.
  }

  \uncover<+->{
    Vemos que esta técnica es exacta para polinomios de grado \(1\),
    por lo que extendemos las series a términos cúbicos.
  }
\end{frame}

\begin{frame}
  \setcounter{beamerpauses}{2}
  \frametitle{Análisis del error}

  \uncover<+->{
    Con \(d = x - x_i\),
    expandiendo \(F(x)\) por serie de Taylor:
    \begin{align*}
      F(x)
        &= F(x_i)
              + F'(x_i) d
              + \frac{F''(x_i)}{2} d^2
              + \frac{F'''(x_i)}{6} d^3
              + O(d^4) \\
        &= 0
             + f(x_i) d
             + \frac{1}{2} f'(x_i) d^2
             + \frac{1}{6} f''(x_i) d^3
          + O(d^4) \\
      F(x_{i + 1})
        &= f(x_i) h
             + \frac{1}{2} f'(x_i) h^2
             + \frac{1}{6} f''(x_i) h^3
             + O(h^4)
    \end{align*}
  }
\end{frame}

\begin{frame}
  \setcounter{beamerpauses}{2}
  \frametitle{Análisis del error}

  \uncover<+->{
    El error del intervalo es:
  }
  \begin{align*}
    \uncover<.->{
      E_i
        &= \int_{x_i}^{x_{i + 1}} f(x)\, \mathrm{d} x
          - h f\left( \frac{x_i + x_{i + 1}}{2} \right)
    }
    \uncover<+->{
      \intertext{Expandimos el segundo término por Taylor:}
        &= F(x_{i + 1})
          - h \left(
                 f(x_i)
                   + f'(x_i) \frac{h}{2}
                   + \frac{1}{2} f''(x_i) \frac{h^2}{4}
                   + O(h^3)
               \right) \\
    }
    \uncover<+->{
        &= \frac{1}{24} f''(x_i) h^3 + O(h^4)
    }
  \end{align*}
\end{frame}

\subsection{Trapezoides}

\begin{frame}
  \setcounter{beamerpauses}{2}
  \frametitle{Trapezoides}

  \uncover<+->{
    El siguiente paso es aproximar la función
    por rectas que pasan por \((x_i, f(x_i))\) y \((x_{i + 1}, f(x_{i + 1}))\).
  }
  \uncover<+->{
    La figura geométrica resultante
    (un cuadrilátero con dos lados paralelos
     y uno de los lados entre ellos en ángulo recto)
     se llama \emph{trapezoide}.
  }
\end{frame}

\begin{frame}
  \setcounter{beamerpauses}{2}
  \frametitle{Trapezoides}

  \begin{center}
    \pgfimage[width = 0.85\textwidth]{trapezoids}
  \end{center}
\end{frame}

\begin{frame}
  \setcounter{beamerpauses}{2}
  \frametitle{Trapezoides}

  \uncover<+->{
    El área del trapezoide formado por las rectas \(x = x_i\),
    \(x = x_{i + 1}\),
    el eje \(X\)
    y la recta que une
    los puntos \((x_i, f(x_i))\) y \((x_{i + 1}, f(x_{i + 1}))\)
    es simplemente:
    \begin{equation*}
      \frac{(x_{i + 1} - x_i) (f(x_i) + f(x_{i + 1}))}{2}
    \end{equation*}
  }
  \uncover<+->{
    La regla compuesta es:
    \begin{equation*}
       \int_a^b f(x) \mathrm{d} x
         \approx \frac{b - a}{2 n}
                    \left(
                      (f(x_0) + f(x_n)) + 2 \sum_{0 < i < n} f(x_i)
                    \right)
     \end{equation*}
  }
\end{frame}

\begin{frame}
  \frametitle{Análisis del error}

  \uncover<+->{
    Consideremos un intervalo,
    tenemos para el error:
    \begin{equation*}
      E_i
        = \int_{x_i}^{x_{i + 1}} f(x) \mathrm{d} x
            - \frac{x_{i + 1} - x_i}{2} (f(x_i) + f(x_{i + 1}))
    \end{equation*}
  }
  \uncover<+->{
    Sea \(c\) el centro del intervalo:
    \begin{equation*}
      c
        = \frac{x_i + x_{i + 1}}{2}
    \end{equation*}
    con lo que:
    \begin{equation*}
      x_{i + 1} - c
        = c - x_i
        = \frac{x_{i + 1} - x_i}{2}
    \end{equation*}
  }
\end{frame}

\begin{frame}
  \frametitle{Análisis del error}

  La expresión para el error
  corresponde a integración por partes
  (tome \(u = f(x)\),
   \(\mathrm{d} v = \mathrm{d} (x - c) = \mathrm{d} x\)):
  \begin{equation*}
    -E_i
      = \int_{x_i}^{x_{i + 1}} (x - c) f'(x) \mathrm{d} x
  \end{equation*}
  Nuevamente integrando por partes,
  como \(x_{i + 1} - c = c - x_i = (x_{i + 1} - x_i) / 2\):
  \begin{align*}
    -E_i
      &= \left. \frac{1}{2}(x - c)^2 f'(x) \right|_{x_i}^{x_{i + 1}}
           - \frac{1}{2} \int_{x_i}^{x_{i + 1}}
               (x - c)^2 f''(x) \, \mathrm{d} x \\
      &= \frac{(x_{i + 1} - x_i)^2}{4} (f'(x_{i + 1}) - f'(x_i))
           - \frac{1}{2} \int_{x_i}^{x_{i + 1}}
               (x - c)^2 f''(x) \, \mathrm{d} x
  \end{align*}
\end{frame}

\begin{frame}
  \frametitle{Análisis del error}

  \uncover<+->{
    El primer término
    es una constante por la integral de \(f''(x)\):
    \begin{align*}
      -E_i
        &= \frac{1}{2}
             \int_{x_i}^{x_{i + 1}}
                      \left(
                        \left(
                          \frac{x_{i + 1} - x_i}{2}
                        \right)^2
                          - (x - c)^2
                      \right) f''(x) \mathrm{d} x
    \end{align*}
  }
  \uncover<+->{
    El polinomio bajo la integral no cambia de signo en \([x_i, x_{i + 1}]\),
    suponiendo \(f''\) continua en el intervalo
    podemos aplicar el teorema del valor intermedio de la integral,
    que nos asegura que hay \(\xi_i \in [x_i, x_{i + 1}]\) tal que:
  }
  \begin{align*}
    \uncover<.->{
      -E_i
        &= \frac{1}{2} f''(\xi_i)
                     \int_{x_i}^{x_{i + 1}}
                        \left(
                          \left(
                            \frac{x_{i + 1} - x_i}{2}
                          \right)^2
                            - (x - c)^2
                      \right) \mathrm{d} x \\
    }
    \uncover<+->{
        &= f''(\xi_i) \frac{(x_{i + 1} - x_i)^3}{12} \\
    }
    \uncover<+->{
      E_i
        &= -\frac{f''(\xi_i)}{12} h^3
    }
  \end{align*}
\end{frame}

\begin{frame}
  \frametitle{Análisis del error}

  \uncover<+->{
    Nuevamente tenemos para el intervalo completo,
    para algún \(\xi \in [a, b]\):
    \begin{align*}
      E
        &= \sum_{0 \le i < n} E_i \\
        &= -\frac{h^3}{12}\sum_{0 \le i < n} f''(\xi_i) \\
        &= -\frac{(b - a) h^2 f''(\xi)}{12}
    \end{align*}
  }
\end{frame}

\subsection{Regla de Simpson}

\begin{frame}
  \setcounter{beamerpauses}{2}
  \frametitle{Regla de Simpson}

  \uncover<+->{
    Siguiente paso es interpolar mediante una cuadrática.
  }
\end{frame}

\begin{frame}
  \setcounter{beamerpauses}{2}
  \frametitle{Simpson}

  Consideremos un intervalo:
  \begin{center}
    \pgfimage[width = 0.8\textwidth]{quadratic}
  \end{center}
\end{frame}

\begin{frame}
  \setcounter{beamerpauses}{2}
  \frametitle{Regla de Simpson}

  \uncover<+->{
    Claro que integrar el polinomio en forma de Lagrange es feo.
  }
  \uncover<+->{
    Pero si la fórmula es exacta para todos los polinomios de grado 2,
    será exacta para polinomios de grados 0, 1 y 2.
    Planteamos un sistema de ecuaciones,
    usando los polinomios simples \(1, x, x^2\)
    con puntos \(-1, 0, 1\) por simetría.
  }
  \uncover<+->{
    Nuestra fórmula de cuadratura será:
    \begin{equation*}
      \int_{-1}^1 f(x) \, \mathrm{d} x
        \approx A_{-1} f(-1) + A_0 f(0) + A_{+1} f(1)
    \end{equation*}
  }
\end{frame}

\begin{frame}
  \setcounter{beamerpauses}{2}
  \frametitle{Regla de Simpson}

  \uncover<+->{
    Planteamos el sistema de ecuaciones:
    \begin{alignat*}{4}
      &\int_{-1}^1 1 \, \mathrm{d} x
         &{} = {}& 2
         && = A_{-1}\cdot 1	   & + A_0 \cdot 1   & + A_{+1} \cdot 1 \\
      &\int_{-1}^1 x \, \mathrm{d} x
         &{} = {}& 0
         && = A_{-1} \cdot (-1)	   & + A_0 \cdot 0   & + A_{+1} \cdot 1	 \\
      &\int_{-1}^1 x^2 \, \mathrm{d} x
         &{} = {}& \frac{2}{3}
         && = A_{-1}\cdot (-1)^2    & + A_0 \cdot 0^2 & + A_{+1} \cdot 1^2
    \end{alignat*}
  }
  \uncover<+->{
    Solución es \(A_{-1} = A_{+1} = 1 / 3\), \(A_0 = 4 / 3\).
  }
\end{frame}

\begin{frame}
  \setcounter{beamerpauses}{2}
  \frametitle{Regla de Simpson}

  \uncover<+->{
    La regla compuesta,
    para un número impar \(n\) de puntos resulta ser:
  }
  \begin{align*}
    \uncover<+->{
    \int_a^b f(x) \, \mathrm{d} x
       &= \sum_{0 \le k \le n / 2}
            \left(
              \frac{1}{3} f(x_{2 k})
                + \frac{4}{3} f(x_{2 k + 1})
                + \frac{1}{3} f(x_{2 k + 2})
            \right) \\
    }
    \uncover<+->{
       &= \frac{2}{3} \sum_{0 \le k \le n / 2} f(x_{2 k})
            + \frac{4}{3} \sum_{0 \le k \le n / 2} f(x_{2 k + 1})
            - \frac{1}{3} f(x_0) - \frac{1}{3} f(x_n)
    }
  \end{align*}
\end{frame}

\begin{frame}
  \setcounter{beamerpauses}{2}
  \frametitle{Regla de Simpson}

  \uncover<+->{
    Vemos que también se cumple:
    \begin{alignat*}{4}
      &\int_{-1}^1 x^3 \, \mathrm{d} x
         &{} = {}& 0
         && = A_{-1}\cdot(-1)^3 & + A_0 \cdot 0^3 & + A_{+1} \cdot 1^3
    \end{alignat*}
  }
  \uncover<+->{
    Vale decir,
    la regla de Simpson es exacta para polinomios hasta grado 3,
    cuando teníamos derecho de esperar exacta solo hasta grado 2.
    El mismo fenómeno ocurrirá con todas las reglas
    con un número impar de puntos
    distribuidos simétricamente.
  }
  \uncover<+->{
    Ya habíamos visto esto con la regla del punto medio.
  }
\end{frame}

\begin{frame}
  \setcounter{beamerpauses}{2}
  \frametitle{Análisis del error}

  \begin{theorem}
    Sea \(f \in C^4 [a, b]\),
    llamamos \(h = (b - a) / 2\),
    \(x_0 = a\), \(x_1 = x_0 + h\), \(x_2 = b\).
    Hay \(\xi \in [a, b]\) tal que el error de la regla de Simpson cumple:
    \begin{equation*}
      E
        = \int_a^b f(x) \, \mathrm{d} x
             - \frac{h}{3} (f(x_0) + 4 f(x_1) + f(x_2))
        = - \frac{h^5}{90} f^{(4)}(\xi)
    \end{equation*}
  \end{theorem}
\end{frame}

\begin{frame}
  \frametitle{Análisis del error}

  \textbf{\large\textcolor{blue}{Demostración}}

  Cambie variable \(x(t) = x_1 + h t\),
  \(t \in [-1, 1]\).
  Defina \(F(t) = f(x(t))\):
  \begin{equation*}
    \int_a^b f(x) \, \mathrm{d} x
      = h \int_{-1}^1 F(\tau) \, \mathrm{d} \tau
  \end{equation*}
  y el error es:
  \begin{align*}
    E
      &= \int_a^b f(x) \, \mathrm{d} x
           - \frac{h}{3} (f(x_0) + 4 f(x_1) + f(x_2)) \\
      &= h \left(
              \int_{-1}^1 F(\tau) \, \mathrm{d} \tau
                - \frac{1}{3} \left( F(-1) + 4 F(0) + F(1) \right)
           \right)
  \end{align*}
\end{frame}

\begin{frame}
  \frametitle{Análisis del error}

  Para \(t \in [-1, 1]\) defina:
  \begin{equation*}
    G(t)
      = \int_{-t}^t F(\tau) \, \mathrm{d} \tau
          - \frac{1}{3} \left( F(-t) + 4 F(0) + F(t) \right)
  \end{equation*}
  El error que buscamos es \(E = h G(1)\).
  Consideremos \(H(t) = G(t) - t^5 G(1)\).
  Como \(H(0) = H(1) = 0\),
  por el teorema de Rolle hay \(\xi_1 \in (0, 1)\)
  tal que \(H'(\xi_1) = 0\).
  Como también \(H'(0) = 0\),
  hay \(\xi_2 \in (0, \xi_1) \subset (0, 1)\) tal que \(H''(\xi_2) = 0\),
  y de la misma forma hay \(\xi_3 \in (0, \xi_2) \subseteq (0, 1)\)
  tal que \(H'''(\xi_3) = 0\).
\end{frame}

\begin{frame}
  \frametitle{Análisis del error}

  La tercera derivada de \(G\) es:
  \begin{equation*}
    G'''(t)
      = \frac{t}{3} (F'''(t) - F'''(-t))
  \end{equation*}
  de donde:
  \begin{equation*}
    H'''(\xi_3)
      = - \frac{\xi_3}{3} (F'''(\xi_3) - F'''(- \xi_3)) - 60 \xi_3^2 G(1)
      = 0
  \end{equation*}
  Dividiendo por \(2 \xi_3^2 = \xi_3^2 - (- \xi_3)^2\),
  legal ya que \(\xi_3 \ne 0\):
  \begin{equation*}
    \frac{F'''(\xi_3) - F'''(-\xi_3)}{\xi_3 - (- \xi_3)}
      = - 90 G(1)
  \end{equation*}
\end{frame}

\begin{frame}
  \frametitle{Análisis del error}

  Por el teorema del valor intermedio de la derivada
  hay \(\xi_4 \in (-\xi_3, \xi_3) \subset (-1, 1)\) tal que:
  \begin{equation*}
    F^{(4)}(\xi_4)
      = - 90 G(1)
  \end{equation*}
  y el error:
  \begin{align*}
    E
      &= - \frac{h}{90} F^{(4)}(\xi_4) \\
      &= - \frac{h^5}{90} f^{(4)}(x_1 + h \xi_4) \\
      &= - \frac{h^5}{90} f^{(4)}(\xi)
  \end{align*}
  donde \(\xi = x_1 + h \xi_3 \in (a, b)\).
  \qed
\end{frame}

\begin{frame}
  \frametitle{Análisis del error}

  \uncover<+->{
    El caso más simple es dos intervalos,
    \(n = 2\) y \(h = (x_{i + 2} - x_i) / 2\):
    \begin{equation*}
      E_i
        = -\frac{(x_{i + 2} - x_i) h^4 f^{(4)}(\xi_i)}{180}
    \end{equation*}
  }
  \uncover<+->{
    Nuevamente tenemos para el intervalo completo,
    para algún \(\xi \in [a, b]\)
    \begin{align*}
      E
        &= \sum_i E_i \\
        &= -\frac{(b - a) h^4}{n \cdot 90}
              \sum_{0 \le i < n / 2} f''(\xi_{2 i}) \\
        &= -\frac{(b - a) h^4 f''(\xi)}{180}
    \end{align*}
  }
\end{frame}

\section{Fórmulas de orden mayor}

\begin{frame}
  \setcounter{beamerpauses}{2}
  \frametitle{Fórmulas de grado mayor}

  \uncover<+->{
    Podemos definir fórmulas similares interpolando polinomios de grado mayor.
  }
  \uncover<+->{
    Pero vimos el fenómeno de Runge,
    puntos igualmente espaciados pueden llevar a enormes errores
    del polinomio interpolador.
    Por esto suelen usarse fórmulas compuestas
    de fórmulas como trapezoides o Simpson.
  }
  \uncover<+->{
    Otra opción es usar nodos de Chebyshev,
    que garantizan error más bien parejo en el intervalo.
  }

  \uncover<+->{
    Una opción mejor es usar fórmulas de Gauß,
    nuestro tema siguiente.
  }
\end{frame}

\section{Aceleración de la convergencia}

\begin{frame}
  \frametitle{Aceleración de Richardson}

  \uncover<+->{
    Supongamos que tenemos una aproximación a una cantidad \(A^*\)
    que cumple:
    \begin{equation*}
      A(h)
        = A^* + C h^n + O(h^{n + 1})
    \end{equation*}
  }
  \uncover<+->{
    Omitiendo el término del error,
    calculando \(A(h)\) para dos valores de \(h\)
    obtenemos un sistema de ecuaciones para \(A^*\) y \(C\),
    resolviendo para \(A^*\) obtenemos una aproximación mejor.
  }
  \uncover<+->{
    \begin{align*}
      A(h)
        &= A^* + C h^n \\
      A(h/t)
        &= A^* + C (h/t)^n
    \end{align*}
    Obtenemos una nueva aproximación \(A^+\):
    \begin{equation*}
      A^+
        = \frac{t^n A(h/t) - A(h)}{t^n - 1}
    \end{equation*}
  }
\end{frame}

\begin{frame}
  \frametitle{Análisis del error}

  Vemos que:
  \begin{align*}
    A^+
      &= \frac{t^n (A^* + C (h/t)^n + O(h^{n + 1}))
                 - (A^* + C h^n + O(h^{n + 1}))}
              {t^n - 1} \\
      &= A^* + O(h^{n + 1})
  \end{align*}
  Usando dos valores de \(h\) aumentamos el orden del método.
  Incluso nos da una estimación del error en \(A^+ - A(h / t)\).
\end{frame}

\begin{frame}
  \setcounter{beamerpauses}{2}
  \frametitle{Integración de Romberg}

  \uncover<+->{
    Una aplicación es integración de Romberg:
    aproxima la integral mediante trapezoides
    para \(h\) y \(h/2\),
    y usa extrapolación de Richardson
    con \(t = 2\) para obtener una aproximación mejor.
    Pueden reutilizarse nodos,
    el trabajo requerido es solo aproximadamente el para el paso menor.
  }
  \uncover<+->{
    Si la estimación del error resultante es demasiado grande,
    disminuimos el paso a la mitad
    (calculando la función para puntos intermedios).
  }

  \uncover<+->{
    Esto puede extenderse a \emph{cuadratura adaptativa}:
    vamos monitoreando el error en cada subintervalo,
    deteniendo el refinamiento en este
    cuando hemos alcanzado la precisión requerida para él.
    Esto es aplicable a funciones con comportamientos distintos
    en el intervalo de interés.
  }
\end{frame}

\section{Resumen}

\begin{frame}
  \setcounter{beamerpauses}{2}
  \frametitle{Resumen}

  \begin{itemize}
  \item
    Definimos el problema de cuadratura,
    y las técnicas que consideramos.
  \item
    Planteamos aproximar por polinomios de grado 0, 1 y 2,
    y derivamos el error de esas reglas de cuadratura.
  \item
    Aceleración de Richardson es una técnica de aplicación general.
    Aplicada a trapezoides lleva a la cuadratura de Romberg.
  \item
    Podemos refinar a cuadratura adaptativa.
  \end{itemize}
\end{frame}
\end{document}

%%% Local Variables:
%%% mode: latex
%%% TeX-master: t
%%% ispell-local-dictionary: "spanish"
%%% End:

% LocalWords:  glyphtounicode subintervalo sub blue monitoreando
% LocalWords:  Memorabilia width rectangles midpoints trapezoids
% LocalWords:  quadratic handout
