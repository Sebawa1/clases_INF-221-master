\input glyphtounicode%
\pdfgentounicode=1
\documentclass[english, spanish, fleqn,%
hyperref = {colorlinks, urlcolor = blue}%
%, handout%
]{beamer}
%\usepackage{beamerarticle}

\usepackage{ifpdf}

\usepackage{beamerthemesplit}

\usepackage{fourier}
\usepackage{amsmath}
\usepackage{amsthm}
\usepackage[group-digits = false, copy-decimal-marker = true,
            exponent-product = \cdot]{siunitx}
\usepackage{enumitem}
\usepackage{babel}
\usepackage{csquotes}
\usepackage{listings}
\mode<article>{\usepackage[colorlinks, urlcolor = blue]{hyperref}}

\ifdefined\HCode
  \def\pgfsysdriver{pgfsys-dvisvgm4ht.def}
\fi

\usefonttheme{professionalfonts}

\lstloadlanguages{[ANSI]C, sh}

\beamerdefaultoverlayspecification{<+->}

\title{Propagación de errores}

\author[Horst H. von Brand]{Horst H. von Brand\\
  \href{mailto:vonbrand@inf.utfsm.cl}{vonbrand@inf.utfsm.cl}}

\institute[DI UTFSM]{Departamento de Informática\\
                     Universidad Técnica Federico Santa María}
\date{}

\begin{document}
\lstset{basicstyle = \small\sffamily, commentstyle = \small\slshape,
        tabsize = 4,
        frame = lines, numbers = none,
        showstringspaces = false,
        xleftmargin = 1em, xrightmargin = 1em
       }

\renewcommand{\lstlistingname}{Listado}

\frame{\maketitle}

\begin{frame}<handout>
  \frametitle{Contenido}

  \tableofcontents
\end{frame}

\section{Introducción}

\begin{frame}
  \frametitle{Fuentes de error}

  \uncover<+->{
    Para obtener valores numéricos se construye un modelo,
    se extraen relaciones relevantes,
    y se resuelven las ecuaciones resultantes.
    Hay varias fuentes de error en esto:
  }
  \begin{enumerate}[label={(\roman*)}]
  \item
    \label{enum:error:fisico}
    El modelo físico es una simplificación de la realidad
  \item
    \label{enum:error:parametros}
    Los parámetros que entran al modelo se conocen con precisión finita
  \item
    \label{enum:error:aproximacion}
    Aproximaciones usadas para resolver el modelo matemático
  \item
    \label{enum:error:redondeo}
    Errores de redondeo en los cálculos
  \end{enumerate}
  \uncover<+->{
    El punto~\ref{enum:error:fisico} es de modelamiento matemático,
    \ref{enum:error:parametros} depende de mediciones del experimento,
    \ref{enum:error:aproximacion} es del cálculo
    mientras el punto~\ref{enum:error:redondeo} viene de representar
    infinitos números reales en espacio finito.
    Acá interesan los puntos~\ref{enum:error:aproximacion}
    y~\ref{enum:error:redondeo}.
  }
\end{frame}

\begin{frame}
  \setcounter{beamerpauses}{2}
  \frametitle{Error absoluto y relativo}

  \uncover<+->{
    Sea \(x^*\) el valor exacto de alguna cantidad,
    \(x\) una aproximación.
  }

  \uncover<+->{
    El \emph{error absoluto} de \(x\) es \(\lvert x - x^* \rvert\),
    el \emph{error relativo} es \(\lvert (x - x^*) / x^* \rvert\)
    (siempre que \(x^* \ne 0\),
     claro).
  }

  \uncover<+->{
    Si el error es \textquote{suficientemente pequeño}
    el error en la cantidad \(f(x^*)\) al calcular \(f(x)\)
    es aproximadamente (nuevamente serie de Taylor):
    \begin{equation*}
      \lvert f(x) - f(x^*) \rvert
        \approx \lvert f'(x^*) \rvert \cdot \lvert x - x^* \rvert
    \end{equation*}
  }

  \uncover<+->{
    La maquinaria de asintóticas permite expresar esto más precisamente,
    de ser necesario.
  }
\end{frame}

\section{Punto flotante}

\begin{frame}
  \setcounter{beamerpauses}{2}
  \frametitle{Punto flotante}

  \uncover<+->{
    Formato y operaciones definidas por IEEE~745,
   también estándar internacional,
   de forma que las operaciones cumplen una regla simple:
   \begin{equation*}
     \mathrm{fl}(x)
       = x (1 + \delta)
       \quad  \lvert \delta \rvert \le \varepsilon
   \end{equation*}
   Acá \(\varepsilon\) describe la precisión de los valores,
   se le conoce como \emph{epsilon de la máquina}.
   En el caso de C,
   en \lstinline[language = sh]!float.h!
   define \lstinline[language = C]!FLT_EPSILON!
   para \lstinline[language = C]{float}
   y \lstinline[language = C]!DBL_EPSILON!
   para \lstinline[language = C]{double};
   para C++ están en el encabezado \lstinline[language = sh]!limits!.
 }
\end{frame}

\begin{frame}
  \setcounter{beamerpauses}{2}
  \frametitle{Punto flotante --- multiplicación y división}

  \uncover<+->{
    Tenemos por ejemplo:
   \begin{align*}
     \mathrm{fl}(\operatorname{fl}(x) \cdot \operatorname{fl}(y))
       &= \mathrm{fl}(x (1 + \delta_x) \cdot y (1 + \delta_y)) \\
       &= (x (1 + \delta_x) \cdot y (1 + \delta_y)) (1 + \delta_{x y}) \\
   \end{align*}
   El error relativo cumple:
   \begin{align*}
     \delta_x + \delta_y + \delta_{x y}
        + \delta_x \delta_y
            + \delta_x \delta_{x y}
            + \delta_y \delta_{x y}
        + \delta_x \delta_y \delta_{x y}
        \le 3 \varepsilon + 3 \varepsilon^2 + \varepsilon^3
   \end{align*}
 }
\end{frame}

\begin{frame}
  \setcounter{beamerpauses}{2}
  \frametitle{Punto flotante --- suma y resta}

  \uncover<+->{
    El caso de sumas y restas es diferente:
    \begin{align*}
      \mathrm{fl}(\operatorname{fl}(x) + \operatorname{fl}(y))
        &= \mathrm{fl}(x (1 + \delta_x) + y (1 + \delta_y)) \\
        &= (x (1 + \delta_x) + y (1 + \delta_y)) (1 + \delta_{x + y}) \\
        &= x + y
             + x (\delta_x + \delta_{x + y} + \delta_x \delta_{x + y})
             + y (\delta_y + \delta_{x + y} + \delta_y \delta_{x + y})
    \end{align*}
    El error absoluto cumple:
    \begin{equation*}
      x (\delta_x + \delta_{x + y} + \delta_x \delta_{x + y})
        + y (\delta_y + \delta_{x + y} + \delta_y \delta_{x + y})
        \le (\lvert x \rvert + \lvert y \rvert) (2 \varepsilon + \varepsilon^2)
    \end{equation*}
    Si ambos operandos son del mismo signo,
    el error relativo es aproximadamente \(2 \varepsilon\);
    si tienen signo opuesto puede producirse \emph{cancelación catastrófica}.
  }
\end{frame}

\begin{frame}
  \setcounter{beamerpauses}{2}
  \frametitle{ Ejemplo}

  \uncover<+->{
    Consideremos el polinomio:
    \begin{equation*}
      p(x)
        = x^4 - 4 x^3 + 6 x^2 - 4 x + 1
    \end{equation*}
  }
  \uncover<+->{
    Todos los coeficientes se representan exactamente en punto flotante.
  }

  \uncover<+->{
    Es claro que \(p(x) = (x - 1)^4\).
  }
\end{frame}

\begin{frame}
  \frametitle{Ejemplo --- cálculo del polinomio}

  Una manera eficiente de calcular polinomios es el método de Horner,
  que corresponde a escribir:
  \begin{equation*}
    a_n x^n + a_{n - 1} x^{n - 1} + \dotsb + a_0
      = (\dotsc (a_n x + a_{n - 1}) x + \dotsb a_1) x + a_0
  \end{equation*}
  \uncover<+->{
    Esto corresponde a una multiplicación y una suma por grado,
    bastante menos que lo que usa el método ingenuo
    de calcular las potencias de \(x\),
    multiplicar por los coeficientes y sumar.
  }
\end{frame}

\begin{frame}[fragile]
  \frametitle{Ejemplo --- cálculo del polinomio}

  Acá efectuaremos cálculos en precisión simple,
  (6 cifras significativas,
   aproximadamente),
  para que quede patente el efecto de errores intermedios en el resultado.

  El polinomio se describe por el arreglo de coeficientes:
  \\[0.5ex]
  \lstinputlisting[language = C,
                   linerange = {5-6}]
                  {poly.c}
\end{frame}

\begin{frame}[fragile]
  \frametitle{Ejemplo --- cálculo del polinomio}

  El cálculo del polinomio es:
  \\[0.5ex]
  \lstinputlisting[language = C,
                   linerange = {8-15}]
                  {poly.c}
\end{frame}

\begin{frame}
  \frametitle{Ejemplo --- resultados numéricos}

  \includegraphics[width = 0.85\textwidth]{plot}

  \uncover<+->{
    Si buscáramos el cero,
    los valores numéricos apoyarían valores entre \num{0,98}
    y \num{1,02}.
  }
\end{frame}

\section{Estabilidad, condicionamiento}

\begin{frame}
  \setcounter{beamerpauses}{2}
  \frametitle{Estabilidad}

  \uncover<+->{
    La \emph{estabilidad} es una característica del algoritmo.
    Se refiere a que el algoritmo automáticamente corrige errores intermedios.
  }

  \uncover<+->{
    Un algoritmo inestable amplifica errores,
    cosa para nada deseable.
  }
\end{frame}

\begin{frame}
  \setcounter{beamerpauses}{2}
  \frametitle{Condicionamiento}

  \uncover<+->{
    El \emph{condicionamiento} es una característica del problema.
    Se refiere a que pequeños cambios en los datos de entrada
    cambian fuertemente los resultados.
  }

  \uncover<+->{
    Obtener resultados precisos para un problema mal condicionado
    siempre será difícil.
  }

  \uncover<+->{
    El \emph{número de condición} es el error relativo del resultado
    dividido por el error relativo de los datos de entrada.
  }
\end{frame}

\begin{frame}
  \setcounter{beamerpauses}{2}
  \frametitle{El polinomio pérfido}

  \uncover<+->{
    Para probar rutinas de punto flotante,
    Wilkinson calculó los coeficientes de un polinomio y buscó sus ceros.
    Obtuvo resultados muy raros,
    pero no pudo hallar errores en sus programas.
  }

  \uncover<+->{
    Investigación más detallada reveló
    que minúsculos cambios en los coeficientes
    (redondeo en los cálculos)
    pueden tener efecto dramático sobre los ceros.
  }
\end{frame}

\begin{frame}
  \setcounter{beamerpauses}{2}
  \frametitle{El polinomio pérfido}

  \uncover<+->{
    Para ilustrar el fenómeno,
    consideremos el polinomio:
    \begin{equation*}
      p(x)
        = (x - 1) (x - 2) \dotsm (x - 10)
    \end{equation*}
  }
  \uncover<+->{
    Introduciremos una pequeña perturbación en un solo término:
    \begin{equation*}
      q(x)
        = p(x) + x^5
    \end{equation*}
    Es \(- 902055 x^5\) en \(p\),
    un cambio de \SI{1,1e-4}{\percent} en el coeficiente.
  }
  \uncover<+->{
    Los ceros de \(q(x)\) son:
    \begin{align*}
      &\num{1,0000027558}, \quad \num{1,99921},
        \quad \num{3,02591}, \quad \num{3,82275}, \\
      &\num{5,24676 \pm 0,751485 i}, \quad
         \num{7,57271 \pm 1,11728 i}, \quad
         \num{9,75659 \pm 0,368389 i}
    \end{align*}
  }
\end{frame}

\begin{frame}
  \setcounter{beamerpauses}{2}
  \frametitle{Explicación}

  \uncover<+->{
    Resulta que cerca de un cero de \(p(x)\),
    \(q(x) \approx x^5\)\ldots
  }

  \uncover<+->{
    Un cero \textquote{chico} de \(p\) no se ve (demasiado) afectado,
    uno grande bastante.
  }
\end{frame}

\section{Resumen}

\begin{frame}
  \setcounter{beamerpauses}{2}
  \frametitle{Resumen}

  \begin{itemize}
  \item
    Fuentes de error en resultados numéricos
  \item
    Errores propios del algoritmo
    (estabilidad,
     aproximaciones en el método de cálculo)
   \item
     Errores atribuibles a redondeo
   \item
     Condicionamiento del problema,
     pequeños cambios en los datos dan grandes cambios en resultados
   \end{itemize}

   \uncover<+->{
     Métodos para estimar los efectos anteriores
     se apoyan fuertemente en análisis asintótico.
   }
   \uncover<+->{
     En nuestra somera introducción al análisis numérico
     por lo general omitiremos estos efectos.
   }
\end{frame}
\end{document}

%%% Local Variables:
%%% mode: latex
%%% TeX-master: t
%%% ispell-local-dictionary: "spanish"
%%% End:

% LocalWords:  glyphtounicode presentation modelamiento fl epsilon ex
