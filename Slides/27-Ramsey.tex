\input glyphtounicode%
\pdfgentounicode=1
\documentclass[english, spanish, fleqn,%
hyperref = {colorlinks, urlcolor = blue}%
%, handout%
]{beamer}

%% ]{article}
%% \usepackage{beamerarticle}

\usepackage{beamerthemesplit}

\usepackage{fourier}
\usepackage{amsmath}
\usepackage{amsthm}
\usepackage[es-noquoting]{babel}
\usepackage{csquotes}
\usepackage{tikz}

\ifdefined\HCode
  \def\pgfsysdriver{pgfsys-dvisvgm4ht.def}
\fi

\usefonttheme{professionalfonts}

\DeclareMathOperator{\Exp}{\mathbb{E}}
\DeclareMathOperator{\var}{var}

\beamerdefaultoverlayspecification{<+->}

\title{Cotas inferiores a números de Ramsey}

\author[Horst H. von Brand]{Horst H. von Brand\\
  \href{mailto:vonbrand@inf.utfsm.cl}{vonbrand@inf.utfsm.cl}}

\institute[DI UTFSM]{Departamento de Informática\\
                     Universidad Técnica Federico Santa María}
\date{}

\begin{document}
\frame{\maketitle}

\begin{frame}<handout>
  \frametitle{Contenido}

  \tableofcontents
\end{frame}

\section{Números de Ramsey}

\begin{frame}
  \setcounter{beamerpauses}{2}
  \frametitle{Números de Ramsey}

  \uncover<+->{
    El teorema de Ramsey dice que todo grafo de tamaño \(R(r, s)\)
    contiene una \emph{\foreignlanguage{english}{clique}} de tamaño \(r\)
    (un \(K_r\))
    o un conjunto independiente
    (vértices no unidos por arcos)
    de tamaño \(s\).
  }
  \uncover<+->{
    El punto del teorema es que \(R(r, s)\) es finito.
  }
  \uncover<+->{
    Obtener valores de \(R(r, s)\)
    ha demostrado ser extraordinariamente difícil.
  }
  \\ \medskip
  \uncover<+->{
    Suele plantearse en términos de \(K_n\)
    algunos de cuyos arcos se colorean de rojo
    (presentes en el grafo)
    los demás en azul
    (ausentes),
    diciendo que contiene un \(K_r\) rojo o un \(K_s\) azul.
 }
\end{frame}

\begin{frame}
  \setcounter{beamerpauses}{2}
  \frametitle{Valor de \(R(3, 3)\)}

  \uncover<+->{
    Un ejemplo simple es:
  }
  \begin{theorem}<+->
    \(R(3, 3) = 6\)
  \end{theorem}
\end{frame}

\begin{frame}
  \setcounter{beamerpauses}{2}
  \frametitle{Valor de \(R(3, 3)\)}

  \uncover<+->{
    \textcolor{blue}{Demostración.} \\
    Consideremos \(K_6\),
    con sus arcos coloreados de rojo o azul.
    Elija \(v\) entre sus vértices,
    entre los vértices restantes
    hay al menos \(3\) unidos con arcos del mismo color a \(v\),
    digamos rojo.
    Si un par de estos \(3\) vértices están conectados por un arco rojo,
    con \(v\) forman un \(K_3\) rojo.
    Si no,
    están unidos por arcos azules entre sí,
    y forman un \(K_3\) azul.
    Esto demuestra que \(R(3, 3) \le 6\).
  }
\end{frame}

\begin{frame}
  \setcounter{beamerpauses}{2}
  \frametitle{Valor de \(R(3, 3)\)}

  \uncover<+->{
    Por otro lado,
    la figura muestra \(K_5\)
    con arcos coloreados de rojo y azul
    que no contiene \(K_3\) del mismo color,
    mostrando que \(R(3, 3) > 5\).
    \begin{center}
      % Idea for drawing filched from
      %	   http://www.texample.net/tikz/examples/combinatorial-graphs
      \begin{tikzpicture}[every node/.style
                            = {shape = circle, fill = black!75, draw = black}]
        \draw[thick, blue] \foreach \x in {18, 90, ..., 378}
        {
          (\x:1.5) -- (\x+72:1.5)
        };
        \draw[thick, red] \foreach \x in {18, 162, ..., 738}
        {
          (\x:1.5) node{} -- (\x+144:1.5)
        };
      \end{tikzpicture}
    \end{center}
    \qed
  }
\end{frame}

\section{El método probabilístico}

\begin{frame}
  \setcounter{beamerpauses}{2}
  \frametitle{El método probabilístico}

  \uncover<+->{
    El método probabilístico
    consiste en demostrar usando herramientas de probabilidades
    que un objeto existe.
  }
\end{frame}

\begin{frame}
  \setcounter{beamerpauses}{2}
  \frametitle{Cotas inferiores a \(R(r, r)\)}

  \uncover<+->{
    Coloreemos \(K_n\) al azar,
    asignándole el color rojo o azul a cada arco independientemente
    con la misma probabilidad.
  }
  \uncover<+->{
    Para un conjunto \(S\) de vértices,
    sea \(X(S) = 1\)
    si todos los arcos entre vértices en \(S\) son del mismo color,
    \(X(S) = 0\) en caso contrario.
    El número de \(K_r\) monocromáticos es simplemente la suma de \(X(S)\)
    sobre todos los subconjuntos de \(r\) vértices.
  }
\end{frame}

\begin{frame}
  \setcounter{beamerpauses}{2}
  \frametitle{Cotas inferiores a \(R(r, r)\)}

  \uncover<+->{
    Sea \(S\) un conjunto de \(r\) vértices.
    La probabilidad de que todos los arcos entre ellos
    sean del mismo color es:
    \begin{equation*}
      2 \cdot 2^{- \binom{r}{2}}
    \end{equation*}
    (el factor \(2\) es por los dos colores).
  }
  \uncover<+->{
    El número esperado de \(r\)\nobreakdash-subgrafos monocromáticos es:
    \begin{align*}
      \Exp\left[ \sum_{\lvert S \rvert = r} X(S) \right]
        &=  \sum_{\lvert S \rvert = r} \Exp[X(S)] \\
        &= \binom{n}{r} 2^{1 - \binom{r}{2}}
    \end{align*}
  }
\end{frame}

\begin{frame}
  \setcounter{beamerpauses}{2}
  \frametitle{Cotas inferiores a \(R(r, r)\)}

  \uncover<+->{
    Si el número esperado de \(K_r\) monocromáticos es menor a \(1\),
    debe haber grafos sin \(K_r\) monocromáticos.
  }
  \uncover<+->{
    O sea,
    si:
    \begin{equation*}
      \binom{n}{r}
        < 2^{\binom{r}{2} - 1}
    \end{equation*}
    hay un coloreo de los arcos de \(K_n\)
    tal que no contiene \(K_r\) rojos ni azules.
  }
\end{frame}

\section{Resumen}

\begin{frame}
  \setcounter{beamerpauses}{2}
  \frametitle{Resumen}

  \begin{itemize}
  \item
    Planteamos el método probabilísitico.
  \item
    Discutimos números de Ramsey.
  \item
    Demostramos \(R(3, 3) = 6\).
  \item
    Hallamos una cota inferior a \(R(r, r)\).
  \end{itemize}
\end{frame}
\end{document}

%%% Local Variables:
%%% mode: latex
%%% TeX-master: t
%%% ispell-local-dictionary: "spanish"
%%% End:

% LocalWords:  glyphtounicode english clique blue subgrafos
% LocalWords:  probabilísitico
