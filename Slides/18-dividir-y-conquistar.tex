\input glyphtounicode%
\pdfgentounicode=1
\documentclass[english, spanish, fleqn,%
hyperref = {colorlinks, urlcolor = blue}%
%, handout%
]{beamer}

%% ]{article}
%% \usepackage{beamerarticle}

\usepackage{beamerthemesplit}

\usepackage[es-noquoting]{babel}
\usepackage[group-digits = false, copy-decimal-marker = true]{siunitx}
\usepackage{csquotes}
\usepackage{fourier}
\usepackage{amsmath}
\usepackage{amsthm}
\usepackage{dcolumn}
\usepackage[noline, noend]{algorithm2e}
\usepackage{listings}

\usefonttheme{professionalfonts}

%%%
%%% For listings
%%%

\lstloadlanguages{Python}


\beamerdefaultoverlayspecification{<+->}

\title{Dividir y conquistar}

\author[Horst H. von Brand]{Horst H. von Brand\\
  \href{mailto:vonbrand@inf.utfsm.cl}{vonbrand@inf.utfsm.cl}}

\institute[DI UTFSM]{Departamento de Informática\\
                     Universidad Técnica Federico Santa María}
\date{}

\begin{document}
%%%
%%% For listings
%%%

\lstset{basicstyle = \sffamily, commentstyle = \slshape,
        frame = none, numbers = none,
        tabsize = 4,
        showstringspaces = false,
        xleftmargin = 1em, xrightmargin = 1em
       }
\renewcommand{\lstlistingname}{Listado}

%%%
%%% For algorithm2e
%%%

\SetAlgorithmName{Algoritmo}{Algoritmo}{Índice de algoritmos}

\SetAlCapSty{mdseries}
\SetKwProg{Function}{function}{}{end}
\SetKwProg{Procedure}{procedure}{}{end}
\SetKw{Variables}{variables}
\SetKw{Downto}{downto}
\SetKwBlock{Loop}{loop}{end}
\SetKw{Continue}{continue}
\SetKw{Break}{break}
\SetKw{KwStep}{step}

\frame{\maketitle}

\begin{frame}<handout>
  \frametitle{Contenido}

  \tableofcontents
\end{frame}

\section{Técnica general}

\begin{frame}
  \setcounter{beamerpauses}{2}
  \frametitle{La técnica general}

  \uncover<+->{
    Una técnica fructífera para diseñar buenos algoritmos
    es dividir el problema original en problemas menores del mismo tipo,
    resolver éstos y combinar los resultados.
  }

  \uncover<+->{
    Ejemplos de esta estrategia son ordenamiento por intercalación
    (\emph{\foreignlanguage{english}{mergesort}}),
    búsqueda binaria,
    algoritmos rápidos para calcular potencias,
    también \emph{\foreignlanguage{english}{quicksort}}.
  }

  \uncover<+->{
    Veremos varios ejemplos adicionales,
    y discutiremos técnicas de análisis para estos algoritmos.
  }
\end{frame}

\section{Ejemplos}

\begin{frame}
  \setcounter{beamerpauses}{2}
  \frametitle{Potencias}

  \uncover<+->{
    Es obvio que:
    \begin{align*}
      a^0
        &= 1 \\
      a^1
        &= a \\
      a^n
        &= \left(a^{\lfloor n / 2 \rfloor}\right)^2 \cdot a^{[2 \nmid n]}
    \end{align*}
  }
\end{frame}

\begin{frame}
  \frametitle{Potencias}

  Esto sugiere el algoritmo:
  \begin{algorithm}[H]
    \DontPrintSemicolon

    \Function{\(\mathrm{power}(a, n)\)}{
      \uIf{\(n = 0\)}{
        \Return \(1\) \;
      }
      \uElseIf{\(n = 1\)}{
        \Return \(a\) \;
      }
      \Else{
        \(t \gets \mathrm{power}(a, \lfloor n / 2 \rfloor)\) \;
        \(t \gets t \cdot t\)  \;
        \If{\(2 \nmid n\)}{
          \(t \gets t \cdot a\) \;
        }
        \Return \(t\) \;
      }
    }
  \end{algorithm}
\end{frame}

\begin{frame}
  \setcounter{beamerpauses}{2}
  \frametitle{Potencias}

  \uncover<+->{
    En términos del número de multiplicaciones al calcular \(a^n\),
    \(P(n)\),
    obtenemos la recurrencia:
    \begin{equation*}
      P(n)
        = P(\lfloor n / 2 \rfloor) + 1 + [2 \nmid n]
        \qquad P(0) = P(1) = 0
    \end{equation*}
  }
\end{frame}

\begin{frame}
  \setcounter{beamerpauses}{2}
  \frametitle{\emph{\foreignlanguage{english}{Mergesort}}}

  \uncover<+->{
    Debemos ordenar una secuencia de largo \(n\).
  }
  \\ \medskip
  \uncover<+->{
    Si \(n \le 1\) no hay nada que hacer.
    Si \(n > 1\),
    la dividimos en dos subsecuencias de largo similar
    (\(\lfloor n / 2 \rfloor\) y \(\lceil n / 2 \rceil\)),
    ordenamos las mitades recursivamente
    e intercalamos los resultados.
  }
  \\ \smallskip
  \uncover<+->{
    Intercalar dos secuencias ordenadas de largos \(m\) y \(n\)
    toma \(m + n\) operaciones
    (debemos crear la nueva secuencia)
    y \(\min\{m, n\}\) comparaciones
    (debemos comparar los elementos de la secuencia más corta
     con elementos de la otra para determinar en qué orden van).
  }
  \\ \medskip
  \uncover<+->{
    Si llamamos \(M(n)\) al número de comparaciones
    para ordenar \(n\) elementos
    usando \emph{\foreignlanguage{english}{mergesort}},
    obtenemos la recurrencia:
    \begin{equation*}
      M(n)
        = M(\lfloor n / 2 \rfloor) + M(\lceil n / 2 \rceil)
            + \lfloor n / 2 \rfloor
      \qquad
      M(0) = M(1) = 0
    \end{equation*}
  }
\end{frame}

\begin{frame}
  \setcounter{beamerpauses}{2}
  \frametitle{Búsqueda binaria}

  \uncover<+->{
    Debemos ubicar el elemento \(e\)
    en un arreglo ordenado \(\mathbf{a}\) de largo \(n\),
    o reportar que no está.
  }
  \\ \medskip
  \uncover<+->{
    Si \(n = 0\),
    reportamos que no está.
    Si \(n = 1\),
    comparamos con el único elemento.
    Si \(n > 1\),
    comparamos con el elemento en la posición \(\lfloor n / 2 \rfloor\).
    Si resultan iguales,
    damos esa posición.
    Si es menor
    buscamos
    (recursivamente)
    en la primera mitad del arreglo,
    si es mayor en la segunda.
  }
  \\ \medskip
  \uncover<+->{
    Si \(B(n)\) es el número máximo de comparaciones
    al buscar en un arreglo de \(n\) elementos,
    podemos aproximarlo como buscar en mitades:
    \begin{equation*}
      B(n)
        = B(\lfloor n / 2 \rfloor) + 1
      \qquad
      B(0) = 0, B(1) = 1
    \end{equation*}
  }
\end{frame}

\begin{frame}
  \setcounter{beamerpauses}{2}
  \frametitle{Multiplicación de Karatsuba}

  \uncover<+->{
    Nos dan a multiplicar dos números \(A\) y \(B\),
    de \(2 n\)~dígitos cada uno.
  }
  \uncover<+->{
    Una estrategia es dividir los números en mitades:
    \begin{equation*}
      A
        = A_h \cdot 10^n + A_l
      \qquad
      B
        = B_h \cdot 10^n + B_l
    \end{equation*}
  }
  \uncover<+->{
    Álgebra elemental da:
    \begin{align*}
      A \cdot B
        &= (A_h \cdot 10^n + A_l) \cdot (B_h \cdot 10^n + B_l) \\
        &= A_h B_h \cdot 10^{2 n}
             + (A_h B_l + A_l B_h) \cdot 10^n
             + A_l B_l
    \end{align*}
  }
\end{frame}

\begin{frame}
  \setcounter{beamerpauses}{2}
  \frametitle{Multiplicación de Karatsuba}

  \uncover<+->{
    Notando que:
    \begin{equation*}
      (A_h + A_l) \cdot (B_h + B_l)
        = A_h B_h + A_h B_l + A_l B_h + A_l B_l
    \end{equation*}
    podemos evaluar los coeficientes mediante tres multiplicaciones
    (no cuatro):
    \begin{align*}
      &A_h B_h \\
      &A_l B_l \\
      &A_h B_l + A_l B_h
         = (A_h + A_l) \cdot (B_h + B_l) - A_h B_h - A_l B_l
    \end{align*}
  }
  \uncover<+->{
    Llamando \(K(n)\) al número de operaciones de un dígito:
    \begin{equation*}
      K(2 n)
        = 3 K(n) + 4 n
      \qquad
      K(1) = 1
    \end{equation*}
  }
\end{frame}

\section{El teorema maestro}

\begin{frame}
  \setcounter{beamerpauses}{2}
  \frametitle{Características comunes de las recurrencias}

  \uncover<+->{
    Las recurrencias discutidas son:
  }
  \begin{align*}
    \uncover<+->{
       M(n)
         &= M(\lfloor n / 2 \rfloor) + M(\lceil n / 2 \rceil)
             + \lfloor n / 2 \rfloor
       &&
       M(0) = M(1) = 0 \\
    }
    \uncover<+->{
      B(n)
        &= B(\lfloor n / 2 \rfloor) + 1
      &&
      B(0) = 0, B(1) = 1 \\
    }
    \uncover<+->{
      K(2 n)
        &= 3 K(n) + 4 n
      &&
      K(1) = 1
    }
  \end{align*}
  \uncover<+->{
    Sea \(C(n)\) el costo para tamaño \(n\).
    Las recurrencias pueden escribirse aproximadamente como:
    \begin{align*}
      C(n)
        &= a C(n / b) + f(n)
      &&\qquad
      C(1)
        = 1
    \end{align*}
  }
  \uncover<+->{
    (Omitimos pisos y cielos,
     y despreciamos términos menores
     para obtener una recurrencia más simple.
     El valor inicial no tiene real impacto.
     Puede justificarse rigurosamente que aproxima la solución exacta.)
  }
  \\ \smallskip
  \uncover<+->{
    En castellano:
    Para resolver un problema de tamaño \(n\),
    dividimos en \(a\) problemas de tamaño \(n / b\),
    los resolvemos recursivamente
    y combinamos las soluciones.
  }
  \uncover<+->{
    El costo de dividir en subproblemas
    y luego combinar soluciones es \(f(n)\).
  }
\end{frame}

\begin{frame}
  \setcounter{beamerpauses}{2}
  \frametitle{Recurrencia simplificada}

  \uncover<+->{
    Podemos simplificar la recurrencia si suponemos que \(n\)
    es una potencia de \(b\).
    Es de interés el caso \(a \ge 1\),
    \(b > 1\), \(f(n) > 0\).
    Cambiemos variables:
    \begin{equation*}
      n
        = b^r
      \qquad
      C(b^r)
         = W(r)
    \end{equation*}
    para obtener:
    \begin{equation*}
      W(r)
        = a W(r - 1) + f(b^r)
      \qquad
      W(0)
        = 1
    \end{equation*}
  }
  \uncover<+->{
    Una recurrencia lineal de primer orden con coeficientes constantes.
  }
\end{frame}

\begin{frame}
  \setcounter{beamerpauses}{2}
  \frametitle{Solución de la recurrencia}

  \uncover<+->{
    La solución es rutina.
    Dividimos por \(a^r\):
  }
  \begin{align*}
    \uncover<+->{
      \frac{W(r)}{a^r} - \frac{W(r - 1)}{a^{r - 1}}
        &= a^{-r} f(b^r) \\
    }
    \uncover<+->{
      \sum_{1 \le r \le s} \left(
                            \frac{W(r)}{a^r} - \frac{W(r - 1)}{a^{r - 1}}
                         \right)
        &= \sum_{1 \le r \le s} a^{-r} f(b^r)
    }
  \end{align*}
  \uncover<+->{
    Esto se simplifica a:
    \begin{equation*}
      W(s)
        = W(0) a^s + a^s \sum_{1 \le r \le s} a^{-r} f(b^r)
    \end{equation*}
  }
\end{frame}

\begin{frame}
  \setcounter{beamerpauses}{2}
  \frametitle{Solución de la recurrencia}

  \uncover<+->{
    Definamos \(\alpha = \log_b a\),
    con lo que:
    \begin{equation*}
      a^r
        = \left( b^{\log_b a} \right)^r
        = \left( b^\alpha \right)^r
        = \left( b^r \right)^\alpha
        = n^\alpha
    \end{equation*}
  }
  \uncover<+->{
    La solución a la recurrencia en términos de las variables originales es:
    \begin{equation*}
      C(n)
        = C(0) n^\alpha
            + n^\alpha \sum_{1 \le r \le \log_b n} a^{-r} f(b^r)
    \end{equation*}
  }
\end{frame}

\begin{frame}
  \setcounter{beamerpauses}{2}
  \frametitle{Cotas de la sumatoria}

  \uncover<+->{
    Es crucial la sumatoria:
    \begin{equation*}
      \sum_{1 \le r \le \log_b n} a^{-r} f(b^r)
    \end{equation*}
  }
  \uncover<+->{
    Hay varias situaciones a analizar.
    Primero,
    si la suma infinita respectiva converge,
    lo que ocurre si \(f(n) = O(n^c)\) con \(c < \alpha\),
    podemos acotar la sumatoria por su límite y queda:
    \begin{equation*}
      C(0) n^\alpha
        \le C(n)
        \le C(0) n^\alpha + n^\alpha \sum_{r \ge 1} a^{-r} f(b^r)
    \end{equation*}
  }
  \uncover<+->{
    Concluimos:
    \begin{equation*}
      C(n)
        = \Theta(n^\alpha)
    \end{equation*}
  }
\end{frame}

\begin{frame}
  \setcounter{beamerpauses}{2}
  \frametitle{Cotas de la sumatoria}

  \uncover<+->{
    Otra situación se da si la suma diverge.
    Si \(f(n) = \Omega(n^c)\) con \(c > \alpha\),
    el último término de la suma es el mayor,
    con lo que obtenemos la estimación \(C(n) = \Omega(f(n))\).
  }
  \uncover<+->{
    Suponiendo además \(a f(n / b) \le k f(n)\)
    para algún \(k < 1\),
    podemos escribir:
    \begin{align*}
      \sum_{1 \le r \le \log_b n} a^{-r} f(b^r)
        &=   a^{- \log_b n} \sum_{0 \le r < \log_b n} a^r f(n / b^r) \\
        &\le n^{-\alpha} \sum_{0 \le r < \log_b n} k^r f(n) / a \\
        &<   \frac{f(n)}{a n^\alpha} \sum_{r > 0} k^r \\
        &=   \frac{f(n)}{a n^\alpha} \cdot \frac{1}{1 - k}
    \end{align*}
  }
\end{frame}

\begin{frame}
  \setcounter{beamerpauses}{2}
  \frametitle{Cotas de la sumatoria}

  \uncover<+->{
    Con lo anterior esto da si \(f(n) = \Omega(n^c)\) con \(c > \alpha\)
    y \(a f(n / b) \le k f(n)\) para algún \(k < 1\):
    \begin{equation*}
      C(0) n^\alpha + f(n)
        \le C(n)
        \le C(0) n^\alpha +  \frac{f(n)}{a} \cdot \frac{1}{1 - k}
    \end{equation*}
  }
  \uncover<+->{
    Podemos absorber los términos en \(n^\alpha\)
    ya que \(f(n) = \Omega(n^\alpha)\),
    y tenemos:
    \begin{equation*}
      C(n)
        = \Theta(f(n))
   \end{equation*}
  }
\end{frame}

\begin{frame}
  \setcounter{beamerpauses}{2}
  \frametitle{Cotas de la sumatoria}

  \uncover<+->{
    Quedan las situaciones intermedias.
    Las más interesantes en la práctica se resumen en:
    \begin{equation*}
      f(n)
        = \Theta(n^\alpha \log^\beta n)
    \end{equation*}
  }
  \uncover<+->{
    La suma de marras es:
    \begin{align*}
      \sum_{1 \le r \le \log_b n} a^{-r} f(b^r)
        &= \sum_{1 \le r \le \log_b n} a^{-r} \Theta(a^r r^\beta) \\
        &= \sum_{1 \le r \le \log_b n} \Theta(r^\beta) \\
        &= \Theta\left(
                   \sum_{1 \le r \le \log_b n} r^\beta
                 \right)
    \end{align*}
  }
\end{frame}

\begin{frame}
  \setcounter{beamerpauses}{2}
  \frametitle{Cotas de la sumatoria}

  \uncover<+->{
    La suma es convergente si \(\beta < -1\),
    si \(\beta = -1\) son números harmónicos
    (sabemos \(H_n \sim \ln n\)),
    para \(\beta > -1\) podemos acotar mediante una integral.
  }
  \uncover<+->{
    En resumen,
    si \(f(n) = \Theta(n^\alpha \log^\beta n)\):
    \begin{equation*}
      C(n)
        = \begin{cases}
            \Theta(n^\alpha)		       & \beta < -1 \\
            \Theta(n^\alpha \log \, \log n)	& \beta = -1 \\
            \Theta(n^\alpha \log^{\beta + 1} n)	 & \beta > -1
          \end{cases}
    \end{equation*}
  }
\end{frame}

\begin{frame}
  \setcounter{beamerpauses}{2}
  \frametitle{El teorema maestro}

  \uncover<+->{
    La solución a la recurrencia:
    \begin{align*}
      C(n)
        &= a C(n / b) + f(n)
      &&
         C(0) = 1
    \end{align*}
    donde \(\alpha = \log_b a\)
    y \(f(n)\) es como se indica
    cumple:
  }
  \begin{align*}
    \uncover<+->{
      C(n)
        &= \Theta(n^\alpha)
      && f(n) = O(n^c) \text{\ con \(c < \alpha\)} \\
    }
    \uncover<+->{
      C(n)
        &= \Theta(n^\alpha )
      && f(n) = \Theta(n^\alpha \log^\beta n) \text{\ con \(\beta < -1\)} \\
    }
    \uncover<+->{
      C(n)
        &= \Theta(n^\alpha \log \, \log n)
      && f(n) = \Theta(n^\alpha \log^\beta n) \text{\ con \(\beta = -1\)} \\
    }
    \uncover<+->{
      C(n)
        &= \Theta(n^\alpha \log^{\beta + 1} n)
      && f(n) = \Theta(n^\alpha \log^\beta n) \text{\ con \(\beta > -1\)} \\
    }
    \uncover<+->{
      C(n)
        &= \Theta(f(n))
      && f(n) = \Omega(n^c) \text{\ con \(c > \alpha\))
                                  y \(a f(n / b) \le k f(n)\)}
    }
  \end{align*}
\end{frame}

\begin{frame}
  \setcounter{beamerpauses}{2}
  \frametitle{Nuestros ejemplos}

  \uncover<+->{
    \begin{align*}
      P(n)
        &= P(\lfloor n / 2 \rfloor) + 1 + [2 \nmid n]
        && P(0) = P(1) = 0 \\
      M(n)
        &= M(\lfloor n / 2 \rfloor) + M(\lceil n / 2 \rceil)
            + \lfloor n / 2 \rfloor
        && M(0) = M(1) = 0 \\
      B(n)
        &= B(\lfloor n / 2 \rfloor) + 1
        && B(0) = 0, B(1) = 1 \\
      K(2 n)
        &= 3 K(n) + 4 n
        && K(1) = 1
    \end{align*}
  }

  \begin{center}
    \uncover<+->{
      \begin{tabular}{lrr*{2}{>{\(}l<{\)}}}
        \multicolumn{1}{c}{\textbf{Algoritmo}}
          & \multicolumn{1}{c}{\boldmath\(a\)\unboldmath}
          & \multicolumn{1}{c}{\boldmath\(b\)\unboldmath}
          & \multicolumn{1}{c}{\boldmath\(f(n)\)\unboldmath}
          & \multicolumn{1}{c}{\boldmath\(C(n)\)\unboldmath} \\
        \hline
    }
      \uncover<+->{
        Potencia	 & 1 & 2 & \Theta(1) & \Theta(\log n) \\
      }
      \uncover<+->{
        Mergesort	 & 2 & 2 & \Theta(n) & \Theta(n \log n) \\
      }
      \uncover<+->{
        Búsqueda binaria & 1 & 2 & \Theta(1) & \Theta(\log n) \\
      }
      \uncover<+->{
        Karatsuba	 & 3 & 2 & \Theta(n) & \Theta(n^{\log_2 3})
      }
    \end{tabular}
  \end{center}
\end{frame}

\section{El teorema de Akra-Bazzi}

\begin{frame}
  \setcounter{beamerpauses}{2}
  \frametitle{El teorema de Akra-Bazzi}

  \uncover<+->{
    Uno de los problemas de la derivación precedente
    es que supone divisiones iguales
    (no en \(1 / 3\) y \(2 / 3\),
     por ejemplo),
    además que es preocupante el barrer bajo la alfombra pisos y techos
    (ya indicamos que puede justificarse rigurosamente,
     claro).
  }
  \uncover<+->{
    Este tipo de recurrencias se han estudiado intensamente,
    sus soluciones muestran comportamientos bastante extraños.
    Pero salvo algunas situaciones simples son difíciles de tratar.
    Afortunadamente nuestras cotas son válidas.
  }
  \\ \medskip
  \uncover<+->{
    Un camino distinto tomaron Akra y Bazzi,
    planteando una recurrencia sobre los reales
    (ahí no hay problema en basarse en los casos \(17 / 3\) y \(3 / 2\)
     anteriores).
    Leighton corrigió,
    completó
    y extendió lo anterior.
 }
\end{frame}

\begin{frame}
  \frametitle{El teorema de Akra-Bazzi}

  \begin{theorem}
    Sea la recurrencia para \(z \ge z_0\):
    \begin{equation*}
      T(z)
        = g(z) + \sum_{1 \le k \le n} a_k T(b_k z + h_k(z))
    \end{equation*}
    con \(z_0, a_k, b_k\) constantes,
    sujeta a:
    \begin{itemize}
    \item<1->
      Hay suficientes casos base
    \item<1->
      Para todo \(k\) es \(a_k > 0\) y \(0 < b_k < 1\)
    \item<1->
      Hay una constante \(c\) tal que \( \lvert g'(z) \rvert = O(z^c)\)
    \item<1->
      Para todo \(k\) es \(h_k(z) = O(z / \log^2 z)\)
    \end{itemize}
  \end{theorem}
\end{frame}

\begin{frame}
  \frametitle{El teorema de Akra-Bazzi}

  \emph{
    Si \(p\) es la única solución real de:
    \begin{equation*}
      \sum_{1 \le k \le n} a_k b_k^p
        = 1
    \end{equation*}
    entonces la solución cumple:
    \begin{equation*}
      T(z)
        = \Theta \left(
                   z^p \left(
                         1 +\int_1^z \frac{g(u)}{u^{p + 1}} \, \mathrm{d} u
                       \right)
                 \right)
    \end{equation*}
  }
\end{frame}

\begin{frame}
  \setcounter{beamerpauses}{2}
  \frametitle{Demostración}

  \uncover<+->{
    La demostración es un festival de aplicaciones del teorema del valor medio
    para acotar integrales
    y otras maniobras turbias.
  }
\end{frame}

\begin{frame}
  \setcounter{beamerpauses}{2}
  \frametitle{Aplicaciones}

  \uncover<+->{
    Vimos que la recurrencia correcta para ordenamiento por intercalación
    es:
    \begin{equation*}
      M(n)
        = M(\lfloor n / 2 \rfloor) + M(\lceil n / 2 \rceil)
            + \lfloor n / 2 \rfloor
      \qquad
      M(0) = M(1) = 0
    \end{equation*}
  }
  \uncover<+->{
    Es directamente aplicable el teorema de Akra-Bazzi,
    con \(a_1 = a_2 = 1\),
    \(b_1 = b_2 = 1 / 2\),
    \(h_1(z) = z / 2 - \lfloor z / 2 \rfloor\),
    \(h_2(z) = z / 2 - \lceil z / 2 \rceil\),
    \(g(z) = z\).
  }
  \uncover<+->{
    De \(2 \cdot (1/2)^p = 1\) obtenemos \(p = 1\) y:
    \begin{equation*}
      M(n)
        = \Theta\left(
                  n^1 \left(
                        1 + \int_1^n \frac{u\, \mathrm{d} u}{u^2}
                      \right)
                \right)
        = \Theta(n \log n)
    \end{equation*}
  }
\end{frame}

\begin{frame}
  \setcounter{beamerpauses}{2}
  \frametitle{Aplicaciones}

  \uncover<+->{
    Al analizar árboles de búsqueda aleatorizados
    resulta la recurrencia:
    \begin{equation*}
      T(n)
        = \frac{1}{4} T(n / 4) + \frac{3}{4} T(3 n / 4) + 1
    \end{equation*}
  }
  \uncover<+->{
    Obtenemos:
    \begin{equation*}
      \frac{1}{4} \left( \frac{1}{4} \right)^p
        + \frac{3}{4} \left( \frac{3}{4} \right)^p
       = 1
    \end{equation*}
    que da \(p = 0\) y:
    \begin{equation*}
      T(n)
        = \Theta\left(
                  n^0 \left(
                        1 + \int_1^n \frac{1\, \mathrm{d} u}{u}
                      \right)
                \right)
        = \Theta(\log n)
    \end{equation*}
  }
\end{frame}

\begin{frame}
  \setcounter{beamerpauses}{2}
  \frametitle{Aplicaciones}

  \uncover<+->{
    El problema del valor recursivo de \(n\)
    lleva a la recurrencia para el número de llamadas \(R(n)\):
    \begin{equation*}
      R(n)
        = R(n / 2) + R(n / 3) + R(n / 4) + 1
    \end{equation*}
  }
  \uncover<+->{
    Obtenemos:
    \begin{equation*}
      \left( \frac{1}{2} \right)^p
        + \left( \frac{1}{3} \right)^p
        + \left( \frac{1}{4} \right)^p
       = 1
    \end{equation*}
    que da \(p = \num{1,082131}\) y:
    \begin{equation*}
      R(n)
        = \Theta\left(
                  n^p \left(
                        1 + \int_1^n \frac{1\, \mathrm{d} u}{u^{p + 1}}
                      \right)
                \right)
        = \Theta(n^p)
    \end{equation*}
  }
\end{frame}

\section{Resumen}

\begin{frame}
  \setcounter{beamerpauses}{2}
  \frametitle{Resumen}

  \begin{itemize}
  \item
    Dividir y conquistar es una técnica fructífera de diseño de algoritmos.
  \item
    Su análisis exacto es altamente complejo\ldots
  \item
    El teorema maestro da cotas asintóticas a la solución de las recurrencias
    en muchos casos de interés práctico.
  \item
    Una variante muy útil del teorema maestro es el teorema de Akra-Bazzi.
  \end{itemize}
\end{frame}
\end{document}

%%% Local Variables:
%%% mode: latex
%%% TeX-master: t
%%% ispell-local-dictionary: "spanish"
%%% End:

% LocalWords:  glyphtounicode mdseries Function function end Downto
% LocalWords:  Procedure procedure downto Loop loop Continue continue
% LocalWords:  Break break KwStep step english mergesort quicksort
% LocalWords:  subsecuencias lrr aleatorizados sumatoria handout
