\input glyphtounicode%
\pdfgentounicode=1
\documentclass[english, spanish, fleqn,%
hyperref = {colorlinks, urlcolor = blue}%
%, handout%
]{beamer}

%% ]{article}
%% \usepackage{beamerarticle}

\usepackage{beamerthemesplit}

\usepackage[es-noquoting]{babel}
\usepackage{csquotes}
\usepackage{fourier}
\usepackage{amsmath}
\usepackage{amsthm}
\usepackage{listings}
\usepackage{pgfplots}

\pgfplotsset{compat=1.16}

\usefonttheme{professionalfonts}

%%%
%%% For listings
%%%

\lstloadlanguages{[ANSI]C}


\beamerdefaultoverlayspecification{<+->}

\title{Aproximar ceros}

\author[Horst H. von Brand]{Horst H. von Brand\\
  \href{mailto:vonbrand@inf.utfsm.cl}{vonbrand@inf.utfsm.cl}}

\institute[DI UTFSM]{Departamento de Informática\\
                     Universidad Técnica Federico Santa María}
\date{}

\begin{document}
%%%
%%% For listings
%%%

\lstset{basicstyle = \sffamily, commentstyle = \slshape,
        frame = lines, numbers = none,
        showstringspaces = false,
        float, floatplacement = ht, captionpos = b,
        xleftmargin = 1em, xrightmargin = 1em
       }
\renewcommand{\lstlistingname}{Listado}

\frame{\maketitle}

\begin{frame}<handout>
  \frametitle{Contenido}

  \tableofcontents
\end{frame}

\section{El problema}

\begin{frame}
  \setcounter{beamerpauses}{2}
  \frametitle{El problema}

  \uncover<+->{
    Es común querer resolver ecuaciones,
    o sea,
    hallar \(x^*\) tal que:
    \begin{equation*}
      f(x^*)
        = 0
    \end{equation*}
  }
  \uncover<+->{
    \noindent
    Esto es hallar un cero de la función \(f\).
  }
\end{frame}

\begin{frame}
  \setcounter{beamerpauses}{2}
  \frametitle{El problema}

  \uncover<+->{
    Aprendimos fórmulas generales para funciones lineales y cuadráticas,
    una serie de técnicas
    que permiten reducir algunos polinomios de grado mayor,
    métodos para hallar ceros racionales de polinomios,
    y una variedad de trucos para resolver situaciones diferentes.
  }
  \uncover<+->{
    Una búsqueda en Internet indica que hay fórmulas generales
    para polinomios de tercer y cuarto grado,
    pero son muy complicadas.
  }
  \medskip \\
  \uncover<+->{
    La triste realidad es que para la mayor parte de las funciones de interés
    no hay fórmulas que nos dan los ceros.
    Incluso puede demostrarse que no hay fórmulas algebraicas
    (usando sumas, multiplicaciones y raíces)
    para polinomios generales de grado quinto o superior.
  }
\end{frame}

\section{Métodos numéricos para aproximar ceros}

\begin{frame}
  \setcounter{beamerpauses}{2}
  \frametitle{Métodos numéricos}

  \uncover<+->{
    El problema a resolver entonces
    dada una función continua \(f(x)\),
    aproximar un cero \(x^*\) de la función \(f\).
  }
  \uncover<+->{
    Generalmente partiremos al menos
    con alguna idea general de la ubicación del \(x^*\) de interés.
  }
  \uncover<+->{
    Buscamos algún método sistemático
    que construya aproximaciones mejores de \(x^*\),
    ojalá con el mínimo trabajo.
  }
  \uncover<+->{
    (La función \(f\) puede ser muy costosa de computar.)
  }
  \bigskip

  \uncover<+->{
    Hay un sinnúmero de algoritmos para resolver este importante problema.
    Estudiaremos en detalle solo los más importantes.
    En el apunte se discuten algunos adicionales.
  }
\end{frame}

\subsection{Método de bisección}

\begin{frame}
  \setcounter{beamerpauses}{2}
  \frametitle{El método de bisección}

  \uncover<+->{
    Ya lo deben haber visto en un ramo de cálculo,
    al demostrar que una función continua sobre \([a, b]\)
    toma todos los valores entre \(f(a)\) y \(f(b)\).
  }
  \bigskip

  \uncover<+->{
    Partimos la iteración con valores \(x_0\) y \(x_1\) tales que
    \(f\) cambia de signo entre ellos
    (o sea,
     \(f(x_0) \cdot f(x_1) < 0\)).
  }
  \uncover<+->{
    Dividimos el intervalo en dos,
    reteniendo una mitad en que cambie de signo la función
    (si no le apuntamos de una al cero,
     claro)
    hasta tener un intervalo suficientemente corto.
  }
\end{frame}

\begin{frame}
  \frametitle{El método de bisección}

  \begin{center}
    \begin{tikzpicture}
      \begin{axis}[axis x line = middle, axis y line = left,
                   ytick = \empty,
                   xlabel = \(x\), ylabel = {\(y = f(x)\)},
                   xmin = 0, xmax = 1,
                   xtick = {0.1, 0.45, 0.8},
                   xticklabels = {\(x_0\), \(x_2\), \(x_1\)},
                   ymin = -0.65, ymax = 1.1],
         \addplot[color = red, thick, domain = 0.02:0.95]{exp(-x) - x};
         \draw[color = blue] (axis cs: 0.1, 0) -- (axis cs: 0.1, 0.80483);
         \draw[color = blue] (axis cs: 0.8, -0.35067) -- (axis cs: 0.8, 0);
      \end{axis}
    \end{tikzpicture}
  \end{center}
\end{frame}

\begin{frame}
  \frametitle{Bisección (programa)}
  \framesubtitle{Encabezado}

  \lstinputlisting[basicstyle = \small\sffamily,
                   frame = none,
                   linerange = {30-31}]
                  {bisection.c}
\end{frame}

\begin{frame}
  \frametitle{Bisección (programa)}
  \framesubtitle{Cuerpo}

  \lstinputlisting[basicstyle = \small\sffamily,
                   frame = none,
                   linerange = {33-45}]
                  {bisection.c}
\end{frame}

\subsection{Método de Newton}

\begin{frame}
  \setcounter{beamerpauses}{2}
  \frametitle{Método de Newton}

  \uncover<+->{
    Otra idea es aproximar la función por una tangente en el punto actual
    y determinar dónde corta el eje \(X\).
  }
  \uncover<+->{
    La ecuación de la tangente en \(x_n\) es:
    \begin{equation*}
      \frac{y - y_n}{x - x_n}
        = f'(x_n)
    \end{equation*}
    donde interesa \(x\) para \(y = 0\).
  }
  \medskip  \\
  \uncover<+->{
    Resulta la iteración:
    \begin{equation*}
      x_{n + 1}
        = x_n - \frac{f(x_n)}{f'(x_n)}
    \end{equation*}
  }
  \uncover<+->{
    Iniciamos la iteración eligiendo \(x_0\),
    que intentamos sea cercano al cero buscado.
  }
\end{frame}

\begin{frame}
  \frametitle{Método de Newton}

  \begin{center}
    \begin{tikzpicture}
      \begin{axis}[
                    axis lines = center,
                    xlabel = {\(x\)},
                    ylabel = {\(y\)},
                    every axis y
                      label/.style = {at = (current axis.above origin),
                        anchor = south},
                    every axis x
                      label/.style = {at = (current axis.right of origin),
                        anchor = west},
                    xmin =  0,	 xmax = 0.5,
                    ymin = -0.3, ymax = 1.2,
                    xtick	= {3/20,    17139/56020},
                    xticklabels = {\small \(x_0\),
                                   \small \(x_1\)},
                    ytick = {0},
                    yticklabels = {}
                  ]
        \addplot [color = blue, domain = 0.02:0.37]
          {(17139 - 56020 * x) / 21480};
        \addplot [domain = 0.01:0.55, color = red, thick]
           {(4740000 * x^4 - 7646000 * x^3 + 4231950 * x^2
              - 1167595 * x + 157926) / 136500};
        \draw [dashed, color = blue] (axis cs:3/20, 0) -- (axis cs:3/20, 2/5);
      \end{axis}
    \end{tikzpicture}
\end{center}
\end{frame}

\begin{frame}
  \frametitle{Método de Newton (programa)}
  \framesubtitle{Encabezado}

  \lstinputlisting[basicstyle = \small\sffamily,
                   frame = none,
                   linerange = {32-33}]
                  {newton.c}
\end{frame}

\begin{frame}
  \frametitle{Método de Newton (programa)}
  \framesubtitle{Cuerpo}

  \lstinputlisting[basicstyle = \small\sffamily,
                   frame = none,
                   linerange = {35-42}]
                  {newton.c}
\end{frame}

\subsection{Método de la secante}

\begin{frame}
  \setcounter{beamerpauses}{2}
  \frametitle{Método de secante}

  \uncover<+->{
    Si es incómodo o costoso calcular la derivada,
    podemos aproximar la tangente por la secante entre los últimos dos puntos.
  }
  \uncover<+->{
    La ecuación de la secante que pasa por \((x_n, y_n)\)
    y \((x_{n + 1}, y_{n + 1})\) es:
    \begin{equation*}
      \frac{y - y_{n + 1}}{x - x_{n + 1}}
        = \frac{y_{n + 1} - y_n}{x_{n + 1} - x_n}
    \end{equation*}
    donde interesa \(x\) para \(y = 0\).
  }
  \medskip  \\
  \uncover<+->{
    Resulta la iteración:
    \begin{equation*}
      x_{n + 2}
        = x_{n + 1} - y_{n + 1} \frac{x_{n + 1} - x_n}{y_{n + 1} - y_n}
    \end{equation*}
  }

  \uncover<+->{
    Resista la tentación de la fórmula simétrica:
    \begin{equation*}
      x_{n + 2}
        = \frac{x_n y_{n + 1} - x_{n + 1} y_n}{y_{n + 1} - y_n}
    \end{equation*}
  }
\end{frame}

\begin{frame}
  \frametitle{Método de secante}

  \begin{center}
    \begin{tikzpicture}
      \begin{axis}[
                    axis lines = center,
                    xlabel = {\(x\)},
                    ylabel = {\(y\)},
                    every axis y
                      label/.style = {at = (current axis.above origin),
                        anchor = south},
                    every axis x
                      label/.style = {at = (current axis.right of origin),
                        anchor = west},
                    xmin = 0,	 xmax = 0.5,
                    ymin = -0.6, ymax = 1.2,
                    xtick = {1/20, 3/10, 93/260},
                    xticklabels = {\small \(x_0\),
                                   \small \(x_1\),
                                   \small \(x_2\)},
                    ytick = {0},
                    yticklabels = {},
                    height = 6cm, width = 9cm
                  ]
        \addplot [blue]
           {(93 - 260 * x) / 100};
        \addplot [domain = 0.01:0.5, red]
           {(4740000 * x^4 - 7646000 * x^3 + 4231950 * x^2
              - 1167595 * x + 157926) / 136500};

        \draw [dashed, blue] (axis cs:1/20, 0) -- (axis cs:1/20, 4/5);
        \draw [dashed, blue] (axis cs:3/10, 0) -- (axis cs:3/10, 3/20);
      \end{axis}
    \end{tikzpicture}
\end{center}
\end{frame}

\begin{frame}
  \frametitle{Método de secante (programa)}
  \framesubtitle{Encabezado}

  \lstinputlisting[basicstyle = \small\sffamily,
                   frame = none,
                   linerange = {31-32}]
                  {secant.c}
\end{frame}

\begin{frame}
  \frametitle{Método de secante (programa)}
  \framesubtitle{Cuerpo}

  \lstinputlisting[basicstyle = \small\sffamily,
                   frame = none,
                   linerange = {34-43}]
                  {secant.c}
\end{frame}

\subsection{Iteración de punto fijo}

\begin{frame}
  \setcounter{beamerpauses}{2}
  \frametitle{Iteración de punto fijo}

  \uncover<+->{
    Un \emph{punto fijo} de la función \(g\) es \(x^*\)
    tal que \(x^* = g(x^*)\).
  }
  \uncover<+->{
    Es natural intentar aproximar un punto fijo por:
    \begin{equation*}
      x_{n + 1}
        = g(x_n)
    \end{equation*}
  }
  \uncover<+->{
    Iniciamos la iteración eligiendo \(x_0\),
    cercano al cero buscado.
  }
  \bigskip \\
  \uncover<+->{
    Siempre podemos reescribir la ecuación \(f(x) = 0\) como un punto fijo,
    por ejemplo \(x = x - \alpha f(x)\) para \(\alpha\) adecuado
    (incluso variable).
    Mejor opción puede ser despejar del término más significativo.
  }
\end{frame}

\begin{frame}
  \frametitle{Punto fijo (gráficamente)}

  \begin{center}
    \begin{tikzpicture}[scale = 1.2]
      \begin{axis}[
                    axis lines = center,
                    xlabel = {\(X\)},
                    ylabel = {\(Y\)},
                    every axis y
                      label/.style = {at = (current axis.above origin),
                        anchor = north},
                    every axis x
                      label/.style = {at = (current axis.right of origin),
                        anchor = east},
                    xmin = 0, xmax = 1.75,
                    ymin = 0, ymax = 1.75,
                    xtick = {0.81},
                    xticklabels = {\(x^*\)},
                    ytick = {0},
                    yticklabels = {},
                    height = 6cm, width = 6cm
                  ]
        \addplot [domain = 0:1.45, color = blue]
           {x} node [yshift = 6pt, xshift = 5pt] (yx) {\(y = x\)};
        \addplot [domain = 0.45:1.6, red] {(1.5 - 0.7 * x)^3}
           node [yshift = 10pt] (g) {\(g(x)\)};
        \draw [dashed, blue] (axis cs:0.81, 0) -- (axis cs:0.81, 1.11);
      \end{axis}
    \end{tikzpicture}
  \end{center}
\end{frame}

\begin{frame}
  \frametitle{Iteración de punto fijo -- convergencia}

  \begin{center}
    \begin{tikzpicture}[scale = 0.92]
      \begin{axis}[domain = 0:0.9,
                   axis x line = bottom, axis y line = left,
                   xmin = 0, xmax = 0.9,
                   ymin = 0, ymax = 0.9,
                   xtick = {0.1, 0.812, 0.25930573, 0.7084739004713133,
                     0.3257801870289462, 0.6545450715920982,
                     0.36698903565963403, 0.5},
                   xticklabels = {\(x_0^{\phantom{\ast}}\),
                                  \(x_1^{\phantom{\ast}}\),
                                  \(x_2^{\phantom{\ast}}\),
                                  \(x_3^{\phantom{\ast}}\),
                                  \(x_4^{\phantom{\ast}}\),
                                  \(\),
                                  \(\),
                                  \(x^{\ast}_{\phantom{0}}\)},
                   ytick style = {draw = none},
                   yticklabels = {}]

        \addplot[mark = none, blue] {x};
        \addplot[mark = none, red]
          {1 * x^3 / 1 - 7 * x^2 / 5 - 1 * x / 4 + 17 / 20};

      \draw [blue, very thin, dashed]
         (axis cs:0.1, 0) -- (axis cs:0.1, 0.812);
      \draw [blue, very thin, dashed]
         (axis cs:0.1, 0.812) -- (axis cs:0.812, 0.812);
      \draw [blue, very thin, dashed]
         (axis cs:0.812, 0.812) -- (axis cs:0.812, 0.25930573);
      \draw [blue, very thin, dashed]
         (axis cs:0.812, 0.25930573) -- (axis cs:0.25930573, 0.25930573);
      \draw [blue, very thin, dashed]
         (axis cs:0.25930573, 0.25930573)
              -- (axis cs:0.25930573, 0.7084739004713133);
      \draw [blue, very thin, dashed]
         (axis cs:0.25930573, 0.25930573)
              -- (axis cs:0.25930573, 0.7084739004713133);
      \draw [blue, very thin, dashed]
         (axis cs:0.25930573, 0.7084739004713133)
              -- (axis cs:0.7084739004713133, 0.7084739004713133);
      \draw [blue, very thin, dashed]
         (axis cs:0.7084739004713133, 0.7084739004713133)
              -- (axis cs:0.7084739004713133, 0.3257801870289462);
      \draw [blue, very thin, dashed]
         (axis cs:0.7084739004713133, 0.7084739004713133)
              -- (axis cs:0.7084739004713133, 0.3257801870289462);
      \draw [blue, very thin, dashed]
         (axis cs:0.7084739004713133, 0.3257801870289462)
              -- (axis cs:0.3257801870289462, 0.3257801870289462);
      \draw [blue, very thin, dashed]
         (axis cs:0.3257801870289462, 0.3257801870289462)
              -- (axis cs:0.3257801870289462, 0.6545450715920982);
      \draw [blue, very thin, dashed]
         (axis cs:0.3257801870289462, 0.3257801870289462)
              -- (axis cs:0.3257801870289462, 0.6545450715920982);
      \draw [blue, very thin, dashed]
         (axis cs:0.3257801870289462, 0.6545450715920982)
              -- (axis cs:0.6545450715920982, 0.6545450715920982);
      \draw [blue, very thin, dashed]
         (axis cs:0.6545450715920982, 0.6545450715920982)
              -- (axis cs:0.6545450715920982, 0.36698903565963403);
      \draw [blue, very thin, dashed]
         (axis cs:0.6545450715920982, 0.36698903565963403)
              -- (axis cs:0.36698903565963403, 0.36698903565963403);
      \end{axis}
    \end{tikzpicture}

    Le llaman diagrama de telaraña
    (\emph{\foreignlanguage{english}{cobweb diagram}} en inglés).
  \end{center}
\end{frame}

\begin{frame}
  \frametitle{Iteración de punto fijo -- divergencia}

  \begin{center}
    \begin{tikzpicture}
      \begin{axis}[domain = 0:1,
                   axis x line = bottom, axis y line = left,
                   xmin = 0, xmax = 1,
                   ymin = 0, ymax = 1,
                   xtick = {0.3, 0.7425, 0.245202598, 0.8053675633204307,
                     0.19829350349601982, 0.8561644490113705,
                     0.1687199716882174, 0.5},
                   xticklabels = {\(x_0^{\phantom{\ast}}\),
                                  \(x_1^{\phantom{\ast}}\),
                                  \(x_2^{\phantom{\ast}}\),
                                  \(x_3^{\phantom{\ast}}\),
                                  \(\),
                                  \(x_5^{\phantom{\ast}}\),
                                  \(x_6^{\phantom{\ast}}\),
                                  \(x^{\ast}_{\phantom{0}}\)},
                   ytick style = {draw = none},
                   yticklabels = {}]

        \addplot[mark = none, blue] {x};
        \addplot[mark = none, red]
          {5 * x^3 / 4 - 25 * x^2 / 16 - 23 * x / 40 + 327 / 320};

      \draw [blue, very thin, dashed]
         (axis cs:0.3, 0) -- (axis cs:0.3, 0.7425);
      \draw [blue, very thin, dashed]
         (axis cs:0.3, 0.7425) -- (axis cs:0.7425, 0.7425);
      \draw [blue, very thin, dashed]
         (axis cs:0.7425, 0.7425) -- (axis cs:0.7425, 0.245202598);
      \draw [blue, very thin, dashed]
         (axis cs:0.7425, 0.245202598) -- (axis cs:0.245202598, 0.245202598);
      \draw [blue, very thin, dashed]
         (axis cs:0.245202598, 0.245202598)
             -- (axis cs:0.245202598, 0.8053675633204307);
      \draw [blue, very thin, dashed]
         (axis cs:0.245202598, 0.245202598)
             -- (axis cs:0.245202598, 0.8053675633204307);
      \draw [blue, very thin, dashed]
      (axis cs:0.245202598, 0.8053675633204307)
             -- (axis cs:0.8053675633204307, 0.8053675633204307);
      \draw [blue, very thin, dashed]
         (axis cs:0.8053675633204307, 0.8053675633204307)
             -- (axis cs:0.8053675633204307, 0.19829350349601982);
      \draw [blue, very thin, dashed]
         (axis cs:0.8053675633204307, 0.8053675633204307)
             -- (axis cs:0.8053675633204307, 0.19829350349601982);
      \draw [blue, very thin, dashed]
         (axis cs:0.8053675633204307, 0.19829350349601982)
             -- (axis cs:0.19829350349601982, 0.19829350349601982);
      \draw [blue, very thin, dashed]
         (axis cs:0.19829350349601982, 0.19829350349601982)
             -- (axis cs:0.19829350349601982, 0.8561644490113705);
      \draw [blue, very thin, dashed]
         (axis cs:0.19829350349601982, 0.19829350349601982)
              -- (axis cs:0.19829350349601982, 0.8561644490113705);
      \draw [blue, very thin, dashed]
         (axis cs:0.19829350349601982, 0.8561644490113705)
              -- (axis cs:0.8561644490113705, 0.8561644490113705);
      \draw [blue, very thin, dashed]
         (axis cs:0.8561644490113705, 0.8561644490113705)
              -- (axis cs:0.8561644490113705, 0.1687199716882174);
      \draw [blue, very thin, dashed]
         (axis cs:0.8561644490113705, 0.1687199716882174)
              -- (axis cs:0.1687199716882174, 0.1687199716882174);
      \end{axis}
    \end{tikzpicture}
  \end{center}
\end{frame}

\begin{frame}
  \frametitle{Iteración de punto fijo -- convergencia}

  \begin{center}
    \begin{tikzpicture}
      \begin{axis}[domain = 0:1,
                   axis x line = bottom, axis y line = left,
                   xmin = 0, xmax = 1,
                   ymin = 0, ymax = 1,
                   xtick = {0.1, 0.5000066, 0.678700259368431587,
                     0.7312996515938205, 0.74701294446039,
                     0.7518459248664618, 0.7533485113148211, 0.75},
                   xticklabels = {\(x_0^{\phantom{\ast}}\),
                                  \(x_1^{\phantom{\ast}}\),
                                  \(x_2^{\phantom{\ast}}\),
                                  \(\),
                                  \(\),
                                  \(\),
                                  \(\),
                                  \(x^{\ast}_{\phantom{0}}\)},
                   ytick style = {draw = none},
                   yticklabels = {}]

        \addplot[mark = none, blue] {x};
        \addplot[mark = none, red]
                {680 * x^3 / 1521 - 1270 * x^2 / 1521
                  + 16409 * x / 20280 + 5773 / 13520};

      \draw [blue, very thin, dashed]
         (axis cs:0.1, 0) -- (axis cs:0.1, 0.5000066);
      \draw [blue, very thin, dashed]
         (axis cs:0.1, 0.5000066) -- (axis cs:0.5000066, 0.5000066);
      \draw [blue, very thin, dashed]
         (axis cs:0.5000066, 0.5000066)
              -- (axis cs:0.5000066, 0.678700259368431587);
      \draw [blue, very thin, dashed]
         (axis cs:0.5000066, 0.678700259368431587)
              -- (axis cs:0.678700259368431587, 0.678700259368431587);
      \draw [blue, very thin, dashed]
         (axis cs:0.678700259368431587, 0.678700259368431587)
              -- (axis cs:0.678700259368431587, 0.7312996515938205);
      \draw [blue, very thin, dashed]
         (axis cs:0.678700259368431587, 0.678700259368431587)
              -- (axis cs:0.678700259368431587, 0.7312996515938205);
      \draw [blue, very thin, dashed]
         (axis cs:0.678700259368431587, 0.7312996515938205)
              -- (axis cs:0.7312996515938205, 0.7312996515938205);
      \draw [blue, very thin, dashed]
         (axis cs:0.7312996515938205, 0.7312996515938205)
              -- (axis cs:0.7312996515938205, 0.74701294446039);
      \draw [blue, very thin, dashed]
         (axis cs:0.7312996515938205, 0.7312996515938205)
              -- (axis cs:0.7312996515938205, 0.74701294446039);
      \draw [blue, very thin, dashed]
         (axis cs:0.7312996515938205, 0.74701294446039)
              -- (axis cs:0.74701294446039, 0.74701294446039);
      \draw [blue, very thin, dashed]
         (axis cs:0.74701294446039, 0.74701294446039)
              -- (axis cs:0.74701294446039, 0.7518459248664618);
      \draw [blue, very thin, dashed]
         (axis cs:0.74701294446039, 0.74701294446039)
              -- (axis cs:0.74701294446039, 0.7518459248664618);
      \draw [blue, very thin, dashed]
         (axis cs:0.74701294446039, 0.7518459248664618)
              -- (axis cs:0.7518459248664618, 0.7518459248664618);
      \draw [blue, very thin, dashed]
         (axis cs:0.7518459248664618, 0.7518459248664618)
              -- (axis cs:0.7518459248664618, 0.7533485113148211);
      \draw [blue, very thin, dashed]
         (axis cs:0.7518459248664618, 0.7533485113148211)
              -- (axis cs:0.7533485113148211, 0.7533485113148211);
      \end{axis}
    \end{tikzpicture}
  \end{center}
\end{frame}

\begin{frame}
  \frametitle{Iteración de punto fijo -- divergencia}

  \begin{center}
    \begin{tikzpicture}
      \begin{axis}[domain = 0:1,
                   axis x line = bottom, axis y line = left,
                   xmin = 0, xmax = 1,
                   ymin = 0, ymax = 1,
                   xtick = {0.3, 0.323028, 0.355531714912121,
                     0.40050722374171244, 0.4615564206393762,
                     0.5436710682749981, 0.6563153774050898, 0.25},
                   xticklabels = {\(x_0^{\phantom{\ast}}\),
                                  \(\),
                                  \(\),
                                  \(x_3^{\phantom{\ast}}\),
                                  \(x_4^{\phantom{\ast}}\),
                                  \(x_5^{\phantom{\ast}}\),
                                  \(x_6^{\phantom{\ast}}\),
                                  \(x^{\ast}_{\phantom{0}}\)},
                   ytick style = {draw = none},
                   yticklabels = {}]

        \addplot[mark = none, blue] {x};
        \addplot[mark = none, red]
                {2450 * x^3 / 1521 - 3160 * x^2 / 1521 + 27217 * x / 12168
                  - 553 / 2704};

      \draw [blue, very thin, dashed]
         (axis cs:0.3, 0) -- (axis cs:0.3, 0.323028);
      \draw [blue, very thin, dashed]
         (axis cs:0.3, 0.323028) -- (axis cs:0.323028, 0.323028);
      \draw [blue, very thin, dashed]
         (axis cs:0.323028, 0.323028) -- (axis cs:0.323028, 0.355531714912121);
      \draw [blue, very thin, dashed]
         (axis cs:0.323028, 0.355531714912121)
              -- (axis cs:0.355531714912121, 0.355531714912121);
      \draw [blue, very thin, dashed]
         (axis cs:0.355531714912121, 0.355531714912121)
              -- (axis cs:0.355531714912121, 0.40050722374171244);
      \draw [blue, very thin, dashed]
         (axis cs:0.355531714912121, 0.355531714912121)
              -- (axis cs:0.355531714912121, 0.40050722374171244);
      \draw [blue, very thin, dashed]
         (axis cs:0.355531714912121, 0.40050722374171244)
              -- (axis cs:0.40050722374171244, 0.40050722374171244);
      \draw [blue, very thin, dashed]
         (axis cs:0.40050722374171244, 0.40050722374171244)
              -- (axis cs:0.40050722374171244, 0.4615564206393762);
      \draw [blue, very thin, dashed]
         (axis cs:0.40050722374171244, 0.40050722374171244)
              -- (axis cs:0.40050722374171244, 0.4615564206393762);
      \draw [blue, very thin, dashed]
         (axis cs:0.40050722374171244, 0.4615564206393762)
              -- (axis cs:0.4615564206393762, 0.4615564206393762);
      \draw [blue, very thin, dashed]
         (axis cs:0.4615564206393762, 0.4615564206393762)
              -- (axis cs:0.4615564206393762, 0.5436710682749981);
      \draw [blue, very thin, dashed]
         (axis cs:0.4615564206393762, 0.4615564206393762)
              -- (axis cs:0.4615564206393762, 0.5436710682749981);
      \draw [blue, very thin, dashed]
         (axis cs:0.4615564206393762, 0.5436710682749981)
              -- (axis cs:0.5436710682749981, 0.5436710682749981);
      \draw [blue, very thin, dashed]
         (axis cs:0.5436710682749981, 0.5436710682749981)
              -- (axis cs:0.5436710682749981, 0.6563153774050898);
      \draw [blue, very thin, dashed]
         (axis cs:0.5436710682749981, 0.6563153774050898)
              -- (axis cs:0.6563153774050898, 0.6563153774050898);
      \end{axis}
    \end{tikzpicture}
  \end{center}
\end{frame}

\begin{frame}
  \frametitle{Iteración de punto fijo (programa)}
  \framesubtitle{Encabezado}

  \lstinputlisting[basicstyle = \small\sffamily,
                   frame = none,
                   linerange = {30-31}]
                  {fpi.c}
\end{frame}

\begin{frame}
  \frametitle{Iteración de punto fijo (programa)}
  \framesubtitle{Cuerpo}

  \lstinputlisting[basicstyle = \small\sffamily,
                   frame = none,
                   linerange = {33-41}]
                  {fpi.c}
\end{frame}

\section{Resumen}

\begin{frame}
  \setcounter{beamerpauses}{2}
  \frametitle{Resumen}

  \begin{itemize}
  \item
    Un problema importante es aproximar ceros
  \item
    Vimos varias ideas aplicables a este problema.
    Los programas básicos son simples.
    \uncover<+->{
      Armar rutinas de uso general es otra historia\ldots
    }
  \item
    En el apunte se discuten algunos métodos adicionales.
  \item
    Resta ver cuándo convergen,
    y qué tan rápido
  \end{itemize}
\end{frame}

\end{document}

%%% Local Variables:
%%% mode: latex
%%% TeX-master: t
%%% ispell-local-dictionary: "spanish"
%%% End:

% LocalWords:  glyphtounicode english cobweb diagram
