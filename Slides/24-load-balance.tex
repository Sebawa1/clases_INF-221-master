\input glyphtounicode%
\pdfgentounicode=1
\documentclass[english, spanish, fleqn,%
hyperref = {colorlinks, urlcolor = blue}%
%, handout%
]{beamer}

%% ]{article}
%% \usepackage{beamerarticle}

\usepackage{beamerthemesplit}

\usepackage{fourier}
\usepackage[group-digits = false, copy-decimal-marker = true]{siunitx}
\usepackage{amsmath}
\usepackage{amsthm}
\usepackage{dcolumn}
\usepackage[es-noquoting]{babel}
\usepackage{csquotes}

\ifdefined\HCode
  \def\pgfsysdriver{pgfsys-dvisvgm4ht.def}
\fi

\usefonttheme{professionalfonts}

\DeclareMathOperator{\Exp}{\mathbb{E}}
\DeclareMathOperator{\var}{var}

\beamerdefaultoverlayspecification{<+->}

\title{Balance de carga}

\author[Horst H. von Brand]{Horst H. von Brand\\
  \href{mailto:vonbrand@inf.utfsm.cl}{vonbrand@inf.utfsm.cl}}

\institute[DI UTFSM]{Departamento de Informática\\
                     Universidad Técnica Federico Santa María}
\date{}

\begin{document}
\frame{\maketitle}

\begin{frame}<handout>
  \frametitle{Contenido}

  \tableofcontents
\end{frame}

\section{Justificación}

\begin{frame}
  \setcounter{beamerpauses}{2}
  \frametitle{Asignar al azar}

  \uncover<+->{
    Cuando hay que elegir entre diversos recursos,
    asignar al azar reparte la carga.
    Particularmente interesante en sistemas distribuidos,
    es decisión local sin conocimiento global.
  }
\end{frame}

\begin{frame}
  \setcounter{beamerpauses}{2}
  \frametitle{Ejemplo a discutir}

  \uncover<+->{
    Supongamos un nuevo sitio social,
    \emph{MalaLeche},
    donde se reclama por todo
    (incluso el funcionamiento del sitio).
  }
  \uncover<+->{
    El tráfico es alto,
    se requieren varios procesadores.
    Si alguna de las máquinas se ve sobrepasada,
    el rendimiento sufre
    (con los consiguientes reclamos).
  }
  \uncover<+->{
    Queremos determinar cuántos equipos se requieren para manejar la carga
    si se distribuye al azar.
  }
\end{frame}

\begin{frame}
  \frametitle{Especificaciones ejemplo}

  \uncover<+->{
    Específicamente,
    MalaLeche recibe \num[group-digits]{24 000} peticiones
    en cada período de \num{10} minutos.
  }
  \uncover<+->{
    Las peticiones
    toman a lo más \SI{1}{\second} de procesamiento;
    aunque la mayoría son triviales
    (quejarse de la ortografía del mensaje precedente
     y similares),
     el tiempo promedio de ejecución es \SI{0,25}{\second}.
  }
  \uncover<+->{
    Midiendo el trabajo en unidades de \SI{1}{\second} de procesamiento,
    si a alguno de los servidores
    se le asignan más de \num{600} unidades de trabajo en \SI{10}{\minute},
    se cae y produce problemas.
  }
\end{frame}

\begin{frame}
  \frametitle{Especificaciones ejemplo}

  \uncover<+->{
    La carga total de MalaLeche
    de \(\num{24 000} \cdot \num{0,25} = \num{6 000}\) unidades de trabajo
    cada \SI{10}{\minute}
    indica que se requieren \num{010} máquinas
    trabajando al \SI{100}{\percent} con balance de carga perfecto.
  }
  \uncover<+->{
    Necesitaremos más de \num{10} para acomodar fluctuaciones en la carga
    y balance imperfecto de carga,
    la pregunta es cuántos se requieren.
  }
\end{frame}

\begin{frame}
  \frametitle{Desarrollo}

  \uncover<+->{
    Específicamente,
    nos interesa el número \(m\) de servidores que hace muy poco probable
    que alguno se vea sobrecargado al asignarle
    más de \(600\) unidades de trabajo en un período de \(10\) minutos.
  }
  \\ \medskip

  \uncover<+->{
    Acotemos la probabilidad de que el primer servidor se sobrecargue
    en un período dado.
  }
  \uncover<+->{
    Sea \(T\) las unidades de trabajo asignadas a ese servidor,
    buscamos una cota superior a \(\Pr[T \ge \num{600}]\).
  }
  \uncover<+->{
    Sea \(t_i\) el tiempo que la primera máquina dedica a la tarea \(i\),
    con lo que \(t_i = 0\) si se asigna a otra máquina.
    Así,
    con \(n = 24\,000\):
    \begin{equation*}
      T
        = \sum_{1 \le i \le n} t_i
    \end{equation*}
  }
\end{frame}

\begin{frame}
  \frametitle{Desarrollo}

  \uncover<+->{
    Podemos usar las cotas de Chernoff
    si las variables son independientes
    y en el rango \([0, 1]\).
  }
  \uncover<+->{
    La primera condición se cumple
    si la asignación de tareas a los servidores
    no depende de su tiempo de ejecución,
    la segunda se da porque ninguna tarea toma más de una unidad.
  }
\end{frame}

\begin{frame}
  \frametitle{Desarrollo}

  \uncover<+->{
    Hay \num{24 000} tareas,
    cada una de tiempo de procesamiento esperado de \SI{0,25}{\second}.
    Asignado tareas al azar a los servidores,
    la carga esperada para el primer servidor es:
    \begin{align*}
      \Exp[T]
        &= \frac{\num{24 000} \cdot \num{0,25}}{m} \\
        &= \frac{\num{6 000}}{m}
    \end{align*}
    Como vimos,
    con \(m < 10\),
    esperamos que se sobrecargue,
    con \(m = 10\) está al \SI{100}{\percent} de capacidad.
  }
\end{frame}

\begin{frame}
  \frametitle{Desarrollo}

  \uncover<+->{
    Nos interesa el límite:
    \begin{equation*}
      600 = c \Exp[T]
    \end{equation*}
    con lo que \(c = m / 10\).
  }
  \uncover<+->{
    La cota de Chernoff es:
    \begin{align*}
      \Pr[T \ge 600]
        &= \Pr\left[ T \ge \frac{m}{10} \cdot \Exp[T] \right] \\
        &\le \mathrm{e}^{-\beta(m / 10) \cdot \num{6 000} / m}
    \end{align*}
    donde:
    \begin{equation*}
      \beta(c)
        = c \, \ln c - c + 1
    \end{equation*}
  }
\end{frame}

\begin{frame}
  \frametitle{Desarrollo}

  \uncover<+->{
    Por la cota de la unión,
    la probabilidad de que \emph{alguna} de las máquinas
    se sobrecargue en una hora cualquiera es:
    \begin{align*}
      \Pr[\text{alguna se sobrecarga}]
        &\le \sum_{1 \le i \le m} \Pr[\text{el servidor \(i\) se sobrecarga}] \\
        &=   m \Pr[\text{el servidor \(1\) se sobrecarga}] \\
        &\le m \mathrm{e}^{-\beta(m / 10) \cdot \num{6 000} / m}
    \end{align*}
  }
\end{frame}

\begin{frame}
  \frametitle{Desarrollo}

  \uncover<+->{
    Algunos valores se tabulan a continuación:
    \begin{center}
      \begin{tabular}[ht]{>{\(}c<{\):}>{\(}l<{\)}}
        m = 11 & \num{0,784} \\
        m = 12 & \num{0,000999} \\
        m = 13 & \num{0,0000000760}
      \end{tabular}
    \end{center}
  }
  \uncover<+->{
    O sea,
    con \num{11} máquinas alguna puede caerse casi inmediatamente,
    \num{12} debieran durar unos días,
    mientras \num{13} dan para un siglo o dos.
  }
\end{frame}

\section{Elegir entre dos}

\begin{frame}
  \setcounter{beamerpauses}{2}
  \frametitle{Elegir entre dos}

  \uncover<+->{
    Si se revisan al azar dos de las máquinas
    y se asigna la tarea a  aquella con menos carga,
    la carga máxima esperada disminuye muy substancialmente.
  }
  \uncover<+->{
    No es necesario revisar todas las máquinas,
    es fácil de hacer las consultas en paralelo.
  }
\end{frame}

\begin{frame}
  \setcounter{beamerpauses}{2}
  \frametitle{Distribución al azar}

  \uncover<+->{
    El resultado clásico
    es que si se distribuyen \(m\) bolas
    en \(m\) casilleros,
    con alta probabilidad el casillero más lleno contiene:
    \begin{equation*}
      \frac{\ln m}{\ln \ln m} \, (1 + o(1))
    \end{equation*}
  }

\end{frame}

\begin{frame}
  \setcounter{beamerpauses}{2}
  \frametitle{Elegir el menos ocupado}

  \uncover<+->{
    Si hay \(m\) casilleros
    y se distribuyen \(n\)
    (\(n > m\))
    bolas entre ellos,
    si se eligen \(d\) casilleros al azar cada vez
    y se pone la bola en el más vacío
    donde \(d \ge 2\),
    el largo de la cola más larga con alta probabilidad es:
    \begin{equation*}
      \frac{\ln \ln m}{\ln d} \, (1 + o(1)) + \Theta(n/m)
    \end{equation*}
  }
  \uncover<+->{
    Vale decir,
    de \(d = 1\) a \(d = 2\) hay una mejora exponencial;
    para \(d \ge 2\) el cambio es solo en un factor constante moderado.
  }
\end{frame}

\section{Resumen}

\begin{frame}
  \setcounter{beamerpauses}{2}
  \frametitle{Resumen}

  \begin{itemize}
  \item
    Distribuir carga equitativamente es un problema espinudo.
  \item
    Simplemente asignar al azar da buenos resultados.
  \item
    Mejora substancialmente si revisamos dos máquinas al azar
    y asignamos la tarea a la menos ocupada.
  \end{itemize}
\end{frame}
\end{document}

%%% Local Variables:
%%% mode: latex
%%% TeX-master: t
%%% ispell-local-dictionary: "spanish"
%%% End:

% LocalWords:  glyphtounicode MalaLeche
