\input glyphtounicode%
\pdfgentounicode=1
\documentclass[english, spanish, fleqn,%
hyperref = {colorlinks, urlcolor = blue}%
%, handout%
]{beamer}

%% ]{article}
%% \usepackage{beamerarticle}

\usepackage{beamerthemesplit}

\usepackage{fourier}
\usepackage{amsmath}
\usepackage{amsthm}
\usepackage[es-noquoting]{babel}
\usepackage{csquotes}

\ifdefined\HCode
  \def\pgfsysdriver{pgfsys-dvisvgm4ht.def}
\fi

\usefonttheme{professionalfonts}

\beamerdefaultoverlayspecification{<+->}

\title{Cuadraura de Gauß}

\author[Horst H. von Brand]{Horst H. von Brand\\
  \href{mailto:vonbrand@inf.utfsm.cl}{vonbrand@inf.utfsm.cl}}

\institute[DI UTFSM]{Departamento de Informática\\
                     Universidad Técnica Federico Santa María}
\date{}

\begin{document}
\frame{\maketitle}

\begin{frame}<handout>
  \frametitle{Contenido}

  \tableofcontents
\end{frame}

\section{Mejores cuadraturas}

\begin{frame}
  \setcounter{beamerpauses}{2}
  \frametitle{Elegir nodos}

  \uncover<+->{
    Si \emph{elegimos} los \(n + 1\) puntos \(x_0, \dotsc, x_n\)
    en vez de aceptar los dados,
    tenemos \(2 n + 2\)~variables,
    con las que podemos cumplir \(2 n + 2\)~condiciones,
    como hacer que la regla sea exacta para polinomios de grado \(2 n + 1\).
  }
  \\ \medskip
  \uncover<+->{
    La forma obvia es plantear el sistema de ecuaciones
    (no lineal)\ldots
  }
  \uncover<+->{
    Daremos un desvío.
  }
\end{frame}

\section{Polinomios ortogonales}

\begin{frame}
  \frametitle{Norma de funciones}

  \begin{definition}
    Sea el intervalo \([a, b]\)
    (los extremos pueden ser infinitos),
    y una función peso \(w \colon [a, b] \to \mathbb{R}\),
    continua y positiva.
    Dadas funciones \(f\) y \(g\),
    continuas sobre el intervalo,
    su \emph{producto interno}
    (sobre \([a, b]\) respecto de \(w\)) es:
    \begin{equation*}
      \langle f, g \rangle_w
        = \int_a^b w(x) f(x) g(x) \mathrm{d} x
    \end{equation*}
  \end{definition}
\end{frame}

\begin{frame}
  \frametitle{Norma de funciones}

  \begin{definition}[cont]
    La \emph{norma}
    (sobre \([a, b]\) respecto de \(w\)) de \(f\) es:
    \begin{equation*}
      \lVert f \rVert_w
        = \sqrt{\langle f, f \rangle_w}
    \end{equation*}

    Decimos que \(f\) y \(g\) son \emph{ortogonales}
    (sobre \([a, b]\) con peso \(w\))
    si \(\langle f, g \rangle_w = 0\).
    Decimos que \(f\) está \emph{normalizada}
    (sobre \([a, b]\) con peso \(w\))
    si \(\lVert f \rVert_w = 1\)
  \end{definition}
\end{frame}

\begin{frame}
  \setcounter{beamerpauses}{2}
  \frametitle{Espacio normado}

  \uncover<+->{
    Con la definición anterior las funciones continuas sobre \([a, b]\)
    y las definiciones obvias de suma de funciones
    y multiplicación por un escalar
    forman un espacio vectorial normado.
  }

  \uncover<+->{
    Nos interesa particularmente el conjunto de polinomios.
    Llamaremos \(\Pi_n\) al conjunto de polinomios de grado a lo más \(n\).
  }
\end{frame}

\begin{frame}
  \setcounter{beamerpauses}{2}
  \frametitle{Polinomios ortogonales}

  \begin{definition}<+->
    Sea \(w\) una función de peso sobre \([a, b]\).
    Diremos que la secuencia de polinomios \(p_n(x)\),
    donde \(p_n\) es de grado~\(n\),
    es \emph{ortogonal} si para \(i \ne j\)
    es \(\langle p_i, p_j \rangle_w = 0\).
    Son \emph{ortonormales} si además \(\lVert p_n \rVert = 1\).
  \end{definition}
  \bigskip
  \uncover<+->{
    Pueden construirse de cualquier conjunto de vectores independientes
    (en nuestro caso,
     por ejemplo \(\{1, x, x^2, \dotsc\}\))
    por el proceso de Gram-Schmidt.
  }
\end{frame}

\begin{frame}
  \setcounter{beamerpauses}{2}
  \frametitle{Polinomios ortogonales}

  \uncover<+->{
    Note que si \(p_k(x)\) es un polinomio ortogonal de grado \(k\),
    todo polinomio \(q(x)\) de grado \(n\) puede expresarse:
    \begin{equation*}
      q(x)
        = a_n p_n(x) + a_{n - 1} p_{n - 1}(x) + \dotsb + a_0 p_0(x)
    \end{equation*}
  }
  \uncover<+->{
    (Use \(p_n(x)\) para eliminar el término en \(x^n\),
     y continúe con el siguiente término.)
   }

   \uncover<+->{
     En consecuencia:
   }
  \begin{align*}
    \uncover<+->{
      \int_a^b w(x) p_{n + 1}(x) q(x) \, \mathrm{d} x
        &= \int_a^b w(x) p_{n + 1}(x)
                      \left(
                        a_n p_n(x) + a_{n - 1} p_{n - 1}(x) + \dotsm + a_0 p_0(x)
                      \right) \, \mathrm{d} x \\
    }
    \uncover<+->{
        &= 0
    }
  \end{align*}
\end{frame}

\begin{frame}
  \frametitle{Ceros de polinomios ortogonales}

  \begin{theorem}
    Sea \(p(x)\) un polinomio ortogonal de grado~\(n\) sobre \([a, b]\)
    con función de peso \(w\).
    Entonces \(p\) tiene \(n\) ceros simples en \([a, b]\).
  \end{theorem}
\end{frame}

\begin{frame}
  \frametitle{Ceros de polinomios ortogonales}

  \begin{proof}
    Sean \(x_1, x_2, \dotsc, x_r\) los ceros
    (posiblemente repetidos)
    de \(p\) en \([a, b]\).
    Sea \(q(x) = (x - x_1) \dotsm (x - x_r)\),
    \(\deg(q) = r \le n\).
    Entonces \(p(x) q(x)\) no cambia de signo en el intervalo:
    \begin{equation*}
      \int_a^b w(x) p(x) q(x) \, \mathrm{d} x
        \ne 0
    \end{equation*}
    Esto es imposible si \(\deg(q) < \deg(p)\).

    Supongamos \(x_1\) un cero múltiple de \(p\),
    consideremos \(g(x) = p(x) / (x - x_1)^2\):
    \begin{equation*}
      \int_a^b w(x) p(x) g(x) \, \mathrm{d} x
        = \int_a^b w(x) \left(\frac{p(x)}{x - x_1}\right)^2 \, \mathrm{d} x
        > 0
    \end{equation*}
    contradicción.
  \end{proof}
\end{frame}

\begin{frame}
  \setcounter{beamerpauses}{2}
  \frametitle{Cuadratura de Gauß}

  \uncover<+->{
    Buscamos reglas de cuadratura de la forma:
    \begin{equation*}
      \int_a^b w(x) f(x) \, \mathrm{d} x
         \approx \sum_{0 \le k \le n} A_k f(x_k)
    \end{equation*}
  }
  \uncover<+->{
    Si es exacta para polinomios de grado hasta \(n\),
    debe ser:
    \begin{equation*}
      A_k
        = \int_a^b w(x) \ell_k(x) \, \mathrm{d} x
    \end{equation*}
  }
\end{frame}

\begin{frame}
  \frametitle{Cuadratura de Gauß}

  \begin{theorem}
    Sea \(p(x)\) de grado \(n + 1\),
    ortogonal a \(\Pi_n\).
    Si \(x_i\) para \(0 \le i \le n\) son los ceros de \(p\),
    la regla de cuadratura es exacta para \(f \in \Pi_{2 n + 1}\).
  \end{theorem}
\end{frame}

\begin{frame}
  \frametitle{Cuadratura de Gauß}

  \begin{proof}
    Sea \(f \in \Pi_{2 n + 1}\).
    Por el algoritmo de división:
    \begin{equation*}
      f(x)
         = q(x) p(x) + r(x)
    \end{equation*}
    con \(q, r \in \Pi_n\).
    Así,
    es \(f(x_i) = r(x_i)\) para \(0 \le i \le n\)
    y:
    \begin{equation*}
      \int_a^b w(x) f(x) \, \mathrm{d} x
        = \int_a^b w(x) (p(x) q(x) + r(x)) \, \mathrm{d} x
        = \int_a^b w(x) r(x) \, \mathrm{d} x
    \end{equation*}
    Pero por construcción el método es exacto para \(r \in \Pi_n\):
    \begin{equation*}
      \int_a^b w(x) r(x) \, \mathrm{d} x
        = \sum_{0 \le i \le n} A_i r(x_i)
        = \sum_{0 \le i \le n} A_i f(x_i)
    \end{equation*}
  \end{proof}
\end{frame}

\begin{frame}
  \frametitle{Cuadratura de Gauß}

  \begin{lemma}
    En una cuadratura de Gauß
    los coeficientes son positivos
    y:
    \begin{equation*}
      \sum_{0 \le i \le n} A_i
        = \int_a^b w(x) \, \mathrm{d} x
    \end{equation*}
  \end{lemma}
\end{frame}

\begin{frame}
  \frametitle{Cuadratura de Gauß}

  \begin{proof}
    Fijemos \(n\),
    y sea \(p(x) \in \Pi_{n + 1}\) \(w\)\nobreakdash-ortogonal a \(\Pi_n\),
    con lo que \(p(x_i) = 0\) para \(0 \le i \le n\).
    Elegimos \(j\), y definimos \(q(x) = p(x) / (x - x_j)\),
    con lo que \(\deg(q^2) = 2 n\) y es exacta:
    \begin{equation*}
      \int_a^b w(x) q^2(x) \, \mathrm{d} x
        = \sum_{0 \le i \le n} A_i q^2(x_i)
        = A_j q^2(x_j)
    \end{equation*}
    O sea,
    \(A_j > 0\).

    También es exacta:
    \begin{equation*}
      \int_a^b w(x) \, \mathrm{d} x
        = \sum_{0 \le i \le n} A_i
    \end{equation*}
  \end{proof}
\end{frame}

\section{Resumen}

\begin{frame}
  \setcounter{beamerpauses}{2}
  \frametitle{Resumen}

  \begin{itemize}
  \item
    Elegir los nodos permite obtener reglas de cuadratura
    exactas para grado mayor
  \item
    Usando polinomios ortogonales
    podemos obtener fórmulas elegantes para los nodos.
    Incluso podemos incorporar una función de peso.
  \end{itemize}
\end{frame}

\end{document}

%%% Local Variables:
%%% mode: latex
%%% TeX-master: t
%%% ispell-local-dictionary: "spanish"
%%% End:

% LocalWords:  glyphtounicode normado ortonormales
