\input glyphtounicode%
\pdfgentounicode=1
\documentclass[english, spanish, fleqn,%
hyperref = {colorlinks, urlcolor = blue}%
%, handout%
]{beamer}

%% ]{article}
%% \usepackage{beamerarticle}

\usepackage{beamerthemesplit}

\usepackage{fourier}
\usepackage[es-noquoting]{babel}
\usepackage{amsmath}
\usepackage{amsthm}
\usepackage{csquotes}
\usepackage[noline, noend]{algorithm2e}
\usepackage{listings}
\usepackage[basic]{complexity}
\usepackage{tikz}

\ifdefined\HCode
  \def\pgfsysdriver{pgfsys-dvisvgm4ht.def}
\fi

\usefonttheme{professionalfonts}

%%%
%%% For listings
%%%

\lstloadlanguages{[ANSI]C++}

\usetikzlibrary{positioning}

\beamerdefaultoverlayspecification{<+->}

\title{Branch and bound}

\author[Horst H. von Brand]{Horst H. von Brand\\
  \href{mailto:vonbrand@inf.utfsm.cl}{vonbrand@inf.utfsm.cl}}

\institute[DI UTFSM]{Departamento de Informática\\
                     Universidad Técnica Federico Santa María}
\date{}

\begin{document}
%%%
%%% For algorithm2e
%%%

\SetAlgorithmName{Algoritmo}{Algoritmo}{Índice de algoritmos}

\SetAlCapSty{mdseries}
\SetKwProg{Function}{function}{}{end}
\SetKwProg{Procedure}{procedure}{}{end}
\SetKw{Variables}{variables}
\SetKw{Downto}{downto}
\SetKwBlock{Loop}{loop}{end}
\SetKw{Continue}{continue}
\SetKw{Break}{break}
\SetKw{KwStep}{step}

%%%
%%% For listings
%%%

\lstset{basicstyle = \sffamily, commentstyle = \slshape,
        frame = none, numbers = none,
        tabsize = 4,
        showstringspaces = false
       }
\renewcommand{\lstlistingname}{Listado}

\frame{\maketitle}

\begin{frame}<handout>
  \frametitle{Contenido}

  \tableofcontents
\end{frame}

\section{El problema}

\begin{frame}
  \setcounter{beamerpauses}{2}
  \frametitle{Optimización}

  \uncover<+->{
    Muchos problemas solicitan el mejor objeto de entre una colección,
    sujeto a ciertas condiciones.
  }
  \uncover<+->{
    Para concretizar,
    supondremos que buscamos maximizar \(f(v)\)
    para el objeto \(v\).
    La situación de buscar un mínimo es simétrica.
  }

  \uncover<+->{
    Si la construcción del objeto se puede llevar a cabo en etapas,
    podemos aplicar técnicas
    afines a \emph{\foreignlanguage{english}{backtracking}}.
  }
  \uncover<+->{
    Si contamos con una función de cota \(b(v)\)
    que para un objeto parcial \(v\)
    nos da una cota superior al valor de \(f\)
    para todos los objetos que se pueden construir a partir de \(v\),
    podemos usar el valor de \(b\) para podar.
  }
  \uncover<+->{
    A esta idea se le llama \emph{\foreignlanguage{english}{branch and bound}}
    (literalmente,
     \textquote{ramificar y acotar} en inglés:
     \emph{ramificar} para explorar alternativas,
     \emph{acotar} descarta alternativas que sabemos no llevan al máximo).
   }
 \end{frame}

 \begin{frame}
   \setcounter{beamerpauses}{2}
   \frametitle{Algoritmo genérico}

   \uncover<+->{
     Como ya se dijo,
     tiene mucho en común con \emph{\foreignlanguage{english}{backtracking}}.
     La idea es mantener el máximo valor
     entre los objetos completos ya examinados,
     cualquier objeto parcialmente construido con cota menor que ese valor
     claramente es imposible que se complete al máximo,
     puede descartarse.
   }
   \\ \medskip
   \uncover<+->{
     Modelamos nuestro algoritmo genérico
     alrededor de búsqueda no recursiva,
     por ser maleable.
   }
   \uncover<+->{
     Nada impide combinar cotas con una heurística
     que ordene el cómputo para explorar primero alternativas más prometedoras
     (búsqueda mejor primero,
      \emph{\foreignlanguage{english}{Best First Search}}).
   }
   \uncover<+->{
     Puede ser que el valor de la función cota sea una heurística adecuada,
     o puede que contemos con otro criterio.
   }
 \end{frame}

 \begin{frame}
   \setcounter{beamerpauses}{2}
   \frametitle{Cotas}

   \uncover<+->{
     Una función \(b\) como se describe,
     que siempre sobrestima el valor posible de \(f\),
     se llama \emph{admisible}.
   }
   \uncover<+->{
     De no tener esa garantía,
     corremos el riesgo de descartar la solución óptima.
   }
   \\ \medskip
   \uncover<+->{
     Es claro que una función \(b\) que dé una cota más ajustada
     es preferible,
     pero hay que balancear el costo de calcular una cota cercana
     (probablemente más costosa de calcular)
     con el costo de revisar opciones que podrían haberse descartado.
   }
 \end{frame}

 \section{El algoritmo genérico}

\begin{frame}
  \setcounter{beamerpauses}{2}
  \frametitle{Convenciones}

  \uncover<+->{
    Suponemos dado el ámbito de trabajo.
  }
  \uncover<+->{
    Como antes,
    usaremos \(N(v)\) para denotar los vecinos del objeto (parcial) \(v\).
    La función \(g(v)\) dice si \(v\) es una meta
    (objeto completo),
    la función \(f(v)\) da el valor de un objeto meta \(v\),
    mientras \(b(v)\) es la cota superior discutida antes.
  }
  \\ \smallskip
  \uncover<+->{
    Acá \(N(v)\),
    los objetos vecinos al objeto (parcial) \(v\),
    depende de nuestra
    (probablemente arbitraria)
    división en etapas.
  }
  \uncover<+->{
    Es claro que \(g\) y \(f\) son parte de la descripción del problema.
  }
  \\ \medskip
  \uncover<+->{
    El algoritmo siguiente asume que si \(g(s)\) entonces \(b(s) = f(s)\)
    (esto es simple de forzar en \(b\)).
  }
 \end{frame}

\begin{frame}
  \setcounter{beamerpauses}{2}
  \frametitle{Convenciones}

  \uncover<+->{
    Comenzamos la búsqueda con un objeto \(s\),
    posiblemente el objeto \textquote{vacío}.
  }
  \uncover<+->{
    Cota inferior obvia es menos infinito,
    en la práctica dar un valor imposiblemente bajo.
  }
  \uncover<+->{
    Otra idea es construir algún objeto inicial no demasiado malo,
    si hay una forma simple de hacerlo.
  }
\end{frame}

\begin{frame}
  \frametitle{Branch and Bound}

  \small
  \begin{algorithm}[H]
    \DontPrintSemicolon

    \Function{\(\mathrm{BB}(b, s)\)}{
      \(Q \leftarrow \mathrm{Queue}\);
         \(\mathrm{enqueue}(Q, s)\) \;
      \(C \leftarrow \{ s \}\) \;
      \(B \leftarrow -\infty\) \;

      \While{\(\neg \mathrm{empty}(Q)\)}{
        \(s \leftarrow \mathrm{dequeue}(Q)\) \;
        \If{\(s \notin C \wedge b(s) \ge B\)}{
          \(C \leftarrow C \cup \{ s \}\);
          \If{\(g(s)\)}{
            \(B \leftarrow f(s)\) \;
            \(M \leftarrow s\) \;
          }
          \ForEach{\(v \in N(s)\)}{
            \If{\(v \notin C \wedge b(v) \ge B\)}{
              \(\mathrm{enqueue}(Q, v)\) \;
            }
          }
        }
      }
      \Return \(M\) \;
    }
  \end{algorithm}
\end{frame}

\section{Ejemplos}

\begin{frame}
  \setcounter{beamerpauses}{2}
  \frametitle{Problema \textsc{Knapsack}}

  \uncover<+->{
    El problema \textsc{Knapsack}
    (mochila),
    pone un conjunto de objetos de pesos positivos \(w_i\)
    con sus respectivos valores positivos \(v_i\),
    y una capacidad de la mochila \(C\).
    La idea es llenar la mochila
    (no sobrepasar su capacidad),
    logrando el valor máximo.
  }
  \\ \medskip
  \uncover<+->{
    En su momento demostramos que el problema de decisión respectivo
    (¿puede lograrse al menos el valor \(c\)?)
    es \NP\nobreakdash-completo,
    por lo que no esperamos un algoritmo eficiente.
  }
  \uncover<+->{
    Vale decir,
    es extremadamente poco probable
    que haya alguna manera \textquote{inteligente} de resolver esto,
    habrá que recurrir a una búsqueda exhaustiva
    revisando las distintas posibilidades.
  }
\end{frame}

\begin{frame}
  \setcounter{beamerpauses}{2}
  \frametitle{\textsc{Knapsack} continuo}

  \uncover<+->{
    Resulta que hay un problema similar que sí admite una solución eficiente%
  }%
  \uncover<+->{%
    : si permitimos fracciones de ítem,
    demostraremos más adelante que la solución óptima
    se obtiene considerando los objetos en orden de \(v_i / w_i\) descendiente,
    colocando la máxima fracción del ítem que cabe en la mochila.
  }
  \\ \medskip
  \uncover<+->{
    Relajamos la condición de ítem íntegro,
    podemos usar la solución a este problema como cota a nuestro problema
    (las soluciones necesariamente serán mejores).
  }
\end{frame}

\begin{frame}
  \setcounter{beamerpauses}{2}
  \frametitle{Estrategia para \textsc{Knapsack}}

  \uncover<+->{
    La estrategia para resolver \textsc{Knapsack} es entonces
    ir considerando los ítem uno a uno,
    viendo la opción de incluirlo o excluirlo.
    Usamos el valor de la mochila con los objetos ya elegidos
    sumado al resultado de \textsc{Knapsack} continuo para los aún no fijados
    como función \(b\).
  }
  \\ \medskip
  \uncover<+->{
    Un orden natural para ello es el orden de \(v_i / w_i\) decreciente
    impuesto por \textsc{Knapsack} continuo.
  }
  \uncover<+->{
    Tiene sentido intentar primero la opción de agregar el ítem \(i\)
    (si se puede),
    dado que intuitivamente eso debiera dar valores mayores de \(b\)
    (y finalmente de \(f\) cuando lleguemos a una decisión
     sobre todos los ítem),
    y vimos que una cota alta es deseable al maximizar.
  }
  \uncover<+->{
    (Podemos considerar lo anterior
     como usar una función heurística rudimentaria
     que guía la expansión.)
  }
\end{frame}

\begin{frame}
  \setcounter{beamerpauses}{2}
  \frametitle{Programa para \textsc{Knapsack}}
  \framesubtitle{Encabezados}

  \small
  \lstinputlisting[firstline = 1, lastline = 4]{knapsack.cc}
\end{frame}

\begin{frame}
  \setcounter{beamerpauses}{2}
  \frametitle{Programa para \textsc{Knapsack}}
  \framesubtitle{Variables globales --- el problema}

  \small
  \lstinputlisting[firstline = 6, lastline = 9]{knapsack.cc}
\end{frame}

\begin{frame}
  \setcounter{beamerpauses}{2}
  \frametitle{Programa para \textsc{Knapsack}}
  \framesubtitle{Variables globales --- posición actual}

  \small
  \lstinputlisting[firstline = 11, lastline = 18]{knapsack.cc}
\end{frame}

\begin{frame}
  \setcounter{beamerpauses}{2}
  \frametitle{Programa para \textsc{Knapsack}}
  \framesubtitle{Función cota}

  \small
  \lstinputlisting[firstline = 21, lastline = 33]{knapsack.cc}
\end{frame}

\begin{frame}
  \setcounter{beamerpauses}{2}
  \frametitle{Programa para \textsc{Knapsack}}
  \framesubtitle{Función \lstinline!bb(int)!, prototipo/fin}

  \small
  \lstinputlisting[linerange = {35-37}]{knapsack.cc}
  \hspace{4em}(Cuerpo de la función acá)
  \lstinputlisting[linerange = {53-53}]{knapsack.cc}
\end{frame}

\begin{frame}
  \setcounter{beamerpauses}{2}
  \frametitle{Programa para \textsc{Knapsack}}
  \framesubtitle{Función \lstinline!bb(int)!, fin (considerados todos)}

  \small
  \lstinputlisting[linerange = {37-44}]{knapsack.cc}
\end{frame}

\begin{frame}
  \setcounter{beamerpauses}{2}
  \frametitle{Programa para \textsc{Knapsack}}
  \framesubtitle{Función \lstinline!bb(int)!, poda}

  \small
  \lstinputlisting[linerange = {45-46}]{knapsack.cc}
\end{frame}

\begin{frame}
  \setcounter{beamerpauses}{2}
  \frametitle{Programa para \textsc{Knapsack}}
  \framesubtitle{Función \lstinline!bb(int)!, llamadas recursivas}

  \small
  \lstinputlisting[linerange = {47-52}]{knapsack.cc}
\end{frame}

\begin{frame}
  \setcounter{beamerpauses}{2}
  \frametitle{Programa para \textsc{Knapsack}}
  \framesubtitle{Programa principal}

  \small
  \lstinputlisting[linerange = {55-68}]{knapsack.cc}
\end{frame}

\section{Resumen}

\begin{frame}
  \setcounter{beamerpauses}{2}
  \frametitle{Resumen}

  \begin{itemize}
  \item
    Discutimos la técnica de búsqueda
    \emph{\foreignlanguage{english}{branch and bound}}.
  \item
    Mostramos una aplicación concreta,
    el problema \textsc{Knapsack}.
    \uncover<+->{
      La estrategia planteada es la base para algoritmos eficientes
      para resolver problemas de programación lineal entera.
    }
  \end{itemize}
\end{frame}
\end{document}

%%% Local Variables:
%%% mode: latex
%%% TeX-master: t
%%% ispell-local-dictionary: "spanish"
%%% End:

% LocalWords:  glyphtounicode mdseries Function function end Downto
% LocalWords:  Procedure procedure downto Loop loop Continue continue
% LocalWords:  Break break KwStep step english backtracking branch
% LocalWords:  and bound Knapsack handout Best First Search
