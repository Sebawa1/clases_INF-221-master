\input glyphtounicode%
\pdfgentounicode=1
\documentclass[english, spanish, fleqn,%
hyperref = {colorlinks, urlcolor = blue}%
%, handout%
]{beamer}

%% ]{article}
%% \usepackage{beamerarticle}

\usepackage{beamerthemesplit}

\usepackage{fourier}
\usepackage[normalem]{ulem}
\usepackage{amsmath}
\usepackage{amsthm}
\usepackage[es-noquoting]{babel}
\usepackage{csquotes}
\usepackage[noline, noend]{algorithm2e}
\usepackage{listings}

\ifdefined\HCode
  \def\pgfsysdriver{pgfsys-dvisvgm4ht.def}
\fi

\usefonttheme{professionalfonts}

% Overlay aware \sout
% Filched from TeX.SE 35767, update to answer by Roelof Spijker
% https://tex.stackexchange.com/a/35769/21275

\renewcommand<>{\sout}[1]{
  \only#2{\beameroriginal{\sout}{#1}}
  \invisible#2{#1}
}

%%%
%%% For listings
%%%

\lstloadlanguages{[ANSI]C, sh}

\beamerdefaultoverlayspecification{<+->}

\title{El programa rsync}

\author[Horst H. von Brand]{Horst H. von Brand\\
  \href{mailto:vonbrand@inf.utfsm.cl}{vonbrand@inf.utfsm.cl}}

\institute[DI UTFSM]{Departamento de Informática\\
                     Universidad Técnica Federico Santa María}
\date{}

\begin{document}
%%%
%%% For listings
%%%

\lstset{basicstyle = \sffamily, commentstyle = \slshape,
        frame = none, numbers = none,
        tabsize = 4,
        showstringspaces = false,
        xleftmargin = 1em, xrightmargin = 1em
       }
\renewcommand{\lstlistingname}{Listado}

%%%
%%% For algorithm2e
%%%

\SetAlgorithmName{Algoritmo}{Algoritmo}{Índice de algoritmos}

\SetAlCapSty{mdseries}
\SetKwProg{Function}{function}{}{end}
\SetKwProg{Procedure}{procedure}{}{end}
\SetKw{Variables}{variables}
\SetKw{Downto}{downto}
\SetKwBlock{Loop}{loop}{end}
\SetKw{Continue}{continue}
\SetKw{Break}{break}
\SetKw{KwStep}{step}

\frame{\maketitle}

\begin{frame}<handout>
  \frametitle{Contenido}

  \tableofcontents
\end{frame}

\section{Problema a resolver}

\begin{frame}
  \setcounter{beamerpauses}{2}
  \frametitle{Sincronizar archivos}

  \uncover<+->{
    En situaciones como administrar repositorios de software
    o respaldos,
    y también al mantener sincronizados
    computadores personales diversos,
    aparece el problema de copiar grandes archivos,
    que solo han cambiado un poco.
  }
  \uncover<+->{
    La solución de copiar todo es ineficiente,
    y era enfurecedora cuando los enlaces muy rápidos
    eran de unos pocos megabit por segundo.
  }
\end{frame}

\begin{frame}
  \setcounter{beamerpauses}{2}
  \frametitle{Solución}

  \uncover<1->{
    Se requiere algún método de comparar archivos remotos
    (para saber qué partes transferir)
    \emph{sin copiarlos completos}.
  }
  \\ \medskip
  \uncover<2->{
    \mode<beamer>{
      \sout<3->{
        \alert<2>{
          \textbf<2>{¡Imposible!}
        }
      }
    }
    \mode<handout>{
      Parece imposible\ldots
    }
  }
  \medskip
  \uncover<3->{
    Es lo que hace \lstinline[language = sh]!rsync(1)!.
  }
\end{frame}

\section{Huellas digitales}

\begin{frame}
  \setcounter{beamerpauses}{2}
  \frametitle{Buscar un patrón fijo en una palabra}

  \uncover<+->{
    El problema es buscar un patrón fijo
    (le llaman \emph{\foreignlanguage{english}{needle}}, aguja)
    en una palabra
    (\emph{\foreignlanguage{english}{haystack}}, pajar).
  }

  \uncover<+->{
    La idea obvia de ir comparando
    partiendo en cada posición
    es ineficiente.
  }

  \uncover<+->{
    Los algoritmos tradicionales más eficientes
    (\href{https://en.wikipedia.org/wiki/Knuth\%E2\%80\%93Morris\%E2\%80\%93Pratt_algorithm}
          {Knuth-Morris-Pratt},
     \href{https://en.wikipedia.org/wiki/Boyer\%E2\%80\%93Moore_string-search_algorithm}
          {Boyer-Moore})
    son bastante complicados.
  }
\end{frame}

\begin{frame}
  \setcounter{beamerpauses}{2}
  \frametitle{Algoritmo de Karp y Rabin}

  \uncover<+->{
    Un algoritmo aleatorizado muy simple es calcular una huella de la aguja,
    y luego calcular la huella partiendo en cada posición del pajar.
  }
  \uncover<+->{
    Suena muy ineficiente calcular una función \foreignlanguage{english}{hash}
    para cada posición,
    pero tenemos la libertad de elegir la función.
  }
\end{frame}

\begin{frame}
  \setcounter{beamerpauses}{2}
  \frametitle{\emph{\foreignlanguage{english}{Rolling hash}}}

  \uncover<+->{
    Hay funciones \foreignlanguage{english}{hash} fáciles de actualizar.
  }

  \uncover<+->{
    Sea una palabra \(\sigma = x_0 x_1 \ldots x_{k - 1}\) de largo \(k\)
    (los \(x_i\) son los caracteres)
    y definamos:
    \begin{equation*}
      h(\sigma)
        = \left( \sum_{0 \le i < k} a^i x_i \right) \bmod m
    \end{equation*}
  }
  \uncover<+->{
   Podemos actualizar \(h\) para mover el rango fácilmente.
   Si tenemos \(u = h(x_0 \ldots x_{k - 1})\),
    podemos calcular:
    \begin{equation*}
      h(x_1 \ldots x_k)
        = ((u - x_0) \cdot a^{-1} + x_k \cdot a^{k - 1}) \bmod m
     \end{equation*}
  }
  \uncover<+->{
    Solo si \(a\) es relativamente primo con \(m\)
    tiene inverso,
    claro está.
  }
\end{frame}

\begin{frame}
  \setcounter{beamerpauses}{2}
  \frametitle{\emph{\foreignlanguage{english}{Rolling hash}}}

  \uncover<+->{
    Consideraciones similares
    a las para parámetros de generadores de números aleatorios
    de la forma \(x_{n + 1} = (a x_n + c) \bmod m\)
    sugieren usar \(m\) un número primo grande
    (interesante resulta \(2^{31} - 1\),
     cerca del máximo entero que se puede representar en \(32\)~bits)
    y \(a\) una raíz primitiva módulo \(m\)
    (por suerte son bastante numerosas,
     una búsqueda al azar rápidamente da con una).
  }
  \uncover<+->{
    Otra opción interesante es usar \(m\) una potencia de \(2\)
    (la operación módulo es manipular bits,
     si \(m\) es el tamaño de palabra del computador
     simplemente ignorar rebalses),
    en cuyo caso debe ser \(a \equiv 3\) o \(a \equiv 5 \pmod{8}\).
  }
\end{frame}

\begin{frame}
  \setcounter{beamerpauses}{2}
  \frametitle{Algoritmo de Karp y Rabin}

  \uncover<+->{
    El algoritmo de Karp y Rabin usa la idea anterior
    (claro que la función que usan es diferente,
     en vez de calcular módulo \(m\)
     trabaja sobre \(\mathrm{GF}(2^k)\)).
  }
  \uncover<+->{
    Hay una probabilidad de falso positivo,
    para confirmar un hallazgo compara caracter a caracter.
  }
\end{frame}

\section{El programa \texttt{rsync}}

\begin{frame}
  \setcounter{beamerpauses}{2}
  \frametitle{El programa \texttt{rsync(1)}}

  \uncover<+->{
    El programa \lstinline[language = sh]!rsync(1)!
    usa la idea del algoritmo de Karp-Rabin:
    divide el archivo origen en bloques
    y calcula un \foreignlanguage{english}{hash}
    (usa una variante \emph{\foreignlanguage{english}{rolling hash}}
     del algoritmo Adler32)
    para cada uno de ellos,
    y los envía al destino.
    El destino hace lo propio,
    y considera sin modificar bloques
    con el mismo \foreignlanguage{english}{hash}.
  }
  \uncover<+->{
    En caso de que no coincida,
    considera correr el bloque
    (así maneja el caso común de inserciones en el origen)
    usando el \emph{\foreignlanguage{english}{rolling hash}},
    solo si tampoco encuentra un calce transfiere el nuevo bloque.
  }
\end{frame}

\begin{frame}
  \setcounter{beamerpauses}{2}
  \frametitle{Debilidades}

  \uncover<+->{
    Los \emph{\foreignlanguage{english}{rolling hash}} discutidos
    son susceptibles de ataques
    (colisiones).
  }

  \uncover<+->{
    Mitigación en el caso de Karp-Rabin es elegir los parámetros al azar,
    aunque no es demasiado relevante
    (confirmar un posible calce es simple).
  }

  \uncover<+->{
    El programa \lstinline[language = sh]!rsync(1)!
    usa un \foreignlanguage{english}{hash} fijo,
    puede engañarse fácilmente.
  }
  \uncover<+->{
    Soluciones serían usar una función elegida al azar,
    acordada al comienzo de la transacción;
    incluso usar varias funciones independientes en paralelo.
  }
\end{frame}

\section{Resumen}

\begin{frame}
  \setcounter{beamerpauses}{2}
  \frametitle{Resumen}

  \begin{itemize}
  \item
    Vimos dos aplicaciones de la idea de huellas digitales:
    El algoritmo de Karp-Rabin
    y el popular \lstinline[language = sh]!rsync(1)!.
  \end{itemize}

\end{frame}

\end{document}

%%% Local Variables:
%%% mode: latex
%%% TeX-master: t
%%% ispell-local-dictionary: "spanish"
%%% End:

% LocalWords:  glyphtounicode mdseries Function function end Downto
% LocalWords:  Procedure procedure downto Loop loop Continue continue
% LocalWords:  Break break KwStep step enfurecedora megabit english
% LocalWords:  needle haystack aleatorizado hash Rolling caracter
% LocalWords:  rsync rolling beamer handout
