\input glyphtounicode%
\pdfgentounicode=1
\documentclass[english, spanish, fleqn,%
hyperref = {colorlinks, urlcolor = blue}%
%, handout%
]{beamer}

%% ]{article}
%% \usepackage{beamerarticle}

\usepackage{beamerthemesplit}

\usepackage[es-noquoting]{babel}
\usepackage{csquotes}
\usepackage[group-digits = false, copy-decimal-marker = true]{siunitx}
\usepackage{fourier}
\usepackage{amsmath}
\usepackage{amsthm}
\usepackage{dcolumn}
\usepackage[noline, noend]{algorithm2e}
\usepackage{listings}

\usefonttheme{professionalfonts}

%%%
%%% For listings
%%%

\lstloadlanguages{Python}

\beamerdefaultoverlayspecification{<+->}

\title{\foreignlanguage{english}{Quicksort}}

\author[Horst H. von Brand]{Horst H. von Brand\\
  \href{mailto:vonbrand@inf.utfsm.cl}{vonbrand@inf.utfsm.cl}}

\institute[DI UTFSM]{Departamento de Informática\\
                     Universidad Técnica Federico Santa María}
\date{}

\begin{document}
%%%
%%% For listings
%%%

\lstset{basicstyle = \sffamily, commentstyle = \slshape,
        frame = none, numbers = none,
        tabsize = 4,
        showstringspaces = false,
        xleftmargin = 1em, xrightmargin = 1em
       }
\renewcommand{\lstlistingname}{Listado}

%%%
%%% For algorithm2e
%%%

\SetAlgorithmName{Algoritmo}{Algoritmo}{Índice de algoritmos}

\SetAlCapSty{mdseries}
\SetKwProg{Function}{function}{}{end}
\SetKwProg{Procedure}{procedure}{}{end}
\SetKw{Variables}{variables}
\SetKw{Downto}{downto}
\SetKwBlock{Loop}{loop}{end}
\SetKw{Continue}{continue}
\SetKw{Break}{break}
\SetKw{KwStep}{step}

\frame{\maketitle}

\begin{frame}<handout>
  \frametitle{Contenido}

  \tableofcontents
\end{frame}

\section{Idea básica}

\begin{frame}
  \setcounter{beamerpauses}{2}
  \frametitle{La idea básica}

  \uncover<+->{
    No siempre tenemos control sobre la división.
    El ejemplo clásico
    es \emph{\foreignlanguage{english}{\foreignlanguage{english}{Quicksort}}}.
  }

  \uncover<+->{
    La idea
    de \emph{\foreignlanguage{english}{\foreignlanguage{english}{Quicksort}}}
    es simple:
  }
  \begin{itemize}
  \item
    Elija un elemento \emph{pivote} entre los elementos a ordenar
  \item
    Reorganice de forma que los elementos menores al pivote
    estén antes de los mayores al pivote.
  \item
    Ordene recursivamente
    los tramos antes y después de la posición
    que separa los elementos \textquote{chicos} de los \textquote{grandes}.
  \end{itemize}
\end{frame}

\begin{frame}
  \setcounter{beamerpauses}{2}
  \frametitle{La idea básica}

  \uncover<+->{
    Es crítico que al particionar no terminemos
    con alguno de los subtramos el tramo original completo.
    Eso se evita al elegir el pivote entre los elementos del rango.
  }
  \\ \medskip
  \uncover<+->{
    Hay varias formas de efectuar la crítica operación de particionar.
    En todo caso,
    es claro ver que se puede efectuar en tiempo \(\Theta(n)\).
  }
\end{frame}

\begin{frame}
  \setcounter{beamerpauses}{2}
  \frametitle{La idea básica}

  \uncover<+->{
    Si la división fuera siempre en mitades,
    la recurrencia para el número de operaciones sería aproximadamente:
    \begin{equation*}
      T(n) = 2 T(n / 2) + n
    \end{equation*}
    y el teorema maestro da \(T(n) = \Theta(n \log n)\).
  }
  \uncover<+->{
    Si la división es en fracciones fijas,
    digamos \(\alpha\) y \(1 - \alpha\),
    es:
    \begin{equation*}
      T(n) = T(\alpha n) + T((1 - \alpha) n) + n
    \end{equation*}
    El teorema de Akra-Bazzi tiene \(p = 1\) con lo que también:
    \begin{equation*}
      T(n)
        = \Theta\left(
                  z \int_1^n \frac{u \, \mathrm{d} u}{u^2}
                \right)
        = \Theta(n \log n)
    \end{equation*}
  }
\end{frame}

\begin{frame}
  \setcounter{beamerpauses}{2}
  \frametitle{La idea básica}

  \uncover<+->{
    Claro que si la división
    es en un número fijo de elementos \(k\) en el primer tramo
    (con lo que es condición de contorno el valor \(T(k)\)):
    \begin{equation*}
      T(n)
        = T(k) + T(n - k) + n
    \end{equation*}
  }
  \uncover<+->{
    Solución es
    (suponiendo \(n\) múltiplo de \(k\)):
    \begin{equation*}
      T(n)
        = \frac{(n / k - 1) (n + 2 n / k + 2 T(k))}{2} + T(k)
        = \Theta(n^2 / k)
    \end{equation*}
  }
  \uncover<+->{
    Harto preocupante,
    el peor caso es \(O(n^2)\).
  }
\end{frame}

\begin{frame}
  \setcounter{beamerpauses}{2}
  \frametitle{La idea básica}

  \uncover<+->{
    Analizaremos el caso promedio,
    más adelante demostraremos que los peores casos
    son extraordinariamente poco probables
    \emph{siempre que los datos vengan ordenados al azar}.
  }
  \uncover<+->{
    Plantearemos algunas variantes,
    más eficientes
    o resistentes a ataques que fuercen el peor caso.
  }
  \\ \medskip
  \uncover<+->{
    La mayor parte del trabajo se hace en el proceso de particionar,
    hacerlo eficiente es crucial.
  }
\end{frame}

\section{Desarrollo de un programa}

\begin{frame}
  \setcounter{beamerpauses}{2}
  \frametitle{Desarrollo de un programa}

  \uncover<+->{
    Debemos ordenar el tramo \lstinline!A[m] .. A[n - 1]! del arreglo.
  }
  \\ \medskip
  \uncover<+->{
    Si \(m + 1 \ge n\) no hay nada que hacer.
  }

  \uncover<+->{
    Si \(m + 1 < n\),
    reorganizamos con un \(p\) por definir tal que
    \lstinline!A[m .. k - 1] <= p! y \lstinline!p < A[k .. n - 1]!
  }
\end{frame}

\begin{frame}[fragile]
  \frametitle{Desarrollo de un programa}
  \framesubtitle{Particionar}

  Una manera de particionar es el código siguiente,
  debido a Lomuto:
  \small
  \begin{lstlisting}[gobble = 4]
    p = A[n - 1];
    j = m;
    for(i = m; i < n - 1; i++)
        if(A[i]) <= p) {
            swap(A[i], A[j]); j++;
        }
    swap(A[j], A[n - 1]);
  \end{lstlisting}
\end{frame}

\begin{frame}[fragile]
  \frametitle{Desarrollo de un programa}
  \framesubtitle{Particionar}

  Mucho más eficiente es el esquema de Hoare:
  \small
  \begin{lstlisting}[gobble = 4]
    p = A[(m + n) / 2];
    int i = m - 1, j = n;
    for(;;) {
        while(A[++i] < p)
            ;
        while(p < A[--j])
            ;
        if(i >= j) break;
        swap(A[i], A[j]);
    }
  \end{lstlisting}
\end{frame}

\begin{frame}
  \setcounter{beamerpauses}{2}
  \frametitle{Desarrollo de un programa}
  \framesubtitle{Particionar}

  \uncover<+->{
    Este código es corto,
    pero lejos de trivial.
  }
  \uncover<+->{
    Aumenta \lstinline!i! pasando elementos \textquote{chicos}
    parando con elementos mayores o iguales,
    o disminuye \lstinline!j! pasando elementos \textquote{grandes},
    para al encontrar un elemento menor o igual.
    Inicialmente el pivote está entre \lstinline!i! y \lstinline!j!,
    esto sirve para detener los ciclos
    sin tener que verificar específicamente no salir del rango.
  }
  \uncover<+->{
    Sabemos que al final \lstinline!A[m .. j] <= p!
    y \lstinline!A[j + 1 .. n - 1] >= p!.
  }
  \uncover<+->{
    El pivote queda en \lstinline!A[j]!.
  }

  \uncover<+->{
    Solo intercambiamos elementos,
    el resultado es una permutación de los datos originales.
  }
\end{frame}

\begin{frame}[fragile]
  \frametitle{Desarrollo de un programa}
  \framesubtitle{Primer esbozo}

  \small
  \begin{lstlisting}[gobble = 4]
    int quicksort(type A[], int m, int n)
    {
        if(m + 1 < n) {
            type p = A[(m + n) / 2];
            int i = m - 1, j = n;
            for(;;) {
                while(A[++i] < p);
                while(p < A[--j]);
                if(i >= j) break;
                swap(A[i], A[j]);
            }
            quicksort(A, m, j + 1);
            quicksort(A, j + 1, n);
        }
    }
  \end{lstlisting}
\end{frame}

\begin{frame}
  \setcounter{beamerpauses}{2}
  \frametitle{Desarrollo de un programa}
  \framesubtitle{Particionar}

  \uncover<+->{
    Si hay elementos iguales al pivote,
    debe tenerse cuidado que no queden todos en el mismo tramo.
    Si fueran todos los elementos iguales,
    caemos en el peor caso.
  }
  \uncover<+->{
    En un computador determinista no es fácil dividirlos mitad y mitad.
    Pero podemos simplemente dejarlos fuera,
    como lo hace el particionamiento de Hoare.
  }
\end{frame}

\begin{frame}
  \setcounter{beamerpauses}{2}
  \frametitle{Desarrollo de un programa}
  \framesubtitle{Particionar}

  \uncover<+->{
    Es común que los datos vengan \textquote{casi ordenados}
    (o sea,
     los elementos generalmente no están lejos de sus posiciones en orden).
    Garantía de peor caso al elegir el primero
    (o el último)
    de pivote,
    nuevamente.
  }

  \uncover<+->{
    La técnica de particionamiento de Hoare elige como pivote
    un elemento del centro.
    Otra opción es elegir el pivote al azar en el rango.
    Esto también evita ataques que usan la elección del pivote
    para forzar el peor caso.
  }

  \uncover<+->{
    Otra posibilidad
    (que disminuye además la probabilidad del peor caso)
    es tomar una muestra
    (por ejemplo,
     primer y último elementos,
     y el central)
    y usar la mediana de la muestra como pivote.
  }
\end{frame}

\begin{frame}
  \setcounter{beamerpauses}{2}
  \frametitle{Desarrollo de un programa}
  \framesubtitle{Particionar}

  \uncover<+->{
    Otra opción es dividir en tres tramos:
    menores al pivote,
    iguales al pivote
    y mayores al pivote.
  }
  \uncover<+->{
    Se conoce como el problema de la bandera holandesa
    (en inglés,
     \emph{\foreignlanguage{english}{Dutch national flag problem}})
    por la bandera de franjas de tres colores,
    bautizado así por Dijkstra.
  }
\end{frame}

\begin{frame}[fragile]
  \frametitle{Desarrollo de un programa}
  \framesubtitle{Particionar}

  Lo siguiente mantiene \lstinline!A[m .. i - 1] < p!,
  \lstinline!A[i .. j - 1] == p!
  y \lstinline!A[k .. n - 1] > p!;
  \lstinline!A[j .. k - 1]! indecisos.
  \small
  \begin{lstlisting}[gobble = 4]
     int i = m, j = m, k = n;
     while(j < k) {
         if(A[j] < p) {
             swap(A[i], A[j]); i++; j++;
         }
         else if(A[j] > p) {
             k--; swap(A[j], A[k]);
         }
         else
            j++;
         }
     }
  \end{lstlisting}
\end{frame}

\begin{frame}
  \setcounter{beamerpauses}{2}
  \frametitle{Desarrollo de un programa}
  \framesubtitle{Evitar recursión}

  \uncover<+->{
    \foreignlanguage{english}{Quicksort} hace dos llamadas recursivas.
  }
  \uncover<+->{
    Podemos reemplazar la última
    (llamada de cola)
    por ajustar argumentos y volver al comienzo.
  }
  \uncover<+->{
    Note que compíladores competentes hacen esta modificación automáticamente,
    verifique antes de hacerla a mano.
  }
  \uncover<+->{
    Si arreglamos de forma de hacer la llamada recursiva para el tramo menor,
    puede haber a lo más \(\log_2 n\) llamadas recursivas pendientes.
  }
\end{frame}

\begin{frame}[fragile]
  \frametitle{Desarrollo de un programa}
  \framesubtitle{Tercer esbozo}

  \small
  \begin{lstlisting}[gobble = 4]
    int quicksort(type A[], int m, int n)
    {
        while(m + 1 < n) {
            /* Partition A[m .. j; j + 1 .. n - 1] */
            if(j - m > n - j) {
                quicksort(A, j + 1, n); n = j + 1;
            }
            else {
                quicksort(A, m, j + 1); m = j + 1;
            }
        }
    }
  \end{lstlisting}
\end{frame}

\begin{frame}
  \setcounter{beamerpauses}{2}
  \frametitle{Desarrollo de un programa}
  \framesubtitle{Evitar recursión}

  \uncover<+->{
    Podemos eliminar completamente las llamadas recursivas
    de la última versión
    manejando una pila de rangos directamente.
  }
  \uncover<+->{
    Manejar una pila de pares de índices
    definitivamente será más eficiente
    que una llamada y retorno a función.
  }
\end{frame}

\begin{frame}
  \setcounter{beamerpauses}{2}
  \frametitle{Desarrollo de un programa}
  \framesubtitle{Otras consideraciones}

  \uncover<+->{
    \foreignlanguage{english}{Quicksort}
    es muy eficiente para manejar arreglos grandes,
    pero ineficiente para arreglos pequeños.
  }
  \uncover<+->{
    Una opción es cambiar a otro método cuando el tramo es corto.
  }
  \uncover<+->{
    Resulta que ordenamiento por inserción hace trabajo proporcional
    a la suma de las distancias de los elementos de sus posiciones finales,
    con lo que una opción es simplemente cortar la recursión
    con tramos cortos
    y completar con inserción sobre el total.
  }
\end{frame}

\begin{frame}
  \setcounter{beamerpauses}{2}
  \frametitle{Desarrollo de un programa}
  \framesubtitle{Otras consideraciones}

  \uncover<+->{
    Para evitar el peor caso,
    puede monitorearse la profundidad de la recursión.
    Si se hace demasiado grande,
    abortar y usar un método más lento,
    pero garantizadamente \(O(n \log n)\).
  }
  \uncover<+->{
    Es la idea de \emph{\foreignlanguage{english}{Introsort}},
    que cambia a \emph{\foreignlanguage{english}{Heapsort}}.
  }
\end{frame}

\section{Análisis del algoritmo}

\begin{frame}
  \setcounter{beamerpauses}{2}
  \frametitle{Análisis del algoritmo básico}

  \uncover<+->{
    Solo analizaremos el algoritmo básico en detalle,
    dando algunas pistas sobre las variantes.
  }
  \uncover<+->{
    Como medida principal usaremos el número de comparaciones.
  }
\end{frame}

\subsection{Número de comparaciones}

\begin{frame}
  \setcounter{beamerpauses}{2}
  \frametitle{Número de comparaciones}

  \uncover<+->{
    Si \(C(n)\) es el número de comparaciones
    que hace \foreignlanguage{english}{Quicksort}
    al ordenar un arreglo de \(n\)~elementos,
    vemos que particionar efectúa \(n - 1\) comparaciones,
    y luego ordena recursivamente
    tramos de \(k\) y \(n - 1 - k\) elementos para algún \(0 \le k \le n - 1\)
    (excluimos el pivote,
     en su posición final).
  }
  \uncover<+->{
    Si todos los elementos son diferentes
    y el pivote se elige al azar,
    cada división es igualmente probable:
    \begin{equation*}
      C(n)
        = \frac{1}{n - 1} \sum_{0 \le k \le n - 1} (C(k) + C(n - 1 - k))
             + n - 1
      \qquad
      C(0)
        = 0
    \end{equation*}
  }
\end{frame}

\begin{frame}
  \setcounter{beamerpauses}{2}
  \frametitle{Número de comparaciones}

  \uncover<+->{
    Ajustando índices,
    notando que estamos sumando los \(C(k)\)
    en orden creciente y decreciente:
    \begin{equation*}
      C(n + 1)
        = \frac{2}{n + 1} \sum_{0 \le k \le n} C(k) + n
      \qquad
      C(0)
        = 0
    \end{equation*}
  }
  \uncover<+->{
    Reorganizando:
    \begin{equation*}
      (n + 1) C(n + 1)
        = 2 \sum_{0 \le k \le n} C(k) + n (n + 1)
      \qquad
      C(0)
        = 0
    \end{equation*}
  }
\end{frame}

\begin{frame}
  \setcounter{beamerpauses}{2}
  \frametitle{Número de comparaciones}

  \uncover<+->{
    Definimos la función generatriz:
    \begin{equation*}
      g(z)
        = \sum_{n \ge 0} C(n) z^n
    \end{equation*}
    Aplicando las propiedades respectivas a la recurrencia,
    usando su condición inicial:
    \begin{equation*}
      z \frac{\mathrm{d}}{\mathrm{d} z} \frac{g(z)}{z} + \frac{g(z)}{z}
        = \frac{2 g(z)}{1 - z}	+ \frac{2 z}{(1 - z)^3}
    \end{equation*}
  }
\end{frame}

\begin{frame}
  \setcounter{beamerpauses}{2}
  \frametitle{Número de comparaciones}

  \uncover<+->{
    Solución es:
    \begin{equation*}
      g(z)
        = - 2 \frac{\ln (1 - z) + z}{(1 - z)^2}
    \end{equation*}
  }
  \uncover<+->{
    Obtenemos:
    \begin{align*}
      C(n)
        &=    [z^n] g(z) \\
        &=    2 (n + 1) H_n - 4 n \\
        &\sim 2 n \ln n
    \end{align*}
  }
\end{frame}

\begin{frame}
  \setcounter{beamerpauses}{2}
  \frametitle{Contraste con el número mínimo de comparaciones}

  \uncover<+->{
    Sabemos que para ordenar \(n\) elementos
    se requieren al menos \(\lceil \log_2 n! \rceil\) comparaciones.
  }

  \uncover<+->{
    Tenemos la fórmula de Stirling para el factorial:
    \begin{align*}
      n!
        &\sim \sqrt{2 \pi n} \left( \frac{n}{\mathrm{e}} \right)^n \\
      \log_2 n!
        &\sim \log_2 \sqrt{2 \pi}
                + \frac{1}{2} \log_2 n
                + n \log_2 n
                - n \log_2 \mathrm{e} \\
        &\sim n \log_2 n
    \end{align*}
  }
\end{frame}

\begin{frame}
  \setcounter{beamerpauses}{2}
  \frametitle{Contraste con el número mínimo de comparaciones}

  \uncover<+->{
    Resulta para el número mínimo de comparaciones \(M(n)\):
    \begin{align*}
      M(n)
        &\sim \log_2 n! \\
        &\sim n \log_2 n \\
        &= \frac{1}{\ln 2} n \ln n \\
        &= \num{1,4430} \, n \ln n
    \end{align*}
  }
  \uncover<+->{
    \foreignlanguage{english}{Quicksort} en promedio no hace mucho más de las comparaciones requeridas.
  }
\end{frame}

\subsection{Número de intercambios}

\begin{frame}
  \setcounter{beamerpauses}{2}
  \frametitle{Número de intercambios}

  \uncover<+->{
    Para completar el análisis detallado del algoritmo,
    hace falta contabilizar el número promedio de intercambios
    en el proceso de particionamiento.
  }
  \uncover<+->{
    En esta medida difieren las técnicas mostradas.
  }
  \\ \medskip
  \uncover<+->{
    En el proceso solo es relevante el orden relativo de los elementos,
    si son todos diferentes
    podemos suponer que son simplemente los números de \(1\) a~\(n\).
  }
  \\ \smallskip
  \uncover<+->{
    Para hacer tratable el análisis,
    supondremos que las \(n!\) permutaciones de los elementos
    son igualmente probables.
  }
  \uncover<+->{
    Una propiedad crucial de las técnicas de particionamiento planteadas
    es que no reordenan los tramos resultantes:
    si el tramo original está ordenado al azar,
    lo están los tramos resultantes.
  }
\end{frame}

\begin{frame}
  \setcounter{beamerpauses}{2}
  \frametitle{Intercambios en particionamiento de Lomuto}

  \uncover<+->{
    El particionamiento de Lomuto revisa el arreglo completo
    e intercambia cada vez que encuentra un elemento mayor que el pivote.
  }
  \uncover<+->{
    Si el pivote es \(p\),
    efectúa \(p - 1\) intercambios.
    En promedio,
    suponiendo que el pivote se elige al azar:
    \begin{equation*}
      \frac{1}{n} \sum_{1 \le p \le n} (p - 1)
        = \frac{n - 1}{2}
    \end{equation*}
  }
\end{frame}

\begin{frame}
  \setcounter{beamerpauses}{2}
  \frametitle{Intercambios en particionamiento de Hoare}

  \uncover<+->{
    Particionamiento de Hoare
    busca elementos mayores que el pivote desde la izquierda
    y elementos menores desde la derecha,
    para intercambiarlos.
  }
  \uncover<+->{
    Que siempre encuentra un par a intercambiar es parte de la demostración
    de correctitud.
  }
\end{frame}

\begin{frame}
  \setcounter{beamerpauses}{2}
  \frametitle{Intercambios en particionamiento de Hoare}

  \uncover<+->{
    El análisis en este caso es algo más enredado.
    Si el pivote elegido es \(p\),
    hay \(n - p\) elementos mayores en el arreglo.
    De ellos elegimos los que caen antes del pivote,
    en \(p - 1\)~posiciones.
    En promedio,
    el número de elementos mayores que el pivote antes de la posición \(p\)
    (el número de pares que se intercambian)
    es:
    \begin{equation*}
      \frac{(n - p) (p - 1)}{n - 1}
    \end{equation*}
    Promediando sobre las posiciones del pivote:
    \begin{equation*}
      \frac{1}{n} \sum_{1 \le p \le n} \frac{(n - p) (p - 1)}{n - 1}
        = \frac{n - 2}{6}
    \end{equation*}
  }
\end{frame}

\section{Variantes}

\begin{frame}
  \setcounter{beamerpauses}{2}
  \frametitle{Análisis del algoritmo mediana de tres}

  \uncover<+->{
    En el algoritmo básico la posición final del pivote
    es una cualquiera en el rango \(1 .. n\).
  }
  \uncover<+->{
    Si tomamos una muestra de \(3\) elementos
    y elegimos la mediana de pivote,
    el pivote está en la posición \(r + 1\) en el arreglo ordenado
    es que elegimos uno entre los \(n - r - 1\)~mayores al pivote
    y uno entre los \(r\)~menores.
    En total,
    la probabilidad es:
    \begin{align*}
      \frac{r (n - r - 1)}{\binom{n}{3}}
    \end{align*}
  }
\end{frame}

\begin{frame}
  \setcounter{beamerpauses}{2}
  \frametitle{Análisis del algoritmo mediana de tres}

  \uncover<+->{
    La recurrencia es ahora,
    después de reorganizar como antes
    (son \(3\) comparaciones para hallar la mediana,
     después de ordenar esos tres elementos quedan \(n - 2\) a clasificar):
    \begin{equation*}
      \binom{n}{3} C(n + 1)
        = 2 \sum_{0 \le k \le n} (r - 1) (n - r) C(r)
            + (n + 3 - 2) \binom{n}{3}
      \qquad
      C(3)
        = 3
    \end{equation*}
  }
\end{frame}

\begin{frame}
  \setcounter{beamerpauses}{2}
  \frametitle{Análisis de las variantes}

  \uncover<+->{
    La recurrencia para mediana de tres
    puede resolverse con técnicas similares a las anteriores.
  }
  \uncover<+->{
    Resulta que se gana aproximadamente un \SI{6}{\percent} en rendimiento.
  }
  \\ \medskip
  \uncover<+->{
    Poniendo como valores iniciales hasta \(C[M]\) el costo respectivo
    del método simple para tramos cortos,
    se obtiene una recurrencia que permite calcular el valor de \(M\)
    óptimo.
  }
  \uncover<+->{
    El mínimo depende de los detalles exactos,
    pero se mueve alrededor de \num{10}.
  }
\end{frame}

\section{Resumen}

\begin{frame}
  \setcounter{beamerpauses}{2}
  \frametitle{Resumen}

  \begin{itemize}
  \item
    \foreignlanguage{english}{Quicksort} es un algoritmo muy importante.
  \item
    La idea general es simple,
    pero hay muchas variantes a contrastar.
  \item
    El algoritmo es frágil,
    debe tenerse mucho cuidado al escribir su propia versión.
  \item
    Sorprendentemente,
    es posible analizar el algoritmo en forma muy detallada.
  \end{itemize}
\end{frame}
\end{document}

%%% Local Variables:
%%% mode: latex
%%% TeX-master: t
%%% ispell-local-dictionary: "spanish"
%%% End:

% LocalWords:  glyphtounicode mdseries Function function end Downto
% LocalWords:  Procedure procedure downto Loop loop Continue continue
% LocalWords:  Break break KwStep step english Quicksort particionar
% LocalWords:  subtramos particionamiento Dutch national flag problem
% LocalWords:  monitorearse garantizadamente Introsort Heapsort
% LocalWords:  correctitud
