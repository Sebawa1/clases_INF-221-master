\input glyphtounicode%
\pdfgentounicode=1
\documentclass[english, spanish, fleqn,%
hyperref = {colorlinks, urlcolor = blue}%
%, handout%
]{beamer}

%% ]{article}
%% \usepackage{beamerarticle}

\usepackage{beamerthemesplit}

\usepackage[es-noquoting]{babel}
\usepackage{csquotes}
\usepackage{fourier}
\usepackage{amsmath}
\usepackage{amsthm}
\usepackage[noline, noend]{algorithm2e}
\usepackage{listings}

\usefonttheme{professionalfonts}

%%%
%%% For listings
%%%

\lstloadlanguages{Python}


\beamerdefaultoverlayspecification{<+->}

\title{Búsqueda inteligente en grafos: \(A^*\)}

\author[Horst H. von Brand]{Horst H. von Brand\\
  \href{mailto:vonbrand@inf.utfsm.cl}{vonbrand@inf.utfsm.cl}}

\institute[DI UTFSM]{Departamento de Informática\\
                     Universidad Técnica Federico Santa María}
\date{}

\begin{document}
%%%
%%% For listings
%%%

\lstset{basicstyle = \sffamily, commentstyle = \slshape,
        frame = none, numbers = none,
        tabsize = 4,
        showstringspaces = false,
        xleftmargin = 1em, xrightmargin = 1em
       }
\renewcommand{\lstlistingname}{Listado}

%%%
%%% For algorithm2e
%%%

\SetAlgorithmName{Algoritmo}{Algoritmo}{Índice de algoritmos}

\SetAlCapSty{mdseries}
\SetKwProg{Function}{function}{}{end}
\SetKwProg{Procedure}{procedure}{}{end}
\SetKw{Variables}{variables}
\SetKw{Downto}{downto}
\SetKwBlock{Loop}{loop}{end}
\SetKw{Continue}{continue}
\SetKw{Break}{break}
\SetKw{KwStep}{step}

\frame{\maketitle}

\begin{frame}<handout>
  \frametitle{Contenido}

  \tableofcontents
\end{frame}

\section{Situación general}

\begin{frame}
  \setcounter{beamerpauses}{2}
  \frametitle{El problema}

  \uncover<+->{
    Es común buscar el camino más barato de un nodo origen en un grafo
    a una meta.
    Muchas veces no conocemos el destino,
    pero tenemos un criterio que nos dice si un nodo es meta.
  }
  \\ \medskip
  \uncover<+->{
    Una opción es usar el algoritmo de Dijkstra,
    pero al entregar ese los caminos más cortos a todos los nodos,
    puede trabajar mucho en caminos que van en la dirección equivocada.
  }
  \\ \medskip
  \uncover<+->{
    Si contamos con alguna estimación de la distancia
    de un nodo a una meta,
    podemos usar ésta como guía
    para decidir qué nodo expandir a continuación.
  }
  \uncover<+->{
    Debemos incluir el costo del mejor camino conocido al nodo actual
    en la estimación,
    nos interesa el costo mínimo del camino completo.
  }
\end{frame}

\begin{frame}
  \setcounter{beamerpauses}{2}
  \frametitle{Consideraciones}

  \uncover<+->{
    Suponemos un grafo dirigido \(G = (V, E)\),
    una función de costo \(c \colon E \to \mathbb{R}\),
    donde siempre es \(c(e) \ge \delta > 0\)
    (esto evita caminos de largo infinito con costo finito),
    un conjunto de \emph{fuentes} \(S \subset V\),
    un conjunto de \emph{metas} \(T \subset V\)
    y un operador \emph{sucesor} \(\Gamma \colon V \to 2^V\).
    No suponemos que \(G\) es finito,
    suponemos eso sí que el número de nodos vecinos (descendientes)
    es siempre finito
    (es \emph{localmente finito}).
  }
  \uncover<+->{
    El subgrafo \(G_v\) es el nodo \(v\) junto con todos sus descendientes.
    Dado un nodo fuente \(s \in S\)
    nos interesa hallar en \(G_s\) el nodo \(t \in T\)
    que minimiza el costo del camino
    (la suma de los costos de los arcos)
    de \(s\) a \(t\).
  }
  \uncover<+->{
    Al costo mínimo de un camino de \(u\) a \(v\) lo anotaremos \(h(u, v)\),
    para abreviar escribiremos \(h(v)\) para \(\min_{t \in T} \{ h(v, t) \}\)
    (\(h(v)\) es el costo del camino óptimo desde \(v\) a un destino).
  }
\end{frame}

\begin{frame}
  \setcounter{beamerpauses}{2}
  \frametitle{Consideraciones}

  \uncover<+->{
    Un algoritmo es \emph{admisible}
    si garantiza hallar un camino óptimo de \(s\) a una meta
    para todo grafo.
  }
  \uncover<+->{
    Algoritmos admisibles podrán expandir diferentes nodos,
    o hacerlo en distinto orden.
  }
  \uncover<+->{
    Interesa que el algoritmo expanda el mínimo número de nodos.
    Expandir nodos que se sabe que no pueden estar en un camino óptimo
    es desperdiciar esfuerzo,
    ignorar nodos que están en un camino óptimo
    puede hacer que no lo encuentre y no ser admisible.
  }
  \uncover<+->{
    Interesan algoritmos admisibles y eficientes.
  }
\end{frame}

\begin{frame}
  \setcounter{beamerpauses}{2}
  \frametitle{Consideraciones}

  \uncover<+->{
    Supondremos una \emph{función de evaluación}
    \(\widehat{f} \colon V \to \mathbb{R}\),
    de manera de expandir a continuación
    el nodo de mínimo valor de \(\widehat{f}\).
  }
  \uncover<+->{
    Esto sugiere el algoritmo \(A^*\)
    que contempla una cola de prioridad \(Q\).
    Diremos que nodos en \(Q\)
    (al igual que nodos aún no considerados)
    están \emph{abiertos},
    y marcaremos ciertos nodos como \emph{cerrados}
    para no considerarlos nuevamente.
  }
\end{frame}

\section{El algoritmo \(A^*\)}

\begin{frame}
  \frametitle{El algoritmo \(A^*\)}

  \begin{algorithm}[H]
    \DontPrintSemicolon

    \Function{\(A^* (G, s, T)\)}{
      \(\mathrm{Insert}(Q, s, \widehat{f}(s))\) \;
      \While{\(\neg \mathrm{empty}(Q)\)}{
        \(v \leftarrow \mathrm{DeleteMin}(Q)\) \;
        Mark \(v\) closed \;
        \If{\(v \in T\)}{
          \Return \(v\) \;
        }
        \ForEach{\(u \in \Gamma(v)\)}{
          \uIf{\(u \text{\ is not closed}\)}{
            \(\mathrm{Insert}(Q, u, \widehat{f}(u))\) \;
          }
          \ElseIf{\(\text{new \(\widehat{f}(u)\) less than old value}\)}{
            Remove closed mark from \(u\) \;
            \(\mathrm{Insert}(Q, u, \widehat{f}(u))\) \;
          }
        }
      }
    }
  \end{algorithm}
\end{frame}

\begin{frame}
  \setcounter{beamerpauses}{2}
  \frametitle{El algoritmo \(A^*\)}

  \uncover<+->{
    En realidad nos interesa el camino de \(s\) a \(T\),
    la modificación obvia
    es registrar el nodo padre de \(v\) cuando lo marcamos cerrado
    (y corregirlo al volverlo a cerrar),
    finalmente seguimos la lista desde el nodo meta alcanzado hacia atrás
    para reconstruir el camino buscado.
  }
\end{frame}

\begin{frame}
  \setcounter{beamerpauses}{2}
  \frametitle{El algoritmo \(A^*\)}

  \uncover<+->{
    El algoritmo
    supone una cola de prioridad que permite modificar prioridades.
    La mayoría de las versiones disponibles en bibliotecas
    supone prioridades inmutables.
  }
  \uncover<+->{
    En el caso que el algoritmo solicita modificar prioridades,
    podemos simplemente insertar el nodo con la nueva prioridad.
    Cuando nuevamente encontremos el nodo,
    será con prioridad más alta y lo descartaremos.
  }
  \uncover<+->{
    Ensucia la cola con datos inútiles,
    pero parece no tener un impacto demasiado alto en el rendimiento,
    como muestran experimentos.
  }
\end{frame}

\subsection{La función de evaluación}

\begin{frame}
  \setcounter{beamerpauses}{2}
  \frametitle{La función de evaluación}

  \uncover<+->{
    Para el subgrafo \(G_s\) sea \(f(v)\) el costo óptimo
    de un camino de \(s\) a \(T\),
    con la restricción que el camino pase por \(v\).
  }
  \uncover<+->{
    Note que \(f(s) = h(s)\),
    que \(f(v) = f(s)\) para todo \(v\) en un camino óptimo,
    y que \(f(v) > f(s)\) si \(v\) no está en un camino óptimo.
    No conocemos \(f\)
    (determinar su valor es precisamente el objetivo del ejercicio),
    pero es razonable usar una estimación de \(f\)
    como función de evaluación \(\widehat{f}\)
  }
\end{frame}

\begin{frame}
  \setcounter{beamerpauses}{2}
  \frametitle{Definiciones}

  \uncover<+->{
    Es natural descomponer \(f(v)\) en dos partes:
    el costo óptimo para llegar de \(s\) a \(v\)
    (le llamaremos \(g(v)\))
    y el costo óptimo para llegar de \(v\) a una meta
    (antes lo definimos como \(h(v)\)).
    \begin{equation*}
      f(v)
        = g(v) + h(v)
    \end{equation*}
  }
\end{frame}

\begin{frame}
  \setcounter{beamerpauses}{2}
  \frametitle{Definiciones}

  \uncover<+->{
    Buscamos una estimación \(\widehat{f}(v)\),
    que igualmente podemos descomponer:
    \begin{equation*}
      \widehat{f}(v)
        = \widehat{g}(v) + \widehat{h}(v)
    \end{equation*}
    Acá \(\widehat{g}(v)\) es una estimación de \(g(v)\),
    a la mano está el mejor costo conocido.
    Siempre será \(\widehat{g}(v) \ge g(v)\).

    A su vez,
    \(\widehat{h}(v)\) es una estimación de \(h(v)\).
    Vemos que nos conviene \(\widehat{h}(v) \le h(v)\)
    (buscamos subestimar \(f(v)\) para evitar descartar soluciones viables).
  }
\end{frame}

\subsection{Algunas consecuencias}

\begin{frame}
  \frametitle{Conocemos \(g\) para un nodo en \(P\)}

  \begin{lemma}
    Para un nodo no cerrado \(v\) y un camino óptimo \(P\) de \(s\) a \(v\),
    hay un nodo abierto \(v'\) en \(P\)
    con \(\widehat{g}(v') = g(v')\).
  \end{lemma}
\end{frame}

\begin{frame}
  \setcounter{beamerpauses}{2}
  \frametitle{Conocemos \(g\) para un nodo en \(P\)}
  \framesubtitle{Demostración}

  Sea \(P = \langle s = v_0, \dotsc, v_k = v \rangle\).
  \uncover<+->{
    Si \(s\) está abierto
    (no se ha completado ninguna iteración),
    tome \(s = v'\),
    \(\widehat{g}(s) = g(s) = 0\),
    el lema se cumple trivialmente.
  }
  \uncover<+->{
    Si \(s\) está cerrado,
    sea \(\Delta\) el conjunto de nodos cerrados \(v_i\) en \(P\)
    para los que \(\widehat{g}(v_i) = g(v_i)\).
  }
  \uncover<+->{
    Sabemos que \(\Delta \ne \varnothing\),
    ya que \(s \in \Delta\).
    Sea \(v^*\) el último nodo cerrado de \(P\),
    \(v^* \ne v\) porque \(v\) está abierto.
  }
\end{frame}

\begin{frame}
  \setcounter{beamerpauses}{2}
  \frametitle{Conocemos \(g\) para un nodo en \(P\)}
  \framesubtitle{Demostración}

  \uncover<+->{
    Sea \(v'\) el sucesor de \(v^*\) en \(P\).
    Entonces:
    \begin{align*}
      \widehat{g}(v')
        &\le \widehat{g}(v^*) + c(v^* v')
          && \text{por definición de \(\widehat{g}\)} \\
      \widehat{g}(v^*)
        &=   g(v^*)
          && \text{porque \(v^* \in \Delta\)} \\
      g(v')
        &=   g(v^*) + c(v^* v')
          && \text{dado que \(P\) es óptimo}
    \end{align*}
  }
  \uncover<+->{
    Concluimos que \(\widehat{g}(v') \le g(v')\),
    como \(\widehat{g}(v') \ge g(v')\) resulta \(\widehat{g}(v') = g(v')\),
    y por la definición de \(\Delta\),
    \(v'\) está abierto.
    \qed
  }
\end{frame}

\begin{frame}
  \frametitle{Hay un nodo abierto
                   con \(\widehat{f}(v') \le f(s)\) en \(P\)}

  \begin{corollary}
    Suponga que para todo \(v\) es \(\widehat{h}(v) \le h(v)\),
    y que \(A^*\) no ha terminado.
    Entonces para todo camino óptimo \(P\) de \(s\) a \(T\)
    hay un nodo abierto \(v' \in P\) con \(\widehat{f}(v') \le f(s)\).
  \end{corollary}
\end{frame}

\begin{frame}
  \setcounter{beamerpauses}{2}
  \frametitle{Hay un nodo abierto
                   con \(\widehat{f}(v') \le f(s)\) en \(P\)}
  \framesubtitle{Demostración}

  \uncover<+->{
    La demostración es por contradicción.
  }
  \uncover<+->{
    Hay dos casos a considerar:
  }
  \begin{description}
  \item[No termina:]
    Sea \(t \in T\),
    alcanzable desde \(s\) en un número finito de pasos
    con costo mínimo \(f(s)\).
    Como el costo de cada arco es a lo menos \(\delta\),
    se alcanza \(t\) en a lo más \(M = f(s) / \delta\) pasos,
    y para todos los vértices \(v\)
    más lejos de \(s\) que \(M\) es:
    \begin{equation*}
      \widehat{f}(v)
        \ge \widehat{g}(v)
        \ge g(v)
        \ge M \delta
    \end{equation*}
  \end{description}
\end{frame}

\begin{frame}
  \setcounter{beamerpauses}{2}
  \frametitle{Hay un nodo abierto
                   con \(\widehat{f}(v') \le f(s)\) en \(P\)}
  \framesubtitle{Demostración}

  \begin{description}
  \item[]
    O sea,
    ningún nodo a distancia mayor a \(M\) de \(s\) se expande,
    ya que por el corolario
    habrá un nodo abierto \(v'\)en un camino óptimo
    tal que \(\widehat{f}(v') \le f(s) < f(v)\).
    \uncover<+->{
      El algoritmo elegirá \(v'\) en vez de \(v\).
      Hay un número finito de nodos a distancia a lo más \(M\),
      cada uno de ellos puede ser reabierto solo un número finito de veces,
      ya que hay un número finito de caminos que pasan por él,
      y se reabre solo si calculamos un \(\widehat{g}(v)\) menor.
    }
  \end{description}
\end{frame}

\begin{frame}
  \setcounter{beamerpauses}{2}
  \frametitle{Hay un nodo abierto
                   con \(\widehat{f}(v') \le f(s)\) en \(P\)}
  \framesubtitle{Demostración}

  \begin{description}
  \item[Entrega un camino no óptimo:]
    Supongamos que \(A^*\) termina en el nodo \(t\)
    con \(\widehat{f}(t) = \widehat{g}(t) > f(s)\).
    Por el corolario,
    había un nodo abierto \(v'\) en un camino óptimo
    con \(\widehat{f}(v') \le f(s) < \widehat{f}(t)\).
    Se habría elegido \(v'\) para ser expandido en vez de \(t\),
    con lo que \(A^*\) no habría terminado.
    \qed
  \end{description}
\end{frame}

\section{\(A^*\) es óptimo}

\begin{frame}
  \setcounter{beamerpauses}{2}
  \frametitle{\(A^*\) es óptimo}

  \uncover<+->{
    Hemos demostrado que si \(\widehat{h}(v) \le h(v)\),
    \(A^*\) es admisible.
    Una cota inferior obvia es \(\widehat{h}(v) = 0\),
    con la que \(A^*\) es ciego
    (el resultado es esencialmente el algoritmo de Dijkstra).
  }

  \uncover<+->{
    En el problema original de movimiento de un robot en un área con obstáculos,
    una cota inferior a la distancia a recorrer
    es la distancia entre dos puntos,
    obviando los obstáculos.
  }
  \uncover<+->{
    En general,
    si omitimos algunas de las restricciones del problema,
    obtendremos un costo no mayor,
    o sea un valor admisible de \(\widehat{h}(v)\).
  }
  \uncover<+->{
    Resulta que \(A^*\) es óptimo,
    en el sentido que expande el mínimo número de nodos
    entre todos los algoritmos que usan la misma información
    (la misma cota \(\widehat{h}\)).
    Esto porque un algoritmo que \emph{no} expanda
    todos los nodos con \(\widehat{f}(v) < f(s)\) para la meta \(s\)
    puede omitir el camino óptimo.
  }
\end{frame}

\begin{frame}
  \setcounter{beamerpauses}{2}
  \frametitle{\(A^*\) es óptimo}

  \uncover<+->{
    Diremos que la estimación \(\widehat{h}\)
    cumple la condición de \emph{monotonía} si:
    \begin{equation*}
      h(u, v) + \widehat{h}(u)
        \ge \widehat{h}(v)
    \end{equation*}
  }
  \uncover<+->{
    La condición
    expresa que la estimación \(\widehat{h}(v)\)
    no puede mejorarse usando datos correspondientes de otros nodos.
  }

  \uncover<+->{
    Resulta que si \(\widehat{h}\) cumple monotonía,
    nunca se reconsideran nodos.
  }
\end{frame}

\begin{frame}
  \frametitle{Consecuencias de monotonía}

  \begin{lemma}
    Suponga que se cumple
    la condición de monotonía y que \(A^*\) cerró el nodo \(v\).
    Entonces \(\widehat{g}(v) = g(v)\).
  \end{lemma}
\end{frame}

\begin{frame}
  \setcounter{beamerpauses}{2}
  \frametitle{Consecuencias de monotonía}
  \framesubtitle{Demostración}

  \uncover<+->{
    Por contradicción.
    Considere el subgrafo \(G_s\) justo antes de cerrar \(v\),
    y suponga que \(\widehat{g}(v) > g(v)\).
    Sea \(P\) un camino óptimo de \(s\) a \(v\),
    como \(\widehat{g}(v) > g(v)\) el algoritmo no lo encontró.
  }
  \uncover<+->{
    Por el lema,
    hay un nodo abierto \(v' \in P\) con \(\widehat{g}(v') = g(v')\).
    Por suposición,
    \(v \ne v'\),
    con lo que:
    \begin{align*}
      g(v)
        &= g(v') + h(v', v) \\
        &= \widehat{g}(v') + h(v', v)
    \end{align*}
  }
\end{frame}

\begin{frame}
  \setcounter{beamerpauses}{2}
  \frametitle{Consecuencias de monotonía}
  \framesubtitle{Demostración}

  \uncover<+->{
    Vale decir:
    \begin{align*}
      \widehat{g}(v)
        &> \widehat{g}(v') + h(v', v) \\
    \intertext{Sumando \(\widehat{h}\) a ambos lados:}
      \widehat{g}(v) + \widehat{h}(v)
        &> \widehat{g}(v') + h(v', v) + \widehat{h}(v') \\
    \intertext{Aplicando monotonía al lado derecho:}
      \widehat{g}(v) + \widehat{h}(v)
        &> \widehat{g}(v') + \widehat{h}(v')
    \end{align*}
  }
\end{frame}

\begin{frame}
  \setcounter{beamerpauses}{2}
  \frametitle{Consecuencias de monotonía}
  \framesubtitle{Demostración}

  \uncover<+->{
    Por la definición de \(\widehat{f}\):
    \begin{align*}
      \widehat{f}(v)
        &> \widehat{f}(v')
    \end{align*}
  }
  \uncover<+->{
    Pero en tal caso \(A^*\) hubiese expandido \(v'\),
    que estaba disponible,
    en vez de \(v\).
    \qed
  }
\end{frame}

\section{Resumen}

\begin{frame}
  \setcounter{beamerpauses}{2}
  \frametitle{Resumen}

  \begin{itemize}
  \item
    Describimos el algoritmo \(A^*\).
  \item
    Tiene el inconveniente de que exige una cola de prioridad
    que permite modificar prioridades.
    \uncover<+->{
      Si no está disponible esa operación,
      podemos simplemente reinsertar el nodo con la nueva prioridad,
      con poco impacto en el rendimiento.
    }
  \item
    Demostramos que es óptimo
    \uncover<+->{
      (expande el mínimo número de nodos
       entre algoritmos que usan la misma cota).
    }
  \item
    Si se cumple monotonía,
    cuando se cierra un nodo se conoce el mínimo costo de un camino a él.
  \end{itemize}
\end{frame}
\end{document}

%%% Local Variables:
%%% mode: latex
%%% TeX-master: t
%%% ispell-local-dictionary: "spanish"
%%% End:

% LocalWords:  glyphtounicode mdseries Function function end Downto
% LocalWords:  Procedure procedure downto Continue continue Break old
% LocalWords:  break KwStep step Mark closed less than value Remove
% LocalWords:  mark from Loop loop subgrafo
