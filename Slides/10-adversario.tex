\input glyphtounicode%
\pdfgentounicode=1
\documentclass[english, spanish, fleqn,%
hyperref = {colorlinks, urlcolor = blue}%
%, handout%
]{beamer}

%% ]{article}
%% \usepackage{beamerarticle}

\usepackage{beamerthemesplit}

\usepackage[utf8]{inputenc}
\usepackage{fourier}
\usepackage{amsmath}
\usepackage{amsthm}
\usepackage{siunitx}
\usepackage[es-noquoting]{babel}
\usepackage[noline, noend]{algorithm2e}
\usepackage{csquotes}

\ifdefined\HCode
  \def\pgfsysdriver{pgfsys-dvisvgm4ht.def}
\fi

\usefonttheme{professionalfonts}

%%%
%%% For algorithm2e
%%%

\SetAlgorithmName{Algoritmo}{Algoritmo}{Índice de algoritmos}

\SetAlCapSty{mdseries}
\SetKwProg{Function}{function}{}{end}

\beamerdefaultoverlayspecification{<+->}

\title{Argumento de adversario}

\author[Horst H. von Brand]{Horst H. von Brand\\
  \href{mailto:vonbrand@inf.utfsm.cl}{vonbrand@inf.utfsm.cl}}

\institute[DI UTFSM]{Departamento de Informática\\
                     Universidad Técnica Federico Santa María}
\date{}

\begin{document}
\frame{\maketitle}

\begin{frame}<handout>
  \frametitle{Contenido}

  \tableofcontents
\end{frame}

\section{Idea básica}

\begin{frame}
  \setcounter{beamerpauses}{2}
  \frametitle{Idea básica}

  \uncover<+->{
    Una técnica útil para demostrar una cota
    es imaginar un adversario omnisciente
    que da al algoritmo los peores posibles datos.
  }
  \uncover<+->{
    El adversario puede ajustar su comportamiento
    según lo que el algoritmo solicita,
    solo debe ser consistente con sus respuestas previas.
  }
\end{frame}

\begin{frame}
  \setcounter{beamerpauses}{2}
  \frametitle{Un ejemplo simple}

  \uncover<+->{
    Nos dan un arreglo de \(n\) bits,
    y piden determinar si hay algún bit \(1\) en él.
  }
  \uncover<+->{
    Parece obvio que para asegurar que no hay ningún \(1\),
    debemos revisar todos los bits.
  }
  \uncover<+->{
    ¿Cómo asegurarnos que no hay alguna astuta estrategia
    que evita revisarlos todos?
  }
\end{frame}

\begin{frame}
  \setcounter{beamerpauses}{2}
  \frametitle{Un adversario}

  \uncover<+->{
    Construiremos un adversario
    que entrega los bits solicitados por el algoritmo,
    de forma de poder entregar un arreglo consistente con sus respuestas
    que contiene un \(1\) si se le pide.
  }
  \uncover<+->{
    La idea es forzar que hayan varios arreglos posibles
    consistentes con las respuestas dadas.
    Si se solicita el bit \(i\),
    el adversario retorna \(0\) y registra esa respuesta.
    Mientras queden bits sin consultar,
    estos pueden ser \(0\) o \(1\),
    si se le pide al adversario
    (que asegura que hay un \(1\))
    que muestre sus cartas,
    muestra el arreglo con algún no consultado en \(1\).
  }
  \uncover<+->{
    Para asegurar que no hay \(1\),
    hay que consultar los \(n\) bits,
    como queríamos demostrar.
  }
\end{frame}

\section{Propiedades elusivas}

\begin{frame}
  \setcounter{beamerpauses}{2}
  \frametitle{Propiedades elusivas}

  \begin{definition}<+->
    Una propiedad de una estructura de datos se llama \emph{elusiva}
    si para determinar si la estructura cumple es necesario revisarla completa.
  \end{definition}
  \smallskip
  \uncover<+->{
    Demostramos antes
    que la propiedad \textquote{el arreglo de bits contiene un \(1\)}
    es elusiva.
  }
\end{frame}

\begin{frame}
  \setcounter{beamerpauses}{2}
  \frametitle{Palabras que contienen \(01\)}

  \uncover<+->{
    Después de palabras que contienen (o no) \(1\),
    es natural preguntar por otros patrones.
  }
  \uncover<+->{
    Podemos verificar si una palabra contiene \(01\)
    revisando todos sus bits.
  }
  \uncover<+->{
    La pregunta es si se puede hacer sin revisar la palabra entera.
  }
\end{frame}

\begin{frame}
  \setcounter{beamerpauses}{2}
  \frametitle{Palabras que contienen \(01\)}

  \uncover<+->{
    Si el largo \(n\) de la palabra es impar,
    podemos revisar sus posiciones pares,
    \(B[2], B[4], \dotsc, B[n - 1]\).
  }
  \uncover<+->{
    Si para algún \(i < j\) vemos \(B[i] = 0\) y \(B[j] = 1\),
    podemos parar:
    sabemos que hay \(01\) entremedio.
  }
  \uncover<+->{
    Si hay solo \(1\) seguidos por \(0\),
    no tiene sentido revisar el bit entre el último \(1\) y el primer \(0\),
    los demás son posibles candidatos a completar \(01\).
  }
  \uncover<+->{
    Si hay solo \(0\),
    debemos revisar si hay \(1\) en las posiciones \(3, 5, \dotsc, n\)
    (un \(1\) en la primera posición iría seguido por \(0\));
     si hay solo \(1\),
    debemos revisar si hay \(0\) las posiciones \(1, 3, \dotsc, n - 2\)
    (un \(0\) en la última posición seguiría un \(1\)).
  }
  \uncover<+->{
    O sea,
    basta revisar \(n - 1\) bits.
  }
\end{frame}

\begin{frame}
  \setcounter{beamerpauses}{2}
  \frametitle{Palabras que contienen \(01\)}

  \uncover<+->{
    Si el largo de la palabra es par,
    un argumento de adversario demuestra que deben revisarse todos los bits.
  }

  \uncover<+->{
    Una palabra sin \(01\) debe tener la forma \(1^* 0^*\).
  }
  \uncover<+->{
    El adversario mantiene índices \(r\) y \(s\),
    el rango \(1, \dotsc, r\) son \(1\)
    y el rango \(s, \dotsc, n\) son \(0\)
    con el rango entre \(r + 1\) y \(s - 1\) indeciso.
  }
  \uncover<+->{
    Si el rango indeciso siempre tiene largo par,
    puede contener \(01\).
  }
\end{frame}

\begin{frame}
  \setcounter{beamerpauses}{2}
  \frametitle{Palabras que contienen \(01\)}

  \uncover<+->{
    Inicialmente \(r = 0\) y \(s = n + 1\).
  }

  \uncover<+->{
    El adversario mantiene la invariante que \(s - r\) es par.
  }
  \uncover<+->{
    Si el algoritmo solicita el bit \(i\) con \(i \le r\),
    retorna \(1\).
    Si el algoritmo solicita el bit \(i\) con \(i \ge s\),
    retorna \(0\).
  }
  \uncover<+->{
    Si se solicita el bit \(i\) con \(r < i < s\),
    ajusta \(r\) o \(s\) a \(i\) y retorna el bit que corresponde.
  }

  \uncover<+->{
    Como \(s - r\) siempre es par,
    si no se han consultado todos los bit hay al menos \(2\) bits contiguos
    que pueden ser \(01\).
  }
\end{frame}

\begin{frame}
  \frametitle{Ocultar \(01\)}

  \begin{algorithm}[H]
    \DontPrintSemicolon

    \(r \leftarrow 0; s \leftarrow n + 1\) \;
    \;
    \Function{\(\mathrm{Hide01}(i)\)}{
      \uIf{\(i \le r\)}{
        \(B[i] \leftarrow 1\) \;
      }
      \uElseIf{\(i \ge s\)}{
        \(B[i] \leftarrow 0\) \;
      }
      \uElseIf{\(i - r\) is even}{
        \(B[i] \leftarrow 0\) ; \(s \leftarrow i\) \;
      }
      \Else {
        \(B[i] \leftarrow 1\) ; \(r \leftarrow i\) \;
      }
      \Return \(B[i]\) \;
    }
  \end{algorithm}
\end{frame}

\begin{frame}
  \setcounter{beamerpauses}{2}
  \frametitle{Patrones elusivos}

  \uncover<+->{
    O sea,
    el patrón \(01\)
    (y simétricamente \(10\))
    es elusivo solo si el largo de la palabra es par.
  }
  \uncover<+->{
    Resulta que los únicos patrones elusivos para todos los largos
    son los patrones de un símbolo \(0\) y \(1\).
  }
\end{frame}

\section{Ordenar por comparaciones}

\begin{frame}
  \setcounter{beamerpauses}{2}
  \frametitle{Ordenar por comparaciones}

  \uncover<+->{
    Tenemos \(n\) elementos distintos,
    que debemos ordenar de menor a mayor usando comparaciones.
  }
  \uncover<+->{
    Hay \(n!\) permutaciones de \(n\)~elementos,
    el adversario mantiene el conjunto \(L\)
    de las permutaciones consistentes con sus respuestas hasta el momento.
  }
\end{frame}

\begin{frame}
  \setcounter{beamerpauses}{2}
  \frametitle{Ordenar por comparaciones}

  \uncover<+->{
    Inicialmente \(L\) son todas las permutaciones.
  }

  \uncover<+->{
    La consulta \textquote{¿es \(a[i] < a[j]\)?}
    divide las permutaciones posibles
    en conjuntos de permutaciones \(L_{\text{Si}}\) y \(L_{\text{No}}\),
    el adversario da la respuesta que corresponde al conjunto mayor.
  }
  \uncover<+->{
    Como cada comparación reduce el conjunto de permutaciones
    en a lo más la mitad,
    y debemos llegar a reducirlo a una sola,
    se requieren al menos \(\log_2 n! = \Theta(n \log n)\) comparaciones.
  }
\end{frame}

\begin{frame}
  \setcounter{beamerpauses}{2}
  \frametitle{Ordenar por comparaciones}

  \uncover<+->{
    Note que el adversario
    no depende de las comparaciones hechas por el algoritmo.
  }
  \uncover<+->{
    Obviamente es mucho trabajo del adversario,
    pero es una construcción teórica.
  }
\end{frame}

\section{Propiedades de grafos}

\begin{frame}
  \setcounter{beamerpauses}{2}
  \frametitle{Propiedades de grafos}

  \uncover<+->{
    Supondremos grafos representados mediante matrices de adyacencia.
  }
\end{frame}

\begin{frame}
  \setcounter{beamerpauses}{2}
  \frametitle{Grafo vacío}

  \uncover<+->{
    Un ejemplo obvio es preguntar si el grafo es vacío
    (no tiene arcos).
  }
  \uncover<+->{
    Dado el conjunto de vértices \(V\),
    el adversario mantiene el grafo \(G\)
    (esencialmente lo no revisado).
  }
  \uncover<+->{
    Inicialmente \(G\) es el grafo completo sobre \(V\).
    Cada vez que el algoritmo consulta si \(u v\) es un arco del grafo,
    el adversario responde \textquote{No} y elimina \(u v\) de \(G\).
    Si el algoritmo no consulta por algún posible arco,
    el grafo vacío \(E = (V, \varnothing)\)
    y el grafo no vacío \(G\) que mantiene el adversario
    son ambos consistentes con las respuestas dadas,
    el algoritmo no puede responder correctamente.
  }
\end{frame}

\begin{frame}
  \setcounter{beamerpauses}{2}
  \frametitle{Conectividad}

  \uncover<+->{
    El adversario mantiene dos grafos,
    \(S\) y \(T\)
    (por \textquote{Sí} y \textquote{Tal vez}).
    El grafo \(S\) contiene los arcos
    que el algoritmo sabe pertenecen al grafo,
    \(T\) además contiene los arcos aún no considerados.
  }
  \uncover<+->{
    Inicialmente \(S = (V, \varnothing)\) es el grafo vacío
    y \(T\) es el grafo completo sobre \(V\).
  }
  \uncover<+->{
    Supondremos que el algoritmo no hace consultas redundantes,
    si consulta sobre el arco \(e\) es que no ha consultado sobre él antes
    (o sea,
    \(e\) pertenece a \(T \smallsetminus S\)).
  }
\end{frame}

\begin{frame}
  \frametitle{Ocultar conectividad}

  \begin{algorithm}[H]
    \DontPrintSemicolon

    \(S \leftarrow (V, \varnothing);
      T \leftarrow (V, \{ u v \colon u \ne v \})\) \;
    \;
    \Function{\(\mathrm{HideConnected}(e)\)}{
      \If{\(T \smallsetminus \{e\}\) is connected}{
        \(T \leftarrow T \smallsetminus \{ e \}\) \;
        \Return \(\mathrm{False}\) \;
      }
      {
        \(S \leftarrow S \cup \{ e \}\) \;
        \Return \(\mathrm{True}\) \;
      }
    }
  \end{algorithm}
\end{frame}

\begin{frame}
  \setcounter{beamerpauses}{2}
  \frametitle{Invariantes}

  \begin{description}
  \item[\boldmath\(S\) es subgrafo de \(T\)\unboldmath:]
    Esto es obvio.
  \item[\boldmath\(T\) es conexo\unboldmath:]
    También es obvio.
  \item[\boldmath Si \(T\) tiene un ciclo,
        ninguno de sus arcos está en \(S\)\unboldmath:]
    Si el algoritmo hubiese consultado por uno de los arcos del ciclo,
    el adversario lo habría eliminado de \(T\) y no lo agregaría a \(S\).
  \item[\boldmath\(S\) es acíclico\unboldmath:]
    Esto es inmediato del invariante anterior.
  \item[\boldmath Si \(S \ne T\), \(S\) no es conexo\unboldmath:]
    Sabemos que los grafos conexos acíclicos son árboles.
    Supongamos \(S\) es un árbol y \(e\) es un arco en \(T\) pero no en \(S\).
    Entonces \(S \cup \{ e \}\) tiene un ciclo,
    que está también en \(T\).
    Esto viola el tercer invariante.
  \end{description}
\end{frame}

\begin{frame}
  \setcounter{beamerpauses}{2}
  \frametitle{Conclusión}

  \uncover<+->{
    Si el algoritmo no revisa todos los arcos,
    \(S \ne T\),
    ambos consistentes con las respuestas dadas,
    pero uno es conexo y el otro no.
    La conectividad es elusiva.
  }
\end{frame}

\section{Máximo y mínimo}

\begin{frame}
  \setcounter{beamerpauses}{2}
  \frametitle{Máximo y mínimo}

  \uncover<+->{
    Para \(n\) elementos,
    obvio es \(n - 1\) comparaciones para obtener el máximo
    y \(n - 2\) para identificar el mínimo entre los demás,
    total \(2 n - 3\).
  }
\end{frame}

\begin{frame}
  \setcounter{beamerpauses}{2}
  \frametitle{Máximo y mínimo}

  \uncover<+->{
    \begin{algorithm}[H]
      \DontPrintSemicolon

      \Function{\(\mathrm{MaxMin}(\mathbf{a})\)}{
        \(m \leftarrow a[1]; M \leftarrow a[1]\) \;
        \For{\(i \leftarrow 2\) \KwTo \(\mathrm{length}(\mathbf{a})\)}{
           \uIf{\(a[i] > M\)}{
             \(M \leftarrow a[i]\) \;
           }
           \ElseIf{\(a[i] < m\)}{
            \(m \leftarrow a[i]\) \;
          }
        }
        \Return \((M, m)\) \;
      }
    \end{algorithm}
  }
  \smallskip
  \uncover<+->{
    Este algoritmo bastante obvio
    para un arreglo de largo \(n\)
    efectúa \(n - 1\) comparaciones con \(M\) y \(n - 2\) con \(m\),
    para un total de \(2 n - 3\).
  }
\end{frame}

\begin{frame}
  \setcounter{beamerpauses}{2}
  \frametitle{Máximo y mínimo}

  \uncover<+->{
    Mejor es dividir en pares,
    comparar los pares y dejar los menores en \(L\) y los mayores en \(H\).
    Si hay un número impar de elementos al último lo agregamos a ambos.
  }
  \uncover<+->{
    Resulta \(\lvert L \rvert = \lvert H \rvert = \lceil n / 2 \rceil\).
    Esto consume \(\lfloor n / 2 \rfloor\) comparaciones.
  }

  \uncover<+->{
    Obtenemos el mínimo de \(L\) con \(\lvert L \rvert - 1\) comparaciones,
    el máximo de \(H\) con \(\lvert H \rvert - 1\) comparaciones,
    para un total de:
    \begin{equation*}
      \left\lfloor \frac{n}{2} \right\rfloor
        + \left\lceil \frac{n}{2} \right\rceil - 1
        + \left\lceil \frac{n}{2} \right\rceil - 1
        = \left\lceil \frac{3 n}{2} \right\rceil - 2
    \end{equation*}
  }
\end{frame}

\begin{frame}
  \setcounter{beamerpauses}{2}
  \frametitle{Máximo y mínimo}

  \uncover<+->{
    Para una cota inferior,
    usamos un argumento de adversario.
  }
  \uncover<+->{
    El adversario mantiene marcas en los elementos,
    \(+\) si puede ser el máximo y \(-\) si puede ser el mínimo.
  }
  \uncover<+->{
    Inicialmente todos los elementos llevan marcas \(+\) y \(-\),
    un total de \(2 n\) marcas.
    Al final debe quedar exactamente un elemento marcado \(+\)
    y uno marcado \(-\)
    (el adversario puede declarar cualquier elemento marcado \(+\) como máximo
    y cualquier elemento marcado \(-\) como mínimo).
  }
\end{frame}

\begin{frame}
  \setcounter{beamerpauses}{2}
  \frametitle{Máximo y mínimo}

  \uncover<+->{
    Al comparar dos elementos marcados \(\pm\),
    el adversario declara uno mayor y el otro menor,
    eliminando las marcas \(-\) y \(+\) de los dos elementos,
    respectivamente.
  }
  \uncover<+->{
    En todos los otros casos
    el adversario puede responder
    de manera de eliminar a lo más una de las marcas.
  }
  \uncover<+->{
    Al comparar un elemento marcado \(\pm\) con uno marcado \(-\),
    declara que el segundo es menor y deja la marca \(+\)
    (el marcado \(\pm\) ya no es el mínimo)
    y \(-\)
    (sigue pudiendo ser el menor).
  }
  \uncover<+->{
    El caso de comparar \(\pm\) con \(+\) es similar.
  }
  \uncover<+->{
    Comparar dos elementos marcados ambos \(+\) o \(-\)
    elimina la marca de uno de ellos y declara al otro mayor
    (respectivamente menor),
    comparar uno marcado \(+\) con uno marcado \(-\) no requiere cambios
    (el primero es mayor).
  }
\end{frame}

\begin{frame}
  \setcounter{beamerpauses}{2}
  \frametitle{Máximo y mínimo}

  \uncover<+->{
    Un algoritmo correcto debe remover \(2 n - 2\) marcas.
    A lo más \(\lfloor n / 2 \rfloor\) comparaciones eliminan dos marcas,
    otras comparaciones eliminan a lo más una.
    Si el número de comparaciones requeridas es a lo menos \(C\):
  }
  \begin{align*}
    \uncover<.->{
      C - \left\lfloor \frac{n}{2} \right\rfloor
        &= 2 n - 2 - 2 \left\lfloor \frac{n}{2} \right\rfloor \\
    }
    \uncover<+->{
      C
        &= \left\lceil \frac{3 n}{2} \right\rceil - 2
    }
  \end{align*}

  \uncover<+->{
    Como tenemos una cota superior igual a la inferior,
    este es el número exacto de comparaciones requerido
    (y el algoritmo esbozado es óptimo).
  }
\end{frame}

\section{Resumen}

\begin{frame}
  \setcounter{beamerpauses}{2}
  \frametitle{Resumen}

  \begin{itemize}
  \item
    Planteamos argumento de adversario,
    permiten obtener cotas para el rendimiento de algoritmos
    y el trabajo necesario para resolver ciertos problemas.
  \item
    Vimos propiedades elusivas.
    \uncover<+->{
      Significan que el algoritmo obvio de revisar todo
      es óptimo.
    }
  \item
    Discutimos varias aplicaciones:
    patrones en palabras,
    propiedades de grafos,
    ordenamiento,
    máximo y mínimo.
  \item
    Para máximo y mínimo
    hallamos un algoritmo que hace exactamente el trabajo mínimo,
    es óptimo
    (en término de número de comparaciones).
  \end{itemize}
\end{frame}
\end{document}

%%% Local Variables:
%%% mode: latex
%%% TeX-master: t
%%% ispell-local-dictionary: "spanish"
%%% End:

% LocalWords:  glyphtounicode is even connected subgrafo handout
