\input glyphtounicode%
\pdfgentounicode=1
\documentclass[english, spanish, fleqn,%
hyperref = {colorlinks, urlcolor = blue}%
%, handout%
]{beamer}

%% ]{article}
%% \usepackage{beamerarticle}

\usepackage{beamerthemesplit}

\usepackage{fourier}
\usepackage{amsmath}
\usepackage{amsthm}
\usepackage{siunitx}
\usepackage[es-noquoting]{babel}
\usepackage{csquotes}
\usepackage{enumitem}
\usepackage{dcolumn, colortbl}
\usepackage{complexity}
\usepackage{listings}
\usepackage[noline, noend]{algorithm2e}

\ifdefined\HCode
  \def\pgfsysdriver{pgfsys-dvisvgm4ht.def}
\fi

\usefonttheme{professionalfonts}

%%%
%%% For listings
%%%

\lstloadlanguages{[ANSI] C, Python}

\beamerdefaultoverlayspecification{<+->}

\title{Programación dinámica}

\author[Horst H. von Brand]{Horst H. von Brand\\
  \href{mailto:vonbrand@inf.utfsm.cl}{vonbrand@inf.utfsm.cl}}

\institute[DI UTFSM]{Departamento de Informática\\
                     Universidad Técnica Federico Santa María}
\date{}

\begin{document}
%%%
%%% For listings
%%%

\lstset{basicstyle = \sffamily, commentstyle = \slshape,
        frame = none, numbers = none,
        tabsize = 4,
        showstringspaces = false,
        xleftmargin = 1em, xrightmargin = 1em
       }
\renewcommand{\lstlistingname}{Listado}

%%%
%%% For algorithm2e
%%%

\SetAlgorithmName{Algoritmo}{Algoritmo}{Índice de algoritmos}

\SetAlCapSty{mdseries}
\SetKwProg{Function}{function}{}{end}
\SetKwProg{Procedure}{procedure}{}{end}
\SetKw{Variables}{variables}
\SetKw{Downto}{downto}
\SetKwBlock{Loop}{loop}{end}
\SetKw{Continue}{continue}
\SetKw{Break}{break}
\SetKw{KwStep}{step}

\frame{\maketitle}

\begin{frame}<handout>
  \frametitle{Contenido}

  \tableofcontents
\end{frame}

\section{Idea general}

\begin{frame}
  \setcounter{beamerpauses}{2}
  \frametitle{Idea general}

  \uncover<+->{
    Vimos que algoritmos voraces son aplicables
    en problemas que podemos dividir en etapas,
    siempre que contemos con un criterio
    que en cada etapa nos dice cuál es la elección correcta.
  }
  \uncover<+->{
    Una alternativa cuando eso no se da es alguna búsqueda exhaustiva,
    analizando todas las posibles elecciones.
    Eso lleva directamente
    a algoritmos como \emph{\foreignlanguage{english}{backtracking}}
    o \emph{\foreignlanguage{english}{branch and bound}}.
  }
  \uncover<+->{
    En muchos casos esta idea
    lleva a resolver repetidas veces el mismo (sub)problema.
    Obtenemos un algoritmo mucho más eficiente
    si guardamos el resultado de resolver el subproblema en vez de recalcular.
  }
  \uncover<+->{
    Hay varias maneras de organizar esto.
  }
\end{frame}

\section{Programación dinámica}

\begin{frame}
  \setcounter{beamerpauses}{2}
  \frametitle{Variaciones de programación dinámica}

  \uncover<+->{
    Una idea es plantear un programa
    (probablemente recursivo)
    para resolver el problema,
    y reemplazar las llamadas recursivas
    por una consulta a alguna estructura adecuada
    que registra valores calculados
    antes de efectuar el cálculo.
  }
  \uncover<+->{
    A esto le llaman \emph{memorización},
    o más comúnmente,
    \emph{memoización}.
  }
  \\ \smallskip
  \uncover<+->{
    Otra idea es calcular sistemáticamente las soluciones a subproblemas,
    de forma que estén disponibles cuando se requieren.
    Las soluciones se organizan en alguna estructura de datos apropiada,
    comúnmente llamada \textquote{tabla}.
    Esto es propiamente \emph{programación dinámica}.
  }
  \uncover<+->{
    A veces puede ahorrarse memoria
    reteniendo únicamente las soluciones que se usarán.
  }
\end{frame}

\begin{frame}
  \setcounter{beamerpauses}{2}
  \frametitle{Números de Fibonacci}

  \uncover<+->{
    Para clarificar ideas,
    consideremos el cálculo de números de Fibonacci,
    definidos por la recurrencia:
    \begin{equation*}
      F_{n + 2}
        = F_{n + 1} + F_n
        \qquad F_0 = 0, F_1 = 1
    \end{equation*}
  }
  \\ \medskip
  \uncover<+->{
    Ilustraremos las distintas estrategias con el problema de calcular \(F_n\)
    (admitimos que es de juguete).
  }
\end{frame}

\begin{frame}
  \setcounter{beamerpauses}{2}
  \frametitle{Números de Fibonacci}

  \uncover<+->{
    Sabemos que:
    \begin{align*}
      F_n
       &=    \frac{\tau^n + (1 - \tau)^n}{\sqrt{5}}
          && \tau = \frac{1 + \sqrt{5}}{2} = 1,61803 \\
       &\sim \frac{\tau^n}{\sqrt{5}}
    \end{align*}
  }
  \uncover<+->{
    En \(64\)~bits cabe \(F_n\) hasta:
    \begin{equation*}
      n
        = \frac{64 \ln 2 + \ln \sqrt{5}}{\ln \tau}
        = 93,86
    \end{equation*}
  }
\end{frame}

\begin{frame}
  \frametitle{Números de Fibonacci}
  \framesubtitle{Cálculo recursivo ingenuo}

  \lstinputlisting[basicstyle = \small,
                   language = C,
                   firstline = 8, lastline = 13]
                  {fibonacci-recursive.c}
\end{frame}

\begin{frame}
  \setcounter{beamerpauses}{2}
  \frametitle{Números de Fibonacci}
  \framesubtitle{Cálculo recursivo ingenuo}

  \uncover<+->{
    Tomando como medida de tiempo el número de llamadas,
    llamando \(C_n\) al número de llamadas para calcular \(F_n\)
    resulta:
    \begin{equation*}
      C_{n + 2}
        = 1 + C_{n + 1} + C_n
        \qquad C_0 = C_1 = 1
    \end{equation*}
  }
  \uncover<+->{
    Técnicas tradicionales de solución de recurrencias dan:
    \begin{equation*}
      C_n
        = 2 F_{n + 1} - 1
    \end{equation*}
    (O verifique substituyendo.)
  }
  \medskip
  \uncover<+->{
    Hay \(n\) niveles de recursión,
    con lo que el espacio requerido es proporcional a \(n\).
  }
\end{frame}

\begin{frame}
  \frametitle{Números de Fibonacci}
  \framesubtitle{Versión memoizada}

  \lstinputlisting[basicstyle = \small,
                   language = C,
                   firstline = 11, lastline = 23]
                  {fibonacci-memoized.c}
\end{frame}

\begin{frame}
  \setcounter{beamerpauses}{2}
  \frametitle{Números de Fibonacci}
  \framesubtitle{Versión memoizada}

  \uncover<+->{
    Este programa efectúa \(2 n - 1\)~llamadas,
    y usa memoria proporcional a \(n\)
    (nuestra versión usa memoria constante,
     pero es simple reorganizarla
     para que se solicite memoria solo según se necesita).
  }
\end{frame}

\begin{frame}
  \frametitle{Números de Fibonacci}
  \framesubtitle{Programación dinámica}

  \lstinputlisting[basicstyle = \small,
                   language = C,
                   firstline = 8, lastline = 17]
                  {fibonacci-dp.c}
\end{frame}

\begin{frame}
  \setcounter{beamerpauses}{2}
  \frametitle{Números de Fibonacci}
  \framesubtitle{Programación dinámica}

  \uncover<+->{
    Este programa efectúa trabajo proporcional a~\(n\),
    y usa memoria proporcional a \(n\).
  }
\end{frame}

\begin{frame}
  \setcounter{beamerpauses}{2}
  \frametitle{Números de Fibonacci}
  \framesubtitle{Programación dinámica ahorrando memoria}

  \uncover<+->{
    Nuestra versión siguiente nace de la observación
    que solo se usan \lstinline!mem[i - 1]! y \lstinline!mem[i - 2]!,
    designamos variables \lstinline!b! y \lstinline!a! para estos valores.
  }
\end{frame}

\begin{frame}
  \frametitle{Números de Fibonacci}
  \framesubtitle{Programación dinámica ahorrando memoria}

  \lstinputlisting[basicstyle = \small,
                   language = C,
                   firstline = 8, lastline = 19]
                  {fibonacci-dp-frugal.c}
\end{frame}

\begin{frame}
  \setcounter{beamerpauses}{2}
  \frametitle{Números de Fibonacci}
  \framesubtitle{Programación dinámica ahorrando memoria}

  \uncover<+->{
    Este programa efectúa trabajo proporcional a~\(n\),
    y usa memoria constante.
  }
\end{frame}

\begin{frame}
  \setcounter{beamerpauses}{2}
  \frametitle{Aplicabilidad de programación dinámica}

  \uncover<+->{
    Suponemos un problema \(P\),
    que se resuelve en etapas,
    eligiendo parte de la solución \(p\) en cada una de ellas.
  }
  \begin{description}
  \item[Estructura Inductiva:]
    Dada la elección \(\widehat{p}\),
    queda un subproblema menor \(P'\)
    tal que si \(\Pi'\) es solución viable de \(P'\),
    \(\{\widehat{p}\} \cup \Pi'\) es solución viable de \(P\)
    (\(P'\) no tiene \textquote{restricciones externas}).
  \item[Subestructura Óptima:]
    Si \(P'\) queda de \(P\) al sacar \(\hat p\),
    y  \(\Pi'\) es óptima para \(P'\),
    \(\Pi' \cup \{ \widehat{p} \}\) es óptima para \(P\).
  \item[Elección Completa:]
    Entre todas las opciones,
    elegimos aquel \(\widehat{p}\) que da el mejor resultado
    combinado con \(\Pi'\),
    una solución óptima para el problema resultante
    de \(P \smallsetminus \{ \widehat{p} \}\).
  \end{description}
\end{frame}

\begin{frame}
  \setcounter{beamerpauses}{2}
  \frametitle{Aplicabilidad de programación dinámica}

  \uncover<+->{
    A diferencia de un algoritmo voraz,
    no conocemos un criterio \textquote{local}
    que nos permita elegir \(\widehat{p}\),
    debemos considerar varias opciones.
    Esto lleva naturalmente a una recursión:
    resuelva los subproblemas recursivamente,
    y elija aquella combinación que da la solución global.
  }
  \\ \medskip
  \uncover<+->{
    Lo anterior está planteado en términos de buscar un óptimo,
    pero puede adaptarse para determinar si hay o no soluciones.
  }
\end{frame}

\section{Estructura general}

\begin{frame}
  \setcounter{beamerpauses}{2}
  \frametitle{Desarrollo de un programa}

  \begin{enumerate}[font = \textbf, label = {(\alph*)}]
  \item \textbf{Plantear la recurrencia:}
    Esta es la parte crítica,
    depende íntimamente del problema.
    Debemos identificar los subproblemas relevantes,
    cómo se combinan para una solución,
    identificar todas las alternativas relevantes
    y cómo elegir la mejor de las alternativas.
  \item \textbf{Escribir un programa recursivo:}
    Generalmente es una traducción mecánica de la recurrencia.
  \end{enumerate}
\end{frame}

\begin{frame}
  \setcounter{beamerpauses}{2}
  \frametitle{Desarrollo de un programa}

  \begin{enumerate}[font = \textbf, label = {(\alph*)}, start = 3]
  \item \textbf{Identificar subproblemas:}
    \label{step:dp:identify}
    Vea todas las formas en las que el programa recursivo se llama a sí mismo.
    Determine el conjunto de posibles argumentos.
  \item \textbf{Defina una estructura de datos:}
    Requerimos almacenar resultados para todas las combinaciones posibles
    de argumentos.
    Esto generalmente lleva a alguna clase de tabla,
    pero perfectamente puede ser apropiada una estructura diferente.
  \end{enumerate}
\end{frame}

\begin{frame}
  \setcounter{beamerpauses}{2}
  \frametitle{Desarrollo de un programa}

  \begin{enumerate}[font = \textbf, label = {(\alph*)}, start = 5]
  \item \textbf{Identifique dependencias:}
    Debemos organizar los cálculos de forma que valores requeridos
    ya se hayan calculado antes.
    Por ejemplo,
    diagrama las dependencias.
  \item \textbf{Determine un buen orden de cálculo:}
    El orden en que funciona el programa recursivo es una guía.
    Interesa un orden simple de programar.
    Las dependencias descubiertas en el paso anterior
    definen un orden parcial entre subproblemas,
    buscamos una extensión lineal.
    Esto es crítico,
    tenga cuidado.
  \end{enumerate}
\end{frame}

\begin{frame}
  \setcounter{beamerpauses}{2}
  \frametitle{Desarrollo de un programa}

  \begin{enumerate}[font = \textbf, label = {(\alph*)}, start = 7]
  \item \textbf{Analice requerimientos de tiempo y espacio:}
    Depende fundamentalmente del número de subproblemas
    y lo que se debe hacer para cada uno de ellos.
    Incluso es posible de obtener directamente
    luego del paso~\ref{step:dp:identify}.
  \item \textbf{Escriba el algoritmo:}
    Esto es inmediato si se desarrollaron cuidadosamente los pasos anteriores.
  \end{enumerate}
\end{frame}

\begin{frame}
  \setcounter{beamerpauses}{2}
  \frametitle{Sobre la recurrencia}

  \uncover<+->{
    Las recurrencias que hemos usado \textquote{miran hacia atrás},
    el programa resultante naturalmente construye \textquote{hacia adelante}.
  }
  \uncover<+->{
    Es igualmente válido plantear una recurrencia
    que \textquote{mira hacia adelante}
    (considerar lo que falta para llegar a la meta,
     no el camino desde el inicio).
    El programa que resulta construye \textquote{hacia atrás}.
  }

  \uncover<+->{
    Cuál elegir depende de las preferencias del desarrollador
    y las particularidades del problema.
  }
\end{frame}

\section{Ejemplo detallado}

\begin{frame}
  \setcounter{beamerpauses}{2}
  \frametitle{Un ejemplo detallado}

  \uncover<+->{
    Para el entero no negativo \(n\) se define su \emph{valor recursivo}
    como el máximo entre \(n\)
    y la suma de los valores recursivos
    de \(\lfloor n / 2 \rfloor\),
    \(\lfloor n / 3 \rfloor\) y \(\lfloor n / 4 \rfloor\).
  }

  \uncover<+->{
    Es claro que habrán muchos cálculos repetidos
    al efectuar esto recursivamente,
    y éstos son independientes entre sí.
    Es la situación ideal para considerar programación dinámica.
  }
\end{frame}

\begin{frame}
  \setcounter{beamerpauses}{2}
  \frametitle{Un ejemplo detallado}
  \framesubtitle{Recurrencia}

  \uncover<+->{
    La recurrencia está básicamente dada por el enunciado:
    \begin{equation*}
      m(n)
        = \max \{
                   n,
                   m(\lfloor n / 2 \rfloor)
                     + m(\lfloor n / 3 \rfloor)
                     + m(\lfloor n / 4 \rfloor)
               \}
    \end{equation*}
    Es claro que \(m(0) = 0, m(1) = 1\).
  }
\end{frame}

\begin{frame}
  \setcounter{beamerpauses}{2}
  \frametitle{Un ejemplo detallado}
  \framesubtitle{Programa recursivo}

  \uncover<+->{
    El programa recursivo es bastante obvio,
    es una traducción inmediata
    de la recurrencia.
    \lstinputlisting[basicstyle = \small,
                     language = Python,
                     linerange = {3-8}]
                    {max_sum_rec.py}
  }
\end{frame}

\begin{frame}
  \setcounter{beamerpauses}{2}
  \frametitle{Un ejemplo detallado}
  \framesubtitle{Identificar subproblemas}

  \uncover<+->{
    De la recurrencia es obvio que subproblemas inmediatos para \(n\)
    son los para \(\lfloor n / 2 \rfloor\),
     \(\lfloor n / 3 \rfloor\) y \(\lfloor n / 4 \rfloor\).
  }
\end{frame}

\begin{frame}
  \setcounter{beamerpauses}{2}
  \frametitle{Un ejemplo detallado}
  \framesubtitle{Defina una estructura de datos}

  \uncover<+->{
    Se requiere únicamente el valor de \(m\) para valores selectos de \(n\),
    un arreglo
    (lista en Python)
    es suficiente.
    Bastan \(\lfloor n / 2 \rfloor\) elementos,
    elementos posteriores no se usan.
  }
\end{frame}

\begin{frame}
  \setcounter{beamerpauses}{2}
  \frametitle{Un ejemplo detallado}
  \framesubtitle{Identifique dependencias}

  \uncover<+->{
    Ya lo vimos,
    \(m(n)\)
    depende de los valores para \(\lfloor n / 2 \rfloor\),
    \(\lfloor n / 3 \rfloor\) y \(\lfloor n / 4 \rfloor\).
  }
\end{frame}

\begin{frame}
  \setcounter{beamerpauses}{2}
  \frametitle{Un ejemplo detallado}
  \framesubtitle{Determine un buen orden de cálculo}

  \uncover<+->{
    Como \(m(n)\)
    depende únicamente de valores de \(m\) para argumentos menores,
    un orden de cálculo obvio es llenar el arreglo secuencialmente.
  }
\end{frame}

\begin{frame}
  \setcounter{beamerpauses}{2}
  \frametitle{Un ejemplo detallado}
  \framesubtitle{Analice requerimientos de tiempo y espacio}

  \uncover<+->{
    Es claro que la memoria extra usada
    es el arreglo para guardar valores de \(m\)
    (espacio \(\Theta(n)\))
    y el cálculo de cada elemento hace referencia a tres elementos adicionales,
    es fijo
    (tiempo \(\Theta(n)\)).
  }
\end{frame}

\begin{frame}
  \setcounter{beamerpauses}{2}
  \frametitle{Un ejemplo detallado}
  \framesubtitle{Escriba el algoritmo}

  \uncover<+->{
    Escribir el programa final
    es juntar lo discutido.
    \label{item:dp-example-algorithm}
    \lstinputlisting[basicstyle = \small,
                     language = Python,
                     linerange = {3-14}]
                    {max_sum_dp.py}
  }
\end{frame}

\begin{frame}
  \setcounter{beamerpauses}{2}
  \frametitle{Un ejemplo detallado}
  \framesubtitle{Comentarios finales}

  \uncover<+->{
    Es útil tener a la mano
    el programa recursivo
    para contrastar resultados.
  }
  \uncover<+->{
    En general,
    el programa recursivo será demasiado ineficiente para tamaños grandes,
    habrá que elegir casos de prueba pequeños.
  }
\end{frame}

\section{Distancia de edición}

\begin{frame}
  \setcounter{beamerpauses}{2}
  \frametitle{Distancia de edición}

  \uncover<+->{
    La \emph{distancia de edición}
    (también llamada \emph{distancia de Levenshtein})
    entre dos palabras es el mínimo número de cambios de símbolo,
    inserciones y eliminaciones de símbolos individuales
    que transforman una en la otra.
  }
  \uncover<+->{
    Calcular esta distancia
    (con el problema relacionado de hallar la secuencia de operaciones)
    es recurrente.
    Por ejemplo,
    es lo que intenta hallar el programa \texttt{diff(1)} estándar de Unix
    o lo que muestra como diferencias entre una versión y otra
    el sistema de control de versiones predilecto,
    es una operación común en biología molecular
    (hallar similitudes entre genes),
    es útil a la hora de sugerir reemplazos en revisión de ortografía.
  }
\end{frame}

\begin{frame}
  \setcounter{beamerpauses}{2}
  \frametitle{Distancia de edición}

  \uncover<+->{
    Podemos ilustrar el proceso de edición
    anotando las palabras alineadas en columnas,
    con espacios donde se inserta o elimina un símbolo.
  }
  \uncover<+->{
    Por ejemplo,
    para editar \texttt{casas} a \texttt{calle}:
    \\ \smallskip
    \begin{tabular}{l}
      \texttt{cas\textvisiblespace as} \\
      \texttt{calle\textvisiblespace}
    \end{tabular}
  }
  \\ \smallskip
  \uncover<+->{
    Son \num{4} ediciones:
    cambiar `\texttt{s}' a `\texttt{l}',
    insertar `\texttt{l}',
    cambiar `\texttt{a}' a `\texttt{e}',
    eliminar `\texttt{s}'.
  }
  \uncover<+->{
    El mínimo son \num{3} ediciones:
    cambiar la primera `\texttt{s}' por `\texttt{l}',
    cambiar la segunda `\texttt{a}' por `\texttt{l}',
    cambiar la segunda `\texttt{s}' por `\texttt{e}'.
  }
\end{frame}

\begin{frame}
  \setcounter{beamerpauses}{2}
  \frametitle{Estructura recursiva}

  \uncover<+->{
    El primer paso es formular el problema en forma recursiva.
  }

  \uncover<+->{
    Nuestra representación en columnas tiene la crucial propiedad
    de subestructura óptima:
    si eliminamos la última columna,
    las demás representan una secuencia mínima para el resto de las palabras.
  }
  \uncover<+->{
    La demostración es por contradicción:
    si hubiera una secuencia más corta para el prefijo,
    con la columna última daría una secuencia más corta para el total.
  }

  \uncover<+->{
    Conociendo la última columna,
    el hada de recursión se hace cargo del resto.
  }
\end{frame}

\begin{frame}
  \setcounter{beamerpauses}{2}
  \frametitle{Estructura recursiva}

  \uncover<+->{
    Dicho de otra forma:
    el alineamiento buscado representa una secuencia de operaciones,
    ordenadas
    (sin particular razón)
    de derecha a izquierda.
  }
  \uncover<+->{
    Es una secuencia de decisiones,
    una por columna.
    En medio del proceso,
    tenemos alineados sufijos de las palabras.
    \\ \smallskip
    \begin{center}
      \begin{tabular}{|c|*{6}{>{\columncolor{orange}}p{0.7em}|}}
        \hline
        ALG  & O & R & I & T & M & O \\
        ELEG & A & N &	 & T & E &   \\
        \hline
      \end{tabular}
    \end{center}
    \smallskip
    El resto de las operaciones es independiente de lo ya acordado.
  }
\end{frame}

\begin{frame}
  \setcounter{beamerpauses}{2}
  \frametitle{Estructura recursiva}

  \uncover<+->{
    Para palabras \(A[1 .. m]\) y \(B[1 .. n]\) podemos plantear lo siguiente.
  }
\end{frame}

\begin{frame}
  \setcounter{beamerpauses}{2}
  \frametitle{Estructura recursiva}

  \uncover<+->{
    Para índices \(i, j\),
    sea \(\mathrm{edit}(i, j)\) la distancia entre los prefijos
    \(A[1 .. i]\) y \(B[1 .. j]\).
  }
  \uncover<+->{
    Para \(i, j\) positivos hay exactamente cuatro posibilidades
    para la última columna:
  }
  \begin{description}
  \item[Inserción:]
    La última entrada de la fila superior es vacía.
    En tal caso
    \(\mathrm{edit}(i, j) = \operatorname{edit}(i, j - 1) + 1\)
    (el \({}+ 1\) es por la última inserción).
  \item[Eliminación:]
    La última entrada de la fila inferior es vacía.
    En tal caso
    \(\mathrm{edit}(i, j) = \operatorname{edit}(i - 1, j) + 1\)
  \item[Reemplazo:]
    Las últimas entradas difieren.
    En tal caso
    \(\mathrm{edit}(i, j) = \operatorname{edit}(i - 1, j - 1) + 1\)
  \item[Coinciden:]
    Las últimas entradas son iguales.
    En tal caso
    \(\mathrm{edit}(i, j) = \operatorname{edit}(i - 1, j - 1)\)
  \end{description}
\end{frame}

\begin{frame}
  \setcounter{beamerpauses}{2}
  \frametitle{Estructura recursiva}

  \uncover<+->{
    Este análisis falla si \(i = 0\) o \(j = 0\),
    pero esos casos son simples de manejar directamente:
  }
  \begin{itemize}
  \item
    Transformar la palabra vacía en una palabra de largo \(j\)
    requiere \(j\) inserciones,
    \(\mathrm{edit}(0, j) = j\).
  \item
    Transformar una palabra de largo \(i\) en la palabra vacía
    requiere \(i\) eliminaciones,
    \(\mathrm{edit}(i, 0) = i\).
  \end{itemize}
  \uncover<+->{
    Prueba de sanidad:
    transformar la palabra vacía en la palabra vacía tiene costo cero.
  }
\end{frame}

\begin{frame}
  \setcounter{beamerpauses}{2}
  \frametitle{Estructura recursiva}

  Resulta la siguiente recursión para \(\mathrm{edit}\):
  \begin{equation*}
    \mathrm{edit}(i, j)
      = \begin{cases}
           i & \text{si \(j = 0\)} \\
           j & \text{si \(i = 0\)} \\
           \min
             \begin{Bmatrix}
               \mathrm{edit}(i, j - 1) + 1 \\
               \mathrm{edit}(i - 1, j) + 1 \\
               \mathrm{edit}(i - 1, j - 1) + [A[i] \ne B[j]]
             \end{Bmatrix}
             & \text{caso contrario}
        \end{cases}
  \end{equation*}
\end{frame}

\begin{frame}
  \setcounter{beamerpauses}{2}
  \frametitle{Programación dinámica}

  \begin{description}
  \item[Subproblemas:]
    Cada subproblema se identifica
    por índices \(0 \le i \le m\) y \(0 \le j \le n\).
  \item[Memoización:]
    Requerimos un arreglo bidimensional
    para almacenar \(\mathrm{edit}(i, j)\),
    \(e[0 .. m, 0 .. n]\).
  \item[Dependencias:]
    La entrada \(e[i, j]\) depende solo de los tres vecinos
    \(e[i - 1, j], e[i, j - 1], e[i - 1, j - 1]\).
  \item[Orden de llenado:]
    Si llenamos el arreglo fila por fila,
    cada fila de izquierda a derecha,
    al calcular un elemento están disponibles los de los que depende.
    \uncover<+->{
      (No es el único que funciona,
       pero es simple.)
    }
  \item[Costos:]
    Son \(m n\) entradas en el arreglo,
    llenar cada una toma tiempo constante.
  \end{description}
\end{frame}

\begin{frame}
  \frametitle{Algoritmo final}

  \selectlanguage{english}
  \begin{algorithm}[H]
    \DontPrintSemicolon

    \Function{\(\mathrm{EditDistance}(A[1 .. m], B[1 .. n])\)}{
      \ForEach{\(0 \le j \le n\)}{
        \(e[0, j] \leftarrow j\) \;
      }
      \For{\(i \leftarrow 1\) \KwTo \(m\)}{
        \(e[i, 0] \leftarrow i\) \;
        \For{\(j \leftarrow i\) \KwTo \(n\)}{
          \(\mathit{ins} \leftarrow e[i, j - 1] + 1\);
          \(\mathit{del} \leftarrow e[i - 1, j] + 1\) \;
          \eIf{\(A[i] = B[j]\)}{
            \(\mathit{rep} \leftarrow e[i - 1, j - 1]\) \;
          }{
            \(\mathit{rep} \leftarrow e[i - 1, j - 1] + 1\) \;
          }
          \(e[i, j] \leftarrow \min\{ \mathit{ins},
                                      \mathit{del},
                                      \mathit{rep} \}\) \;
        }
        \Return \(e[m, n]\) \;
      }
    }
  \end{algorithm}
  \selectlanguage{spanish}
\end{frame}

\begin{frame}
  \setcounter{beamerpauses}{2}
  \frametitle{Obtener las operaciones}

  \uncover<+->{
    Rescatar las operaciones requiere almacenar en el arreglo
    cuál de las tres (o cuatro)
    operaciones fue la que dio el óptimo de ese casillero,
    y recorrer hacia atrás desde la meta.
  }
  \\ \medskip
  \uncover<+->{
    Puede ahorrarse espacio en el cálculo de la distancia
    reteniendo solo las entradas que serán usadas luego.
  }
  \uncover<+->{
    Para reconstruir las operaciones,
    pueden retenerse parte de los datos y recalcular el resto.
    El apunte referencia detalles.
  }
  \\ \medskip
  \uncover<+->{
    Hay algoritmos más rápidos
    en el caso que la distancia es pequeña
    (común en aplicaciones:
     no se comparan archivos/genes radicalmente diferentes).
  }
\end{frame}

\begin{frame}
  \setcounter{beamerpauses}{2}
  \frametitle{Máximo conjunto independiente de un árbol}

  \uncover<+->{
    Recordamos que un \emph{conjunto independiente}
    (en inglés,
     \emph{\foreignlanguage{english}{independent set}})
    de un grafo es un conjunto de vértices entre los que no hay arcos.
  }
  \uncover<+->{
    Vimos que determinar si un grafo
    tiene un conjunto independiente de tamaño al menos \(k\)
    (el problema \textsc{IS})
    es \NP\nobreakdash-completo.
  }
  \uncover<+->{
    Demostraremos un elegante algoritmo
    para hallar el tamaño de un conjunto independiente de un árbol.
  }
\end{frame}

\begin{frame}
  \setcounter{beamerpauses}{2}
  \frametitle{Máximo conjunto independiente de un árbol}

  \uncover<+->{
    Analizando el árbol desde un nodo raíz
    (arbitrario,
     realmente),
    vemos que un conjunto independiente incluye a la raíz
    (en cuyo caso ninguno de los hijos de ésta puede pertenecer)
    o la excluye
    (los hijos de la raíz pueden pertenecer).
  }
\end{frame}

\begin{frame}
  \setcounter{beamerpauses}{2}
  \frametitle{Máximo conjunto independiente de un árbol}

  \uncover<+->{
    La idea general está clara entonces:
    calcular para los subárboles las cantidades \(E_v\)
    (excluye la raíz del subárbol con raíz en \(v\))
    e \(I_v\)
    (incluye la raíz del subárbol con raíz en \(v\)).
  }
  \uncover<+->{
    Calculamos entonces para el vértice \(v\):
    \begin{align*}
      I_v
        &= 1 + \sum_{t \in N(v)} E_t \\
      E_v
        &= \sum_{t \in N(v)} \max \{ I_t, E_t \}
    \end{align*}
    El tamaño máximo de un conjunto independiente es simplemente
    \(\max \{ I_r, E_r \}\) para la raíz \(r\) del árbol.
  }
\end{frame}

\begin{frame}
  \setcounter{beamerpauses}{2}
  \frametitle{Máximo conjunto independiente de un árbol}

  \uncover<+->{
    El planteo recursivo llama programación dinámica a gritos.
  }
  \uncover<+->{
    Claro que acá no tiene sentido una \textquote{tabla}
    (arreglo),
    almacenamos \(I_v, E_v\) en los nodos del árbol.
  }
  \uncover<+->{
    Un orden válido de cálculo simple está dado por un recorrido en postorden.
  }
\end{frame}

\section{Resumen}

\begin{frame}
  \setcounter{beamerpauses}{2}
  \frametitle{Resumen}

  \begin{itemize}
  \item
    Planteamos la estrategia de programación dinámica.
    \uncover<+->{
      Es una estrategia de muy amplia aplicabilidad,
      que vale la pena asimilar bien.
    }
  \item
    Discutimos de condiciones de aplicabilidad.
    \uncover<+->{
      Hay similitud con algoritmos voraces.
    }
    \uncover<+->{
      Como esos nunca funcionan\ldots
    }
  \item
    Describimos una estrategia general para desarrollar estos algoritmos,
    ilustrada con un ejemplo realista.
  \item
    El ejemplo de \textsc{IS} máximo de árboles recalca
    que la \textquote{tabla} no es central.
  \item
    Discusión más detallada y ejemplos
    en el \href{http://jeffe.cs.illinois.edu/teaching/algorithms}
               {texto de Jeff Erickson}.
  \end{itemize}
\end{frame}
\end{document}

%%% Local Variables:
%%% mode: latex
%%% TeX-master: t
%%% ispell-local-dictionary: "spanish"
%%% End:

% LocalWords:  glyphtounicode mdseries Function function end Downto
% LocalWords:  Procedure procedure downto Loop loop Continue continue
% LocalWords:  Break break KwStep step english backtracking branch em
% LocalWords:  and bound sub memoización memoizada computines cont IS
% LocalWords:  diff cas orange ALG ELEG edit independent set Jeff
% LocalWords:  postorden
