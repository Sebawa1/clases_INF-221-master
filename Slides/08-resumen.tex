\input glyphtounicode%
\pdfgentounicode=1
\documentclass[english, spanish, fleqn,%
hyperref = {colorlinks, urlcolor = blue}%
%, handout%
]{beamer}

%% ]{article}
%% \usepackage{beamerarticle}

\usepackage{beamerthemesplit}

\usepackage{fourier}
\usepackage[group-digits = false, copy-decimal-marker = true]{siunitx}
\usepackage{amsmath}
\usepackage{amsthm}
\usepackage[es-noquoting]{babel}
\usepackage{csquotes}

\ifdefined\HCode
  \def\pgfsysdriver{pgfsys-dvisvgm4ht.def}
\fi

\usefonttheme{professionalfonts}

\beamerdefaultoverlayspecification{<+->}

\title{Comentarios finales\\
         Análisis numérico}

\author[Horst H. von Brand]{Horst H. von Brand\\
  \href{mailto:vonbrand@inf.utfsm.cl}{vonbrand@inf.utfsm.cl}}

\institute[DI UTFSM]{Departamento de Informática\\
                     Universidad Técnica Federico Santa María}
\date{}

\begin{document}
\frame{\maketitle}

\begin{frame}<handout>
  \frametitle{Contenido}

  \tableofcontents
\end{frame}

\section{Áreas que vimos y que no vimos}

\begin{frame}
  \setcounter{beamerpauses}{2}
  \frametitle{Áreas que revisamos}

  \uncover<+->{
    Vimos
    (someramente)
    tema de errores en las operaciones
    y su propagación.
  }
  \uncover<+->{
    Discutimos cómo evaluar y describir errores en los algoritmos.
  }
  \\ \medskip
  \uncover<+->{
    Las áreas que discutimos en algún detalle
    son la solución de ecuaciones,
    interpolación
    y cuadratura.
  }
  \uncover<+->{
    De ellas vimos algunos de los algoritmos y técnicas más importantes,
    discutiendo técnicas para acotar errores.
  }
\end{frame}

\begin{frame}
  \setcounter{beamerpauses}{2}
  \frametitle{Áreas sin discutir}

  \uncover<+->{
    El análisis numérico es un área enorme.
    Temas que no tocamos incluyen:
  }
  \begin{itemize}
  \item
    Operaciones aritméticas
    (algo se ve en \emph{Arquitectura y Organización de Computadores}, INF-245)
  \item
    Cálculo de las funciones elementales%
    \uncover<+->{%
      (calcular \(\sqrt{x}\),
       potencias,
       logaritmos o funciones trigonométricas
       no es para nada trivial)
     }
  \item
    Cálculo de funciones comunes en ciencia e ingeniería,
    como ser \(\Gamma\)
    (extensión de factoriales a los reales o complejos),
    funciones cumulativas de distribuciones de probabilidad,
    funciones de Bessel o la función \(W\) de Lambert,
    requieren técnicas especializadas.
  \item
    Está el amplio tema del cálculo de una variedad de constantes:
    \(\pi\), \(\gamma\), \ldots{}
    Cada una con su propia subindustria.
  \end{itemize}
\end{frame}

\begin{frame}
  \setcounter{beamerpauses}{2}
  \frametitle{Áreas sin discutir}
  \begin{itemize}
   \item
     Resolver sistemas de ecuaciones lineales es un tema importante.
   \item
     Es común resolver sistemas de ecuaciones no lineales.
   \item
     Está el área gigantesca de solución de ecuaciones diferenciales,
     tanto ordinarias como parciales.
  \end{itemize}
\end{frame}

% \begin{frame}
%   \setcounter{beamerpauses}{2}
%   \frametitle{Cálculo de \(\pi\)}

%   \uncover<+->{
%     La técnica clásica
%     (desde Arquímedes)
%     es aproximar \(\pi\)
%     mediante el perímetro de polígonos regulares inscrito y circunscrito
%     al círculo.
%   }
%   \uncover<+->{
%     Considerando el hexágono inscrito
%     (de perímetro \(6 r\)
%     concluimos \(3 < \pi\).
%   }
%   \uncover<+->{
%     El hexágono circunscrito está formado por triángulos equiláteros
%     de altura \(r\),
%     de lado \(2 r / \sqrt{3}\),
%     dando \(\pi < 6 / \sqrt{3} = 3,4610\).
%   }
%   \uncover<+->{
%     Promediando ambos valores obtenemos \(3,3221\).
%   }

%   \end{frame}

%   Método de Newton (con serie binomial, etc)
%   Series de Machin
%   Ramanujan (?)
%   Spigot series

\end{document}

%%% Local Variables:
%%% mode: latex
%%% TeX-master: t
%%% ispell-local-dictionary: "spanish"
%%% End:

% LocalWords:  glyphtounicode INF subindustria
