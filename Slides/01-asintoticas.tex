\input glyphtounicode%
\pdfgentounicode=1
\documentclass[english, spanish, fleqn,%
hyperref = {colorlinks, urlcolor = blue}%
%, handout%
]{beamer}
%\usepackage{beamerarticle}

\usepackage{ifpdf}

\usepackage{beamerthemesplit}

\usepackage{fourier}
\usepackage{amsmath}
\usepackage{amsthm}
\usepackage{siunitx}
\usepackage{babel}
\usepackage{csquotes}
\mode<article>{\usepackage[colorlinks, urlcolor = blue]{hyperref}}

\ifdefined\HCode
  \def\pgfsysdriver{pgfsys-dvisvgm4ht.def}
\fi

\usefonttheme{professionalfonts}

\beamerdefaultoverlayspecification{<+->}

\title{Asintóticas}

\author[Horst H. von Brand]{Horst H. von Brand\\
  \href{mailto:vonbrand@inf.utfsm.cl}{vonbrand@inf.utfsm.cl}}

\institute[DI UTFSM]{Departamento de Informática\\
                     Universidad Técnica Federico Santa María}
\date{}

\begin{document}
\frame{\maketitle}

\begin{frame}<handout>
  \frametitle{Contenido}

  \tableofcontents
\end{frame}

\section{Introducción}

\begin{frame}
  \frametitle{Justificación}

  \uncover<+->{
    En cálculos numéricos,
    interesa obtener aproximaciones precisas a los valores buscados.
    O sea,
    acotar el error cometido en la aproximación numérica.
  }

  \uncover<+->{
    Al evaluar algoritmos,
    interesa estimar el uso de recursos
    (tiempo,
     memoria,
     tráfico de red,
     \ldots)
    en términos del \textquote{tamaño} del problema a resolver.
  }

  \uncover<+->{
    En estas
    (y muchas otras)
    aplicaciones no interesa el valor exacto
    (si lo conociéramos,
     el ejercicio mismo no tiene sentido)
    interesa dar estimaciones o cotas simples
    suficientemente precisas.
  }
  \uncover<+->{
    Herramienta fundamental son las \emph{estimaciones asintóticas}.
  }
\end{frame}

\section{Definiciones}

\begin{frame}
  \frametitle{Notaciones asintóticas}

  Hay varias notaciones,
  con definiciones a veces bastante diferentes.
  Adoptaremos las de
  \href{https://faculty.math.illinois.edu/~hildebr/lecture-notes/asymptotics.pdf}
       {Hildebrand},
  a su vez afines a las de \href{https://danluu.com/knuth-big-o.pdf}{Knuth}.
\end{frame}

\subsection{\emph{Big-Oh}}

\begin{frame}
  \frametitle{\emph{Big Oh}}

  La notación más común es \emph{\foreignlanguage{english}{big Oh}}
  (O mayúscula),
  que primero introduciremos en su uso más común,
  comportamiento de funciones de una variable real
  (o entera)
  cuando tiende a infinito.
\end{frame}

\begin{frame}
  \setcounter{beamerpauses}{2}
  \frametitle{\emph{Big Oh} --- definición \(x\) grande}

   \uncover<+->{
     Dadas funciones \(f(x)\), \(g(x)\)
     definidas para todo \(x\) suficientemente grande
     decimos que:
     \begin{equation*}
       f(x)
         = O(g(x))
     \end{equation*}
     si hay constantes \(c > 0\) y \(x_0\) tales que:
     \begin{equation*}
       \lvert f(x) \rvert
         \le c \, \lvert g(x) \rvert  \qquad (x \ge x_0)
     \end{equation*}
   }
   \uncover<+->{
     Normalmente el valor de \(x_0\) no es importante,
     de ser relevante anotaremos \((x \ge x_0)\).
   }

   \uncover<+->{
     Para enfatizar que \(c\) o \(x_0\) dependen de algún parámetro \(\alpha\),
     anotamos \(O_\alpha(\cdot)\).
   }
\end{frame}

\begin{frame}
  \setcounter{beamerpauses}{2}
  \frametitle{\emph{Big Oh} --- definición general}

   \uncover<+->{
     Dadas funciones \(f(x)\), \(g(x)\)
     definidas para todo \(x\) relevante,
     decimos que:
     \begin{equation*}
       f(x)
         = O(g(x))
         ¸\qquad (x_0 \le x \le x_1)
     \end{equation*}
     si hay una constante \(c > 0\) tal que:
     \begin{equation*}
       \lvert f(x) \rvert
         \le c \, \lvert g(x) \rvert  \qquad (x_0 \le x \le x_1)
     \end{equation*}
   }

  \uncover<+->{
     Esta es una \emph{estimación \(O\)},
     decimos \emph{\(f\) es de orden \(g\)},
     \(c\) es la \emph{constante-\(O\)},
     \(x_0 \le x \le x_1\) es el \emph{rango de validez} de la estimación.
   }
\end{frame}

\begin{frame}
  \setcounter{beamerpauses}{2}
  \frametitle{\emph{Big Oh} ---comentarios}

  \uncover<+->{
    Un caso común es que estemos interesados en un entorno de \num{0},
    \(\lvert x \rvert \le \varepsilon\) para algún \(\varepsilon\)
    que no interesa especificar más.
  }

  \uncover<+->{
    No daremos intervalos explícitamente más que en casos especiales.
    Las situaciones comunes son que estemos interesados en \(x \to \infty\)
    o \(x \to 0\),
    quedará claro del contexto de cuál estamos hablando.
  }
\end{frame}

\begin{frame}
  \setcounter{beamerpauses}{2}
  \frametitle{\emph{big-Oh} --- comentarios}

  \uncover<+->{
    Generalmente el valor de \(c\) no importa.
    En muchos casos podemos calcular un valor explícito,
    aunque es muy tedioso incluso si no buscamos el mejor valor posible.
    Muchas veces puede deducirse su existencia de consideraciones de continuidad
    y similares.
  }

  \uncover<+->{
    El punto de esta notación
    --- y de las otras notaciones asintóticas ---
    es que permite expresar en forma simple y sugestiva,
    que la desigualdad se cumple sin tener que exhibirla en detalle.
  }
\end{frame}

\begin{frame}
  \setcounter{beamerpauses}{2}
  \frametitle{\emph{big-Oh} --- ecuaciones}

  \uncover<+->{
    En expresiones interpretamos \(O(g(x))\) como
    \textquote{reemplaza una función que cumple la estimación-O}.
  }

  \uncover<+->{
    Por ejemplo:
    \begin{equation*}
      \ln (1 + x)
         = x + O(x^2) \qquad (\lvert x \rvert \le 1/2)
    \end{equation*}
    significa:

    \begin{displayquote}
      \(\ln(1 + x) = x + f(x)\),
      donde \(f(x)\) satisface \(\lvert f(x) \rvert \le c \, x^2\)
      para alguna constante \(c\) siempre que \(\lvert x \rvert \le 1/2\)
    \end{displayquote}
  }
\end{frame}

\begin{frame}
  \setcounter{beamerpauses}{2}
  \frametitle{\emph{big-Oh} --- interpretación}

  \uncover<+->{
    La igualdad es notación cómoda,
    pero fundamentalmente incorrecta.
  }
  \uncover<+->{
    Lo correcto es interpretar expresiones que contienen \(O(\cdot)\)
    como un conjunto de funciones,
    y el signo igual
    realmente dice que el conjunto de funciones a la izquierda
    es un subconjunto del conjunto del lado derecho.
  }
  \uncover<+->{
    Si no hay \(O(\cdot)\),
    es simplemente el conjunto de esa única función.
  }

  \uncover<+->{
    La notación de igualdad es demasiado cómoda y ampliamente usada;
    han habido heroicos esfuerzos de cambiarla por algo más descriptivo,
    sin real efecto.
  }
\end{frame}

\begin{frame}
  \setcounter{beamerpauses}{2}
  \frametitle{\emph{big-Oh} --- advertencias}

  \uncover<+->{
    La \textquote{igualdad} así definida
    ni siquiera es una relación de equivalencia.
  }

  \uncover<+->{
    Es reflexiva y transitiva,
    pero no simétrica.
  }
  \uncover<+->{
    Por ejemplo,
    de \(x = O(x^2)\) no podemos concluir que \(x^2 = O(x)\).
  }
\end{frame}

\subsection{Otras notaciones}

\begin{frame}
  \setcounter{beamerpauses}{2}
  \frametitle{Otras notaciones}

  \uncover<+->{
    Hay una variedad de notaciones adicionales,
    algo menos usadas.
    Nuevamente,
    normalmente se subentiende algún rango \(x \ge x_0\).
  }
  \uncover<+->{
    Las advertencias sobre \(O(\cdot)\) son aplicables a ellas también.
  }

  \uncover<+->{
    \begin{equation*}
      f(x)
        = \Omega(g(x))
        \qquad \text{Hay \(c > 0\)
                     tal que
                     \(\lvert f(x) \rvert \ge c \, \lvert g(x) \rvert\)}
    \end{equation*}
    \(f(x)\) es de orden al menos \(g(x)\)
  }
  \uncover<+->{
    \begin{equation*}
      f(x)
        = \Theta(g(x))
        \qquad \text{\(f(x) = O(g(x))\) y \(f(x) = \Omega(g(x))\)}
    \end{equation*}
    \(f(x)\) es del mismo orden que \(g(x)\)
  }
\end{frame}

\begin{frame}
  \setcounter{beamerpauses}{2}
  \frametitle{Otras notaciones (cont)}

  \uncover<+->{
    Hay otras notaciones,
    usadas comúnmente para situaciones
    en que interesa \(x\) en un entorno de \(a\)
    (frecuentemente \(a = 0\),
     aunque también \(a = \infty\)).
  }
  \uncover<+->{
    \begin{equation*}
      f(x)
        = o(g(x))
        \qquad \lim_{x \to a} \frac{f(x)}{g(x)} = 0
    \end{equation*}
    \(f(x)\) es de menor orden que \(g(x)\)
  }
  \uncover<+->{
    \begin{equation*}
      f(x)
        = \omega(g(x))
        \qquad \lim_{x \to a} \frac{f(x)}{g(x)} = \pm\infty
    \end{equation*}
    \(f(x)\) es de mayor orden que \(g(x)\)
  }
\end{frame}

\begin{frame}
  \setcounter{beamerpauses}{2}
  \frametitle{Otras notaciones (cont)}

  \uncover<+->{
    \begin{equation*}
      f(x)
        \sim g(x)
        \qquad \lim_{x \to a} \frac{f(x)}{g(x)} = 1
    \end{equation*}
    \(f(x)\) es asintóticamente
    (o asintóticamente equivalente a) \(g(x)\)
  }
\end{frame}

\begin{frame}
  \setcounter{beamerpauses}{2}
  \frametitle{Serie de Taylor}

  \uncover<+->{
    Herramienta fundamental es la serie de Taylor centrada en \(a\):
    \begin{equation*}
      f(x)
        = \sum_{0 \le k \le n} \frac{1}{k!} f^{(k)}(a) (x - a)^k
           + R_{n + 1}(x)
    \end{equation*}
  }
  \uncover<+->{
    La forma de Lagrange del residuo es:
    \begin{equation*}
      R_{n + 1}(x)
        = \frac{1}{(n + 1)!} f^{(n + 1)}(\xi) (x - a)^{n + 1}
    \end{equation*}
    con \(\xi\) entre \(a\) y \(x\).
  }
\end{frame}

\begin{frame}
  \setcounter{beamerpauses}{2}
  \frametitle{Serie de Taylor}

  \uncover<+->{
    Note que para todo \(x \in [a, b]\):
  }
  \begin{align*}
    \uncover<+->{
      R_{n + 1}(x)
        &=   \frac{f^{(n + 1)}(\xi)}{(n + 1)!} (x - a)^{n + 1} \\
    }
    \uncover<+->{
     \rvert  R_{n + 1}(x) \rvert
        &\le \frac{1}{(n + 1)!} \max_{\xi \in [a, b]} \{ f^{(n + 1)}(\xi) \}
                \cdot \lvert x - a \rvert^{n + 1} \\
    }
    \uncover<+->{
      \intertext{Esto es:}
      R_{n + 1}(x)
        &= O\left( (x - a)^{n + 1} \right)
          \quad(x \in [a, b])
    }
  \end{align*}
\end{frame}

\subsection{Ejemplos}

\begin{frame}
  \setcounter{beamerpauses}{2}
  \frametitle{Ejemplos}

  \begin{description}
  \item[\boldmath\(x = O(\mathrm{e}^x)\)\unboldmath:]
    \uncover<+->{
      Consideremos \(x \mathrm{e}^{-x}\),
      con derivada \(\mathrm{e}^{-x} (1 - x)\)
      (negativa para \(x > 1\),
      positiva para \(x < 1\) y cero únicamente para \(x = 1\)).
      O sea,
      \(x \mathrm{e}^{-x}\) tiene un máximo para \(x = 1\),
      con lo que para todo \(x \in \mathbb{R}\)
      y todo \(c \ge \mathrm{e}^{-1}\):
      \begin{equation*}
        x
          \le c \, \mathrm{e}^x
      \end{equation*}
    }

    \uncover<+->{
      Otra forma de verlo:
      \(x \mathrm{e}^{-x}\) es continua en todo \(\mathbb{R}\),
      negativa para \(x\) negativo,
      tiende a cero para \(x \to a\),
      por lo que es acotada en todo \(\mathbb{R}\).
    }
  \end{description}
\end{frame}

\begin{frame}
  \setcounter{beamerpauses}{2}
  \frametitle{Ejemplos}

  \begin{description}
  \item[\boldmath\(\ln (1 + x) = x - x^2 / 2 + O(x^3)
            \quad (\lvert x \rvert \le 1/2)\)\unboldmath:]
    \uncover<+->{
      Conocemos la serie de Taylor,
      válida para \(\lvert x \rvert < 1\):
      \begin{equation*}
        \ln (1 + x)
          = \sum_{n \ge 1} \frac{(-1)^{n + 1} x^n}{n}
      \end{equation*}
    }
    \uncover<+->{
      En el rango \(\lvert x \rvert \le 1/2\) indicado:
      \begin{align*}
        \left\lvert \ln (1 + x) - x  + \frac{x^2}{2} \right\rvert
          &\le \sum_{n \ge 3} \frac{\lvert x \rvert^n}{n} \\
          &<   \lvert x^3 \rvert
                  \frac{1}{3} \sum_{n \ge 0}\lvert x \rvert^n
          &\le	 \lvert x^3 \rvert
                   \cdot \frac{1}{3} \sum_{n \ge 0} 2^{-n} \\
          &=   \frac{2}{3} \lvert x^3 \rvert
      \end{align*}
    }
  \end{description}
\end{frame}

\subsection{Fórmulas útiles}

\begin{frame}
  \frametitle{Algunas fórmulas útiles}

  Acá \(\alpha \in \mathbb{R}\)
  y \(C > 0\).
  La constante \(1/2\) en los rangos \(\lvert z \rvert \le 1/2\)
  puede reemplazarse por \(c < 1\).
  \begin{alignat*}{3}
    \frac{1}{1 + z}
       &= 1 + O(\lvert z \rvert)
       &&\quad& (\lvert z \rvert \le 1/2) \\
    (1 + z)^\alpha
       &= 1 + O_\alpha(\lvert z \rvert)
       &&& (\lvert z \rvert \le 1/2) \\
    (1 + z)^\alpha
       &= 1 + \alpha z + O_\alpha(\lvert z \rvert^2)
       &&& (\lvert z \rvert \le 1/2) \\
    \ln (1 + z)
       &= O(\lvert z \rvert)
       &&& (\lvert z \rvert \le 1/2) \\
    \ln (1 + z)
       &= z + O(\lvert z \rvert^2)
       &&& (\lvert z \rvert \le 1/2) \\
    \ln (1 + z)
       &= z - \frac{z^2}{2} + O(\lvert z \rvert^3)
       &&& (\lvert z \rvert \le 1/2)
  \end{alignat*}
\end{frame}

\begin{frame}
  \frametitle{Algunas fórmulas útiles (cont)}

  \begin{alignat*}{3}
    \mathrm{e}^z
       &= 1 + O_C (\lvert z \rvert)
       &&\qquad& (\lvert z \rvert \le C) \\
    \mathrm{e}^z
       &= 1 + z + O_C (\lvert z \rvert^2)
       &&& (\lvert z \rvert \le C) \\
    \mathrm{e}^z
       &= 1 + z + \frac{z^2}{2} + O_C (\lvert z \rvert^3)
       &&& (\lvert z \rvert \le C)
  \end{alignat*}
\end{frame}

\subsection{Propiedades}

\begin{frame}
  \setcounter{beamerpauses}{2}
  \frametitle{Propiedades}

  \begin{description}
  \item[Constantes:]
    Si \(C > 0\),
    \(f(x) = O(C g(x))\) es equivalente a \(f(x) = O(g(x))\).
  \item[Absorber términos:]
    Para \(c\) constante y \(g(x) = O(h(x))\),
    \(f(x) = c g(x) + O(h(x))\)
    es equivalente a \(f(x) = O(h(x))\).
  \item[Transitividad:]
    Si \(f(x) = O(g(x))\) y \(g(x) = O(h(x))\)
    entonces \(f(x) = O(h(x))\).
  \item[Sumas y productos:]
    Si \(f_1(x) = O(g_1(x))\) y \(f_2(x) = O(g_2(x))\),
    entonces
     \(f_1(x) + f_2(x) = O(\lvert g_1(x) \rvert + \lvert g_2(x) \rvert)\)
    y
     \(f_1(x) \cdot f_2(x) = O(\lvert g_1(x) g_2(x) \rvert)\).
  \end{description}
\end{frame}

\begin{frame}
  \setcounter{beamerpauses}{2}
  \frametitle{Propiedades}

  \begin{description}
  \item[Diferencias:]
    Si \(f_1(x) = O(g_1(x))\) y \(f_2(x) = O(g_2(x))\),
    entonces
    \(f_1(x) - f_2(x) = O(\lvert g_1(x) \rvert + \lvert g_2(x) \rvert)\).

    Términos \(O(\cdot)\) \emph{nunca} se cancelan
    (representan funciones posiblemente diferentes).
  \end{description}
\end{frame}

\begin{frame}
  \setcounter{beamerpauses}{2}
  \frametitle{Propiedades}

  \begin{description}
  \item[Sumas de un número arbitrario de términos:]
    Se extiende a un número arbitrario
    o infinito de términos \(O\)
    si las estimaciones valen \emph{uniformemente},
    O sea,
    si para todo \(n\) vale \(f_n(x) \le c g_n(x)\) si \(x \ge x_0\),
    entonces también:
    \begin{align*}
      \sum_n f_n(x)
        &= \sum_n O(\lvert g_n(x) \rvert) \\
        &= O\left( \sum_n \lvert g_n(x) \rvert \right)
    \end{align*}
  \end{description}
\end{frame}

\begin{frame}
  \setcounter{beamerpauses}{2}
  \frametitle{Propiedades}

  \begin{description}
  \item[Distribuyendo términos:]
    Si \(f_1(x) = O(g_1(x))\) y \(f_2(x) = O(g_2(x))\),
    para cualquier \(h(x)\) es
    \(h(x) (f_1(x) + f_2(x))
        = O(\lvert h(x) \rvert
             \cdot (\lvert g_1(x) \rvert + \lvert g_2(x) \rvert))\).
  \item[Extraer factores:]
    Si \(f(x) = O(g(x) h(x))\) entonces \(f(x) = g(x) O(h(x))\).
  \end{description}
\end{frame}

\subsection{Simplificar}

\begin{frame}
  \setcounter{beamerpauses}{2}
  \frametitle{Simplificaciones}

  En lo que sigue,
  nos interesa el caso en que \(\phi(x) \to 0\).
  \begin{align*}
    \uncover<+->{\frac{1}{1 + O(\phi(x))} = 1 + O(\phi(x)) \\}
    \uncover<+->{(1 + O(\phi(x)))^p = 1 + O_p(\phi(x)) \\}
    \uncover<+->{\ln (1 + O(\phi(x))) = O(\phi(x)) \\}
    \uncover<+->{\exp (O(\phi(x))) = 1 + O(\phi(x))}
  \end{align*}
  \uncover<+->{
    Estas se demuestran fácilmente de las series de Taylor respectivas.
  }
\end{frame}

\begin{frame}
  \frametitle{Trucos}

  \begin{description}
  \item[Logaritmos:]
    Ayudan a simplificar productos y potencias.
    Luego exponenciar.
  \item[Factor principal:]
    Extraerlo,
    y estimar el resto.
  \item[Término máximo:]
    Estimar una suma como el número de términos por el mayor de ellos.
    El máximo término da una cota inferior útil.
  \end{description}
\end{frame}

\section{Ejemplos}

\begin{frame}
  \setcounter{beamerpauses}{2}
  \frametitle{Ejemplos}

  \uncover<+->{
    Estimar \(\mathrm{e}^{-x} / (1 - x)\)
    para \(x\) pequeño.
  }
  \\ \medskip
  \uncover<+->{
    Series de Taylor de las partes:
  }
  \begin{align*}
    \uncover<+->{
      \frac{\mathrm{e}^{-x}}{1 - x}
        &= \left(1 - x + \frac{x^2}{2} + O(x^3)\right)
             \cdot \left(1 + x + x^2 + O(x^3)\right) \\
    }
    \uncover<+->{
        &= 1 + \frac{x^2}{2} + O(x^3)
    }
  \end{align*}
\end{frame}

\begin{frame}
  \setcounter{beamerpauses}{2}
  \frametitle{Ejemplos}

  Estimar \((a_1 + a_2 + \dotsb + a_n)^p\) para \(p \ge 1\)
  si los  \(a_k > 0\).
  Note que acá \(n\) y \(p\) son constantes.
  \\[3ex]
  \uncover<+->{
    Resulta \((a_1 + a_2 + \dotsb + a_n)^p = \Theta(a_1^p + \dotsb + a_n^p)\)
  }
  \\[1ex]
  \uncover<+->{
    Debemos demostrar cotas
    superior (\(O(\cdot)\)) e inferior (\(\Omega(\cdot)\)).
  }
\end{frame}

\begin{frame}
  \setcounter{beamerpauses}{2}
  \frametitle{Ejemplos}

  \uncover<+->{
    Cota superior:
    \begin{equation*}
      \left( \sum_{1 \le k \le n} a_k \right)^p
        \le \left( n \max_{1 \le k \le n}\{  a_k \} \right)^p
        = n^p \left( \max_{1 \le k \le n} \{  a_k^p \} \right)
        \le n^p \left( \sum_{1 \le k \le n} a_k^p \right)
    \end{equation*}
    Por las suposiciones,
    \(n^p\) es constante.
  }
\end{frame}

\begin{frame}
  \setcounter{beamerpauses}{2}
  \frametitle{Ejemplos}

  \uncover<+->{
    Cota inferior:
    \begin{equation*}
      \sum_{1 \le k \le n} a_k^p
        \le n \max_{1 \le k \le n} \{ a_k^p \}
        \le n \left( \sum_{0 \le k \le n} a_k \right)^p
    \end{equation*}
  }
  \uncover<+->{
    O sea:
    \begin{equation*}
      \left( \sum_{0 \le k \le n} a_k \right)^p
        \le \frac{1}{n} \sum_{1 \le k \le n} a_k^p
    \end{equation*}
  }
\end{frame}

\begin{frame}
  \setcounter{beamerpauses}{2}
  \frametitle{Ejemplos}

  Determine el comportamiento de \(f(x) = \sqrt{x^2 + 1}\)
  cuando \(x \to \infty\).

  \uncover<+->{
    Como \(x > 0\),
    es \(\sqrt{x^2} = x\).
    Bajo la raíz, domina \(x^2\),
    esperamos \(f(x)\) cercano a \(\sqrt{x^2} = x\).
  }

  \begin{align*}
    \uncover<+->{f(x) &= \sqrt{x^2 + 1} \\}
    \uncover<+->{     &= \sqrt{x^2 (1 + x^{-2})} \\}
    \uncover<+->{     &= x \sqrt{1 + x^{-2}} \\}
    \uncover<+->{     &= x \left( 1 + O(x^{-2}) \right)}
  \end{align*}
\end{frame}

\begin{frame}
  \setcounter{beamerpauses}{2}
  \frametitle{Ejemplos}

  ¿Qué tan buena es la estimación siguiente
  (promediar diferencias \(x \pm \delta\))?
  \begin{equation*}
    f'(x)
      = \frac{f(x + \delta) - f(x - \delta)}{2 \delta}
   \end{equation*}

   \uncover<+->{
     Expandimos \(f\) por Taylor alrededor de \(x\):
     \begin{equation*}
       f(x + h)
         = f(x) + f'(x) h + \frac{f''(x)}{2} h^2 + \frac{f'''(x) h^3}{6}
              + O(h^4)
    \end{equation*}
  }
  \uncover<+->{En consecuencia:}
  \begin{align*}
    \uncover<+->{
      \frac{f(x + \delta) + f(x - \delta)}{2 \delta}
        &= \frac{2 f'(x) \delta + f'''(x) \delta^3 / 3
                  + O(\delta^4)}
                {2 \delta} \\
    }
    \uncover<+->{
       &= f'(x) + \frac{f'''(x)}{6} \delta^2
                  + O(\delta^3)
    }
  \end{align*}
\end{frame}

\section{Resumen}

\begin{frame}
  \setcounter{beamerpauses}{2}
  \frametitle{Resumen}

  \begin{itemize}
  \item
    Definición de notaciones asintóticas
  \item
    Algunos ejemplos.
    Para muchos más,
    y detalles adicionales,
    vea el el apunte,
    las notas de
    \href{https://faculty.math.illinois.edu/~hildebr/lecture-notes/asymptotics.pdf}
         {Hildebrand}
    o el \href{https://aofa.cs.princeton.edu/home}
              {texto de Sedgewick y Flajolet}.
  \item
    Soltura al interpretar y manipular este tipo de expresiones
    es importante en mucho de lo que sigue del curso.
  \item
    Usaremos resultados del cálculo con frecuencia,
    para un recordatorio vea el apéndice del apunte.
  \end{itemize}
\end{frame}
\end{document}

%%% Local Variables:
%%% mode: latex
%%% TeX-master: t
%%% ispell-local-dictionary: "spanish"
%%% End:

% LocalWords:  presentation Big english big deducirse cont ex handout
% LocalWords:  asintóticamente exponenciar glyphtounicode
