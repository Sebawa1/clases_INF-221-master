\input glyphtounicode%
\pdfgentounicode=1
\documentclass[english, spanish, fleqn,%
hyperref = {colorlinks, urlcolor = blue}%
%, handout%
]{beamer}

%% ]{article}
%% \usepackage{beamerarticle}

\usepackage{beamerthemesplit}

\usepackage[es-noquoting]{babel}
\usepackage{csquotes}
\usepackage{fourier}
\usepackage[group-digits = false, copy-decimal-marker = true]{siunitx}
\usepackage{amsmath}
\usepackage{amsthm}
\usepackage{enumitem}
\usepackage[noline, noend]{algorithm2e}
\usepackage{listings}
\usepackage{chessboard}

\usefonttheme{professionalfonts}

%%%
%%% For listings
%%%

\lstloadlanguages{Python}


\beamerdefaultoverlayspecification{<+->}

\title{Backtracking}

\author[Horst H. von Brand]{Horst H. von Brand\\
  \href{mailto:vonbrand@inf.utfsm.cl}{vonbrand@inf.utfsm.cl}}

\institute[DI UTFSM]{Departamento de Informática\\
                     Universidad Técnica Federico Santa María}
\date{}

\begin{document}
%%%
%%% For listings
%%%

\lstset{basicstyle = \sffamily, commentstyle = \slshape,
        frame = none, numbers = none,
        tabsize = 4,
        showstringspaces = false,
        xleftmargin = 1em, xrightmargin = 1em
       }
\renewcommand{\lstlistingname}{Listado}

%%%
%%% For algorithm2e
%%%

\SetAlgorithmName{Algoritmo}{Algoritmo}{Índice de algoritmos}

\SetAlCapSty{mdseries}
\SetKwProg{Function}{function}{}{end}
\SetKwProg{Procedure}{procedure}{}{end}
\SetKw{Variables}{variables}
\SetKw{Downto}{downto}
\SetKwBlock{Loop}{loop}{end}
\SetKw{Continue}{continue}
\SetKw{Break}{break}
\SetKw{KwStep}{step}

\frame{\maketitle}

\begin{frame}<handout>
  \frametitle{Contenido}

  \tableofcontents
\end{frame}

\section{Situación general}

\begin{frame}
  \setcounter{beamerpauses}{2}
  \frametitle{El problema}

  \uncover<+->{
    Es común buscar un objeto que cumple ciertas condiciones.
    A falta de una idea mejor,
    podemos revisar todas las configuraciones posibles.
  }
  \uncover<+->{
    Requerimos entonces alguna forma de generarlas sistemáticamente.
  }
  \uncover<+->{
    En muchos casos podemos irlas generando incrementalmente,
    y abandonar si vemos que el camino actual no puede completarse al éxito.
  }
  \\ \medskip
  \uncover<+->{
    Hacer lo anterior recursivamente
    se llama \emph{\foreignlanguage{english}{backtracking}}
    (puede traducirse en \textquote{avanzar y retroceder},
     \textquote{andar y desandar}):
    se avanza un paso,
    si estamos en un camino sin salida
    retrocedemos para intentar la siguiente opción.
  }
\end{frame}

\begin{frame}
  \setcounter{beamerpauses}{2}
  \frametitle{Consideraciones}

  \uncover<+->{
    Si tenemos \(d\) puntos en que debemos decidir,
    cada uno de ellos con \(m\) opciones,
    revisar todas las posibilidades es \(m^d\)~casos.
  }
  \uncover<+->{
    Incluso para valores moderados de \(d\) y \(m\)
    esto es un número astronómico
    (es crecimiento exponencial en \(d\)).
  }
  \uncover<+->{
    Es imperativo disminuir opciones
    (\emph{podar} ramas del árbol de búsqueda)
     para lograr tiempos de ejecución razonables.
   }
\end{frame}

\begin{frame}
  \setcounter{beamerpauses}{2}
  \frametitle{El algoritmo general}

  \uncover<+->{
    Es un recorrido en un árbol de objetos construidos a medias,
    con un único objeto parcial activo.
    Corresponde a búsqueda en profundidad,
    efectuada naturalmente mediante recursión
    (aunque podríamos usar búsqueda en profundidad no recursiva
     de ser necesario).
  }
\end{frame}

\begin{frame}
  \frametitle{El algoritmo general}

    \begin{algorithm}[H]
      \DontPrintSemicolon

      \tcc*[l]{Set up basic object} \;

      \Function{\(\mathrm{backtrack}()\)}{
        \eIf{\(\mathrm{object}\) is complete}{
          Process \(\mathrm{object}\) \;
        }
        {
          \(S \leftarrow \text{set of possible next steps}\) \;
          \ForEach{\(s \in S\)}{
            Do step \(s\) on \(\mathrm{object}\) \;
            \(\mathrm{backtrack}()\) \;
            Undo step \(s\) on \(\mathrm{object}\) \;
          }
        }
      }
    \end{algorithm}
\end{frame}

\section{Esquema del algoritmo}

\begin{frame}
  \setcounter{beamerpauses}{2}
  \frametitle{El algoritmo general}

  \uncover<+->{
    El orden de los pasos puede estar dado por el problema,
    o ser completamente artificial.
  }
  \uncover<+->{
    Obviamente es central evitar repeticiones.
  }
  \\ \medskip
  \uncover<+->{
    Si solo buscamos una solución
    (o solo interesa saber si hay soluciones),
    cortamos la recursión temprano.
  }
  \uncover<+->{
    El algoritmo no especifica el orden en que se intentan elementos de \(S\).
    Si solo interesa una solución
    (o buscamos algún óptimo),
    podemos extraerlos en algún orden prometedor.
  }
\end{frame}

\begin{frame}
  \setcounter{beamerpauses}{2}
  \frametitle{Receta para \emph{backtracking}}

  \uncover<+->{
    Resolver un problema usando \emph{\foreignlanguage{english}{backtracking}}
    significa efectuar los siguientes pasos:
  }
  \begin{enumerate}[font = \textbf, label = {(\alph*)}]
  \item
    Definir los pasos de la construcción del objeto buscado.
  \item
    Diseñar la estructura de datos para objetos parcialmente construidos.
  \item
    Definir cómo determinar si se ha llegado a destino.
  \item
    Diseñar estructuras auxiliares
    que ayuden a determinar pasos siguientes.
  \item
    Definir cómo hacer y deshacer modificaciones.
  \item
    Definir criterios de abortar la búsqueda.
  \item
    Definir qué hacer con objetos completos.
  \end{enumerate}
\end{frame}

\section{Ejemplos}

\subsection{Generar permutaciones}

\begin{frame}
  \setcounter{beamerpauses}{2}
  \frametitle{Generar permutaciones}

  \uncover<+->{
    En Python,
    realmente no hay nada que hacer.
    La biblioteca \lstinline!itertools! las entrega directamente:

    \lstinputlisting[firstline = 3, lastline = 6]{permutations-library.py}
  }
  \bigskip
  \uncover<+->{
    Igual interesa ver cómo hacerlo.
  }
\end{frame}

\begin{frame}
  \setcounter{beamerpauses}{2}
  \frametitle{Generar permutaciones}

  \uncover<+->{
    En nuestro esquema,
    si se han elegido \(a_1, \dotsc, a_k\) para las primeras posiciones,
    para \(a_{k + 1}\) podemos elegir cualquiera de las aún no elegidas.
  }
  \uncover<+->{
    Es natural representar el conjunto de elementos disponibles
    como un arreglo de \lstinline!boolean!.
  }
\end{frame}

\begin{frame}
  \frametitle{Generar permutaciones}
  \framesubtitle{Definición de variables}

  \lstinputlisting[firstline = 3, lastline = 7]{permutations.py}
\end{frame}

\begin{frame}
  \frametitle{Generar permutaciones}
  \framesubtitle{Funciones auxiliares: ubicar, retractar}

  \lstinputlisting[firstline = 9, lastline = 14]{permutations.py}
\end{frame}

\begin{frame}
  \frametitle{Generar permutaciones}
  \framesubtitle{Funciones auxiliares: mostrar solución}

  \lstinputlisting[firstline = 16, lastline = 19]{permutations.py}
\end{frame}

\begin{frame}
  \frametitle{Generar permutaciones}
  \framesubtitle{Función central}

  \lstinputlisting[firstline = 21, lastline = 29]{permutations.py}
\end{frame}

\begin{frame}
  \frametitle{Generar permutaciones}
  \framesubtitle{Programa principal}

  \lstinputlisting[firstline = 31, lastline = 31]{permutations.py}
\end{frame}

\subsection{Rotulaciones graciosas}

\begin{frame}
  \setcounter{beamerpauses}{2}
  \frametitle{Rotulación graciosa de un grafo}

  \uncover<+->{
    Rotulamos los vértices de un grafo de \(n + 1\)~vértices
    con \(0, \dotsc, n\) sin repeticiones,
    y el arco \(u v\)
    con la diferencia absoluta entre los rótulos de \(u\) y \(v\).
    Si cada diferencia entre \(1\) y \(n\) aparece exactamente una vez,
    se llama una \emph{rotulación graciosa}.
  }
  \\ \bigskip
  \uncover<+->{
    Es claro que entre grafos conexos solo árboles
    pueden tener rotulaciones graciosas
    (un árbol de \(n + 1\)~vértices tiene \(n\)~arcos).
  }
  \uncover<+->{
    Un problema abierto es si todo árbol admite una rotulación graciosa.
  }
  \\ \medskip
  \uncover<+->{
    Un camino de largo \(n\) siempre puede rotularse graciosamente
    como (\(n, 0, n - 1, 1, n - 2, 2, \dotsc, \lceil n / 2 \rceil\)).
    Interesa cuántas rotulaciones graciosas tiene.
  }
\end{frame}

\begin{frame}
  \setcounter{beamerpauses}{2}
  \frametitle{Generar y filtrar}

  \uncover<+->{
    Una opción es \emph{generar y filtrar}:
    las rotulaciones graciosas son permutaciones de \(0, \dotsc, n\)
    con la condición adicional sobre diferencias
    entre elementos vecinos.
  }
\end{frame}

\begin{frame}
  \setcounter{beamerpauses}{2}
  \frametitle{Generar y filtrar}
  \framesubtitle{Preliminares}

  \lstinputlisting[firstline = 4, lastline = 6]{graceful-generate-filter.py}
\end{frame}

\begin{frame}
  \setcounter{beamerpauses}{2}
  \frametitle{Generar y filtrar}
  \framesubtitle{Filtrar}

  \lstinputlisting[firstline = 8, lastline = 16]{graceful-generate-filter.py}
\end{frame}

\begin{frame}
  \setcounter{beamerpauses}{2}
  \frametitle{Generar y filtrar}
  \framesubtitle{Programa principal}

  \lstinputlisting[firstline = 21, gobble = 4]{graceful-generate-filter.py}
\end{frame}

\begin{frame}
  \setcounter{beamerpauses}{2}
  \frametitle{Resultados}

  \uncover<+->{
    El programa precedente
    dice que hay \num{648} rotulaciones graciosas de \(P_{10}\).
  }
  \uncover<+->{
    Hay un total
    de \(11! = \num[group-digits = true]{39 916 800}\)~permutaciones
    de \(0, \dotsc, 10\),
    las rotulaciones graciosas son una fracción mínima.
  }
  \uncover<+->{
    El tiempo para generar todas las permutaciones,
    solo para descartar la inmensa mayoría,
    es enorme.
  }
\end{frame}

\begin{frame}
  \setcounter{beamerpauses}{2}
  \frametitle{Planteo vía
                   \emph{\foreignlanguage{english}{backtracking}}}

  \uncover<+->{
    Las permutaciones que comienzan (\(0, 1, 2, \dotsc\))
    no tienen posibilidad alguna
    (ya se repite la diferencia \(1\)),
    son una gran fracción del total.
  }

  \uncover<+->{
    Usaremos nuestro generador de permutaciones,
    podando si alguna diferencia se repite.
  }
\end{frame}

\begin{frame}
  \frametitle{Rotulaciones graciosas}
  \framesubtitle{Definición de variables globales}

  \lstinputlisting[firstline = 5, lastline = 10]{graceful.py}
\end{frame}

\begin{frame}
  \frametitle{Rotulaciones graciosas}
  \framesubtitle{Funciones auxiliares: ubicar, retractar}

  \lstinputlisting[firstline = 12, lastline = 19]{graceful.py}
\end{frame}

\begin{frame}
  \frametitle{Rotulaciones graciosas}
  \framesubtitle{Función central}

  \lstinputlisting[firstline = 21, lastline = 31]{graceful.py}
\end{frame}

\begin{frame}
  \frametitle{Rotulaciones graciosas}
  \framesubtitle{Programa principal}
  \lstinputlisting[firstline = 35, gobble = 4]{graceful.py}
\end{frame}

\begin{frame}
  \setcounter{beamerpauses}{2}
  \frametitle{Comparación}

  \uncover<+->{
    Ambos programas dan \num{648}~rotulaciones graciosas
    para \(P_{10}\).
  }
  \uncover<+->{
    El programa basado en \emph{\foreignlanguage{english}{backtracking}}
    toma \SI{2,045}{\second},
    generar y filtrar toma \SI{98,114}{\second}.
    Y eso que seguramente la generación de permutaciones
    está escrita en C nativo,
    cuidadosamente optimizada.
  }

  \uncover<+->{
    Los tiempos aumentan muy rápidamente con \(n\).
  }
\end{frame}

\subsection{El problema de ocho reinas}

\begin{frame}
  \frametitle{Movidas de una reina en ajedrez}

  La reina es la pieza más poderosa del ajedrez.
  Ataca en vertical,
  horizontal
  y ambas diagonales.

  \setchessboard{setpieces = {Qc5}}
  \begin{center}
    \chessboard[pgfstyle = {[fill]circle},
                padding = -1ex,
                smallboard, showmover = false,
                backfields  = {a5, b5, d5, e5, f5, g5, h5,
                               c1, c2, c3, c4, c6, c7, c8,
                               a3, b4, d6, e7, f8,
                               a7, b6, d4, e3, f2, g1}
               ]
  \end{center}
\end{frame}

\begin{frame}
  \setcounter{beamerpauses}{2}
  \frametitle{El problema de las 8 reinas}

  \uncover<+->{
    Un problema antiguo es el de las \emph{ocho reinas}:
    ¿pueden ubicarse ocho reinas en un tablero de ajedrez
    de forma que ninguna pueda atacar a otra?
  }
  \uncover<+->{
    Es claro que no pueden ser más de ocho,
    puede haber a lo más una por fila.
  }
  \uncover<+->{
    La pregunta obvia siguiente es cuántas soluciones hay.
  }
\end{frame}

\begin{frame}
  \setcounter{beamerpauses}{2}
  \frametitle{Planteo vía
                   \emph{\foreignlanguage{english}{backtracking}}}

  \uncover<+->{
    Es obvio dividir en pasos de ubicar una reina en el tablero.
  }

  \uncover<+->{
    Ubicar la reina en cualquiera de los \(64\)~casilleros
    (o al menos los libres)
    da \(64^{\underline{8}} = 178\,462\,987\,637\,760\)
    alternativas a revisar.
  }
  \uncover<+->{
    Queda de ejercicio calcular cuánto tiempo toma
    suponiendo que nuestro computador
    revisa un millón de posiciones por segundo.
  }
\end{frame}

\begin{frame}
  \setcounter{beamerpauses}{2}
  \frametitle{Planteo vía
                   \emph{\foreignlanguage{english}{backtracking}}}

  \uncover<+->{
    Observando que puede haber a lo más una reina por columna,
    podemos definir ocho etapas,
    una por columna;
    viendo que cada reina ubicada quita al menos la fila en que está
    para las siguientes como opción,
    esto es menos de \(8! = \num[group-digits = true]{40 320}\)~posibilidades.
    Bastante manejable.
  }
  \uncover<+->{
    Debemos diseñar una estructura auxiliar
    que nos permita determinar rápidamente si un casillero está libre
    (no es atacado por las reinas ya posicionadas),
    que sea rápido de actualizar al posicionar una reina
    o retirarla.
  }
\end{frame}

\begin{frame}
  \setcounter{beamerpauses}{2}
  \frametitle{Planteo vía
                   \emph{\foreignlanguage{english}{backtracking}}}

  \uncover<+->{
    Como puede haber a lo más una reina por fila y diagonal,
    es natural tener un \lstinline!boolean!
    para cada una que indica si está libre.
  }
  \\ \smallskip
  \uncover<+->{
    Las diagonales son rectas con pendiente \(\pm 1\).
    Las diagonales que pasan por fila y columna \(r_0, c_0\)
    tienen \(r \pm c\) fijos.
  }
  \\ \smallskip
  \uncover<+->{
    Hay \(8\) filas,
    \(0 \le r \le 7\);
    \(15\) diagonales en subida,
    \(- 7 \le r - c \le 7\);
    \(15\) diagonales en bajada,
    \(0 \le r + c \le 14\).
  }
  \\ \smallskip
  \uncover<+->{
    Usamos el arreglo \lstinline[language = Python]{rfree}
    indexado por \(r\) para las filas libres,
    el arreglo \lstinline[language = Python]{du}
    indexado por \(7 + r - c\) para las diagonales en subida
    y el arreglo \lstinline[language = Python]{dd}
    indexado por \(r + c\) para las diagonales en bajada.
  }
\end{frame}

\begin{frame}
  \frametitle{Solución del problema de 8 reinas}
  \framesubtitle{Definición de variables}

  \lstinputlisting[firstline = 3, lastline = 12]{8queens}
\end{frame}

\begin{frame}
  \frametitle{Solución del problema de 8 reinas}
  \framesubtitle{Funciones auxiliares: ubicar, retractar}

  \lstinputlisting[firstline = 14, lastline = 19]{8queens}
\end{frame}

\begin{frame}
  \frametitle{Solución del problema de 8 reinas}
  \framesubtitle{Funciones auxiliares: mostrar solución}

  \lstinputlisting[firstline = 21, lastline = 28]{8queens}
\end{frame}

\begin{frame}
  \frametitle{Solución del problema de 8 reinas}
  \framesubtitle{Función central}

  \lstinputlisting[firstline = 30, lastline = 41]{8queens}
\end{frame}

\begin{frame}
  \frametitle{Solución del problema de 8 reinas}
  \framesubtitle{Programa principal}

  \lstinputlisting[firstline = 43, lastline = 45]{8queens}
\end{frame}

\begin{frame}
  \frametitle{Soluciones}

  \uncover<+->{
    El programa arroja \(92\) soluciones.
  }

  \uncover<+->{Una de ellas es:
    \begin{center}
      \setchessboard{setpieces = {Qa1, Qb5, Qc8, Qd6,
                                  Qe3, Qf7, Qg2, Qh4}}
      \chessboard[smallboard, showmover = false]
    \end{center}
  }
\end{frame}

\subsection{El algoritmo de Warnsdorf}

\begin{frame}
  \frametitle{Movidas de un caballo en ajedrez}

  El caballo en ajedrez tiene movidas particulares.

  \begin{center}
    \setchessboard{setpieces = {Nd5}}
    \chessboard[pgfstyle = {[fill]circle},
                padding = -1ex,
                smallboard, showmover = false,
                backfields  = {b6, c7, e7, f6, f4, e3, c3, b4}
               ]
  \end{center}
\end{frame}

\begin{frame}
  \setcounter{beamerpauses}{2}
  \frametitle{El algoritmo de Warnsdorf}

  \uncover<+->{
    Un problema muy antiguo es hallar saltos del caballo
    que visiten cada casillero exactamente una vez.
  }

  \uncover<+->{
    Warnsdorf publicó una solución general en 1823,
    en cada salto ir a aquel casillero que tiene menos vecinos aún sin visitar.
    Esto no necesita reconsiderar pasos.
  }

  \uncover<+->{
    Corresponde a hallar un camino o ciclo hamiltoniano
    en el grafo de casilleros unidos por posibles saltos.
    En el caso general,
    la idea de Warnsdorf resulta ser una heurística
    que dirige la búsqueda eficientemente.
  }
\end{frame}

\section{Resumen}

\begin{frame}
  \setcounter{beamerpauses}{2}
  \frametitle{Resumen}

  \begin{itemize}
  \item
    Idea general.
    El algoritmo recursivo genérico.
  \item
    Ejemplos:
    generar permutaciones,
    rotulaciones graciosas,
    el problema de las 8~reinas
  \item
    El algoritmo de Warnsdorf.
    Heurística de \textquote{el más restringido primero}.
  \end{itemize}
\end{frame}
\end{document}

%%% Local Variables:
%%% mode: latex
%%% TeX-master: t
%%% ispell-local-dictionary: "spanish"
%%% End:

% LocalWords:  glyphtounicode mdseries Function function end Downto
% LocalWords:  Procedure procedure downto Continue continue Break Qc
% LocalWords:  break KwStep step irlas english backtracking setpieces
% LocalWords:  pgfstyle fill circle padding ex smallboard showmover
% LocalWords:  false backfields Qa Qb Qd Qe Qf Qg Qh rotularse Loop
% LocalWords:  loop is solution Process move Undo handout Nd Set up
% LocalWords:  hamiltoniano basic object on
