\input glyphtounicode%
\pdfgentounicode=1
\documentclass[english, spanish, fleqn,%
hyperref = {colorlinks, urlcolor = blue}%
%, handout%
]{beamer}

%% ]{article}
%% \usepackage{beamerarticle}

\usepackage{beamerthemesplit}

\usepackage{fourier}
\usepackage{amsmath}
\usepackage{amsthm}
\usepackage[es-noquoting]{babel}
\usepackage{csquotes}
\usepackage[noline, noend]{algorithm2e}
\usepackage{tikz}

\ifdefined\HCode
  \def\pgfsysdriver{pgfsys-dvisvgm4ht.def}
\fi

\usefonttheme{professionalfonts}

\usetikzlibrary{positioning}

\beamerdefaultoverlayspecification{<+->}

\title{Recorrer grafos}

\author[Horst H. von Brand]{Horst H. von Brand\\
  \href{mailto:vonbrand@inf.utfsm.cl}{vonbrand@inf.utfsm.cl}}

\institute[DI UTFSM]{Departamento de Informática\\
                     Universidad Técnica Federico Santa María}
\date{}

\begin{document}
%%%
%%% For algorithm2e
%%%

\SetAlgorithmName{Algoritmo}{Algoritmo}{Índice de algoritmos}

\SetAlCapSty{mdseries}
\SetKwProg{Function}{function}{}{end}
\SetKwProg{Procedure}{procedure}{}{end}
\SetKw{Variables}{variables}
\SetKw{Downto}{downto}
\SetKwBlock{Loop}{loop}{end}
\SetKw{Continue}{continue}
\SetKw{Break}{break}
\SetKw{KwStep}{step}

\frame{\maketitle}

\begin{frame}<handout>
  \frametitle{Contenido}

  \tableofcontents
\end{frame}

\section{El problema}

\begin{frame}
  \setcounter{beamerpauses}{2}
  \frametitle{Grafos}

  \uncover<+->{
    Un grafo describe las conexiones entre objetos
    (vértices).
    Como tal,
    es una estructura cómoda para describir una gran variedad de situaciones,
    manipular estas estructuras es una tarea común.
  }
  \uncover<+->{
    Un requerimiento común es visitar todos los vértices
    alcanzables desde alguno en particular,
    lo que se conoce como \emph{recorrer el grafo}.
  }
  \uncover<+->{
    Ejemplos típicos son los recorridos estándar
    (preorden, postorden, inorden)
    de árboles binarios
    (que en realidad no son grafos).
  }
\end{frame}

\begin{frame}
  \setcounter{beamerpauses}{2}
  \frametitle{Representaciones de grafos}

  \uncover<+->{
    Tradicionalmente se representan grafos
    mediante \emph{matrices de adyacencia}
    (una matriz de booleanos indexada por vértices,
     verdadero indica que los vértices son vecinos)
    o \emph{listas de adyacencia}
    (se asocia una lista de vecinos a cada vértice).
  }

  \uncover<+->{
    Otra opción es representar vértices como nodos
    y los arcos como punteros entre nodos.
    Es la representación común de árboles binarios
    (que,
     repetimos,
     no son grafos).
  }

  \uncover<+->{
    Muchos de los grafos de interés práctico son demasiado grandes
    para generar completamente
    (más aún si solo se explorará una fracción mínima),
    en tal caso suelen representarse mediante una rutina
    que entrega los vecinos del nodo actual.
  }
\end{frame}

\begin{frame}
  \setcounter{beamerpauses}{2}
  \frametitle{Representación de grafos}

  \uncover<+->{
    Para uniformidad,
    supondremos un grafo \(G = (V, E)\)
    de vértices \(V\) y arcos \(E\).
    Para referirnos a los vértices de \(G\) anotamos \(V_G\),
    para sus arcos usaremos \(E_G\).
    Los vecinos de \(v \in V_G\) los anotaremos:
    \begin{equation*}
      N_G(v)
        = \{ u \in V_G \colon u v \in E_G \}
    \end{equation*}
  }

  \uncover<+->{
    Usaremos conjuntos (\emph{\foreignlanguage{english}{set}}),
    pilas (\emph{\foreignlanguage{english}{stack}}) y
    colas (\emph{\foreignlanguage{english}{queue}})
    como estructuras de datos.
  }
  \uncover<+->{
    Bibliotecas proveen estas estructuras básicas para muchos lenguajes,
    es fácil armar al menos versiones rudimentarias
    (ineficientes o limitadas)
     si no están disponibles.
  }
\end{frame}

\section{Métodos de búsqueda}

\begin{frame}
  \setcounter{beamerpauses}{2}
  \frametitle{Métodos de búsqueda}

  \uncover<+->{
    Desarrollaremos los métodos tradicionales
    para visitar todos los vértices alcanzables en el grafo
    desde un vértice dado.
  }
  \uncover<+->{
    Las estructuras básicas son aplicables también a grafos dirigidos
    (digrafos)
    y a estructuras que no son grafos
    (como árboles binarios o árboles ordenados).
  }

  \uncover<+->{
    Usaremos un conjunto \(C\),
    comúnmente llamado \emph{cerrado}
    (\emph{\foreignlanguage{english}{closed}} en inglés)
    para registrar los vértices ya considerados
    de forma de evitar caer en ciclos.
    Si sabemos que la estructura es acíclica,
    podemos obviar el conjunto de vértices ya visitados.
  }
  \uncover<+->{
    Una variante más o menos obvia es abandonar el recorrido
    en cuanto se halla el vértice buscado.
  }
\end{frame}

\begin{frame}
  \setcounter{beamerpauses}{2}
  \frametitle{Reportar soluciones}

  \uncover<+->{
    Generalmente no basta con dar con un vértice particular,
    se requiere el camino que lleva a él.
  }
  \uncover<+->{
    La técnica general para acomodar esto
    es almacenar los vértices visitados,
    y a cada uno de ellos asociar el vértice padre
    (desde el cual llegamos a él).
    Recorriendo estas listas obtenemos el camino a la meta
    en reversa.
  }
\end{frame}

\subsection{Búsqueda en profundidad}

\begin{frame}
  \setcounter{beamerpauses}{2}
  \frametitle{Búsqueda en profundidad}

  \uncover<+->{
    En inglés,
    se le conoce como \emph{\foreignlanguage{english}{depth first search}},
    abreviado DFS.
  }

  \uncover<+->{
    Es recorrer un camino del grafo hasta el final
    (llegar a un vértice que no tenga vecinos no visitados aún),
    para devolverse al vértice anterior e continuar desde allí.
  }
\end{frame}

\begin{frame}
  \frametitle{Búsqueda en profundidad}
  \framesubtitle{Versión recursiva}

  \begin{algorithm}[H]
    \DontPrintSemicolon

    \(C \leftarrow \varnothing\) \;

    \Procedure{\(\mathrm{DFS}(G, s)\)}{
        \(C \leftarrow C \cup \{ s \}\) \;
        \(\mathrm{process}(s)\) \;
        \ForEach{\(v \in N_G(s)\)}{
          \If{\(v \notin C\)}{
            \(\mathrm{DFS}(G, v)\) \;
        }
      }
    }
  \end{algorithm}
\end{frame}

\begin{frame}
  \setcounter{beamerpauses}{2}
  \frametitle{Clasificar arcos}

  \uncover<+->{
    Es claro que los arcos que sigue búsqueda en profundidad
    forman un árbol recubridor de \(G\) si este es conexo.
    A sus arcos los llamamos \emph{arcos de árbol}
    o \emph{arcos hacia adelante}
    (en inglés \emph{\foreignlanguage{english}{forward edges}}).
  }
  \uncover<+->{
    A los demás arcos de \(G\)
    (que llevan a vértices ya visitados)
    los llamamos \emph{arcos reversos}
    (en inglés \emph{\foreignlanguage{english}{back edges}}).
  }
  \\ \medskip
  \uncover<+->{
    Note que esta clasificación depende de las elecciones del algoritmo,
    no son realmente propiedades del grafo.
  }
\end{frame}

\begin{frame}
  \setcounter{beamerpauses}{2}
  \frametitle{Búsqueda en profundidad}
  \framesubtitle{Versión iterativa}

  \uncover<+->{
    Notando que lo que hacemos es dejar pendientes los vértices no visitados
    para profundizar la búsqueda desde el actual,
    podemos evitar la recursión explícita
    manejando la pila de pendientes manualmente.
  }
  \uncover<+->{
    Elementos de la pila son los vértices pendientes de visitar.
  }
\end{frame}

\begin{frame}
  \frametitle{Búsqueda en profundidad}
  \framesubtitle{Versión iterativa}

  \small
  \begin{algorithm}[H]
    \DontPrintSemicolon

    \Procedure{\(\mathrm{DFS}(G, s)\)}{
      \(S \leftarrow \mathrm{Stack}\);
      \(\mathrm{push}(S, s)\);
      \(C \leftarrow \varnothing\) \;
      \While{\(\neg \mathrm{empty}(S)\)}{
        \(s \leftarrow \mathrm{pop}(S)\) \;
        \If{\(s \notin C\)}{
          \(C \leftarrow C \cup \{ s \}\) \;
          \(\mathrm{process}(s)\) \;
          \ForEach{\(v \in N_G(s)\)}{
            \If{\(v \notin C\)}{
              \(\mathrm{push}(S, v)\) \;
            }
          }
        }
      }
    }
  \end{algorithm}
\end{frame}

\subsection{Búsqueda a lo ancho}

\begin{frame}
  \setcounter{beamerpauses}{2}
  \frametitle{Búsqueda a lo ancho}

  \uncover<+->{
    En inglés,
    se le conoce como \emph{\foreignlanguage{english}{breadth first search}},
    abreviado BFS.
  }

  \uncover<+->{
    La idea acá es visitar los vecinos de \(s\),
    luego de haberlos visitado todos
    comenzamos a revisar los vecinos de los vecinos de \(s\),
    y así sucesivamente.
  }
  \uncover<+->{
    Sorprendentemente es un cambio menor:
    cambiar la pila por una cola de vértices pendientes.
  }
\end{frame}


\begin{frame}
  \frametitle{Búsqueda a lo ancho}

  \small
  \begin{algorithm}[H]
    \DontPrintSemicolon

    \Procedure{\(\mathrm{BFS}(G, s)\)}{
      \(Q \leftarrow \mathrm{Queue}\);
      \(\mathrm{enqueue}(Q, s)\);
      \(C \leftarrow \varnothing\) \;
      \While{\(\neg \mathrm{empty}(Q)\)}{
        \(s \leftarrow \mathrm{dequeue}(Q)\) \;
        \If{\(s \notin C\)}{
          \(C \leftarrow C \cup \{ s \}\) \;
          \(\mathrm{process}(s)\) \;
          \ForEach{\(v \in N_G(s)\)}{
            \If{\(v \notin C\)}{
              \(\mathrm{enqueue}(Q, v)\) \;
            }
          }
        }
      }
    }
  \end{algorithm}
\end{frame}

\begin{frame}
  \setcounter{beamerpauses}{2}
  \frametitle{Comentarios}
  \framesubtitle{Búsqueda en profundidad}

  \uncover<+->{
    El programa más simple es búsqueda en profundidad recursiva.
  }
  \uncover<+->{
    Vemos que nuestro planteo de \emph{\foreignlanguage{english}{backtracking}}
    en su corazón es búsqueda en profundidad recursiva.
    Es claro que podemos reformularlo en forma iterativa,
    y adosarle evitar ciclos de ser necesario.
  }

  \uncover<+->{
    Búsqueda en profundidad tiene la virtud de solo usar espacio
    (en la pila explícita en la versión iterativa,
     o en la pila implícita usada para recursión)
    proporcional a la profundidad máxima.
  }
  \uncover<+->{
    Para aplicaciones tipo \emph{\foreignlanguage{english}{backtracking}}
    podemos usar una única estructura global que se manipula.
  }

  \uncover<+->{
    Puede perderse en un largo camino,
    sin hallar soluciones vecinas.
  }
\end{frame}

\begin{frame}
  \setcounter{beamerpauses}{2}
  \frametitle{Comentarios}
  \framesubtitle{Búsqueda a lo ancho}

  \uncover<+->{
    Búsqueda a lo ancho garantiza hallar el vértice más cercano,
    al ir desarrollando el grafo en oleadas.
  }
  \uncover<+->{
    Requiere almacenar gran parte del árbol sobre los nodos considerados.
  }
\end{frame}

\subsection{Búsqueda mejor primero}

\begin{frame}
  \setcounter{beamerpauses}{2}
  \frametitle{Heurísticas}

  \uncover<+->{
    Los métodos precedentes se llaman \emph{ciegos}.
    Solo consideran la estructura del grafo,
    no los vértices.
  }
  \uncover<+->{
    Una variación obvia es \emph{búsqueda heurística},
    que a cada vértice \(v\) le asocia un valor \(h(v)\),
    tal que un vértice de menor valor de \(h\) es más prometedor
    (sospechamos es más cercano a la solución).
  }
  \uncover<+->{
    Si supiéramos cuál expandir para llegar al destino,
    bastaría elegir ese.
    Generalmente solo contamos con alguna estimación,
    una \emph{heurística}
    (del griego para \textquote{descubrir})
    que guíe la búsqueda.
  }
  \uncover<+->{
    Cambia la cola por una \emph{cola de prioridad},
    y vamos extrayendo y considerando los vértices en ella
    en orden de \(h\) creciente.
  }
\end{frame}

\begin{frame}
  \setcounter{beamerpauses}{2}
  \frametitle{Búsqueda mejor primero}

  \uncover<+->{
    Juntamos las observaciones precedentes,
    añadiendo una \emph{heurística} \(h \colon V \to \mathbb{R}\),
    tal que menor valor de \(h\) significa más prometedor de expandir.
  }
  \uncover<+->{
    La llamamos \emph{búsqueda de mejor primero},
    en inglés \emph{\foreignlanguage{english}{Best First Search}},
    confusamente también abreviado BFS.
  }
  \\ \medskip
  \uncover<+->{
    En vez de una cola,
    usamos una cola de prioridad.
  }
  \uncover<+->{
    El resto del algoritmo de búsqueda a lo ancho queda casi idéntico.
  }
\end{frame}

\begin{frame}
  \frametitle{Búsqueda mejor primero}

  \small
  \begin{algorithm}[H]
    \DontPrintSemicolon

    \Procedure{\(\mathrm{BFS}(G, h, s)\)}{
      \(Q \leftarrow \mathrm{PriorityQueue}\);
      \(\mathrm{insert}(Q, s, h(s))\);
      \(C \leftarrow \varnothing\) \;
      \While{\(\neg \mathrm{empty}(Q)\)}{
        \(s \leftarrow \mathrm{deletemin}(Q)\) \;
        \If{\(s \notin C\)}{
          \(C \leftarrow C \cup \{ s \}\) \;
          \(\mathrm{process}(s)\) \;
          \ForEach{\(v \in N_G(s)\)}{
            \If{\(v \notin C\)}{
              \(\mathrm{insert}(Q, v, h(v))\) \;
            }
          }
        }
      }
    }
  \end{algorithm}
\end{frame}

\section{Aplicaciones}

\begin{frame}
  \setcounter{beamerpauses}{2}
  \frametitle{Aplicaciones de recorridos de grafos}

  \uncover<+->{
    Una aplicación obvia es determinar si un grafo es conexo:
    si una búsqueda alcanza todos los vértices,
    es conexo.
  }
  \uncover<+->{
    Para determinar componentes conexos,
    basta comenzar nuevamente la búsqueda
    las veces que sea necesario iniciando con un vértice no visitado.
  }
\end{frame}

\begin{frame}
  \setcounter{beamerpauses}{2}
  \frametitle{Puntos de articulación}

  \uncover<+->{
    Un \emph{punto de articulación} de un grafo conexo
    es un vértice tal que si se elimina del grafo
    el resultado no es conexo.
  }
  \uncover<+->{
    El vértice \(C\) es un punto de articulación del grafo:
    \\ \medskip
    \begin{center}
      \begin{tikzpicture}[node distance = 1.8cm, on grid]
        \node[circle, draw] at (0, 3)		(A) {\(A\)};
        \node[circle, draw] at (0, 0)		(B) {\(B\)};
        \node[circle, draw] at (2, 1.5)		(C) {\(C\)};
        \node[circle, draw] at (4, 3)		(D) {\(D\)};
        \node[circle, draw] at (4, 0)		(E) {\(E\)};
        \node[circle, draw] at (6, 1.5)		(F) {\(F\)};
        \node[circle, draw] at (2, -0.6)	(G) {\(G\)};

        \draw (C) -- (A) -- (B) -- (C) -- (D) -- (F) -- (E) -- (C) -- (G);
      \end{tikzpicture}
    \end{center}
  }
\end{frame}

\begin{frame}
  \setcounter{beamerpauses}{2}
  \frametitle{Puntos de articulación}

  \uncover<+->{
    Un vértice \(v\) es punto de articulación de \(G\) si:
  }
  \begin{itemize}
  \item
    \(v\) es raíz del árbol DFS
    y tiene al menos dos hijos
  \item
    \(v\) no es raíz del árbol DFS
    y tiene un hijo \(u\)
    tal que ninguno de los descendientes de \(u\)
    tiene un arco reverso a un ancestro de \(v\)
  \end{itemize}
  \medskip
  \uncover<+->{
    Hallar puntos de articulación es buscar vértices con estas condiciones.
  }
  \\ \medskip
  \uncover<+->{
    Acá DFS es central:
    halla todos los vértices alcanzables desde un hijo de la raíz
    antes de procesar los demás vecinos de ésta.
  }
\end{frame}

\begin{frame}
  \setcounter{beamerpauses}{2}
  \frametitle{Determinar si un grafo es bipartito}

  \uncover<+->{
    Grafos bipartitos
    (o sea, grafos \(G = (X \cup Y, E)\)
     tales que solo hay arcos entre vértices \(X\) e \(Y\))
    son importantes en aplicaciones.
  }
  \uncover<+->{
    Determinar si \(G\) es bipartito
    (e identificar \(X\) e \(Y\))
    es elegir un vértice de partida y colorearlo de rojo,
    recorrer el grafo,
    cada vez que se encuentra un nuevo vértice
    colorearlo del color opuesto a su padre.
    En caso de arcos reversos
    verificar que el color del otro vértice es opuesto al actual.
  }
  \uncover<+->{
    Si se agota una componente conexa,
    procesar las demás de la misma forma.
  }
  \uncover<+->{
    Finalmente,
    designar \(X\) e \(Y\) como los conjuntos de vértices de ambos colores.
  }
\end{frame}

\begin{frame}
  \setcounter{beamerpauses}{2}
  \frametitle{Algoritmos en teoría de lenguajes}

  \uncover<+->{
    Varios de nuestros algoritmos en teoría de autómatas y lenguajes
    (como construir un DFA dado un NFA,
     determinar símbolos inútiles de una gramática)
    son recorridos de grafos.
  }
\end{frame}

\section{Resumen}

\begin{frame}
  \setcounter{beamerpauses}{2}
  \frametitle{Resumen}

  \begin{itemize}
  \item
    Discutimos las técnicas de recorrido de grafos más comunes:
    en profundidad y a lo ancho.
  \item
    Algoritmos afines son aplicables a estructuras que no son grafos:
    árboles binarios,
    árboles ordenados,
    digrafos,
    multigrafos.
  \item
    Contrastamos ventajas y desventajas de búsqueda en profundidad
    y a lo ancho.
  \item
    Discutimos la variante \emph{mejor primero}
  \item
    Describimos algunas aplicaciones.
  \end{itemize}
\end{frame}
\end{document}

%%% Local Variables:
%%% mode: latex
%%% TeX-master: t
%%% ispell-local-dictionary: "spanish"
%%% End:

% LocalWords:  glyphtounicode mdseries Function function end Downto
% LocalWords:  Procedure procedure downto Loop loop Continue continue
% LocalWords:  Break break KwStep step preorden postorden inorden set
% LocalWords:  english stack queue closed depth first search DFS BFS
% LocalWords:  breadth backtracking reformularlo Best recubridor back
% LocalWords:  forward edges DFA NFA multigrafos
