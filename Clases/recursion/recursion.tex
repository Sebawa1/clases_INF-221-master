\bibliographystyle{babplain-fl}

\chapter{Recursión}
\label{cha:recursion}

% To do: Compare with Smid

  Nuestra herramienta principal
  es \emph{reducir} problemas a problemas más \textquote{simples}.
  Por ejemplo:
  de una expresión regular obtener
  un programa eficiente para reconocer un patrón.
  Para resolver este problema usamos:
  \begin{itemize}
  \item
    Expresión Regular a NFA (por ejemplo Thompson)
  \item
    NFA a DFA (algoritmo de subconjuntos)
  \item
    DFA a DFA mínimo (hay varias opciones)
  \item Interpretar el DFA o traducirlo a código.
  \end{itemize}
  Es la estrategia que usamos en Informática Teórica
  (INF-155),
  pero allí lo hicimos para demostrar que hay problemas difíciles.

  Al resolver un problema,
  lo dividimos en subproblemas diversos y combinamos resultados.
  Notar que al hacer esto
  (por ejemplo,
   invocar una rutina de biblioteca,
   como \texttt{printf(3)} en C)
  confiamos en que la solución al subproblema
  hace su trabajo correctamente.
  \emph{Confiamos} en terceros.
  De la misma manera,
  al invocar una función que nosotros escribimos,
  \emph{confiamos} en que hace su trabajo correctamente.
  La consideramos una caja negra,
  cuyo funcionamiento no nos interesa,
  en realidad ni siquiera nos incumbe.
  Podemos incluso cambiarla por una versión más eficiente,
  sin cambiar el programa.
  Desde el otro ángulo,
  quien escribe la solución al subproblema no considera dónde se usará,
  se concentra en cumplir con las especificaciones de su tarea.

  Usar recursión es lo mismo,
  solo que la función se invoca a sí misma
  (directa o indirectamente).

  Recursión es inducción en forma de programa.
  Igual que en inducción requerimos casos base y paso de inducción.

  La belleza de la recursión es que nos debemos preocupar
  solo del problema entre manos,
  subproblemas se resuelven automáticamente.
  Una extensa discusión,
  introduciendo el tema gradualmente,
  ofrece Roberts~%
    \cite{roberts86:_thinking_recursively}.
  Lúcida es la exposición de Hetland~%
    \cite[capítulo~4]{hetland14:_python_algorithms};
  otra opción es Smid~%
    \cite[capítulo~4]{smid19:_discr_struct_comput_scien}
  quien discute varios ejemplos de definiciones por inducción
  que llevan a programas recursivos nada obvios.

  Para escribir un programa recursivo hay varios pasos clave
  a considerar:
  \begin{itemize}
  \item
    Describir la solución al problema como la combinación
    de resultados de problemas similares,
    con datos distintos.
  \item
    Determinar situaciones en las cuales la solución es tan simple
    que puede resolverse directamente.
  \end{itemize}
  Para verificar que el programa resultante es correcto,
  debemos asegurarnos que la descomposición es correcta,
  que las soluciones de los casos base
  (no recursivos)
  son correctos
  y que cada vez las llamadas son a \textquote{casos más cercanos}
  a los casos base.
  Note que solo nos preocupamos de detallar el primer paso de la solución,
  los demás pasos son automáticos.
  Es lo que Erickson~%
    \cite{erickson19:_algorithms}
  llama \emph{\foreignlanguage{english}{recursion fairy}},
  el hada de la recursión,
  haciéndose cargo.
  Esto hace atractiva la recursión como técnica de programación,
  es frecuente que dé soluciones simples,
  transparentes.

\section{Factoriales}
\label{sec:factoriales}

  La definición de factoriales es recursiva:
  \begin{equation*}
    n!
      = \begin{cases}
          1		    & n = 0 \\
          n \cdot (n - 1)!  & n > 0
        \end{cases}
  \end{equation*}
  De acá es obvia la versión recursiva
  del listado~\ref{lst:factorial-recursive}.
  \lstinputlisting[float, floatplacement = h,
                   language = C,
                   firstline = 5, lastline = 11,
                   caption = {Factoriales, versión recursiva},
                   label = lst:factorial-recursive]
                  {code/factorial.c}
  Este es un ejemplo un tanto tonto,
  ya que es fácil ver que eventualmente llegaremos al caso base \(n = 0\),
  retornando \num{1},
  luego
  (trepando las llamadas recursivas)
  vemos que en cada paso multiplicamos por el siguiente valor.
  Esto lleva a la versión iterativa~\ref{lst:factorial-iterative}.
  \lstinputlisting[float, floatplacement = h,
                   language = C,
                   firstline = 15,
                   caption = {Factoriales, versión iterativa},
                   label = lst:factorial-iterative]
                  {code/factorial.c}
  En este caso simple la relación entre las llamadas
  es simple de dilucidar,
  dando una solución iterativa simple.

\section{Recorrer árbol binario}
\label{sec:binary-tree-traversal}

  La estructura de árbol binario tiene definición recursiva.
  Un árbol binario es:
  \begin{itemize}
  \item
    Vacío
  \item
    Consta de un nodo \emph{raíz},
    conectado a un subárbol binario izquierdo
    y un subárbol binario derecho.
  \end{itemize}
  Con esta definición recursiva,
  procesar estas estructuras
  (por ejemplo,
   visitar cada uno de sus nodos)
  tiene la obvia expresión recursiva
  del listado~\ref{lst:binary-tree-traversal}.
  \lstinputlisting[float, floatplacement = h,
                   language = C,
                   firstline = 5,
                   caption = {Recorrer un árbol binario},
                   label = lst:binary-tree-traversal]
                  {code/binary-tree-traversal.c}
  Esto también puede reducirse a una versión iterativa,
  pero será necesario manejar explícitamente las tareas pendientes
  mediante una pila explícita.
  En tales casos es preferible que el lenguaje de programación
  se haga cargo de estos detalles burocráticos.

\section{Torres de Hanoi}

  Según la leyenda
  (inventada por Édouard Lucas en el siglo~XIX),
  en un monasterio en Hanoi
  hay \num{64} placas redondas de oro
  en tres agujas de diamante.
  Una profecía dice que se crearon cuando se creó el mundo,
  con las placas ordenadas de mayor a menor en una de las agujas,
  los monjes del monasterio tienen la tarea de mover las placas a otra aguja.
  Solo se permite mover una placa a la vez
  y nunca se debe ubicar una placa sobre una menor.
  La figura~\ref{13::TorreHanoi1}
  muestra la situación inicial de una versión simplificada,
  con menos placas,
  común como acertijo.

  Lo que buscan los monjes es mover todas las placas de \(A\) a \(C\),
  usando \(B\) como auxiliar.
  \begin{figure}[ht]
    \centering
    \newcommand{\hanoiDisk}[4]
                  {\draw [fill = #4] (#1 - #3/2, #2)
                    -- ++ (#3, 0) -- ++ (0, 0.15)
                    -- ++ (-#3, 0) -- cycle} %(x, y, width, color)
    \begin{tikzpicture}[scale = 0.75]
      % Dibujo de la primera plataforma
      \hanoiDisk{0}{0}{2.5}{green!60!black!80};
      \hanoiDisk{2.5}{0}{2.5}{blue!60!black!60};
      \hanoiDisk{5}{0}{2.5}{green!60!black!80};
      \draw (0, 0.15) -- ++ (0, 2);
      \draw (2.5, 0.15) -- ++ (0, 2);
      \draw (5, 0.15) -- ++ (0, 2);

      % Dibujo de los discos de la primera plataforma
      \hanoiDisk{0}{0.15}{2}{gray!40};
      \hanoiDisk{0}{0.15 + 0.15}{1.8}{gray!20};
      \hanoiDisk{0}{0.15 + 2 * 0.15}{1.6}{gray!40};
      \hanoiDisk{0}{0.15 + 3 * 0.15}{1.4}{gray!20};
      \hanoiDisk{0}{0.15 + 4 * 0.15}{1.2}{gray!40};
      \hanoiDisk{0}{0.15 + 5 * 0.15}{1.0}{gray!20};
      \hanoiDisk{0}{0.15 + 6 * 0.15}{0.8}{gray!40};
      \hanoiDisk{0}{0.15 + 7 * 0.15}{0.6}{gray!20};
      \hanoiDisk{0}{0.15 + 8 * 0.15}{0.4}{gray!40};
      \node at (0, -0.25) (A) {\(A\)};
      \node at (2.5, -0.25) (B) {\(B\)};
      \node at (5, -0.25) (C) {\(C\)};

      % Dibujo de la segunda plataforma
      \hanoiDisk{0   + 8.5}{0}{2.5}{green!60!black!80};
      \hanoiDisk{2.5 + 8.5}{0}{2.5}{blue!60!black!60};
      \hanoiDisk{5   + 8.5}{0}{2.5}{green!60!black!80};
      \draw (0	 + 8.5, 0.15) -- ++ (0, 2);
      \draw (2.5 + 8.5, 0.15) -- ++ (0, 2);
      \draw (5	 + 8.5, 0.15) -- ++ (0, 2);

      % Dibujo de los discos de la segunda plataforma
      \hanoiDisk{5 + 8.5}{0.15}{2}{gray!40};
      \hanoiDisk{5 + 8.5}{0.15 + 0.15}{1.8}{gray!20};
      \hanoiDisk{5 + 8.5}{0.15 + 2 * 0.15}{1.6}{gray!40};
      \hanoiDisk{5 + 8.5}{0.15 + 3 * 0.15}{1.4}{gray!20};
      \hanoiDisk{5 + 8.5}{0.15 + 4 * 0.15}{1.2}{gray!40};
      \hanoiDisk{5 + 8.5}{0.15 + 5 * 0.15}{1.0}{gray!20};
      \hanoiDisk{5 + 8.5}{0.15 + 6 * 0.15}{0.8}{gray!40};
      \hanoiDisk{5 + 8.5}{0.15 + 7 * 0.15}{0.6}{gray!20};
      \hanoiDisk{5 + 8.5}{0.15 + 8 * 0.15}{0.4}{gray!40};
      \node at (0   + 8.5, -0.25) (A2) {\(A\)};
      \node at (2.5 + 8.5, -0.25) (B2) {\(B\)};
      \node at (5   + 8.5, -0.25) (C2) {\(C\)};

      \draw [-latex', thick, red!60!black!80] (6.35, 1.25) -- ++ (0.75, 0);
    \end{tikzpicture}
    \caption{Mover las placas desde la plataforma \(A\) a la \(C\).}
    \label{13::TorreHanoi1}
  \end{figure}

\subsection{Solución (recursiva)}

  Una solución
  es como se muestra en la figura~\ref{13::TorreHanoi2}.
  \begin{figure}[ht]
    \centering
    \newcommand{\hanoiDisk}[4]
         {\draw [fill = #4] (#1 - #3/2, #2) -- ++ (#3, 0)
           -- ++ (0, 0.15) -- ++ (-#3, 0) -- cycle} %(x, y, width, color)
    \begin{tikzpicture}[scale = 0.75]
      % Dibujo de la primera plataforma
      \hanoiDisk{0}{0}{2.5}{green!60!black!80};
      \hanoiDisk{2.5}{0}{2.5}{blue!60!black!60};
      \hanoiDisk{5}{0}{2.5}{green!60!black!80};
      \draw (0, 0.15) -- ++ (0, 2);
      \draw (2.5, 0.15) -- ++ (0, 2);
      \draw (5, 0.15) -- ++ (0, 2);

      % Dibujo de los discos de la primera plataforma
      \hanoiDisk{0}{0.15}{2}{gray!40};
      \hanoiDisk{2.5}{0.15 + 0 * 0.15}{1.8}{gray!20};
      \hanoiDisk{2.5}{0.15 + 1 * 0.15}{1.6}{gray!40};
      \hanoiDisk{2.5}{0.15 + 2 * 0.15}{1.4}{gray!20};
      \hanoiDisk{2.5}{0.15 + 3 * 0.15}{1.2}{gray!40};
      \hanoiDisk{2.5}{0.15 + 4 * 0.15}{1.0}{gray!20};
      \hanoiDisk{2.5}{0.15 + 5 * 0.15}{0.8}{gray!40};
      \hanoiDisk{2.5}{0.15 + 6 * 0.15}{0.6}{gray!20};
      \hanoiDisk{2.5}{0.15 + 7 * 0.15}{0.4}{gray!40};
      \node at (0, -0.25) (A) {\(A\)};
      \node at (2.5, -0.25) (B) {\(B\)};
      \node at (5, -0.25) (C) {\(C\)};

      % Dibujo de la segunda plataforma
      \hanoiDisk{0 + 8.5}{0}{2.5}{green!60!black!80};
      \hanoiDisk{2.5 + 8.5}{0}{2.5}{blue!60!black!60};
      \hanoiDisk{5 + 8.5}{0}{2.5}{green!60!black!80};
      \draw (0	 + 8.5, 0.15) -- ++ (0, 2);
      \draw (2.5 + 8.5, 0.15) -- ++ (0, 2);
      \draw (5	 + 8.5, 0.15) -- ++ (0, 2);

      % Dibujo de los discos de la segunda plataforma
      \hanoiDisk{5 + 8.5}{0.15}{2}{gray!40};
      \hanoiDisk{2.5 + 8.5}{0.15 + 0 * 0.15}{1.8}{gray!20};
      \hanoiDisk{2.5 + 8.5}{0.15 + 1 * 0.15}{1.6}{gray!40};
      \hanoiDisk{2.5 + 8.5}{0.15 + 2 * 0.15}{1.4}{gray!20};
      \hanoiDisk{2.5 + 8.5}{0.15 + 3 * 0.15}{1.2}{gray!40};
      \hanoiDisk{2.5 + 8.5}{0.15 + 4 * 0.15}{1.0}{gray!20};
      \hanoiDisk{2.5 + 8.5}{0.15 + 5 * 0.15}{0.8}{gray!40};
      \hanoiDisk{2.5 + 8.5}{0.15 + 6 * 0.15}{0.6}{gray!20};
      \hanoiDisk{2.5 + 8.5}{0.15 + 7 * 0.15}{0.4}{gray!40};
      \node at (0   + 8.5, -0.25) (A2) {\(A\)};
      \node at (2.5 + 8.5, -0.25) (B2) {\(B\)};
      \node at (5   + 8.5, -0.25) (C2) {\(C\)};

      \draw [-latex', thick, red!60!black!80] (6.35, 1.25) -- ++ (0.75, 0);
    \end{tikzpicture}
    \caption{Solo nos queda mover el último disco (más grande)
             de \(A\) a \(C\) directamente.}
    \label{13::TorreHanoi2}
  \end{figure}
  Esto es solución porque traduce el problema de mover \(n\) piezas
  a mover \(n - 1\) recursivamente,
  luego mover \num{1}
  (trivial, figura~\ref{13::TorreHanoi2}),
  luego mover \(n - 1\) recursivamente.
  Es critico en este diseño no considerar más que el disco mayor,
  en el que nos concentramos.
  Dejamos el completar la tarea al hada de recursión,
  la llamada recursiva es simplemente una caja negra
  cuyo funcionamiento interno no es de nuestra incumbencia.

\paragraph{Problema:}

  Mover \(n\) piezas de \(A\) a \(C\).

  \begin{proof}
    \begin{description}
    \item[Base:]
      Si \(n = 0\),
      no hay que hacer nada.
    \item[Inducción:]
      Supongamos que sabemos mover \(k\) piezas de \(i\) a \(j\)
      (con \(i \ne j\)).
      Para mover \(k + 1\) piezas de \(A\) a \(C\)
      (con \(B\) de \textquote{apoyo}):
      \begin{itemize}
      \item
        Movemos las \(k\) piezas superiores de \(A\) a \(B\).
      \item
        Movemos la pieza inferior de \(A\) a \(C\).
      \item
        Movemos las \(k\) piezas de \(B\) a \(C\)
        (con \(A\) de \textquote{apoyo}).
      \end{itemize}
    \end{description}
  \end{proof}
  Pseudocódigo es el algoritmo~\ref{alg:Hanoi}.
  Usamos las variables \(\mathnormal{src}\) para el origen,
  \(\mathnormal{dst}\) para el destino
  y \(\mathnormal{tmp}\) para el auxiliar
  (libre).
  Estas variables valen \(A\), \(B\) o \(C\),
  según sea.
  La solución al problema está dada por la llamada
  \(\mathop{hanoi}(64, A, C, B)\)
  (mueva \num{64} platos de \(A\) a \(C\),
   usando \(B\) de intermediario).
  \begin{algorithm}[ht]
    \DontPrintSemicolon\Indp

    \Procedure{\(\mathrm{hanoi}(n,
                     \mathnormal{src}, \mathnormal{dst}, \mathnormal{tmp})\)}{
      \If{\(n > 0\)}{
        \(\mathrm{hanoi}(n - 1,
              \mathnormal{src}, \mathnormal{tmp}, \mathnormal{dst})\) \;
        Move a disk from \(\mathnormal{src}\) to \(\mathnormal{dst}\) \;
        \(\mathrm{hanoi}(n - 1,
              \mathnormal{tmp}, \mathnormal{dst}, \mathnormal{src})\) \;
      }
    }
    \caption{Solución recursiva a las torres de Hanoi}
    \label{alg:Hanoi}
  \end{algorithm}
  Por la estructura de la solución,
  siempre cumplimos con las restricciones.

  Una pregunta obvia es el número de movidas que hace nuestro algoritmo.
  \begin{proposition}
    \label{pro:Hanoi-recursive}
    El algoritmo recursivo~\ref{alg:Hanoi}
    ejecuta \(2^n - 1\)~movidas para transferir \(n\) placas de \(A\) a \(C\).
  \end{proposition}
  \begin{proof}
    Sea \(T(n)\) el número de movidas para transferir \(n\) platos
    usando el algoritmo~\ref{alg:Hanoi}.
    La recursión lleva a la recurrencia:
    \begin{align*}
      T(0)
        &= 0 \\
      T(n + 1)
        &= 2 T(n) + 1,\qquad n \ge 0
    \end{align*}
    Técnicas estándar de solución de recurrencias entregan:
    \begin{equation*}
      T(n)
        = 2^n - 1
    \end{equation*}
    \qedhere
  \end{proof}
  \begin{proposition}
    \label{pro:Hanoi-optimal}
    En el problema de las torres de Hanoi
    se requieren al menos \(2^n - 1\) movidas para transferir \(n\) placas.
  \end{proposition}
  \begin{proof}
    La demostración es por inducción.
    Llamemos \(R(n)\) el número de movidas requeridas
    para transferir \(n\) placas.
    \begin{description}
    \item[Base:]
      Si no hay placas,
      no hay movidas,
      o sea,
      \(R(0) = 0 \ge 2^0 - 1\).
    \item[Inducción:]
      Supongamos que para mover \(k\)
      discos debemos hacer \(R(k) \ge 2^k - 1\) movidas,
      y consideremos mover \(k + 1\) discos.
      El disco mayor no puede ayudar en el proceso,
      siempre está abajo.
      Para poder moverlo,
      deberemos mover \(k\) discos de \(A\) a \(B\),
      luego mover el disco mayor de \(A\) a \(C\),
      finalmente mover los que están en \(B\) a \(C\).
      Esto suma:
      \begin{align*}
        R(k + 1)
          &\ge R(k) + R(k) + 1 \\
          &\ge 2 (2^k - 1) + 1 \\
          &=   2^{k + 1} - 1
      \end{align*}
    \end{description}
    Por inducción,
    se requieren \(R(n) \ge 2^n - 1\) movidas para todo \(n \in \mathbb{N}\).
  \end{proof}
  Como la cota inferior es exactamente lo que da nuestro algoritmo,
  este es óptimo.

\section{Interpretar expresiones regulares}
\label{sec:regex-interpreter}

  Un excelente ejemplo de recursión es el intérprete de
  (una versión muy restringida)
  de expresiones regulares escrito por Pike~
    descrito y comentado por Kernighan~%
    \cite{kernighan07:_regex_matcher}
  (también publicado en el texto de Kernighan y Pike~%
     \cite{kernighan99:_practice_programming}).
  Es una muy instructiva ilustración de recursión.
  Reconoce solo las siguientes construcciones:

  \begin{tabular}[h]{cl}
    c			   & Calza el caracter 'c'
                             (salvo los especiales a continuación) \\
    .			   & Calza cualquier caracter \\
    \textasciicircum	   & Calza el comienzo
                             del \emph{\foreignlanguage{english}{string}} \\
    \textdollar		   & Calza el final
                             del \emph{\foreignlanguage{english}{string}} \\
    \textasteriskcentered  & Cero o más ocurrencias del caracter anterior
  \end{tabular}

  El código
  (ligeramente editado para usanza moderna en C)
  es el listado~\ref{lst:Pike-regex},
  que usa el encabezado~\ref{lst:Pike-regex-h}.
  \lstinputlisting[language = C,
                   firstline = 7,
                   caption = {Declaraciones para la rutina de Pike},
                   label = lst:Pike-regex-h]{code/match.h}
  \lstinputlisting[language = C,
                   firstline = 9,
                   caption = {La rutina de Pike},
                   label = lst:Pike-regex]{code/match.c}
  Usa el simple expediente de intentar calzar en cada punto de la línea actual,
  llamando a \lstinline[language = C]!matchhere! en cada posición.
  La función \lstinline[language = C]!matchhere! es el corazón del código,
  ve si la expresión calza en la posición actual.
  Según si el primer caracter que queda por manejar de la expresión
  calza en la posición actual del código,
  recursivamente considera el resto de la expresión y del texto;
  si el caracter viene seguido por un asterisco
  invoca a \lstinline[language = C]!matchstar!,
  que intenta repetir el caracter inicial
  (al que se aplica asterisco)
  cero o más veces,
  llamando recursivamente a \lstinline[language = C]!matchhere!
  con el resto de la expresión y del texto.

\section{Mergesort}

  Queremos ordenar \(A[1, \ldots, n]\).
  Nuestro pseudocódigo para mergesort es el del algoritmo~\ref{alg:mergesort}.
  \begin{algorithm}[ht]
    \DontPrintSemicolon\Indp

    \Procedure{\(\mathrm{mergesort}(A[1, \dotsc, n])\)}{
      \If{\(n > 1\)}{
        \(\mathrm{mergesort}
             (A[1, \dotsc,
                 \left\lfloor \frac{n}{2} \right\rfloor])\) \;
        \(\mathrm{mergesort}
             (A[\left\lfloor \frac{n}{2} \right\rfloor + 1,
                 n])\) \;
        \(\mathrm{merge}
             (A[1, \dotsc,
                 \left\lfloor \frac{n}{2} \right\rfloor],
              A[\left\lfloor \frac{n}{2} \right\rfloor + 1,
                 n])\) \;
      }
    }
    \caption{Mergesort}
    \label{alg:mergesort}
  \end{algorithm}
  De la misma forma que lo hicimos con las torres de Hanoi:
  interesa el número de operaciones.
  Para simplificar el análisis,
  contabilizamos el número de copias de datos,
  por cada copia hay un número acotado de operaciones adicionales.
  Sea \(M(n)\) el número de copias para completar el ordenamiento.
  Es claro que:
  \begin{align}
    M(0)
      &= M(1) \notag \\
    M(n)
      &= M\left( \left\lfloor \frac{n}{2} \right\rfloor \right)
           + M\left( \left\lceil \frac{n}{2} \right\rceil \right)
           + m\left( \left\lfloor \frac{n}{2} \right\rfloor,
                     \left\lceil \frac{n}{2} \right\rceil \right)
                 \label{13::M1}
  \end{align}
  Acá \(m(a, b)\)
  es el costo de intercalar dos grupos de tamaños \(a\) y \(b\).
  Si consideramos el costo simplemente en términos de elementos a mover,
  vemos que \(m(a, b) = a + b\),
  todos los elementos se mueven de las áreas respectivas a la de salida,
  y el último término de~\eqref{13::M1} es simplemente \(n\).
  Discutiremos recurrencias como esta
  en el capítulo~\ref{cha:dividir-conquistar},
  donde concluiremos que:
  \begin{equation*}
    M(n)
      = O(n \log n)
  \end{equation*}
  Dentro de un factor constante,
  mergesort es óptimo
  (vimos en el apunte de \emph{Fundamentos de Informática}~%
     \cite{brand17:_fundamentos_informatica}
   que si solo se comparan elementos
   para ordenar \(n\)~elementos
   se requieren~\(\Omega(n \log n)\) comparaciones).

\section*{Ejercicios}
\label{sec:ejercicios-13}

  \begin{enumerate}
  \item
    Para plantear soluciones alternativas
    a problemas naturalmente expresados en forma recursiva
    sí debemos considerar las actividades de las llamadas recursivas,
    generalmente llegando al caso base y trabajar de allí en reversa.
    Para verificar que las soluciones son correctas,
    deben demostrarse propiedades de la solución,
    como por ejemplo en el caso de torres de Hanoi,
    suponiendo que las agujas están en los vértices de un triángulo equilátero,
    demostrar que:
    \begin{itemize}
    \item
      Cada disco se mueve siempre en la misma dirección
    \item
      Mover un disco pequeño siempre alterna con la única movida legal.
    \end{itemize}
  \item
    Una estrategia alternativa para las torres de Hanoi
    es considerar que las agujas
    están ubicadas en los vértices de un triángulo equilátero.
    Repita lo siguiente hasta completar la solución:
    \begin{itemize}
    \item
      Mueva el disco más pequeño una posición a la izquierda
    \item
      Haga la movida legal que no involucra al disco más pequeño.
    \end{itemize}
    Esta estrategia es más fácil de recordar para resolver el acertijo a mano,
    pero demostrar que es correcta es mucho más trabajo
    que para la solución recursiva del texto.
  \item
    Una solución no recursiva al problema de torres de Hanoi
    es como sigue.
    Nombre las posiciones como \(A\), \(B\) y \(C\),
    con los discos originalmente en \(A\) y deben quedar en \(C\).
    Si el número de discos es par,
    repita lo siguiente hasta completar la tarea:
    \begin{itemize}
    \item
      Haga la movida legal entre \(A\) y \(B\)
      (en cualquier dirección)
    \item
      Haga la movida legal entre \(A\) y \(C\)
      (en cualquier dirección)
    \item
      Haga la movida legal entre \(B\) y \(C\)
      (en cualquier dirección)
    \end{itemize}
    Si el número de discos es impar,
    repita lo siguiente hasta completar la tarea:
    \begin{itemize}
    \item
      Haga la movida legal entre \(A\) y \(C\)
      (en cualquier dirección)
    \item
      Haga la movida legal entre \(A\) y \(B\)
      (en cualquier dirección)
    \item
      Haga la movida legal entre \(B\) y \(C\)
      (en cualquier dirección)
    \end{itemize}
    \begin{enumerate}
    \item
      Determine el número de movidas para traspasar \(n\) discos
      de \(A\) a \(C\).
    \item
      Demuestre que esta estrategia resuelve el problema.
    \end{enumerate}
  \end{enumerate}

% To do:
% - Exercises (e.g. Smid)

\bibliography{../referencias}

%%% Local Variables:
%%% mode: latex
%%% TeX-master: "../INF-221_notas"
%%% ispell-local-dictionary: "spanish"
%%% End:

% LocalWords:  Recursión NFA DFA INF subproblemas printf C recursión
% LocalWords:  subproblema english recursion fairy Hanoi XIX ht c to
% LocalWords:  recursivamente Pseudocódigo recurrencia recurrencias
% LocalWords:  caracter string Mergesort pseudocódigo mergesort fill
% LocalWords:  equilátero Édouard cycle Move disk from cl
