\bibliographystyle{babplain-fl}

\chapter{Recorrido de Grafos}
\label{cha:recorrido-grafos}

  Un grafo describe las conexiones entre objetos
  (vértices).
  Como tal,
  es una estructura cómoda para describir una gran variedad de situaciones,
  manipular estas estructuras es una tarea común.
  Hay una variedad de situaciones en las que se debe revisar completo
  algún grafo.
  Ejemplos típicos son los recorridos estándar
  (preorden, postorden, inorden)
  de árboles binarios
  (que en realidad no son grafos).

  Tradicionalmente se representan grafos
  mediante \emph{matrices de adyacencia}
  (una matriz de booleanos indexada por vértices,
   verdadero indica que los vértices son vecinos)
  o \emph{listas de adyacencia}
  (se asocia una lista de vecinos a cada vértice).
  Otra opción
  (particularmente para digrafos,
   en los cuales los arcos tienen una dirección)
  es representar vértices por nodos en una estructura enlazada
  y los arcos como punteros.
  En muchos casos de interés,
  el grafo a recorrer no se conoce explícitamente,
  dado un vértice en él se generan descendientes (vecinos) bajo demanda.

  Como es un área de múltiples aplicaciones,
  hay \emph{\foreignlanguage{english}{software}} genérico disponible
  para muchas tareas comunes,
  como el descrito por Knuth~%
    \cite{knuth09:_sgb}.

  Para uniformidad,
  supondremos un grafo \(G = (V, E)\)
  de vértices \(V\) y arcos \(E\).
  Para referirnos a los vértices de \(G\) anotamos \(V_G\),
  para sus arcos usaremos \(E_G\).
  El vecindario
  (\emph{\foreignlanguage{english}{neighborhood}})
  de \(v \in V_G\) lo anotaremos:
  \begin{equation*}
    N_G(v)
      = \{ u \in V_G \colon u v \in E_G \}
  \end{equation*}

  En nuestros algoritmos
  usaremos conjuntos (\emph{\foreignlanguage{english}{set}}),
  pilas (\emph{\foreignlanguage{english}{stack}}) y
  colas (\emph{\foreignlanguage{english}{queue}})
  como estructuras de datos.
  Bibliotecas proveen estas estructuras básicas para muchos lenguajes,
  es fácil armar al menos versiones rudimentarias
  (ineficientes o limitadas)
  si no están disponibles.
  Los algoritmos que discutiremos
  son aplicables con modificaciones menores a grafos dirigidos
  (digrafos)
  y a estructuras que no son grafos
  (como árboles binarios o árboles ordenados).

  Muchos algoritmos usarán un conjunto \(C\),
  comúnmente llamado \emph{cerrado}
  (\emph{\foreignlanguage{english}{closed}} en inglés)
  para registrar los vértices ya considerados
  de forma de evitar caer en ciclos.
  Si sabemos que la estructura es acíclica,
  podemos obviar el conjunto de vértices ya visitados.

  Generalmente no basta con dar con un vértice particular,
  se requiere el camino que lleva a él.
  La técnica general para acomodar esto
  es almacenar los vértices visitados,
  y a cada uno de ellos asociar el vértice padre
  (desde el cual llegamos a él).
  Recorriendo estas listas obtenemos el camino a la meta
  en reversa.

\section{Búsqueda en profundidad}
\label{seq:DFS}

  En inglés,
  se le conoce como \emph{\foreignlanguage{english}{depth first search}},
  abreviado DFS.
  Es recorrer un camino del grafo hasta el final
  (llegar a un vértice que no tenga vecinos no visitados aún),
  para devolverse al vértice anterior e continuar desde allí.

\subsection{Búsqueda en profundidad -- versión recursiva}
\label{sec:DFS-recursive}

  La forma más directa de describir esto es mediante un programa recursivo,
  vea el algoritmo~\ref{alg:DFS-recursive}.
  \begin{algorithm}[H]
    \DontPrintSemicolon

    \(C \gets \varnothing\) \;

    \Procedure{\(\mathrm{DFS}(G, s)\)}{
        \(C \gets C \cup \{ s \}\) \;
        \(\mathrm{process}(s)\) \;
        \ForEach{\(v \in N_G(s)\)}{
          \If{\(v \notin C\)}{
            \(\mathrm{DFS}(G, v)\) \;
        }
      }
    }
    \caption{Búsqueda en profundidad -- versión recursiva}
    \label{alg:DFS-recursive}
  \end{algorithm}

\subsection{Clasificar arcos}
\label{sec:clasify-arcs}

  Es claro que los arcos que sigue búsqueda en profundidad
  forman un árbol recubridor de \(G\) si este es conexo.
  A sus arcos los llamamos \emph{arcos de árbol}
  (en inglés \emph{\foreignlanguage{english}{tree edges}})
  o \emph{arcos hacia adelante}
  (en inglés \emph{\foreignlanguage{english}{forward edges}}).
  A los demás arcos de \(G\)
  (que llevan a vértices ya visitados)
  los llamamos \emph{arcos reversos}
  (en inglés \emph{\foreignlanguage{english}{back edges}}).

  Note que esta clasificación depende de las elecciones del algoritmo,
  no son realmente propiedades del grafo.

\subsection{Búsqueda en profundidad -- versión iterativa}
\label{sec:DFS-iterative}

  Notando que lo que hacemos es dejar pendientes los vértices no visitados
  para profundizar la búsqueda desde el actual,
  podemos evitar la recursión explícita
  manejando la pila de vértices pendientes manualmente.
  Vea el algoritmo~\ref{alg:DFS-iterative}.
  \begin{algorithm}[ht]
    \DontPrintSemicolon

    \Procedure{\(\mathrm{DFS}(G, s)\)}{
      \(S \gets \mathrm{Stack}\);
      \(\mathrm{push}(S, s)\);
      \(C \gets \varnothing\) \;
      \While{\(\neg \mathrm{empty}(S)\)}{
        \(s \gets \mathrm{pop}(S)\) \;
        \If{\(s \notin C\)}{
          \(C \gets C \cup \{ s \}\) \;
          \(\mathrm{process}(s)\) \;
          \ForEach{\(v \in N_G(s)\)}{
            \If{\(v \notin C\)}{
              \(\mathrm{push}(S, v)\) \;
            }
          }
        }
      }
    }
    \caption{Búsqueda en profundidad -- versión iterativa}
    \label{alg:DFS-iterative}
  \end{algorithm}

\section{Búsqueda a lo ancho}
\label{sec:BFS}

  En inglés,
  se le conoce como \emph{\foreignlanguage{english}{breadth first search}},
  abreviado BFS.

  La idea acá es visitar los vecinos de \(s\),
  luego de haberlos visitado todos
  comenzamos a revisar los vecinos de los vecinos de \(s\),
  y así sucesivamente.

  Sorprendentemente es un cambio menor
  respecto de la búsqueda en profundidad iterativa,
  cambiar la pila por una cola de vértices pendientes,
  vea el algoritmo~\ref{alg:BFS}.
  \begin{algorithm}[ht]
    \DontPrintSemicolon

    \Procedure{\(\mathrm{BFS}(G, s)\)}{
      \(Q \gets \mathrm{Queue}\);
      \(\mathrm{enqueue}(Q, s)\);
      \(C \gets \varnothing\) \;
      \While{\(\neg \mathrm{empty}(Q)\)}{
        \(s \gets \mathrm{dequeue}(Q)\) \;
        \If{\(s \notin C\)}{
          \(C \gets C \cup \{ s \}\) \;
          \(\mathrm{process}(s)\) \;
          \ForEach{\(v \in N_G(s)\)}{
            \If{\(v \notin C\)}{
              \(\mathrm{enqueue}(Q, v)\) \;
            }
          }
        }
      }
    }
    \caption{Búsqueda a lo ancho}
    \label{alg:BFS}
  \end{algorithm}

\section{Comentarios}
\label{sec:comments-graph-traversal}

  El programa más simple es búsqueda en profundidad recursiva.
  Búsqueda en profundidad tiene la virtud de solo usar espacio
  (en la pila explícita en la versión iterativa,
   o en la pila implícita usada para recursión)
  proporcional a la profundidad máxima.

  Búsqueda a lo ancho garantiza hallar el vértice más cercano,
  al ir desarrollando el grafo en oleadas.
  Requiere almacenar gran parte del árbol sobre los nodos considerados.
  Al no revisar los vértices en orden de hacer/deshacer,
  para aplicaciones de revisión sistemática
  no basta con una estructura global.

\section{Aplicaciones}
\label{sec:graph-traversal-uses}

  Un uso simple es determinar si un grafo es conexo,
  o identificar sus componentes conexos.

  Determinar si un grafo es bipartito es tomar un vértice,
  asignarle un color y darle el color contrario a sus vecinos;
  esto se repite hasta colorear todos los vértices o hallar una contradicción.

  Muchos de los algoritmos sobre grafos que veremos
  son variantes de búsqueda en su corazón.

\bibliography{../referencias}

%%% Local Variables:
%%% mode: latex
%%% TeX-master: "../INF-221_notas"
%%% ispell-local-dictionary: "spanish"
%%% End:

