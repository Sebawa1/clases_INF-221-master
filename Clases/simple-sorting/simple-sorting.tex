\bibliographystyle{babplain-fl}

\chapter{Métodos simples de ordenamiento}
\label{cha:metodos-simples-ordenamiento}

  Como una ilustración de técnicas de análisis de algoritmos más avanzadas,
  analizaremos los métodos de ordenamiento más simples.
  Tenemos los métodos
  de la burbuja
  (listado~\ref{lst:bubblesort},
   ver también la discusión histórica de Astrachan~%
     \cite{astrachan03:_bubble_sort}),
  selección,
  (listado~\ref{lst:selection}),
  e inserción
  (listado~\ref{lst:insertion}).
  Es fácil ver que todos ellos tienen tiempo de ejecución \(O(n^2)\).
  El método de selección tiene mejor caso \(O(n^2)\),
  los otros dos tienen mejor caso \(O(n)\).
  \lstinputlisting[float,
                   language = C,
                   firstline = 8,
                   caption = {Método de la burbuja},
                   label = lst:bubblesort]
                  {code/bubblesort.c}
  \lstinputlisting[float,
                   language = C,
                   firstline = 8,
                   caption = {Método de selección},
                   label = lst:selection]
                  {code/selection.c}
  \lstinputlisting[float,
                   language = C,
                   firstline = 8,
                   caption = {Método de inserción},
                   label = lst:insertion]
                  {code/insertion.c}

\section{Rendimiento de métodos simples de ordenamiento}
\label{sec:rendimiento-metodos-simples}

  Nos interesa obtener información más detallada que esta
  sobre el rendimiento de estos algoritmos.
  En particular,
  interesa el tiempo promedio de ejecución.
  Para ello debemos considerar una distribución de los datos de entrada
  (valores repetidos,
   orden original del arreglo).
  Para simplificar,
  supondremos que no hay datos repetidos,
  y como los algoritmos únicamente comparan elementos
  podemos asumir que la entrada es una permutación de \(1, \dotsc, n\).
  O sea,
  requerimos la distribución de las permutaciones dadas al algoritmo.
  Esto en general es imposible de conseguir
  (y engorroso de tratar),
  así que supondremos que todas las permutaciones son igualmente probables.
  Esto reduce el análisis detallado a derivar propiedades promedio
  de las permutaciones.
  \begin{definition}
    Una \emph{inversión} de la permutación \(\pi\) de \(1, \dotsc, n\)
    es un par de índices \(i\), \(j\) tales que \(i < j\)
    y \(\pi(i) > \pi(j)\).
  \end{definition}
  El número mínimo de inversiones es \num{0}
  (el arreglo ordenado no tiene inversiones),
  el máximo es \(n (n - 1) / 2\)
  (en el arreglo ordenado de mayor a menor
   el elemento en la posición~\(i\)
   participa en \(i - 1\) inversiones con elementos previos,
   sumando para \(1 \le i \le n\) se tiene el valor citado).

  Es claro que en el método de la burbuja cada intercambio
  elimina exactamente una inversión.
  Vale decir,
  el número de asignaciones de elementos
  es tres veces el número de inversiones.

  El método de inserción funciona esencialmente como el de la burbuja,
  solo que en vez de intercambiar en cada paso
  deja un espacio libre en la posición original,
  copia cada elemento una posición hacia arriba
  si es mayor que el elemento bajo consideración,
  moviendo la posición libre hacia abajo;
  finalmente ubica el elemento en su posición
  (la que queda libre después de los malabares anteriores).
  Compare con la figura~\ref{fig:insercion}.
  \begin{figure}[ht]
    \centering
    \begin{tikzpicture}[scale = 0.75]
        \draw[thick] (0, 0) rectangle (13, 1);
        \draw[thick] (2, 0) rectangle (3, 1);
        \draw[thick, fill = blue!10] (3, 0) rectangle (9, 1);
        \draw[thick] (8, 0) rectangle (9, 1);
        \draw[thick] (8, 3) rectangle (9, 4);
          \node at (9.1, 3.5) [right] {tmp};

        \draw[-latex'] (8.5, 1.1) -- (8.5, 2.9);
        \draw (8.6, 2) node[circle, inner sep = 0.5pt, draw, right] {\num{1}};

        \draw[-latex'] (4, 0.5) -- (7, 0.5);
          \draw (5.5, 1.15) node[circle, inner sep = 0.5pt, draw, above]
             {\num{2}};

        \draw[-latex'] (7.9, 3.5) to [in = 75, out = 180] (2.5, 1.1);
           \draw (4, 3.3) node[circle, inner sep = 0.5pt, draw] {\num{3}};
      \end{tikzpicture}
    \caption{Operación del método de inserción}
    \label{fig:insercion}
  \end{figure}
  Nuevamente,
  si suponemos que el espacio vacío
  eventualmente será ocupado por el valor temporal,
  cada asignación elimina una inversión.
  Esto se resume en dos asignaciones para cada elemento,
  y una asignación adicional para cada inversión.

\section{Funciones generatrices cumulativas}
\label{sec:func-gener-cumul}

  Comparar los métodos
  es entonces esencialmente obtener el número medio de inversiones
  en las permutaciones de \(1, \dotsc, n\).
  Para ello recurrimos a nuestra técnica preferida,
  funciones generatrices.
  De manera muy similar
  a como contabilizamos las estructuras de un tamaño dado
  mediante funciones generatrices
  podemos representar el total de alguna característica.
  Dividiendo por el número de estructuras del tamaño respectivo
  tenemos el promedio del valor de interés.
  Véase el apéndice~\ref{apx:symbolic-method-dummies}
  o el apunte de Fundamentos de Informática~%
    \cite{brand17:_fundamentos_informatica}.

\section{Análisis de burbuja e inserción}
\label{sec:analisis-burbuja-insercion}

  Nuestros objetos de interés son permutaciones,
  objetos rotulados.
  Corresponde usar funciones generatrices exponenciales.

  Anotemos \(\iota(\pi)\) para el número de inversiones
  de la permutación \(\pi\),
  y definamos la función generatriz cumulativa:
  \begin{equation}
    \label{eq:I-def}
    I(z)
      = \sum_{\pi \in \mathscr{P}}
          \iota(\pi) \frac{z^{\lvert \pi \rvert}}{\lvert \pi \rvert !}
  \end{equation}
  En particular,
  nos interesa el número promedio de inversiones
  para permutaciones de tamaño~\(n\).

  Podemos describir permutaciones mediante la expresión simbólica:
  \begin{equation}
    \label{eq:P-class}
    \mathscr{P}
      = \mathscr{E} + \mathscr{P} \star \mathscr{Z}
  \end{equation}
  Vale decir,
  una permutación es vacía
  o es una permutación combinada con un elemento adicional.
  Dada la permutación \(\pi\)
  construimos permutaciones de largo \(\lvert \pi \rvert + 1\)
  añadiendo un nuevo elemento vía la operación \(\star\).
  Si elegimos \(j\) como nuevo último elemento,
  habrán \(j - 1\) elementos menores que él antes,
  o sea serán \(j - 1\) inversiones adicionales.
  Estamos creando \(\lvert \pi \rvert + 1\) nuevas permutaciones,
  cada una de las cuales conserva las inversiones que tiene,
  y agrega entre \num{0} y \(\lvert \pi \rvert\) nuevas inversiones
  dependiendo del valor elegido como último.
  El total de inversiones en el conjunto de permutaciones así creado
  a partir de \(\pi\) es:
  \begin{equation}
    \label{eq:iota-decomposed}
    (\lvert \pi \rvert + 1) \iota(\pi)
      + \sum_{0 \le k \le \lvert \pi \rvert} k
      = (\lvert \pi \rvert + 1) \iota(\pi)
          +  \frac{\lvert \pi \rvert ( \lvert \pi \rvert + 1)}{2}
  \end{equation}
  Con esto tenemos la descomposición para la función generatriz cumulativa
  (la permutación de cero elementos no tiene inversiones):
  \begin{align}
    I(z)
      &= \iota(\varepsilon)
           + \sum_{\pi \in \mathscr{P}}
               \left(
                 (\lvert \pi \rvert + 1) \iota(\pi)
                     +	\frac{\lvert \pi \rvert ( \lvert \pi \rvert + 1)}{2}
               \right)
               \frac{z^{\lvert \pi \rvert + 1}}{(\lvert \pi \rvert + 1)!}
                   \label{eq:Inv-decomposed} \\
      &= \sum_{\pi \in \mathscr{P}}
           \iota(\pi) \frac{z^{\lvert \pi \rvert + 1}}{\lvert \pi \rvert !}
           + \frac{1}{2}
               \sum_{\pi \in \mathscr{P}}
                 \frac{z^{\lvert \pi \rvert + 1}}{\lvert \pi \rvert !}
                 \lvert \pi \rvert \notag \\
   \intertext{Como hay \(k!\) permutaciones de tamaño \(k\),
              sumando sobre tamaños resulta:}
      &= z I(z) + \frac{1}{2} z \sum_{k \ge 0} k z^k \notag \\
      &= z I(z) + \frac{z^2}{2 (1 - z)^2}
  \end{align}
  Despejando:
  \begin{equation}
    \label{eq:Inv-explicit}
    I(z)
      = \frac{1}{2} \frac{z^2}{(1 - z)^3}
  \end{equation}
  Obtenemos el número promedio de inversiones directamente,
  ya que hay \(n!\) permutaciones de tamaño \(n\),
  y el promedio casualmente es el coeficiente de \(z^n\)
  en la función generatriz exponencial:
  \begin{align}
    \Exp_n[\iota]
      &= [z^n] I(z)
           \label{En-iota} \\
      &= \frac{1}{2} [z^n] \frac{z^2}{(1 - z)^3} \notag \\
      &= \frac{1}{2} [z^{n - 2}] (1 - z)^{-3} \notag \\
      &= \frac{1}{2} \binom{-3}{n - 2} \notag \\
      &= \frac{1}{2} \binom{n - 2 + (3 - 1)}{3 - 1} \notag \\
      &= \frac{1}{2} \binom{n}{2}
           \notag \\
      &= \frac{n (n - 1)}{4}
          \label{En-iota-explicit}
  \end{align}
  Podemos resumir las anteriores
  como número asintótico de asignaciones al ordenar \(n\) elementos
  en el cuadro~\ref{tab:bubble-insertion}.
  \begin{table}[ht]
    \centering
    \begin{tabular}{l*{3}{|>{\(}c<{\)}}}
      \multicolumn{1}{c|}{\textbf{Algoritmo}} &
        \multicolumn{1}{c|}{\textbf{Min}} &
        \multicolumn{1}{c|}{\textbf{Prom}} &
        \multicolumn{1}{c}{\textbf{Máx}} \\
       \hline
       Burbuja	 & 0   & 3 n^2 / 4 & 3 n^2 / 2 \\
       Inserción & 2 n & n^2 / 4   &   n^2 / 2
      \end{tabular}
    \caption{Comparación entre métodos de burbuja e inserción}
    \label{tab:bubble-insertion}
  \end{table}
  Queda claro
  (salvo para optimistas incurables)
  que el método de inserción es mejor.

\section{Análisis de selección}
\label{sec:analisis-seleccion}

  El método de selección tiene sentido
  si queremos minimizar el número de copias
  (podemos simplemente copiar y comparar claves,
   y copiar elementos solo para ubicarlos en su lugar).
  En tal caso interesa fundamentalmente el número de asignaciones.

  Para el método de selección el número de asignaciones
  está dado por el número de veces que hallamos un elemento mayor,
  ver el listado~\ref{lst:maximo}.
  \lstinputlisting[float,
                   language=C,
                   firstline = 6,
                   xleftmargin=3em, numbers=left,
                   caption={Hallar el máximo},
                   label=lst:maximo]
                   {code/maximum.c}
  o sea,
  el número de máximos de izquierda a derecha en la permutación.
  Todas las operaciones se efectúan \(n\)~veces,
  salvo las actualizaciones a la variable~\lstinline[language = C]!max!.
  Es evidente que el número de veces que se actualiza
  \lstinline[language = C]!max! es \(O(n)\),
  pero interesa una respuesta más precisa.
  Para obtener este valor,
  podemos optar por funciones generatrices cumulativas
  o bivariadas.
  Exploraremos ambas opciones como ejemplos en lo que sigue.

  Si suponemos que todos los valores son diferentes,
  y que todas las maneras de ordenarlos son igualmente probables,
  estamos buscando el número promedio de máximos de izquierda a derecha
  de permutaciones.
  Podemos describir la clase de permutaciones simbólicamente
  como:
  \begin{equation}
    \label{eq:P-class-again}
    \mathscr{P}
      = \mathscr{E} + \mathscr{P} \star \mathscr{Z}
  \end{equation}
  Llamaremos \(\chi(\pi)\) al número de máximos de izquierda a derecha
  en la permutación \(\pi\).

\subsection{Función generatriz cumulativa}
\label{sec:max-fg-cumulativa}

  Definimos la función generatriz cumulativa exponencial:
  \begin{equation}
    \label{eq:max-fg-cum}
    \widehat{C}(z)
      = \sum_{\pi \in \mathscr{P}}
          \chi(\pi) \frac{z^{\lvert \pi \rvert}}{\lvert \pi \rvert !}
  \end{equation}
  Como el último elemento de la permutación
  es un máximo de izquierda a derecha
  si es el máximo de todos ellos
  (y los demás máximos de izquierda a derecha se mantienen al rerotular),
  usando la convención de Iverson
  podemos expresar el número de máximos de izquierda a derecha
  en la permutación resultante de \(\pi \star (1)\)
  si se asigna el rótulo \(j\) al elemento nuevo como:
  \begin{equation}
    \label{eq:app:chi+1}
    \chi(\pi) + [j = \lvert \pi \rvert + 1]
  \end{equation}
  con lo que en total
  para el conjunto de permutaciones \(\pi \star (1)\) es:
  \begin{equation*}
    \sum_{1 \le j \le \lvert \pi \rvert + 1}
      (\chi(\pi) + [j = \lvert \pi \rvert + 1])
      = (\lvert \pi \rvert + 1 ) \chi(\pi)
          + 1
  \end{equation*}
  con lo que de~\eqref{eq:P-class-again} resulta:
  \begin{align*}
    \widehat{C}(z)
      &= \chi(\varepsilon) \frac{z^0}{0!}
           + \sum_{\pi \in \mathscr{P}}
               ((\lvert \pi \rvert + 1) \chi(\pi) + 1)
                 \frac{z^{\lvert \pi \rvert + 1}}{(\lvert \pi \rvert + 1)!} \\
      &= 0
           + z \sum_{\pi \in \mathscr{P}}
                 \chi(\pi) \frac{z^{\lvert \pi \rvert}}{\lvert \pi \rvert !}
           + \sum_{\pi \in \mathscr{P}}
               \frac{z^{\lvert \pi \rvert + 1}}{(\lvert \pi \rvert + 1)!} \\
  \intertext{Como hay \(n!\) permutaciones de largo \(n\),
             la última suma se simplifica:}
      &= z \widehat{C}(z)
           + \sum_{n \ge 0} n! \frac{z^{n + 1}}{(n + 1)!} \\
             \notag \\
      &= z \widehat{C}(z)
           + \sum_{n \ge 0} \frac{z^{n + 1}}{n + 1} \\
      &= z \widehat{C}(z) + \ln \frac{1}{1 - z}
  \end{align*}
  Despejamos:
  \begin{equation}
    \label{eq:max-fg-cum-explicit}
    \widehat{C}(z)
      = \frac{1}{1 - z} \ln \frac{1}{1 - z}
  \end{equation}
  Nos interesa el coeficiente de \(z^n\) de~\eqref{eq:max-fg-cum-explicit},
  que casualmente es directamente el valor promedio que buscamos
  (ver por ejemplo el apunte de Fundamentos de Informática~%
     \cite{brand17:_fundamentos_informatica}
   para desarrollo de las funciones generatrices empleadas):
  \begin{equation}
    \label{eq:1}
    [z^n] \widehat{C}(z)
      = H_n
  \end{equation}
  En promedio a \lstinline[language = C]!max! se asigna un nuevo valor
  \(H_n = \ln n + \gamma + O(1/n)\) veces
  al buscar el máximo de \(n\) valores.

\subsection{Función generatriz bivariada}
\label{sec:max-fg-bivariada}

  La función generatriz de probabilidad de que una permutación de tamaño \(n\)
  tenga \(k\) máximos de izquierda a derecha es:
  \begin{equation}
    \label{eq:M-pgf}
    M(z, u)
      = \sum_{\pi \in \mathscr{P}}
          \frac{z^{\lvert \pi \rvert}}{\lvert \pi \rvert !}
            u^{\chi(\pi)}
  \end{equation}
  Esto casualmente es la función generatriz exponencial bivariada
  correspondiente a la clase~\eqref{eq:P-class-again}.

  Razonando como en la sección~\ref{sec:max-fg-cumulativa}
  obtenemos:
  \begin{align}
    M(z, u)
      &= \frac{z^{\lvert \varepsilon \rvert}}{\lvert \varepsilon \rvert !}
           u^{\chi(\varepsilon)}
           + \sum_{\pi \in \mathscr{P}}
               \sum_{1 \le j \le \lvert \pi \rvert + 1}
                 \frac{z^{\lvert \pi \rvert + 1}}
                      {(\lvert \pi \rvert + 1)!}
                   u^{\chi(\pi) + [j = \lvert \pi \rvert + 1]}
                      \label{eq:M-decomposed} \\
      &= \frac{z^0}{0!} u^0
           + \sum_{\pi \in \mathscr{P}}
               \frac{z^{\lvert \pi \rvert + 1}}{(\lvert \pi \rvert + 1)!}
                  u^{\chi(\pi)}
               \sum_{1 \le j \le \lvert \pi \rvert + 1}
                 u^{[j = \lvert \pi \rvert + 1]}
                      \notag  \\
      &= 1
           + \sum_{\pi \in \mathscr{P}}
               \frac{z^{\lvert \pi \rvert + 1}}{(\lvert \pi \rvert + 1)!}
                  u^{\chi(\pi)} (\lvert \pi \rvert + u)
                      \label{eq:M-decomposed-result}
  \end{align}
  Derivando respecto de \(z\)
  (indicamos derivadas por subíndices para simplificar notación):
  \begin{align*}
    M_z(z, u)
      &= \sum_{\pi \in \mathscr{P}}
           \frac{z^{\lvert \pi \rvert}}{\lvert \pi \rvert !}
           u^{\chi(\pi)}
           (\lvert \pi \rvert + u) \\
      &= z M_z(z, u) + u M(z, u)
  \end{align*}
  Vale decir:
  \begin{equation}
    \label{eq:M-pde}
    (1 - z) M_z(z, u) - u M(z, u)
      = 0
  \end{equation}
  En~\eqref{eq:M-pde} la variable~\(u\) interviene como parámetro,
  esta es una ecuación diferencial ordinaria.
  Como \(M(0, u) = 1\),
  la solución es:
  \begin{equation*}
    \label{eq:M-solution}
    M(z, u)
      = \left( \frac{1}{1 - z} \right)^u
  \end{equation*}
  Las derivadas de interés son:
  \begin{align*}
    M_u(z, u)
      &= \frac{1}{(1 - z)^u} \ln \frac{1}{1 - z} \\
    [z^n] M_u(z, 1)
      &= H_n \\
    M_{u u}(z, u)
      &= \frac{1}{(1 - z)^u} \ln^2 \frac{1}{1 - z} \\
    [z^n] M_{u u}(z, 1)
      &= \sum_{1 \le k \le n} H_k \notag \\
      &= (n + 1) H_n - n
  \end{align*}
  Obtenemos:
  \begin{align}
    \Exp[ \chi_n ]
      &= [z^n] M_u (z, 1) \\
      &= H_n \\
    \var[ \chi_n ]
      &= M_{u u} (z, 1) - \left( M_u (z, 1) \right)^2 \notag \\
      &= ((n + 1) H_n - n) + H_n - H_n^2 \notag \\
      &= (n + 2) H_n - n - H_n^2
  \end{align}

\subsection{Ordenamiento por selección}
\label{sec:ordenamiento-seleccion}

  Volvamos a nuestro algoritmo de ordenamiento.
  Llamemos \(X_i\) al número de asignaciones a \lstinline[language = C]!max!
  en las distintas rondas del algoritmo
  (ciclo externo sobre \lstinline[language = C]!i!).
  Nos interesan:
  \begin{align}
    \Exp\left[ \sum_{1 \le i \le n} X_i \right]
      &= \sum_{1 \le i \le n} \Exp[ X_i ] \notag \\
      &= \sum_{1 \le i \le n} H_i \notag \\
      &= (n + 1) H_n - n
  \end{align}
  Vemos que los \(X_i\) son variables independientes,
  así que:
  \begin{align*}
    \var\left[ \sum_{1 \le i \le n} X_i \right]
      &= \sum_{1 \le i \le n} \var[ X_i ] \\
      &= \sum_{1 \le i \le n} ((i + 2) H_i - i - H_i^2) \\
      &= \sum_{1 \le i \le n} i H_i
           + 2 \sum_{1 \le i \le n} H_i
           - \frac{n (n + 1)}{2}
           - \sum_{1 \le i \le n} H_i^2
      &= \sum_{1 \le i \le n} i H_i
           + 2 ((n + 1) H_n - n)
           - \frac{n (n + 1)}{2}
           - \sum_{1 \le i \le n} H_i^2
  \end{align*}
  Atacamos las distintas sumas:
  \begin{align*}
    \sum_{1 \le i \le n} i H_i
      &= [z^n]
           \frac{z}{1 - z} \mathrm{D} \frac{1}{1 - z} \ln \frac{1}{1 - z} \\
      &= [z^n] \left(
                 \frac{z}{(1 - z)^3} \ln \frac{1}{1 - z}
                   + \frac{z}{(1 - z)^3}
               \right) \\
      &= [z^{n - 1}] \frac{1}{(1 - z)^3} \ln \frac{1}{1 - z}
          + \binom{-3}{n - 1} \\
      &= \binom{n - 1 + 2}{2} (H_{n - 1 + 2} - H_2)
           + \binom{n - 1 + 3 - 1}{3 - 1} \\
      &= \frac{n (n + 1)}{2} \left( H_{n + 1} - \frac{3}{2} \right)
           + \frac{(n + 1) (n + 2)}{2} \\
      &= \frac{n (n + 1)}{2} \left( H_{n + 1} - \frac{1}{2} \right) \\
      &= \frac{n (n + 1)}{2} H_n - \frac{n (n - 1)}{4}
  \end{align*}
  La otra suma se puede resolver sumando por partes:
    % MSE 933656
  \begin{align*}
    \sum_{1 \le i \le n} H_i^2
      &= ((n + 1) H_n - n) H_n
           - \sum_{1 \le i \le n - 1} \frac{(i + 1) H_i - i}{i + 1} \\
      &= (n + 1) H_n^2 - n H_n
          - (n + 1) H_n + n + H_n + (n - 1) - H_n + 1 \\
      &= (n + 1) H_n^2 - (2 n + 1) H_n + 2 n
  \end{align*}
  Uniendo las piezas y simplificando:
  \begin{equation}
    \var\left[ \sum_{1 \le i \le n} X_i \right]
      = \frac{n^2 + 9 n + 6}{2} H_n
          - (n + 1) H_n^2
          - \frac{n (3 n + 17)}{4}
  \end{equation}
  O sea,
  asintóticamente el número de asignaciones
  (recuerde que un intercambio son tres asignaciones)
  en el método de selección
  es mínimo \num{0}, máximo \(3 n\), promedio \(3 n \ln n\),
  varianza \(9 n^2 \ln n / 2\).

% To do: heapsort

\bibliography{../referencias}

%%% Local Variables:
%%% mode: latex
%%% TeX-master: "../INF-221_notas"
%%% ispell-local-dictionary: "spanish"
%%% End:

% LocalWords:  cumulativas cumulativa Min Prom Máx bivariadas eq max
% LocalWords:  rerotular class again fg cum explicit bivariada pde
% LocalWords:  asintóticamente
