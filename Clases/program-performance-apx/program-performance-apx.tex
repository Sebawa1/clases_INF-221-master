\bibliographystyle{babplain-fl}

\chapter{Rendimiento de programas}
\label{apx:rendimiento-programas}

  Formalizaremos algunas técnicas que hemos usado
  para evaluar rendimiento de programas.
  Detalle exhaustivo de la idea da Knuth~%
    \cite{knuth97:_fundam_algor},
  una discusión más accesible y sistematizada es la de Rossmanith~%
    \cite{rossmanith12:_analysis_algorithms}.

\section{Número de veces que se ejecuta cada línea}
\label{sec:ejecucion-lineas}

  El primer paso para una evaluación rigurosa de los recursos gastados
  es determinar cuántas veces se ejecuta cada línea del algoritmo.
  Contando con costos precisos de cada una de ellas
  puede calcularse el costo total.
  Dividimos el programa en bloques básicos
  (designados por letras mayúsculas, \(A, B, \dotsc\),
  secuencias de instrucciones que siempre se ejecutan juntas,
  y las unimos por arcos dirigidos \(e_i\) entre bloques
  siempre que existe la posibilidad de que el control pase de un bloque a otro.
  Habrá un bloque en el que comienza el programa,
  y bloques en los cuales termina.
  Los conectamos por arcos ficticios.
  Llamaremos \(E_i\) al número de veces
  que el control pasa por el arco \(e_i\).
  La técnica se basa en lo siguiente,
  resultado bastante evidente si se considera que el control
  nunca se ve atrapado en un ciclo infinito:
  \begin{theorem}[Ley de Kirchhoff]
    \label{theo:Kirchhoff}
    Sea \(X\) un bloque básico,
    y sean \(\mathscr{I}\) el conjunto de \(i\) tales que el arco \(e_i\)
    entra a \(X\)
    y \(\mathscr{O}\) el conjunto de \(i\) tales que el arco \(e_i\)
    sale de \(X\).
    Entonces:
    \begin{equation*}
      \sum_{i \in \mathscr{I}} E_i
        = \sum_{i \in \mathscr{O}} E_i
    \end{equation*}
    La suma es el número de veces que se ejecutan las instrucciones en \(X\).
  \end{theorem}
  El primer paso es descomponer en bloques básicos,
  e identificar los arcos \(e_i\).
  Luego se halla un árbol recubridor del grafo,
  obviando que los arcos son dirigidos.
  Arcos que no pertenecen al árbol recubridor crean ciclos,
  que llamaremos \emph{ciclos fundamentales}.
  Describimos los ciclos eligiendo un arco perteneciente al árbol recubridor
  como positivo y recorremos el ciclo en esa dirección,
  asignando signos positivo o negativo según si el arco
  va en la dirección en que seguimos el ciclo o no.
  Lo interesante es que cada ciclo da una solución a la ley de Kirchhoff,
  si asignamos \(E_i = 0\) a los arcos que no pertenecen al ciclo
  y \(E_i = 1\) a los que pertenecen.
  Como las ecuaciones son lineales,
  combinaciones lineales de soluciones nuevamente son soluciones.
  Podemos expresar los flujos a través del ciclo \(C_i\) como \(\lambda_i\),
  planteando un sistema de ecuaciones.
  Esto nos permite expresar todos los flujos \(E_i\)
  en términos de un subconjunto de ellos.

\subsection{Descomposición en bloques básicos}
\label{sec:bloques-basicos}

  El primer paso es descomponer el programa en \emph{bloques básicos},
  secuencias de instrucciones que se ejecutan siempre en orden.
  Comienzan en el destino de control de flujo
  y terminan con alguna instrucción que
  (condicionalmente o no)
  cambia el flujo de control.
  Consideremos por ejemplo el algoritmo de Prim,
  algoritmo~\ref{alg:apx:Prim},
  para un árbol recubridor mínimo de grafo.
  Datos de entrada son un grafo conexo \(G = (V, E)\),
  una función de peso \(w \colon E \to \mathbb{R}\)
  y un vértice inicial \(s \in V\)
  (arbitrario).
  Anotaremos \(N(v)\) para el conjunto de vértices vecinos a \(v\) en \(G\).

  Representamos la solución mediante arreglos,
  el arreglo \(\pi[v]\) contendrá el padre en el árbol recubridor mínimo
  con \(s\) de raíz,
  \(\mathrm{key}[v]\) es el costo del camino más corto conocido
  entre el árbol actual y ese vértice.
  Suponemos la operación \(\mathrm{min-from}(M)\)
  que extrae un vértice del conjunto de vértices \(M\)
  con mínimo valor de \(\mathrm{key}[v]\).
  \begin{algorithm}[ht]
    \DontPrintSemicolon\Indp

    \ForEach{\(u \in V\)}{
      \(\mathrm{key}[u] \gets \infty\) \;
      \(\pi[u] \gets \mathrm{\mathbf{nil}}\) \;
    }
    \(\mathrm{key}[s] \gets 0\) \;
    \(M \gets V\) \;
    \While{\(M \ne \varnothing\)}{
      \(u \gets \mathrm{min-from}(M)\) \;
      \ForEach{\(v \in N(u)\)}{
        \If{\((v \in M) \wedge (w(u, v) < \mathrm{key}[v])\)}{
          \(\pi[v] \gets u\) \;
          \(\mathrm{key}[v] \gets w(u, v)\) \;
        }
      }
    }
    \caption{Algoritmo de Prim}
    \label{alg:apx:Prim}
  \end{algorithm}
  Este programa se representa en forma de diagrama de flujo
  (grafo de bloques básicos)
  en la figura~\ref{fig:Prim}.
  La convención es que flujos \textquote{hacia abajo}
  representan salida con condición al final del bloque verdadera,
  flujos \textquote{hacia el lado} representan condiciones falsas.
  \begin{figure}[ht]
    \centering
    \begin{tikzpicture}[block/.style = {rectangle, rounded corners, draw,
                                        minimum width  = 10em,
                                        minimum height =  2em,
                                        text width     =  9em},
                        flow/.style = {-latex'},
                        stflow/.style ={thick, -latex'}]
      \node	      (0) {};
      \node[block, below of = 0, yshift = 1.5em, label = north west:{\(A\)}]
                      (A) {\(M \leftarrow V\)};
      \coordinate[below of = A, yshift = 2.5em]
                      (AB) {};
      \node[block, below of = AB, yshift = 2.5em, label = north west:{\(B\)}]
                      (B) {\(M \ne \varnothing\)};
      \node[block, below of = B, label = north west:{\(C\)}]
                      (C) {\(u \leftarrow \mathrm{any-from}(M)\) \\
                           \(\mathrm{key}[u] \leftarrow \infty\) \\
                           \(\pi[u] \leftarrow \mathrm{\mathbf{nil}}\)};
      \coordinate[below of = C, yshift = 2.5em]
                      (CD) {};
      \node[block, below of = CD, yshift = 2.5em, label = north west:{\(D\)}]
                      (D) {\(\mathrm{key}[s] \leftarrow 0\) \\
                           \(M \leftarrow V\)};
      \node[below of = D, yshift = 2.5em]
                      (DE) {};
      \node[block, below of = DE, yshift = 2.5em, label = north west:{\(E\)}]
                      (E) {\(M \ne \varnothing\)};

      \node[block, below of = E, yshift = 0.5em, label = north west:{\(F\)}]
                      (F) {\(u \leftarrow \mathrm{min-from}(M)\) \\
                           \(N \leftarrow N(u)\)};
      \node[below of = F, yshift = 2.5em]
                      (FG) {};
      \node[block, below of = FG, yshift = 2.5em, label = north west:{\(G\)}]
                      (G) {\(N \ne \varnothing\)};
      \node[block, below of = G, yshift = 0.5em, label = north west:{\(H\)}]
                      (H) {\(v \leftarrow \mathrm{any-from}(N)\) \\
                           \(v \in M\)};
      \node[block, below of = H, yshift = 0.5em, label = north west:{\(I\)}]
                      (I) {\(w(u, v) < \mathrm{key}[v]\)};
      \node[block, below of = I, yshift = 0.5em, label = north west:{\(J\)}]
                      (J) {\(\pi[v] \leftarrow v\) \\
                           \(\mathrm{key}[v] \leftarrow w(u, v)\)};
      \coordinate[below of = J, yshift = 2.5em]
                      (N) {};

      \coordinate[right of = AB, node distance =    8em] (ABR)	{};
      \coordinate[right of = AB, node distance =  2.3em] (BA)	{};
      \coordinate[left of  = B,	 node distance =    8em] (BL)	{};
      \coordinate[left of  = CD, node distance =  2.2em] (DA)	{};
      \coordinate[left of  = CD, node distance =    8em] (CDL)	{};
      \coordinate[right of = CD, node distance =    8em] (CDR)	{};
      \coordinate[right of = DE, node distance =  2.3em] (EA)	{};
      \coordinate[right of = DE, node distance =    8em] (DER)	{};
      \coordinate[left of  = E,	 node distance =   10em] (EL)	{};
      \coordinate[left of  = FG, node distance =  2.3em] (GA)	{};
      \coordinate[left of  = FG, node distance =    8em] (FGL1) {};
      \coordinate[left of  = FG, node distance =   10em] (FGL2) {};
      \coordinate[left of  = FG, node distance =   12em] (FGL3) {};
      \coordinate[right of = G,	 node distance =    8em] (GR)	{};
      \coordinate[left of  = H,	 node distance =    8em] (HL)	{};
      \coordinate[left of  = I,	 node distance =   10em] (IL)	{};
      \coordinate[left of  = N,	 node distance =   12em] (NL)	{};

      \draw[flow]   (0)	   -- node [left]  {\(e_0\)}	(A);
      \draw[stflow] (A)	   -- node [left]  {\(e_1\)}	(B);
      \draw[stflow] (B)	   -- node [left]  {\(e_2\)}	(C);
      \draw[flow]   (C)	   -- (CD) -- (CDR) -- node [right]  {\(e_3\)} (ABR)
                           -- (BA) -- (BA|-B.north);
      \draw[stflow] (B)	   -- (BL) -- node [left] {\(e_4\)} (CDL)
                           -- (DA) -- (DA|-D.north);
      \draw[stflow] (D)	   -- node [left]  {\(e_5\)}	(E);
      \draw[stflow] (E)	   -- node [left]  {\(e_6\)}	(F);
      \draw[stflow] (F)	   -- node [right] {\(e_7\)}	(G);
      \draw[stflow] (G)	   -- node [left]  {\(e_8\)}	(H);
      \draw[stflow] (H)	   -- node [left]  {\(e_9\)}	(I);
      \draw[stflow] (I)	   -- node [left]  {\(e_{10}\)} (J);
      \draw[flow]   (I)	   -- node [above] {\(e_{11}\)} (IL)
                           -- (FGL2);
      \draw[flow]   (J)	   -- (N) -- node [above, near end] {\(e_{12}\)} (NL)
                           -- (FGL3) -- (GA) -- (GA|-G.north);
      \draw[flow]   (H)	   -- (HL) -- node [right] {\(e_{13}\)} (FGL1);
      \draw[flow]   (G)	   -- (GR)  -- node [right]  {\(e_{14}\)} (DER)
                           -- (EA) -- (EA|-E.north);
      \draw[flow]   (E)	   -- node [above, near end] {\(e_{15}\)} (EL);
    \end{tikzpicture}
    \caption{Diagrama de flujo del algoritmo de Prim}
    \label{fig:Prim}
  \end{figure}
  Note que hemos separado la inicialización de los ciclos
  de sus cuerpos,
  se ejecutan aparte.
  Para representar el elegir un vértice cualquiera
  estamos usando la operación
  \(\mathrm{any-from}(M)\),
  que elige un elemento cualquiera del conjunto \(M\),
  lo elimina de este y lo retorna.

\subsection{Aplicar ley de Kirchhoff}
\label{sec:aplicar-Kirchhoff}

  En la figura~\ref{fig:Prim} hemos marcado los arcos
  \(e_1, e_2, e_4, e_5, e_6, e_7, e_8, e_9\) y \(e_{10}\)
  (estamos identificando \(e_0 = e_{15}\),
   un flujo ficticio a través del \textquote{exterior})
  del árbol recubridor elegido en negrita.
  Esto da los ciclos fundamentales representados como:
  \begin{align*}
    C_0
      &= e_0 + e_1 + e_4 + e_5 \\
    C_1
      &= e_2 + e_3 \\
    C_2
      &= e_6 + e_7 + e_{14} \\
    C_3
      &= e_8 + e_{13} \\
    C_4
      &= e_8 + e_9 + e_{11} \\
    C_5
      &= e_8 + e_9 + e_{10} + e_{12}
  \end{align*}
  Por la ley de Kirchhoff podemos representar los flujos como:
  \begin{align*}
    \begin{pmatrix}
      E_0 \\
      E_1 \\
      E_2 \\
      E_3 \\
      E_4 \\
      E_5 \\
      E_6 \\
      E_7 \\
      E_8 \\
      E_9 \\
      E_{10} \\
      E_{11} \\
      E_{12} \\
      E_{13} \\
      E_{14}
    \end{pmatrix}
      &= \lambda_0
           \begin{pmatrix}
             1 \\
             1 \\
             0 \\
             0 \\
             1 \\
             1 \\
             0 \\
             0 \\
             0 \\
             0 \\
             0 \\
             0 \\
             0 \\
             0 \\
             0
           \end{pmatrix}
         + \lambda_1
           \begin{pmatrix}
             0 \\
             0 \\
             1 \\
             1 \\
             0 \\
             0 \\
             0 \\
             0 \\
             0 \\
             0 \\
             0 \\
             0 \\
             0 \\
             0 \\
             0
           \end{pmatrix}
         + \lambda_2
           \begin{pmatrix}
             0 \\
             0 \\
             0 \\
             0 \\
             0 \\
             0 \\
             1 \\
             1 \\
             0 \\
             0 \\
             0 \\
             0 \\
             0 \\
             0 \\
             1
           \end{pmatrix}
         + \lambda_3
           \begin{pmatrix}
             0 \\
             0 \\
             0 \\
             0 \\
             0 \\
             0 \\
             0 \\
             0 \\
             1 \\
             0 \\
             0 \\
             0 \\
             0 \\
             1 \\
             0
           \end{pmatrix}
         + \lambda_4
           \begin{pmatrix}
             0 \\
             0 \\
             0 \\
             0 \\
             0 \\
             0 \\
             0 \\
             0 \\
             1 \\
             1 \\
             0 \\
             1 \\
             0 \\
             0 \\
             0
           \end{pmatrix}
         + \lambda_5
           \begin{pmatrix}
             0 \\
             0 \\
             0 \\
             0 \\
             0 \\
             0 \\
             0 \\
             0 \\
             1 \\
             1 \\
             1 \\
             0 \\
             1 \\
             0 \\
             0
           \end{pmatrix} \\
      &= \begin{pmatrix}
           1 & 0 & 0 & 0 & 0 & 0 \\
           1 & 0 & 0 & 0 & 0 & 0 \\
           0 & 1 & 0 & 0 & 0 & 0 \\
           0 & 1 & 0 & 0 & 0 & 0 \\
           1 & 0 & 0 & 0 & 0 & 0 \\
           1 & 0 & 0 & 0 & 0 & 0 \\
           0 & 0 & 1 & 0 & 0 & 0 \\
           0 & 0 & 1 & 0 & 0 & 0 \\
           0 & 0 & 0 & 1 & 1 & 1 \\
           0 & 0 & 0 & 0 & 1 & 1 \\
           0 & 0 & 0 & 0 & 0 & 1 \\
           0 & 0 & 0 & 0 & 1 & 0 \\
           0 & 0 & 0 & 0 & 0 & 1 \\
           0 & 0 & 0 & 1 & 0 & 0 \\
           0 & 0 & 1 & 0 & 0 & 0
         \end{pmatrix}
           \cdot \begin{pmatrix}
                   \lambda_0 \\
                   \lambda_1 \\
                   \lambda_2 \\
                   \lambda_3 \\
                   \lambda_4 \\
                   \lambda_5
                 \end{pmatrix}
  \end{align*}
  Interesa identificar un conjunto de seis flujos cuyas filas en la matriz
  sean linealmente independientes,
  y expresar las demás en términos de ellas.
  Notamos que \(E_0 = E_1 = E_4 = E_5\),
  \(E_2 = E_3\),
  también \(E_6 = E_7 = E_{14}\)
  y finalmente \(E_{10} = E_{12}\)
  ya que sus filas son iguales.
  Una mirada al diagrama de flujo~\ref{fig:Prim}
  confirma estas igualdades.

  Buscamos ahora expresar los demás flujos
  en términos de un conjunto de flujos linealmente independientes.
  Tenemos varias igualdades ya,
  completando con las expresiones para \(E_8\) y \(E_9\)
  todos quedan expresados en términos de
  \(E_0, E_2, E_6, E_{10}, E_{11}\) y~\(E_{13}\):
  \begin{align*}
    E_1
      &= E_0 \\
    E_3
      &= E_2 \\
    E_4
      &= E_0 \\
    E_5
      &= E_0 \\
    E_7
      &= E_6 \\
    E_8
      &= E_9 + E_{13} \\
    E_9
      &= E_{10} + E_{11} \\
    E_{12}
      &= E_{10} \\
    E_{14}
      &= E_6
  \end{align*}

  Sabemos que el programa se invoca una única vez,
  por lo que \(E_0 = 1\).
  Los números de veces que se repiten los demás ciclos
  deberemos determinarlos.
  Conociendo los flujos,
  podemos calcular cuántas veces se ejecuta cada vértice
  (bloque básico).
  Llamando como el bloque en el diagrama de flujo
  al número de veces que se le ejecuta resulta:
  \begin{align*}
    A
      &= E_0 \\
    B
      &= E_0 + E_3 \\
      &= E_0 + E_2 \\
    C
      &= E_2 \\
    D
      &= E_4 \\
      &= E_0 \\
    E
      &= E_5 + E_{14} \\
      &= E_0 + E_6 \\
    F
      &= E_6 \\
    G
      &= E_7 + E_{11} + E_{12} + E_{13} \\
      &= E_6 + E_{10} + E_{11} + E_{13} \\
    H
      &= E_8 \\
      &= E_9 + E_{13} \\
    I
      &= E_9 \\
    J
      &= E_{10}
  \end{align*}
  Del programa vemos que:
  \begin{align*}
    E_0
      &= 1 \\
    E_2
      &= E_6 = \lvert V \rvert
  \end{align*}
  Los demás valores dependen del grafo,
  del orden en que se eligen nodos
  y de la función \(w\).

\subsection{Llamadas a funciones}
\label{sec:llamadas-funciones}

  En el caso que el programa llame a una función,
  simplemente el costo de esa línea será el costo de esa función
  para esos argumentos específicos.
  En caso que sea una llamada recursiva,
  esto da lugar a una recurrencia.
  Por ejemplo,
  considere el programa~\ref{lst:apx-snapl}.
  \lstinputlisting[language = C++,
                   firstline = 3,
                   xleftmargin = 3em,
                   numbers = none,
                   caption = {El programa \texttt{snapl}},
                   label = lst:apx-snapl]
                  {code/snapl.cc}
  Nos interesa el número \(a_n\) de números
  que escribe la llamada \lstinline[language = C++]!snapl(n)!.

  Resulta la recurrencia:
  \begin{equation*}
    a_n
      = 3 a_{n - 2} + 2 a_{n - 3}
      \quad a_0 = 0, a_1 = 1, a_2 = 0
  \end{equation*}
  Técnicas estándar de solución de recurrencias dan:
  \begin{equation*}
    a_n
      = \frac{2^{n + 1} - (-1)^n (3 n + 2)}{9}
  \end{equation*}

\section{Análisis probabilístico}
\label{sec:analisis-probabilistico}

  Analicemos el programa~\ref{lst:apx-maximo}.
  \lstinputlisting[language = C,
                   firstline = 6,
                   xleftmargin = 3em,
                   numbers = left, stepnumber = 1, numberstyle = {\small},
                   caption = {Hallar el máximo},
                   label = lst:apx-maximo]
                   {code/maximum.c}
  Es claro que\ldots{}

  Nos interesa específicamente el número de asignaciones
  a la variable \lstinline[language = C]!max!.
  Al efecto,
  supondremos que los elementos son todos distintos,
  son una permutación elegida uniformemente al azar de \(n\) elementos.

  El mínimo es \num{1}
  (si \lstinline[language = C]!a[0]! es el máximo).
  La probabilidad de este caso es \(1/n\)
  (hay \(n\) valores posibles de \lstinline[language = C]!a[0]!,
   todos igualmente probables,
   solo uno de ellos corresponde al evento de interés).

  El máximo es \(n\)
  (si vienen en orden).
  La probabilidad de este caso es \(1/n!\)
  (hay \(n!\) permutaciones,
   solo una de ellas con los elementos en orden).

  Para el promedio,
  consideremos el caso que para \(i\) se ejecuta la línea~9.
  Podemos describirlo diciendo que elegimos \(i + 1\) elementos para
  \lstinline[language = C]!a[0]! a \lstinline[language = C]!a[i]!
  entre los \(n\),
  uno de ellos es el máximo
  (en \lstinline[language = C]!a[i]!).
  Se permutan los demás,
  \(i\) antes de \lstinline[language = C]!a[i]!
  y \(n - i - 1\) después.
  Como en total hay \(n!\) permutaciones
  la probabilidad de este evento es:
  \begin{align*}
    \frac{\binom{n}{i + 1} i! (n - i - 1)!}{n!}
      &= \frac{n!}{(i + 1)! (n - i - 1)!} \cdot \frac{i! (n - i - 1)!}{n!} \\
      &= \frac{1}{i + 1}
  \end{align*}
  Por la linealidad del valor esperado,
  el número esperado de veces que se ejecuta la línea~9 es:
  \begin{align*}
    \sum_{1 \le i < n} \Pr[\text{línea 9 se ejecuta para \(i\)}]
      &= \sum_{1 \le i < n} \frac{1}{i + 1} \\
      &= \sum_{2 \le i \le n} \frac{1}{i} \\
      &= H_n - 1
  \end{align*}
  A las asignaciones de la línea~9
  debemos sumar la asignación de la línea~4,
  dando que el valor esperado del número total
  de asignaciones a \lstinline[language = C]!max!
  es \(H_n \sim \ln n\).

  En resumen,
  las líneas 6 y 10 se ejecutan siempre \num{1} vez,
  las líneas 7 y 8 siempre se ejecutan \(n - 1\) veces,
  mientras la línea 9 se ejecuta
  mínimo \num{0}~veces,
  máximo~\(n - 1\) veces,
  en promedio \(H_n - 1\) veces.
  Asignaciones en total son así
  mínimo \num{1}~vez,
  máximo \(n\)~veces,,
  promedio \(H_n \sim \ln n\)~veces.

\bibliography{../referencias}

%%% Local Variables:
%%% mode: latex
%%% TeX-master: "../INF-221_notas"
%%% ispell-local-dictionary: "spanish"
%%% End:

% LocalWords:  recubridor
