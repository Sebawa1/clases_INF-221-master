\bibliographystyle{babplain-fl}

\chapter{Dividir y Conquistar -- Soluciones Exactas}
\label{cha:dividir-conquistar-solucion}

  Lo discutido en el capítulo~\ref{cha:dividir-conquistar}
  da soluciones aproximadas a las recurrencias resultantes
  de algoritmos de dividir y conquistar.
  Las soluciones de las recurrencias exactas
  (con pisos y cielos)
  tienen soluciones muy extrañas.
  Como ejemplo,
  resolveremos exactamente la recurrencia
  para el número de comparaciones de mergesort:
  \begin{equation}
    \label{eq:mergesort-exact}
    M(n)
      = M(\lfloor n / 2 \rfloor)
          + M(\lceil n / 2 \rceil)
          + n
    \quad
    M(0)
      = M(1)
      = 0
  \end{equation}
  Seguimos a Sedgewick y Flajolet~%
    \cite{sedgewick13:_introd_anal_algor}
  para resolver la recurrencia.
  Si definimos diferencias:
  \begin{equation*}
    d(n)
      = M(n + 1) - M(n)
  \end{equation*}
  considerando por separado los casos \(n\) par e impar y simplificando
  concluimos que vale la recurrencia auxiliar:
  \begin{equation*}
    d(n)
      = d(\lfloor n / 2 \rfloor) + 1
    \quad
    d(1)
      = 1
  \end{equation*}
  Iterando esta recurrencia vemos que \(d(n) = \lfloor \log_2 n \rfloor + 1\)
  con lo que la solución es:
  \begin{equation}
    \label{eq:mergesort-interim}
    M(n)
      = n - 1 + \sum_{1 \le k < n} \lfloor \log_2 k \rfloor
  \end{equation}
  Hay varias formas de expresar la suma en forma cerrada.
  Consideremos primero la suma:
  \begin{equation*}
    \sum_{1 \le k < 2^r} \lfloor \log_2 k \rfloor
  \end{equation*}
  Vemos que en el rango \(2^s \le k < 2^{s + 1}\)
  es \(\lfloor \log_2 k \rfloor = s\),
  en ese rango hay \(2^s\) valores.
  O sea:
  \begin{align*}
    \sum_{1 \le k < 2^r} \lfloor \log_2 k \rfloor
      &= \sum_{0 \le s < r} s \cdot 2^s \\
      &= (r - 2) \cdot 2^r + 2
  \end{align*}
  La suma original es esto más algunos términos
  después de la última potencia de dos:
  \begin{align*}
    \sum_{1 \le k < n} \lfloor \log_2 k \rfloor
      &= \sum_{1 \le k < 2^{\lfloor \log_2 n \rfloor}} \lfloor \log_2 k \rfloor
           + \sum_{2^{\lfloor \log_2 n \rfloor} \le k < n}
               \lfloor \log_2 k \rfloor \\
      &= (\lfloor \log_2 n \rfloor - 2) \cdot 2^{\lfloor \log_2 n \rfloor} + 2
           + \sum_{2^{\lfloor \log_2 n \rfloor} \le k < n}
               \lfloor \log_2 n \rfloor \\
      &= (\lfloor \log_2 n \rfloor - 2) \cdot 2^{\lfloor \log_2 n \rfloor} + 2
           + (n - 2^{\lfloor \log_2 n \rfloor})
               \lfloor \log_2 n \rfloor \\
      &= n \cdot \lfloor \log_2 n \rfloor
           - 2^{\lfloor \log_2 n \rfloor + 1}
           + 2
  \end{align*}
  Con esto,
  de~\ref{eq:mergesort-interim}
  obtenemos:
  \begin{equation*}
    \label{eq:mergesort-solution}
    M(n)
      = n \cdot \lfloor \log_2 n \rfloor
          + 2 n
          - 2 {\lfloor \log_2 n \rfloor + 1}
  \end{equation*}
  Usando la parte fraccionaria:
  \begin{equation*}
    \{ x \}
      = x - \lfloor x \rfloor
  \end{equation*}
  podemos expresar esto como:
  \begin{align*}
    M(n)
      &= n \log_2 n - n \{ \log_2 n \}
          + 2 n
          - n \cdot 2^{1 - \{ \log_2 n \}} \\
      &= n \log_2 n
          + n \left(
                1 + 1 - \{ \log_2 n \} - 2^{1 - \{ \log_2 n \}}
              \right) \\
      &= n \log_2 n + n \phi(1 - \{ \log_2 n \})
  \end{align*}
  Acá es:
  \begin{equation}
    \label{eq:phi}
    \phi(x)
      = 1 + x - 2^x
  \end{equation}
  Interesa el máximo de esto en el rango \(0 < x \le 1\)
  (es \(\phi(0) = \phi(1) = 0\);
   esta función es convexa,
   su segunda derivada es negativa),
  que se da para \(x^* = - \ln \ln 2 / \ln 2 \approx 0,5288\),
  donde vale:
  \begin{equation*}
    \phi(x^*)
      = 1 - \frac{\ln \ln 2}{\ln 2} - \frac{1}{\ln 2}
      \approx 0,0861
  \end{equation*}
  O sea,
  tenemos \(0 \le \phi(x) \le 0,0861\).
  Nuestra solución aproximada es muy cercana a la solución exacta.

  Los primeros resultados generales sobre este tipo de recurrencias
  son los de Erdős, Hildebrand, Odlyzko, Pudaite y Reznik~%
    \cite{erdos87:_asymptotic_behavior_family_sequences},
  quienes demuestran
  (usando técnicas muy diferentes)
  que la recurrencia:
  \begin{equation*}
    a(n)
      = \sum_{1 \le k \le s} r_k a(\lfloor n / m_k \rfloor)
      \qquad a(0) = 1
  \end{equation*}
  tiene soluciones diferentes dependiendo si hay enteros \(d, u_k\)
  tales que \(m_k = d^{u_k}\) o no
  (elegimos el valor máximo de \(d\) en este caso,
   o sea,
   los \(u_k\) relativamente primos).
  Sea \(\tau\) la única solución real a:
  \begin{equation*}
    \sum_{1 \le k \le s} \frac{r_k}{m_k^\tau}
      = 1
  \end{equation*}
  Si hay tales enteros,
  llaman el caso \emph{entramado}
  (en inglés, \emph{\foreignlanguage{english}{lattice}}),
  en caso contrario \emph{ordinario}.
  En el caso ordinario,
  la solución cumple:
  \begin{equation*}
    \lim_{n \to \infty} \frac{a(n)}{n^\tau}
      = c_o
  \end{equation*}
  para una constante explícita \(c_o\) fácil de calcular;
  en el caso entramado ese límite no existe,
  pero existe:
  \begin{equation*}
    \lim_{k \to \infty} \frac{a(d^k)}{f^{k \tau}}
      = c_l
  \end{equation*}
  para otra constante fácil de calcular.

  Hwang, Janson y Tsai~%
    \cite{hwang17:_exact_asymp_solut_divid_conquer}
  dan soluciones exactas a recurrencias de la forma
  \begin{equation*}
    f(n)
      = f(\lfloor n / 2 \rfloor) + f(\lceil n / 2 \rceil) + g(n)
  \end{equation*}
  que resultan ser de la forma:
  \begin{equation*}
    f(n)
      = F(n) + n P( \log_2 n) - Q(n)
  \end{equation*}
  donde \(F(n) = 0\) o más que lineal,
  \(P\) es una función con período~\num{1}
  y \(Q(n) = o(n)\).
  Muestran cómo expresar estas funciones explícitamente.

  Drmota y Szpankowski~%
    \cite{drmota13:_master_theorem_discrete}
  discuten recurrencias discretas de la forma
  \begin{equation*}
    T(n)
      = a_n + \sum_{1 \le j \le m} b_j T(\rfloor p_j n + \delta_j \lfloor)
  \end{equation*}
  donde la secuencia \(a_n\) es conocida,
  \(b_j\) y \(\delta_j\) son constantes dadas.

\bibliography{../referencias}

%%% Local Variables:
%%% mode: latex
%%% TeX-master: "../INF-221_notas"
%%% ispell-local-dictionary: "spanish"
%%% End:

% LocalWords:  mergesort english lattice
