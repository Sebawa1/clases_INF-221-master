\bibliographystyle{babplain-fl}

\chapter{Matrimonios Estables}
\label{cha:gale-shapley}

  Nuestro interés principal es algoritmos que operan con objetos discretos,
  de los que estudia la combinatoria.
  Un primer ejemplo muestra que situaciones a primera vista simples
  pueden tener profundidades insospechadas.

\section{El problema}
\label{sec:stable-marriage-problem}

  El problema a resolver es el de matrimonios estables
  (\emph{\foreignlanguage{english}{stable marriage problem}}):
  Dados un conjunto de mujeres \(\mathscr{X}\)
  y otro de hombres \(\mathscr{Y}\),
  donde \(\lvert \mathscr{X} \rvert = \lvert \mathscr{Y} \rvert\),
  cada mujer define un orden de deseabilidad para los hombres,
  y similarmente los hombres con las mujeres,
  donde suponemos que no hay empates.
  \begin{definition}
    Un conjunto de matrimonios se dice \emph{inestable}
    si incluye parejas \(u v\) y \(x y\),
    tales que \(u\) prefiere a \(y\) frente a \(v\)
    y \(y\) prefiere a \(u\) frente a \(x\).
  \end{definition}
  Si el conjunto es inestable,
  \(u\) y \(x\) pueden mejorar sus elecciones divorciándose
  y volviéndose a casar.
  Buscamos matrimonios estables.
  Este problema y variantes aparece en un amplio rango de situaciones,
  resumidas por Iwama y Miyazaki~%
    \cite{iwama08:_stable_marriage_probl_survey}.

  Hallar un conjunto estable de matrimonios parece ser simplemente
  \textquote{cumpla las preferencias},
  pero reflexión más profunda muestra que ni siquiera es obvio
  que tal conjunto existe.
  Es claro que no todos pueden obtener su mejor opción,
  si hay un hombre que es el preferido de dos o más mujeres,
  una de ellas al menos deberá conformarse con otro.

  Una posibilidad es asignar parejas al azar,
  y en caso de que cambiando parejas se pueda mejorar,
  permitir divorcios y nuevos matrimonios.
  Sin embargo,
  el siguiente ejemplo
  (adaptado de Knuth~%
     \cite{knuth96:_stable_marriage})
  muestra que esto no siempre termina.
  Considere las preferencias del cuadro~\ref{tab:sm-divorces-remarriages},
  donde simplemente damos las mujeres en orden de preferencia para cada hombre
  y viceversa.
  \begin{table}[ht]
    \centering
    \subfloat[Hombres]{
      \begin{tabular}{>{\(}c<{\):}*{3}{>{\(}c<{\)}}}
        1 & 2 & 1 & 3 \\
        2 & \multicolumn{3}{c}{arbitrario} \\
        3 & 1 & 2 & 3
      \end{tabular}
      \label{subtab:sm-dr-men}
    }
    \hspace*{3em}
    \subfloat[Mujeres]{
      \begin{tabular}{>{\(}c<{\):}*{3}{>{\(}c<{\)}}}
        1 & 1 & 3 & 2 \\
        2 & 3 & 1 & 2 \\
        3 & \multicolumn{3}{c}{arbitrario}
      \end{tabular}
      \label{subtab:sm-dr-women}
    }
    \caption{Contraejemplo para divorcios y matrimonios}
    \label{tab:sm-divorces-remarriages}
  \end{table}
  Una secuencia de divorcios y matrimonios entra en un ciclo;
  pero hay soluciones estables,
  como \(((x_1, y_2), (x_2, y_3), (x_3, y_1))\)
  o \(((x_1, y_1), (x_2, y_3), (x_3, y_2))\).

  Una solución puede hallarse mediante un algoritmo bastante sencillo,
  y este algoritmo muestra incidentalmente
  que los matrimonios estables siempre existen.
  El algoritmo fue
  discutido formalmente por primera vez por Gale y Shapley~%
    \cite{gale62:_stable_marriage}.
  \begin{theorem}
    \label{theo:stable-marriage-exists}
    Siempre hay un conjunto de matrimonios estable.
  \end{theorem}
  \begin{proof}
    Consideremos el modelo tradicional,
    en el cual los hombres proponen matrimonio a las mujeres,
    y estas aceptan o no.
    Efectuamos varias rondas,
    en cada ronda los hombres sin pareja proponen matrimonio
    a la mujer más alta en su preferencia que no lo ha rechazado aún,
    cada mujer elige entre las propuestas que recibe
    al hombre más alto en sus preferencias
    y se compromete provisoriamente.
    Si una mujer provisoriamente comprometida recibe una mejor oferta,
    disuelve el compromiso
    y se compromete con el nuevo pretendiente.

    Note que una vez que una mujer recibe una propuesta,
    nunca más queda libre
    (puede cambiar de novio,
     claro).
    En cada ronda un hombre propone a mujeres
    hasta hallar una que lo acepte,
    el número de mujeres no comprometidas disminuye.
    Los hombres pueden proponer siempre en orden de preferencia decreciente
    (las mujeres que ya lo rechazaron solo pueden mejorar su pretendiente,
     no lo aceptarán después).
    El número total de propuestas es a lo más \(n^2 - 2 n + 2\) propuestas,
    si \(\lvert \mathscr{X} \rvert = \lvert \mathscr{Y} \rvert = n\).
    Una vez que todas las mujeres han recibido propuestas
    el noviazgo se declara terminado
    y los compromisos se formalizan.

    El resultado es estable,
    cosa que demostramos por contradicción.
    Supongamos que \(x\) tiene a \(y\) de pareja,
    pero prefiere a \(y'\),
    quien a su vez prefiere a \(x\) sobre su marido \(x'\).
    Entonces \(x\) propuso matrimonio a \(y'\) antes que a \(y\),
    y fue rechazado a cambio de alguien a quien \(y'\) prefiere a \(x\).
    Si \(y'\) cambió su compromiso en el intertanto,
    fue por alguien a quien prefiere aún más que a \(x\).
    O sea,
    \(y'\) prefiere a \(x'\),
    no hay inestabilidad.
  \end{proof}
  Podemos extender trivialmente al caso
  \(\lvert \mathscr{X} \rvert \ne \lvert \mathscr{Y} \rvert\),
  simplemente sobrarán hombres o mujeres que no encuentran pareja,
  y por el mismo razonamiento los matrimonios acordados son estables.

  Una pregunta obvia es si la solución es única,
  y la relación entre esta
  y la que da el algoritmo simétrico en que proponen las mujeres.
  Resulta que pueden haber muchas soluciones,
  como demuestra Knuth~%
    \cite{knuth96:_stable_marriage}
  con el ejemplo del cuadro~\ref{tab:sm-example-many}.
  Se muestran solo las dos primeras preferencias de los hombres
  y las últimas de las mujeres
  (las demás no importan,
   no entran en soluciones estables).
  \begin{table}[ht]
    \centering
    \subfloat[Hombres]{
      \begin{tabular}{>{\(}r<{\)}@{:\;}*{2}{>{\(}c<{\)}}@{\;\({} \cdots {}\)}}
        1     & 1     & 2    \\
        2     & 2     & 1    \\
        \multicolumn{1}{r}{\(\vdots\)\phantom{:}} &
          \multicolumn{1}{c}{\(\vdots\)} &
          \multicolumn{1}{c}{\(\vdots\)} \\
        n - 1 & n - 1 & n     \\
        n     & n     & n - 1
      \end{tabular}
      \label{subtab:sm-ex-men}
    }
    \hspace*{5em}
    \subfloat[Mujeres]{
      \begin{tabular}{>{\(}r<{\)}@{:\;\({} \cdots {}\)\;}>{\(}c<{\)}}
        1     & 1    \\
        2     & 2    \\
        \multicolumn{1}{r}{\(\vdots\)\phantom{:\;\({} \cdots {}\)}} &
          \multicolumn{1}{c}{\(\vdots\)} \\
        n - 1 & n - 1 \\
        n     & n
      \end{tabular}
      \label{subtab:sm-ex-women}
    }
    \caption{Preferencias para muchas soluciones,
             \(n\) par}
    \label{tab:sm-example-many}
  \end{table}
  Vemos que el hombre \num{1} puede formar parejas estables
  con las mujeres \num{1} o \num{2},
  mientras el hombre \num{2} queda con \num{2} o \num{1};
  el hombre \num{3} con \num{3} o \num{4},
  dejando la otra para \num{4};
  y así sucesivamente.
  Ambas posibilidades son estables,
  si dos hombres se conforman con sus segundas opciones,
  son la última preferencia para su primera opción
  y ella nunca lo preferirá.
  Esto da \(2^{n/2}\) soluciones posibles.

  Incidentalmente,
  si se permiten preferencias incompletas
  (\textquote{prefiero muerto que casado con\ldots}),
  puede no haber solución estable.
  Nuevamente Knuth~%
    \cite{knuth96:_stable_marriage}
  plantea un ejemplo.
  \begin{table}[ht]
    \centering
    \subfloat[Hombres]{
      \begin{tabular}{>{\(}c<{\)}@{:\;}*{3}{>{\(}c<{\)}}}
        1 & 1 &	&     \\
        2 & 3 & 2 & 1 \\
        3 & 3 & 1 &
      \end{tabular}
      \label{subtab:sm-inc-men}
    }
    \hspace*{3em}
    \subfloat[Mujeres]{
      \begin{tabular}{>{\(}c<{\)}@{:\;}*{3}{>{\(}c<{\)}}}
        1 & 3 & 1 & 2 \\
        2 & 2 & 1 & 3 \\
        3 & 1 & 2 & 3
      \end{tabular}
      \label{subtab:sm-inc-women}
    }
    \caption{Preferencias incompletas}
    \label{tab:sm-incomplete}
  \end{table}
  En el cuadro~\ref{tab:sm-incomplete} la única posibilidad es
  \(((x_1, y_1), (x_2, y_2), (x_3, y_3))\),
  pero esta es inestable por \(x_2\) e \(y_3\).

  En realidad,
  se da:
  \begin{theorem}
    \label{theo:sm-optimal}
    La solución dada por el algoritmo de Gale-Shapley
    es la mejor posible para los hombres.
  \end{theorem}
  \begin{proof}
    Llame a una mujer \emph{posible} para \(x\)
    si hay una solución estable que la da como pareja para \(x\).
    Demostraremos que cada hombre se casa
    con su mujer posible más alta en sus preferencias
    por inducción.
    Supongamos que en un cierto punto del algoritmo
    ningún hombre ha sido rechazado por una mujer posible.
    Al inicio esto se cumple vacuamente;
    si siempre se cumple durante la ejecución del algoritmo
    se cumple al final,
    y cada hombre se casa con su pareja posible preferida.
    Suponga que en este punto \(y\),
    habiendo recibido una propuesta mejor,
    rechaza a \(x\).
    Debemos demostrar que \(y\) es imposible para \(x\).
    Si \(y\) elige a \(x'\),
    es porque prefiere a \(x'\) sobre \(x\);
    y \(x'\) se propuso a \(y\) porque todas sus mejores opciones
    lo rechazaron.
    Por inducción,
    \(x'\) es imposible para esas otras opciones.
    Si se casara \(x\) con \(y\),
    \(x'\) deberá conformarse con \(y'\),
    que considera menos deseable.
    Pero esta configuración es inestable,
    \(x'\) e \(y\) estarían dispuestos a intercambiar parejas.
    En resumen,
    \(x\) es imposible para \(y\).
  \end{proof}
  Solo si la solución es única
  los resultados de propuestas de hombres y propuestas de mujeres coinciden.
  Gusfield~%
    \cite{gusfield87:_three_fast_algor_four_probl_stabl_marriag}
  da algoritmos eficientes para hallar todas las combinaciones estables
  (¡pueden ser muchas!)
  y otros problemas afines.

\section{Postulación a carreras}
\label{sec:postulacion-carreras}

  Una extensión es considerar estudiantes \(\mathscr{A}\)
  que postulan a universidades \(\mathscr{U}\),
  donde la universidad \(u \in \mathscr{U}\) ofrece \(q_u\) vacantes.
  Nuevamente,
  cada estudiante tiene una lista de prioridades de las universidades
  y cada universidad una lista de preferencias de postulantes.
  Todos postulan a su universidad preferida
  entre las que aún no lo han rechazado,
  y la universidad con \(q\) cupos elige los \(q\) mejores
  entre los que tiene en la lista actualmente y los nuevos llegados
  (posiblemente rechazando a quienes no cumplen requisitos mínimos,
   con lo que la lista podría tener menos de \(q\) postulantes).
  El proceso termina cuando todos los estudiantes
  o están en una lista de espera
  o han sido rechazados por todas las universidades
  a las que pueden postular.
  En forma similar al teorema~\ref{theo:stable-marriage-exists}
  se demuestra que el resultado es estable
  (ningún estudiante se cambiaría de universidad con otro),
  como en el teorema~\ref{theo:sm-optimal}
  los postulantes obtienen sus mejores cupos posibles.

\section*{Ejercicios}
\label{sec:ejercicios-07-previa}

  \begin{enumerate}
  \item
    Encuentre las soluciones que entrega el algoritmo
    orientado a hombres y a mujeres
    para las preferencias del cuadro~\ref{tab:sm-exercise}.
    \begin{table}[ht]
      \centering
      \subfloat[Hombres]{
        \begin{tabular}{>{\(}c<{\)}@{:}*{8}{@{\;\;}>{\(}c<{\)}}}
          1 & 5 & 7 & 1 & 2 & 6 & 8 & 4 & 3 \\
          2 & 2 & 3 & 7 & 5 & 4 & 1 & 8 & 6 \\
          3 & 8 & 5 & 1 & 4 & 6 & 2 & 3 & 7 \\
          4 & 3 & 2 & 7 & 5 & 1 & 6 & 8 & 4 \\
          5 & 7 & 2 & 5 & 1 & 3 & 6 & 8 & 4 \\
          6 & 1 & 6 & 7 & 5 & 8 & 4 & 2 & 3 \\
          7 & 2 & 5 & 7 & 6 & 3 & 4 & 8 & 1 \\
          8 & 3 & 8 & 4 & 5 & 7 & 2 & 6 & 1
        \end{tabular}
        \label{subtab:sm-exer-men}
      }
      \hspace*{2em}
      \subfloat[Mujeres]{
        \begin{tabular}{>{\(}c<{\)}@{:}*{8}{@{\;\;}>{\(}c<{\)}}}
          1 & 5 & 3 & 7 & 6 & 1 & 2 & 8 & 4 \\
          2 & 8 & 6 & 3 & 5 & 7 & 2 & 1 & 4 \\
          3 & 1 & 5 & 6 & 2 & 4 & 8 & 7 & 3 \\
          4 & 8 & 7 & 3 & 2 & 4 & 1 & 5 & 6 \\
          5 & 6 & 4 & 7 & 3 & 8 & 1 & 2 & 5 \\
          6 & 2 & 8 & 5 & 3 & 4 & 6 & 7 & 1 \\
          7 & 7 & 5 & 2 & 1 & 8 & 6 & 4 & 3 \\
          8 & 7 & 4 & 1 & 5 & 2 & 3 & 6 & 8
        \end{tabular}
        \label{subtab:sm-exer-women}
      }
      \caption{Preferencias para ejercicio}
      \label{tab:sm-exercise}
    \end{table}
  \item
    Acote el número de propuestas de matrimonio aceptadas,
    como menciona la demostración
    del teorema~\ref{theo:stable-marriage-exists}.
  \item
    Demuestre que a lo más un hombre
    recibe su última elección con el algoritmo dado.
    En consecuencia,
    si hay una asignación estable en la cual varios hombres
    se deben conformar con sus últimas preferencias,
    hay varias soluciones.
  \item
    El algoritmo esbozado en el teorema~\ref{theo:stable-marriage-exists}
    no especifica el orden en que los hombres se proponen.
    Demuestre que cualquiera sea este orden,
    la asignación resultante es la misma.
  \item
    Demuestre que el algoritmo
    como esbozado en teorema~\ref{theo:stable-marriage-exists}
    a cada mujer le asigna la peor de sus preferencias
    entre todas las soluciones estables.
  \item
    Demuestre que el algoritmo de llenado de cupos en universidades
    cumple lo enunciado.
  \end{enumerate}

% To do:
% - Related problems: Roommate
% - Other algorithms cited by Knuth
% - Complete bibliography

\bibliography{../referencias}

%%% Local Variables:
%%% mode: latex
%%% TeX-master: ../INF-221_clases
%%% ispell-local-dictionary: "spanish"
%%% End:

% LocalWords:  english stable marriage problem deseabilidad
% LocalWords:  intertanto
