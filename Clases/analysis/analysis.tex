%\bibliographystyle{babplain-fl}

\chapter{Algunos resultados del análisis}
\label{apx:analysis}

  En derivaciones en análisis numérico
  usamos varios resultados conocidos
  (y no tan conocidos)
  del análisis.
  Resumimos los más importantes,
  con sus demostraciones.

  Primeramente,
  algunas definiciones previas.
  \begin{definition}
    El conjunto de funciones continuas sobre el intervalo cerrado \([a, b]\)
    se anota \(C[a, b]\).
    Usamos \(C^n[a, b]\) para denotar el conjunto de funciones reales
    con \(n\)\nobreakdash-ésima derivada continua
    sobre el intervalo \([a, b]\).
    Notaciones correspondientes se usan para intervalos abiertos
    en alguno de sus extremos,
    como \(C(a, b)\) para funciones continuas
    en el intervalo abierto \((a. b)\).
  \end{definition}

  Un \textquote{resultado obvio}
  (cuya demostración en realidad es bastante compleja,
   y omitiremos)
  es el siguiente:
  \begin{theorem}[Valores extremos]
    \label{theo:extreme value}
    Sea \(f\) una función continua sobre el intervalo \([a, b]\).
    Sean el mínimo y el máximo de \(f\) sobre el intervalo,
    \(m = \min_{x \in [a, b]} \{ f(x) \}\)
    y \(M = \max_{x \in [a, b]} \{ f(x) \}\).
    Entonces para todo \(v \in [m, M]\)
    hay \(\zeta \in [a, b]\) tal que \(f(\zeta) = v\).
  \end{theorem}
  O sea,
  \(f\) toma todos los valores entre \(m\) y \(M\).

  \begin{theorem}[Rolle]
    \label{theo:Rolle}
    Sea \(f \colon [a, b] \to \mathbb{R}\) continua,
    diferenciable en \((a, b)\),
    tal que \(f(a) = f(b)\).
    Entonces hay \(\zeta \in (a, b)\) tal que \(f'(\zeta) = 0\).
  \end{theorem}
  En particular,
  considerando el caso \(f(a) = f(b) = 0\)
  dice que entre cada par de ceros de \(f\) hay al menos un cero de \(f'\).
  \begin{proof}
    Como \(f\) es continua en \([a, b]\),
    alcanza sus valores máximo y mínimo en el intervalo.
    Si máximo y mínimo coinciden en ambos extremos,
    la función es constante,
    su derivada se anula en \((a, b)\)
    y podemos elegir \(\zeta \in (a, b)\) arbitrariamente.

    Supongamos entonces que el máximo no ocurre en los extremos
    (la situación con el mínimo es similar),
    llamemos \(\zeta\) al punto máximo.
    Para \(\zeta + h \in (a, b)\) consideremos la expresión:
    \begin{equation*}
      \frac{f(\zeta + h) - f(\zeta)}{h}
    \end{equation*}
    Para \(h\) positivo esta expresión es negativa
    (numerador negativo,
     denominador positivo)
    para \(h\) negativo es positiva
    (numerador negativo,
     denominador negativo).
    Concluimos que:
    \begin{align*}
      \lim_{h \to 0} \frac{f(\zeta + h) - f(\zeta)}{h}
        &= f'(\zeta) \\
        &= 0
    \end{align*}
    El límite existe ya que supusimos que \(f\) es diferenciable.
    Esto es lo que queríamos demostrar.
  \end{proof}
  Un resultado que usamos es el siguiente:
  \begin{theorem}[Rolle extendido]
    \label{theo:Rolle-extended}
    Sea \(f \colon [a, b] \to \mathbb{R}\) continua,
    diferenciable \(n\) veces en \((a, b)\),
    tal que \(f(a) = f(b) = 0\)
    y con \(n + 1\)~ceros distintos en \([a, b]\).
    Entonces hay \(\zeta \in (a, b)\) tal que \(f^{(n)}(\zeta) = 0\).
  \end{theorem}
  \begin{proof}
    La demostración es por inducción.
    \begin{description}
    \item[Base:]
      Para \(n = 0\) tenemos el teorema de Rolle,
      teorema~\ref{theo:Rolle}.
    \item[Inducción:]
      Supongamos que vale para \(n\),
      y consideremos la función \(f'(x)\).
      Por hipótesis \(f'\) es continua en \((a, b)\).
      Por el teorema de Rolle aplicado
      a los intervalos
        \([x_0, x_1], [x_1, x_2], \dotsc, [x_n, x_{n + 1}]\),
      cada uno de estos contiene un cero de \(f'(x)\),
      llamémosles \(\zeta_0, \zeta_1, \dotsc, \zeta_n\).
      Por inducción,
      el intervalo \([\zeta_0, \zeta_n] \subset [a, b]\)
      contiene un cero \(\zeta\)
      de la \(n\)\nobreakdash-ésima derivada de \(f'(x)\),
      que es \(f^{(n + 1)}(x)\).
    \end{description}
    Por inducción vale para todo \(n \in \mathbb{N}_0\).
  \end{proof}
  \begin{theorem}[Valor medio]
    \label{theo:mean-value}
    Sea \(f \colon [a, b] \to \mathbb{R}\) continua en \([a, b]\)
    y derivable en \((a, b)\).
    Entonces hay \(\zeta \in (a, b)\) tal que:
    \begin{equation*}
      f'(\zeta)
        = \frac{f(b) - f(a)}{b - a}
    \end{equation*}
  \end{theorem}
  \begin{proof}
    Consideremos la función:
    \begin{equation*}
      g(x)
        = f(x) - \frac{f(b) - f(a)}{b - a} x
    \end{equation*}
    Entonces \(g\) cumple las hipótesis del teorema de Rolle,
    \ref{theo:Rolle}.
    Vale decir,
    hay un punto \(\zeta \in (a, b)\) tal que:
    \begin{align*}
      g'(\zeta)
        &= 0 \\
        &= f'(\zeta) - \frac{f(b) - f(a)}{b - a}
    \end{align*}
    El resultado sigue.
  \end{proof}
  El siguiente se conoce como teorema del valor medio extendido,
  o teorema de valor medio de Cauchy.
  \begin{theorem}[Valor medio extendido]
    \label{theo:mean-value-extended}
    Sean \(f, g\) funciones continuas en el intervalo \([a, b]\),
    derivables en \((a, b)\).
    Entonces hay \(\zeta \in (a, b)\) tal que:
    \begin{equation*}
      (f(b) - f(a)) g'(\zeta)
        = (g(b) - g(a)) f'(\zeta)
    \end{equation*}
  \end{theorem}
  Obviamente,
  si \(g(a) \ne g(b)\) y \(g'(\zeta) \ne 0\) esto equivale a:
  \begin{equation*}
    \frac{f'(\zeta)}{g'(\zeta)}
      = \frac{f(b) - f(a)}{g(b) - g(a)}
  \end{equation*}
  \begin{proof}
    Si \(g(a) = g(b)\),
    por el teorema de Rolle
    (teorema~\ref{theo:Rolle})
    hay \(\zeta \in (a, b)\) tal que \(g'(\zeta) = 0\),
    para este el teorema se cumple trivialmente.

    Suponga ahora \(g(a) \ne g(b)\).
    Defina:
    \begin{equation*}
      h(x)
        = f(x) - \frac{f(b) - f(a)}{g(b) - g(a)} g(x)
    \end{equation*}
    Como \(f, g\) son continuas en \([a, b]\) y derivables en \((a, b)\),
    esto se cumple para \(h\),
    al satisfacer las hipótesis del teorema de Rolle
    hay \(\zeta \in (a, b)\) tal que:
    \begin{align*}
      h'(\zeta)
        &= 0 \\
        &= f'(\zeta) - \frac{f(b) - f(a)}{g(b) - g(a)} g'(\zeta)
    \end{align*}
    El resultado es inmediato.
  \end{proof}
  El teorema del valor medio,
  teorema~\ref{theo:mean-value},
  es el caso particular \(g(x) = x\).

  Finalmente:
  \begin{theorem}
    \label{theo:mean-value-integral}
    Sea \(f \colon [a, b] \to \mathrm{R}\) continua,
    y sea \(g \colon [a, b] \to \mathbb{R}\)
    integrable que no cambia de signo en \([a, b]\).
    Entonces hay \(\zeta \in [a, b]\) tal que:
    \begin{equation*}
      \int_a^b f(x) g(x) \mathrm{d} x
        = f(\zeta) \int_a^b g(x) \mathrm{d} x
    \end{equation*}
  \end{theorem}
  El caso especial \(g(x) = 1\) da el resultado conocido:
  \begin{equation*}
    \int_a^b f(x) \mathrm{d} x
      = f(\zeta) (b - a)
  \end{equation*}
  \begin{proof}
    Supongamos \(g\) es no negativa,
    el caso de \(g\) no positiva es similar.
    Sabemos que hay valores mínimo y máximo de \(f\) sobre \([a, b]\),
    llamémosles \(m, M\),
    tales que para \(x \in [a, b]\):
    \begin{equation*}
      m \le f(x) \le M
    \end{equation*}
    Como \(g\) no es negativa:
    \begin{equation*}
      m \int_a^b g(x) \mathrm{d} x
        \le \int_a^b f(x) g(x) \mathrm{d} x
        \le M \int_a^b g(x) \mathrm{d} x
    \end{equation*}
    Si ahora:
    \begin{equation*}
      \int_a^b g(x) \mathrm{d} x
        = 0
    \end{equation*}
    podemos elegir \(\zeta \in [a, b]\) arbitrario.
    Si la última integral no se anula,
    por el teorema de los valores extremos,
    teorema~\ref{theo:extreme value},
    hay \(\zeta \in [a, b]\) con el que se cumple lo prometido.
  \end{proof}
  Un resultado importantísimo es:
  \begin{theorem}[Taylor]
    \label{theo:taylor}
    Sea \(f \in C^{n + 1}(a, x)\) con \(f^{(n + 1)} \in C[a, x]\).
    Entonces hay \(\xi \in [a, x]\) tal que:
    \begin{equation*}
      f(x)
        = \sum_{0 \le k \le n} \frac{f^{(k)}(a)}{k!} (x - a)^k
            + \frac{f^{(n + 1)}(\xi)}{(n + 1)!} (x - a)^{n + 1}
    \end{equation*}
  \end{theorem}
  A esta se le conoce como \emph{la forma de Lagrange del residuo},
  es la más útil para nuestros fines.
  \begin{proof}
    Sean:
    \begin{align*}
      F(t)
        &= \sum_{0 \le k \le n} \frac{f^{(k)}(t)}{k!} (x - t)^k \\
      G(t)
        &= (t - x)^{n + 1}
    \end{align*}
    Por el teorema~\ref{theo:mean-value-extended},
    tenemos que hay \(\xi \in (a, x)\) tal que:
    \begin{equation*}
      \frac{F'(\xi)}{G'(\xi)}
        = \frac{F(x) - F(a)}{G(x) - G(a)}
    \end{equation*}
    Vemos que:
    \begin{align*}
      F'(t)
        &= \frac{f^{(n + 1)}(t)}{n!} (x - t)^n \\
      G'(t)
        &= (n + 1) (t - x)^n
    \end{align*}
    Substituyendo y simplificando queda:
    \begin{equation*}
      f(x)
        = \sum_{0 \le k \le n} \frac{f^{(k)}(a)}{k!} (x - a)^k
            + \frac{f^{(n + 1)}(\xi)}{(n + 1)!} (x - a)^{k + 1}
    \end{equation*}
    Esto es lo que había que demostrar.
  \end{proof}

%\bibliography{../referencias}

%%% Local Variables:
%%% mode: latex
%%% TeX-master: "../INF-221_notas"
%%% ispell-local-dictionary: "spanish"
%%% End:

% LocalWords:  ésima diferenciable
