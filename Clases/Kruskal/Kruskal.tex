\bibliographystyle{babplain-fl}

\chapter{Algoritmo de Kruskal, Union-Find}
\label{cha:algoritmo-de-kruskal}

  Un problema recurrente es hallar el árbol recubridor mínimo
  (\emph{\foreignlanguage{english}{Minimal Spanning Tree}} en inglés,
    abreviado MST)
  de un grafo rotulado conexo.
  En detalle,
  dado un grafo \(G = (V, E)\) conexo,
  con arcos rotulados por \(w \colon E \to \mathbb{R}^{+}\)
  (el rótulo representa costo del arco),
  hallar un árbol recubridor de costo total
  (suma de los costos de los arcos)
  mínimo.
  Una solución a este problema da el algoritmo de Kruskal~%
    \cite{kruskal56:_MST},
  (algoritmo~\ref{alg:Kruskal},
   que ya discutimos en el capítulo~\ref{cha:greedy-algorithm}).
  La idea es ir construyendo un bosque
  (conjunto de árboles),
  uniendo sucesivamente árboles
  mediante arcos de costo mínimo.
  Inicialmente el bosque es simplemente cada vértice por separado,
  al final es un árbol recubridor mínimo.
  \begin{algorithm}[ht]
    \DontPrintSemicolon\Indp

    Sort \(E\) by increeaing cost \;
    \(S \gets \varnothing\) \;
    \For{\(v \in V\)}{
      Add \(\{v\}\) to \(S\) \;
    }
    \(T \gets \varnothing\) \;
    \For{\(u v \in E\)}{
      \If{\(u\) and \(v\) don't belong to the same set of \(S\)}{
        Add \(u v\) to \(T\) \;
        Join the sets to which \(u\) and \(v\) belong in \(S\) \;
      }
    }
    \Return (V, T) \;

    \caption{Algoritmo de Kruskal}
    \label{alg:Kruskal}
  \end{algorithm}
  Este es un ejemplo clásico de algoritmo voraz.
  Nos interesa derivar un programa eficiente
  (y deducir su complejidad).
  Es claro que la manipulación del conjunto \(S\)
  es crucial.
  Parte de la discusión que sigue viene de Erickson~%
    \cite[clase~17]{erickson19:_algorithms}.

\section{Una estructura de datos
       para \emph{\foreignlanguage{english}{union-find}}}
\label{sec:union-find-estructura}

  Abstrayendo las operaciones empleadas,
  vemos que requerimos una estructura de datos
  que representa una partición de un conjunto universo \(V\),
  con operaciones de inicializar con elementos solitarios,
  hallar
  (\emph{\foreignlanguage{english}{find}})
  la partición a la que pertenece un elemento,
  y unir particiones disjuntas
  (\emph{\foreignlanguage{english}{union}}).
  Por ellas se le llama
  problema de \emph{\foreignlanguage{english}{union-find}}.
  \begin{figure}[ht]
    \centering
    \begin{tikzpicture}[edge from parent/.style = {draw, latex'-}]
      \node (a) at (0, 0) {\num{3}}
        child {node {\num{10}}
          child {node{\num{15}}}
          child {node{\num{17}}}};
      \path[-latex'] (a) edge [loop above] (a);

      \node (b) at (3, 0) {\num{11}}
        child {node {\num{4}}};
      \path[-latex'] (b) edge [loop above] (b);

      \node (c) at (6, 0) {\num{13}}
        child {node {\num{5}}}
        child {node {\num{6}}}
        child {node {\num{14}}
          child {node {\num{9}}
            child {node {\num{8}}}
            child {node {\num{16}}}
            child {node {\num{20}}
              child {node {\num{18}}}
              child {node {\num{19}}}}}};
      \path[-latex'] (c) edge [loop above] (c);
    \end{tikzpicture}
    \caption{Esquema de la estructura para
             \emph{\foreignlanguage{english}{union-find}}.}
    \label{fig:union-find-estructura}
  \end{figure}
  La manera de identificar a un subconjunto de \(V\) es irrelevante,
  podemos elegir un elemento cualquiera como representante.
  Para hallar el representante del conjunto al que pertenece \(v\)
  podemos hacer que cada elemento apunte a un elemento padre,
  siguiendo la lista hasta el final hallamos el representante.
  Unir dos conjuntos es hacer que el representante de uno
  quede como padre del representante del otro,
  ver la figura~\ref{fig:union-find-estructura}.
  Para simplificar algunos de los algoritmos,
  hacemos que las raíces de los árboles apunten a ellas mismas.
  En diagramas sucesivos omitiremos las flechas
  y los bucles en las raíces.

  La operación \(\mathrm{find}\) depende de la altura
  de los árboles que se construyan,
  interesa construir árboles bajos.
  Nos conviene hacer que el representante con el árbol menor
  dependa del representante con el árbol más alto,
  ya que esto no aumenta la altura.
  Si mantenemos un arreglo \(\mathrm{rank}\) con la altura del árbol
  de cada representante,
  basta poner de hijo al representante de altura menor.
  Solo en caso de empate la altura aumenta,
  elegimos uno de los dos como nuevo representante
  con \(\mathrm{rank}\) uno mayor.
  Inicialmente \(\mathrm{rank}\) es cero para todos los vértices.
  Note que solo cuando se unen dos árboles de la misma altura
  se ajusta \(\mathrm{rank}\) del nuevo representante.
  Es un simple ejercicio de inducción demostrar que de esta manera
  si \(v\) es representante de una clase,
  esta contiene al menos \(2^{\mathrm{rank}[v]}\) elementos.
  En consecuencia,
  el máximo camino posible de un vértice a su representante en la clase \(C\)
  es de largo \(\log_2 \lvert C \rvert\).
  El costo para cada operación es \(O(\log n)\).
  Esta estructura y su manipulación fueron propuestas por
  Galler y Fisher~%
    \cite{galler64:_improved_equiv_algorithm},
  aunque su análisis tomó años.
  La idea se basa en arreglos globales \(\mathrm{rank}\)
  (la altura del árbol con raíz en el vértice)
  y \(\mathrm{parent}\)
  (el padre del vértice,
   para la raíz es el mismo para simplificar el código).
  Vea los algoritmos
  para \(\mathrm{MakeSets}\)
  (crea la estructura inicial),
  \(\mathrm{union}\)
  (une las clases de \(u\) y \(v\))
  y \(\mathrm{find}\)
  (halla el representante para la clase de \(v\)),
  respectivamente~\ref{alg:union-find-MakeSets},
  \ref{alg:union-find-union},
  y~\ref{alg:union-find-find}.
  \begin{algorithm}[ht]
    \DontPrintSemicolon\Indp

    \Procedure{\(\operatorname{MakeSets}(V)\)}{
      \For{\(v \in V\)}{
        \(\mathrm{rank}[v] \gets 0\) \;
        \(\mathrm{parent}[v] \gets v\) \;
      }
    }
    \caption{Algoritmo para crear conjuntos}
    \label{alg:union-find-MakeSets}
  \end{algorithm}
  \begin{algorithm}[ht]
    \DontPrintSemicolon\Indp

    \Function{\(\operatorname{find}(v)\)}{
      \While{\(v \ne \mathrm{parent}[v]\)}{
        \(v \gets \mathrm{parent}[v]\) \;
      }
      \Return \(v\) \;
    }
    \caption{Algoritmo para encontrar representante}
    \label{alg:union-find-find}
  \end{algorithm}
  \begin{algorithm}[ht]
    \DontPrintSemicolon\Indp

    \Procedure{\(\operatorname{union}(u, v)\)}{
      \uIf{\(\mathrm{rank}[u] > \mathrm{rank}[v]\)}{
        \(\mathrm{parent}[v] \gets u\) \;
      }
      \Else{
        \(\mathrm{parent}[u] \gets v\) \;
        \If{\(\mathrm{rank}[u] = \mathrm{rank}[v]\)}{
          \(\mathrm{rank}[v] \gets \mathrm{rank}[v] + 1\) \;
        }
      }
    }
    \caption{Algoritmo para unir conjuntos dados representantes}
    \label{alg:union-find-union}
  \end{algorithm}

\bibliography{../referencias}

%%% Local Variables:
%%% mode: latex
%%% TeX-master: "../INF-221_notas"
%%% ispell-local-dictionary: "spanish"
%%% End:

% LocalWords:  Union Find recubridor english Spanning Tree MST Sort
% LocalWords:  by increeaing cost Add to and belong the same set of
% LocalWords:  Join sets which in union find
