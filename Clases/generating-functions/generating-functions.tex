\bibliographystyle{babplain-fl}

\chapter{Píldoras de funciones generatrices}
\label{apx:funciones-generatrices}

  La idea de funciones generatrices es usar una serie
  para representar una secuencia.
  Resulta que en muchos casos manipular la serie es mucho más fácil
  que trabajar con la secuencia,
  y se pueden obtener resultados sorprendentes en forma muy sencilla.
  Véase el apéndice~\ref{apx:symbolic-method-dummies} para ejemplos.
  Mucho más detalle se da en los apuntes de Fundamentos de Informática~%
    \cite{brand17:_fundamentos_informatica},
  en los textos de Wilf~%
    \cite{wilf06:_gfology}
  y de Flajolet y Sedgewick~%
    \cite{flajolet09:_analy_combin},
  y muy particularmente aplicado a nuestro tema en el de
  Sedgewick y Flajolet~%
    \cite{sedgewick13:_introd_anal_algor}.

  Wilf~\cite{wilf06:_gfology} expresa
  que la función generatriz es una línea de ropa
  de la cual se cuelgan los coeficientes para exhibición.
  Entendemos el exponente de \(z\) como un contador,
  índice del coeficiente correspondiente.
  Como veremos,
  operaciones sobre la función generatriz
  corresponden a actuar sobre la secuencia,
  en muchos casos resulta más sencillo manipular la serie
  que trabajar con la secuencia.
  Para nuestros efectos,
  en general basta manipularlos como si fueran \textquote{polinomios infinitos}.
  Si tenemos la suerte que la serie converge para algún rango
  alrededor de \(z = 0\)
  (como en nuestros ejemplos),
  podremos aplicar las herramientas del cálculo.

\section{Ejemplos combinatorios}
\label{sec:gf-ejemplos-combinatorios}

  Acá nos interesa analizar las sumas resultantes
  al lanzar combinaciones de los dados no tradicionales.
  Un dado puede representarse por una tupla
  que indica el número de caras con cada valor,
  agregando el valor cero para completar.
  Los dados tradicionales quedan representados por:
  \begin{align*}
    \mathrm{D}
      \text{\ (caras \(\{1, 2, 3, 4, 5, 6\}\))}
      &\colon \langle 0, 1, 1, 1, 1, 1, 1 \rangle
  \end{align*}
  Los dados no transitivos que consideraremos quedan descritos por:
  \begin{align*}
    \mathrm{A}
      \text{\ (caras \(\{1, 1, 3, 5, 5, 6\}\))}
      &\colon \langle 0, 2, 0, 1, 0, 2, 1 \rangle \\
    \mathrm{B}
      \text{\ (caras \(\{2, 3, 3, 4, 4, 5\}\))}
      &\colon \langle 0, 0, 1, 2, 2, 1, 0 \rangle \\
    \mathrm{C}
      \text{\ (caras \(\{1, 2, 2, 4, 6, 6\}\))}
      &\colon \langle 0, 1, 2, 0, 1, 0, 2 \rangle
  \end{align*}
  Para lanzar
  la suma \num{7} con dos dados
  tenemos las opciones
  desde que el primero aporte \num{1} y el segundo \num{6}
  hasta aportes respectivos de \num{6} y \num{1}.
  Debemos considerar además el número de caras del valor considerado
  en cada dado.
  Para los dados \(\mathrm{A}\) y \(\mathrm{C}\)
  esto se traduce en:
  \begin{equation*}
    2\cdot 2 + 0 \cdot 0 + 1 \cdot 1 + 0 \cdot 0 + 2 \cdot 2 + 1 \cdot 1
      = 10
  \end{equation*}
  posibilidades.
  Esta es exactamente la manera en que se calcula
  el coeficiente de \(z^7\) al multiplicar polinomios
  con el coeficiente de \(z^n\)
  dando el número de caras con valor \(n\) para cada dado:
  \begin{align*}
    A(z)
      &= 2 z		   +   z^3	   + 2 z^5 +   z^6 \\
    B(z)
      &=	 z^2	   + 2 z^3 + 2 z^4 +   z^5 \\
    C(z)
      &=   z + 2 z^2	   + z^4		   + 2 z^6
  \end{align*}
  El coeficiente de \(z^7\)
  en el producto \(A(z) \cdot C(z)\) nos da el valor buscado:
  \begin{equation*}
    A(z) \cdot C(z)
      = 2 z^2 + 4 z^3 +	  z^4 + 4 z^5	 + 2 z^6    + 10 z^7
              + 2 z^8 + 4 z^9 +	  z^{10} + 4 z^{11} + 2 z^{12}
  \end{equation*}
  Una sencilla operación
  considera todas las combinaciones posibles.
  Note que los polinomios del caso nos interesan
  puramente por sus propiedades algebraicas,
  los valores de los polinomios para diversos valores de \(z\)
  no interesan en lo más mínimo.

  Al lanzar dos dados tradicionales
  las sumas \num{2} y \num{12} se pueden obtener de una única manera,
  mientras para \num{4} hay tres (\(1 + 3 = 2 + 2 = 3 + 1\)).
  Representamos un dado mediante el polinomio:
  \begin{equation}
    \label{eq:gf-dado}
    D(z)
      = z + z^2 + z^3 + z^4 + z^5 + z^6
  \end{equation}
  con lo cual para lanzamientos de dos dados:
  \begin{equation}
    \label{eq:dos-dados}
    D^2(z)
      = z^2 + 2 z^3 + 3 z^4 + 4 z^5 + 5 z^6 + 6 z^7
          + 5 z^8 + 4 z^9 + 3 z^{10} + 2 z^{11} + z^{12}
  \end{equation}
  Interesa hallar dados marcados en forma diferente
  que den la misma distribución de las sumas
  (\textquote{dados locos}).%
    \index{dados!locos|see{Sicherman, dados de}}
  Para construirlos
  buscamos polinomios \(D_1(z)\) y \(D_2(z)\)
  que den el producto~\eqref{eq:dos-dados}.
  Queremos además que ambas representen dados,
  o sea tengan \num{6} caras,
  y que cada cara debe estar marcada por al menos un punto.
  El número de caras
  es simplemente el valor de la función en \(z = 1\);
  que cada cara esté marcada con al menos un punto
  se traduce en que la función generatriz sea divisible por \(z\)
  (el número de caras marcadas con cero,
   o sea el coeficiente de \(z^0\),
   debe ser cero),
  O sea:
  \begin{equation}
    \label{eq:dados-locos-caras}
    D_1(1)
      = D_2(1)
      = 6
  \end{equation}
  Factorizamos~\eqref{eq:gf-dado}:
  \begin{equation}
    \label{eq:gf-dado-factorizada}
    D(z)
      = z (z + 1) (z^2 - z + 1) (z^2 + z + 1)
  \end{equation}
  Los factores \(z\) y \(z^2 - z + 1\) tienen valor \num{1} para \(z = 1\),
  \(z + 1\) da \num{2} y \(z^2 + z + 1\) da \num{3}.
  Tanto \(D_1(z)\) como \(D_2(z)\)
  deben tener los factores \(z\),
  \(z + 1\) y \(z^2 + z + 1\);
  solo quedan por redistribuir los dos factores \(z^2 - z + 1\) en \(D^2(z)\):
  \begin{align}
    D_1(z)
      &= z (z + 1) (z^2 + z + 1) \notag \\
      &= z + 2 z^2 + 2 z^3 + z^4 \label{eq:Sicherman-1} \\
    D_2(z)
      &= z (z + 1) (z^2 - z + 1)^2 (z^2 + z + 1) \notag \\
      &= z + z^3 + z^4 + z^5 + z^6 + z^8 \label{eq:Sicherman-2}
  \end{align}
  Los dados marcados con \(\{ 1, 2, 2, 3, 3, 4 \}\)
  y \(\{ 1, 3, 4, 5, 6, 8 \}\)
  se conocen como \emph{dados de Sicherman}~%
    \cite{gardner78_2:_mathem_games}.

\section{Definiciones formales}
\label{sec:gf-definiciones-formales}

  Sea una secuencia
  \(\left\langle a_n \right\rangle_{n \ge 0}
     = \left\langle
         a_0, a_1, a_2, \dotsc, a_n, \dotsc
       \right\rangle\).
  La \emph{función generatriz} (ordinaria) de la secuencia es
  la serie de potencias:%
    \index{generatriz!ordinaria|textbfhy}
  \begin{equation*}
    A(z)=\sum_{0 \le n} a_n z^n
  \end{equation*}
  Anotaremos
  \(A(z)
     \ogf \left\langle a_n\right\rangle_{n \ge 0}\) en este caso
  (\emph{ogf} es por
     \emph{\foreignlanguage{english}
                           {Ordinary Generating Function}}).

  La \emph{función generatriz exponencial}%
    \index{generatriz!exponencial|textbfhy}
  de la secuencia es la serie:
  \begin{equation*}
    \widehat{A}(z)
      = \sum_{0 \le n} a_n \, \frac{z^n}{n!}
  \end{equation*}
  Anotaremos
  \(\widehat{A}(z)
     \egf \left\langle a_n\right\rangle_{n \ge 0}\) en este caso
  (\emph{egf} es por
     \emph{\foreignlanguage{english}
                           {Exponential Generating Function}}).

  Por comodidad,
  a veces escribiremos estas relaciones
  con la función generatriz al lado derecho.

\subsection{Reglas OGF}
\label{sec:reglas-OGF}
\index{generatriz!ordinaria!reglas}

  Las propiedades siguientes de funciones generatrices ordinarias
  son directamente las definiciones del caso
  o son muy simples de demostrar,
  sus justificaciones detalladas quedarán de ejercicios.

  \begin{description}
  \item[Linealidad:]
    Si \(A(z) \ogf \left\langle a_n \right\rangle_{n \ge 0}\)
    y \(B(z) \ogf \left\langle b_n \right\rangle_{n \ge 0}\),
    y \(\alpha\) y \(\beta\) son constantes,
    entonces:
    \begin{equation*}
      \alpha A(z) + \beta B(z)
         \ogf \left\langle
                \alpha a_n + \beta b_n
              \right\rangle_{n \ge 0}
    \end{equation*}
  \item[Secuencia desplazada a la izquierda:]
    Si
    \(A(z) \ogf \left\langle a_n \right\rangle_{n \ge 0}\),
    entonces:
    \begin{equation*}
      \frac{A(z) - a_0 - a_1 z - \dotsb - a_{k - 1} z^{k - 1}}{z^k}
        \ogf \left\langle a_{n + k}\right\rangle_{n \ge 0}
    \end{equation*}
  \item[Multiplicar por \(n\):]
    Consideremos:
    \begin{align*}
      A(z)
        &\ogf \left\langle a_n\right\rangle_{n \ge 0} \\
      z \, \frac{\mathrm{d}}{\mathrm{d} z} A(z)
        &\ogf \left\langle n a_n\right\rangle_{n \ge 0}
    \end{align*}
    Esta operación
    se expresa en términos del operador \(z \mathrm{D}\)
    (acá \(\mathrm{D}\) es por derivada,
     para abreviar).
    Además:
    \begin{equation*}
      (z \mathrm{D})^2 A(z)
        = z D (z D A(z))
        \ogf \left\langle n^2 a_n\right\rangle_{n \ge 0}
    \end{equation*}
    Note que
      \((z \mathrm{D})^2 = z \mathrm{D} + z^2 \mathrm{D}^2\)
    es diferente de \(z^2 \mathrm{D}^2\).
  \item[Multiplicar por un polinomio en \(n\):]
    Si \(p(n)\) es un polinomio,
    usando el operador \(z \mathrm{D}\) varias veces
    para potencias de \(n\)
    y por linealidad:
    \begin{align*}
      p(z \mathrm{D}) A(z)
        &\ogf \left\langle p(n) a_n \right\rangle_{n \ge 0}
    \end{align*}
  \item[Convolución:]
    Si \(A(z) \ogf \left\langle a_n \right\rangle_{n \ge 0}\)
    y \(B(z) \ogf \left\langle b_n \right\rangle_{n \ge 0}\)
    entonces:
    \begin{equation*}
      A(z) \cdot B(z)
        \ogf \left\langle
               \sum_{0 \le k \le n} a_k b_{n - k}
              \right\rangle_{n \ge 0}
    \end{equation*}
  \item
    Sea \(k\) un entero positivo
    y \(A(z) \ogf \left\langle a_n\right\rangle_{n \ge 0}\),
    entonces:
    \begin{equation*}
      (A(z))^k
        \ogf \left\langle \sum_{n_1 + n_2 + \dotsb + n_k = n}
               \left( a_{n_1} \cdot a_{n_2} \dotsm a_{n_k} \right)
             \right\rangle_{n \ge 0}
    \end{equation*}
    Vale la pena tener presente el caso especial:
    \begin{equation*}
      (A(z))^2
        \ogf \left\langle
               \sum_{0 \le i \le n} a_i a_{n - i}
             \right\rangle_{n \ge 0}
    \end{equation*}
  \item[Sumas parciales:]
    Supongamos:
    \begin{equation*}
      A(z) \ogf \left\langle a_n \right\rangle_{n \ge 0}
    \end{equation*}
    Podemos escribir:
    \begin{equation*}
      \sum_{0 \le k \le n} a_k
        = \sum_{0 \le k \le n} 1 \cdot a_k
    \end{equation*}
    Esto no es más que la convolución de las secuencias
    \(\left\langle 1 \right\rangle_{n \ge 0}\)
    y \(\left\langle a_n \right\rangle_{n \ge 0}\),
    y la función generatriz de la primera es nuestra vieja conocida,
    la serie geométrica,
    con lo que:
    \begin{equation}
      \label{eq:sumas-parciales}
      \frac{A(z)}{1 - z}
        \ogf \left\langle
               \sum_{0 \le k \le n} a_k
             \right\rangle_{n \ge 0}
    \end{equation}
  \end{description}

\subsection{Reglas EGF}
\label{sec:reglas-EGF}
\index{generatriz!exponencial!reglas}

  Las siguientes resumen propiedades
  de las funciones generatrices exponenciales.
  Son simples de demostrar,
  y las justificaciones que no se dan acá quedarán de ejercicios.
  \begin{description}
  \item[Linealidad:]
    Si \(\widehat{A}(z)
           \egf \left\langle a_n \right\rangle_{n \ge 0}\)
    y \(\widehat{B}(z)
          \egf \left\langle b_n \right\rangle_{n \ge 0}\),
    y \(\alpha\) y \(\beta\) son constantes,
    entonces:
    \begin{equation*}
      \alpha \widehat{A}(z) + \beta \widehat{B}(z)
        \egf \left\langle
               \alpha a_n + \beta b_n
             \right\rangle_{n \ge 0}
    \end{equation*}
  \item[Secuencia desplazada a la izquierda:]
    Si \(\widehat{A}(z)
           \egf \left\langle a_n \right\rangle_{n \ge 0}\),
    entonces,
    usando nuevamente \(\mathrm{D}\) para el operador derivada:
    \begin{equation*}
      \mathrm{D}^k \widehat{A}(z)
        \egf \left\langle a_{n + k} \right\rangle_{n \ge 0}
    \end{equation*}
  \item[Multiplicación por un polinomio en \(n\):]
    Si es \(\widehat{A}(z)
              \egf \left\langle a_n \right\rangle_{n \ge 0}\),
    y \(p\) es un polinomio,
    entonces:
    \begin{equation*}
      p(z \mathrm{D}) \widehat{A}(z)
        \egf \left\langle p(n) a_n \right\rangle_{n \ge 0}
    \end{equation*}
    Es la misma que en funciones generatrices ordinarias,
    ya que la operación \(z \mathrm{D}\)
    no altera el exponente en \(z^n\).
  \item[Convolución binomial:]
    Si \(\widehat{A}(z)
           \egf \left\langle a_n \right\rangle_{n \ge 0}\) y
    \(\widehat{B}(z)
        \egf \left\langle b_n \right\rangle_{n \ge 0}\)
    entonces:
    \begin{align*}
      \widehat{A}(z) \cdot \widehat{B}(z)
        &= \sum_{n \ge 0}\biggl( \,
                           \sum_{0 \le k \le n}
                           \frac{a_k}{k!} \, \frac{b_{n - k}}
                                                  {(n - k)!}
                         \biggr) z^n \\
        &= \sum_{n \ge 0} \biggl( \,
                            \sum_{0 \le k \le n}
                               \binom{n}{k} \, a_k b_{n - k}
                          \biggr)
               \frac{z^n}{n!}
    \end{align*}
    Vale decir:
    \begin{equation*}
      \widehat{A}(z) \cdot \widehat{B}(z)
        \egf \left\langle
               \sum_{0 \le k \le n} \binom{n}{k} \, a_k b_{n - k}
             \right\rangle_{n \ge 0}
    \end{equation*}
  \item[Términos individuales:]
    Es fácil ver que si
    \(\widehat{A}(z)
        \egf \left\langle a_n \right\rangle_{n \ge 0}\) entonces:
    \begin{equation*}
      a_n = \widehat{A}^{(n)}(0)
    \end{equation*}
    Esto en realidad no es más que el teorema de Maclaurin.%
      \index{Maclaurin, teorema de}
  \end{description}

\section{\protect\boldmath
           \texorpdfstring{El truco \(z \mathrm{D}\log\)}
                          {Derivada logarítmica}%
       \protect\unboldmath}
\index{derivada logaritmica@derivada logarítmica|textbfhy}

  Los logaritmos ayudan a simplificar expresiones con exponenciales
  y potencias.
  Pero terminamos con el logaritmo de una suma
  si el argumento es una serie,
  que es algo bastante feo de contemplar.
  Eliminar el logaritmo se logra derivando:
  \begin{equation*}
    \frac{\mathrm{d} \ln(A)}{\mathrm{d} z} = \frac{A'}{A}
  \end{equation*}
  Esto es mucho más decente.
  Multiplicamos por \(z\)
  para reponer la potencia \textquote{perdida} al derivar.

\section{Algunas series útiles}
\label{ref:series-utiles}
\index{serie de potencias!series utiles@series útiles}

  Las expansiones en serie que más aparecen son las siguientes.

\subsection{Serie geométrica}
\label{sec:serie-geometrica}
\index{serie de potencias!geometrica@geométrica}

  Es la serie más común en aplicaciones.
  Si \(\lvert z \rvert < 1\),
  se cumple:
  \begin{equation}
    \label{eq:serie-geometrica-b}
    \sum_{n \ge 0} z^n
      = \frac{1}{1 - z}
  \end{equation}
  Una variante importante es la siguiente,
  expansión válida para \(\lvert a z \rvert < 1\)
  (con la convención \(0^0 = 1\)):
  \begin{equation}
    \label{eq:serie-geometrica-c}
    \sum_{n \ge 0} a^n z^n
      = \frac{1}{1 - a z}
  \end{equation}

\subsection{Teorema del binomio}
\label{sec:teorema-binomio}
\index{serie de potencias!binomio}

  Una de las series más importantes
  es la expansión de la potencia de un binomio:
  \begin{equation}
    \label{eq:serie-binomio}
    \sum_{n \ge 0} \binom{\alpha}{n} \, z^n
       = (1 + z)^\alpha
  \end{equation}
  Siempre que \(\lvert z \rvert < 1\)
  esto vale incluso para \(\alpha\) complejos,
  si definimos:
  \begin{equation}
    \label{eq:coeficiente-binomial}
    \binom{\alpha}{k}
       = \frac{\alpha}{1} \cdot \frac{\alpha - 1}{2}
            \cdot \frac{\alpha - 2}{3}
            \cdot \dots
            \cdot \frac{\alpha - k + 1}{k}
       = \frac{\alpha^{\underline{k}}}{k!}
  \end{equation}
  y (consistente con la convención que productos vacíos son \num{1})
  siempre es:
  \begin{equation}
    \label{eq:binomial(alpha,0)}
    \binom{\alpha}{0}
      = 1
  \end{equation}
  A los coeficientes~\eqref{eq:coeficiente-binomial}
  se les llama \emph{coeficientes binomiales}
  por su conexión con la potencia de un binomio.
  La expansión~\eqref{eq:serie-binomio}
  (también conocida como \emph{fórmula de Newton}
   si \(\alpha\) no es un natural)
  es fácil de demostrar por el teorema de Maclaurin.
  Resulta que~\eqref{eq:serie-geometrica-b} es
  un caso particular de~\eqref{eq:serie-binomio}.

  Si \(\alpha\) es un entero positivo,
  la serie~\eqref{eq:serie-binomio} se reduce a un polinomio
  (el factor \(\alpha^{\underline{k}}\) se anula si \(k > \alpha\))
  y la relación es válida para todo \(z\).
  Además,
  en caso que \(n\) sea natural podemos escribir:
  \begin{equation}
    \label{eq:coeficiente-binomial-factorial}
    \binom{n}{k}
       = \frac{n!}{k! (n - k)!}
  \end{equation}
  Es claro que:
  \begin{equation}
    \label{eq:coeficiente-binomial-contorno}
    \binom{n}{k}
      = 0 \text{\ si \(k < 0\) o \(k > n\)}
  \end{equation}
  Esto con~\eqref{eq:coeficiente-binomial-factorial}
  sugiere la convención:
  \begin{equation}
    \label{eq:1/k!-convention}
    \frac{1}{k!}
      = 0 \quad \text{si \(k < 0\)}
  \end{equation}
  Note la simetría:
  \begin{equation}
    \label{eq:coeficiente-binomial-simetria}
    \binom{n}{k}
      = \binom{n}{n - k}
  \end{equation}

  Casos especiales notables de coeficientes binomiales
  para \(\alpha \notin \mathbb{N}\) son los siguientes:
  \begin{description}
  \item[\boldmath Caso \(\alpha = 1 / 2\):\unboldmath]
    Tenemos,
    como siempre:
    \begin{equation}
      \label{eq:binomial(1/2,0)}
      \binom{1/2}{0}
        = 1
    \end{equation}
    Cuando \(k \ge 1\):
    \begin{align}
      \binom{1/2}{k}
         &= \frac{\frac{1}{2} \cdot (\frac{1}{2}-1)
               \dotsm (\frac{1}{2} - k + 1)}{k!} \notag \\
         &= \frac{1}{2^k}
               \cdot \frac{1 \cdot (1 - 2) \cdot (1 - 4)
                             \dotsm (1 - 2 k + 2)}{k!} \notag \\
         &= \frac{(-1)^{k - 1}}{2^k k!}
               \cdot (1 \cdot 3 \dotsm (2 k - 3)) \notag \\
         &= \frac{(-1)^{k - 1}}{2^k k!}
               \cdot \frac{1 \cdot 2 \cdot 3 \cdot 4
                              \cdot \dotsm
                              \cdot (2 k - 3) \cdot (2 k - 2)}
                          {2 \cdot 4 \cdot 6 \dotsm (2 k - 2)}
                                \notag \\
         &= \frac{(-1)^{k - 1}}{2^k k!}
               \cdot \frac{(2 k - 2)!}{2^{k - 1} (k - 1)!}
                  \notag \\
         &= \frac{(-1)^{k - 1}}{2^{2 k - 1} \cdot k}
               \cdot \frac{(2 k - 2)!}{(k - 1)! \, (k - 1)!}
                  \notag \\
         &= \frac{(-1)^{k - 1}}{2^{2 k - 1} \cdot k}
               \cdot \binom{2 k - 2}{k - 1}
            \label{eq:binomial(1/2,k)}
    \end{align}
    Hay que tener cuidado,
    la última fórmula no cubre el caso \(k = 0\).

  \item[\boldmath Caso \(\alpha = -1/2\):\unboldmath]
    Mucha de la derivación es similar a la del caso anterior.
    Tenemos,
    para \(k > 0\):
    \begin{align}
      \binom{-1/2}{k}
        &= \frac{(-1/2) \cdot (-1/2 - 1) \cdot \dotsm
                   \cdot (-1/2 - k + 1)}
                {k!} \notag \\
        &= (-1)^k \frac{1}{2^k}
             \cdot \frac{1 \cdot 3 \dotsm (2 k - 1)}{k!} \notag \\
        &= (-1)^k \frac{1}{2^k}
             \cdot \frac{(2 k)!}{k! \, 2^k \, k!} \notag \\
        &= (-1)^k \frac{1}{2^{2 k}} \, \binom{2 k}{k}
            \label{eq:binomial(-1/2,k)}
    \end{align}
    Esta fórmula con \(k = 0\) da:
    \begin{equation*}
      \binom{-1/2}{0} = 1
    \end{equation*}
    así no se necesita hacer un caso especial acá.
  \item[\boldmath Caso \(\alpha = -n\):\unboldmath]
    Cuando \(\alpha\) es un entero negativo,
    podemos escribir:
    \begin{equation}
      \label{eq:binomial(-n,k)}
      \binom{-n}{k}
        = \frac{(-n)^{\underline{k}}}{k!}
        = (-1)^k \, \frac{n^{\overline{k}}}{k!}
        = (-1)^k \, \frac{(n + k - 1)^{\underline{k}}}{k!}
        = (-1)^k \, \binom{k + n - 1}{n - 1}
    \end{equation}
    Una fórmula cómoda es:
    \begin{equation}
      \label{eq:serie-binomio-negativo}
      \frac{1}{(1 - z)^{n + 1}}
        = \sum_{k \ge 0} \binom{n + k}{n} \, z^k
    \end{equation}
  \end{description}
  Interesantes resultan las series al variar \(n\), no \(k\).
  Sabemos que \(\binom{n}{k} = 0\) si no es que \(0 \le k \le n\),
  podremos ahorrarnos los límites de las sumas para simplificar.
  El índice \(n\) o \(n + k\) para \(k\) fijo
  da lo mismo al sumar sobre todo \(n \in \mathbb{Z}\):
  \begin{align}
    \sum_n \binom{n}{k} \, z^n
      &= \sum_n \binom{n + k}{k} \, z^{n + k} \notag \\
      &= z^k \sum_n \binom{n + k}{n} \, z^n \notag \\
      &= \frac{z^k}{(1 - z)^{k + 1}}
            \label{eq:serie-binomio-n}
  \end{align}
  Al final usamos~\eqref{eq:serie-binomio-negativo}.
  Omitir los rangos de los índices ahorró interminables ajustes.
\subsection{Otras series}
\label{sec:otras-series}

  Una serie común es la exponencial:%
    \index{serie de potencias!exponencial}
  \begin{equation}
    \label{eq:exponencial}
    \mathrm{e}^z
      = \sum_{n \ge 0} \frac{z^n}{n!}
  \end{equation}
  A veces aparecen funciones trigonométricas:%
    \index{serie de potencias!seno}%
    \index{serie de potencias!coseno}
  \begin{equation*}
    \sin z
      = \sum_{n \ge 0} (-1)^n \frac{z^{2 n + 1}}{(2 n + 1)!} \qquad
    \cos z
      = \sum_{n \ge 0} (-1)^n \frac{z^{2 n}}{(2 n)!}
  \end{equation*}
  o hiperbólicas:%
    \index{serie de potencias!seno hiperbolico@seno hiperbólico}%
    \index{serie de potencias!coseno hiperbolico@coseno hiperbólico}
  \begin{equation*}
    \sinh z
      = \sum_{n \ge 0} \frac{z^{2 n + 1}}{(2 n + 1)!} \qquad
    \cosh z
      = \sum_{n \ge 0} \frac{z^{2 n}}{(2 n)!}
  \end{equation*}
  Una relación útil es la fórmula de Euler:%
    \index{Euler, formula de (exponencial complejo)@Euler, fórmula de (exponencial complejo)}
  \begin{equation}
    \label{eq:formula-Euler-exponencial}
    \mathrm{e}^{u + \mathrm{i} v}
      = \mathrm{e}^u (\cos v + \mathrm{i} \sin v)
  \end{equation}

  Es frecuente la serie para el logaritmo,%
    \index{serie de potencias!logaritmo}
  que podemos hallar integrando término a término:
  \begin{align}
    \frac{\mathrm{d}}{\mathrm{d} z} \, \ln (1 - z)
      &= - \frac{1}{1 - z}
       = - \sum_{n \ge 0} z^n \notag \\
    \ln (1 - z)
      &= - \sum_{n \ge 1} \frac{z^n}{n}
           \label{eq:ln(1-z)}
  \end{align}
  Muchos ejemplos adicionales de series útiles
  se hallan en el texto de Wilf~\cite{wilf06:_gfology}.

\section{Notación para coeficientes}
\label{sec:funciones-generatrices:notacion}
\index{serie de potencias!extraer coeficiente}

  Comúnmente extraeremos el coeficiente de un término de una serie.
  Generalmente no hay términos con potencias negativas de \(z\),
  tales coeficientes serán cero.
  Para esto,
  dadas las series:
  \begin{equation*}
    A(z)
      = \sum_{n \ge 0} a_n z^n
    \hspace{3em}
    B(z)
      = \sum_{n \ge 0} b_n z^n
  \end{equation*}
  usaremos la notación:
  \begin{equation*}
     \left[ z^n \right] A(z) = a_n
  \end{equation*}
  Tenemos algunas propiedades simples:
  \begin{align*}
    \left[ z^n \right] z^k A(z)
      &= \begin{cases}
           0				  & n < k \\
           \left[ z^{n - k} \right] A(z)  & n \ge k
         \end{cases} \\
    \left[ z^n \right] (\alpha A(z) + \beta B(z))
      &= \alpha \left[ z^n \right] A(z)
           + \beta \left[ z^n \right] B(z)
  \end{align*}

\section{Aceite de serpiente}
\label{sec:snake-oil}
\index{generatriz!aceite de serpiente}

  La manera tradicional de simplificar sumatorias
  (particularmente las que involucran coeficientes binomiales)
  es aplicar identidades u otras manipulaciones de los índices,
  como magistralmente exponen Knuth~%
    \cite{knuth97:_fundam_algor}
  y Graham, Knuth y~Patashnik~%
    \cite{graham94:_concr_mathem}.
  Acá mostramos un método alternativo,
  que no requiere saber y aplicar
  una enorme variedad de identidades.
  Wilf~%
    \cite{wilf06:_gfology}
  le llama \emph{\foreignlanguage{english}{Snake Oil Method}},
  por la cura milagrosa que se ve en las películas del viejo oeste.
  La técnica es bastante simple:
  \begin{enumerate}
  \item
    Identificar la variable libre,
    llamémosle \(n\),
    de la que depende la suma.
    Sea \(f(n)\) nuestra suma.
  \item
    Sea \(F(z)\) la función generatriz ordinaria
    de la secuencia \(\langle f(n) \rangle_{n \ge 0}\).
  \item
    Multiplique la suma por \(z^n\) y sume sobre \(n\).
    Tenemos \(F(z)\) expresado como una doble suma,
    sobre \(n\) y la variable de la suma original.
  \item
    Intercambie el orden de las sumas,
    y exprese la suma interna en forma simple y cerrada.
  \item
    Encuentre los coeficientes,
    son los valores de \(f(n)\) buscados.
  \end{enumerate}
  Sorprende la alta tasa de éxitos de la técnica.
  Tiene la ventaja de que no requiere mayor creatividad;
  resulta claro cuándo funciona
  y es obvio cuando falla.

  Usaremos la convención que toda suma sin restricciones
  es sobre el rango \(-\infty\) a \(\infty\).
  Como los coeficientes binomiales \(\binom{n}{k}\)
  que usaremos en los ejemplos se anulan cuando
  \(k\) no está en el rango \([0, n]\),
  esto evita interminables ajustes de índices.

  Nuestro ejemplo viene de Riordan~%
    \cite{riordan68:_combin_ident},
  donde se resuelve mediante delicadas maniobras.
  Nuestro desarrollo sigue a Dobrushkin~%
    \cite{dobrushkin10:_method_algor_analysis}.

  Evaluar:
  \begin{equation*}
    h_n
      = \sum_{0 \le k \le n}
          (-1)^{n - k} \, 4^k \, \binom{n + k + 1}{2 k + 1}
  \end{equation*}

  Definimos \(H(z)\) como la función generatriz de los \(h_n\);
  multiplicamos por \(z^n\),
  sumamos para \(n \ge 0\)
  e intercambiamos orden de suma:
  \begin{align*}
    H(z)
      &= \sum_{n \ge 0} z^n
           \sum_{0 \le k \le n}
             (-1)^{n - k} \, 4^k \, \binom{n + k + 1}{2 k + 1} \\
      &= \sum_{n \ge 0}
           \sum_{0 \le k \le n}
             (-4)^k \, (-z)^n \, \binom{n + k + 1}{2 k + 1} \\
      &= \sum_{k \ge 0}
           (-4)^k \,
           \sum_{n \ge k}
             \binom{n + k + 1}{2 k + 1} \, (-z)^n
  \end{align*}
  Para completar el trabajo necesitamos la suma interna.
  Haciendo el cambio de variable \(r = n - k\):
  \begin{equation*}
    \sum_{n \ge k} \binom{n + k + 1}{2 k + 1} \, (-z)^n
      = (-z)^k \, \sum_{r \ge 0}
                    \binom{r + 2 k + 1}{2 k + 1} \, (-z)^r
      = \frac{(-z)^k}{(1 + z)^{2 k + 2}}
  \end{equation*}
  Substituyendo en lo anterior:
  \begin{equation*}
    H(z)
      = \sum_{k \ge 0} \frac{(4 z)^k}{(1 + z)^{2 k + 2}}
      = \frac{1}{(1 + z)^2}
          \cdot \frac{1}{1 - \frac{4 z}{(1 + z)^2}}
      = \frac{1}{(1 - z)^2}
  \end{equation*}
  Resta extraer los coeficientes,
  lo que da:
  \begin{equation*}
    h_n
      = (-1)^n \, \binom{-2}{n}
      = \binom{n + 1}{1}
      = n + 1
  \end{equation*}

\section{La fórmula de inversión de Lagrange}
\label{sec:Lagrange-inversion}

  Es común encontrarse con ecuaciones de la forma
  \begin{equation*}
    u = t \phi(u)
  \end{equation*}
  donde \(\phi\) es una función dada de \(u\).
  Esta relación define \(u\) en función de \(t\),
  y \textquote{estamos despejando \(u\) en términos de \(t\)}.
  Fue demostrada por Lagrange%
    \index{Lagrange, inversion de@Lagrange, inversión de|textbfhy}%
    \index{Lagrange-Burmann, inversion de@Lagrange-Bürmann, inversión de|see{Lagrange, inversión de}}
  y casi simultáneamente por Bürmann,%
    \index{Burmann, Hans Heinrich@Bürmann, Hans Heinrich}
  la forma dada acá es la de Bürmann.
  \begin{theorem}[Fórmula de inversión de Lagrange]
    \label{theo:LIF}
    Sean \(f(u)\) y \(\phi(u)\) series de potencias en \(u\),
    con \(\phi(0) = 1\).
    Entonces hay una única serie \(u = u(t)\) que cumple:
    \begin{equation*}
      u = t \phi(u)
    \end{equation*}
    Además,
    el valor \(f(u(t))\) de \(f\) en el cero \(u = u(t)\)
    cumple:
    \begin{equation*}
      \left[ t^n \right] \left\{ f(u(t)) \right\}
         = \frac{1}{n} \, \left[ u^{n - 1} \right] \,
                            \left\{ f'(u) \phi(u)^n \right\}
    \end{equation*}
  \end{theorem}
  Dadas \(f\) y \(\phi\),
  esta fórmula da los coeficientes de \(f(u(t))\) en bandeja.
  No demostraremos este resultado,
  ya que nos llevaría demasiado fuera del rango de este ramo.
  La demostración del teorema puede encontrarse en el texto de Wilf~%
    \cite{wilf06:_gfology}.

\bibliography{../referencias}

%%% Local Variables:
%%% mode: latex
%%% TeX-master: "../INF-221_notas"
%%% ispell-local-dictionary: "spanish"
%%% End:

% LocalWords:  see eq Factorizamos gf textbfhy ogf english Ordinary
% LocalWords:  Generating Function egf Exponential geometrica Snake
% LocalWords:  Oil Method inversion Burmann Bürmann Hans Heinrich
