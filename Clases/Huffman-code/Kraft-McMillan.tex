\section{La desigualdad de Kraft-McMillan}
\label{sec:Kraft-McMillan}

  La desigualdad de Kraft~%
    \cite{kraft49:_coding}
  limita los largos de los códigos en un código prefijo.
  A su vez,
  McMillan~%
    \cite{mcmillan56:_two_inequalities}
  demostró que si un código no cumple la desigualdad de Kraft
  no puede decodificarse en forma única.

  Demostraremos ambas.
  \begin{theorem}
    Sean los símbolos del alfabeto \(\Sigma = \{a_1, a_2, \dotsc, a_n\}\)
    codificados mediante un código decodificable de forma única
    sobre un alfabeto de tamaño \(r\),
    con palabras de código respectivamente \(\ell_1, \ell_2, \ldots, \ell_n\).
    Entonces:
    \begin{equation}
      \label{eq:Kraft-inequality}
      \sum_{1 \le k \le n} r^{\ell_k}
        \le 1
    \end{equation}
    Por el contrario,
    para un conjunto de números naturales \(\ell_1, \ell_2, \ldots, \ell_n\)
    que cumplen la desigualdad indicada hay un código con  palabras de código
    de los respectivos largos que es decodificable en forma única.
  \end{theorem}
  \begin{proof}
    Primero demostraremos que la desigualdad de Kraft~\ref{eq:Kraft-inequality}
    se cumple para códigos prefijo
    y que hay un código prefijo con esos largos si se cumple la desigualdad.
    Sin pérdida de generalidad,
    suponemos \(\ell_1 \le \ell_2  \le \dotsb \le \ell_n\).
    Sea \(A\) el árbol \(r\)\nobreakdash-ario completo de altura \(\ell_n\),
    con lo que podemos considerar las palabras del código prefijo
    como nodos de \(A\).
    Sea \(A_k\) el conjunto de hojas del subárbol que tiene el nodo \(v_k\)
    (correspondiente al código de \(a_k\))
    de \(A\) como raíz.
    Como la altura del subárbol con raíz \(v_k\) es \(\ell_n - \ell_k\),
    tiene \(\lvert A_k \rvert = r^{\ell_n - \ell_k}\) hojas.
    Como es un código prefijo,
    sabemos que \(A_r \cap A_s = \varnothing\) si \(r \ne s\).
    Como el total de hojas de \(A\) es \(r^{\ell_n}\),
    concluimos que:
    \begin{equation*}
      \left\lvert \bigcup_{1 \le k \le n} A_k \right\rvert
        =   \sum_{1 \le k \le n} \lvert A_k \rvert
        =   \sum_{1 \le k \le n} r^{\ell_n - \ell_k}
        \le r^{\ell_n}
    \end{equation*}
    de donde sigue la desigualdad~\ref{eq:Kraft-inequality}.

    Para el recíproco,
    dada una secuencia de enteros \(\ell_1 \le \ell_2 \le \dotsb \le \ell_n\)
    que cumplen la desigualdad~\ref{eq:Kraft-inequality},
    podemos construir un código prefijo con códigos de esos largos.
    La idea es elegir arbitrariamente
    un código de largo \(\ell_1\) para \(a_1\).
    En el árbol \(A\)
    esto corresponde a eliminar como posibles códigos
    los prefijos de ese código.
    También elimina los descendientes del vértice \(v_i\),
    en particular las hojas \(A_1\),
    que son \(r^{\ell_n - \ell_1}\).
    Podemos elegir del resto del árbol algún código de largo \(\ell_2\),
    lo que elimina \(r^{\ell_n - \ell_2}\) hojas adicionales.

    Siguiendo de esta forma,
    si se cumple la desigualdad vemos que esto es posible en cada paso,
    construimos un código prefijo con los largos prescritos.
    Como \(\ell_{k + 1} \ge \ell_k\),
    por la forma de los subárboles
    no podemos quedar con un conjunto de hojas que no tienen ascendente común,
    hay cómo elegir cada uno de los códigos.

    Ahora demostraremos que si el código se puede decodificar en forma única,
    entonces cumple la desigualdad de Kraft.
    El recíproco de esto ya lo demostramos,
    un código prefijo obviamente se puede decodificar en forma única.

    Sea:
    \begin{equation*}
      C
        = \sum_{1 \le k \le n} r^{-\ell_k}
    \end{equation*}
    La idea es acotar \(C^m\) para \(m \in \mathbb{N}\)
    y demostrar que la cota solo es posible para todo \(m\) si \(C \le 1\).
    Escribimos:
    \begin{align*}
      C^m
        &= \left( \sum_{1 \le k \le n} r^{-\ell_k} \right)^m \\
        &= \sum_{1 \le r_1 \le n}
             \sum_{1 \le r_2 \le n} \cdots \sum_{1 \le r_n \le n}
               r^{-\ell_{r_1} - \ell_{r_2} - \dotsm - \ell_{r_n}}
    \end{align*}
    Considere ahora todas las palabras en \(\Sigma^m\),
    si las codificamos con el código supuesto todos los códigos serán distintos.
    Esto se traduce en que cada término de la suma
    corresponde a una palabra en \(\Sigma^m\).
    Si llamamos \(q_\ell\)
    el número de codificaciones de \(\Sigma^m\) de largo \(\ell\),
    resultando:
    \begin{equation*}
      C^m
        = \sum_{\ell \ge 1} q_\ell r^{-\ell}
    \end{equation*}
    Para un alfabeto de \(r\) símbolos
    hay solo \(r^\ell\) códigos de largo \(\ell\),
    con lo que tenemos la cota simple \(q_\ell \le r^\ell\).
    Si \(\ell_{\mathtt{max}}\)
    es el largo máximo de un código
    esto da la cota:
    \begin{align*}
      C^m
        &\le \sum_{1 \le \ell \le m \ell_{\mathtt{max}}} r^{\ell} r^{-\ell} \\
        &=   m \ell_{\mathtt{max}} \\
      C
        &\le \left( m \ell_{\mathtt{max}} \right)^{1 / m}
    \end{align*}
    Esto debe cumpĺirse para todo \(m\).
    Pero:
    \begin{equation*}
      \lim_{m \to \infty} \left( m \ell_{\mathtt{max}} \right)^{1 / m}
        = 1
    \end{equation*}
    O sea,
    debe ser \(C \le 1\),
    de lo contrario la desigualdad
    se violaría para \(m\) suficientemente grande.
  \end{proof}

%%% Local Variables:
%%% mode: latex
%%% TeX-master: "../INF-221_notas"
%%% End:

% LocalWords:  decodificable max
