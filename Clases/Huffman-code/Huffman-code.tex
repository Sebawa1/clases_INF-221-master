\bibliographystyle{babplain-fl}

\chapter{Código Huffman}
\label{cha:Huffman}

  El código Huffman~%
    \cite{huffman52:_method_const_minim_redun_codes}
  es una aplicación muy importante de algoritmo voraz.
  Lo desarrolló como estudiante de pregrado,
  cuando se presentó la alternativa de dar un examen final
  o escribir un trabajo sobre la optimalidad de ciertos códigos
  en su ramo de teoría de comunicaciones.
  Vio que no podría resolver el problema planteado,
  y estaba a punto de abandonar cuando se le ocurrió la idea de este algoritmo,
  demostrando que es óptimo.
  En esto le ganó a sus profesores,
  que estaban desarrollando el código Shannon-Fano que resulta no ser óptimo.

  Dado un texto,
  formado por símbolos,
  buscamos codificarlo eficientemente.
  Si cada símbolo se codifica en \(k\) bits
  (total \(2^k\) símbolos posibles),
  de texto de largo \(n\) usa \(n k\) bits.
  En texto,
  las frecuencias son \emph{muy} desiguales.
  Por ejemplo,
  en la novela Moby Dick
  aparece~\num{117\,194} veces la letra~'e',
  y~\num{640} veces la~'z'.
  Nuestro principal objetivo es asignarle códigos cortos
  a los símbolos que más se repiten,
  a costa de codificaciones largas para símbolos poco frecuentes.

  Pero hay que tener cuidado:
  \begin{align*}
    a &\mapsto 0 \\
    b &\mapsto 1 \\
    c &\mapsto 01
  \end{align*}
  Con esta codificación escribimos:
  \begin{equation}
    \label{09::PrimerCodigo}
    a b a b c \rightsquigarrow 010101
  \end{equation}
  Pero también:
  \begin{equation}
    \label{09::SegundoCodigo}
    c c c \rightsquigarrow 010101
  \end{equation}
  ¡Se produce ambigüedad entre~\eqref{09::PrimerCodigo}
  y~\eqref{09::SegundoCodigo}!

  Obviamente nos interesan códigos que tengan decodificación única.
  Condición suficiente para evitar ambigüedades es que
  ningún código sea prefijo de otro
  (\textquote{\emph{\foreignlanguage{english}{prefix-free code}}}
   o \textquote{\emph{\foreignlanguage{english}{prefix code}}}).
  La desigualdad de Kraft\nobreakdash-McMillan~%
    \cite{kraft49:_coding, mcmillan56:_two_inequalities}
  muestra que si un código puede decodificarse en forma única,
  hay un código prefijo con códigos del mismo largo para cada símbolo.
  Esto es importante porque hace eficiente el decodificar.

  Podemos describir un código prefijo binario como un árbol binario,
  las hojas son los símbolos y el camino desde la raíz a la hoja es el código
  (por ejemplo,
   ir al hijo izquierdo es 0,
   ir al derecho 1).
  Dadas las frecuencias con que aparecen los símbolos en las hojas,
  podemos asignar frecuencias a nodos internos del árbol
  como la suma de las frecuencias de todas las hojas descendientes
  (o sea,
   la suma de la frecuencia de los hijos
   en los nodos internos).
  Un ejemplo es la figura~\ref{09::EjemploArbolHuffman},
  el código representado es el del cuadro~\ref{tab:EjemploArbolHuffman}.
  \begin{figure}[ht]
    \centering
    \begin{tikzpicture}[scale = 0.75]
      \coordinate (r)	 at (3, 3);
      \coordinate (0)	 at (0, 2);
      \coordinate (1)	 at (6, 2);
      \coordinate (10)	 at (4, 1);
      \coordinate (100)	 at (3, 0);
      \coordinate (101)	 at (5, 0);
      \coordinate (11)	 at (8, 1);

      \draw (r)	 -- node [above] {\num{0}} (0);
      \draw (r)	 -- node [above] {\num{1}} (1);
      \draw (1)	 -- node [above] {\num{0}} (10);
      \draw (1)	 -- node [above] {\num{1}} (11);
      \draw (10) -- node [above left] {\num{0}} (100);
      \draw (10) -- node [above right] {\num{1}} (101);

      \draw [fill] (r)	 circle [radius = 3pt];
      \draw [fill] (1)	 circle [radius = 3pt];
      \draw [fill] (10)	 circle [radius = 3pt];

      \node [below = 0.1 of 0]	 {\(a\)};
      \node [below = 0.1 of 100] {\(b\)};
      \node [below = 0.1 of 101] {\(d\)};
      \node [below = 0.1 of 11]	 {\(c\)};
    \end{tikzpicture}
    \caption{Ejemplo de código prefijo como árbol}
    \label{09::EjemploArbolHuffman}
  \end{figure}
  \begin{table}[ht]
    \centering
    \begin{tabular}{>{\(}c<{\)}>{\(}l<{\)}}
      \multicolumn{1}{c}{\textbf{Símbolo}} &
        \multicolumn{1}{c}{\textbf{Código}} \\
      \hline
        a & 0	\\
        b & 100 \\
        c & 11	\\
        d & 101
    \end{tabular}
    \caption{Código representado en la figura~\ref{09::EjemploArbolHuffman}}
    \label{tab:EjemploArbolHuffman}
  \end{table}
  Queremos nodos caros
  (alta frecuencia)
  cerca de la raíz,
  nodos baratos
  (bajas frecuencias)
  lejos.
  Una idea es agrupar desde las hojas,
  partiendo con los nodos más baratos,
  considerando igualmente nodos internos resultantes en iteraciones sucesivas.

\section{Descripción del problema}

  Para atacar el problema,
  y finalmente demostrar que el algoritmo esbozado da un óptimo,
  hay que describirlo formalmente.

  Dada una secuencia \(T\) sobre \(\Sigma = \{x_1, \ldots, x_n\}\),
  donde \(x_i\) aparece con frecuencia \(f_i\),
  construir una función de codificación
    \(C \colon \Sigma \rightarrow \text{cadenas de bits}\),
  tal que \(C\) es un código prefijo
  y el número total de bits para representar \(T\) se minimiza.
  \begin{definition}
    La \emph{profundidad} de la hoja \(\ell _i\),
    anotada \(d(\ell _i)\),
    es el largo del camino de la raíz a esa hoja.
  \end{definition}
  En el código descrito por el árbol \(R\)
  el símbolo \(x_i\) queda codificado por \(d(x_i)\) bits,
  el texto completo queda representado por el siguiente número de bits:
  \begin{equation*}
    B(R)
      = \sum_i f_i d(x_i)
  \end{equation*}
  Observamanos que:
  \begin{itemize}
  \item
    Si \(R\) es óptimo,
    todo nodo interno tiene dos hijos.

    Si hubiese un nodo interno con un único hijo,
    podríamos acortar los caminos
    (códigos)
    de descendientes de su hijo
    haciéndolos depender directamente del nodo.
  \item
    Hay dos hojas \(x_a\), \(x_b\)
    a la profundidad máxima que son hermanos.

    Por el punto anterior,
    no pueden haber nodos internos con un único hijo,
    toda hoja tiene un hermano.
  \end{itemize}
  Intuitivamente,
  buscamos letras poco frecuentes a altas profundidades,
  frecuentes a profundidades bajas.
  Lo que hace el algoritmo de Huffman
  es asignar desde los símbolos menos frecuentes,
  agrupando colecciones de símbolos.
  Sea \(L = (\ell_1, \ldots, \ell_n)\) el conjunto de hojas
  para todos los símbolos,
  y sea \(f_i\) la frecuencia de la letra \(x_i\).
  Hallar las dos letras de frecuencia mínima,
  digamos \(x_a\) y \(x_b\) con frecuencias \(f_a\) y \(f_b\).
  Unir sus hojas en la hoja \(\ell_{a }\) con frecuencia \(f_a+f_b\)
  dando un árbol \(R_{a b}\)
  (figura~\ref{09::RabTree}):
  \begin{figure}[ht]
    \centering
    \begin{tikzpicture}
          [every state/.style = {draw = black,
                                 fill = black,
                                 minimum size = 2mm}
          ]
      \node [state] (d0) {};
      \node [above of = d0, yshift = -42] (dummy:ce) {\(x_{ab}\)};
      \node [below left of = d0] (xa) {\(x_a\)};
      \node [below right of = d0] (xb) {\(x_b\)};
      \path (d0) edge node [above left] {\num{0}} (xa)
      edge node [above right] {\num{1}} (xb);
    \end{tikzpicture}
    \caption{El nodo \(x_{a b}\) es la unión entre \(x_a\) y \(x_b\).}
    \label{09::RabTree}
  \end{figure}
  Recursivamente resolver el problema con:
  \begin{equation}
    L
      = \{ \ell _1, \ldots, \ell _n \}
           \smallsetminus \{ \ell _a, \ell _b \} \cup \{ \ell _{a b}\}
  \end{equation}
  y frecuencias ajustadas (\(\ell_{a b}\rightsquigarrow f_a+f_b\))
  \begin{ejemplo}
    Considere el cuadro~\ref{09::CuadroEjemplo1}.
    \begin{table}[ht]
      \centering
      \begin{tabular}{>{\(}c<{\)}|c}
        \multicolumn{1}{c|}{\textbf{Símbolo}}
            & \textbf{Frecuencia} \\
        \hline
        a & \phantom{0}9 \\
        b & \phantom{0}4 \\
        c & \phantom{0}2 \\
        d & 15 \\
        e & \phantom{0}3 \\
        f & 17
      \end{tabular}
      \caption{Frecuencias de los símbolos \(a, b, c, d, e, f\).}
      \label{09::CuadroEjemplo1}
    \end{table}
    El algoritmo de Huffman
    encuentra los dos símbolos de menor frecuencia
    y crea un subárbol con ellos.
    En el cuadro~\ref{09::CuadroEjemplo1},
    se tiene que los símbolos \(c\) y \(e\)
    son los de menor frecuencia.
    Luego,
    formamos un subárbol con ellos
    (figura~\ref{09::EjemploArbol1}).
    \begin{figure}[ht]
      \centering
      \begin{tikzpicture}
          [every state/.style = {draw = black,
                                 fill = black,
                                 minimum size = 2mm}
          ]
        \node [state] (ce) {};
        \node [above of = ce, yshift = -42] (dummy:ce) {\(c e\)};
        \node [below left of = ce] (c) {\(c\)};
        \node [below right of = ce] (e) {\(e\)};
        \path (ce) edge node [above left] {\num{0}} (c)
           edge node [above right] {\num{1}} (e);
      \end{tikzpicture}
      \caption{Hojas son los símbolos que menos se repiten.}
      \label{09::EjemploArbol1}
    \end{figure}
    Agregamos este \textquote{nodo conjunto} al cuadro~\ref{09::CuadroEjemplo1},
    cuya frecuencia es equivalente
    al peso del árbol de la figura~\ref{09::EjemploArbol1}
    (suma de las frecuencias de \(c\) y \(e\)).
    El resultado se aprecia en el cuadro~\ref{09::CuadroEjemplo2}.
    \begin{table}[ht]
      \centering
      \begin{tabular}{>{\(}c<{\)}|c}
        \multicolumn{1}{c|}{\textbf{Símbolo}} & \textbf{Frecuencia} \\
        \hline
          a   & \phantom{0}9 \\
          b   & \phantom{0}4 \\
          d   & 15 \\
          f   & 17 \\
          c e & \phantom{0}5
      \end{tabular}
      \caption{Nodo conjunto \(c e\)
               con frecuencia la suma de las de \(c\) y \(e\).}
      \label{09::CuadroEjemplo2}
    \end{table}
    Repetimos el proceso,
    es decir,
    escogemos dos símbolos del cuadro~\ref{09::CuadroEjemplo2}
    que tienen menor frecuencia
    y creamos un nuevo subárbol.
    Estos símbolos son \(c e\) y \(b\).
    La figura~\ref{09::EjemploArbol2} muestra el árbol resultante.
    \begin{figure}[ht]
      \centering
      \begin{tikzpicture}
          [every state/.style = {draw = black,
                                 fill = black,
                                 minimum size = 2mm}
          ]
        \node [state] (ce) {};
        \node [above of = ce, yshift = -42] (dummy:ce) {\(c e\)};
        \node [below left of = ce] (c) {\(c\)};
        \node [below right of = ce] (e) {\(e\)};
        \path (ce) edge node [above left] {\num{0}} (c)
          edge node [above right] {\num{1}} (e);

        \node [above left of = ce, state] (bce) {};
        \node [below left of = bce] (b) {\(b\)};
        \node [above of = bce, yshift = -42] (dummy:bce) {\(b c e\)};
        \path (bce) edge node [above left] {\num{0}} (b)
          edge node [above right] {\num{1}} (ce);
      \end{tikzpicture}
      \caption{Hojas son los símbolos que menos se repiten.}
      \label{09::EjemploArbol2}
    \end{figure}
    Reemplazamos los símbolos \(b\) y \(c e\)
    del cuadro~\ref{09::CuadroEjemplo2}
    con \(b c e\) de frecuencia \(f_{bce} = 9\).
    El resultado queda en el cuadro~\ref{09::CuadroEjemplo3}.
    \begin{table}[ht]
      \centering
      \begin{tabular}{>{\(}c<{\)}|c}
        \multicolumn{1}{c|}{\textbf{Símbolo}} & \textbf{Frecuencia} \\
        \hline
          a	& \phantom{0}9 \\
          d	& 15 \\
          f	& 17 \\
          b c e & \phantom{0}9
      \end{tabular}
      \caption{Reemplazando los símbolos \(b\) y \(c e\)
               del cuadro~\ref{09::CuadroEjemplo2}.}
      \label{09::CuadroEjemplo3}
    \end{table}
    Iteramos nuevamente.
    En el cuadro~\ref{09::CuadroEjemplo3}
    se tiene que los dos símbolos con menor frecuencia
    son \(b c e\) y \(a\).
    El árbol resultante es el de la figura~\ref{09::EjemploArbol3}.
    \begin{figure}[ht]
      \centering
      \begin{tikzpicture}
          [every state/.style = {draw = black,
                                 fill = black,
                                 minimum size = 2mm}
          ]
        \node [state] (ce) {};
        \node [above of = ce, yshift = -42] (dummy:ce) {\(c e\)};
        \node [below left of = ce] (c) {\(c\)};
        \node [below right of = ce] (e) {\(e\)};
        \path (ce) edge node [above left] {\num{0}} (c)
          edge node [above right] {\num{1}} (e);

        \node [above left of = ce, state] (bce) {};
        \node [below left of = bce] (b) {\(b\)};
        \node [above of = bce, yshift = -42, xshift = 5] (dummy:bce)
          {\(b c e\)};
        \path (bce) edge node [above left] {\num{0}} (b)
          edge node [above right] {\num{1}} (ce);

        \node [state, above left of = bce] (abce) {};
        \node [below left of = abce] (a) {\(a\)};
        \node [above of = abce, yshift = -42] (dummy:abce)
          {\(a b c e\)};

        \path (abce) edge node [above right] {\num{1}} (bce)
          edge node [above left] {\num{0}} (a);
      \end{tikzpicture}
      \caption{Árbol con peso de \(f_{bce}+f_a = 18\).}
      \label{09::EjemploArbol3}
    \end{figure}
    Quitamos estos símbolos del cuadro~\ref{09::CuadroEjemplo3}
    y los reemplazamos por \(a b c e\).
    El resultado es el cuadro~\ref{09::CuadroEjemplo4}.
    \begin{table}[ht]
      \centering
      \begin{tabular}{>{\(}c<{\)}|c}
        \multicolumn{1}{c|}{\textbf{Símbolo}} & \textbf{Frecuencia} \\
        \hline
        d	& 15 \\
        f	& 17 \\
        a b c e & 18
      \end{tabular}
      \caption{Reemplazando los símbolos \(b c e\) y \(a\)
               del cuadro~\ref{09::CuadroEjemplo3}.}
      \label{09::CuadroEjemplo4}
    \end{table}
    En el cuadro~\ref{09::CuadroEjemplo4}
    vemos que los símbolos con menor frecuencia son \(d\) y \(f\).
    Tomamos estos dos símbolos
    y creamos un nuevo árbol que los tenga como hojas
    (figura~\ref{09::EjemploArbol4}).
    \begin{figure}[ht]
      \centering
      \begin{tikzpicture}
          [every state/.style = {draw = black,
                                 fill = black,
                                 minimum size = 2mm}
          ]
        \node [state] (df) {};
        \node [below left of = df] (d) {\(d\)};
        \node [below right of = df] (f) {\(f\)};
        \node [above of = df, yshift = -42] (dummy:df) {\(df\)};
        \path (df) edge node [above left] {\num{0}} (d)
          edge node [above right] {\num{1}} (f);
      \end{tikzpicture}
      \caption{Hojas son los símbolos de menor frecuencia
               del cuadro~\ref{09::CuadroEjemplo4}.}
      \label{09::EjemploArbol4}
    \end{figure}
    Sacamos esos símbolos
    y los reemplazamos por \(d f\),
    dando el cuadro~\ref{09::CuadroEjemplo5}.
    \begin{table}[ht]
      \centering
      \begin{tabular}{c|c}
        \multicolumn{1}{c|}{\textbf{Símbolo}} & \textbf{Frecuencia} \\
        \hline
          d f	  & 32\\
          a b c e & 18
      \end{tabular}
      \caption{Reemplazando \(d\) y \(f\)
               del cuadro~\ref{09::CuadroEjemplo4}.}
      \label{09::CuadroEjemplo5}
    \end{table}
    Tomamos los dos últimos símbolos
    y creamos el árbol final
    de la figura~\ref{09::EjemploArbol5}.
    \begin{figure}[ht]
      \centering
      \begin{tikzpicture}
          [every state/.style = {draw = black,
                                 fill = black,
                                 minimum size = 2mm}
          ]
        \node [state] (ce) {};
        \node [above of = ce, yshift = -42] (dummy:ce) {\(c e\)};
        \node [below left of = ce] (c) {\(c\)};
        \node [below right of = ce] (e) {\(e\)};
        \path (ce) edge node [above left] {\num{0}} (c)
          edge node [above right] {\num{1}} (e);

        \node [above left of = ce, state] (bce) {};
        \node [below left of = bce] (b) {\(b\)};
        \node [above of = bce, yshift = -42, xshift = 5] (dummy:bce)
          {\(b c e\)};
        \path (bce) edge node [above left] {\num{0}} (b)
          edge node [above right] {\num{1}} (ce);

        \node [state, above left of = bce] (abce) {};
        \node [below left of = abce] (a) {\(a\)};
        \node [above of = abce, yshift = -42, xshift = 10] (dummy:abce)
          {\(a b c e\)};

        \path (abce) edge node [above right] {\num{1}} (bce)
          edge node [above left] {\num{0}} (a);

        \node [state, above left of = abce, xshift = -20, yshift = -10]
          (final) {};

        \node [state, below left of = final, yshift = 10, xshift = -10] (df) {};
        \node [below left of = df] (d) {\(d\)};
        \node [below right of = df] (f) {\(f\)};
        \node [above of = df, yshift = -42, xshift = -5] (dummy:df) {\(df\)};
        \path (df) edge node [above left] {\num{0}} (d)
          edge node [above right] {\num{1}} (f);

        \path (final) edge node [above left] {\num{0}} (df)
          edge node [above right] {\num{1}} (abce);

      \end{tikzpicture}
      \caption{Este árbol tiene un peso de \(f_{bce} + f_a = 18\).}
      \label{09::EjemploArbol5}
    \end{figure}
    La codificación resultante
    se lee directamente del árbol,
    es la del cuadro~\ref{tab:huffman-example-code}.
    \begin{table}[ht]
      \centering
      \begin{tabular}{>{\(}c<{\)}|r|l}
        \multicolumn{1}{c|}{\textbf{Símbolo}}
          & \multicolumn{1}{c|}{\textbf{Freq}}
          & \multicolumn{1}{c}{\textbf{Código}} \\
        \hline
        a &  9 & 10   \\
        b &  4 & 110  \\
        c &  2 & 1110 \\
        d & 15 & 00   \\
        e &  3 & 1111 \\
        f & 17 & 01
        \end{tabular}
      \caption{Codificación del ejemplo}
      \label{tab:huffman-example-code}
    \end{table}
    Si usáramos un código de largo fijo,
    requeriríamos \(\lceil \log_2 6 \rceil = 3\) bits por símbolo.
    Nuestro código da \num{2,28}~bits en promedio.
  \end{ejemplo}

\section{Algoritmo}

  Sucesivamente:
  \begin{enumerate}
  \item
    Tome los dos símbolos con menos frecuencia de su tabla
    y reemplácelos por un nuevo símbolo que representa a ambos.
    Supongamos que estos símbolos son \(x_a\) y \(x_b\),
    entonces el nuevo símbolo es \(x_{a b}\).
    La frecuencia de este símbolo conjunto
    será la suma de la frecuencia de \(x_a\) y \(x_b\).
  \item
    Cree un árbol que tenga como raíz
    al símbolo conjunto \(x_{a b}\) con \(x_{a}\) y \(x_{b}\) como hijos.
  \item
    Volver al paso 1 hasta que nuestra tabla esté formada
    por solo un símbolo conjunto,
    que representará a todos los símbolos de \(\Sigma\).
  \end{enumerate}
  Llegamos a la parte entretenida:
  demostrar que el algoritmo de Huffman halla un árbol óptimo.
  \begin{proof}
    Para demostrar que da un óptimo,
    usamos el teorema~\ref{theo:esquema-voraz}.
    \begin{description}
    \item[Elección Voraz:]
      Sea \(L\) la instancia original
      (o sea, el texto completo,
      con la frecuencia de cada símbolo respectivo),
      sean \(\ell_a\) y \(\ell_b\) las hojas menos frecuentes.
      Entonces hay un árbol óptimo que incluye \(R_{a b}\).

      Sea \(R\) un árbol óptimo para \(L\).
      Si \(R_{a b}\) es parte de \(R\),
      salimos a carretear.
      Si el árbol \(R_{a b}\) \emph{no} es parte de \(R\),
      sean \(\ell_x\), \(\ell_y\) dos hojas en \(R\) con padre común
      (hermanos),
      con \(\delta = d(\ell _x) = d(\ell_y)\) máximo.

      Claramente,
      \(a\) o \(b\) pueden coincidir con \(x\) o \(y\).
      Consideraremos el caso en que son diferentes,
      la situación en que alguno coincide es similar.

      Obtenga \(R^*\)
      intercambiando \(x\leftrightarrow a\), \(y\leftrightarrow b\),
      \(R^*\) contiene \(R_{a b}\).
      Sea \(B(R)\) el número de bits usados por el árbol \(R\)
      (la profundidad \(d\) hace referencia al árbol original \(R\)).
       En el árbol \(R^*\):
       \begin{align*}
         B(R^*)
           &= B(R) - (f_x + f_y) \delta
                   - f_a d(\ell_a)
                   - f_b d(\ell_b)
                   + (f_a + f_b) \delta
                   + f_x d(\ell_a) + f_y d(\ell_b) \\
           &= B(R) - \underbrace{(f_x-f_a)}_{> 0}
                     \underbrace{(\delta +d(\ell_a)}_{> 0}
                   - \underbrace{(f_y - f_b)}_{> 0}
                     \underbrace{(\delta + d(\ell_b))}_{> 0}
       \end{align*}
       Pero \(R\) es óptimo.
       Hemos llegado a una contradicción.
    \item[Estructura inductiva:]
      Elegir un (sub)árbol no interfiere con los demás.
    \item[Subestructura óptima:]
      Sean \(x, y\) los símbolos menos frecuentes,
      con frecuencias \(f_x\) y \(f_y\),
      respectivamente.
      Sea \(R'\) el árbol óptimo para el problema con el \textquote{símbolo}
      \(x y\) con frecuencia \(f_x + f_y\).
      Debemos demostrar que el árbol \(R\),
      con \(x y\) reemplazado por el subárbol respectivo con \(x\) e \(y\)
      es óptimo.
      Usaremos primas para distinguir valores en \(R'\) de valores en \(R\).
      Primeramente:
      \begin{align*}
        B(R)
          &= \sum_{s \in \Sigma} f_s d(s) \\
          &= \sum_{s \in \Sigma \smallsetminus \{ x, y \}} f_s d(s)
               + f_x d(x) + f_y d(y) \\
          &= \sum_{s \in \Sigma \smallsetminus \{ x, y \}} f_s d(s)
               + (f_x + f_y) (d'(x y) + 1) \\
          &= \sum_{s \in \Sigma \smallsetminus \{ x, y \}} f_s d(s)
               + f'_{x y} d'(x y) + f_x + f_y \\
          &= B(R') + f_x + f_y
      \end{align*}
      Para llegar a una contradicción,
      suponga que \(R\) no es óptimo,
      y sea \(T\) un árbol óptimo con \(x\) e \(y\) de hojas hermanas
      (sabemos que existe por lo anterior).
      Sea \(T'\) el árbol que resulta de eliminar \(x\) e \(y\).
      Podemos ver \(T'\) como un árbol
      para el alfabeto \(\Sigma \smallsetminus \{ x, y \} \cup \{ x y \}\)
      (agrupa \(x\) con \(y\)).
      Podemos repetir el cálculo anterior para obtener
      \(B(T) = B(T') + f_x + f_y\).
      O sea:
      \begin{align*}
        B(R')
          &= B(R) - f_x - f_y \\
          &> B(T) - f_x - f_y \\
          &= B(T')
      \end{align*}
      Pero supusimos que \(R'\) era óptimo.
    \end{description}
  \end{proof}
% To do:
% - Own (Python) programs
% - Apply to a (Spanish?) text, compute savings

\section{Comentarios finales}
\label{sec:Huffman-comments}

  Este esquema es óptimo bajo el supuesto que los símbolos aparecen al azar.
  Pero sabemos que esto no es así,
  por ejemplo,
  ciertas combinaciones están prohibidas en castellano.
  En un texto particular habrán palabras que se repiten con frecuencia,
  aprovechar estas particularidades es otro problema.
  Los algoritmos de compresión de datos que se usan hoy aprovechan esto,
  pero en algún nivel usan código Huffman por ser simple y eficiente.
  Discusión del ubicuo formato Zip
  y un programa ejemplo
  (que muestra cómo la teoría precedente
   --- junto con otras técnicas de compresión de datos ---
   se puede llevar a código práctico)
  ofrece Wennborg~%
    \cite{wennborg20:_zip_files}.

\section{La desigualdad de Kraft-McMillan}
\label{sec:Kraft-McMillan}

  La desigualdad de Kraft~%
    \cite{kraft49:_coding}
  limita los largos de los códigos en un código prefijo.
  A su vez,
  McMillan~%
    \cite{mcmillan56:_two_inequalities}
  demostró que si un código no cumple la desigualdad de Kraft
  no puede decodificarse en forma única.

  Demostraremos ambas.
  \begin{theorem}
    Sean los símbolos del alfabeto \(\Sigma = \{a_1, a_2, \dotsc, a_n\}\)
    codificados mediante un código decodificable de forma única
    sobre un alfabeto de tamaño \(r\),
    con palabras de código respectivamente \(\ell_1, \ell_2, \ldots, \ell_n\).
    Entonces:
    \begin{equation}
      \label{eq:Kraft-inequality}
      \sum_{1 \le k \le n} r^{\ell_k}
        \le 1
    \end{equation}
    Por el contrario,
    para un conjunto de números naturales \(\ell_1, \ell_2, \ldots, \ell_n\)
    que cumplen la desigualdad indicada hay un código con  palabras de código
    de los respectivos largos que es decodificable en forma única.
  \end{theorem}
  \begin{proof}
    Primero demostraremos que la desigualdad de Kraft~\ref{eq:Kraft-inequality}
    se cumple para códigos prefijo
    y que hay un código prefijo con esos largos si se cumple la desigualdad.
    Sin pérdida de generalidad,
    suponemos \(\ell_1 \le \ell_2  \le \dotsb \le \ell_n\).
    Sea \(A\) el árbol \(r\)\nobreakdash-ario completo de altura \(\ell_n\),
    con lo que podemos considerar las palabras del código prefijo
    como nodos de \(A\).
    Sea \(A_k\) el conjunto de hojas del subárbol que tiene el nodo \(v_k\)
    (correspondiente al código de \(a_k\))
    de \(A\) como raíz.
    Como la altura del subárbol con raíz \(v_k\) es \(\ell_n - \ell_k\),
    tiene \(\lvert A_k \rvert = r^{\ell_n - \ell_k}\) hojas.
    Como es un código prefijo,
    sabemos que \(A_r \cap A_s = \varnothing\) si \(r \ne s\).
    Como el total de hojas de \(A\) es \(r^{\ell_n}\),
    concluimos que:
    \begin{equation*}
      \left\lvert \bigcup_{1 \le k \le n} A_k \right\rvert
        =   \sum_{1 \le k \le n} \lvert A_k \rvert
        =   \sum_{1 \le k \le n} r^{\ell_n - \ell_k}
        \le r^{\ell_n}
    \end{equation*}
    de donde sigue la desigualdad~\ref{eq:Kraft-inequality}.

    Para el recíproco,
    dada una secuencia de enteros \(\ell_1 \le \ell_2 \le \dotsb \le \ell_n\)
    que cumplen la desigualdad~\ref{eq:Kraft-inequality},
    podemos construir un código prefijo con códigos de esos largos.
    La idea es elegir arbitrariamente
    un código de largo \(\ell_1\) para \(a_1\).
    En el árbol \(A\)
    esto corresponde a eliminar como posibles códigos
    los prefijos de ese código.
    También elimina los descendientes del vértice \(v_i\),
    en particular las hojas \(A_1\),
    que son \(r^{\ell_n - \ell_1}\).
    Podemos elegir del resto del árbol algún código de largo \(\ell_2\),
    lo que elimina \(r^{\ell_n - \ell_2}\) hojas adicionales.

    Siguiendo de esta forma,
    si se cumple la desigualdad vemos que esto es posible en cada paso,
    construimos un código prefijo con los largos prescritos.
    Como \(\ell_{k + 1} \ge \ell_k\),
    por la forma de los subárboles
    no podemos quedar con un conjunto de hojas que no tienen ascendente común,
    hay cómo elegir cada uno de los códigos.

    Ahora demostraremos que si el código se puede decodificar en forma única,
    entonces cumple la desigualdad de Kraft.
    El recíproco de esto ya lo demostramos,
    un código prefijo obviamente se puede decodificar en forma única.

    Sea:
    \begin{equation*}
      C
        = \sum_{1 \le k \le n} r^{-\ell_k}
    \end{equation*}
    La idea es acotar \(C^m\) para \(m \in \mathbb{N}\)
    y demostrar que la cota solo es posible para todo \(m\) si \(C \le 1\).
    Escribimos:
    \begin{align*}
      C^m
        &= \left( \sum_{1 \le k \le n} r^{-\ell_k} \right)^m \\
        &= \sum_{1 \le r_1 \le n}
             \sum_{1 \le r_2 \le n} \cdots \sum_{1 \le r_n \le n}
               r^{-\ell_{r_1} - \ell_{r_2} - \dotsm - \ell_{r_n}}
    \end{align*}
    Considere ahora todas las palabras en \(\Sigma^m\),
    si las codificamos con el código supuesto todos los códigos serán distintos.
    Esto se traduce en que cada término de la suma
    corresponde a una palabra en \(\Sigma^m\).
    Si llamamos \(q_\ell\)
    el número de codificaciones de \(\Sigma^m\) de largo \(\ell\),
    resultando:
    \begin{equation*}
      C^m
        = \sum_{\ell \ge 1} q_\ell r^{-\ell}
    \end{equation*}
    Para un alfabeto de \(r\) símbolos
    hay solo \(r^\ell\) códigos de largo \(\ell\),
    con lo que tenemos la cota simple \(q_\ell \le r^\ell\).
    Si \(\ell_{\mathtt{max}}\)
    es el largo máximo de un código
    esto da la cota:
    \begin{align*}
      C^m
        &\le \sum_{1 \le \ell \le m \ell_{\mathtt{max}}} r^{\ell} r^{-\ell} \\
        &=   m \ell_{\mathtt{max}} \\
      C
        &\le \left( m \ell_{\mathtt{max}} \right)^{1 / m}
    \end{align*}
    Esto debe cumpĺirse para todo \(m\).
    Pero:
    \begin{equation*}
      \lim_{m \to \infty} \left( m \ell_{\mathtt{max}} \right)^{1 / m}
        = 1
    \end{equation*}
    O sea,
    debe ser \(C \le 1\),
    de lo contrario la desigualdad
    se violaría para \(m\) suficientemente grande.
  \end{proof}

%%% Local Variables:
%%% mode: latex
%%% TeX-master: "../INF-221_notas"
%%% End:

% LocalWords:  decodificable max
    

\section*{Ejercicios}
\label{sec:ejercicios-09}

  \begin{enumerate}
  \item
    Escriba un programa que lea las frecuencias de un conjunto de símbolos
    (para simplificar,
     considere caracteres ASCII únicamente)
    con sus frecuencias,
    y retorne una codificación de Huffman para ellos.
  \end{enumerate}

\bibliography{../referencias}

%%% Local Variables:
%%% mode: latex
%%% TeX-master: "../INF-221_notas"
%%% ispell-local-dictionary: "spanish"
%%% End:

% LocalWords:  pregrado optimalidad Moby Dick english prefix free sub
% LocalWords:  code Observamanos Freq Zip
