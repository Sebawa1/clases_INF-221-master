\bibliographystyle{babplain-fl}

\chapter{Sistemas de subconjuntos y matroides}
\label{cha:matroids}

  Para muchos problemas hay una respuesta
  a la pregunta de por qué funcionan los algoritmos voraces.
  \begin{definition}
    Un \emph{sistema de subconjuntos} es un conjunto \(\mathscr{I}\)
    de subconjuntos de un conjunto \(\mathscr{E}\)
    (el \emph{conjunto base})
    de modo que \(\mathscr{I}\) es cerrado bajo inclusión.
  \end{definition}
  O sea,
  si \(\mathscr{B} \in \mathscr{I}\) y \(\mathscr{A} \subseteq \mathscr{B}\)
  entonces \(\mathscr{A} \in \mathscr{I}\).
  Note en particular que si \(\mathscr{I} \ne \varnothing\),
  siempre es \(\varnothing \in \mathscr{I}\).
  Cunningham~%
    \cite{cunningham12:_coming_matroids}
  reseña la curiosa historia de estas estructuras,
  y su creciente relevancia en optimización combinatoria.
  Oxley~%
    \cite{oxley03:_matroids,oxley14:_matroids}
  da una visión general de las matemáticas relevantes
  (la segunda referencia es una versión expandida).

  El \emph{problema de optimización} para un sistema de subconjuntos
  asigna un peso positivo a cada elemento de \(\mathscr{E}\),
  y busca un conjunto \(\mathscr{X} \in \mathscr{I}\)
  cuyo peso sea máximo en todos los conjuntos de \(\mathscr{I}\).
  Es claro que podemos definir en forma afín la búsqueda de un mínimo.
  Retendremos esta definición
  para no complicar innecesariamente la discusión general.
  Parte de la discusión y los ejemplos que siguen vienen de Erickson~%
    \cite{erickson19:_algorithms}.

  Algunos ejemplos:
  \begin{itemize}
  \item
    Sea \(\mathscr{E}\) un conjunto cualquiera de vectores
    de un espacio vectorial \(V\),
    y sea \(\mathscr{I}\) el conjunto de subconjuntos de \(\mathscr{E}\)
    que son linealmente independientes.
  \item
    Considere un grafo \(G = (V, E)\).
    Considere subconjuntos de los vértices entre los que no hay arcos.
    Determinar si hay tal conjunto de \(k\) vértices
    es el problema \foreignlanguage{english}{\textsc{Independent~Set}}.
  \item
    Considere nuevamente un grafo \(G = (V, E)\).
    Considere subconjuntos de los vértices
    que están conectados todos entre sí
    (un grafo \(K_k\) como subgrafo de \(G\).
    Determinar si hay tal conjunto de \(k\) vértices
    es el problema \foreignlanguage{english}{\textsc{Clique}}.
  \item
    Considere un conjunto de tareas que usan recursos comunes
    con instantes de inicio y fin fijos.
    Conjuntos de tareas que pueden ejecutarse sin interferencia
    forman un sistema de subconjuntos.
  \item
    Considere:
    \begin{align*}
      \mathscr{E}
        &= \{ a, b, c \} \\
      \mathscr{I}
        &= \{ \varnothing, \{ a \}, \{ b \}, \{ a, b \}, \{ b, c \} \}
    \end{align*}
    Podemos verificar directamente que es cerrado bajo inclusión.
    Una manera alternativa de describir \(\mathscr{I}\)
    es como todos los conjuntos que no contienen a \(a\) y \(c\).
  \item
    Sea \(\mathscr{E}\) el conjunto de arcos de un grafo,
    y sean \(\mathscr{I}\) los conjuntos de arcos que no comparten vértices
    (se les llama \emph{\foreignlanguage{english}{matching}}
     del grafo).
  \end{itemize}

\section{Algoritmo voraz genérico}
\label{sec:greedy-generic}

  Dado un sistema finito de subconjuntos \((\mathscr{E}, \mathscr{I})\)
  hallamos un conjunto en \(\mathscr{I}\)
  mediante el algoritmo~\ref{alg:greedy-generic}.
  \begin{algorithm}[ht]
    \DontPrintSemicolon\Indp

    \Function{\(\mathrm{Greedy}(\mathscr{E}, \mathscr{I})\)}{
      \(\mathscr{X} \gets \varnothing\) \;
      Sort the elements of \(\mathscr{E}\) by decreasin weight \;
      \For{\(x \in \mathscr{E}\)}{
        \If{\(\mathscr{X} \cup \{ x \} \in \mathscr{I}\)}{
          \(\mathscr{X} \gets \mathscr{X} \cup \{ x \}\) \;
        }
      }
      \Return \(\mathscr{X}\) \;
    }
    \caption{Algoritmo voraz genérico}
    \label{alg:greedy-generic}
  \end{algorithm}
  El algoritmo retorna un conjunto \emph{maximal}
  (no se le pueden agregar elementos de \(\mathscr{E}\)
   sin salir de \(\mathscr{I}\)),
  pero no necesariamente \emph{máximo}
  (no hay elementos de \(\mathscr{I}\) de mayor peso).
  Nuestro problema de optimización pide un conjunto máximo.

  Nuestro resultado es que hay una propiedad
  de sistemas de subconjuntos finitos
  que garantiza que el algoritmo voraz~\ref{alg:greedy-generic}
  da un conjunto máximo para todas las funciones de peso.

  En nuestros ejemplos previos:
  \begin{itemize}
  \item
    Si consideramos los subconjuntos de \(\{ a, b, c \}\)
    que no tienen \(\{ a, c \}\) como subconjunto,
    y asignamos peso a los elementos,
    agregaremos \(b\) y aquél de \(\{ a, c \}\) de mayor peso.
  \item
    En los conjuntos de arcos que no forman ciclos,
    lo que tenemos
    es el problema \emph{\foreignlanguage{english}{Maximal Weight Forest}}
    (hallar el bosque de mayor peso que es subgrafo de \(G\),
     abreviado MWF).
    Este problema es equivalente a MST,
    si el máximo peso de un arco en MST es \(m\),
    asígnele peso \(2 m - w(e)\) al arco \(e\).
    El algoritmo~\ref{alg:greedy-generic} para MWF aplicado a esto
    entrega un MST
    (es el algoritmo de Kruskal~\ref{alg:kruskal} disfrazado).
  \end{itemize}

\section{Matroides y algoritmos voraces}
\label{sec:matroid-greedy}

  Diremos que un sistema de subconjuntos \((\mathscr{E}, \mathscr{I})\)
  tiene la \emph{propiedad de intercambio}
  si:
  \begin{equation}
    \label{eq:subset-system-exchange}
    \forall \mathscr{A}, \mathscr{B} \in \mathscr{I},
      (\lvert \mathscr{A} \rvert < \lvert\mathscr{B} \rvert)
        \Longrightarrow
          (\exists e \in \mathscr{B} \smallsetminus \mathscr{A}
             \text{\ tal que\ } \mathscr{A} \cup \{ e \} \in \mathscr{I})
  \end{equation}
  En estos términos:
  \begin{definition}
    Un \emph{matroide} es un sistema de subconjuntos
    \(M = (\mathscr{E}, \mathscr{I})\)
    con la propiedad de intercambio.
    Llamamos \emph{conjuntos independientes}
    a los elementos de \(\mathscr{I}\).
  \end{definition}
  Se les llama \emph{conjuntos dependientes}
  (¡Sorpresa!)
  a los subconjuntos de \(\mathscr{E}\) que no pertenecen a \(\mathscr{I}\).
  El \emph{rango} de \(\mathscr{X} \subseteq \mathscr{E}\)
  es la cardinalidad de su máximo subconjunto independiente.
  Un conjunto independiente de dice que es una \emph{base} de \(M\)
  si no es subconjunto de ningún conjunto independiente
  (es un conjunto independiente maximal).
  A un conjunto dependiente cuyos subconjuntos propios
  son todos independientes se le llama \emph{circuito}.

  Algunos ejemplos,
  varios de los cuales aparecerán nuevamente luego.
  Que son matroides quedará como ejercicio:
  \begin{description}
  \item[Matroide uniforme \boldmath\(U_{k, n}\)\unboldmath:]
    Un subconjunto \(X \subseteq \{ 1, 2, \dotsc, n \}\)
    es independiente si y solo si \(\lvert X \rvert \le k\).

    Todo subconjunto de \(k\) elementos es una base;
    todo conjunto de \(k + 1\) o más elementos es un circuito.
  \item[Matroide gráfico \boldmath\(\mathscr{M}(G)\)\unboldmath:]
    Sea \(G = (V, E)\) un grafo.
    Un subconjunto de \(E\) es independiente si no contiene ciclos.

    Una base del matroide es un árbol recubridor de \(G\);
    un circuito es un ciclo en \(G\).
  \item[Matroide cográfico \boldmath\(\mathscr{M}^*(G)\)\unboldmath:]
    Sea \(G = (V, E)\) un grafo.
    Un subconjunto \(I \subseteq E\) es independiente
    si el subgrafo complementario \((V, E \smallsetminus I)\) es conexo.

    Una base del matroide es el complemento de un árbol recubridor de \(G\);
    un circuito es un \emph{cociclo} de \(G\),
    un conjunto mínimo de arcos que desconecta a \(G\).
  \item[Matroide de correspondencias:]
    Sea \(G = (V, E)\) un grafo.
    Un conjunto \(I \subseteq V\) es independiente si y solo si
    hay un \emph{\foreignlanguage{english}{matching}}
    (conjunto de arcos que no tienen vértices en común)
    que los cubre.
  \item[Caminos disjuntos:]
    Sea \(G = (V, E)\) un grafo dirigido,
    y sea \(s\) un vértice fijo de \(G\).
    Un subconjunto \(I \subseteq V\) es independiente si y solo si
    hay caminos que no comparten arcos desde \(s\) a cada elemento de \(I\).
  \end{description}

  El resultado general es de Radó y Edmonds~%
    \cite{edmonds71:_matroid_greed_algor}:
  \begin{theorem}[Radó-Edmonds]
    \label{theo:greedy-matroid}
    Dado un sistema de subconjuntos \((\mathscr{E}, \mathscr{I})\),
    las siguientes son equivalentes:
    \begin{enumerate}
    \item
      El algoritmo voraz~\ref{alg:greedy-generic}
      entrega una solución óptima
      (mínima o máxima)
      para toda función de peso
    \item
      El sistema de subconjuntos es un matroide
    \end{enumerate}
  \end{theorem}
  \begin{proof}
    Demostramos implicancia en ambas direcciones.

    Usamos contradicción para demostrar
    que si \(M= (\mathscr{E}, \mathscr{I})\) es un matroide
    entonces el algoritmo voraz entrega un óptimo
    (máximo en la demostración,
     demostrar el caso de mínimo es simétrico).
    Sea \(\mathscr{A} = \{ a_1, a_2, \dotsc, a_k \}\)
    la solución entregada por el algoritmo voraz,
    y sea \(\mathscr{B} = \{ b_1, b_2, \cdots, b_{k'} \}\) una solución óptima,
    donde suponemos \(w(\mathscr{B}) > w(\mathscr{A})\).
    Primero,
    \(k = k'\),
    ya que si fuera \(k' \ne k\) por la propiedad de intercambio
    podríamos agregar un elemento del conjunto mayor al otro.
    Esto o contradice la optimalidad de \(\mathscr{B}\)
    o contradice el que el algoritmo terminó con \(\mathscr{A}\).
    Luego,
    podemos suponer que los elementos de \(\mathscr{A}\) y \(\mathscr{B}\)
    se listan en orden de mayor a menor
    (en \(\mathscr{A}\) es el orden en que los incluyó nuestro algoritmo),
    y consideremos el mínimo \(s\) tal que \(w(b_s) > w(a_s)\).
    Sean los subconjuntos:
    \begin{align*}
      \alpha
        &= \{ a_1, \cdots, a_{s - 1} \} \\
      \beta
        &= \{ b_1, \cdots, b_s \}
    \end{align*}
    Por ser parte de \(\mathscr{I}\)
    los conjuntos \(\mathscr{A}\) y \(\mathscr{B}\),
    también son parte de \(\mathscr{I}\) los conjuntos \(\alpha\) y \(\beta\).
    Por la propiedad de intercambio,
    hay \(t\) con \(1 \le t \le s\)
    tal que \(b_t \in \beta \smallsetminus \alpha\)
    y \(\alpha \cup \{ b_t \} \in \mathscr{I}\).
    Pero \(w(b_t) \ge w(b_s) > w(a_s)\),
    y nuestro algoritmo hubiese preferido \(b_t\) a \(a_s\).

    Al revés,
    si el algoritmo voraz siempre entrega un óptimo
    entonces \((\mathscr{E}, \mathscr{I})\) es un matroide.
    Para esto basta demostrar que el algoritmo voraz
    puede no entregar un óptimo
    si \((\mathscr{E}, \mathscr{I})\) no es un matroide.
    Si \((\mathscr{E}, \mathscr{I})\) no es un matroide,
    entonces:
    \begin{equation*}
      \exists \mathscr{A}, \mathscr{B} \in \mathscr{I},
        (\lvert \mathscr{A} \rvert < \lvert\mathscr{B} \rvert)
          \wedge
            (\not\exists e \in \mathscr{B} \smallsetminus \mathscr{A}
               \text{\ tal que\ } \mathscr{A} \cup \{ e \} \in \mathscr{I})
    \end{equation*}
    Sean \(m = \lvert \mathscr{A} \rvert\)
    y \(n = \lvert \mathscr{E} \rvert\).
    Defina:
    \begin{equation*}
      w(e)
        = \begin{cases}
             m + 2	 & e \in \mathscr{A} \\
             m + 1	 & e \in \mathscr{B} \smallsetminus \mathscr{A} \\
             1 / (2 n)	 & \text{caso contrario}
          \end{cases}
    \end{equation*}
    El algoritmo voraz retorna \(\mathscr{A}\),
    con peso a lo más \(m (m + 2) + 1 / 2 = m^2 + 2 m + 1 / 2\);
    una solución mejor es \(\mathscr{B}\)
    con peso al menos \((m + 1)^2 = m^2 + 2 m + 1\).
  \end{proof}
  Otra propiedad interesante resulta de lo siguiente:
  \begin{definition}
    Un sistema de subconjuntos \((\mathscr{E}, \mathscr{I})\)
    tiene la \emph{propiedad de cardinalidad}
    si:
    \begin{equation}
      \label{eq:propiedad-cardinalidad}
      \forall \mathscr{E}' \subseteq \mathscr{E},
         (A, B \in \mathscr{I}
            \text{\ subconjuntos maximales de \(\mathscr{E'}\)})
           \implies ( \lvert A \rvert = \lvert B \rvert )
    \end{equation}
  \end{definition}
  Decimos que \(A \in \mathscr{I}\)
  es \emph{subconjunto maximal} de \(\mathscr{E}'\)
  si \(A \subseteq \mathscr{E}'\) y no hay \(a \in \mathscr{E}'\)
  tal que \(A \cup \{ a \} \in \mathscr{I}\).
  Con esto tenemos:
  \begin{theorem}[Propiedad de cardinalidad]
    Sea un sistema de subconjuntos \((\mathscr{E}, \mathscr{I})\).
    Entonces \((\mathscr{E}, \mathscr{I})\) es un matroide
    si y solo si cumple la propiedad de cardinalidad.
  \end{theorem}
  \begin{proof}
    Es un si y solo si,
    demostramos implicancia en ambas direcciones.

    Sea \(\mathscr{E}\) un matroide,
    \(A, B\) subconjuntos maximales de \(\mathscr{E}' \subseteq \mathscr{E}\).
    Debemos demostrar \(\lvert A \rvert = \lvert B \rvert\),
    cosa que haremos por contradicción.
    Supongamos \(\lvert A \rvert < \lvert B \rvert\),
    por la propiedad de intercambio:
    \begin{equation*}
      \exists e \in B \smallsetminus A,
        (A \cup \{ e \} \in \mathscr{I})
    \end{equation*}
    Note que \(A \cup \{ e \} \in \mathscr{E}'\)
    ya que \(e \in B \subseteq \mathscr{E}'\),
    o sea,
    \(A\) no sería maximal.
    El caso \(\lvert A \rvert > \lvert B \rvert\) es simétrico.

    Al revés,
    debemos demostrar que
    si \((\mathscr{E}, \mathscr{I})\) no es matroide
    entonces hay \(\mathscr{E}'\) y \(A, B \in \mathscr{I}\)
    con \(A, B\) maximales en \(\mathscr{E}'\)
    con \(\lvert A \rvert \ne \lvert B \rvert\).
    Si \((\mathscr{E}, \mathscr{I})\) no es matroide:
    \begin{equation*}
      \exists A, C \in \mathscr{I},
        \lvert A \rvert < \lvert C \rvert \wedge
          \not\exists e \in C \smallsetminus A
          \text{\ con\ } A \cup \{ e \} \in \mathscr{I}
    \end{equation*}
    Defina \(\mathscr{E}' = A \cup C\),
    note que \(A\) es maximal en \(\mathscr{E}'\).
    Hay \(B \in \mathscr{I}\) tal que \(C \subseteq B\)
    y \(B\) es maximal en \(\mathscr{E}'\).
    Pero \(\lvert B \rvert \ge \lvert A \rvert + 1\),
    como debíamos demostrar.
  \end{proof}
  Tenemos dos propiedades diferentes
  (intercambio y cardinalidad)
  que describen los matroides.

  Note que nuestro primer ejemplo de algoritmo voraz
  (programar observaciones de ALMA)
  tiene un sistema de subconjuntos natural asociado:
  dos tareas son independientes si no traslapan
  (conjuntos independientes son los \textsc{Independent~Set}
   del grafo en el cual cada observación es un vértice,
   y dos vértices están unidos si traslapan).
  Esto \emph{no} es un matroide,
  pueden haber conjuntos independientes maximales de tamaños distintos.
  Nuestro algoritmo voraz entrega un óptimo por la elección
  de función de peso
  (todos iguales),
  para otras funciones de peso falla.

  Algunos ejemplos de matroides
  y los algoritmos voraces correspondientes:
  \begin{itemize}
  \item
    Los subconjuntos de un conjunto finito \(\mathscr{U}\)
    de cardinalidad a lo más \(k\).

    Podemos hallar el subconjunto más pesado usando el algoritmo voraz.
  \item
    Sea \(G = (V, E)\) un grafo,
    \(s \in V\) un vértice cualquiera.
    Los arcos de caminos desde \(s\) en el grafo son un matroide.
  \item
    Matroide de columnas.
    Sea \(\mathbf{A}\) una matriz,
    \(\mathscr{E}
        = \{ \mathbf{x}
               \colon \mathbf{x} \text{\ es columna de \(\mathbf{A}\)} \}\),
    y sea \(\mathscr{I}\) los conjuntos de columnas linealmente independientes.

    Podemos hallar la base más pesada para las columnas de \(\mathbf{A}\)
    usando el algoritmo voraz.
  \end{itemize}
  Nuestro último ejemplo de sistema de subconjuntos
  (arcos independientes en un grafo)
  no es un matroide,
  y el algoritmo voraz no siempre da un máximo.
  Considere el grafo de la figura~\ref{fig:edge-labelled-graph}.
  \begin{figure}
    \centering
    \tikzstyle{vertex} = [circle, fill = black!75, draw = black]
    \begin{tikzpicture}
       \node[vertex] at (0, {2 * sqrt(3)})		      (A)   {};
       \node[vertex] at (2, {2 * sqrt(3)})		      (B)   {};
       \node[vertex] at (4, {2 * sqrt(3)})		      (C)   {};

       \node[vertex] at (1, {sqrt(3)})		      (D)   {};
       \node[vertex] at (3, {sqrt(3)})		      (E)   {};

       \node[vertex] at (2, 0)			      (F)   {};

       \path (A) edge node[above] {\num{3}} (B)
             (A) edge node[above] {\num{2}} (D)
             (B) edge node[above] {\num{2}} (C)
             (B) edge node[above] {\num{2}} (E)
             (C) edge node[below] {\num{2}} (E)
             (D) edge node[above] {\num{3}} (E)
             (D) edge node[above] {\num{2}} (F)
             (E) edge node[below] {\num{2}} (F);
    \end{tikzpicture}
    \caption{Un grafo con arcos rotulados}
    \label{fig:edge-labelled-graph}
  \end{figure}
  Al buscar un conjunto de arcos de máximo peso
  que no comparten vértices,
  el algoritmo voraz da el conjunto maximal
  que consta de los arcos de peso~\num{3},
  para un total de~\num{6}
  (ver~\ref{fig:edge-labelled-graph-greedy});
  \begin{figure}
    \centering
    \tikzstyle{vertex} = [circle, fill = black!75, draw = black]
    \subfloat[Resultado del algoritmo voraz]{
      \begin{tikzpicture}
         \node[vertex] at (0, {2 * sqrt(3)})		      (A)   {};
         \node[vertex] at (2, {2 * sqrt(3)})		      (B)   {};
         \node[vertex] at (4, {2 * sqrt(3)})		      (C)   {};

         \node[vertex] at (1, {sqrt(3)})		      (D)   {};
         \node[vertex] at (3, {sqrt(3)})		      (E)   {};

         \node[vertex] at (2, 0)			      (F)   {};

         \draw[very thick] (A) edge node[above] {\num{3}}	      (B)
                           (D) edge node[above] {\num{3}}	      (E);

         \draw[thin, gray] (D) edge node[above] {\num{2}}	      (F)
                           (E) edge node[below] {\num{2}}	      (F)
                           (A) edge node[above] {\num{2}}	      (D)
                           (B) edge node[above] {\num{2}}	      (C)
                           (B) edge node[above] {\num{2}}	      (E)
                           (C) edge node[below] {\num{2}}	      (E);
      \end{tikzpicture}
    \label{fig:edge-labelled-graph-greedy}
  }
  \qquad
  \subfloat[Un óptimo]{
    \begin{tikzpicture}
       \node[vertex] at (0, {2 * sqrt(3)})		      (A)   {};
       \node[vertex] at (2, {2 * sqrt(3)})		      (B)   {};
       \node[vertex] at (4, {2 * sqrt(3)})		      (C)   {};

       \node[vertex] at (1, {sqrt(3)})		      (D)   {};
       \node[vertex] at (3, {sqrt(3)})		      (E)   {};

       \node[vertex] at (2, 0)			      (F)   {};

       \draw[very thick] (A) edge node[above] {\num{3}} (B)
                         (C) edge node[below] {\num{2}} (E)
                         (D) edge node[above] {\num{2}} (F);

       \draw[thin, gray] (A) edge node[above] {\num{2}} (D)
                         (B) edge node[above] {\num{2}} (C)
                         (B) edge node[above] {\num{2}} (E)
                         (D) edge node[above] {\num{3}} (E)
                         (E) edge node[below] {\num{2}} (F);
    \end{tikzpicture}
    \label{fig:edge-labelled-graph-maximum}
  }
  \end{figure}
  pero el máximo es~\num{7}
  (ver~\ref{fig:edge-labelled-graph-maximum}).

  La teoría de matroides nació de consideraciones
  sobre conjuntos linealmente independientes de vectores,
  y se vio aplicable a teoría de grafos.
  Rápidamente se extendió a otras áreas,
  un ejemplo es el texto de Lawler~%
    \cite{lawler76:_combinatorial_optimization_matroids}
  sobre optimización combinatoria y matroides.

\section{Programación con plazos fatales}
\label{sec:deadline-scheduling}

  Suponga que debe completar \(n\) tareas en \(n\) días,
  donde cada tarea demanda un día completo de dedicación.
  Cada tarea tiene un plazo fatal,
  si se completa después de su plazo fatal hay un costo a pagar.
  Se busca el orden en el cual ejecutar las tareas
  de forma de pagar el mínimo costo.

  Formalmente,
  numeramos las tareas de \num{1} a \(n\),
  dados un arreglo de plazos fatales \(D\)
  (un entero entre \num{1} y \(n\))
  y uno de penalizaciones \(P\)
  (números reales no negativos).
  Un programa \(\pi\) es una permutación de \(\{ 1, 2, \dotsc, n \}\).
  Buscamos un programa \(\pi\) que minimice:
  \begin{equation*}
    \sum_{1 \le i \le n} P[i] \cdot [ \pi(i) > D[i] ]
  \end{equation*}

  No parece para nada un ejemplo de optimización de matroide,
  allí solicitan un subconjunto y buscamos una permutación.
  Sorprendentemente,
  hay un matroide disfrazado en él.
  Para el programa \(\pi\),
  diga que las tareas para las cuales \(\pi(i) > D[i]\)
  están \emph{atrasadas},
  las demás \emph{a tiempo}.
  La observación trivial de que el costo de una programación
  queda determinada por sus tareas a tiempo lleva a revelar el matroide.

  Llame \emph{realista} a un conjunto de tareas \(X\)
  tal que hay una programación \(\pi\) en la que todas las tareas de \(X\)
  se completan a tiempo.
  Podemos caracterizar los subconjuntos realistas de la siguiente forma.
  Sea \(X(t)\) el conjunto de tareas en \(X\)
  con plazo fatal en o antes de \(t\):
  \begin{equation*}
    X(t)
      = \{ i \in X \colon D[i] \le t \}
  \end{equation*}
  En particular,
  \(X(0) = \varnothing\) y \(X(n) = X\).
  \begin{proposition}
    \label{prop:schedule-realistic}
    Sea \(X \subseteq \{ 1, 2, \dotsc, n \}\) un conjunto arbitrario de tareas.
    Entonces \(X\) es realista si y solo si \(\lvert X(t) \rvert \le t\)
    para todo \(t\).
  \end{proposition}
  \begin{proof}
    Es un si y solo si,
    demostramos implicancia en ambas direcciones.

    Sea \(\pi\) una programación
    en la que todas las tareas de \(X\) están a tiempo.
    Sea \(i_t\) la \(t\)\nobreakdash-ésima de \(X\) a completar.
    Por un lado,
    \(\pi(i_t) \ge t\),
    ya que completamos \(t - 1\) tareas antes;
    por el otro,
    \(\pi(i_t) \le D[i_t]\) ya que \(i_t\) está a tiempo.
    Concluimos \(D[i_t] \ge t\),
    por lo que \(\lvert X(t) \rvert \le t\).

    Suponga ahora que \(\lvert X(t) \rvert \le t\) para todo \(t\).
    Si ejecutamos las tareas de \(X\) en orden de plazo fatal,
    completamos las tareas con plazo fatal a más tardar \(t\)
    el día \(t\).
    Para todo \(i \in X\) estamos completando \(i\) antes de \(D[i]\),
    \(X\) es realista.
  \end{proof}
  Llamemos \emph{canónica} una programación para el conjunto de tareas \(X\)
  en la cual se ejecutan las tareas de \(X\) en orden de plazo fatal creciente,
  y las demás tareas en orden arbitrario.
  La proposición~\ref{prop:schedule-realistic}
  nos dice que \(X\) es realista
  si y solo si todas sus tareas se completan a tiempo
  en su programación canónica.
  O sea,
  nuestro problema se puede reformular como
  hallar un subconjunto realista \(X\) que maximice:
  \begin{equation*}
    \sum_{i \in X} P[i]
  \end{equation*}
  Estamos buscando un subconjunto óptimo.
  \begin{proposition}
    \label{prop:realistic=matroid}
    La colección de conjuntos realistas es un matroide.
  \end{proposition}
  \begin{proof}
    El conjunto vacío es realista
    (vacuamente),
    todo subconjunto de un conjunto realista es obviamente realista.
    Resta demostrar que se cumple la propiedad de intercambio.
    Sean entonces \(X\) e \(Y\) conjuntos realistas,
    con \(\lvert X \rvert > \lvert Y \rvert\).

    Sea \(t^*\) el máximo entero
    tal que \(\lvert X(t^*) \rvert \le \lvert Y(t^*) \rvert\).
    Debe existir,
    ya que \(\lvert X(0) \rvert = 0 \le 0 = \lvert Y(0) \rvert\)
    mientras
    \(\lvert X(n) \rvert
        = \lvert X \rvert
        > \lvert Y \rvert
        = \lvert Y(n) \rvert\).
    Por la definición de \(t^*\),
    hay más tareas con plazo fatal \(t^* + 1\) que \(t^*\)
    en \(X\) que en \(Y\).
    O sea,
    podemos elegir \(j \in X \smallsetminus Y\) con plazo fatal \(t^* + 1\).
    Llamemos \(Z = Y \cup \{ j \}\).

    Sea \(t\) arbitrario.
    Si \(t \le t^*\),
    entonces \(\lvert Z(t) \rvert = \lvert Y(t) \rvert \le t\),
    ya que \(Y\) es realista.
    Por otro lado,
    si \(t > t^*\),
    \(\lvert Z(t) \rvert = \lvert Y(t) \rvert + 1 \le \lvert X(t) \rvert\)
    por la definición de \(t^*\) y dado que \(X\) es realista.
    Por la proposición~\ref{prop:schedule-realistic},
    \(Z\) es realista.
    Se cumple la propiedad de intercambio.
  \end{proof}
  Por la proposición~\ref{prop:realistic=matroid},
  nuestro problema de hallar la programación óptima
  es una optimización de matroide,
  el algoritmo voraz~\ref{alg:greedy-schedule} da detalles.
  \begin{algorithm}
    \DontPrintSemicolon\Indp

    Sort \(P\) in decreasing order
      and sort \(D\) the same \;
    \(j \gets 0\) \;
    \For{\(i \gets 1\) \KwTo \(n\)}{
      \(X[j + 1] \gets i\) \;
      \If{\(X[1 .. j + 1]\) is realistic}{
        \(j \gets j + 1\) \;
      }
    }
    \Return the canonical program for \(X[1 .. j]\) \;

    \caption{Algoritmo voraz para programar tareas}
    \label{alg:greedy-schedule}
  \end{algorithm}
  Falta determinar si un conjunto de tareas es realista.
  La proposición~\ref{prop:schedule-realistic}
  da una pista cómo hacerlo,
  dando el algoritmo~\ref{alg:realistic-subset}.
  Esto supone que \(X\) viene ordenado por plazo fatal:
  \begin{equation*}
    i \le j
      \implies D[X[i]] \le D[X[j]]
  \end{equation*}
  \begin{algorithm}
    \DontPrintSemicolon\Indp

    \Function{\(\mathrm{realistic}(X, D)\)}{
      \(N \gets 0\) \;
      \(j \gets 0\) \;
      \For{\(t \gets 1\) \KwTo \(n\)}{
        \If{\(D[X[j]] = t\)}{
          \(N \gets N + 1; j \leftarrow j + 1\) \;
          \If{\(N > t\)}{
            \Return \(F\) \;
          }
        }
      }
      \Return \(T\) \;
    }
    \caption{Determinar si un conjunto es realista}
    \label{alg:realistic-subset}
  \end{algorithm}
  El resultado se ejecuta en tiempo \(O(n^2)\),
  usando estructuras de datos apropiadas esto se reduce a \(O(n \log n)\).
  Detalles quedan de ejercicio.


\section*{Ejercicios}
\label{sec:ejercicios-matroides}

  \begin{enumerate}
  \item
    Demuestre que los ejemplos de sistemas de subconjuntos citados
    realmente lo son.
    ¿Son matroides?
  \item
    Demuestre que el matroide gráfico \(\mathscr{M}(G)\) es un matroide
    para todo grafo \(G\).
    Describa sus bases y circuitos.
  \item
    Demuestre que para todo grafo \(G\)
    el matroide cográfico \(\mathscr{M}^*(G)\) es un matroide.
  \item
    Demuestre que para todo grafo \(G\) el matroide de correspondencias
    es un matroide.
    \\ \textbf{Pista:}
       ¿Qué es la diferencia simétrica de dos correspondencias?
  \item
    Indique qué entrega el algoritmo voraz en cada ejemplo de matroide citado.
  \item
    Sea \(G\) un grafo.
    Un conjunto de ciclos \(\{ c_1, c_2, \dotsc, c_k \}\) de \(G\)
    se llama \emph{redundante} si cada arco de \(G\)
    aparece en un número par de \(c_i\).
    Un conjunto de ciclos es \emph{independiente}
    si no contiene subconjuntos redundantes.
    Un conjunto maximal de ciclos independientes es una \emph{base de ciclos}
    de \(G\).
    \begin{enumerate}
    \item
      Sea \(C\) una base de ciclos de \(G\).
      Demuestre que para cada ciclo \(\gamma\) de \(G\),
      hay un subconjunto \(A \subseteq C\) tal que \(A \cap \{ \gamma \}\)
      es redundante.
      O sea,
      \(\gamma\) es el \textquote{o exclusivo} de los ciclos de \(A\).
    \item
      Demuestre que la colección de conjuntos de ciclos independientes
      es un matroide.
    \item
      Suponga ahora que cada arco de \(G\) tiene un peso,
      que el peso de un ciclo es la suma de los pesos de sus arcos,
      y que el peso de un conjunto de ciclos
      es la suma de los pesos de los ciclos
      (note que arcos que se repiten se cuentan cada vez que aparecen).
      Describa y analice un algoritmo eficiente
      para hallar una base de ciclos de mínimo peso.
      (Esto no es sencillo,
       no es inmediato obtener los ciclos de \(G\)).
    \end{enumerate}
  \item
    Demuestre que el sistema de subconjuntos
    del problema de programar observaciones de ALMA no es un matroide.
    Dé un ejemplo con una función de peso
    (retorno de la observación)
    tal que el algoritmo voraz no entregue una programación óptima.
  \item
    Demuestre cómo programar tareas con plazo fatal en tiempo \(O(n \log n)\).

    \textbf{Pista:} Use una estructura
      que permita determinar si \(X \cup \{ i \}\) es realista
      y agregar \(i\) a \(X\) en tiempo \(O(\log n)\) cada operación.
  \end{enumerate}

\bibliography{../referencias}

%%% Local Variables:
%%% mode: latex
%%% TeX-master: "../INF-221_notas"
%%% ispell-local-dictionary: "spanish"
%%% End:

% LocalWords:  english Independent Set subgrafo Clique matching Sort
% LocalWords:  the elements of by decreasin weight Forest MWF MST in
% LocalWords:  recubridor cográfico cociclo optimalidad vertex circle
% LocalWords:  fill black draw ésima reformular decreasing order and
% LocalWords:  sort same is realistic canonical program for
