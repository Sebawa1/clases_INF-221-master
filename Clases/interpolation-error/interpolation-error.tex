\bibliographystyle{babplain-fl}

\chapter{Error de Interpolación}
\label{cha:interpolacion-error}

  La clase pasada vimos cómo obtener la interpolación
  dado los pares de puntos \((x_k, f(x_k))\) con \(k \in \{0, \dotsc, n \}\)
  que nos daban.
  Nuestro tema de interés ahora es obtener
  el error de esas interpolaciones
  (figura \ref{05::ErrorEjemplo}).
  En particular,
  nos interesa ver cómo depende el error de los puntos elegidos.
  Usaremos algunos resultados del análisis real,
  refiérase a sus apuntes,
  al apéndice~\ref{apx:analysis}
  o a textos como el de Chen~%
    \cite{chen08:_fundam_analy}
  para demostraciones de los teoremas del caso.
  \begin{figure}[ht]
    \centering
    \begin{tikzpicture}
      \begin{axis}[
                    axis lines = center,
                    xlabel = {\(x\)},
                    ylabel = {\(y\)},
                    every axis y
                      label/.style = {at = (current axis.above origin),
                        anchor = south},
                    every axis x
                      label/.style = {at = (current axis.right of origin),
                        anchor = west},
                    xmin = 0,	 xmax = 3,
                    ymin = -0.6, ymax = 2.3,
                    xtick = {0},
                    xticklabels = {},
                    ytick = {0},
                    yticklabels = {},
                    height = 6cm, width = 9cm
                  ]

        \addplot [black!30!blue!90, samples = 100, domain = 0.15:2.4]
          {ln(2 * x + 0.8) + 0.2} node [xshift = 10] (f) {\(f(x)\)};
        \addplot [black!10!red!90, smooth, tension = 0.8]
          coordinates {(0.2, 0.2) (0.55, 1.3) (0.9, 0.6) (1.2, 1.7)
                       (1.5, 0.8) (1.9, 1.9)}
          node [yshift = 5pt, xshift = 4pt] (p) {\(Q_n(x)\)};

        \draw [fill = black!30] (axis cs:0.257, 0.47) circle (0.7mm);
        \draw [fill = black!30] (axis cs:0.72, 1) circle (0.7mm);
        \draw [fill = black!30] (axis cs:1.315, 1.43) circle (0.7mm);
        \draw [fill = black!30] (axis cs:1.85, 1.7) circle (0.7mm);
        \draw [fill = black!30] (axis cs:1.85, 1.7) circle (0.7mm);

        \draw [latex'-latex', thick, green!60!black!90]
          (axis cs:0.888, 1.15) -- (axis cs:0.888, 0.595);
        \draw [fill = black!30] (axis cs:0.888, 0.595) circle (0.7mm);
      \end{axis}
    \end{tikzpicture}
    \caption{El error es la diferencia
             entre \(Q_n(x)\) y \(f(x)\) en cada punto (flecha verde).}
    \label{05::ErrorEjemplo}
  \end{figure}
  \begin{theorem}
    \label{theo:interpolation-error}
    Sea \(f \in C^{n+1}[a, b]\)
    (vale decir,
     \(f\) tiene \(n + 1\) derivadas continuas en el intervalo).
    Sea \(Q_n(x)\) el polinomio de grado \(n\)
    que interpola \(f\)
    en los puntos distintos \(x_0, \dotsc, x_n \in[a, b]\).
    Entonces para todo \(x \in [a, b]\)
    hay \(\zeta \in [a, b]\) tal que
    \begin{equation}
      \label{05::Teo2}
      f(x) - Q_n(x)
        = \underbrace{\frac{1}{(n + 1)!} f^{(n+1)}(\zeta)
              \prod_{0 \le j \le n}(x - x_j)}_{\text{Error}}
    \end{equation}
  \end{theorem}
  \begin{proof}
    Fijemos \(x \in [a, b]\).
    Si \(x = x_j\) para algún \(j\),
    el lado izquierdo y derecho de~\eqref{05::Teo2} se anulan
    para todo \(\zeta\)
    y estamos listos.

    En caso contrario,
    sea
    \begin{equation}
      \omega(x)
        = \prod_{0\leq j\leq n}(x-x_j)
    \end{equation}
    Notamos para referencia futura
    que \(\omega(x)\) es mónico%
      \footnote{Para un polinomio de grado \(n\),
                el coeficiente de \(x^n\) es \num{1}.},
    con lo que \(\omega^{(n+1)}(x) = (n + 1)!\).

    Definamos:
    \begin{equation}
      F(y)
        = f(y)-Q_n(y) - \lambda \omega(y)
    \end{equation}
    donde elegimos \(\lambda\) tal que \(F(x) = 0\).

    La función \(F(y)\) está en \(C^{n+1}[a, b]\),
    y tiene \(n + 2\) ceros distintos en \([a, b]\),
    a saber \(x, x_0, \dotsc, x_n\).
    Por el teorema de Rolle extendido
    (teorema~\ref{theo:Rolle-extended})
    \(F^{(n + 1)}(y)\) tiene un cero en \([a, b]\),
    llamémosle \(\zeta\).
    O sea:
    \begin{align*}
      F^{(n + 1)}(\zeta)
        &= f^{(n + 1)}(\zeta)
             - \cancelto{\scriptstyle 0}{Q_n^{(n + 1)}(\zeta)}
             - \lambda
          \quad \cancelto{\scriptstyle(n + 1)!}{\omega^{(n + 1)}(\zeta)}
         = 0 \\
    \intertext{Vale decir:}
      f^{(n + 1)}(\zeta)
        &= \lambda(n + 1)! \\
    \intertext{Con esto:}
      \lambda
        &= \frac{f^{(n + 1)}(\zeta)}{(n + 1)!} \\
      F(y)
        &= f(y) - Q_n(y) - \frac{f^{(n + 1)}(\zeta)}{(n + 1)!} \omega(y)
    \end{align*}
    El error en el punto \(x\) es:
    \begin{equation*}
      f(x) - Q_n(x)
        = \frac{f^{(n + 1)}(\zeta)}{(n + 1)!} \omega(x)
    \end{equation*}
  \end{proof}
  Acá \(n\) lo elegimos nosotros
  y corresponde al grado de la interpolación.
  Es claro que mientras mayor sea el grado de nuestra interpolación,
  en general menor será el error
  (vea el \((n + 1)!\) como denominador).
  Claro que intervienen las características de la función
  y los puntos elegidos también.

\section{El fenómeno de Runge}
\label{sec:Runge-phenomenon}

  En 1901 Runge~%
    \cite{runge01:_ueber_aequidistante_interpolation}
  observó el fenómeno que lleva su nombre.
  Lo ilustramos con la función que el mismo usó de ejemplo:
  \begin{equation}
    \label{eq:Runge-function}
    f(x)
      = \frac{1}{1 + 25 x^2}
  \end{equation}
  Se grafica~\eqref{eq:Runge-function}
  en el rango \([-1, 1]\) en la figura~\ref{fig:Runge-function}.
  \begin{figure}[ht]
    \centering
    \pgfimage[width = 0.8\textwidth]{images/runge}
    \caption{La función de Runge}
    \label{fig:Runge-function}
  \end{figure}
  Se aprecia que parece ser bastante mansa,
  pero las figuras~\ref{fig:Runge-function-equiespaciados}
  muestran lo que ocurre al interpolar
  con 5, 7, 9 y~11 puntos igualmente espaciados
  entre \(-1\) y~\num{1}.
  \begin{figure}[ht]
    \centering
    \subfloat[\num{5} puntos]{
      \pgfimage[width = 0.475\textwidth]{images/runge-4}
      \label{fig:Runge-function-4}
    }
    \subfloat[\num{7} puntos]{
      \pgfimage[width = 0.475\textwidth]{images/runge-6}
      \label{fig:Runge-function-6}
    } \\
    \subfloat[\num{9} puntos]{
      \pgfimage[width = 0.475\textwidth]{images/runge-8}
      \label{fig:Runge-function-8}
    }
    \subfloat[\num{11} puntos]{
      \pgfimage[width = 0.475\textwidth]{images/runge-10}
      \label{fig:Runge-function-10}
    }
    \caption{Interpolando la función de Runge, puntos equiespaciados}
    \label{fig:Runge-function-equiespaciados}
  \end{figure}
  Se aprecia que alrededor de \num{0},
  el polinomio interpolante se acerca a la función,
  como esperábamos;
  sin embargo,
  en los extremos el error
  (la diferencia absoluta máxima entre la función y el polinomio)
  aumenta.
  La figura~\ref{fig:Ruge-error} muestra cómo evoluciona el error máximo
  con el número de puntos igualmente espaciados.
  \begin{figure}[ht]
    \centering
    \pgfimage[width = 0.8\textwidth]{images/runge-error}
    \caption{Error al interpolar la función de Runge}
    \label{fig:Ruge-error}
  \end{figure}
  Una explicación clara del fenómeno,
  aunque un tanto simplificada para efectos didácticos,
  es la de Epperson~%
    \cite{epperson87:_runge_example}.
  Resulta que depende de las singularidades de la función
  en el plano complejo,
  que no se aprecian en la línea real.

\section{Puntos de Chebyshev}
\label{sec:puntos-de-chebyshev}

  Nace entonces la pregunta de elegir los puntos
  de manera de evitar este comportamiento indeseable.
  Del teorema~\ref{theo:interpolation-error}
  el error de interpolación para los puntos distintos
  \(x_0, \dotsc, x_n \in[a, b]\)
  cumple:
  \begin{equation}
    \label{eq:interpolation-error}
          f(x) - Q_n(x)
        = \frac{1}{(n + 1)!} f^{(n+1)}(\zeta)
            \prod_{0 \le j \le n}(x - x_j)
  \end{equation}
  donde \(a < \zeta < b\).
  Debe destacarse que \(\zeta\)
  depende de los puntos \(x_0, \dotsc, x_n\) además de \(x\),
  minimizar el máximo valor absoluto de esta expresión
  en algún intervalo no es fácil.
  Nos contentaremos con minimizar el polinomio dado por la productoria.
  Para ello emplearemos una secuencia de polinomios especiales.

  Los \emph{polinomios de Chebyshev}
  se definen mediante:
  \begin{equation}
    \label{eq:Chebyshev-recurrence}
    \begin{split}
      T_0(x)
        &= 1 \\
      T_1(x)
        &= x \\
      T_{n + 1}(x)
        &= 2 x T_n(x) - T_{n - 1}(x)
          \quad n \ge 1
    \end{split}
  \end{equation}

  En vez de la recurrencia~\eqref{eq:Chebyshev-recurrence}
  es posible derivar una fórmula explícita para \(T_n(x)\).
  \begin{lemma}
    \label{lem:Chebyshev-explicit}
    Para \(x \in [-1, 1]\):
    \begin{equation}
      \label{eq:Chebyshev-explicit}
      T_n(x)
        = \cos(n \arccos x)
    \end{equation}
  \end{lemma}
  \begin{proof}
    La identidad para el coseno de suma de ángulos da:
    \begin{align*}
      \cos (n + 1) \theta
        &= \cos \theta \cos n \theta - \sin \theta \sin n \theta \\
      \cos (n - 1) \theta
        &= \cos \theta \cos n \theta + \sin \theta \sin n \theta
    \end{align*}
    de donde:
    \begin{equation*}
      \cos (n + 1) \theta
        = 2 \cos \theta \cos n \theta - \cos (n - 1) \theta
    \end{equation*}
    Sea \(\theta = \arccos x\),
    vale decir,
    \(x = \cos \theta\).
    Si definimos:
    \begin{equation*}
      t_n(x)
        = \cos (n \arccos x)
        = \cos n \theta
    \end{equation*}
    Resulta así:
    \begin{align*}
      t_0(x)
        &= 1 \\
      t_1(x)
        &= x \\
      t_{n + 1}(x)
        &= 2 x t_n(x) - t_{n - 1}(x)
          \quad n \ge 1
    \end{align*}
    Esta es exactamente la recurrencia~\eqref{eq:Chebyshev-recurrence}.
  \end{proof}
  De la ecuación~\eqref{eq:Chebyshev-recurrence}
  vemos que:
  \begin{equation*}
    T_n(x)
      = 2^{n - 1} x^n + \dotsb
  \end{equation*}
  Conocemos los ceros de \(T_n(x)\) por el lema~\ref{lem:Chebyshev-explicit}:
  \begin{equation*}
    x_k
      = \cos \frac{(2 k - 1) \pi}{2 n}, k = 1, \dotsc, n
  \end{equation*}
  así que:
  \begin{equation*}
    T_n(x)
      = 2^{n - 1} \prod_{1 \le k \le n} (x - x_k)
  \end{equation*}

  ¿Qué tienen de especial estos polinomios?
  Veremos primero un resultado general sobre polinomios mónicos.
  \begin{theorem}
    \label{theo:maxima-monic-polynomial}
    Si \(p_n(x)\) es un polinomio mónico de grado \(n\),
    entonces:
    \begin{equation}
      \label{eq:maxima-monic-polynomial}
      \max_{-1 \le x \le 1} \lvert p_n(x) \rvert
        \ge 2^{1 - n}
    \end{equation}
  \end{theorem}
  \begin{proof}
    Demostramos~\eqref{eq:maxima-monic-polynomial} por contradicción.
    Supongamos que para el polinomio mónico \(p_n\) de grado \(n\)
    y para todo \(\lvert x \rvert \le 1\) siempre es:
    \begin{equation*}
      \lvert p_n(x) \rvert
        < 2^{1 - n}
    \end{equation*}
    Sea
    \begin{equation*}
      q_n(x)
        = 2^{1 - n} T_n(x)
    \end{equation*}
    con lo que \(q_n\) es mónico,
    y sean \(u_j\) los siguientes \(n + 1\) puntos:
    \begin{equation*}
      u_j
        = \cos \frac{j \pi}{n}
          \quad 0 \le j \le n
    \end{equation*}
    Como por el lema~\ref{lem:Chebyshev-explicit}:
    \begin{align*}
      T_n \left( \cos \frac{j \pi}{n} \right)
        &= \cos \left( n \arccos \left( \cos \frac{j \pi}{n} \right) \right) \\
        &= \cos \left( n \cdot \frac{j \pi}{n} \right) \\
        &= \cos j \pi \\
        &= (-1)^j
    \end{align*}
    tenemos que:
    \begin{equation*}
      (-1)^j q_n(u_j)
        = 2^{1 - n}
    \end{equation*}
    Por lo tanto:
    \begin{equation*}
      (-1)^j p_n(u_j)
        \le \lvert p_n(u_j) \rvert
        < 2^{1 - n}
        = (-1)^j q_n(u_j)
    \end{equation*}
    Concluimos que:
    \begin{equation*}
      (-1)^j (q_n(u_j) - p_n(u_j)) > 0
        \quad 0 \le j \le n
    \end{equation*}
    lo que significa que el polinomio \(q_n(x) - p_n(x)\)
    cambia de signo \(n + 1\) veces en el intervalo.
    Vale decir,
    tiene al menos \(n\) ceros en el intervalo.
    Pero \(p_n\) y \(q_n\) son mónicos,
    por lo que \(q_n(x) - p_n(x)\) tiene grado a lo más \(n - 1\),
    y no puede tener más de \(n - 1\) ceros
    a menos que sea el polinomio cero,
    o sea,
    \(p_n = q_n\).
    Como el máximo de \(p_n\) es menor al máximo de \(q_n\),
    son polinomios distintos y esto es imposible.
  \end{proof}
  Volvamos a nuestro problema de interpolación.
  Queremos hallar los puntos \(x_j\) que minimicen:
  \begin{equation*}
    \max_{-1 \le x \le 1}
      \left\lvert \prod_{0 \le j \le n} (x - x_j) \right\rvert
  \end{equation*}
  Notamos que este polinomio es mónico de grado \(n + 1\),
  por lo que por el teorema~\ref{theo:maxima-monic-polynomial}:
  \begin{equation*}
    \max_{-1 \le x \le 1}
      \left\lvert \prod_{0 \le j \le n} (x - x_j) \right\rvert
      \ge 2^{- n}
  \end{equation*}
  y podemos obtener el mínimo de esto
  si el polinomio es \(2^{-n} T_{n + 1}(x)\),
  o sea,
  los \(x_j\) son los ceros del polinomio de Chebyshev \(T_{n + 1}(x)\).
  Por el lema~\ref{lem:Chebyshev-explicit}
  vemos que:
  \begin{equation}
    \label{eq:Chebyshev-points}
    x_j
      = \cos \frac{2 j + 1}{2 n + 2} \pi
        \quad 0 \le j \le n
  \end{equation}
  Para estos puntos nuestra cota~\eqref{05::Teo2} para el error
  da:
  \begin{align}
    \lvert f(x) - p_n(x) \rvert
      &=   \left\lvert
             2^{-n} T_{n + 1}(x) \frac{f^{(n + 1)}(\zeta(x))}{(n + 1)!}
           \right\rvert \notag \\
      &\le \frac{B_{n + 1}}{2^n (n + 1)!}
             \label{eq.Chebyshev-error}
  \end{align}
  donde:
  \begin{equation}
    \label{eq:Chebyshev-error-coeff}
    B_{n + 1}
      = \max_{x \in [-1, 1]} \lvert f^{(n + 1)}(x) \rvert
  \end{equation}
  Es claro que mediante una transformación lineal
  podemos cambiar el rango de integración.

  Si repetimos el ejercicio de interpolación anterior,
  pero ahora interpolando la función de Runge en los puntos de Chebyshev
  resultan las figuras~\ref{fig:Chebyshev-Runge-function}.
  \begin{figure}[ht]
    \centering
    \subfloat[\num{5} puntos]{
      \pgfimage[width = 0.475\textwidth]{images/chebyshev-4}
      \label{fig:Chebyshev-Runge-function-4}
    }
    \subfloat[\num{7} puntos]{
      \pgfimage[width = 0.475\textwidth]{images/chebyshev-6}
      \label{fig:Chebyshev-Runge-function-6}
    } \\
    \subfloat[\num{9} puntos]{
      \pgfimage[width = 0.475\textwidth]{images/chebyshev-8}
      \label{fig:Chebyshev-Runge-function-8}
    }
    \subfloat[\num{11} puntos]{
      \pgfimage[width = 0.475\textwidth]{images/chebyshev-10}
      \label{fig:Chebyshev-Runge-function-10}
    }
    \caption{Interpolando la función de Runge, puntos de Chebyshev}
    \label{fig:Chebyshev-Runge-function}
  \end{figure}
  La figura~\ref{fig:Chebyshev-error} muestra cómo evoluciona el error máximo
  con el número de puntos de Chebyshev.
  Se ve que las máximas magnitudes de los errores son bastante parejos
  (cabe esperar que lo sean,
   por la forma sinusoidal de \(T_n(x)\) en nuestro rango,
   y siendo el error una función
   --que esperamos varíe lentamente con \(x\)--
   multiplicada por un polinomio de Chebyshev).
  \begin{figure}[ht]
    \centering
    \pgfimage[width = 0.8\textwidth]{images/chebyshev-error}
    \caption{Error al interpolar la función de Runge
             en puntos de Chebyshev}
    \label{fig:Chebyshev-error}
  \end{figure}

\bibliography{../referencias}

%%% Local Variables:
%%% mode: latex
%%% TeX-master: "../INF-221_notas"
%%% ispell-local-dictionary: "spanish"
%%% End:

% LocalWords:  grafica eq function width images runge equiespaciados
% LocalWords:  recurrence maxima monic polynomial
