\bibliographystyle{babplain-fl}

\setchessboard{
  showmover = false,
  label = false
}

\chapter{Backtracking}
\label{cha:backtracking}

  Una idea al resolver problemas complejos
  es ir construyendo la solución incrementalmente,
  explorando distintas ramas
  y volviendo atrás
  (\emph{\foreignlanguage{english}{backtrack}} en inglés)
  si resulta que un camino es sin salida.
  Es aplicable a problemas de búsqueda
  (estamos buscando un objeto con ciertas características,
   debemos contabilizar cuántos hay,
   nos interesa el \textquote{mejor} de ellos).
  En términos generales,
  buscamos construir el objeto buscado por etapas,
  en cada etapa interesa eliminar de consideraciones futuras
  objetos parcialmente construidos que no pueden completarse a lo buscado.
  La manera natural de organizar esto es mediante una estructura global,
  que representa un objeto parcialmente construido,
  en la cual se registra el trabajo de construcción.
  Generalmente habrán varias opciones de paso siguiente,
  intentamos cada una de ellas por turno,
  deshaciendo el cambio si no lleva al destino
  (o debemos explorarlas todas porque interesa hallarlos todos).
  Es básicamente un recorrido parcial
  de un árbol de objetos construidos a medias,
  en el cual tenemos un único objeto parcial activo.
  Esto corresponde a búsqueda en profundidad,
  efectuada naturalmente mediante recursión
  (aunque podríamos usar búsqueda en profundidad no recursiva
   de ser necesario).

  El esquema general es el dado por el algoritmo~\ref{alg:backtrack-schema}.
  \begin{algorithm}
    \DontPrintSemicolon

    \tcc*[l]{Set up basic object} \;

    \Function{\(\mathrm{backtrack}()\)}{
      \eIf{\(\mathrm{object}\) is complete}{
        Process \(\mathrm{object}\) \;
      }
      {
        \(S \gets \text{set of possible next steps}\) \;
        \ForEach{\(s \in S\)}{
          Do step \(s\) on \(\mathrm{object}\) \;
          \(\mathrm{backtrack}()\) \;
          Undo step \(s\) on \(\mathrm{object}\) \;
        }
      }
    }
    \caption{Esquema básico de \emph{\foreignlanguage{english}{backtracking}}}
    \label{alg:backtrack-schema}
  \end{algorithm}

  Resolver un problema usando \emph{\foreignlanguage{english}{backtracking}}
  significa efectuar los siguientes pasos:
  \begin{enumerate}[font = \textbf, label = {(\alph*)}]
  \item
    Definir los pasos de la construcción del objeto buscado.
  \item
    Diseñar la estructura de datos
    que representa objetos parcialmente construidos.
  \item
    Definir cómo determinar si se ha llegado a destino.
  \item
    Diseñar posibles estructuras auxiliares
    que ayuden a determinar pasos siguientes.
  \item
    Definir cómo hacer y deshacer modificaciones.
  \item
    Definir criterios que permitan abortar la búsqueda
    porque el objeto parcial no puede completarse a una solución.
    Esto puede requerir estructuras globales auxiliares.
    Esto es parte clave de la determinación de los pasos candidatos.
  \item
    Definir qué hacer con objetos completos.
    La opción de terminar la búsqueda es poco natural
    en la versión recursiva,
    considere basar su programa en búsqueda en profundidad no recursiva
    si no se requiere exploración completa.
  \end{enumerate}
  El esbozo anterior supone que todos los datos son globales.
  Puede tener sentido que la función \(\mathrm{backtrack}()\)
  tome como argumento el número del paso a tomar,
  o incluso la lista de decisiones que llevan a la situación presente.

  El esbozo no puede explicitar el orden en que se intentan pasos candidatos,
  Una elaboración obvia si solo interesa hallar un objeto
  es intentar los candidatos
  en algún orden que prometa hallar pronto un destino.

  \begin{ejemplo}[Un clásico]
    En el ajedrez la reina es la pieza más poderosa.
    Amenaza los casilleros en su fila y columna,
    y los ubicados en diagonal.
    La figura~\ref{fig:reina-amenaza} muestra los casilleros
    que amenaza una reina en el ajedrez.
    \begin{figure}[ht]
      \centering
      \setchessboard{setpieces = {Qc5}}
      \chessboard[pgfstyle = {[fill]circle},
                  padding = -1ex,
                  backfields  = {a5, b5, d5, e5, f5, g5, h5,
                                 c1, c2, c3, c4, c6, c7, c8,
                                 a3, b4, d6, e7, f8,
                                 a7, b6, d4, e3, f2, g1}
                 ]
      \caption{Los casilleros amenazados por una reina}
      \label{fig:reina-amenaza}
    \end{figure}
    Un problema clásico
    (propuesto por Max Bezzel en 1848)
    es determinar
    si se pueden ubicar ocho reinas en el tablero
    de manera que ninguna pueda amenazar a otra.
    Claramente no pueden ser más de ocho,
    puede haber a lo más una reina por columna.
    Resolver este problema
    mediante \emph{\foreignlanguage{english}{backtracking}}
    fue un ejemplo de programación estructurada de Dijkstra~%
      \cite{dijkstra72:_structured_programming}.

    Siguiendo los pasos:
    \begin{enumerate}[font = \textbf, label = {(\alph*)}]
    \item
      \textbf{Pasos de la construcción:}
      Agregamos reinas una a una al tablero.
      Un orden simple útil
      (porque elimina posiciones imposibles)
      es agregar reinas por columnas en orden.
    \item
      \textbf{Objetos parciales:}
      Basta registrar la fila en que se ubica la reina de cada columna
    \item
      \textbf{Estamos en el destino:}
      Ubicamos la reina en la última columna.
    \item
      \textbf{Estructuras auxiliares:}
      Es útil registrar qué filas y diagonales están libres.
    \item
      \textbf{Hacer/deshacer:}
      A cada columna le corresponde la posición de la reina,
      hacer y deshacer es actualizar esto.
      Se deben marcar libres filas y diagonales de la reina que se remueve,
      y marcar ocupadas las de la reina que se ubica.
    \item
      \textbf{Criterios para abortar:}
      No intentamos posicionar reinas en la columna en filas ya ocupadas.
    \item
      \textbf{Objetos completos:}
      Solo los contabilizamos
      o escribimos la solución,
      según sea el caso.
    \end{enumerate}
    En este caso:
    \begin{itemize}
    \item
      Ubicar reina en columna 1, 2, \ldots
    \item
      Registrar filas libres
      (para omitir ocupadas al ubicar la siguiente reina)
    \item
      Registrar diagonales libres.

      Reina en \(r, c\):
      \begin{align*}
        y - c
          &= 1 \cdot (x - r) \\
        r - c
          &= x - y \text{\ es constante} \\
        y - c
          &= -1 \cdot (x - r) \\
        c - r
          &= x + y \text{\ es constante}
      \end{align*}
    \end{itemize}
    Por ejemplo,
    luego de ubicadas las primeras tres reinas
    en las filas \num{1}, \num{3} y \num{5} la configuración resultante
    es la de la figura~\ref{fig:tres-reinas}.
    \begin{figure}[ht]
      \centering
      \setchessboard{setpieces = {Qa1, Qb3, Qc5}}
      \chessboard[pgfstyle = {[fill]circle},
                  padding = -1ex,
                  backfields  = {a2, a3, a4, a5, a6, a7, a8,
                                   b1, c1, d1, e1, f1, g1, h1,
                                   b2, c3, d4, e5, f6, g7, h8,
                                 b1, b2, b4, b5, b6, b7, b8,
                                   a3, c3, d3, e3, f3, g3, h3,
                                   a2, c4, d5, e6, f7, g8,
                                   a4, c2, d1,
                                 c1, c2, c3, c4, c6, c7, c8,
                                    a5, b5, d5, e5, f5, g5, h5,
                                    a7, b6, d4, e3, f2, g1,
                                    a3, b4, d6, e7, f8
                                }
                 ]
      \caption{Configuración con tres reinas}
      \label{fig:tres-reinas}
    \end{figure}
    Vemos que las filas y diagonales amenazadas por estas tres
    restringen muchísimo las posiciones viables
    para la cuarta y siguientes.
    Con estas tres reinas,
    para la cuarta reina quedan solo \num{3} posibilidades.

    Elegimos Python (!),
    en Python los arreglos tienen índices desde \num{0},
    rango de las variables \(r\), \(c\) es de \num{0} a~\num{7}.
    Interesan los rangos de las expresiones:
    \begin{description}
    \item[\boldmath\(r - c\)\unboldmath:]
      Rango es \(-7, \dotsc, 7\),
      sumar \num{7} para llevar al rango \(0, \dotsc, 14\).
    \item[\boldmath\(r + c\)\unboldmath:]
      Rango es \(0, \dotsc, 14\)
    \end{description}
    El programa final es el del listado~\ref{lst:8queens}.
    \lstinputlisting[float,
                     language = Python,
                     firstline = 3,
                     caption = {Ocho reinas en Python},
                     label = lst:8queens]
                    {code/8queens}
    La figura~\ref{fig:8reinas} muestra una de las \num{92}~soluciones.
    \begin{figure}[ht]
      \centering
      \setchessboard{setpieces = {Qa1, Qb5, Qc8, Qd6,
                                  Qe3, Qf7, Qg2, Qh4}}
      \chessboard
      \caption{Una solución para el problema de 8 reinas.}
      \label{fig:8reinas}
    \end{figure}
  \end{ejemplo}

\section{El algoritmo de Warnsdorf}
\label{sec:Warnsdorf-algo}

  Un problema también relacionado al ajedrez
  es determinar si un caballo puede visitar cada casillero del tablero
  exactamente una vez.
  La figura~\ref{fig:knight-moves} muestra las posibles movidas de un caballo.
  \begin{figure}[ht]
    \centering
    \setchessboard{setpieces = {Nd5}}
    \chessboard[pgfstyle = {[fill]circle},
                padding = -1ex,
                backfields  = {b6, c7, e7, f6, f4, e3, c3, b4}
               ]
    \caption{Movidas posibles de un caballo}
    \label{fig:knight-moves}
  \end{figure}
  Warnsdorf~%
    \cite{warnsdorf23:_roesselsprung_loesung}
  halló un algoritmo simple para resolver este problema
  sin tener que reconsiderar movidas:
  en cada paso elija la posición siguiente como aquella
  que da menos movidas posibles a continuación.

  En forma abstracta,
  este problema corresponde a determinar
  si el grafo con vértices los casilleros
  y arcos los posibles saltos entre ellos
  tiene un camino o ciclo hamiltonianos
  (que visitan cada vértice exactamente una vez).
  En el caso de grafos generales esto no evita reconsideraciones,
  pero resulta un criterio eficiente para hallar soluciones.
  La idea de considerar a continuación
  la configuración con menos vecinos es una heurística útil en muchos casos.

\section{Sudoku}
\label{sec:sudoku}

  En Sudoku se plantea una grilla de \(9 \times 9\) casillas
  a ser llenadas con los dígitos \num{1} a \num{9}
  de forma que cada fila, columna y subcuadrado de \(3 \times 3\)
  contenga cada dígito exactamente una vez.
  Un ejemplo de Sudoku
  es el dado en la figura~\ref{fig:Sudoku-dificil}.
  \begin{figure}[ht]
    \centering
    \subfloat[Problema]{
      \begin{tabular}{|*{3}{c@{\;\,}c@{\;\,}c|}}
        \hline
           &   &   &
           &   &   &
           & 1 & 2 \\

           &   &   &
           & 3 & 5 &
           &   &   \\

           &   &   &
         6 &   &   &
           & 7 &   \\
        \hline
         7 &   &   &
           &   &   &
         3 &   &   \\

           &   &   &
         4 &   &   &
         8 &   &   \\

         1 &   &   &
           &   &   &
           &   &   \\
        \hline
           &   &   &
         1 & 2 &   &
           &   &   \\

           & 8 &   &
           &   &   &
           & 4 &   \\

           & 5 &   &
           &   &   &
         6 &   &   \\
        \hline
      \end{tabular}
    }
    \hspace*{3em}
    \subfloat[Solución]{
      \begin{tabular}{|*{3}{c@{\;\,}c@{\;\,}c|}}
        \hline
         6 & 7 & 3 &
         8 & 9 & 4 &
         5 & 1 & 2 \\

         9 & 1 & 2 &
         7 & 3 & 5 &
         4 & 8 & 6 \\

         8 & 4 & 5 &
         6 & 1 & 2 &
         9 & 7 & 3 \\
        \hline
         7 & 9 & 8 &
         2 & 6 & 1 &
         3 & 5 & 4 \\

         5 & 2 & 6 &
         4 & 7 & 3 &
         8 & 9 & 1 \\

         1 & 3 & 4 &
         5 & 8 & 9 &
         2 & 6 & 7 \\
        \hline
         4 & 6 & 9 &
         1 & 2 & 8 &
         7 & 3 & 5 \\

         2 & 8 & 7 &
         3 & 5 & 6 &
         1 & 4 & 9 \\

         3 & 5 & 1 &
         9 & 4 & 7 &
         6 & 2 & 8 \\
        \hline
        \end{tabular}
    }
    \caption{Sudoku muy difícil}
    \label{fig:Sudoku-dificil}
  \end{figure}
  Estrategias para elegir siguiente casillero a llenar:
  \begin{itemize}
  \item
    Al azar/primero libre.
  \item
    Más restringido.
  \end{itemize}
  Estrategias para podar:
  \begin{itemize}
  \item
    Cuenta local
    (revisar si quedan opciones fila/columna/cuadrante)
  \item
    \emph{\foreignlanguage{english}{Look ahead}}
    (revisar si quedan casilleros sin opciones)
  \end{itemize}
  Norvig~%
    \cite{norvig:_sudoku}
  discute algunas estrategias y da un programa que resuelve Sudoku,
  reportando resultados.
  Skiena~%
    \cite{skiena08:_algor_desig_manual}
  reporta resultados del cuadro~\ref{tab:backtracking-Sudoku}
  para distintos problemas.
  \begin{table}[ht]
    \centering
    \begin{tabular}{l|l|r|r|r}
      \multicolumn{1}{c|}{\textbf{Combinaciones}}
         & \textbf{Criterio de Poda}
         & \multicolumn{1}{c|}{\textbf{Simple}}
         & \multicolumn{1}{c|}{\textbf{Mediano}}
         & \multicolumn{1}{c}{\textbf{Difícil}} \\
      \hline
      Azar & Local & 1\,904\,832 & 863\,305 & No \ldots\\
      Azar & Look ahead & 127 & 142 & 12\,507\,212\\
      Restringida & Local & 48 & 84 & 1\,243\,838\\
      Restringida & Look ahead & 48 & 65 & 10\,374
    \end{tabular}
    \caption{Rendimiento de variantes
             de \emph{\foreignlanguage{english}{backtracking}} en Sudoku}
    \label{tab:backtracking-Sudoku}
  \end{table}
  El simple es de los que típicamente se dan para principiantes,
  el mediano se planteó en un campeonato
  (y ninguno de los participantes pudo resolverlo),
  el difícil es el de la figura~\ref{fig:Sudoku-dificil}
  (tiene 17 pistas,
   es de los con menos pistas que tiene una única solución).
  El programa que obtiene estos valores es~%
    \cite{skiena06:_backtr_progr_solve_sudoku}.
  Las estrategias aplicadas por Skiena no son las únicas posibles,
  todo jugador serio aplica un menú de estrategias adicionales,
  como las que discute Norvig~%
    \cite{norvig:_sudoku}.
  Un buen ejemplo es el texto de Zambon~%
    \cite{zambon15:_sudoku_programming_c},
  quien desarrolla programas alrededor de Sudoku,
  incluyendo una gran colección de estrategias,
  usando \emph{\foreignlanguage{english}{backtracking}}
  solo como último recurso.
  En 2012 McGuire, Tugemann y Civario~%
    \cite{mcguire12:_no_16-clue_sudoku}
  demostraron mediante una búsqueda exhaustiva
  que tomó \num{7,1}~millones de horas-núcleo en un supercomputador
  que no hay problemas con solo \num{16}~pistas con solución única.

\section*{Ejercicios}
\label{sec:ejercicios-backtracking}

  \begin{enumerate}
  \item
    Podemos generalizar el problema de las 8~reinas en forma obvia
    al problema de \(n\)~reinas.
    Nuestra estrategia bastante ingenua funciona bien en el caso \(n = 8\),
    pero rápidamente se meterá en problemas si \(n\) es mayor.
    En particular,
    si solo interesa determinar si hay o no solución,
    basta revisar una parte de las opciones.
    Experimente con otras estrategias,
    como elegir la columna que tiene menos filas libres,
    definir un criterio para elegir dentro de la columna
    de manera de restringir al máximo hacia adelante
    (considerar por ejemplo las columnas más cercanas aún libres).
  \item
    Una \emph{rotulación graciosa} de un grafo \(G = (V, E)\)
    con \(n\) vértices asigna los rótulos \num{1} a \(n\) a los vértices,
    tal que al rotular los arcos
    con el valor absoluto de la diferencia entre los rótulos de sus vértices
    los arcos están rotulados con los números de \num{1} a \(n - 1\),
    cada uno exactamente una vez.
    Es claro que esto solo se puede hacer en árboles.

    Escriba un programa para determinar
    cuántas rotulaciones graciosas tiene el grafo \(P_n\).
  \item
    Determine si es posible que un caballo del ajedrez
    visite exactamente una vez cada una de las casillas del tablero.
  \item
    Nuestra solución al problema de 8~reinas
    considera las columnas estrictamente en orden.
    Una variante es considerar a continuación
    la columna con menos casilleros libres.
    Experimente con esta variante,
    contabilizando el número de posiciones de reinas que se intentan
    y también el tiempo total de solución.
  \end{enumerate}

\bibliography{../referencias}

%%% Local Variables:
%%% mode: latex
%%% TeX-master: "../INF-221_notas"
%%% ispell-local-dictionary: "spanish"
%%% End:

% LocalWords:  showmover label Backtracking incrementalmente english
% LocalWords:  backtrack recursión setpieces Qc pgfstyle fill circle
% LocalWords:  padding backfields b f g h c Subproblemas Qa Qb Python
% LocalWords:  Qd Qe Qf Qg Qh backtracking ht subproblemas reusarse
% LocalWords:  subcuadrado Look ahead Zambon supercomputador false ex
% LocalWords:  rotulación rotulaciones Steven is solution Process Set
% LocalWords:  Register move Undo Sudoku up basic object step on Max
% LocalWords:  backtracling Nd hamiltonianos
