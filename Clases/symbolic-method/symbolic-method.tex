\bibliographystyle{babplain-fl}

\chapter{Symbolic Method for Dummies}
\label{apx:symbolic-method-dummies}

  La idea básica es usar funciones generatrices en forma sistemática
  en combinatoria.
  Lo que sigue es un condensado del apunte de Fundamentos de Informática~%
    \cite[capítulo~21]{brand17:_fundamentos_informatica}.
  Ver también Lumbroso y~Morcrette~%
    \cite{lumbroso12:_gentle_intro_analy_combin}.
  Mucho más detalle dan Flajolet y Sedgewick~%
    \cite{flajolet09:_analy_combin},
  Sedgewick y Flajolet~%
    \cite{sedgewick13:_introd_anal_algor}
  se concentran en aplicaciones al análisis de algoritmos.

  La idea es tener una \emph{clase} de objetos,
  que anotaremos mediante letras caligráficas,
  como \(\mathscr{A}\).
  La clase \(\mathscr{A}\) consta de \emph{objetos},
  \(\alpha \in \mathscr{A}\).
  Para el objeto \(\alpha\) hay una noción de \emph{tamaño},
  que anotamos \(\lvert \alpha \rvert\)
  (un número natural,
   generalmente el número de \emph{átomos} que componen \(\alpha\)).
  Al número de objetos de tamaño \(n\) lo anotaremos \(a_n\).
  Usaremos \(\mathscr{A}_n\)
  para referirnos al conjunto de objetos de la clase \(\mathscr{A}\)
  de tamaño \(n\),
  con lo que \(a_n = \lvert \mathscr{A}_n \rvert\).
  Condición adicional es
  que el número de objetos de cada tamaño sea finito.

  A las funciones generatrices ordinaria y exponencial
  correspondientes
  les llamaremos \(A(z)\) y \(\widehat{A}(z)\),
  respectivamente:
  \begin{align*}
    A(z)
      &= \sum_{\alpha \in \mathscr{A}} z^{\lvert \alpha \rvert}
       = \sum_{n \ge 0} a_n z^n \\
    \widehat{A}(z)
      &= \sum_{\alpha \in \mathscr{A}}
           \frac{z^{\lvert \alpha \rvert}}{\lvert \alpha \rvert !}
       = \sum_{n \ge 0} a_n \, \frac{z^n}{n!}
  \end{align*}

  Nuestro siguiente objetivo es construir nuevas clases
  a partir de las que ya tenemos.
  Debe tenerse presente que como lo que nos interesa
  es contar el número de objetos de cada tamaño,
  basta construir objetos con distribución de tamaños adecuada
  (o sea,
   relacionados con lo que deseamos contar por una biyección).
  Comúnmente el tamaño de los objetos es el número de alguna clase de átomos
  que lo componen.
  Al combinar objetos para crear objetos mayores
  los tamaños simplemente se suman.

  Las clases más elementales son \(\varnothing\),
  la clase que no contiene objetos;
  \(\mathscr{E} = \{\epsilon\}\),
  la clase que contiene únicamente el objeto vacío \(\epsilon\)
  (de tamaño nulo);
  y la clase que comúnmente llamaremos \(\mathscr{Z}\),
  conteniendo un único objeto de tamaño uno
  (que llamaremos \(\zeta\) por consistencia).
  Luego definimos operaciones que combinan
  las clases \(\mathscr{A}\) y \(\mathscr{B}\)
  mediante \emph{unión combinatoria} \(\mathscr{A} + \mathscr{B}\),
  en que aparecen los \(\alpha\) y los \(\beta\) con sus tamaños
  (los objetos individuales se \textquote{decoran} con su proveniencia,
   de forma que \(\mathscr{A}\) y  \(\mathscr{B}\)
   no necesitan ser disjuntos;
   pero generalmente nos preocuparemos
   que \(\mathscr{A}\) y \(\mathscr{B}\)
   sean disjuntos,
   o podemos usar el principio de inclusión y exclusión
   para contar los conjuntos de interés).
  Ocasionalmente restaremos objetos de una clase,
  lo que debe interpretarse sin decoraciones
  (estamos dejando fuera ciertos elementos,
   simplemente).
  Usaremos \emph{producto cartesiano}
  \(\mathscr{A} \times \mathscr{B}\),
  cuyos elementos son pares \((\alpha, \beta)\)
  y el tamaño del par
  es \(\lvert \alpha \rvert + \lvert \beta \rvert\).
  Otras operaciones son formar \emph{secuencias}
  de elementos de \(\mathscr{A}\)
  (se anota \(\Seq(\mathscr{A})\)),
  formar \emph{conjuntos}
  \(\Set(\mathscr{A})\)
  y \emph{multiconjuntos}
  \(\MSet(\mathscr{A})\)
  de elementos de \(\mathscr{A}\).

  De incluir objetos de tamaño cero
  en estas construcciones pueden crearse infinitos objetos de un tamaño dado,
  lo que no es una clase según nuestra definición.
  Por ello estas construcciones son aplicables
  solo si \(\mathscr{A}_0 = \varnothing\).

  Es importante recalcar las relaciones y diferencias
  entre las estructuras.
  En una secuencia es central
  el orden de las piezas que la componen.
  Ejemplo son las palabras,
  interesa el orden exacto de las letras
  (y estas pueden repetirse).
  En un conjunto solo interesa si el elemento está presente o no,
  no hay orden.
  En un conjunto un elemento en particular está o no presente,
  a un multiconjunto puede pertenecer varias veces.

  Hay dos grandes opciones:
  Objetos rotulados y no rotulados.
  Consideramos que el objeto \(\alpha\) es \emph{rotulado}
  si sus átomos componentes tienen identidad,
  cosa que se representa rotulándolos de \num{1} a \(\lvert \alpha \rvert\).
  Si los átomos son libremente intercambiables,
  son objetos \emph{no rotulados}.
  Un punto que produce particular confusión
  es que tiene perfecto sentido
  hablar de secuencias de elementos sin rotular.
  La secuencia impone un orden,
  pero elementos iguales se consideran indistinguibles
  (en una palabra interesa el orden de las letras,
   pero al intercambiar dos letras iguales
   la palabra sigue siendo la misma).
  Estos dos casos requieren tratamiento separado,
  y en algunos casos operaciones especializadas.

\section{Objetos no rotulados}
\label{sec:objetos-no-rotulados}

  Nuestro primer teorema
  relaciona las funciones generatrices ordinarias
  respectivas para algunas de las operaciones entre clases
  definidas antes.
  Las funciones generatrices de las clases \(\varnothing\),
  \(\mathscr{E}\) y \(\mathscr{Z}\)
  son,
  respectivamente,
  \num{0}, \num{1} y \(z\).
  En las derivaciones
  de las transferencias de ecuaciones simbólicas
  a ecuaciones para las funciones generatrices
  lo que nos interesa es contar los objetos entre manos,
  recurriremos a biyecciones para ello en algunos de los casos.
  \begin{theorem}[Método simbólico, OGF]
    \label{theo:ms-OGF}
    Sean \(\mathscr{A}\) y \(\mathscr{B}\) clases de objetos,
    con funciones generatrices ordinarias
    respectivamente \(A(z)\) y \(B(z)\).
    Entonces tenemos
    las siguientes funciones generatrices ordinarias:
    \begin{enumerate}
    \item
      Para enumerar \(\mathscr{A} + \mathscr{B}\):
      \begin{equation*}
        A(z) + B(z)
      \end{equation*}
    \item
      Para enumerar \(\mathscr{A} \times \mathscr{B}\):
      \begin{equation*}
        A(z) \cdot B(z)
      \end{equation*}
    \item
      Para enumerar \(\Seq(\mathscr{A})\):
      \begin{equation*}
        \frac{1}{1 - A(z)}
      \end{equation*}
    \item
      Para enumerar \(\Set(\mathscr{A})\):
      \begin{equation*}
        \prod_{\alpha \in \mathscr{A}}
           \left( 1 + z^{\lvert \alpha \rvert} \right)
          = \prod_{n \ge 1} (1 + z^n)^{a_n}
          = \exp \left(
                   \sum_{k \ge 1} \frac{(-1)^{k + 1}}{k} \, A(z^k)
                 \right)
      \end{equation*}
    \item
      Para enumerar \(\MSet(\mathscr{A})\):
      \begin{equation*}
        \prod_{\alpha \in \mathscr{A}}
           \frac{1}{1 - z^{\lvert \alpha \rvert}}
          = \prod_{n \ge 1} \frac{1}{(1 - z^n)^{a_n}}
          = \exp\left(
                   \sum_{k \ge 1} \frac{A(z^k)}{k}
                \right)
      \end{equation*}
    \item
      Para enumerar \(\Cyc(\mathscr{A})\):
      \begin{equation*}
        \sum_{n \ge 1} \frac{\phi(n)}{n} \, \ln \frac{1}{1 - A(z^n)}
      \end{equation*}
    \end{enumerate}
  \end{theorem}
  \begin{proof}
    Usamos libremente resultados sobre funciones generatrices,
    ver~%
      \cite[capítulo~14]{brand17:_fundamentos_informatica},
    en las demostraciones de cada caso.
    Usaremos casos ya demostrados en las demostraciones sucesivas.
    \begin{enumerate}
    \item % A + B
      Si hay \(a_n\) elementos de \(\mathscr{A}\) de tamaño \(n\)
      y \(b_n\) elementos de \(\mathscr{B}\) de tamaño \(n\),
      habrán \(a_n + b_n\) elementos
      de \(\mathscr{A} + \mathscr{B}\)
      de tamaño \(n\).

      Alternativamente,
      usando la notación de Iverson
      (ver~%
        \cite[sección~1.4]{brand17:_fundamentos_informatica}):
      \begin{equation*}
        \sum_{\mathclap{\gamma \in \mathscr{A} + \mathscr{B}}}
          z^{\lvert \gamma \rvert}
          = \; \sum_{\mathclap{\gamma \in \mathscr{A}
                                            + \mathscr{B}}}
                 \left(
                   [\gamma \in \mathscr{A}]
                     z^{\lvert \gamma \rvert}
                      + [\gamma \in \mathscr{B}]
                          z^{\lvert \gamma \rvert}
                 \right)
          = \sum_{\alpha \in \mathscr{A}} z^{\lvert \alpha \rvert}
              + \sum_{\beta \in \mathscr{B}}
                  z^{\lvert \beta \rvert}
          = A(z) + B(z)
      \end{equation*}
    \item % A x B
      Hay:
      \begin{equation*}
        \sum_{0 \le k \le n} a_k b_{n - k}
      \end{equation*}
      maneras de combinar elementos de \(\mathscr{A}\)
      con elementos de \(\mathscr{B}\) cuyos tamaños sumen \(n\),
      y este es precisamente
      el coeficiente de \(z^n\) en \(A(z) \cdot B(z)\).

      Alternativamente:
      \begin{align*}
        \sum_{\mathclap{\gamma \in \mathscr{A} \times \mathscr{B}}}
             z^{\lvert \gamma \rvert}
          = \sum_{\mathclap{\substack{
                              \alpha \in \mathscr{A} \\
                              \beta \in \mathscr{B}
                 }}} z^{\lvert (\alpha, \beta) \rvert}
          = \sum_{\mathclap{\substack{
                              \alpha \in \mathscr{A} \\
                              \beta \in \mathscr{B}
                 }}} z^{\lvert \alpha \rvert + \lvert \beta \rvert}
          = \left(
               \sum_{\alpha \in \mathscr{A}}
                 z^{\lvert \alpha \rvert}
             \right)
               \cdot \left(
                        \sum_{\beta \in \mathscr{B}}
                          z^{\lvert \beta \rvert}
                     \right)
          = A(z) \cdot B(z)
      \end{align*}
    \item % Seq(A)
      Hay una manera de obtener la secuencia de largo 0
      (aporta el objeto vacío \(\epsilon\)),
      las secuencias de largo \num{1}
      son simplemente los elementos de \(\mathscr{A}\),
      las secuencias de largo \num{2}
      son elementos de \(\mathscr{A} \times \mathscr{A}\),
      y así sucesivamente.
      O sea,
      las secuencias se representan mediante:
      \begin{equation*}
        \mathscr{E}
          + \mathscr{A}
          + \mathscr{A} \times \mathscr{A}
          + \mathscr{A} \times \mathscr{A} \times \mathscr{A}
          + \dotsb
      \end{equation*}
      Por la segunda parte
      y la serie geométrica,
      la función generatriz correspondiente es:
      \begin{equation*}
        1 + A(z) + A^2(z) + A^3(z) + \dotsb
          = \frac{1}{1 - A(z)}
      \end{equation*}
    \item % Set(A)
      La clase de los subconjuntos finitos de \(\mathscr{A}\)
      queda representada por el producto simbólico:
      \begin{equation*}
        \prod_{\alpha \in \mathscr{A}} (\mathscr{E} + \{\alpha\})
      \end{equation*}
      ya que
      al distribuir los productos de todas las formas posibles
      aparecen todos los subconjuntos de \(\mathscr{A}\).
      Directamente obtenemos entonces:
      \begin{equation*}
        \prod_{\alpha \in \mathscr{A}}
            \left( 1 + z^{\lvert \alpha \rvert} \right)
          = \prod_{n \ge 0} (1 + z^n)^{a_n}
      \end{equation*}
      Otra forma de verlo es que cada objeto de tamaño \(n\)
      aporta un factor \(1 + z^n\),
      si hay \(a_n\) de estos
      el aporte total es \((1 + z^n)^{a_n}\).
      Esta es la primera parte de lo aseverado.
      Aplicando logaritmo:
      \begin{align*}
        \sum_{\alpha \in \mathscr{A}}
            \ln \left(1 + z^{\lvert \alpha \rvert} \right)
          &= -\sum_{\alpha \in \mathscr{A}}
                \sum_{k \ge 1}
                  \frac{(-1)^k z^{\lvert \alpha \rvert k}}{k}  \\
          &= -\sum_{k \ge 1} \frac{(-1)^k}{k} \,
                \sum_{\alpha \in \mathscr{A}}
                  z^{\lvert \alpha \rvert k} \\
          &= \sum_{k \ge 1} \frac{(-1)^{k + 1} \, A(z^k)}{k}
      \end{align*}
      Exponenciando lo último
      resulta equivalente a la segunda parte.
    \item % MSet(A)
      Podemos considerar un multiconjunto finito
      como la combinación de una secuencia
      para cada tipo de elemento:
      \begin{equation*}
        \prod_{\alpha \in \mathscr{A}} \Seq(\{ \alpha \})
      \end{equation*}
      La función generatriz buscada es:
      \begin{equation*}
        \prod_{\alpha \in \mathscr{A}}
          \frac{1}{1 - z^{\lvert \alpha \rvert}}
          = \prod_{n \ge 0} \frac{1}{(1 - z^n)^{a_n}}
      \end{equation*}
      Esto provee la primera parte.
      Nuevamente aplicamos logaritmo para simplificar:
      \begin{align*}
        \ln \prod_{\alpha \in \mathscr{A}}
               \frac{1}{1 - z^{\lvert \alpha \rvert}}
          &= - \sum_{\alpha \in \mathscr{A}}
                 \ln \left( 1 - z^{\lvert \alpha \rvert} \right) \\
          &= \sum_{\alpha \in \mathscr{A}}
               \sum_{k \ge 1}
                 \frac{z^{k \lvert \alpha \rvert}}{k} \\
          &= \sum_{k \ge 1}
               \frac{1}{k} \,
                 \sum_{\alpha \in \mathscr{A}}
                   z^{k \lvert \alpha \rvert} \\
          &= \sum_{k \ge 1}
               \frac{A(z^k)}{k}
      \end{align*}
    \item % Cyc(A)
      Esta situación es más compleja de tratar,
      vea la discusión en~%
        \cite[sección~21.2.3]{brand17:_fundamentos_informatica}.
      \qedhere
    \end{enumerate}
  \end{proof}

\subsection{Algunas aplicaciones}
\label{sec:ms-ogf-aplicaciones}

  La clase de los árboles binarios \(\mathscr{B}\)
  es por definición es la unión disjunta del árbol vacío
  y la clase de tuplas de un nodo (la raíz)
  y dos árboles binarios.
  O sea:
  \begin{equation*}
    \mathscr{B}
      = \mathscr{E}
          + \mathscr{Z} \times \mathscr{B} \times \mathscr{B}
  \end{equation*}
  de donde directamente obtenemos:
  \begin{equation*}
    B(z)
      = 1 + z B^2(z)
  \end{equation*}
  Con el cambio de variable \(u(z) = B(z) - 1\) queda:
  \begin{equation*}
    u(z)
      = z (1 + u(z))^2
  \end{equation*}
  Es aplicable la fórmula de inversión de Lagrange~%
    \cite[teorema~17.8]{brand17:_fundamentos_informatica}
  con \(\phi(u) = (u + 1)^2\) y \(f(u) = u\):
  \begin{align*}
    \left[ z^n \right] u(z)
      &= \frac{1}{n} \, \left[ u^{n - 1} \right] \, \phi(u)^n \\
      &= \frac{1}{n} \, \left[ u^{n - 1} \right] \, (u + 1)^{2 n} \\
      &= \frac{1}{n} \, \binom{2 n}{n - 1} \\
      &= \frac{1}{n + 1} \, \binom{2 n}{n}
  \end{align*}
  Tenemos,
  como \(u(z) = B(z) - 1\)
  (sabemos que \([z^0] u(z) = 0\))
  y de la condición inicial \(b_0 = 1\):
  \begin{equation*}
    b_n =
    \begin{cases}
      \displaystyle
        \frac{1}{n + 1} \, \binom{2 n}{n}
               & \text{si \(n \ge 1\)} \\
      1
               & \text{si \(n = 0\)}
    \end{cases}
  \end{equation*}
  Casualmente la expresión simplificada para \(n \ge 1\)
  da el valor correcto \(b_0 = 1\).
  Estos son los números de Catalan,
  es \(b_n = C_n\).

  Sea ahora \(\mathscr{A}\)
  la clase de \emph{árboles con raíz ordenados},
  formados por un nodo raíz
  conectado a las raíces de una secuencia de árboles ordenados.
  La idea es que la raíz tiene hijos en un cierto orden.
  Simbólicamente:
  \begin{equation*}
    \mathscr{A}
      = \mathscr{Z} \times \Seq(\mathscr{A})
  \end{equation*}
  El método simbólico entrega directamente la ecuación:
  \begin{equation*}
    A(z)
      = \frac{z}{1 - A(z)}
  \end{equation*}
  Nuevamente es aplicable la fórmula de inversión de Lagrange,
  con \(\phi(A) = (1 - A)^{-1}\) y \(f(A) = A\):
  \begin{align*}
    \left[ z^n \right] A(z)
      &= \frac{1}{n} \, \left[ A^{n - 1} \right] \, \phi(A)^n \\
      &= \frac{1}{n} \, \left[ A^{n - 1} \right] \, (1 - A)^{-n} \\
      &= \frac{1}{n} \, \binom{2 n - 2}{n - 1} \\
      &= C_{n - 1}
  \end{align*}
  Otra vez números de Catalan.

  La manera obvia de representar \(\mathbb{N}_0\)
  es por secuencias de marcas,
  como \(||||\) para 4;
  simbólicamente \(\mathbb{N}_0 = \Seq(\mathscr{Z})\).
  Para calcular el número de multiconjuntos de \(k\) elementos
  tomados entre \(n\),
  un multiconjunto queda representado
  por las cuentas de cada uno los \(n\) elementos de que se compone,
  y eso corresponde a:
  \begin{equation*}
    \mathbb{N}_0 \times \dotsb \times \mathbb{N}_0
      = (\Seq(\mathscr{Z}))^n
  \end{equation*}
  Para obtener el número que nos interesa:
  \begin{align*}
    \multiset{n}{k}
      &= \left[ z^k \right] (1 - z)^{-n} \\
      &= (-1)^n \binom{-n}{k} \\
      &= \binom{n + k - 1}{n}
  \end{align*}
  Una \emph{combinación} de \(n\) es expresarlo como una suma.
  Por ejemplo,
  hay \num{8} combinaciones de \num{4}:
  \begin{equation*}
      4
        = 3 + 1
        = 2 + 2
        = 2 + 1 + 1
        = 1 + 3
        = 1 + 2 + 1
        = 1 + 1 + 2
        = 1 + 1 + 1 + 1
  \end{equation*}
  Sea \(c(n)\) el número de combinaciones de \(n\).
  Como antes,
  tenemos \(\mathbb{N} = \mathscr{Z} \times \Seq(\mathscr{Z})\),
  que da:
  \begin{equation*}
    N(z)
      = \frac{z}{1 - z}
  \end{equation*}
  A su vez,
  una combinación no es más que una secuencia de naturales
  (separados por \(+\)):
  \begin{equation*}
    \mathscr{C}
      = \Seq(\mathbb{N})
  \end{equation*}
  Directamente resulta:
  \begin{align*}
    C(z)
      &= \sum_{n \ge 0} c(n) z^n \\
      &= \frac{1}{1 - N(z)} \\
      &= \frac{1}{2} + \frac{1}{2} \cdot \frac{1}{1 - 2 z} \\
    c(n)
      &= [z^n] C(z) \\
      &= \frac{1}{2} \, [n = 0] + \frac{1}{2} \cdot 2^n \\
      &= \frac{1}{2} \, [n = 0] + 2^{n - 1}
  \end{align*}
  Esto es consistente con \(c(4) = 8\) obtenido arriba.

\section{Objetos rotulados}
\label{sec:objetos-rotulados}

  En la discusión previa solo interesaba el tamaño de los objetos,
  no su disposición particular.
  Consideraremos ahora objetos rotulados,
  donde importa cómo se compone el objeto de sus partes
  (los átomos tienen identidades,
   posiblemente porque se ubican en orden).

  El objeto más simple con partes rotuladas son las permutaciones
  (biyecciones \(\sigma \colon [n] \rightarrow [n]\),
   podemos considerarlas secuencias de átomos numerados).
  Para la función generatriz exponencial tenemos,
  ya que hay \(n!\) permutaciones de \(n\) elementos:
  \begin{equation*}
    \sum_{\sigma}
        \frac{z^{\lvert \sigma \rvert}}{\lvert \sigma \rvert !}
      = \sum_{n \ge 0} n! \, \frac{z^n}{n!}
      = \frac{1}{1 - z}
  \end{equation*}

  Lo siguiente más simple de considerar
  es colecciones de ciclos rotulados.
  Por ejemplo,
  escribimos \((1\;3\;2)\) para el objeto
  en que viene \num{3} luego de \num{1},
  \num{2} sigue a \num{3},
  y a su vez \num{1} sigue a \num{2}.
  Así \((2\;1\;3)\) es solo otra forma de anotar el ciclo anterior,
  que no es lo mismo que \((3\;1\;2)\).
  Interesa definir formas consistentes
  de combinar objetos rotulados.
  Por ejemplo,
  al combinar el ciclo \((1\;2)\) con el ciclo \((1\;3\;2)\)
  resultará un objeto con \num{5} rótulos,
  y debemos ver cómo los distribuimos entre las partes.
  El cuadro~\ref{tab:ciclo+ciclo}
  reseña las posibilidades al respetar
  el orden de los elementos asignados a cada parte.
  \begin{table}[htbp]
    \centering
    \begin{tabular}{*{4}{>{\(}l<{\)}}}
      (1\;2) (3\;5\;4) & (2\;3) (1\;5\;4) & (3\;4) (1\;5\;2) &
          (4\;5) (1\;3\;2) \\
      (1\;3) (2\;5\;4) & (2\;4) (1\;5\;3) & (3\;5) (1\;4\;2) \\
      (1\;4) (2\;5\;3) & (2\;5) (1\;4\;3) \\
      (1\;5) (2\;4\;3)
    \end{tabular}
    \caption{Combinando los ciclos \((1\;2)\) y \((1\;3\;2)\)}
    \label{tab:ciclo+ciclo}
  \end{table}
  Es claro que lo que estamos haciendo es elegir
  un subconjunto de \num{2} rótulos
  de entre los \num{5} para asignárselos al primer ciclo.
  El combinar
  dos clases de objetos \(\mathscr{A}\) y \(\mathscr{B}\)
  de esta forma lo anotaremos \(\mathscr{A} \star \mathscr{B}\).

  Otra operación común es la \emph{composición},
  anotada \(\mathscr{A} \circ \mathscr{B}\).
  La idea es elegir un elemento \(\alpha \in \mathscr{A}\),
  luego elegir \(\lvert \alpha \rvert\) elementos
  de \(\mathscr{B}\),
  y reemplazar los \(\mathscr{B}\) por las partes de \(\alpha\),
  en el orden que están rotuladas;
  para finalmente asignar rótulos a los átomos
  que conforman la.estructura completa
  respetando el orden de los rótulos
  al interior de los \(\mathscr{B}\)
  (igual como lo hicimos para \(\star\)).

  Ocasionalmente es útil \emph{marcar}
  uno de los componentes del objeto,
  operación que anotaremos \(\mathscr{A}^\bullet\).
  Usaremos también la construcción \(\MSet(\mathscr{A})\),
  que podemos considerar como una secuencia de elementos numerados
  obviando el orden.
  Cuidado,
  muchos textos le llaman \(\Set()\) a esta operación.

  Tenemos el siguiente teorema:
  \begin{theorem}[Método simbólico, EGF]
    \label{theo:ms-EGF}
    Sean \(\mathscr{A}\) y \(\mathscr{B}\) clases de objetos rotulados,
    con funciones generatrices exponenciales
    \(\widehat{A}(z)\) y \(\widehat{B}(z)\),
    respectivamente.
    Entonces tenemos
    las siguientes funciones generatrices exponenciales:
    \begin{enumerate}
    \item
      Para enumerar \(\mathscr{A}^\bullet\):
      \begin{equation*}
        z \mathrm{D} \widehat{A}(z)
      \end{equation*}
    \item
      Para enumerar \(\mathscr{A} + \mathscr{B}\):
      \begin{equation*}
        \widehat{A}(z) + \widehat{B}(z)
      \end{equation*}
    \item
      Para enumerar \(\mathscr{A} \star \mathscr{B}\):
      \begin{equation*}
        \widehat{A}(z) \cdot \widehat{B}(z)
      \end{equation*}
    \item
      Para enumerar \(\mathscr{A} \circ \mathscr{B}\):
      \begin{equation*}
        \widehat{A}(\widehat{B}(z))
      \end{equation*}
    \item
      Para enumerar \(\Seq(\mathscr{A})\):
      \begin{equation*}
        \frac{1}{1 - \widehat{A}(z)}
      \end{equation*}
    \item
      Para enumerar \(\MSet(\mathscr{A})\):
      \begin{equation*}
        \mathrm{e}^{\widehat{A}(z)}
      \end{equation*}
    \item
      Para enumerar \(\Cyc(\mathscr{A})\):
      \begin{equation*}
        -\ln(1 - \widehat{A}(z))
      \end{equation*}
    \end{enumerate}
  \end{theorem}
  \begin{proof}
    Usaremos casos ya demostrados en las demostraciones sucesivas.
    \begin{enumerate}
    \item % mark A
      El objeto \(\alpha \in \mathscr{A}\)
      da lugar a \(\lvert \alpha \rvert\) objetos
      al marcar cada uno de sus átomos,
      lo que da la función generatriz exponencial:
      \begin{equation*}
        \sum_{\alpha \in \mathscr{A}}
          \lvert \alpha \rvert
            \frac{z^{\lvert \alpha \rvert}}{\lvert \alpha \rvert !}
      \end{equation*}
      Esto es lo indicado.
    \item % A + B
      Nuevamente trivial.
    \item % A x B
      El número de objetos \(\gamma\) que se obtienen
      al combinar \(\alpha \in \mathscr{A}\)
      con \(\beta \in \mathscr{B}\) es:
      \begin{equation*}
        \binom{\lvert \alpha \rvert + \lvert \beta \rvert}
              {\lvert \alpha \rvert}
      \end{equation*}
      y tenemos la función generatriz exponencial:
      \begin{equation*}
        \sum_{\gamma \in \mathscr{A} \star \mathscr{B}}
            \frac{z^{\lvert \gamma \rvert}}{\lvert \gamma \rvert !}
          = \sum_{\substack{
                     \alpha \in \mathscr{A} \\
                     \beta \in \mathscr{B}
                 }}
               \binom{\lvert \alpha \rvert + \lvert \beta \rvert}
                     {\lvert \alpha \rvert}
                  \frac{z^{\lvert \alpha \rvert
                            + \lvert \beta \rvert}}
                       {(\lvert \alpha \rvert
                            + \lvert \beta \rvert)!}
          = \left(
              \sum_{\alpha \in \mathscr{A}}
                \frac{z^{\lvert \alpha \rvert}}
                     {\lvert \alpha \rvert !}
            \right)
              \cdot \left(
                \sum_{\beta \in \mathscr{B}}
                  \frac{z^{\lvert \beta \rvert}}
                       {\lvert \beta \rvert !}
                    \right)
          = \widehat{A}(z) \cdot \widehat{B}(z)
      \end{equation*}
    \item % A circ B
      Tomemos \(\alpha \in \mathscr{A}\),
      de tamaño \(n = \lvert \alpha \rvert\),
      y \(n\) elementos de \(\mathscr{B}\) en orden
      a ser reemplazados por las partes de \(\alpha\).
      Esa secuencia de \(\mathscr{B}\) es representada por:
      \begin{equation*}
        \mathscr{B}
          \star \mathscr{B}
          \star \dotsb
          \star \mathscr{B}
      \end{equation*}
      con función generatriz exponencial:
      \begin{equation*}
        \widehat{B}^n (z)
      \end{equation*}
      Sumando sobre las contribuciones:
      \begin{equation*}
        \sum_{\alpha \in \mathscr{A}}
           \frac{\widehat{B}^{\lvert \alpha \rvert}(z)}
                {\lvert \alpha \rvert \, !}
      \end{equation*}
      Esto es lo prometido.
    \item % Seq(A)
      Primeramente,
      podemos describir:
      \begin{equation*}
        \Seq(\mathscr{Z})
          = \mathscr{E} + \mathscr{Z} \star \Seq(\mathscr{Z})
      \end{equation*}
      que lleva a ecuación:
      \begin{equation*}
        \widehat{S}(z)
          = 1 + z \widehat{S}(z)
      \end{equation*}
      de donde:
      \begin{equation*}
        \widehat{S}(z)
          = \frac{1}{1 - z}
      \end{equation*}
      Aplicando composición se obtiene lo indicado.
    \item % MSet(A)
      Hay un único multiconjunto de \(n\) elementos rotulados
      (se rotulan simplemente de 1 a  \(n\)),
      con lo que \(\MSet(\mathscr{Z})\) corresponde a:
      \begin{equation*}
        \sum_{n \ge 0} \frac{z^n}{n!}
          = \exp(z)
      \end{equation*}
      Al aplicar composición resulta lo anunciado.

      Otra demostración es considerar el multiconjunto de \(\mathscr{A}\),
      descrito por \(\mathscr{M} = \MSet(\mathscr{A})\).
      Si marcamos uno de los átomos de \(\mathscr{M}\)
      estamos marcando uno de los \(\mathscr{A}\),
      el resto sigue formando un multiconjunto de \(\mathscr{A}\):
      \begin{equation*}
        \mathscr{M}^\bullet
          = \mathscr{A}^\bullet \star \mathscr{M}
      \end{equation*}
      Por lo anterior:
      \begin{equation*}
        z \widehat{M}'(z)
          = z \widehat{A}'(z) \widehat{M}(z)
      \end{equation*}
      Hay un único multiconjunto de tamaño \num{0},
      o sea \(\widehat{M}(0) = 1\);
      y hemos impuesto la condición
      que no hay objetos de tamaño \num{0} en \(\mathscr{A}\),
      vale decir,
      \(\widehat{A}(0) = 0\).
      Así la solución a la ecuación diferencial es:
      \begin{equation*}
        \widehat{M}(z)
          = \exp(\widehat{A}(z))
      \end{equation*}
    \item % Cyc(A)
      Consideremos un ciclo de \(\mathscr{A}\),
      o sea \(\mathscr{C} = \Cyc(\mathscr{A})\).
      Si marcamos los \(\mathscr{C}\),
      estamos marcando uno de los \(\mathscr{A}\),
      y el resto es una secuencia:
      \begin{equation*}
        \mathscr{C}^\bullet
          = \mathscr{A}^\bullet \star \Seq(\mathscr{A})
      \end{equation*}
      Esto se traduce en la ecuación diferencial:
      \begin{equation*}
        z \widehat{C}'(z)
          = z \widehat{A}'(z) \frac{1}{1 - \widehat{A}(z)}
      \end{equation*}
      Integrando bajo el entendido \(\widehat{C}(0) = 0\)
      con \(\widehat{A}(0) = 0\)
      se obtiene lo indicado.
    \end{enumerate}
  \end{proof}

\subsection{Algunas aplicaciones}
\label{sec:ms-egf-aplicaciones}

  Un ejemplo simple es el caso de permutaciones,
  que son simplemente secuencias de elementos rotulados:
  \begin{align*}
    \mathscr{P}
      &= \Seq(\mathscr{Z}) \\
    \widehat{P}(z)
      &= \frac{1}{1 - z} \\
    P_n
      &= n! [z^n] \widehat{P}(z) \\
      &= n!
  \end{align*}

  Consideremos colecciones de ciclos:
  \begin{equation*}
    \MSet(\Cyc(\mathscr{Z}))
  \end{equation*}
  Vemos que esto corresponde a:
  \begin{equation*}
    \exp\left( \ln \frac{1}{1 - z} \right)
      = \frac{1}{1 - z}
  \end{equation*}
  Hay tantas permutaciones de tamaño \(n\) como colecciones de ciclos.
  Una biyección se da ordenando los ciclos
  de manera que se inicien con su mayor elemento,
  y listar los ciclos en orden de máximo elemento creciente.
  Cualquier lista de elementos puede reinterpretarse como ciclos
  de una única manera de esta forma.
  En el fondo,
  podemos representar permutaciones
  como los ciclos ordenados para comenzar con sus elementos máximos.

  Podemos describir permutaciones
  como un conjunto de elementos que quedan fijos
  combinado con otros elementos que están fuera de orden
  (un \emph{desarreglo},
   clase \(\mathscr{D}\)).
  O sea:
  \begin{align*}
    \mathscr{P}
      &= \MSet(\mathscr{Z}) \star \mathscr{D} \\
    \frac{1}{1 - z}
      &= \mathrm{e}^z \cdot \widehat{D}(z) \\
    \widehat{D}(z)
      &= \frac{\mathrm{e}^{-z}}{1 - z}
  \end{align*}
  De acá,
  por propiedades de las funciones generatrices vemos que:
  \begin{equation*}
    [z^n] \widehat{D}(z)
      = \sum_{0 \le k \le n} \frac{(-1)^k}{k!}
  \end{equation*}
  y tenemos,
  usando la notación común para el número de desarreglos de tamaño \(n\):
  \begin{equation*}
    D_n
      = n! \sum_{0 \le k \le n} \frac{(-1)^k}{k!}
  \end{equation*}

  Una \emph{involución} es una permutación \(\pi\)
  tal que \(\pi \circ \pi\) es la identidad.
  Es claro que una involución
  es una colección de ciclos de largos \num{1} y \num{2},
  o sea:
  \begin{equation*}
    \mathscr{I}
      = \MSet(\Cyc_{\le 2}(\mathscr{Z}))
  \end{equation*}
  Revisando la derivación,
  vemos que \(\Cyc_{\le 2}(\mathscr{Z})\) corresponde a:
  \begin{equation*}
    \frac{z}{1} + \frac{z^2}{2}
  \end{equation*}
  y tenemos para la función generatriz:
  \begin{equation*}
    \widehat{I}(z)
      = \exp\left( \frac{z}{1} + \frac{z^2}{2} \right)
  \end{equation*}
  Un paquete de álgebra simbólica da:
  \begin{equation*}
    \widehat{I}(z)
      = 1 +    1 \cdot \frac{z}{1!}
          +    2 \cdot \frac{z^2}{2!}
          +    4 \cdot \frac{z^3}{3!}
          +   10 \cdot \frac{z^4}{4!}
          +   26 \cdot \frac{z^5}{5!}
          +   76 \cdot \frac{z^6}{6!}
          +  232 \cdot \frac{z^7}{7!}
          +  764 \cdot \frac{z^8}{8!}
          + 2620 \cdot \frac{z^9}{9!}
%	  + 9496 \cdot \frac{z^{10}}{10!}
          + \dotsm
  \end{equation*}

  Un \emph{desarreglo} es una permutación sin puntos fijos,
  vale decir,
  sin ciclos de largo \num{1}.
  O sea:
  \begin{equation*}
    \mathscr{D}
      = \MSet(Cyc_{> 1}(\mathscr{Z}))
  \end{equation*}
  Revisando las derivaciones para las operaciones,
  esto corresponde a:
  \begin{align*}
    \widehat{D}(z)
      &= \exp \left( \sum_{k \ge 2} \frac{z^k}{k} \right) \\
      &= \exp \left( \sum_{k \ge 1} \frac{z^k}{k} - z \right) \\
      &= \exp \left( \ln \frac{1}{1 - z} - z \right) \\
      &= \frac{1}{1 - z} \cdot e^{-z}
  \end{align*}
  Igual que obtuvimos antes.

\section{Funciones generatrices cumulativas}
\label{sec:gf-cumulativa}

  Para precisar,
  consideremos una clase de objetos \(\mathscr{A}\).
  Como siempre el número de objetos de tamaño \(n\)
  lo anotaremos \(a_n\),
  con función generatriz:
  \begin{align}
    A(z)
      &= \sum_{\alpha \in \mathscr{A}} z^{\lvert \alpha \rvert}
            \label{eq:A-def} \\
      &= \sum_{n \ge 0} a_n z^n
            \label{eq:A-an}
  \end{align}
  Consideremos no solo el número de objetos,
  sino alguna característica,
  cuyo valor para el objeto \(\alpha\) anotaremos \(\chi(\alpha)\).
  Es natural definir la \emph{función generatriz cumulativa}:
  \begin{equation}
    \label{eq:cogf-def}
    C(z)
      = \sum_{\alpha \in \mathscr{A}} \chi(\alpha) z^{\lvert \alpha \rvert}
  \end{equation}
  Vale decir,
  los coeficientes son la suma de la medida \(\chi\)
  para un tamaño dado:
  \begin{equation}
    \label{eq:cogf-coefficient}
    [z^n] C(z)
      = \sum_{\lvert \alpha \rvert = n} \chi(\alpha)
  \end{equation}
  Así tenemos el valor promedio para objetos de tamaño \(n\):
  \begin{equation}
    \label{eq:chi-expected}
    \Exp_n[\chi]
      = \frac{[z^n] C(z)}{[z^n] A(z)}
  \end{equation}

  La discusión precedente es aplicable
  si tenemos objetos no rotulados entre manos.
  Si corresponden objetos rotulados,
  podemos definir las respectivas funciones generatrices exponenciales:
  \begin{align}
    \widehat{A}(z)
      &= \sum_{\alpha \in \mathscr{A}}
           \frac{z^{\lvert \alpha \rvert}}{\lvert \alpha \rvert !}
                \label{eq:Ahat-def} \\
      &= \sum_{n \ge 0} a_n \frac{z^n}{n!}
                \label{eq:Ahat-an} \\
    \widehat{C}(z)
      &= \sum_{\alpha \in \mathscr{A}}
           \chi(\alpha) \frac{z^{\lvert \alpha \rvert}}{\lvert \alpha \rvert !}
                \label{eq:Chat-def}
  \end{align}
  Nuevamente,
  como los factoriales en los coeficientes se cancelan:
  \begin{equation}
    \label{eq:chi-expected-hat}
    \Exp_n[\chi]
      = \frac{[z^n] \widehat{C}(z)}{[z^n] \widehat{A}(z)}
  \end{equation}

\section{Funciones generatrices bivariadas}
\label{sec:gf-bivariada}

  Consideremos una clase \(\mathscr{A}\),
  con objetos \(\alpha \in \mathscr{A}\) de tamaño \(\lvert \alpha \rvert\);
  y a su vez un parámetro,
  cuyo valor para \(\alpha\) es \(\chi(\alpha)\).
  Si los átomos que componen \(\alpha\) son indistinguibles,
  es natural definir la función generatriz bivariada ordinaria:
  \begin{equation}
    \label{eq:obgf-def}
    A(z, u)
      = \sum_{\alpha \in \mathscr{A}} z^{\lvert \alpha \rvert} u^{\chi(\alpha)}
  \end{equation}
  De la misma forma,
  si los átomos son distinguibles
  es apropiada la función generatriz exponencial:
  \begin{equation}
    \label{eq:ebgf-def}
    \widehat{A}(z, u)
      = \sum_{\alpha \in \mathscr{A}}
          \frac{z^{\lvert \alpha \rvert}}{\lvert \alpha \rvert !}
             u^{\chi(\alpha)}
  \end{equation}
  Es común que nos interese el valor promedio de \(\chi(\alpha)\)
  para objetos de tamaño dado.
  Note que:
  \begin{equation}
    \label{eq:obgf-partial}
    \frac{\partial A}{\partial u}
      = \sum_{\alpha \in \mathscr{A}}
          \chi(\alpha) u^{\chi(\alpha) - 1} z^{\lvert \alpha \rvert}
  \end{equation}
  Así podemos calcular los valores promedios a partir de
  los coeficientes de las siguientes sumas:
  \begin{align}
    \sum_{\alpha \in \mathscr{A}} z^{\lvert \alpha \rvert}
      &= A(z, 1)
          \label{eq:obgf|u=1} \\
    \sum_{\alpha \in \mathscr{A}} \chi(\alpha) z^{\lvert \alpha \rvert}
      &= \left. \frac{\partial A}{\partial u} \right\rvert_{u = 1}
          \label{eq:obgf_u|u=1}
  \end{align}
  Vemos que~\eqref{eq:obgf|u=1}
  no es más que la función generatriz del número de objetos,
  mientras~\eqref{eq:obgf_u|u=1}
  es la función generatriz cumulativa.

\bibliography{../referencias}

%%% Local Variables:
%%% mode: latex
%%% TeX-master: "../INF-221_notas"
%%% ispell-local-dictionary: "spanish"
%%% End:

% LocalWords:  Symbolic Method for Dummies proveniencia multiconjunto
% LocalWords:  multiconjuntos biyecciones Exponenciando Cyc bivariada
% LocalWords:  reinterpretarse cumulativas cumulativa bivariadas eq
% LocalWords:  obgf
