\bibliographystyle{babplain-fl}

\chapter{Interpolación}
\label{cha:interpolacion}

  Hay situaciones en las cuales se conoce el valor de una función
  solo en algunos puntos dados,
  y queremos calcular valores en puntos intermedios.
  Por ejemplo,
  contamos solo con algunos valores medidos,
  o hay una tabla de valores numéricos.
  Este es el problema de interpolación.

  Nos dan los valores exactos de una función desconocida
  en \(n + 1\) puntos \(f(x_0), \ldots, f(x_n)\),
  queremos hallar una función que tome esos valores
  (para calcular valores intermedios).
  El caso más común es utilizar \emph{polinomios}.
  Para esta ocasión veremos
  que hay exactamente \emph{un} polinomio de grado a lo más \(n\)
  que pasa por \(n + 1\) puntos.

  Entre las cosas que podemos hacer
  para hallar la interpolación de \(f(x)\) tenemos:
  \begin{itemize}
  \item
    Obtener los coeficientes
    del polinomio en forma canónica de un sistema de ecuaciones.
  \item
    De forma implícita,
    dando el polinomio en forma no canónica.
  \end{itemize}

\section{Por sistema de ecuaciones}

  Suponiendo \(p(x) = a_0 + a_1 x + \dotsb + a_n x^n\),
  creamos un sistema de ecuaciones de la forma:
  \begin{align*}
    p(x_0)
       = f(x_0)
      &= a_0 + a_1 x_0 + a_2 x_0^2 + \dotsb a_n x_0^n \\
    p(x_1)
       = f(x_1)
      &= a_0 + a_1 x_1 + a_2 x_1^2 + \dotsb a_n x_1^n \\
      &\vdots \\
    p(x_n)
       = f(x_n)
      &= a_0 + a_1 x_n + a_2 x_n^2 + \dotsb a_n x_n^n
  \end{align*}
  que en forma de matriz se escribe como:
  \begin{equation}
    \label{04::Sistema:de:ecuaciones:1}
    \begin{pmatrix}
      1 & x_0 & x_0^2 & \cdots & x_0^n \\
      1 & x_1 & x_1^2 & \cdots & x_1^n \\
      \vdots & \vdots & \vdots & \ddots & \vdots & \\
      1 & x_n & x_n^2 & \cdots & x_n^n
    \end{pmatrix}
    \begin{pmatrix}
      a_0 \\
      a_1 \\
      \vdots \\
      a_n
    \end{pmatrix} =
    \begin{pmatrix}
      f(x_0) \\
      f(x_1) \\
      \vdots \\
      f(x_n)
    \end{pmatrix}
  \end{equation}
  Con igual cantidad de ecuaciones e incógnitas,
  el sistema de ecuaciones~\eqref{04::Sistema:de:ecuaciones:1}
  tiene solución única si y solo si
  el determinante es distinto de cero
  (ver por ejemplo Treil~%
    \cite{treil17:_linear_algeb_done_wrong}):
  \begin{equation}
    \label{04::Determinante:de:Vandermonde = 0}
    \left\lvert
      \begin{matrix}
        1 & \cdots & x_0^n\\
        \vdots & \ddots & \vdots \\
        1 & \cdots & x_n^n
      \end{matrix}
    \right\rvert
      \ne 0
  \end{equation}

  El determinante de la ecuación \eqref{04::Determinante:de:Vandermonde = 0}
  resulta ser el \emph{determinante de Vandermonde}.
  \begin{theorem}
    El determinante de Vandermonde vale:
    \begin{equation*}
      \begin{vmatrix}
        1 & a_1 & a_1^2 & \cdots & a_1^{n - 1} \\
        1 & a_2 & a_2^2 & \cdots & a_2^{n - 1} \\
        \vdots & \vdots & \vdots & \ddots & \vdots \\
        1 & a_n & a_n^2 & \cdots & a_n^{n - 1}
      \end{vmatrix}
        = \prod_{1 \le i < j \le n} (a_j - a_i)
    \end{equation*}
  \end{theorem}
  \begin{proof}
    Por inducción sobre \(n\).
    Para abreviar,
    llamaremos:
    \begin{equation*}
      V_n
        = \begin{vmatrix}
            1 & a_1 & a_1^2 & \cdots & a_1^{n - 1} \\
            1 & a_2 & a_2^2 & \cdots & a_2^{n - 1} \\
            \vdots & \vdots & \vdots & \ddots & \vdots \\
            1 & a_n & a_n^2 & \cdots & a_n^{n - 1}
          \end{vmatrix}
    \end{equation*}
    \begin{description}
    \item[Base:]
      El caso \(n = 1\) es trivial,
      es simplemente:
      \begin{equation*}
        \begin{vmatrix}
          1
        \end{vmatrix}
          = 1
      \end{equation*}
    \item[Inducción:]
      Supongamos que vale para \(k\),
      y consideremos el determinante:
      \begin{equation*}
        \begin{vmatrix}
          1 & a_1 & a_1^2 & \cdots & a_1^k \\
          1 & a_2 & a_2^2 & \cdots & a_2^k \\
          \vdots & \vdots & \vdots & \ddots & \vdots \\
          1 & a_k & a_k^2 & \cdots & a_k^k \\
          1 & x	  & x^2	  & \cdots & x^k \\
        \end{vmatrix}
      \end{equation*}
      Expandiendo por la última fila,
      vemos que es un polinomio en \(x\) de grado a lo más \(k\),
      llamémosle \(f(x)\).
      Como determinantes con filas iguales son cero,
      sabemos que \(f(a_i) = 0\) para \(1 \le i \le k\),
      o sea:
      \begin{align*}
        f(x)
          &= c (x - a_1) (x - a_2) \dotsm (x - a_k) \\
          &= c \prod_{1 \le i \le k} (x - a_i)
      \end{align*}
      Como el grado de \(f\) es a lo más \(k\),
      y tenemos \(k\) factores lineales en \(x\),
      \(c\) es independiente de \(x\).

      La expansión por la última fila,
      nuevamente,
      muestra que el coeficiente de \(x^k\) es \(V_k\),
      con lo que \(c = V_k\).
      Reemplazando \(x \mapsto a_{k + 1}\)
      con nuestra hipótesis de inducción entrega:
      \begin{align*}
        V_{k + 1}
          &= V_k \cdot \prod_{1 \le i \le k} (a_{k + 1} - a_i) \\
          &= \prod_{1 \le i < j \le k} (a_j - a_i)
               \cdot \prod_{1 \le i \le k} (a_{k + 1} - a_i) \\
          &= \prod_{1 \le i < j \le k + 1} (a_j - a_i)
      \end{align*}
    \end{description}
    Por inducción, vale para todo \(n \in \mathbb{N}\).
  \end{proof}

\section{De forma implícita}

  Para encontrar la interpolación de \(f\) de manera implícita
  simplemente inventamos un polinomio que pase por los puntos.

\subsection{Forma de Lagrange}

  Consiste en usar el polinomio:
  \begin{align}
    p(x)
      &= f(x_0) \frac{(x - x_1) (x - x_2) \dotsm (x - x_n)}
                     {(x_0 - x_1) (x_0 - x_2) \dotsm (x_0 - x_n)}
           + f(x_1) \frac{(x -x_0) (x - x_2) \dotsm (x - x_n)}
                         {(x_1 - x_0) (x_1 - x_2) \dotsm (x_1 - x_n)}
           + \dotsb \notag \\
      & \qquad {} + f(x_n) \frac{(x - x_0) \dotsm (x - x_{n - 1})}
                               {(x_n - x_0) \dotsm (x_n - x_{n - 1})}
              \notag \\
     &= \sum_{0 \le k \le n} f(x_k)
          \prod_{\substack{0 \le j \le n \\ j \ne k}}
            \frac{x - x_j}{x_k - x_j}
              \label{04::Lagrange:2}
  \end{align}
  como interpolación de \(f\).
  Es claro que si evalúa \(p\)
  en alguno de los \(x_k\) con \(k \in \{ 0, 1, \dotsc, n\}\) que nos entregan,
  se tiene que \(p(x_k) = f(x_k)\),
  y el polinomio es de grado \(n\).

  Para referencia futura,
  llamaremos:
  \begin{equation}
    \label{eq:Lagrange-bases}
    \ell_i(x)
      = \prod_{\substack{0 \le j \le n \\ j \ne i}}
            \frac{x - x_j}{x_i - x_j}
  \end{equation}
  Estos polinomios de grado \(n\),
  los \emph{polinomios base de Lagrange},
  dependen de los puntos \(x_i\),
  y tienen la particularidad que:
  \begin{equation*}
    \ell_i(x_j)
      = [i = j]
  \end{equation*}
  Acá usamos la convención de Iverson:
  \begin{equation*}
    [\text{condición}]
      = \begin{cases}
          0 & \text{si la condición es falsa} \\
          1 & \text{si la condición es verdadera}
        \end{cases}
  \end{equation*}
  En resumen,
  son linealmente independientes siempre que los \(x_i\) son todos distintos
  (básicamente,
   el determinante de Vandermonde es cero solo si algún \(x_i\) se repite).

  En la práctica,
  la fórmula~\eqref{04::Lagrange:2} es poco deseable,
  requiere \(O(n^2)\) computación para calcular \(p(x)\).
  Una forma alternativa es la \emph{fórmula baricéntrica}
  que recomienda Winrich~%
    \cite{winrich69:_compar_evaluat_schem_interp_polyn}
  al comparar varias alternativas en términos de números de operaciones:
  \begin{equation}
    \label{eq:formula-baricentrica}
    p(x)
      = \frac{\sum_{0 \le j \le n} \frac{w_j f(x_j)}{x - x_j}}
             {\sum_{0 \le j \le n} \frac{w_j}{x - x_j}}
  \end{equation}
  donde:
  \begin{equation}
    \label{eq:wj-Lagrange}
    w_j
      = \left(
          \prod_{\substack{0 \le k \le n \\ k \ne j}} (x_j -x_k)
        \right)^{-1}
  \end{equation}
  (claro que la expresión~\eqref{eq:formula-baricentrica}
   es indefinida cuando \(x = x_i\),
   hay que manejar ese caso en forma separada).
  Berrut y Trefethen~%
    \cite{berrut04:_bary_lagrange_interp} discuten esta fórmula en detalle.
  Higham~%
    \cite{higham04:_numer_stab_baryc_lagrange_interp}
  analiza esta técnica,
  concluyendo que es numéricamente estable
  y recomendable para uso general.
  Por ejemplo,
  es la técnica que emplea SciPy~%
    \cite{2020SciPy-NMeth}
  para interpolación por polinomios.

\subsection{Forma de Newton}
\label{sec:Newton-interpolation}

  Podemos escribir:
  \begin{alignat}{2}
    a_0
     &= f(x_0)
     &\qquad&
     Q_0(x)
       = a_0 \notag \\
  \intertext{Luego para \(k > 0\):}
     a_k
      &= \frac{f(x_k) - Q_{k - 1}(x_k)}
              {\prod_{0 \le i \le k - 1}(x_k - x_i)}
      &&
      Q_k(x)
        = Q_{k - 1}(x) + a_k \prod_{0 \le i \le k - 1} (x - x_i)
           \label{eq:interpolacion-Newton}
  \end{alignat}
  donde \(Q_k\) corresponde a la interpolación de grado \(k\).

\subsubsection{Diferencias divididas}
\label{divided-differences}

  La forma de Newton es engorrosa.
  Una alternativa se obtiene con \emph{diferencias divididas},
  que para los puntos \(x_0, \dotsc, x_n\) se definen mediante:
  \begin{align*}
    f[x_0]
      &= f(x_0) \\
    f[x_0, \dotsc, x_j]
      &= \frac{f(x_j) - Q_{j - 1}(x_j)}{\prod_{0 \le k \le j - 1} (x_j - x_k)}
  \end{align*}
  Con esta notación~\eqref{eq:interpolacion-Newton}
  se escribe:
  \begin{equation}
    \label{eq:interpolacion-diferencias-divididas}
    Q_n(x)
      = f[x_0]
          + f[x_0, x_1] (x - x_0)
          + \dotsb
          + f[x_0, x_1, \dotsc, x_n] \prod_{0 \le k < n}(x - x_k)
  \end{equation}
  La utilidad de la forma~\eqref{eq:interpolacion-diferencias-divididas}
  es por el siguiente resultado:
  \begin{lemma}
    Las diferencias divididas cumplen:
    \begin{equation}
      \label{eq:divided-differences-recurrence}
      f[x_0, \dotsc, x_n]
        = \frac{f[x_1, \dotsc, x_n] - f[x_0, \dotsc, x_{n - 1}]}
               {x_n - x_0}
    \end{equation}
  \end{lemma}
  \begin{proof}
    Para cualquier \(k\),
    sea \(Q_k\) el polinomio de grado a lo más \(k\)
    que interpola \(f\) en los puntos \(x_0, \dotsc, x_k\),
    o sea:
    \begin{equation*}
      Q_k(x_j)
        = f(x_j)
           \quad \text{para \(0 \le j \le k\)}
    \end{equation*}
    Consideremos el polinomio \(P(x)\) único de grado a lo más \(n - 1\)
    que interpola \(f\) en los puntos \(x_1, \dotsc, x_n\).
    Es simple verificar que:
    \begin{equation}
      \label{eq:QvsP}
      Q_n(x)
        = P(x) + \frac{x - x_n}{x_n - x_0} (P(x) - Q_{n - 1}(x))
    \end{equation}
    En~\eqref{eq:QvsP} el coeficiente de \(x^n\) al lado izquierdo
    es \(f[x_0, \dotsc, x_n]\).
    El coeficiente de \(x^{n - 1}\) en \(P(x)\)
    es \(f[x_1, \dotsc, x_n]\),
    y en \(Q_{n - 1}(x)\)
    es \(f[x_0, \dotsc, x_{n - 1}]\).
    O sea,
    el coeficiente de \(x^n\) al lado derecho de~\eqref{eq:QvsP}
    es:
    \begin{equation*}
      \frac{f[x_1, \dotsc, x_n] - f[x_0, \dotsc, x_{n - 1}]}
           {x_n - x_0}
    \end{equation*}
    como aseveramos.
  \end{proof}

\paragraph{Nota:}

  Algunos textos definen:
  \begin{equation*}
      f[x_0, \dotsc, x_n]
        = f[x_1, \dotsc, x_n] - f[x_0, \dotsc, x_{n - 1}]
  \end{equation*}
  en cuyo caso las fórmulas deben ajustarse
  para compensar por los denominadores faltantes.

  Un ejemplo
  (la función es la de Runge,
   ecuación~\eqref{eq:Runge-function})
  es la tabla del cuadro~\ref{tab:interpolacion}.
% Table created by divided-differences script
  \begin{table}[ht]
    \centering
    \begin{tabular}{*{ 8 }{>{\(}l<{\)}}}
      \multicolumn{1}{c}{\boldmath\(x\)\unboldmath} &
         \multicolumn{1}{c}{\boldmath\(y\)\unboldmath} &
         \multicolumn{ 6 }{c}{\textbf{Diferencias divididas}} \\
      \hline
      0,10 & 0,8000 \\
           &	    & -2,4615 \\
      0,30 & 0,3077 &	      & 4,3140 \\
           &	    & -0,9516 &	       & -5,6441 \\
      0,45 & 0,1649 &	      & 2,0564 &	 & 4,7074 \\
           &	    & -0,5403 &	       & -3,5258 &	  & -0,4531 \\
      0,50 & 0,1379 &	      & 1,1749 &	 & 4,4355 &	    & -4,6852\\
           &	    & -0,4229 &	       & -1,7516 &	  & -4,2013 \\
      0,55 & 0,1168 &	      & 0,7371 &	 & 1,9148 \\
           &	    & -0,2754 &	       & -0,8899 \\
      0,70 & 0,0755 &	      & 0,3811 \\
           &	    & -0,1421 \\
      0,90 & 0,0471
    \end{tabular}
    \caption{Tabla para interpolación}
    \label{tab:interpolacion}
  \end{table}
  Las entradas
  (diferencias divididas)
  se obtienen restando los elementos inmediatamente a la izquierda
  divididas por la diferencia de los valores de \(x\) en las diagonales.
  Para interpolar,
  se usa la primera fila diagonal.
  Por ejemplo,
  calculemos las aproximaciones a \(y\) para \(x = 0,47\)
  de los datos del cuadro~\ref{tab:interpolacion}.
  \begin{align*}
    Q_0(0,47)
      &= 0,80000 \\
    Q_1(0,47)
      &= Q_0(0,47) - 2,4315 (0,47 - 0,1) \\
      &= -0,11077 \\
    Q_2(0,47)
      &= Q_1(0,47) + 4,3140 (0,47 - 0,1) (0,47 - 0,3) \\
      &= 0,16058 \\
    Q_3(0,47)
      &= Q_2(0,47) - 5,6441 (0,47 - 0,1) (0,47 - 0,3) (0,47 - 0.45) \\
      &= 0,15348 \\
    Q_4(0,47)
      &= Q_3(0,47)
           + 4,7074 (0,47 - 0,1) (0,47 - 0,3) (0,47 - 0.45) (0,47 - 0,50) \\
      &= 0,15331 \\
    Q_5(0,47)
      &= 0,15330 \\
    Q_6(0,47)
      &= 0,15331
  \end{align*}
  El valor correcto es \num{0,15446}.

\section*{Ejercicios}
\label{sec:ejercicios-04}

  \begin{enumerate}
  \item
    Demuestre la fórmula baricéntrica~\eqref{eq:formula-baricentrica}.
    Defina:
    \begin{equation*}
      \ell(x)
        = \prod_{0 \le j \le n} (x - x_j)
    \end{equation*}
    escriba~\eqref{04::Lagrange:2} como:
    \begin{equation*}
      p(x)
        = \ell(x) \sum_{0 \le j \le n} \frac{w_j f(x_j)}{x - x_j}
    \end{equation*}
    Use esta forma de~\eqref{04::Lagrange:2} para interpolar la función \num{1},
    divida y simplifique.
  \item
    Plantee algoritmos eficientes
    para evaluar el polinomio interpolador en \(x\)
    dados los puntos \(x_0, \dotsc, x_n\)
    usando las fórmulas~\eqref{04::Lagrange:2},
    \eqref{eq:formula-baricentrica}
    y~\eqref{eq:interpolacion-Newton}.
    Separe inicialización que depende de los \(x_j\)
    de cálculos que dependen de \(x\)
    (o sea,
     maneje el caso en que dados los \(x_j\)
     interesa evaluar para varios \(x\)).
    Calcule el número de operaciones de punto flotante
    (\emph{flops})
    para evaluar el polinomio interpolador en \(x\) en cada caso.
  \end{enumerate}

\bibliography{../referencias}

%%% Local Variables:
%%% mode: latex
%%% TeX-master: "../INF-221_notas"
%%% ispell-local-dictionary: "spanish"
%%% End:

% LocalWords:  Interpolación interpolación llamémosle baricéntrica c
% LocalWords:  SciPy flops eq baricentrica sec interpolation QvsP
% LocalWords:  interpolacion function
