% INF-221_notas.tex
%
% Copyright (c) 2016-2018, 2022 Horst H. von Brand
% Derechos reservados. Vea COPYRIGHT para detalles

% Documento maestro de las clases de Algoritmos y Complejidad
% No cambiar formato.

\documentclass[czech, english, german, french, latin, spanish, fleqn]
              {memoir}

% Last language is fallback for babelbib
% Fonts
\usepackage{fourier, bm}
\usepackage{mathrsfs}
\usepackage[group-digits = false]{siunitx}
\usepackage{icomma}
% Languages
\usepackage{babel}
\usepackage{babelbib, cite}
\usepackage{chapterbib}
\setbtxfallbacklanguage{english}
% General LaTeX settings
%\usepackage{fixltx2e}
\usepackage{tikz}
\usepackage{afterpage}
\usepackage{pdflscape}
\usepackage{subfig}
\usepackage{enumitem}
\usepackage{array}
\usepackage{dcolumn}
\usepackage{multirow}
\usepackage{colortbl}
\usepackage{calc}
\usepackage[autostyle]{csquotes}
%\usepackage{fnpct}
   % See http://es.wikipedia.org/wiki/Wikipedia_discusión:Temas_recurrentes/Posición_de_las_referencias
\usepackage[makeroom]{cancel} % Permite usar xcancel en equaciones (tachar en fo% Index, glossary
\usepackage{makeidx}
\newcommand*{\textbfhy}[1]{\textbf{\hyperpage{#1}}}
% Math stuff
\usepackage{amsmath, amsthm, amssymb}
\usepackage{systeme}
\usepackage[all]{xy}
\usepackage{xfrac}
\usepackage{centernot}
\usepackage{scalerel}
\usepackage{mathtools} % Permite el uso del entorno rcases
\DeclareMathOperator*{\argmax}{arg\,max}
\DeclareMathOperator*{\argmin}{arg\,min}
% Chess stuff
\usepackage{chessboard}
% General Computer Science
\usepackage[algochapter, ruled, noline, nofillcomment]{algorithm2e}
\usepackage{listings}
\usepackage[basic]{complexity}
\newcommand\cplusplus{C\nolinebreak[4]\hspace{-.03em}%
                      \raisebox{.3ex}{\relsize{-0.75}{\textbf{++}}}}
% microtype should come after fonts and languages, and most packages
\usepackage[babel = true, stretch = 10]{microtype}
% Hyperref (PDF output settings), should be last
\usepackage[colorlinks, urlcolor = blue]{hyperref}

%%%
%%% From Herbert Voß' Mathmode
%%%

\def\mathllap{\mathpalette\mathllapinternal}
\def\mathllapinternal#1#2{%
        \llap{$\mathsurround=0pt#1{#2}$}% $
}
\def\clap#1{\hbox to 0pt{\hss#1\hss}}
\def\mathclap{\mathpalette\mathclapinternal}
\def\mathclapinternal#1#2{%
        \clap{$\mathsurround=0pt#1{#2}$}%
}
\def\mathrlap{\mathpalette\mathrlapinternal}
\def\mathrlapinternal#1#2{%
        \rlap{$\mathsurround=0pt#1{#2}$}% $
}

%%%
%%% Italic blockquotes (TeX.SE 442547)
%%%

\makeatletter
\patchcmd{\csq@bquote@i}{{#6}}{{\emph{#6}}}{}{}
\makeatother


%%%
%%% siunitx
%%%

\sisetup{output-decimal-marker = {,}}

%%%
%%% PDF information settings
%%%

\hypersetup{pdftitle = {Algoritmos y Complejidad},
            pdfauthor = {Horst H. von Brand},
            pdfcreator = {pdflatex},
            pdfproducer = {LaTeXe with hyperref},
            unicode,
            pdfdisplaydoctitle
           }

%%%
%%% For memoir
%%%

\setsecnumdepth{subsubsection}
\settocdepth{subsection}

\cftsetindents{section}{1.5em}{3.0em}
\cftsetindents{subsection}{4.5em}{3.3em}

\cftsetindents{figure}{0pt}{3.0em}
\cftsetindents{table}{0pt}{3.0em}

% Page layout

\semiisopage
\checkandfixthelayout

% Chapter style

\chapterstyle{ger}
\makeatletter
\renewcommand{\@chapapp}{Clase}
\makeatother

%%%
%%% TikZ stuff
%%%

\usepackage{pgfplots}
\pgfplotsset{compat = 1.18}
\usetikzlibrary{automata, positioning, babel}
\usetikzlibrary{shapes, arrows, arrows.meta}
\usetikzlibrary{fit}
\tikzset{shorten >= 1pt,
         node distance = 1.8cm, on grid,
         >= stealth,
         every initial
           by arrow/.style = -{Straight Barb[length = 1.1ex]},
         initial text = {},
         initial distance = 0.05,
         every state/.style = {draw = navyblue!50,
                               very thick, fill=navyblue!20,
                               minimum size = 5mm},
         bend angle = 35,
         every loop/.style = {looseness = 13}}

\usetikzlibrary{calc, patterns}
     % Permite colocar underbrace en tikz y las flechas curvas
\usetikzlibrary{decorations.pathmorphing}
\usetikzlibrary{decorations.text}

\definecolor{navyblue}{RGB}{0,148,222}

%%%
%%% For listings
%%%

\lstloadlanguages{[60]Algol, [ANSI]C, [ANSI]C++, Python, sh}

\lstset{basicstyle = \sffamily, commentstyle = \slshape,
        frame = lines, numbers = none,
        showstringspaces = false,
        float, floatplacement = ht, captionpos = b,
        xleftmargin = 1em, xrightmargin = 1em
       }
\lstset{literate={á}{{\'a}}1
                 {é}{{\'e}}1
                 {í}{{\'\i}}1
                 {ó}{{\'o}}1
                 {ú}{{\'u}}1
                 {ü}{{\"u}}1
                 {ñ}{{\~n}}1
                 {Á}{{\'A}}1
                 {É}{{\'E}}1
                 {Í}{{\'I}}1
                 {Ó}{{\'O}}1
                 {Ú}{{\'U}}1
                 {Ü}{{\"U}}1
                 {Ñ}{{\~N}}1
                 {¿}{{?`\hspace{-2.5pt}}}1
                 {¡}{{!`\hspace{-2.5pt}}}1
% For Algol
                 {×}{\(\times\)}1
                 {÷}{\(\div\)}1
                 {↑}{\(\uparrow\)}1
                 {≤}{\(\le\)}1
                 {≥}{\(\ge\)}1
                 {≠}{\(\ne\)}1
                 {≡}{\(\equiv\)}1
                 {⊃}{\(\supset\)}1
                 {∧}{\(\wedge\)}1
                 {∨}{\(\vee\)}1
                 {¬}{\(\lnot\)}1
        }

% Pilfered from https://tex.stackexchange.com/a/27648/21275
\renewcommand{\lstlistingname}{Listado}
\renewcommand{\lstlistlistingname}{Índice de listados}
\begingroup\makeatletter
\let\newcounter\@gobble\let\setcounter\@gobbletwo
  \globaldefs\@ne \let\c@loldepth\@ne
  \newlistof{listings}{lol}{\lstlistlistingname}
  \newlistentry{lstlisting}{lol}{0}
\endgroup
\setlength{\cftlstlistingnumwidth}{4em}


%%%
%%% For algorithm2e
%%%

\SetAlgorithmName{Algoritmo}{Algoritmo}{Índice de algoritmos}
\newlistof{listofalgorithms}{loa}{\listalgorithmcfname}
\makeatletter
\renewcommand{\l@algocf}{\@dottedtocline{1}{1em}{3.5em}}
\makeatother

\SetAlCapSty{mdseries}
\SetKwProg{Function}{function}{}{end}
\SetKwProg{Procedure}{procedure}{}{end}
\SetKw{Variables}{variables}
\SetKw{Downto}{downto}
\SetKwBlock{Loop}{loop}{end}
\SetKw{Continue}{continue}
\SetKw{Break}{break}
\SetKw{KwStep}{step}

%%%
%%% AMS-LaTeX theorem stuff
%%%

\theoremstyle{plain}
\newtheorem{theorem}{Teorema}[chapter]
\newtheorem{lemma}[theorem]{Lema}
\newtheorem{corollary}[theorem]{Corolario}
\newtheorem{proposition}{Proposición}[chapter]
\newtheorem{conjecture}{Conjetura}[chapter]
\newtheorem*{axiom}{Axioma}

\theoremstyle{definition}
\newtheorem{definition}{Definición}[chapter]
\newtheorem{example}{Ejemplo}[chapter]

\newtheorem{ejemplo}{Ejemplo}[chapter]
\newtheorem{ejercicio}{Ejercicio}[chapter]

\theoremstyle{remark}
\newtheorem{remark}{Nota}

\newenvironment{solution}[1][Solución]{\begin{trivlist}
\item[\hskip \labelsep {\bfseries #1}]}{\end{trivlist}}

\DeclareMathOperator{\chr}{chr}
\DeclareMathOperator{\lcm}{lcm}
\DeclareMathOperator{\ord}{ord}
\DeclareMathOperator{\sgn}{sgn}
\DeclareMathOperator{\val}{val}

\DeclareMathOperator{\var}{var}

\DeclareMathOperator{\Arg}{Arg}
\DeclareMathOperator{\Log}{Log}
\DeclareMathOperator{\res}{res}
\DeclareMathOperator{\pp}{pp}

\DeclareMathOperator{\inflow}{inflow}
\DeclareMathOperator{\outflow}{outflow}

\DeclareMathOperator{\Seq}{\textsc{Seq}}
\DeclareMathOperator{\Cyc}{\textsc{Cyc}}
\DeclareMathOperator{\Set}{\textsc{Set}}
\DeclareMathOperator{\MSet}{\textsc{MSet}}

\DeclareMathOperator{\ogf}{\stackrel{\text{ogf}}{\longleftrightarrow}}
\DeclareMathOperator{\egf}{\stackrel{\text{egf}}{\longleftrightarrow}}

\DeclareMathOperator{\Exp}{\mathbb{E}}

\newcommand{\cycle}[2]{\genfrac{[}{]}{0pt}{}{#1}{#2}}	 % Stirling 1a especie
\newcommand{\classes}[2]{\genfrac{\{}{\}}{0pt}{}{#1}{#2}} % Stirling 2a especie
\newcommand{\lah}[2]{\genfrac{\lfloor}{\rfloor}{0pt}{}{#1}{#2}}	 % Lah

\newcommand{\multiset}[2]{\left( \!\! \binom{#1}{#2} \!\! \right)}

\newcommand{\Circle}{\raisebox{-1pt}{\scalerel*{\bullet}{\bigodot}}}

%%%
%%% Index, glossary
%%%

\makeindex
\makeglossary

%%%
%%% End of preamble
%%%

\begin{document}
\selectlanguage{spanish}
\bibliographystyle{babplain-fl}

\frontmatter

%% Bastard title page

\thispagestyle{empty}
\vspace*{\fill}
{\fontsize{24}{26}\sffamily\bfseries
  Algoritmos y Complejidad \\
}
\vspace*{\fill}

\cleardoublepage

%% Title page

\thispagestyle{empty}
\vspace*{0.5cm}
\begin{center}
  {
    \fontsize{45}{47}\sffamily\bfseries
    Algoritmos y Complejidad
    \par
  }
  \vspace*{1em}
  {
    \fontsize{35}{37}\sffamily\bfseries
    INF-221
    \par
  }
\end{center}
\vfill
\vfill
\noindent
\begin{center}
  \huge\sffamily
    \href{mailto:vonbrand@inf.utfsm.cl}{Horst H. von Brand}
\end{center}
\vfill
\begin{center}
  \begin{tikzpicture}
    \fill[color = red]
           (45- 12.5:2.5)
              arc[radius = 2.5,
                  start angle = 45 - 12.5, end angle = 135 - 12.5]
           -- (135- 12.5:1.5)
                 arc[radius = 1.5,
                     start angle = 135 - 12.5, end angle = 45 - 12.5]
           -- cycle;
    \fill[color = blue]
           (135- 12.5:2.5)
              arc[radius = 2.5,
                  start angle = 135 - 12.5, end angle = 225 - 12.5]
           -- (225- 12.5:1.5)
                 arc[radius = 1.5,
                     start angle = 225 - 12.5, end angle = 135 - 12.5]
           -- cycle;
    \fill[color = yellow!50!red]
           (225 - 12.5:2.5)
              arc[radius = 2.5,
                  start angle = 225 - 12.5, end angle = 315 - 12.5]
           -- (315- 12.5:1.5)
                 arc[radius = 1.5,
                     start angle = 315 - 12.5, end angle = 225 - 12.5]
           -- cycle;
    \fill[color = green]
           (315 - 12.5:2.5)
              arc[radius = 2.5,
                  start angle = 315 - 12.5, end angle = 405 - 12.5]
           -- (405- 12.5:1.5)
                 arc[radius = 1.5,
                     start angle = 405 -12.5, end angle = 315 - 12.5]
           -- cycle;

    \fill[color = red]
           (135 - 12.5:1.25) -- (135 - 12.5:2.75) -- (135 + 25 - 12.5:2.05)
              -- cycle;
    \fill[color = blue]
           (225 - 12.5:1.25) -- (225 - 12.5:2.75) -- (225 + 25 - 12.5:2.05)
              -- cycle;
    \fill[color = yellow!50!red]
           (315 - 12.5:1.25) -- (315 - 12.5:2.75) -- (315 + 25 - 12.5:2.05)
              -- cycle;
    \fill[color = green]
           (405 - 12.5:1.25) -- (405 - 12.5:2.75) -- (405 + 25 - 12.5:2.05)
              -- cycle;

    \path[postaction = {decorate,
          decoration = {text along path,
                        text align = center,
                        text = {|\color{white}\sffamily\Large|Dise{\~n}o}}}]
          (135:1.75) arc[radius = 1.75, start angle = 135, end angle = 45];
    \path[postaction = {decorate,
          decoration = {text along path,
                        text align = center,
                        text = {|\color{white}\sffamily\Large|An{\'a}lisis}}}]
          (225:1.75) arc[radius = 1.75, start angle = 225, end angle = 135];
    \path[postaction = {decorate,
          decoration = {text along path,
                        text align = center,
                        text = {|\color{white}\sffamily\Large|Programa}}}]
          (225:2.1) arc[radius = 2.1, start angle = 225, end angle = 315];
    \path[postaction = {decorate,
          decoration = {text along path,
                        text align = center,
                        text = {|\color{white}\sffamily\Large|Medir}}}]
          (315:2.1) arc[radius = 2.1, start angle = 315, end angle = 405];
  \end{tikzpicture}
  \vspace*{5em}

  \today
  \vspace*{2\baselineskip}

  {\large\sffamily
   Departamento de Informática\\
   Universidad Técnica Federico Santa María
  }
\end{center}

\clearpage

%% Copyright page
\input{class-version}

\thispagestyle{empty}
\vspace*{2ex}
\begin{small}
  \noindent
  \textcopyright\,2016-\the\year\ Horst H. von Brand \\
  Todos los derechos reservados.\\[1.2em]
  Compuesto por el autor en \LaTeXe{} con Utopia para textos
  y Fourier-GUTenberg para matemáticas.\\[2em]
  Versión {\classversion} del {\classdate} \\[0.7em]
  Se autoriza el uso de esta versión preliminar
  para cualquier fin educacional
  en una institución de enseñanza superior,
  en cuyo caso solo se permite
  el cobro de una tarifa razonable de reproducción.
  Se prohíbe todo uso comercial.
\end{small}

\cleardoublepage

%% Dedication page

\thispagestyle{empty}
\vspace*{\fill}
\vspace*{\fill}
\noindent
  Agradezco a mi familia,
  a quienes he descuidado
  durante el desarrollo del presente texto.

\noindent
  Agradezco a mis estudiantes,
  quienes sufrieron versiones preliminares.

\vspace*{\baselineskip}
\noindent
  El Departamento de Informática
  de la Universidad Técnica Federico Santa María
  provee el ambiente ideal de trabajo.

\vspace*{\baselineskip}
\noindent
  Este documento presenta
  la materia del curso \textquote{Algoritmos y Complejidad} (INF-221)
  como dictado en la Casa Central de Universidad Técnica Federico Santa María
  por el autor.
  Se iniciaron con el esfuerzo de Aldo Berríos,
  quien se dio el trabajo de transcribir las
  (a veces muy desorganizadas)
  clases a \LaTeX,
  ligeramente editadas por el autor,
  incorporando comentarios y correcciones de estudiantes,
  y llevadas al formato actual.
  El resultado fue creciendo con extensiones posteriores
  y extensas reorganizaciones.
\vspace*{\baselineskip}
\noindent
  Especial reconocimiento a Claudio Torres,
  quien leyó y comentó el apunte,
  particularmente la primera parte.
\vspace*{\fill}

\cleardoublepage

\tableofcontents

\clearpage

\listoffigures

\clearpage

\listoftables

\clearpage

\lstlistoflistings

\clearpage

\listofalgorithms

\cleardoublepage

\thispagestyle{plain}
%\include{prefacio}

\mainmatter

\part{Análisis Numérico}
\thispagestyle{empty}

\bibliographystyle{babplain-fl}

\chapter{Análisis numérico}
\label{cha:analisis-numerico}

  El \emph{análisis numérico} trata de métodos computacionales
  para obtener valores numéricos precisos para una variedad de objetos,
  como funciones, integrales, soluciones de ecuaciones y sistemas de estas.
  Adoptamos la definición de Trefethen~%
    \cite{trefethen92:_definition_numerical_analysis},
  que define el área como \emph{estudio de algoritmos para los problemas
  de las matemáticas continuas}.
  Una de las mayores revoluciones se inició en el siglo XVII
  al desarrollarse el cálculo,
  que con la física llevó a modelos precisos de muchos fenómenos de interés,
  que luego se extendieron a otras ciencias.
  Estos modelos rara vez se pueden resolver en forma exacta,
  hay que recurrir a técnicas aproximadas.
  Incluso se da que es más eficiente calcular una solución aproximada
  que usar una engorrosa fórmula exacta.
  El mismo Newton inventó métodos numéricos,
  desarrollos de los cuales hoy llevan su nombre.
  Algunos puntos de interés recientes son los algoritmos CORDIC~%
    \cite{volder59:_cordic_computing_techninque,
          volder59:_cordic_trigon_comput_techn}
  y afines para calcular rápidamente funciones trigonométricas e hiperbólicas
  dígito a dígito si la multiplicación es lenta,
  y el truco~%
    \cite{lomont03:_fast_inverse_square_root}
  para aproximar el inverso de la raíz cuadrada
  basado en la representación de números en punto flotante de IEEE,
  usado por primera vez en el juego Quake~IV.

  Esta es un área enorme,
  en este curso cubriremos solo algunas áreas específicas
  de interés más bien general.
  Un texto general es el de Gautschi~%
    \cite{gautschi12:_numerical_analysis},
  otro buen compendio es el texto de Sauer~%
    \cite{sauer11:_numerical_analysis},
  una referencia útil es el texto de Kincaid y Cheney~%
    \cite{kincaid02:_numerical_analysis}.
  Hay numerosas colecciones de notas de clase disponibles,
  como la de Olver~%
    \cite{olver08:_ln_numerical_analysis},
  de Cowley~%
    \cite{cowley14:_numerical_analysis}
  o de Philip~%
    \cite{philip22:_num_math_i,philip22:_num_math_ii}.
  Es recomendable el texto de Heinhold~%
    \cite{heinhold13:_intuitive_guide_numer_methods},
  que se centra en una visión intuitiva
  y desde el usuario de distintos métodos.
  Un texto que ilustra muchos de los algoritmos más importantes
  en Python~3
  es el de Kiusalaas~%
    \cite{kiusalaas13:_numerical_meth_python3},
  otro es el de Liu~%
    \cite{liu20:_first_semes_num_anal_python}.
  El texto de Máté~%
    \cite{mate04:_intro_numerical_analysis_c}
  discute la teoría y muestra cómo programar las técnicas en C.
  El texto de Acton~%
    \cite{acton90:_numer_methods_usual_work}
  es bastante antiguo
  (una reseña indica que muestra claramente su origen en épocas
   de cálculo manual y computadores rudimentarios),
  pero da una visión desde las trincheras en lenguaje coloquial.
  El texto de Muller et al~%
    \cite{muller18:_handb_float_point_arith}
  da una visión muy detallada de aritmética en punto flotante
  y cómo calcular con precisión un amplio rango de funciones de interés.

  Para obtener valores numéricos de interés en una situación práctica
  se debe construir un modelo de la situación,
  extraer de él las relaciones relevantes,
  y resolver las ecuaciones resultantes para obtener el resultado.
  Hay varias fuentes de error en esto:
  \begin{enumerate}[label={(\roman*)}]
  \item
    \label{enum:error:fisico}
    El modelo físico es una simplificación de la realidad
    (se omite la fricción del aire,
     suponer que la aceleración de la gravedad es constante,
     \ldots)
   \item
    \label{enum:error:parametros}
     Los parámetros que entran al modelo se conocen con precisión finita
   \item
    \label{enum:error:aproximacion}
     Aproximaciones usadas para resolver el modelo matemático
   \item
    \label{enum:error:redondeo}
     Errores de redondeo en los cálculos
  \end{enumerate}
  El punto~\ref{enum:error:fisico} es tema de modelamiento matemático,
  no nos concierne acá.
  De~\ref{enum:error:parametros} se preocupa quien monta el experimento.
  El punto~\ref{enum:error:redondeo} tiene que ver con la representación
  de infinitos números reales en espacio finito,
  que trataremos someramente más adelante
  (sección~\ref{sec:floating-point}).
  El punto~\ref{enum:error:aproximacion} es el tema central de interés
  para nosotros.

  Hay dos formas principales de cuantificar el error:
  \begin{definition}
    \label{def:error-computacional}
    Sea \(x\) un valor real,
    y \(x^*\) una aproximación.
    El \emph{error absoluto} de la aproximación \(x^* \approx x\)
    es \(\lvert x^* - x \rvert\),
    el \emph{error relativo} (siempre que \(x \ne 0\))
    es \(\lvert x^* - x \rvert / \lvert x \rvert\).
  \end{definition}
  Por ejemplo,
  \num{1000} es una aproximación de \num{1024} con error absoluto de \num{24}
  y error relativo de \num{0,023}.
  Diremos que la aproximación
  \emph{tiene \(k\) dígitos decimales significativos}
  si el error relativo es menor que \(10^{-k + 1}\),
  o sea,
  después del primer dígito decimal no cero hay \(k\) dígitos correctos.
  En nuestro caso,
  es \(k = 2\)
  (porque \(0,023 < 10^{-1}\)),
  con lo que \num{1\,000} es una aproximación de \num{2}~cifras significativas
  a \num{1\,024}.

  La misma idea puede aplicarse si estamos hablando de vectores,
  solo que en tal caso usaremos normas,
  \(\lVert \mathbf{x}^* - \mathbf{x} \rVert\) es el error absoluto
  y \(\lVert \mathbf{x}^* - \mathbf{x} \rVert / \lVert \mathbf{x} \rVert\)
  el relativo.

\section{Propiedades de punto flotante}
\label{sec:floating-point}

  Hay que tener presente que los cálculos en punto flotante
  \emph{no} cumplen las conocidas leyes.
  Por ejemplo,
  exactamente:
  \begin{align*}
    \frac{301}{4000} - \frac{300}{4001}
      &=       \frac{301 \cdot 4001 - 300 \cdot 4000}{4000 \cdot 4001} \\
      &=       \frac{4301}{16004000} \\
      &\approx 0,0002687453
  \end{align*}
  Si hacemos los cálculos intermedios con \num{3} cifras,
  obtenemos:
  \begin{align*}
    \frac{301}{4000}
      &= 0,0753 \\
    \frac{300}{4001}
      &= 0,0750 \\
    \frac{301}{4000} - \frac{300}{4001}
      &= 0,0003
  \end{align*}
  El resultado no tiene ni una sola cifra correcta.
  Escribiendo:
  \begin{align*}
    \frac{301}{4000} - \frac{300}{4001}
      &= \frac{301 \cdot 4001 - 300 \cdot 4000}{4000 \cdot 4001} \\
      &= \frac{1,20 \cdot 10^6 - 1,20 \cdot 10^6}{1,60 \cdot 10^7} \\
      &= 0
  \end{align*}
  Esto claramente es incorrecto.

  Los cálculos en punto flotante son aproximaciones de números reales.
  Los formatos y operaciones se han definido
  (por el estándar IEEE~745~%
    \cite{IEEE19:_754_floating_point},
   también estándar internacional~%
    \cite{ISO11:_floating_point})
   de forma que las operaciones cumplen una regla simple:
   \begin{equation}
     \label{eq:fp-def}
     \mathrm{fl}(x)
       = x (1 + \delta)
       \quad  \lvert \delta \rvert \le \varepsilon
   \end{equation}
   Acá \(\varepsilon\) describe la precisión de los valores,
   se le conoce como \emph{epsilon de la máquina}.
   En C este valor se define en el encabezado
   \lstinline[language = C]!float.h!
   para los distintos tipos estándar de punto flotante
   (\lstinline[language = C]!FLT_EPSILON! para \lstinline[language = C]!float!,
    \lstinline[language = C]!DBL_EPSILON!
      para \lstinline[language = C]!double!,
    \lstinline[language = C]!LDBL_EPSILON!
     para \lstinline[language = C]!long double!),
   en C++ el encabezado \lstinline[language = C++]!limits!
   define \lstinline[language = C++]!std::numeric_limits<T>::epsilon()!
   para los diferentes tipos \lstinline[language = C++]!T!.
   Otros lenguajes,
   o bibliotecas que ofrezcan precisiones no estándar,
   documentarán el valor alguna forma.

   Con estas cotas,
   tenemos por ejemplo:
   \begin{align*}
     \mathrm{fl}(\operatorname{fl}(x) \cdot \operatorname{fl}(y))
       &= \mathrm{fl}(x (1 + \delta_x) \cdot y (1 + \delta_y)) \\
       &= (x (1 + \delta_x) \cdot y (1 + \delta_y)) (1 + \delta_{x y}) \\
       &= x y
           (1 + \delta_x + \delta_y + \delta_{x y}
              + \delta_x \delta_y
                  + \delta_x \delta_{x y}
                  + \delta_y \delta_{x y}
              + \delta_x \delta_y \delta_{x y})
   \end{align*}
   El error relativo cumple:
   \begin{align*}
     \delta_x + \delta_y + \delta_{x y}
        + \delta_x \delta_y
            + \delta_x \delta_{x y}
            + \delta_y \delta_{x y}
        + \delta_x \delta_y \delta_{x y}
        \le 3 \varepsilon + 3 \varepsilon^2 + \varepsilon^3
   \end{align*}
   O sea,
   aproximadamente \(3 \varepsilon\).
   Un desarrollo similar vale para división.

   El caso de sumas y restas es diferente:
   \begin{align*}
     \mathrm{fl}(\operatorname{fl}(x) + \operatorname{fl}(y))
       &= \mathrm{fl}(x (1 + \delta_x) + y (1 + \delta_y)) \\
       &= (x (1 + \delta_x) + y (1 + \delta_y)) (1 + \delta_{x + y}) \\
       &= x + y
            + x (\delta_x + \delta_{x + y} + \delta_x \delta_{x + y})
            + y (\delta_y + \delta_{x + y} + \delta_y \delta_{x + y})
   \end{align*}
   El error absoluto cumple:
   \begin{equation*}
     x (\delta_x + \delta_{x + y} + \delta_x \delta_{x + y})
       + y (\delta_y + \delta_{x + y} + \delta_y \delta_{x + y})
       \le (\lvert x \rvert + \lvert y \rvert) (2 \varepsilon + \varepsilon^2)
   \end{equation*}
   Si ambos operandos son del mismo signo,
   el error relativo es aproximadamente \(2 \varepsilon\);
   si tienen signo opuesto puede producirse cancelación catastrófica,
   para \(\lvert x \rvert \le \lvert y \rvert\)
   el error relativo está acotado por:
   \begin{equation*}
     \frac{\lvert x \rvert + \lvert y \rvert}
          {\lvert y \rvert - \lvert x \rvert}
       \cdot 2 \varepsilon
   \end{equation*}
   donde el primer factor es arbitrariamente grande.

   Una discusión más detallada ofrece Haberman~%
     \cite{haberman14:_float_point_demystified_part1,
           haberman16:_float_point_demystified_part2}
   o el clásico texto de Goldberg~%
     \cite{goldberg91:_what_every_cs_should_know_fp}.

\section{Propagación de errores}
\label{sec:propagacion-errores}

  Considere calcular el valor \(y = f(x)\).
  Solo podemos calcular una aproximación \(y^*\),
  hay dos maneras de cuantificar el error asociado a esta aproximación.
  Para simplificar,
  supondremos que la función \(f\) es conocida,
  por lo discutido antes es común que solo se conozca en forma aproximada.

\subsection{Hacia adelante}
\label{sec:error-adelante}

  En inglés,
  se habla de \emph{\foreignlanguage{english}{forward error}}.
  Es una medida de la diferencia entre la aproximación \(y^*\)
  y el valor correcto \(y\),
  ya sea absoluto o relativo.
  Como no conocemos \(y\),
  solo podemos obtener una cota superior.
  Suele ser difícil obtener cotas ajustadas.

  Una técnica es aproximar la función por la serie de Taylor
  centrada en \(x\),
  o sea usar:
  \begin{align*}
    f(x^*)
      = f(x) + f'(x) (x^* - x) + \frac{1}{2} f''(x) (x^* - x)^2 + \dotsb
  \end{align*}
  Suponiendo que el error \(\Delta x = x^* - x\) es de pequeña magnitud,
  obtenemos una aproximación razonable truncando luego del término lineal
  (considerar términos de grado mayor
   complica las cosas,
   para \(\Delta x\) pequeño no aporta mucho).
  Además formalmente solo podemos acotar la magnitud del error,
  no conocemos su signo.
  Una cota aproximada para el error en \(y\) es entonces:
  \begin{equation*}
    \lvert y^* - y \rvert
      \approx \lvert f'(x) \rvert \cdot \lvert \Delta x \rvert
  \end{equation*}
  En caso que hayan más datos de entrada,
  se usa la serie de Taylor en múltiples variables,
  conservando los términos lineales únicamente.

\subsection{Hacia atrás}
\label{sec:error-atras}

  En inglés,
  \emph{\foreignlanguage{english}{backward error}}.
  Acá la pregunta es,
  conozco \(y^*\),
  que es respuesta a algún problema \(f(x^*)\).
  Específicamente,
  nos interesa el mínimo \(\Delta x\) tal que:
  \begin{equation*}
    y^*
      = f(x^* + \Delta x)
  \end{equation*}
  A tal \(\lvert \Delta x \rvert\)
  (o \(\lvert \Delta x \rvert / \lvert x^* \rvert\))
  se le llama el error hacia atrás.
  Suele ser más fácil de calcular,
  y obtener cotas ajustadas.

  Por ejemplo,
  si nos interesa \(y = \sqrt{2}\) y tenemos el valor aproximado \(y^* = 1,4\),
  el error es:
  \begin{description}
  \item[Hacia adelante:]
    \(\lvert \Delta y \rvert
        = \lvert 1,4 - 1,41421\ldots \rvert
        \approx 0,0121\)
  \item[Hacia atrás:]
    \(\lvert \Delta x \rvert
        = \lvert (1,4)^2 - 2 \rvert
        = 0,04\)
  \end{description}

\section{Estabilidad y condicionamiento}
\label{sec:estabilidad-condicionamiento}

  Consideremos el problema de valor inicial:
  \begin{equation*}
    u'(t) - 2 u(t)
      = - \mathrm{e}^{-t}
      \quad u(0) = \frac{1}{3}
  \end{equation*}
  La solución es un simple ejercicio:
  \begin{equation*}
    u(t)
      = \frac{1}{3} \mathrm{e}^{-t}
  \end{equation*}
  Esto decae exponencialmente cuando \(t \to \infty\).

  En un computador,
  con precisión finita en binario,
  no podemos representar \(1/3\) en forma exacta,
  en el mejor caso estamos resolviendo algo como:
  \begin{equation*}
    v' - 2 v
      = - \mathrm{e}^{-t}
      \quad v(0) = \frac{1}{3} + \varepsilon
  \end{equation*}
  donde \(\varepsilon\) es el error en la representación de \(1/3\).
  Su solución es:
  \begin{equation*}
    v(t)
      = \frac{1}{3} \mathrm{e}^{-t} + \varepsilon \mathrm{e}^{2 t}
  \end{equation*}
  Esta solución crece exponencialmente,
  el error cometido solo en el valor inicial
  pronto abruma la verdadera solución.
  Note que estamos resolviendo el problema en forma exacta,
  el error es inherente al problema,
  no a nuestra técnica de solución.

  Vale decir,
  hay problemas cuyas soluciones son muy sensibles a los datos de entrada,
  se les llama \emph{mal condicionados}
  (en inglés \emph{\foreignlanguage{english}{ill-conditioned}}).
  Podemos considerar un problema como una función \(f(x)\)
  de un dato de entrada,
  suele cuantificarse la condición del problema \(y = f(x)\) mediante
  el \emph{número de condición},
  el error relativo de \(y\) dividido por el de \(x\):
  \begin{align*}
    \frac{\frac{\lvert \Delta y \rvert}{\lvert y \rvert}}
         {\frac{\lvert \Delta x \rvert}{\lvert x \rvert}}
      &=       \frac{\lvert \Delta y \rvert}{\lvert \Delta x \rvert}
                 \cdot \left\lvert \frac{x}{y} \right\rvert \\
      &\approx \left\lvert \frac{x f'(x)}{f(x)} \right\rvert
  \end{align*}
  Si el número de condición es alto,
  el problema es mal condicionado.
  Definiciones similares se aplican cuando hay más datos de entrada.

  Un ejemplo simple de este tipo de fenómeno se da
  al calcular \(f(x) = \ln x\)
  para \(x\) cercano a \num{1}.
  El número de condición para este problema puede aproximarse por:
  \begin{align*}
    \frac{\lvert x f'(x) \rvert}{\lvert f(x) \rvert}
      &=  \frac{\lvert x \cdot 1 / x \rvert}{\lvert \ln x \rvert} \\
      &=  \frac{1}{\lvert \ln x \rvert}
  \end{align*}
  Para \(x\) cercano a \num{1},
  \(\ln x\) es cercano a \num{0},
  el problema es mal condicionado.
  Para ayudar a evitar este problema,
  POSIX especifica la función \lstinline[language = C]!log1p(3)!,
  que retorna el logaritmo natural de \(1 + x\).

  Wilkinson~%
    \cite{wilkinson59:_ill_conditioned-_poly-I,
          wilkinson59:_ill_conditioned-_poly-II}
  al probar un conjunto de rutinas de punto flotante
  para un computador temprano
  por casualidad se tropezó con el fenómeno que discutiremos.
  Da un resumen divertido en~%
    \cite{wilkinson84:_perfidious_polynomial}.
  Calculó los coeficientes de un polinomio con ceros conocidos,
  y luego calculó numéricamente los ceros del polinomio resultante.
  Obtuvo resultados muy extraños,
  lo que lo llevó a una semana de búsqueda de errores en sus rutinas,
  sin hallar nada.
  Finalmente cayó en cuenta que mínimas diferencias en los coeficientes
  (como los introducidos al redondear)
  pueden tener gran efecto sobre los ceros.

  Para ilustrar el fenómeno,
  consideremos el polinomio:
  \begin{equation*}
    p(x)
      = (x - 1) (x - 2) \dotsm (x - 10)
  \end{equation*}
  Conocemos sus ceros en forma exacta.
  Introduciremos una pequeña perturbación en un solo término.
  Cabe esperar que los ceros del polinomio:
  \begin{equation*}
    q(x)
      = p(x) + x^5
  \end{equation*}
  sean cercanos,
  la diferencia está en los términos
  \(- 902055 x^5\) en \(p\)
  y \(- 902054 x^5\) en \(q\),
  una diferencia de \(0,0001\%\) en un coeficiente.
  Pero los ceros de \(q(x)\) son:
  \begin{align*}
    &1,0000027558, \quad 1,99921, \quad 3,02591, \quad 3,82275, \\
    &5,24676 \pm 0,751485 \mathrm{i}, \quad
       7,57271 \pm 1,11728 \mathrm{i}, \quad
       9,75659 \pm 0,368389 \mathrm{i}
  \end{align*}
  El menor cero es cercano,
  los demás se alejan crecientemente
  y los últimos seis mutan a complejos conjugados.
  Cerca de un cero de \(p\),
  \(q(x) \approx x^5\),
  que para \(x\) grande es muy grande.

  Otro fenómeno se produce cuando errores intermedios del algoritmo
  se amplifican,
  posiblemente abrumando el resultado.
  Esta situación se llama \emph{inestabilidad}.
  Un ejemplo es calcular la integral:
  \begin{equation}
    \label{eq:integral-recurrencia}
    I_n
      = \int_0^1 x^n \mathrm{e}^{x - 1} \, \mathrm{d} x
  \end{equation}
  Por integración por partes obtenemos la recurrencia:
  \begin{equation*}
    I_n
      = 1 - n I_{n - 1}
      \quad I_0 = 1 - \mathrm{e^{-1}}
  \end{equation*}
  Esta es una forma exacta y eficiente de calcular \(I_n\),
  sin embargo si se efectúa con \num{6} cifras
  el valor calculado de \(I_9\) es negativo.

  Note que estas dos situaciones son diferentes,
  el condicionamiento es inherente al problema,
  la estabilidad es una característica del algoritmo.
  Obtener una solución a un problema mal condicionado será difícil,
  incluso con un algoritmo estable.

\section*{Ejercicios}
\label{sec:ejercicios-01previa}

  \begin{enumerate}
  \item
    En la fórmula tradicional para los ceros de la función cuadrática:
    \begin{equation*}
      a x^2 + b x + c
    \end{equation*}
    \begin{equation*}
      x_1, x_2
        = \frac{-b \pm \sqrt{b^2 - 4 a c}}{2 a}
    \end{equation*}
    si \(b^2\) es mucho mayor que \(4 a c\),
    en uno de los casos se restan términos parecidos,
    y el resultado es muy poco preciso.
    Proponga una técnica para obtener valores precisos de ambos ceros
    usando la fórmula de Vieta,
    \(x_1 x_2 = c / a\).
    Considere además los diferentes casos especiales que se producen,
    si \(a \approx 0\) o \(c \approx 0\).
    Prográmela en Python.
  \item
    Calcule el valor de \(\mathrm{e}^{-5.3}\),
    usando \num{4} cifras significativas en los valores intermedios:
    \begin{enumerate}
    \item
      Usando directamente la serie de Maclaurin
    \item
      Mediante la identidad:
      \begin{equation*}
        \mathrm{e}^{-5,3}
          = \frac{1}{\mathrm{e}^{5,3}}
      \end{equation*}
      y calculando la exponencial mediante la serie
    \end{enumerate}
    Compare con el valor exacto \num{0,00499159390691021621}.
  \item
    Complete el ejemplo de algoritmo inestable
    de cálculo de la integral~\eqref{eq:integral-recurrencia}
    efectuando los cálculos indicados.
    Analice la estabilidad de la iteración.
  \item
    Analice la iteración:
    \begin{equation*}
      x_{n + 2}
        = a x_{n + 1} + b x_n
    \end{equation*}
    desde el punto de vista de estabilidad,
    para distintos valores de \(a\) y \(b\).
  \end{enumerate}

\bibliography{../referencias}

%%% Local Variables:
%%% mode: latex
%%% TeX-master: "../INF-221_notas"
%%% ispell-local-dictionary: "spanish"
%%% End:

% LocalWords:  XVII CORDIC IEEE Quake IV epsilon modelamiento C ill
% LocalWords:  operandos english forward backward conditioned POSIX
% LocalWords:  crecientemente recurrencia Prográmela et fl eq

\bibliographystyle{babplain-fl}

\chapter{Encontrar ceros de funciones}
\label{cha:ceros-funciones}

  Se habla de encontrar la \emph{raíz} de una ecuación:
  \begin{equation*}
    f(x)
      = 0
  \end{equation*}
  mientras se busca un \emph{cero} de la función \(f(x)\).
  El problema de hallar ceros de funciones es muy común.
  Lamentablemente solo en casos muy especiales hay fórmulas exactas,
  la mayor parte de las veces
  debe recurrirse a hallar buenas aproximaciones numéricas.
  Un buen resumen y ejemplos da Wright~%
    \cite{wright04:_nonlinear_root_finding}.
  Incluso se da el caso que hay soluciones exactas,
  pero la fórmula del caso es engorrosa
  (o tiene malas propiedades numéricas),
  resulta más cómodo
  e incluso más exacto usar una técnica iterativa.
  Una discusión detallada de algunas técnicas se hallan en las notas
  de LeMesurier y Roberts~%
    \cite{lemesurier97:_b15_numerical_analysis}.

  El problema es dada \(f(x)\),
  hallar \(x^*\) tal que \(f(x^*) = 0\)
  (\(f\) debe ser continua).
  Por ejemplo:
  ¿para qué valor de \(x\) se cumple \(x = e^{-x}\)?
  Una idea para resolver este problema es simplemente graficar la función.
  \begin{equation}
    f(x)
      = x - e^{-x}
  \end{equation}
  y ver cuáles son los valores de \(x^*\) tal que \(f(x^*) = 0\).
  \begin{figure}[ht]
    \centering
    \begin{tikzpicture}
      \begin{axis}[axis x line = middle, axis y line = left,
                   xlabel = {\(x\)}, ylabel ={\(f(x) = x - \mathrm{e}^{-x}\)},
                   xmin = 0, xmax = 2.3,
                   ymin = -1, ymax = 2.3]

        \addplot [mark = none, red] {x - exp(-x)};
      \end{axis}
    \end{tikzpicture}
    \caption{Gráfica de la función \(x - \mathrm{e}^{-x}\)}
    \label{fig:plot-f}
  \end{figure}
  En la figura~\ref{fig:plot-f} se aprecia un cero cerca de \num{0,6}.

  Otra forma de solucionar el problema
  puede ser graficar \(x\), \(e^{-x}\),
  para buscar las intersecciones
  (figura~\ref{fig:plot-intersect}).
  \begin{figure}[ht]
    \centering
    \begin{tikzpicture}
      \begin{axis}[axis lines = left,
                   xlabel = {\(x\)}, ylabel = {\(y\)},
                   xmin = 0, xmax = 1.1,
                   ymin = 0, ymax = 1.1]

        \addplot [mark = none, domain = 0:1, blue] {x};
        \addplot [mark = none, domain = 0:1, red] {exp(-x)};

        \draw[dashed, blue]
          (axis cs: 0.567143, 0) -- (axis cs: 0.567143, 0.567143);
      \end{axis}
    \end{tikzpicture}
    \caption{Gráficas de \(x\) y \(\mathrm{e}^{-x}\)}
    \label{fig:plot-intersect}
  \end{figure}

\section{Métodos que acotan el tramo}

  Corresponden a métodos para encontrar ceros de funciones
  los cuales usan el \emph{teorema del valor intermedio}
  y que básicamente van encerrando la solución
  hasta acotarla a un tramo suficientemente corto
  (en inglés, \emph{\foreignlanguage{english}{bracketing}}).
  Consideraremos dos métodos de este tipo:
  el método de la bisección y \foreignlanguage{latin}{\emph{regula falsi}}.

\subsection{Método de la Bisección}

  Si tenemos \(x_0, x_1\) tales que \(f(x_0)\cdot f(x_1) < 0\),
  hay un cero de \(f\) en \(\left[ x_0, x_1 \right]\),
  elegimos
  \begin{equation}
    x_2
      = \frac{x_0 + x_1}{2}
  \end{equation}
  con \(\left[ x_0, x_2\right]\) o \(\left[ x_2, x_1\right]\)
  según el cual tenga valores de \(f\) de signo distinto.
  \begin{figure}[ht]
    \centering
    \begin{tikzpicture}
      \begin{axis}[
                    axis lines = center,
                    xlabel = \(x\),
                    ylabel = {\(y = f(x)\)},
                    every axis y
                      label/.style = {at = (current axis.above origin),
                        anchor = south},
                    every axis x
                      label/.style = {at = (current axis.right of origin),
                        anchor = west},
                    xmin = 0,  xmax = 2,
                    ymin = -2, ymax = 2,
                    xtick = {0.3, 0.85, 1.4},
                    xticklabels = {\(x_0\), \(x_2\), \(x_1\)},
                    ytick = {0.8, -0.6},
                    yticklabels = {\(f(x_0)\), \(f(x_1)\)},
                    height = 6cm, width = 9cm
                  ]
        \addplot [samples = 100, smooth, tension = 0.8, red]
          coordinates {(0.3, 0.8) (0.5, 1.7) (0.8, 0.4) (1.1, 0.9)
                       (1.3, -1) (1.4, -0.6)};

        \draw [blue, dashed] (axis cs:0.3, 0) -- (axis cs:0.3, 0.8);
        \draw [blue, dashed] (axis cs:1.4, 0) -- (axis cs:1.4, -0.6);

        \draw [blue, dashed] (axis cs:0, -0.6) -- (axis cs:1.4, -0.6);
        \draw [blue, dashed] (axis cs:0, 0.8) -- (axis cs:0.3, 0.8);
      \end{axis}
    \end{tikzpicture}
    \caption{Si \(f(x_0) \cdot f(x_1) < 0\),
             hay \(x^* \in [x_0, x_1]\)
             donde \(f(x^*) = 0\).}
    \label{01::BiseccionEjemplo1}
  \end{figure}
  Por ejemplo,
  consideremos la figura~\ref{01::BiseccionEjemplo1}.
  Podemos apreciar
  que el intervalo \(\left[x_2, x_1\right]\)
  cumple con \(f(x_2)\cdot f(x_1)<0\).
  En consecuencia,
  por el teorema del valor intermedio
  sabemos que \(x^*\in\left[x_2, x_1\right]\).
  Luego,
  para aproximar \(x^*\)
  obtenemos el valor medio del intervalo \(\left[x_2, x_1\right]\)
  y repetimos el proceso.

  Note que la gracia de todo esto es encontrar solo \emph{un} cero,
  ¡no todos!
  Esto quiere decir lo ideal
  es escoger intervalo \([a, b]\) tal que solo tenga \emph{un} cero.
  Si nuestra función solo toca el eje \(X\)
  (vale decir,
   tiene un cero de multiplicidad par)
  esto claramente no sirve.
  Si hay varios ceros en el intervalo,
  convergerá a uno de ellos.
  \begin{figure}[ht]
    \centering
    \begin{tikzpicture}
      \begin{axis}[
                    axis lines = center,
                    xlabel = \(x\),
                    ylabel = {\(y = f(x)\)},
                    every axis y
                      label/.style = {at = (current axis.above origin),
                        anchor = south},
                    every axis x
                      label/.style = {at = (current axis.right of origin),
                        anchor = west},
                    xmin = 0,  xmax = 2,
                    ymin = -2, ymax = 2,
                    xtick = {0.3, 0.85, 1.4},
                    xticklabels = {\(x_0\), \(x_2\), \(x_1\)},
                    ytick = {0},
                    yticklabels = {},
                    height = 6cm, width = 9cm
                  ]
        \addplot [samples = 100, smooth, tension = 0.8, red]
          coordinates {(0.3, 0.8) (0.5, 1.7) (0.8, -0.9) (1.1, 0.9)
                       (1.3, -1) (1.4, -0.6)};

        \draw [blue, dashed] (axis cs:0.3, 0) -- (axis cs:0.3, 0.8);
        \draw [blue, dashed] (axis cs:1.4, 0) -- (axis cs:1.4, -0.6);
      \end{axis}
    \end{tikzpicture}
    \caption{Una función con tres ceros.}
    \label{01::BiseccionEjemplo2}
  \end{figure}
  Por lo tanto,
  el método de la bisección no abarca todos los casos posibles.

\subsection{Método \foreignlanguage{latin}{\emph{Regula Falsi}}}

  Siempre podemos usar la antigua idea
  (se traza a la antigua Babilonia)
  de tomar dos puntos que acoten un cero
  en la curva
  e interpolar linealmente para obtener una aproximación mejor,
  vea la figura~\ref{01::RegularFalsi:grafico}.
  Esto da:
  \begin{equation}
    \label{eq:regula-falsi-start}
    x_2
      = x_1 - f(x_1) \cdot \frac{x_1 - x_0}{f(x_1) - f(x_0)}
  \end{equation}
  \begin{figure}[ht]
    \centering
    % The function interpolates:
    % 0.2   0.9
    % 0.3   0.7
    % 0.9   0.0
    % 1.7  -0.3
    % 1.9  -0.35
    % see regula-falsi.mc
    \begin{tikzpicture}
      \begin{axis}[
                    axis lines = center,
                    xlabel = {\(x\)},
                    ylabel = {\(y\)},
                    every axis y
                      label/.style = {at = (current axis.above origin),
                        anchor = south},
                    every axis x
                      label/.style = {at = (current axis.right of origin),
                        anchor = west},
                    xmin = 0,	 xmax = 2,
                    ymin = -0.6, ymax = 1,
                    xtick = {3/10},
                    xticklabels = {\(x_0\)},
                    extra x ticks = {32/25, 17/10},
                    extra x tick labels = {\(x_2\), \(x_1\)},
                    extra x tick style = {
                      xticklabel style = {yshift = 0.5ex, anchor = south}
                    },
                    ytick = {0},
                    yticklabels = {},
                    height = 6cm, width = 9cm
                  ]
        \addplot [red, smooth, domain = 0.2:1.9]
          {(19000 * x^4 - 154100 * x^3 + 460190 * x^2 - 659011 * x + 320229)
             / 228480};
        \addplot [blue, samples = 100, domain = 0.3:1.7]
          {(32 - 25 * x) / 35};

        \draw [dashed, blue]
           (axis cs:3/10, 0) -- (axis cs:3/10, 7/10);
        \draw [dashed, blue]
           (axis cs:32/25, 0) -- (axis cs:32/25, -6175323/34000000);
        \draw [dashed, blue]
           (axis cs:17/10, 0) -- (axis cs:17/10, -3/10);
      \end{axis}
    \end{tikzpicture}
    \caption{\foreignlanguage{latin}{\emph{Regula falsi}}}
    \label{01::RegularFalsi:grafico}
  \end{figure}
  De allí procedemos iterando,
  calculando:
  \begin{equation}
    \label{eq:regula-falsi-iteration}
    x_{n + 1}
      = x_n - f(x_n) \cdot \frac{x_n - x_0}{f(x_n) - f(x_0)}
  \end{equation}
  Mantenemos fijo el punto inicial \(x_0\),
  la exigencia de siempre horquillar el cero
  llevará a esta situación una vez estemos suficientemente cerca del cero
  (vea la figura~\ref{01::RegularFalsi:grafico}).

\section{Iteración de Punto Fijo}

  En inglés,
  a esta técnica
  se le llama \emph{\foreignlanguage{english}{fixed point iteration}}
  y se abrevia FPI.
  \begin{definition}
    Sea \(g(x)\) una función.
    Un \emph{punto fijo} de \(g\)
    es \(x^*\) tal que \(x^* = g(x^*)\).
  \end{definition}
  Es natural buscar aproximar un punto fijo de \(g\)
  mediante el siguiente esquema:
  Elegir algún \(x_0\) adecuado,
  y luego calcular sucesivamente:
  \begin{equation*}
    x_{n + 1}
      = g(x_n)
  \end{equation*}
  Dependiendo del punto inicial \(x_0\) y de las características de \(g\)
  esto tendrá
  (o no)
  éxito.

  Encontrar un cero de una función por algún método de punto fijo
  consiste en reescribir:
  \begin{equation*}
    f(x)
      = 0
  \end{equation*}
  en la forma:
  \begin{equation*}
    g(x)
      = x
  \end{equation*}
  Note que siempre podemos escribir:
  \begin{equation*}
    g(x)
      = x - \alpha f(x)
  \end{equation*}
  Queda elegir un \(\alpha \ne 0\) adecuado\ldots
  \begin{ejemplo}
    Supongamos que queremos encontrar una solución a la ecuación:
    \begin{equation*}
      x - \mathrm{e}^{-x}
        = 0
    \end{equation*}
    Alternativas de funciones a que buscar punto fijo
    basadas en la ecuación propuesta son:
    \begin{align*}
      x
        &= \mathrm{e}^{-x} \\
      x
        &= - \ln x \\
      x
        &= x - \frac{2}{3} (x - \mathrm{e}^{-x})
    \end{align*}
    Como al crecer \(x\) aumenta \(x\)
    y disminuye (pero más lentamente) \(\mathrm{e}^{-x}\),
    la función dada es creciente,
    vemos que para \(x = 0\) es \(-1\)
    y para \(x = 1\) es \(1 - \mathrm{e}^{-1} > 0\),
    hay un cero en ese rango.
    Podemos elegir \(x_0 = 0,5\).
  \end{ejemplo}
  Usando iteraciones de punto fijo,
  se tiene que el cero de la función
  corresponde a la intersección entre \(y = x\) y \(g(x)\)
  (figura~\ref{01::FPI}).
  \begin{figure}[ht]
    \centering
    \begin{tikzpicture}
      \begin{axis}[
                    axis lines = center,
                    xlabel = {\(x\)},
                    ylabel = {\(y\)},
                    every axis y
                      label/.style = {at = (current axis.above origin),
                        anchor = south},
                    every axis x
                      label/.style = {at = (current axis.right of origin),
                        anchor = west},
                    xmin = 0, xmax = 2.5,
                    ymin = 0, ymax = 2,
                    xtick = {0.81},
                    xticklabels = {\(x^*\)},
                    ytick = {0},
                    yticklabels = {},
                    height = 6cm, width = 9cm
                  ]
        \addplot [domain = 0:1.4, blue]
           {x} node [yshift = 2pt, xshift = 13pt] (yx) {\(y = x\)};
        \addplot [domain = 0.5:1.6, red] {(1.5 - 0.7 * x)^3}
           node [yshift = 10pt] (g) {\(g(x)\)};

        \draw [dashed, blue] (axis cs:0.81, 0) -- (axis cs:0.81, 1.11);
      \end{axis}
    \end{tikzpicture}
    \caption{La intersección entre \(y = x\)
             e \(y = g(x)\) es un punto fijo}
    \label{01::FPI}
  \end{figure}
  Gráficamente,
  la iteración puede representarse en un gráfico de \(y = g(x)\)
  y la recta \(y = x\),
  el punto fijo es la intersección entre ambas.
  Partiendo del valor \(x_0\), calculamos \(g(x_0)\),
  lo que corresponde a ir en vertical a la curva para \(g\).
  Este valor será \(x_1\),
  cosa que corresponde a ir horizontalmente a la línea \(y = x\),
  dando la intersección como \(x_1\).
  Se puede ver como una espiral si \(g(x)\) es decreciente
  (figura~\ref{01::Espiral:convergente}
   y figura~\ref{01::Espiral:divergente}).
  \begin{figure}[ht]
    \centering
    % Ver cobweb-2.mc, cobwebs.raku para detalles
    \begin{tikzpicture}
      \begin{axis}[domain = 0:1,
                   axis x line = bottom, axis y line = left,
                   xmin = 0, xmax = 1,
                   ymin = 0, ymax = 1,
                   xtick = {0.1, 0.812, 0.25930573, 0.7084739004713133,
                     0.3257801870289462, 0.6545450715920982,
                     0.36698903565963403, 0.5},
                   xticklabels = {\(x_0^{\phantom{\ast}}\),
                                  \(x_1^{\phantom{\ast}}\),
                                  \(x_2^{\phantom{\ast}}\),
                                  \(x_3^{\phantom{\ast}}\),
                                  \(x_4^{\phantom{\ast}}\),
                                  \(\),
                                  \(\),
                                  \(x^{\ast}_{\phantom{0}}\)},
                   ytick style = {draw = none},
                   yticklabels = {}]

        \addplot[mark = none, blue] {x};
        \addplot[mark = none, red]
          {1 * x^3 / 1 - 7 * x^2 / 5 - 1 * x / 4 + 17 / 20};

      \draw [blue, very thin, dashed]
         (axis cs:0.1, 0) -- (axis cs:0.1, 0.812);
      \draw [blue, very thin, dashed]
         (axis cs:0.1, 0.812) -- (axis cs:0.812, 0.812);
      \draw [blue, very thin, dashed]
         (axis cs:0.812, 0.812) -- (axis cs:0.812, 0.25930573);
      \draw [blue, very thin, dashed]
         (axis cs:0.812, 0.25930573) -- (axis cs:0.25930573, 0.25930573);
      \draw [blue, very thin, dashed]
         (axis cs:0.25930573, 0.25930573)
              -- (axis cs:0.25930573, 0.7084739004713133);
      \draw [blue, very thin, dashed]
         (axis cs:0.25930573, 0.25930573)
              -- (axis cs:0.25930573, 0.7084739004713133);
      \draw [blue, very thin, dashed]
         (axis cs:0.25930573, 0.7084739004713133)
              -- (axis cs:0.7084739004713133, 0.7084739004713133);
      \draw [blue, very thin, dashed]
         (axis cs:0.7084739004713133, 0.7084739004713133)
              -- (axis cs:0.7084739004713133, 0.3257801870289462);
      \draw [blue, very thin, dashed]
         (axis cs:0.7084739004713133, 0.7084739004713133)
              -- (axis cs:0.7084739004713133, 0.3257801870289462);
      \draw [blue, very thin, dashed]
         (axis cs:0.7084739004713133, 0.3257801870289462)
              -- (axis cs:0.3257801870289462, 0.3257801870289462);
      \draw [blue, very thin, dashed]
         (axis cs:0.3257801870289462, 0.3257801870289462)
              -- (axis cs:0.3257801870289462, 0.6545450715920982);
      \draw [blue, very thin, dashed]
         (axis cs:0.3257801870289462, 0.3257801870289462)
              -- (axis cs:0.3257801870289462, 0.6545450715920982);
      \draw [blue, very thin, dashed]
         (axis cs:0.3257801870289462, 0.6545450715920982)
              -- (axis cs:0.6545450715920982, 0.6545450715920982);
      \draw [blue, very thin, dashed]
         (axis cs:0.6545450715920982, 0.6545450715920982)
              -- (axis cs:0.6545450715920982, 0.36698903565963403);
      \draw [blue, very thin, dashed]
         (axis cs:0.6545450715920982, 0.36698903565963403)
              -- (axis cs:0.36698903565963403, 0.36698903565963403);
      \end{axis}
    \end{tikzpicture}
    \caption{En este caso la espiral de iteración de punto fijo converge}
    \label{01::Espiral:convergente}
  \end{figure}
  \begin{figure}[ht]
    \centering
    % Ver cobweb-1.mc, cobwebs.raku para detalles
    \begin{tikzpicture}
      \begin{axis}[domain = 0:1,
                   axis x line = bottom, axis y line = left,
                   xmin = 0, xmax = 1,
                   ymin = 0, ymax = 1,
                   xtick = {0.3, 0.7425, 0.245202598, 0.8053675633204307,
                     0.19829350349601982, 0.8561644490113705,
                     0.1687199716882174, 0.5},
                   xticklabels = {\(x_0^{\phantom{\ast}}\),
                                  \(x_1^{\phantom{\ast}}\),
                                  \(x_2^{\phantom{\ast}}\),
                                  \(x_3^{\phantom{\ast}}\),
                                  \(\),
                                  \(x_5^{\phantom{\ast}}\),
                                  \(x_6^{\phantom{\ast}}\),
                                  \(x^{\ast}_{\phantom{0}}\)},
                   ytick style = {draw = none},
                   yticklabels = {}]

        \addplot[mark = none, blue] {x};
        \addplot[mark = none, red]
          {5 * x^3 / 4 - 25 * x^2 / 16 - 23 * x / 40 + 327 / 320};

      \draw [blue, very thin, dashed]
         (axis cs:0.3, 0) -- (axis cs:0.3, 0.7425);
      \draw [blue, very thin, dashed]
         (axis cs:0.3, 0.7425) -- (axis cs:0.7425, 0.7425);
      \draw [blue, very thin, dashed]
         (axis cs:0.7425, 0.7425) -- (axis cs:0.7425, 0.245202598);
      \draw [blue, very thin, dashed]
         (axis cs:0.7425, 0.245202598) -- (axis cs:0.245202598, 0.245202598);
      \draw [blue, very thin, dashed]
         (axis cs:0.245202598, 0.245202598)
             -- (axis cs:0.245202598, 0.8053675633204307);
      \draw [blue, very thin, dashed]
         (axis cs:0.245202598, 0.245202598)
             -- (axis cs:0.245202598, 0.8053675633204307);
      \draw [blue, very thin, dashed]
      (axis cs:0.245202598, 0.8053675633204307)
             -- (axis cs:0.8053675633204307, 0.8053675633204307);
      \draw [blue, very thin, dashed]
         (axis cs:0.8053675633204307, 0.8053675633204307)
             -- (axis cs:0.8053675633204307, 0.19829350349601982);
      \draw [blue, very thin, dashed]
         (axis cs:0.8053675633204307, 0.8053675633204307)
             -- (axis cs:0.8053675633204307, 0.19829350349601982);
      \draw [blue, very thin, dashed]
         (axis cs:0.8053675633204307, 0.19829350349601982)
             -- (axis cs:0.19829350349601982, 0.19829350349601982);
      \draw [blue, very thin, dashed]
         (axis cs:0.19829350349601982, 0.19829350349601982)
             -- (axis cs:0.19829350349601982, 0.8561644490113705);
      \draw [blue, very thin, dashed]
         (axis cs:0.19829350349601982, 0.19829350349601982)
              -- (axis cs:0.19829350349601982, 0.8561644490113705);
      \draw [blue, very thin, dashed]
         (axis cs:0.19829350349601982, 0.8561644490113705)
              -- (axis cs:0.8561644490113705, 0.8561644490113705);
      \draw [blue, very thin, dashed]
         (axis cs:0.8561644490113705, 0.8561644490113705)
              -- (axis cs:0.8561644490113705, 0.1687199716882174);
      \draw [blue, very thin, dashed]
         (axis cs:0.8561644490113705, 0.1687199716882174)
              -- (axis cs:0.1687199716882174, 0.1687199716882174);
      \end{axis}
    \end{tikzpicture}
    \caption{En este caso la espiral de iteración de punto fijo diverge}
    \label{01::Espiral:divergente}
  \end{figure}
  A los diagramas de las figuras~\ref{01::Espiral:convergente}
  y~\ref{01::Espiral:divergente}
  les llaman \emph{diagrama de telaraña}
  (en inglés,
   \emph{\foreignlanguage{english}{cobweb diagram}})
  por razones fáciles de ver.
  Note que para construir estas espirales
  debe partir por el eje \(x\),
  en algún \(x_0\) a su gusto.
  Luego,
  tirar una línea vertical hasta chocar con \(g\).
  En seguida,
  continúe con una línea horizontal hasta llegar a \(y = x\)
  y vuelva a trazar otra vertical hasta \(g(x)\).
  Continúe así hasta encontrar una aproximación suficiente
  al punto fijo.

  Las figuras~\ref{01::Espiral:convergente} y~\ref{01::Espiral:divergente}
  ilustran que si \(g(x)\) es decreciente en el rango de interés
  y la gráfica de \(y = g(x)\) se acerca a la horizontal
  cerca del punto de intersección con \(y = x\),
  el método converge.
  En cambio,
  si la gráfica de \(y = g(x)\) se acerca a la vertical,
  diverge.

  En caso que \(g\) sea creciente en vez de una espiral
  se obtiene una escalera,
  vea las figuras~\ref{fig:FIP-increasing-converges}
  y~\ref{fig:FIP-increasing-diverges}.
  \begin{figure}
    \centering
    % Ver cobweb-4.mc, cobwebs.raku para detalles
    \begin{tikzpicture}
      \begin{axis}[domain = 0:1,
                   axis x line = bottom, axis y line = left,
                   xmin = 0, xmax = 1,
                   ymin = 0, ymax = 1,
                   xtick = {0.1, 0.5000066, 0.678700259368431587,
                     0.7312996515938205, 0.74701294446039,
                     0.7518459248664618, 0.7533485113148211, 0.75},
                   xticklabels = {\(x_0^{\phantom{\ast}}\),
                                  \(x_1^{\phantom{\ast}}\),
                                  \(x_2^{\phantom{\ast}}\),
                                  \(\),
                                  \(\),
                                  \(\),
                                  \(\),
                                  \(x^{\ast}_{\phantom{0}}\)},
                   ytick style = {draw = none},
                   yticklabels = {}]

        \addplot[mark = none, blue] {x};
        \addplot[mark = none, red]
                {680 * x^3 / 1521 - 1270 * x^2 / 1521
                  + 16409 * x / 20280 + 5773 / 13520};

      \draw [blue, very thin, dashed]
         (axis cs:0.1, 0) -- (axis cs:0.1, 0.5000066);
      \draw [blue, very thin, dashed]
         (axis cs:0.1, 0.5000066) -- (axis cs:0.5000066, 0.5000066);
      \draw [blue, very thin, dashed]
         (axis cs:0.5000066, 0.5000066)
              -- (axis cs:0.5000066, 0.678700259368431587);
      \draw [blue, very thin, dashed]
         (axis cs:0.5000066, 0.678700259368431587)
              -- (axis cs:0.678700259368431587, 0.678700259368431587);
      \draw [blue, very thin, dashed]
         (axis cs:0.678700259368431587, 0.678700259368431587)
              -- (axis cs:0.678700259368431587, 0.7312996515938205);
      \draw [blue, very thin, dashed]
         (axis cs:0.678700259368431587, 0.678700259368431587)
              -- (axis cs:0.678700259368431587, 0.7312996515938205);
      \draw [blue, very thin, dashed]
         (axis cs:0.678700259368431587, 0.7312996515938205)
              -- (axis cs:0.7312996515938205, 0.7312996515938205);
      \draw [blue, very thin, dashed]
         (axis cs:0.7312996515938205, 0.7312996515938205)
              -- (axis cs:0.7312996515938205, 0.74701294446039);
      \draw [blue, very thin, dashed]
         (axis cs:0.7312996515938205, 0.7312996515938205)
              -- (axis cs:0.7312996515938205, 0.74701294446039);
      \draw [blue, very thin, dashed]
         (axis cs:0.7312996515938205, 0.74701294446039)
              -- (axis cs:0.74701294446039, 0.74701294446039);
      \draw [blue, very thin, dashed]
         (axis cs:0.74701294446039, 0.74701294446039)
              -- (axis cs:0.74701294446039, 0.7518459248664618);
      \draw [blue, very thin, dashed]
         (axis cs:0.74701294446039, 0.74701294446039)
              -- (axis cs:0.74701294446039, 0.7518459248664618);
      \draw [blue, very thin, dashed]
         (axis cs:0.74701294446039, 0.7518459248664618)
              -- (axis cs:0.7518459248664618, 0.7518459248664618);
      \draw [blue, very thin, dashed]
         (axis cs:0.7518459248664618, 0.7518459248664618)
              -- (axis cs:0.7518459248664618, 0.7533485113148211);
      \draw [blue, very thin, dashed]
         (axis cs:0.7518459248664618, 0.7533485113148211)
              -- (axis cs:0.7533485113148211, 0.7533485113148211);
      \end{axis}
    \end{tikzpicture}
    \caption{La escalera de iteración de punto fijo converge}
    \label{fig:FIP-increasing-converges}
  \end{figure}
  \begin{figure}
    \centering
    % Ver cobweb-3.mc, cobwebs.raku para detalles
    \begin{tikzpicture}
      \begin{axis}[domain = 0:1,
                   axis x line = bottom, axis y line = left,
                   xmin = 0, xmax = 1,
                   ymin = 0, ymax = 1,
                   xtick = {0.3, 0.323028, 0.355531714912121,
                     0.40050722374171244, 0.4615564206393762,
                     0.5436710682749981, 0.6563153774050898, 0.25},
                   xticklabels = {\(x_0^{\phantom{\ast}}\),
                                  \(\),
                                  \(\),
                                  \(x_3^{\phantom{\ast}}\),
                                  \(x_4^{\phantom{\ast}}\),
                                  \(x_5^{\phantom{\ast}}\),
                                  \(x_6^{\phantom{\ast}}\),
                                  \(x^{\ast}_{\phantom{0}}\)},
                   ytick style = {draw = none},
                   yticklabels = {}]

        \addplot[mark = none, blue] {x};
        \addplot[mark = none, red]
                {2450 * x^3 / 1521 - 3160 * x^2 / 1521 + 27217 * x / 12168
                  - 553 / 2704};

      \draw [blue, very thin, dashed]
         (axis cs:0.3, 0) -- (axis cs:0.3, 0.323028);
      \draw [blue, very thin, dashed]
         (axis cs:0.3, 0.323028) -- (axis cs:0.323028, 0.323028);
      \draw [blue, very thin, dashed]
         (axis cs:0.323028, 0.323028) -- (axis cs:0.323028, 0.355531714912121);
      \draw [blue, very thin, dashed]
         (axis cs:0.323028, 0.355531714912121)
              -- (axis cs:0.355531714912121, 0.355531714912121);
      \draw [blue, very thin, dashed]
         (axis cs:0.355531714912121, 0.355531714912121)
              -- (axis cs:0.355531714912121, 0.40050722374171244);
      \draw [blue, very thin, dashed]
         (axis cs:0.355531714912121, 0.355531714912121)
              -- (axis cs:0.355531714912121, 0.40050722374171244);
      \draw [blue, very thin, dashed]
         (axis cs:0.355531714912121, 0.40050722374171244)
              -- (axis cs:0.40050722374171244, 0.40050722374171244);
      \draw [blue, very thin, dashed]
         (axis cs:0.40050722374171244, 0.40050722374171244)
              -- (axis cs:0.40050722374171244, 0.4615564206393762);
      \draw [blue, very thin, dashed]
         (axis cs:0.40050722374171244, 0.40050722374171244)
              -- (axis cs:0.40050722374171244, 0.4615564206393762);
      \draw [blue, very thin, dashed]
         (axis cs:0.40050722374171244, 0.4615564206393762)
              -- (axis cs:0.4615564206393762, 0.4615564206393762);
      \draw [blue, very thin, dashed]
         (axis cs:0.4615564206393762, 0.4615564206393762)
              -- (axis cs:0.4615564206393762, 0.5436710682749981);
      \draw [blue, very thin, dashed]
         (axis cs:0.4615564206393762, 0.4615564206393762)
              -- (axis cs:0.4615564206393762, 0.5436710682749981);
      \draw [blue, very thin, dashed]
         (axis cs:0.4615564206393762, 0.5436710682749981)
              -- (axis cs:0.5436710682749981, 0.5436710682749981);
      \draw [blue, very thin, dashed]
         (axis cs:0.5436710682749981, 0.5436710682749981)
              -- (axis cs:0.5436710682749981, 0.6563153774050898);
      \draw [blue, very thin, dashed]
         (axis cs:0.5436710682749981, 0.6563153774050898)
              -- (axis cs:0.6563153774050898, 0.6563153774050898);
      \end{axis}
    \end{tikzpicture}
    \caption{La escalera de iteración de punto fijo diverge}
    \label{fig:FIP-increasing-diverges}
  \end{figure}


\subsection{Método de la Secante}

  La idea,
  al igual que para \foreignlanguage{latin}{\emph{regula falsi}},
  es calcular el intercepto de la recta que pasa por dos puntos,
  ver figura~\ref{01::secante:grafico}.
  Pero a diferencia de ese método,
  no se mantienen puntos fijos.
  Partiendo con \(x_0, x_1\)
  se calculan valores sucesivos
  usando:
  \begin{equation*}
    x_{n + 2}
      = x_{n + 1}
          - f(x_{n + 1}) \cdot \frac{x_{n + 1} - x_n}{f(x_{n + 1}) - f(x_n)}
  \end{equation*}
  \begin{figure}[ht]
    \centering
    % The function interpolates:
    % 0.05  0.80
    % 0.15  0.40
    % 0.30  0.15
    % 0.40  0.00
    % 0.80 -0.30
    % See function.mc
    \begin{tikzpicture}
      \begin{axis}[
                    axis lines = center,
                    xlabel = {\(x\)},
                    ylabel = {\(y\)},
                    every axis y
                      label/.style = {at = (current axis.above origin),
                        anchor = south},
                    every axis x
                      label/.style = {at = (current axis.right of origin),
                        anchor = west},
                    xmin = 0,	 xmax = 0.5,
                    ymin = -0.6, ymax = 1.2,
                    xtick = {1/20, 3/10, 93/260},
                    xticklabels = {\small \(x_0\),
                                   \small \(x_1\),
                                   \small \(x_2\)},
                    ytick = {0},
                    yticklabels = {},
                    height = 6cm, width = 9cm
                  ]
        \addplot [blue]
           {(93 - 260 * x) / 100};
        \addplot [domain = 0.01:0.5, red]
           {(4740000 * x^4 - 7646000 * x^3 + 4231950 * x^2
              - 1167595 * x + 157926) / 136500};

        \draw [dashed, blue] (axis cs:1/20, 0) -- (axis cs:1/20, 4/5);
        \draw [dashed, blue] (axis cs:3/10, 0) -- (axis cs:3/10, 3/20);
      \end{axis}
    \end{tikzpicture}
    \caption{Una iteración del método de la secante.}
    \label{01::secante:grafico}
  \end{figure}
  La ventaja es que al usar puntos cada vez más cerca del cero
  la aproximación lineal se hace más precisa.
  Lo malo es que no asegura un rango que contiene el cero.

\subsection{Método de la tangente (Newton)}

  La idea consiste en elegir un valor arbitrario \(x_0\)
  que esté razonablemente cerca de \(x^*\) (cero de la función).
  Luego,
  sucesivamente encontramos el intercepto con el eje \(x\)
  de la recta tangente de la curva en \(x_n\) usando la ecuación:
  \begin{equation}\label{01::ecuacion:newton}
    x_{n + 1}
      = x_n - \frac{f(x_n)}{f'(x_n)}
  \end{equation}
  resultando un valor \(x_{n + 1}\)
  que se encuentra más cerca de \(x^*\) (figura~\ref{01::newton:grafico}).
  Finalmente,
  iteramos el proceso
  hasta obtener un valor suficientemente cercano al buscado.
  \begin{figure}[ht]
    \centering
    % The function interpolates:
    % 0.05  0.80
    % 0.15  0.40
    % 0.30  0.15
    % 0.40  0.00
    % 0.80 -0.30
    % See function.mc
    \begin{tikzpicture}
      \begin{axis}[
                    axis lines = center,
                    xlabel = {\(x\)},
                    ylabel = {\(y\)},
                    every axis y
                      label/.style = {at = (current axis.above origin),
                        anchor = south},
                    every axis x
                      label/.style = {at = (current axis.right of origin),
                        anchor = west},
                    xmin =  0,	 xmax = 0.5,
                    ymin = -0.6, ymax = 1.2,
                    xtick	= {3/20,    17139/56020},
                    xticklabels = {\small \(x_0\),
                                   \small \(x_1\)},
                    ytick = {0},
                    yticklabels = {},
                    height = 6cm, width = 9cm
                  ]
        \addplot [blue]
          {(17139 - 56020 * x) / 21480};
        \addplot [domain = 0.01:0.5, red]
           {(4740000 * x^4 - 7646000 * x^3 + 4231950 * x^2
              - 1167595 * x + 157926) / 136500};
        \draw [dashed, blue] (axis cs:3/20, 0) -- (axis cs:3/20, 2/5);
      \end{axis}
    \end{tikzpicture}
    \caption{El valor \(x_1\)
             de~\eqref{01::ecuacion:newton}
             es más cercano a \(x^*\) que \(x_0\).}
     \label{01::newton:grafico}
   \end{figure}

\section{Criterios de término}
\label{sec:stopping-criteria}

  Un tema pendiente es bajo qué condiciones terminar la iteración.
  Si el cero \(x^*\) es grande,
  basta asegurar que el error absoluto es pequeño;
  si es pequeño,
  es más relevante el error relativo.
  El criterio más robusto es exigir que se cumplan ambos.

  Pero desconocemos \(x^*\),
  tenemos que usar aproximaciones o algún esquema indirecto.
  En métodos que acotan el cero
  (como bisección),
  eso es relativamente simple.
  Detectar que nuestro método se comporta mal,
  o estamos cerca de los límites impuestos por errores de redondeo,
  no es fácil.
  Vea por ejemplo la discusión de Kahan del esquema iterativo de Cauchy~%
    \cite{kahan10:_estim_error_bound_cauchy_iteration}
  o sus notas sobre hallar ceros~%
    \cite{kahan16:_lecture_notes_real_root_finding}.

\section*{Ejercicios}
\label{sec:ejercicios-01}

  \begin{enumerate}
  \item
    Complete el ejemplo de aproximar el cero de \(x - \mathrm{e}^{-x}\)
    mediante iteración de punto fijo
    usando las alternativas de función con punto fijo planteadas.
    Ejecute algunas iteraciones y compare sus evoluciones.
  \item
    Usando nuevamente la ecuación \(x - \mathrm{e}^{-x}\),
    compare los métodos de bisección
    y \emph{\foreignlanguage{latin}{regula falsi}}
    con rango inicial \([0, 1]\).
    Compare el número de iteraciones necesarias
    para obtener cinco cifras significativas.
  \item
    Usando nuevamente la ecuación \(x - \mathrm{e}^{-x}\),
    compare los métodos de la secante
    (elija valores iniciales \(x_0 = 0\) y \(x_1 = 1\))
    y de Newton con valor inicial \(x_0 = 0\).
  \end{enumerate}

\bibliography{../referencias}

%%% Local Variables:
%%% mode: latex
%%% TeX-master: "../INF-221_notas"
%%% ispell-local-dictionary: "spanish"
%%% End:

% LocalWords:  recurrirse english bracketing latin falsi fixed point
% LocalWords:  iteration FPI cobweb diagram

\bibliographystyle{babplain-fl}

\chapter{Análisis de Convergencia}
\label{cha:convergencia}

  Contamos con varios métodos iterativos para hallar un cero de una función,
  interesa compararlos según alguna medida de rendimiento.
  Una medida clara es cuánto disminuye el error en cada iteración,
  cosa que es relativamente sencilla de deducir.
  Enfatizamos que en esto no consideramos errores de redondeo,
  consideramos cálculos exactos.
  Claro que en la práctica más nos interesa cuánto tiempo demora,
  además de otras medidas como qué tan complicado es de programar
  y posible información adicional que requiere
  (por ejemplo,
   un método que necesita derivadas es impracticable
   cuando la función se conoce solo
   como el resultado de un complejo cálculo,
   no en forma explícita).
  Otras medidas importantes son el área desde la cual el método converge
  (si tenemos que partir muy cerca del cero de interés,
   no tiene demasiada gracia).
  Pero esto nos llevaría demasiado lejos,
  ver las notas de Kahan~%
    \cite{kahan16:_lecture_notes_real_root_finding}
  para una discusión más detallada.

\section{Orden de Convergencia}

  Hemos derivado los métodos mediante consideraciones heurísticas,
  como aproximar la función mediante una recta.
  Interesa ahora obtener cotas asintóticas precisas para el error.
  Note que muchas derivaciones simplemente
  omiten los términos de error de mayor grado,
  al considerarlos despreciables.
  Acá buscamos justificar esto rigurosamente.

  Si definimos \(e_n = x_n - x^*\),
  se dice que un método es de \emph{orden \(p\)},
  si para \(e_n \to 0\) es:
  \begin{equation}
    \lvert e_{n+1} \rvert
      = C \lvert e_n \rvert^p + o(e_n^p)
  \end{equation}
  Acá \(e_n\) corresponde al error con \(n\) iteraciones.
  Estamos usando \(o(e_n^p)\) como cota,
  lo que interesa es que el error sea de orden menor que \(e_n^p\)
  y no queremos comprometernos más allá
  (como sería indicar algo como \(O(e_n^p)\)).
  \begin{itemize}
  \item
    Si \(p = 1\) hablamos de convergencia lineal.
    En este caso debe ser \(0 < C < 1\) para convergencia.
    Bisección tiene \(C = \frac{1}{2}\).
  \item
    Si \(p = 2\) hablamos de convergencia cuadrática
    (en cada paso, el error se eleva a \(p = 2\)).
  \item
    Si \(p = 3\) hablamos de convergencia cúbica.
  \item
    Si	\(p > 1\) decimos que es superlineal.
  \end{itemize}
  Note que mientras mayor sea el orden de convergencia \(p\),
  más rápido
  (en menos iteraciones)
  encontraremos una aproximación adecuada al valor buscado.
  Pero esto hay que balancearlo con otras consideraciones,
  como si conocemos derivadas
  (puede ser que la función quede definida por un procedimiento complejo,
   que no permita obtenerlas)
  y el trabajo extra por iteración que demande.

  Una manera más representativa de la convergencia
  es el índice de eficiencia asintótica.
  Si en una iteración con orden de convergencia~\(p\)
  se evalúa la función \(n\)~veces,
  el índice es \(p^{1/n}\).
  Esto representa el avance por evaluación de la función.

\section{Análisis de las técnicas descritas}
\label{sec:analisis-tecnicas}

  En lo que sigue,
  en la notación \(O\)
  supondremos un rango de la forma \(\lvert e \rvert \le \varepsilon\),
  para \(\varepsilon\) apropiado,
  y lo omitiremos para abreviar.

  Sea \(f(x)\) la función que buscamos el cero \(x^*\):
  \begin{equation*}
    f(x^*)
     = 0
  \end{equation*}
  Suponiendo \(f(x)\) continua,
  que puede derivarse tres veces en un entorno de \(x^*\),
  por teorema de Taylor
  usando la forma de Lagrange del residuo,
  sabemos que hay \(x^* < \xi < x\)
  (o \(x < \xi < x^*\) si \(x < x^*\))
  tal que:
  \begin{align}
    f(x)
      &= f(x^*)
           + f'(x^*) (x - x^*)
           + \frac{1}{2!} f''(x^*) (x - x^*)^2
           + \frac{1}{3!} f'''(\xi) (x - x^*)^3
              \notag \\
  \intertext{Definiendo \(e = x - x^*\) tenemos la cota asintótica:}
    f(x^* + e)
      &= f'(x^*) e + \frac{1}{2!} f''(x^*) e^2 + O(e^3) \\
      &= f'(x^*) e (1 + M e + O(e^2))
              \label{eq:f-aproximado}
  \end{align}
  donde:
  \begin{equation*}
    M
      = \frac{f''(x^*)}{2 f'(x^*)}
  \end{equation*}

  Llamaremos:
  \begin{equation}
    e_n
      = x_n - x^*
  \end{equation}
  donde \(n\) es el número de la iteración correspondiente.
  El análisis de cómo evoluciona el error al ir iterando
  debe efectuarse por separado para cada uno de los métodos.

\section{Regula Falsi}

  Para \emph{\foreignlanguage{latin}{regula falsi}},
  recordemos la ecuación~\eqref{eq:regula-falsi-iteration}:
  \begin{equation}
    x_{n+1}
      = x_n - f(x_n) \cdot \frac{x_n - x_0}{f(x_n) - f(x_0)}
  \end{equation}
  Sea \(x^*\) el cero,
  en términos del error \(e_n = x_n - x^*\)
  expandiendo por Taylor alrededor del cero
  (basta un solo término,
   como veremos):
  \begin{equation*}
    f(x^* + e)
      = f'(x^*) e + O(e^2)
  \end{equation*}
  \begin{align*}
    e_{n+1}
      &= e_n - f(x_n) \cdot \frac{e_n - e_0}{f(x_n) - f(x_0)} \\
      &= \frac{e_n (f'(x^*) e_0 - f(x_0)) + O(e_n^2)}
              {- f(x_0) + f'(x^*) e_n + O(e_n^2)} \\
      &= \frac{e_n (f'(x^*) e_0 - f(x_0)) + O(e_n^2)}{-f(x_0)}
           \cdot \left( 1 + \frac{f'(x^*)}{f(x_0)} e_n + O(e_n^2) \right) \\
      &= \frac{e_n (f'(x^*) e_0 - f(x_0)) + O(e_n^2)}{-f(x_0)}
           \cdot (1 + O(e_n)) \\
      &= e_n \left(
               1 - \frac{f'(x^*) e_0}{f(x_0)}
                 + O(e_n)
             \right) \\
      &= e_n \left( 1 - \frac{f'(x^*) e_0}{f(x_0)} \right) + O(e_n^2)
  \end{align*}
  La convergencia es lineal.

\section{Método de Newton}

  Considere la figura \ref{02::newton:grafico},
  \begin{figure}[ht]
    \centering
    \begin{tikzpicture}
      \begin{axis}[
                    axis lines = center,
                    xlabel = {\(x\)},
                    ylabel = {\(y\)},
                    every axis y
                      label/.style = {at = (current axis.above origin),
                        anchor = south},
                    every axis x
                      label/.style = {at = (current axis.right of origin),
                        anchor = west},
                    xmin = 0,	 xmax = 2,
                    ymin = -0.6, ymax = 2,
                    xtick = {0.535, 0.88375},
                    xticklabels = {\(x_n\), \(x_{n+1}\)},
                    ytick = {0},
                    yticklabels = {},
                    height = 6cm, width = 9cm
                  ]
        \addplot [black!70!blue!80, samples = 100, smooth, tension = 0.8]
          coordinates {(0.3, 1.3) (0.6, 0.5) (1.3, -0.2) (1.8, -0.3)};
        \addplot [domain = 0.2:1.1, red!80!black, samples = 100]
          {-1.72043 * x + 1.52043};

        \draw [dashed, blue] (axis cs:0.535, 0) -- (axis cs:0.535, 0.6);
      \end{axis}
    \end{tikzpicture}
    \caption{Una iteración del método de Newton.}
    \label{02::newton:grafico}
  \end{figure}
  donde tomamos la tangente a la curva \(y = f(x)\) en \(x_n\),
  hallando el intercepto de la tangente con el eje \(X\) en \(x_{n + 1}\):
  \begin{equation}
    x_{n+1}
      = x_n - \frac{f(x_n)}{f'(x_n)}
  \end{equation}
  Considere la aproximación~\eqref{eq:f-aproximado},
  con \(e_n = x_n - x^*\),
  usando la serie geométrica para simplificar la dependencia del error
  tenemos:
  \begin{align*}
    e_{n+1}
      &= e_n - \frac{f(x_n)}{f'(x_n)} \\
      &= e_n - \frac{f'(x^*) e_n (1 + M e_n) + O(e_n^3)}
                    {f'(x^*) (1 + 2 M e_n) + O(e_n^2)} \\
      &= e_n - e_n \frac{1 + M e_n + O(e_n^2)}{1 + 2 M e_n + O(e_n^2)} \\
      &= e_n
          - e_n
              \left( 1 + M e_n + O(e_n^2) \right)
              \left( 1 - 2 M e_n + O(e_n^2) \right) \\
      &= e_n
           - e_n \left(
                   1 - M e_n - 2 M^2 e_n^2 + O(e_n^2)
                 \right) \\
      &= e_n
          - e_n
              \left( 1 - M e_n + O(e_n^2) \right) \\
      &= M e_n^2 + O(e_n^3) \\
      &= \frac{f''(x^*)}{2 f'(x^*)} \cdot e_n^2 + O(e_n^3)
  \end{align*}
  O sea,
  el método de Newton es cuadrático si \(f'(x^*) \ne 0\).
  En el fondo,
  el método de Newton duplica el número de cifras correctas
  en cada iteración.
  Supongamos que el error cumple \(e_k / x^* = a \cdot 10^{-n}\),
  con \(0 < \lvert a \rvert \le 1/2\),
  o sea,
  conocemos \(x^*\) con \(n\) cifras a la iteración \(k\).
  Por la forma aproximada del error:
  \begin{align*}
    \left \lvert \frac{e_{k + 1}}{x^*} \right\rvert
      &\approx \lvert M \rvert
                  \cdot \left\lvert \frac{e_k^2}{x^*} \right\rvert \\
      &=       \lvert M x^* a^2 \rvert \cdot 10^{-2 n} \\
      &\le     \left\lvert \frac{M x^*}{4} \right\rvert \cdot 10^{-2 n} \\
  \end{align*}
  Esto corresponde a conocer \(x^*\) con \(2 n\) cifras
  (siempre que \(\lvert M x^* / 4 \rvert \le 1/2\),
   claro).

  Simplificar las expresiones asintóticas es una tarea bastante ardua,
  por suerte hay sistemas de álgebra simbólica.
  En particular,
  el sistema SymPy\cite{sympy22:_1.10.1} maneja series
  y la notación \(O(\cdot)\).
  Para indicar una expresión de orden \(x^3\) se escribe
  \lstinline[language = Python]{O(x ** 3, x)}
  (deben explicitarse las variables a las que aplica,
   vea la documentación para detalles adicionales).
  \lstinputlisting[language = Python,
                   linerange = {5-18},
                   caption = {An\'alisis de convergencia de Newton},
                   label = lst:Newton-convergence]{code/Newton.py}
  Usando estas facilidades
  con las expansiones desarrolladas antes
  el programa del listado~\ref{lst:Newton-convergence}
  retorna \lstinline[language = Python]{M*e**2 + O(e**3)},
  o sea es:
  \begin{equation*}
    e_{n + 1}
      = M e_n^2 + O(e_n^3)
  \end{equation*}

\subsection{Convergencia del método de Newton}
\label{sec:Newton-convergence}

  Hay una variedad de condiciones
  que garantizan la convergencia del método de Newton.
  Acá consideraremos algunas simples pero útiles en la práctica,
  dadas por Kinaid y Cheney~%
    \cite{kincaid02:_numerical_analysis}.
  \begin{theorem}[Convergencia del método de Newton]
    \label{theo:Newton-convergence}
    Sea \(f''\) continua y sea \(x^*\) un cero simple de \(f\).
    Entonces hay un entorno alrededor de \(x^*\) y una constante \(C\)
    tal que si se inicia en ese entorno,
    el método de Newton converge a \(x^*\)
    y los errores sucesivos cumplen:
    \begin{equation*}
      \lvert x_{n + 1} - x^* \rvert
        \le C \lvert x_n - x^* \rvert^2
    \end{equation*}
  \end{theorem}
  \begin{proof}
    En términos de \(e_n = x_n - x^*\)
    la iteración de Newton puede escribirse:
    \begin{align*}
      x_{n + 1}
        &= x_n - \frac{f(x_n)}{f'(x_n)} \\
      e_{n + 1}
        &= e_n - \frac{f(x_n)}{f'(x_n)} \\
        &= \frac{f'(x_n) e_n - f(x_n)}{f'(x_n)}
    \end{align*}
    Por el teorema de Taylor hay \(\xi_n\) entre \(x^*\) y \(x_n\) tal que::
    \begin{align*}
      f(x^*)
        &= f(x_n - e_n) \\
      0
        &= f(x_n) - f'(x_n) e_n + \frac{1}{2}f''(\xi_n) e_n^2
    \end{align*}
    Reemplazando,
    tenemos que:
    \begin{equation}
      \label{eq:Newton-error}
      e_{n + 1}
        = - \frac{f''(\xi_n)}{2 f(x_n)} e_n^2
    \end{equation}
    Dado \(\delta\) definamos \(c(\delta)\) mediante:
    \begin{equation*}
      c(\delta)
        = \frac{1}{2}
            \max_{\lvert x - x^* \rvert \le \delta}\{\lvert f''(x) \rvert\}
              / \min_{\lvert x - x^* \rvert \le \delta}\{\lvert f'(x) \rvert\}
    \end{equation*}
    Como \(x^*\) es un cero simple,
    \(f'(x^*) \ne 0\),
    como \(f'(x)\) es continua
    (porque \(f\) tiene segunda derivada continua),
    hay \(\delta\) tal que el denominador es positivo
    (\(f'(x) \ne 0\) en el rango).
    Elijamos \(\delta\) suficientemente chico
    para que \(\delta c(\delta) < 1\).
    Entonces,
    si \(\lvert e_n \rvert \le \delta\):
    \begin{align*}
      \lvert e_{n + 1} \rvert
        &\le \left\lvert \frac{f''(\xi_n) e_n^2}{2 f'(x_n)} \right\rvert \\
        &\le c(\delta) e_n^2 \\
        &\le \delta c(\delta) \lvert e_n \rvert \\
        &<   \lvert e_n \rvert
    \end{align*}
    Vale decir,
    la secuencia \(\langle e_n \rangle\) converge a 0 cuando \(n \to \infty\)
    si \(\lvert x_0 - x^* \rvert < \delta\).
    Podemos tomar \(C = c(\delta)\).
  \end{proof}
  Esto es muy engorroso,
  podemos deducir condiciones más simples de acá.
  \begin{corollary}[Convergencia del método de Newton (práctico)]
    \label{cor:Newton-convergence-simple}
    Sea \(f\) una función que tiene un cero,
    creciente,
    convexa,
    de segunda derivada continua.
    Entonces \(f\) tiene un único cero
    y el método de Newton converge desde todos los puntos de partida.
  \end{corollary}
  \begin{proof}
    Si la función es creciente,
    \(f'(x) > 0\);
    si es convexa,
    \(f''(x) > 0\).
    Por~\eqref{eq:Newton-error},
    vemos que independiente del signo de \(e_n\) es \(e_{n + 1} > 0\),
    o sea \(x_{n + 1} > x^*\).
    Como \(f\) es creciente,
    \(f(x_{n + 1}) > 0\).
    Como \(f'(x_{n + 1}) > 0\)
    (la función es creciente),
    \(0 < e_{n + 2} < e_{n + 1}\).
    Concluimos que la secuencia \(\langle e_n \rangle\) converge a cero.
  \end{proof}

\section{Método de la Secante}

  Considere la figura \ref{02::secante:grafico},
  \begin{figure}[ht]
    \centering
    % The function interpolates:
    % 0.05  0.80
    % 0.15  0.40
    % 0.30  0.15
    % 0.40  0.00
    % 0.80 -0.30
    % See ../zero-finding/function.mc
    \begin{tikzpicture}
      \begin{axis}[
                    axis lines = center,
                    xlabel = {\(x\)},
                    ylabel = {\(y\)},
                    every axis y
                      label/.style = {at = (current axis.above origin),
                        anchor = south},
                    every axis x
                      label/.style = {at = (current axis.right of origin),
                        anchor = west},
                    xmin = 0,	 xmax = 0.5,
                    ymin = -0.6, ymax = 1.2,
                    xtick = {1/20, 3/10, 93/260},
                    xticklabels = {\small \(x_0\),
                                   \small \(x_1\),
                                   \small \(x_2\)},
                    ytick = {0},
                    yticklabels = {},
                    height = 6cm, width = 9cm
                  ]
        \addplot [black!70!blue!80, samples = 100, smooth, tension = 0.8]
           {(93 - 260 * x) / 100};
        \addplot [domain = 0.01:0.5, red!80!black, samples = 100]
           {(4740000 * x^4 - 7646000 * x^3 + 4231950 * x^2
              - 1167595 * x + 157926) / 136500};

        \draw [dashed, blue] (axis cs:1/20, 0) -- (axis cs:1/20, 4/5);
        \draw [dashed, blue] (axis cs:3/10, 0) -- (axis cs:3/10, 3/20);
      \end{axis}
    \end{tikzpicture}
    \caption{Una iteración del método de la secante.}
    \label{02::secante:grafico}
  \end{figure}
  donde interpolamos linealmente
  entre los puntos \((x_n, y_n)\) y \((x_{n + 1}, y_{n + 1}\)
  y hallamos el intercepto en el eje \(X\)
  de la recta resultante en  \(x_{n + 2}\)::
  \begin{equation}
    x_{n+2}
      = x_{n+1} - f(x_{n+1)} \cdot \frac{x_{n+1} - x_n}{f(x_{n+1)} - f(x_n)}
  \end{equation}
  Considere nuevamente la aproximación~\eqref{eq:f-aproximado}:
  \begin{align*}
    x_{n+2}
      &= x_{n + 1} - f(x_{n + 1}) \frac{x_{n + 1} - x_n}
                                       {f(x_{n + 1}) - f(x_n)} \\
    e_{n + 2}
      &= e_{n + 1} - f(x_{n + 1})
          \frac{e_{n + 1} - e_n}{f(x_{n + 1}) - f(x_n)} \\
      &= \frac{e_n f(x_{n + 1}) - e_{n + 1} f(x_n)}
              {f(x_{n + 1}) - f(x_n)} \\
      &= \frac{e_{n + 1} \left(
                           f'(x^*) e_n (1 + M e_n + O(e_n^2))
                         \right)
                 - e_n \left(
                          f'(x^*) e_{n + 1} (1 + M e_{n + 1} + O(e_{n + 1}^2))
                        \right)}
              {f'(x^*) e_n (1 + M e_n + O(e_n^2))
                 - f'(x^*) e_{n + 1} (1 + M e_{n + 1} + O(e_{n + 1}^2))} \\
      &= \frac{M e_n e_{n + 1} f'(x^*)
                 (e_n - e_{n + 1} + O(e_n^2 + e_{n + 1}^2))}
              {f'(x^*) (e_n - e_{n + 1} + O(e_n^2) + O(e_{n + 1}^2))} \\
      &= M e_n e_{n + 1}
           \frac{1 + O(e_n + e_{n + 1})}{1 + O(e_n + e_{n + 1})} \\
      &= M e_n e_{n + 1} \left( 1 + O(e_n + e_{n + 1}) \right)
  \end{align*}
  Suponemos que \(e_{n + 1} = O(e_n)\)
  (los errores disminuyen),
  con lo que:
  \begin{equation}
    \label{eq:secant-error}
    e_{n + 2}
      = M e_n e_{n + 1} \left( 1 + O(e_n) \right)
  \end{equation}
  Si suponemos:
  \begin{equation*}
    e_{n + 1}
      = c e_n^p \left( 1 + O(e_n) \right)
  \end{equation*}
  tenemos:
  \begin{align*}
    e_{n + 2}
      &= c e_{n + 1} \left( 1 + O(e_{n + 1}) \right) \\
      &= c \left( c e_n^p \left( 1 + O(e_n) \right) \right)^p
           \left( 1 + O(e_{n + 1}) \right) \\
      &= c^{p + 1} e_n^{p^2}
           \left( 1 + O(e_n) \right)
           \left( 1 + O(e_{n + 1}) \right) \\
      &= c^{p + 1} e_n^{p^2}
           \left( 1 + O(e_n) \right)
  \end{align*}
  De~\eqref{eq:secant-error} tenemos entonces:
  \begin{align*}
    c^{p + 1} e_n^{p^2} \left( 1 + O(e_n) \right)
      &= c \cdot c e_n^p \left( 1 + O(e_n) \right)
           e_n \left( 1 + O(e_n) \right) \\
    c^{p + 1} e_n^{p^2}
      &= c^2 e_n^{p + 1} \left( 1 + O(e_n) \right)
  \end{align*}
  Comparando las potencias de \(e_n\):
  \begin{equation*}
    p^2
      = p + 1
  \end{equation*}
  De acá,
  dado que \(p > 0\):
  \begin{equation*}
    p
      = \tau
      = \frac{1 + \sqrt{5}}{2}
      \approx 1,618
  \end{equation*}
  Por lo tanto,
  el método de la secante es \emph{superlineal}.
  Tiene la ventaja de ser simple y rápido,
  y no requiere derivadas
  (en muchos casos la función no se conoce explícitamente,
   es el resultado de un proceso complejo).

\section{Iteración de punto fijo}
\label{sec:FPI}

  El método de iteración de punto fijo es simple,
  y adecuado en muchas situaciones.
  Más aún,
  todos los métodos iterativos vistos son de la forma:
  \begin{equation*}
    x_{n + 1}
      = g(x_n)
  \end{equation*}
  con lo que estudiar sus propiedades en mayor detalle
  ilumina los métodos más complejos.

  \begin{definition}
    Sea \(g(x)\) una función.
    Un \emph{punto fijo} de \(g\) es \(x^*\) tal que \(x^* = g(x^*)\).
  \end{definition}
  \begin{theorem}[Punto fijo de Brouwer, una dimensión]
    Sea \(g \colon [a, b] \to [a, b]\) una función continua.
    Entonces \(g\) tiene \emph{al menos} un punto fijo en \([a, b]\).
  \end{theorem}
  \begin{proof}
    Por definición de \(g\),
    sabemos:
    \begin{equation*}
      a \le g(x) \le b
    \end{equation*}
    En particular,
    \(a \le g(a) \le b\) y \(a \le g(b) \le b\).
    Si alguna de las dos se cumple con igualdad estamos listos.

    Consideremos entonces \(a < g(a) < b\) y \(a < g(b) < b\),
    sea \(f(x) = g(x) - x\).
    Vemos que \(f\) es continua,
    con \(f(a) = g(a) - a > 0\)
    y \(f(b) = g(b) - b <0\).
    Por el teorema del valor intermedio,
    hay \(x^* \in [a, b]\) tal que \(f(x^*) = 0\),
    o sea, \(x^* = g(x^*)\).
  \end{proof}
  \begin{definition}
    Sea \(g \colon [a, b] \to [a, b]\).
    Se dice que \(g\) es una \emph{contracción}
    si existe \(L\), \(0 < L < 1\),
    tal que para todo \(x, y \in [a, b]\) es:
    \begin{equation}
      \lvert g(x) - g(y) \rvert
        \le L \lvert x - y \rvert
    \end{equation}
    (condición de Lipschitz,
     \(L\) es la constante de Lipschitz).
  \end{definition}
  \begin{theorem}[Contraction Mapping]
    Suponga que \(g \colon [a, b] \to [a, b]\) es continua
    y cumple la condición de Lipschitz.
    Entonces tiene un único punto fijo en \([a, b]\).
  \end{theorem}
  \begin{proof}
    Por el teorema de Brouwer,
    \(g\) tiene al menos un punto fijo.
    Para demostrar que es único,
    supongamos puntos fijos \(c_1\), \(c_2\):
    \begin{equation}
      \lvert c_1 - c_2 \rvert
        =   \lvert g(c_1) - g(c_2) \rvert
        \le L \lvert c_1 - c_2 \rvert
    \end{equation}
    Como \(0 < L < 1\),
    esto es posible solo si \(c_1 = c_2\).
  \end{proof}
  Esto es algo bastante obvio,
  ya que en el fondo tomamos un área más grande
  y en cada iteración la vamos reduciendo hasta tal punto que \(c_1 = c_2\)
  (figura \ref{03::ContractionMapping}).
  \begin{figure}[ht]
    \centering
    \begin{tikzpicture}
      \draw plot [smooth, tension=0.8, smooth cycle]
        coordinates {(0.8, -1) (1.5, -1.3) (2.1, -1.2) (2.4, -1)
                     (2.5, -0.4) (2.4, 0.4) (2.5, 1) (2.3, 1.4)
                     (1.6, 1.3) (1, 1) (0.6, 0.7) (0.9, 0)};

      \draw plot [smooth, tension=0.8, smooth cycle]
        coordinates {(0.5+4, -1) (1.5+4, -1.3) (2.1+4, -1.2)
                     (2.4+4, -1) (2.5+4, -0.4) (2.4+4, 0.4) (2.5+4, 1)
                     (2.3+4, 1.4) (1.6+4, 1.3) (1+4, 1) (0.6+4, 0.9)
                     (0.6+4, 0)};

      \draw [red!90!black!90] plot [smooth, tension = 0.8, smooth cycle]
        coordinates {(4.7, 0) (5.0, 0.5) (5.6, 0.6) (6.1, 0.6)
                     (6.3, -0.5) (5.9, -0.7) (5.6, -0.5) (5.2, -0.7)};

      \node at (1.4, 0.5) (x1) {\color{blue}\(c_1\)};

      \node at (1.6, -0.9) (x2) {\color{blue}\(c_2\)};

      \node at (1.8+4, -0.3) (x3) {\(g(c_2)\)};

      \node at (1.3+4, .20) (x4) {\(g(c_1)\)};

      \draw [-latex', blue] plot [smooth, tension = 0.8]
        coordinates {(1.6, 0.55) (3.7, 0.7) (0.95+4, .22)}
          node [yshift = 20pt, xshift = -50pt] (g1) {\(g\)};

      \draw [-latex', blue] plot [smooth, tension = 0.8]
        coordinates {(1.8, -0.85) (4, -0.9) (1.4+4, -0.35)}
          node [yshift = -25pt, xshift = -60pt] (g1) {\(g\)};
    \end{tikzpicture}
    \caption{Acotamos el área hasta converger en un punto.}
     \label{03::ContractionMapping}
   \end{figure}
  Definamos la secuencia:
  \begin{equation}
    x_{n+1}
      = g(x_n)
  \end{equation}
  Si \(g\) es una contracción en \([a, b]\),
  la secuencia converge al punto fijo \(x^*\) de \(g\) en \([a, b]\).

  De partida,
  si
  \begin{equation*}
    \lim_{n\rightarrow \infty} x_n
      = x^*
  \end{equation*}
  existe,
  es un punto fijo de \(g\).
  Si \(x_0 \in [a, b]\),
  consideremos:
  \begin{align}
    \lvert x_{n+1} - x^* \rvert
      &= \lvert g(x_n) - g(x^*) \rvert \\
      &\le L \lvert x_n - x^* \rvert
           \label{03::DesigualdadContractionInitial} \\
      &\le \ldots \\
      &\le L^{n+1} \lvert x_0 - x^* \rvert
           \label{03::DesigualdadContraction}
  \end{align}
  Como \(\lvert L \rvert < 1\),
  \(L^n \to 0\),
  y el lado izquierdo también tiende a \num{0}
  (para llegar a \eqref{03::DesigualdadContraction}
   solo tenemos que desarrollar
   paso a paso~\eqref{03::DesigualdadContractionInitial}).

  Supongamos que queremos llegar a \(\lvert x_n-x^* \rvert\le \varepsilon\).
  Sabemos que \(\lvert x_n - x^* \rvert \le L^{n} \lvert x_0 - x^* \rvert\).
  Queremos deshacernos del \(x^*\) desconocido al lado derecho:
  \begin{align*}
    \lvert x_0 - x^* \rvert
      &=   \lvert (x_0-x_1) + (x_1-x^*) \rvert \\
      &\le \lvert x_0 - x_1 \rvert + \lvert x_1 - x^* \rvert \\
      &\le \lvert x_0 - x_1 \rvert +L \lvert x_0 - x^* \rvert \\
    (1 - L) \lvert x_0 - x^* \rvert
      &\le \lvert x_1 - x_0 \rvert \\
    \lvert x_0 - x^* \rvert
      &\le \frac{\lvert x_1-x_0 \rvert}{1 - L}
  \end{align*}
  O sea:
  \begin{equation}
    \lvert x_n - x^* \rvert
      \le \frac{L^n}{1 - L} \lvert x_1 - x_0 \rvert
  \end{equation}
  Queremos \(\lvert x_n - x^* \rvert \le \varepsilon\):
  \begin{align*}
    \varepsilon
      &\le \frac{L^n}{1 - L} \lvert x_1 - x_0 \rvert \\
    L^n
      &\ge \frac{\varepsilon(1 - L)}{\lvert x_1 - x_0 \rvert} \\
    n
      &\ge \frac{1}{\ln L}
              \cdot \ln \frac{\varepsilon(1 - L)}{\lvert x_1 - x_0 \rvert}
  \end{align*}
  No hemos supuesto \(g\) diferenciable,
  pero en casos de interés lo es.

  La condición de Lipschitz es:
  \begin{align*}
    \frac{\lvert g(x) - g(y) \rvert}{\lvert x - y \rvert}
       & \le L \\
    \left\lvert \frac{g(x) - g(y)}{x - y} \right\rvert
       & \le L
  \end{align*}
  Por el teorema del valor intermedio
  (ver por ejemplo las notas de Chen~%
    \cite{chen08:_first_year_calculus}):
  \begin{equation}
    \frac{g(x) - g(y)}{x - y}
      = g'(\zeta),
           \qquad x \le \zeta \le y
  \end{equation}
  por lo tanto,
  \(\lvert g'(\zeta) \rvert \le L\) para \(\zeta \in [a, b]\)
  es condición suficiente para Lipschitz,
  se aplica el teorema de contraction mapping
  y hay un único punto fijo en \([a, b]\) y \(x_{n+1} = g(x_n)\) converge.

  No buscamos encontrar \(\zeta\),
  solo demostrar que existe.
  Y por favor, \(\zeta\) se lee \textquote{zeta}.

\subsection{Análisis de convergencia}
\label{sec:FPI-convergence}

  Igual que antes,
  nos interesa analizar la convergencia de esta técnica.
  Usamos nuevamente series de Taylor:
  \begin{align}
    g(x)
      &= g(x^*) + g'(x^*) (x - x^*) + O((x - x^*)^2) \notag \\
      &= x^* + g'(x^*) (x - x^*) + O((x - x^*)^2)
           \label{eq:g-approx-FPI}
  \end{align}
  Igual que antes,
  con \(e_n = x_n - x^*\) obtenemos:
  \begin{align*}
    x_{n + 1}
      &= g(x_n) \\
      &= x^* + g'(x^*) (x_n - x^*) + O((x_n - x^*)^2) \\
    e_{n + 1}
      &= g'(x^*) e_n + O(e_n^2)
  \end{align*}
  Vemos que la convergencia es lineal si \(g'(x^*) \ne 0\),
  y en particular que converge solo si \(\lvert g'(x^*) \rvert < 1\).

  Más generalmente,
  si las primeras \(p - 1\)~derivadas de \(g\) se anulan en \(x^*\),
  pero la \(p\)\nobreakdash-ésima no,
  de la misma serie de Taylor tenemos:
  \begin{equation*}
    e_{n + 1}
      = \frac{1}{p!} g^{(p)}(x^*) e_n^p + O(e_n^{p + 1})
  \end{equation*}
  La convergencia es de orden~\(p\).

\section{El proceso de extrapolación de Aitken}
\label{sec:Aitken}

  Supongamos que tenemos una secuencia \(\langle x_k \rangle\)
  que converge a \(x^*\) de orden \(p\),
  vale decir existe una constante \(A \ne 0\) tal que:
  \begin{equation}
    \label{eq:linear-limit}
    \lim_{n \to \infty} \frac{x_{n + 1} - x^*}{(x_n - x^*)^p}
      = A
  \end{equation}
  Esto significa que la fracción indicada
  para \(n\) suficientemente grande es aproximadamente el límite.
  Podemos plantear para una aproximación \(x_n^+\) a \(x^*\):
  \begin{equation}
    \label{eq:Aitken-x+-equation}
    \frac{x_{n + 2} - x_n^+}{x_{n + 1} - x_n^+}
      = A
      = \frac{x_{n + 1} - x_n^+}{x_n - x_n^+}
  \end{equation}
  Resolviendo la ecuación~\eqref{eq:Aitken-x+-equation}
  para \(x_n^+\) obtenemos:
  \begin{align}
    x_n^+
      &= \frac{x_{n + 2} x_n - x_{n + 1}^2}{x_{n + 2} - 2 x_{n + 1} + x_n}
            \notag \\
      &= x_n - \frac{(\Delta x_n)^2}{\Delta^2 x_n}
            \label{eq:Aitken-x+}
  \end{align}
  La primera versión es muy inestable numéricamente
  (se restan entre sí valores muy parecidos,
   y el valor se obtiene directamente de estas restas),
  por lo que reordenamos para dar la segunda.
  En la ecuación~\eqref{eq:Aitken-x+} usamos la definición de diferencias:
  \begin{align*}
    \Delta x_n
      &= x_{n + 1} - x_n \\
    \Delta^2 x_n
      &= \Delta x_{n + 1} - \Delta x_n \\
      &= x_{n + 2} - 2 x_{n + 1} + x_n
  \end{align*}
  Por la fórmula~\eqref{eq:Aitken-x+}
  se le llama \emph{proceso delta cuadrado de Aitken}~%
    \cite{aitken27:_delta_squared}.

  Para análisis de convergencia
  veamos qué ocurre si:
  \begin{equation*}
    g(x)
      = g(x^*)
         + g'(x^*) (x - x^*)
         + \frac{1}{2} g''(x^*) (x - x^*)^2 + O((x - x^*)^3)
  \end{equation*}
  Reemplazando \(e = x - x^*\) en~\eqref{eq:Aitken-x+}
  y expandiendo en serie de Taylor alrededor de \(e_n = 0\),
  cortesía de Maxima~%
    \cite{maxima20:_5.44.0},
  vemos que:
  \begin{equation*}
    e_n^+
      = \frac{g'(x^*) g''(x^*)}{2 g'(x^*) - 2} e_n^2 + O(e_n^3)
  \end{equation*}
  Vale decir,
  si la iteración original converge linealmente
  la iteración resultante converge cuadráticamente.
  Note que si \(g'(x^*) = 1\) la iteración original no converge
  y tampoco converge esto;
  si \(g'(x^*) = -1\) esto converge cuadráticamente,
  aunque la iteración original no converge.

  Igual resulta de interés ver qué pasa si se aplica a una secuencia
  con orden de convergencia mayor.
  Supongamos convergencia de orden \(p\):
  \begin{equation*}
    g(x)
      = g(x^*) + c (x - x^*)^p + o((x - x^*)^p)
  \end{equation*}
  O sea,
  es:
  \begin{align*}
    e_{n + 1}
       &= c e_n^p (1 + o(1)) \\
    e_{n + 2}
       &= c e_{n + 1}^p (1 + o(1)) \\
       &= c^{p + 1} e_n^{p^2} (1 + o(1))
  \end{align*}
  Substituyendo y simplificando:
  \begin{align*}
    e_n^+
      &= \frac{e_{n + 2} e_n - e_{n + 1}^2}
              {e_{n + 2} - 2 e_{n + 1} + e_n} \\
      &= \frac{c^{p + 1} e_n^{p^2} (1 + o(1)) \cdot e_n
                 - c^2 e_n^{2 p} (1 + o(1))}
              {c^{p + 1} e_n^{p^2} (1 + o(1))
                 - 2 c e_n^p + e_n} \\
      &= - \frac{(c^2 e_n^{2 p} - c^{p + 1} e_n^{p^2 + 1}) (1 + o(1))}
                {(e_n - 2 c e_n^p + c^{p + 1} e_n^{p^2}) (1 + o(1))} \\
      &= -c e_n^{2 p - 1} (1 + o(1))
  \end{align*}
  El orden aumenta,
  a pesar que el método no fue diseñado para este caso.

  El algoritmo del caso es~\ref{alg:Aitken}.
  \begin{algorithm}[ht]
    \DontPrintSemicolon\Indp

    \Function{\(\mathrm{Aitken}(g, x_0, \varepsilon)\)}{
      \Repeat{\(\lvert x_0 - x_s \rvert \le \varepsilon\)}{
        \(x_s \gets x_0\) \;
        \(x_1 \gets g(x_0)\) \;
        \(x_2 \gets g(x_1)\) \;
        \(\displaystyle
          x_0 \gets x_0 - \frac{(x_1 - x_0)^2}{x_2 - 2 x_1 + x_0}\) \;
      }
      \Return \(x_0\)
    }
    \caption{Iteración de punto fijo con aceleración de Aitken}
    \label{alg:Aitken}
  \end{algorithm}

\section{Comentarios finales}
\label{sec:02-final-comments}

  Una discusión accesible de los problemas
  a resolver para construir una rutina \textquote{para uso general}
  es la de Kahan~\cite{kahan79:_calculator_solve_key}.
  El algoritmo de Brent~%
    \cite{brent71:_algor_guaran_conver_findin_zero_funct}
  es un desarrollo cuidadoso de una rutina robusta para uso general.
  Kahan~%
    \cite{kahan16:_lecture_notes_real_root_finding}
  revisa la teoría que sustenta búsqueda de ceros
  sin suponer que estamos \textquote{muy cerca}.

\section*{Ejercicios}
\label{sec:exercises-02}

  \begin{enumerate}
  \item
    Sea \(g(x)\) con un punto fijo \(x^*\),
    donde \(g'(x^*) \ne 1\).
    Defina:
    \begin{equation*}
      G(x)
        = \frac{x g'(x) - g(x)}{g'(x) - 1}
    \end{equation*}
    \begin{enumerate}
    \item
      Demuestre que la iteración \(x_{n + 1} = G(x_n)\)
      converge cerca de \(x^*\)
    \item
      ¿Cuál es el orden de convergencia de este método?
    \end{enumerate}
  \item
    Computadores tempranos tenían operaciones de multiplicación,
    pero no división.
    Plantee un algoritmo para calcular \(a^{-1}\)
    sin usar divisiones.
  \item
    El método de Olver para hallar ceros de \(f(x)\) es la iteración:
    \begin{equation*}
      x_{n + 1}
        = x_n - \frac{f(x_n)^2 f''(x_n) + 2 f(x_n) f'(x_n)^2}{2 f'(x_n)^2}
    \end{equation*}
    ¿Cuál es el orden de convergencia de este método?
  \item
    El método de Halley puede expresarse:
    \begin{align*}
      x_{n + 1}
        &= x_n
            - \frac{2 f(x_n) f'(x_n)}
                   {2 (f'(x_n))^2 - f(x_n) f''(x_n)} \\
        &= x_n - \frac{f(x_n)}{f'(x_n)}
                   \left(
                     1 - \frac{f(x_n)}{f'(x_n)}
                           \cdot \frac{f''(x_n)}{2 f'(x_n)}
                   \right)^{-1}
    \end{align*}
    Es particularmente útil si pueden simplificarse \(f(x) / f'(x)\)
    o \(f'(x) / f''(x)\).
    Tiene la ventaja de que hace pocos cálculos por iteración,
    lo que lo hace atractivo para polinomios
    en conjunto con el método de Horner
    (que permite calcular derivadas
     casi como subproducto de calcular \(f(x\))).
    ¿Cuál es el orden de convergencia de este método?
  \item
    El método de Steffensen es la iteración:
    \begin{align*}
      x_{n + 1}
        &= x_n - \frac{f(x_n)}{g(x_n)} \\
      g(x)
        &= \frac{f(x + f(x))}{f(x)} - 1
    \end{align*}
    ¿Cuál es el orden de este método?
  \item
    Es claro que para el método de Newton
    el corolario~\ref{cor:Newton-convergence-simple}
    del teorema~\ref{theo:Newton-convergence}
    tiene similares para casos en que \(f\) es decreciente y/o cóncava.
    Plantéelos y demuéstrelos.
    ¿Pueden resumirse en un solo enunciado simple?
  \end{enumerate}

\bibliography{../referencias}

%%% Local Variables:
%%% mode: latex
%%% TeX-master: "../INF-221_notas"
%%% ispell-local-dictionary: "spanish"
%%% End:

% LocalWords:  heurísticas Bisección superlineal latin falsi Mapping
% LocalWords:  Contraction diferenciable contraction mapping ésima em
% LocalWords:  extrapolación reordenamos Maxima cuadráticamente ht eq
% LocalWords:  Plantéelos demuéstrelos iteration secant equation
% LocalWords:  SymPy explicitarse

\bibliographystyle{babplain-fl}

\chapter{Otros métodos para encontrar ceros}
\label{cha:otros-metodos-ceros}

  Discutiremos algunos métodos adicionales para hallar ceros,
  junto con analizar su convergencia.

\section{Ceros múltiples}
\label{sec:multiple-zeros}

  Uno de los problemas con que se debe lidiar
  es el caso de ceros múltiples.
  La figura~\ref{fig:multiple-zero-even}
  muestra una función con un cero doble,
  \begin{figure}[ht]
    \centering
    \begin{tikzpicture}
      \begin{axis}[axis x line = middle, axis y line = left,
                   xlabel = {\(x\)},
                   ylabel ={\(f(x) = x (\mathrm{e}^{0,6 x} - 1)\)},
                   xmin = -1, xmax = 1,
                   ymin = -0.2, ymax = 1]

        \addplot [mark = none, smooth, red] {x * (exp(0.6 * x) - 1)};
      \end{axis}
    \end{tikzpicture}
    \caption{Un ejemplo de cero de multiplicidad par}
    \label{fig:multiple-zero-even}
  \end{figure}
  la figura~\ref{fig:multiple-zero-odd}
  muestra una función con un cero triple.
  \begin{figure}[ht]
    \centering
    \begin{tikzpicture}
      \begin{axis}[axis x line = middle, axis y line = left,
                   xlabel = {\(x\)},
                   ylabel ={\(f(x) = x^2 (\mathrm{e}^{x/2 - 1)} \sin x\)},
                   xmin = -1.5, xmax = 1.2,
                   ymin = -0.5, ymax = 0.8]

        \addplot [mark = none, smooth, red]
                 {x^2 * (exp(0.5 * x - 1) * sin(deg(x))};
      \end{axis}
    \end{tikzpicture}
    \caption{Un ejemplo de cero de multiplicidad impar}
    \label{fig:multiple-zero-odd}
  \end{figure}
  Se ve que lo que tienen en común
  es que en el cero la función es tangente al eje \(X\).
  Esto producirá problemas
  para métodos que se basan en la pendiente de la curva,
  como son el método de Newton o de la secante.
  El hecho que si el cero es de multiplicidad par
  éste es a su vez un mínimo
  hace inútiles métodos como el de bisección
  o \emph{\foreignlanguage{latin}{regula falsi}}
  que se basan en cambios de signo de la función.

  En el caso particular de polinomios
  tenemos una salida simple.
  Si el polinomio \(p(x)\) tiene un cero de multiplicidad \(m\) en \(a\),
  quiere decir que para algún polinomio \(q(x)\) tal que \(q(a) \ne 0\):
  \begin{align*}
    p(x)
      &= (x - a)^m q(x) \\
    p'(x)
      &= (x - a)^{m - 1} (m q(x) + (x - a) q'(x))
  \end{align*}
  Vale decir,
  en este caso podemos calcular:
  \begin{equation*}
    k(x)
      = \frac{p(x)}{\gcd(p(x), p'(x)}
    \end{equation*}
    el polinomio \(k(x)\) tiene los mismos ceros que \(p(x)\),
    pero son todos simples.

\subsection{Método de Newton modificado}
\label{sec:Newton-modificado}

  Si \(f\) tiene un cero múltiple en \(x^*\),
  digamos \(f(x) = (x - x^*)^m g(x)\)
  con \(g(x^*) \ne 0\)
  (\(x^*\) es un cero de multiplicidad \(m\)),
  podemos escribir por el teorema de Taylor:
  \begin{equation*}
    g(x)
      = g(x^*) + O(x - x^*)
  \end{equation*}
  Tenemos además:
  \begin{align*}
    f(x)
      &= (x - x^*)^m (g(x^*) + O(x - x^*)) \\
      &= (x - x^*)^m g(x^*) + O((x - x^*)^{m + 1})) \\
    f'(x)
      &= m (x - x^*)^{m - 1} g(x) + (x - x^*)^m g'(x) \\
      &= m (x - x^*)^{m - 1} (g(x^*) + O(x - x^*))
           + (x - x^*)^m O(x - x^*) \\
      &= m (x - x^*)^{m - 1} g(x^*) + O((x - x^*)^m)
  \end{align*}
  Expandiendo nuevamente la fórmula para el método de Newton:
  \begin{align*}
    e_{n + 1}
      &= e_n - \frac{e_n^m g(x^*) (1 + O(e_n))}
                    {m e_n^{m - 1} g(x^*) (1 + O(e_n))} \\
      &= e_n - \frac{e_n}{m} (1 + O(e_n)) \\
      &= \frac{m - 1}{m} e_n + O(e_n^2)
  \end{align*}
  Vale decir,
  la convergencia es lineal si \(m > 1\).

  Podemos corregir esto,
  notando que si \(f\) tiene un cero múltiple en \(x^*\),
  \(\mu(x) = f(x) / f'(x)\) tiene un cero simple,
  y se recupera la convergencia cuadrática:
  \begin{align}
    x_{n + 1}
      &= x_n - \frac{\mu(x_n)}{\mu'(x_n)} \notag \\
      &= x_n
           - \frac{f(x_n)}{f'(x_n)}
               \left( 1 - \frac{f(x_n) f''(x_n)}{f'(x_n)^2} \right)^{-1}
        \label{eq:modified-Newton}
  \end{align}
  Analizamos la convergencia,
  extendiendo las series:
  \begin{align*}
    f(x)
      &= (x - x^*)^m
           \left(
             g(x^*)
               + g'(x^*) (x - x^*)
               + \frac{1}{2} g''(x^*) (x - x^*)^2
           \right)
           + O((x - x^*)^{m + 3}) \\
    \begin{split}
      f'(x)
        &= m (x - x^*)^{m - 1}
             \left(
               g(x^*)
                 + \frac{m + 1}{m} g'(x^*) (x - x^*)
                 + \frac{m + 2}{2 m} g''(x^*) (x - x^*)^2)
             \right) \\
        &\qquad
             + O((x - x^*)^{m + 2})
    \end{split} \\
    \begin{split}
      f''(x)
        &= m (m - 1) (x - x^*)^{m - 2}
             \left(
               g(x^*)
                 + \frac{m + 1}{m - 1} g'(x^*) (x - x^*)
                 + \frac{(m + 2) (m + 1)}{2 m (m - 1)} g''(x^*) (x - x^*)^2
             \right) \\
        &\qquad
             + O((x - x^*)^{m + 1})
    \end{split}
  \end{align*}
  substituyendo en la ecuación~\eqref{eq:modified-Newton} tenemos:
  \begin{equation}
    \label{eq:modified-Newton-error}
    e_{n + 1}
      = - \frac{g'(x^*)}{m g(x^*)} e_n^2 + O(e_n^3)
  \end{equation}
  Como propusimos,
  el método de Newton modificado siempre tiene convergencia cuadrática.
  Su índice de eficiencia sique siendo \(\sqrt{2} = \num{1,4142}\).

  Resulta de interés el límite:
  \begin{equation}
    \label{eq:modified-Newton-limit}
    \lim_{x \to x^*}
      \left(
        1 - \frac{f(x) f''(x)}{f'(x)^2}
      \right)
      = \frac{1}{m}
  \end{equation}
  Por lo tanto,
  si \(f\) tiene un cero de multiplicidad \(m\) en \(x^*\),
  la iteración:
  \begin{equation}
    \label{eq:modified-Newton-constant-iteration}
    x_{n + 1}
      = x_n - m \frac{f(x_n)}{f'(x_n)}
  \end{equation}
  converge cuadráticamente a \(x^*\).

  Claro que en general no sabemos cuánto es \(m\).
  Una estrategia es estimar \(m\)
  usando la expresión del límite~\eqref{eq:modified-Newton-limit}
  al comienzo,
  y verificar el valor regularmente luego.

\section{Métodos de Householder}
\label{sec:metod-de-householder}

  Una familia de métodos de orden de convergencia arbitrario
  son los de Householder\cite{householder70:_num_single_nonlin_equation}.
  La idea es partir con una aproximación de Padé
  (la razón entre dos polinomios)
  a la función,
  en nuestro caso:
  \begin{equation}
    \label{eq:Pade}
    f(x + h)
      = \frac{a_0 + h}{b_0 + b_1 h + \dotsb + b_{d - 1} h^{d - 1}}
          + O(h^d)
  \end{equation}
  Puede demostrarse que la aproximación~\eqref{eq:Pade} es única.

  La derivación del método de Householder de orden~\(d\)
  parte con la aproximación~\eqref{eq:Pade},
  y considera la serie de Taylor de \(1/f\):
  \begin{equation*}
    \left( \frac{1}{f} \right)(x + h)
      = \left(\frac{1}{f} \right)(x)
          + \left(\frac{1}{f} \right)'(x) h
          + \frac{1}{2!} \left(\frac{1}{f} \right)''(x) h^2
          + \dotsb
          + \frac{1}{d!} \left(\frac{1}{f} \right)^{(d)}(x) h^d
          + O(h^{d + 1})
  \end{equation*}
  Esto debe coincidir con la fracción~\eqref{eq:Pade},
  en particular,
  el coeficiente de \(h^d\) se anula;
  al multiplicar por \(a_0 + h\) este es:
  \begin{equation*}
    0
      = a_0 \frac{\left(\frac{1}{f}\right)^{(d)}(x)}{d!}
          + \frac{\left(\frac{1}{f}\right)^{(d - 1)}(x)}{(d - 1)!}
  \end{equation*}
  de donde despejamos \(a_0\),
  y sabemos que la aproximación~\eqref{eq:Pade} a \(f\)
  se anula para \(h = -a_0\),
  o sea la siguiente aproximación a \(x\) se calcula:
  \begin{equation}
    x_{n + 1}
      = x_n + d \frac{\left( \frac{1}{f} \right)^{(d - 1)}(x_n)}
                     {\left( \frac{1}{f} \right)^{(d)}(x_n)}
  \end{equation}
  Por su derivación,
  es claro que el método de Householder de orden~\(d\)
  converge de orden~\(d + 1\).
  Requiere evaluar la función y \(d\)~derivadas,
  su índice de eficiencia es \((d + 1)^{1 / (d + 1)}\).
  Para \(d = 1\)
  resulta el método de Newton,
  \(d = 2\) da el método de Halley:
  \begin{align}
    \label{eq:Halley-iteration}
    x_{n + 1}
      = x_n
          - \frac{f(x_n)}{f'(x_n)}
              \left(
                1 - \frac{f(x_n) f''(x_n)}{2 f'(x_n)^2}
              \right)^{-1}
  \end{align}
  Para el método de Halley nuestras técnicas para estimar el error dan:
  \begin{equation}
    \label{eq:Halley-iteration-error}
    e_{n + 1}
      = - \frac{2 f'(x^*) f'''(x^*) - 3 f''(x^*)^2}{12 f'(x^*)^2} e_n^3
            + O(e_n^4)
  \end{equation}
  El método de Halley es cúbico,
  como esperábamos.

\section{Otro método cúbico}
\label{sec:metodo-cubico}

  La instructiva derivación siguiente es de Sebah y Gourdon~%
    \cite{sebah01:_newton_method_higher_order_iterations}.
  Podemos extender el método de Newton,
  aproximando la función
  mediante la serie de Taylor hasta el término cuadrático:
  \begin{equation*}
    f(x + h)
      = f(x) + f'(x) h + \frac{1}{2} f''(x) h^2 + O(h^3)
  \end{equation*}
  y buscar \(h\) más cercano a cero que anula el polinomio:
  \begin{equation*}
    h
      = - \frac{f'(x)}{f''(x)}
            \left(
              1 - \sqrt{1 - \frac{2 f(x) f''(x)}{f'(x)^2}}
            \right)
  \end{equation*}
  Podemos ahorrarnos la raíz cuadrada
  expandiendo para \(\alpha\) pequeño
  (buscamos \(f(x) = 0\))
  por el teorema del binomio:
  \begin{align*}
    1 - \sqrt{1 - \alpha}
      &= \frac{\alpha}{2} + \frac{\alpha^2}{8} + O(\alpha^3) \\
      &= \frac{\alpha}{2}
           \left(
             1 + \frac{\alpha}{4}
           \right)
           + O(\alpha^3)
  \end{align*}
  al retener hasta el término cuadrático:
  \begin{align*}
    h
      &\approx - \frac{f'(x)}{f''(x)}
                   \cdot \frac{f(x) f''(x)}{f'(x)^2}
                           \left(
                             1 + \frac{f(x) f''(x)}{2 f'(x)^2}
                           \right) \\
      &=       - \frac{f(x)}{f'(x)}
                   \left(
                     1 + \frac{f(x) f''(x)}{2 f'(x)^2}
                   \right)
  \end{align*}
  queda:
  \begin{equation}
    \label{eq:cubic-iteration}
    x_{n + 1}
      = x_n
          - \frac{f(x_n)}{f'(x_n)}
              \left(
                1 + \frac{f(x_n) f''(x_n)}{2 f'(x_n)^2}
              \right)
  \end{equation}
  Nuestras técnicas para estimar el error dan:
  \begin{equation}
    \label{eq:cubic-iteration-error}
    e_{n + 1}
      = - \frac{f'(x^*) f'''(x^*) - 3 f''(x^*)^2}{6 f'(x^*)^2} e_n^3
            + O(e_n^4)
  \end{equation}
  Como esperábamos,
  el método es cúbico siempre que el cero sea simple,
  y resulta comparable al método de Halley.

\section{Método de Muller}
\label{sec:metodo-de-muller}

  Una extensión del método de la secante es el método de Muller~%
    \cite{muller56:_method_solvin_algeb_equat_using_autom_comput}.
  La idea es tomar los últimos tres puntos,
  interpolar mediante un polinomio de segundo grado
  y elegir el siguiente punto como una intersección con el eje.
  En detalle,
  dados puntos
  \((x_n, y_n), (x_{n + 1}, y_{n + 1}), (x_{n + 2}, y_{n + 2})\),
  buscamos \(x_{n + 3} = x_{n + 2} + h\)
  que anula la función cuadrática que interpola entre esos puntos.
  La función cuadrática que interpola puede escribirse en la forma de Newton
  (lo veremos más adelante,
   sección~\ref{sec:Newton-interpolation}):
  \begin{align*}
    f(x)
      &= f[x_{n + 2}]
           + f[x_{n + 1}, x_{n + 2}] (x - x_{n + 1})
           + f[x_n, x_{n + 1}, x_{n + 2}]
                (x - x_{n + 2}) (x - x_{n + 1}) \\
      &= f[x_{n + 2}]
           + w (x - x_{n + 2})
           + f[x_n, x_{n + 1}, x_{n + 2}] (x - x_{n + 2})^2 \\
       &= f[x_{n + 2}]
           + w h
           + f[x_n, x_{n + 1}, x_{n + 2}] h^2
  \end{align*}
  Acá usamos diferencias divididas:
  \begin{align*}
    f[x_0]
      &= f(x_0) \\
    f[x_0, x_1]
      &= \frac{f[x_1] - f[x_0]}{x_1 - x_0} \\
    f[x_0, x_1, x_2]
      &= \frac{f[x_1, x_2] - f[x_0, x_1]}{x_2 - x_0}
  \end{align*}
  junto con:
  \begin{equation*}
    w
      = f[x_{n + 1}, x_{n + 2}] + f[x_n, x_{n + 2}] - f[x_n, x_{n + 1}]
  \end{equation*}
  Nos interesa el cero de menor valor absoluto de la cuadrática en \(h\).
  Pero eso lleva a cancelación en la fórmula cerca del cero
  (valor pequeño de \(f(x_{n + 2})\)).
  Dividiendo la cuadrática por \(h^2\) obtenemos una cuadrática en \(1 / h\),
  que resolvemos para el cero de mayor magnitud obteniendo:
  \begin{equation}
    h
      = - \frac{2 f(x_{n + 2})}
               {w + \sgn(w)
                      \sqrt{w^2
                              - 4 f(x_{n + 2}) f[x_n, x_{n + 1}, x_{n + 2}]}}
  \end{equation}
  donde usamos la función signo:
  \begin{equation*}
    \sgn(x)
      = \begin{cases}
          -1 & x < 0 \\
           0 & x = 0 \\
           1 & x > 0
        \end{cases}
  \end{equation*}
  O sea,
  la iteración es:
  \begin{equation}
    \label{eq:Muller-iteration}
    x_{n + 3}
      = x_{n + 2}
         - \frac{f(x_{n + 2})}
               {w + \sgn(w)
                      \sqrt{w^2
                              - 4 f(x_{n + 2}) f[x_n, x_{n + 1}, x_{n + 2}]}}
  \end{equation}
  Sospechando que el método es superlineal,
  pero no más que cuadrático,
  usamos:
  \begin{equation*}
    f(x^* + h)
      = f'(x^*) h
          + \frac{1}{2} f''(x^*) h^2
          + \frac{1}{6} f'''(x^*) h^3
          + O(h^4)
  \end{equation*}
  y bajo el supuesto:
  \begin{equation*}
    \lvert e_{n + 1} \rvert
      = C \rvert e_n \lvert^p
  \end{equation*}
  de la iteración~\eqref{eq:Muller-iteration}
  después de simplificar
  resulta \(p^3 - p^2 - p - 1 = 0\),
  cuyo cero real es \num{1,8392867552}.
  Como hay una única evaluación de \(f\) por iteración,
  el índice de eficiencia es \num{1,8393}.
  Tenemos:
  \begin{equation}
    C
      = \left\lvert \frac{f'''(x^*)}{6 f'(x^*)} \right\rvert^{(p - 1) / 2}
  \end{equation}

  Un problema del método de Muller
  es que aún si todos los valores son reales
  la siguiente estimación del cero puede ser compleja.

\section{Interpolación inversa}
\label{sec:interpolacion-inversa}

  En vez de interpolar los \(f(x)\)
  y buscar el punto donde la interpolación se anula,
  podemos interpolar la función inversa y evaluar para \(y = 0\).
  Esto tiene la ventaja de no requerir raíces
  (y no lleva a complejos).
  Si usamos una función cuadrática,
  de la forma de Lagrange evaluada en \(y = 0\) tenemos:
  \begin{equation}
    \label{eq:inverse-interpolation-iteration}
    x_{n + 3}
      = \frac{y_{n + 1} y_{n + 2}}
             {(y_n - y_{n + 1}) (y_n - y_{n + 2})} x_n
          + \frac{y_n y_{n + 2}}
                 {(y_{n + 1} - y_n) (y_{n + 1} - y_{n + 2})} x_{n + 1}
          + \frac{y_n y_{n + 1}}
                 {(y_{n + 2} - y_n) (y_{n + 2} - y_{n + 1})} x_{n + 2}
  \end{equation}

  Substituyendo \(x = x^* + e\),
  evaluando~\eqref{eq:inverse-interpolation-iteration}
  para \(x_0\), \(x_1\) y \(x_2\) para obtener \(x_3\),
  es claro de la fórmula que el término más significativo
  es proporcional a \(e_0 e_1 e_2\),
  calculamos el límite:
  \begin{equation*}
    \lim_{\substack{e_0 \to 0 \\
                    e_1 \to 0 \\
                    e_2 \to 0}} \frac{e_3}{e_0 e_1 e_2}
      = - \frac{f'(x^*) f'''(x^*) - 3 f''(x^*)^2}{6 f'(x^*)}
  \end{equation*}
  Bajo el supuesto \(\lvert e_{n + 1} \rvert = C \lvert e_n \rvert^p\)
  obtenemos la ecuación:
  \begin{equation*}
    C^{p^2 + p + 1} \lvert e_n \rvert^{p^3}
      = \frac{f'(x^*) f'''(x^*) - 3 f''(x^*)^2}{6 f'(x^*)}
           C^{p + 2} \lvert e_n \rvert^{p^2 + p + 1}
  \end{equation*}
  El orden de convergencia es el cero real de \(p^3 - p^2 - p - 1\),
  cuyo valor es \num{1,8392867552},
  y nuevamente este el índice de eficiencia.
  Obtenemos además:
  \begin{equation*}
    C
      = \left\lvert
          \frac{f'(x^*) f'''(x^*) - 3 f''(x^*)^2}{6 f'(x^*)}
        \right\rvert^{1/(p^2 - 1)}
  \end{equation*}

  Un problema de este método es que si resultan dos puntos iguales,
  falla.

\section{El método de Brent}
\label{sec:metodo-Brent}

  Un método híbrido muy popular es el de Brent~%
    \cite[capítulo~4]{brent73:_algo_min_no_derivatives}.
  La idea es usar un método de interpolación
  (interpolación cuadrática inversa o secante)
  si se puede,
  recurriendo a bisección si lo anterior falla
  o se ve que converge lentamente.
  Exige iniciar con puntos que acotan el cero,
  elige el tercer punto por bisección,
  de allí usa interpolación cuadrática inversa si se puede
  (los tres puntos son diferentes)
  o secante,
  y cada cierto número de pasos fuerza usar al menos un paso de bisección
  para asegurar que el rango de búsqueda se encoja con rapidez.
  Garantiza convergencia razonable,
  incluso para funciones de muy mal comportamiento;
  y aprovecha la convergencia rápida de los métodos de interpolación
  cuando son aplicables.

  La versión definitiva es el programa~\ref{lst:Brent} en Algol~60~%
    \cite{backus76:_modif_report_Algol60}
  como dado por Brent~%
    \cite[sección~4.6]{brent71:_algor_guaran_conver_findin_zero_funct},
  reindentado para gustos modernos.
  Brent lo discute en detalle,
  incluyendo consideraciones
  de error de redondeo en cálculo en punto flotante.
  El algoritmo usa el valor \(\varepsilon\)
  (argumento \lstinline[language = {Algol}]!macheps!)
  de precisión relativa de punto flotante,
  y retorna un cero con tolerancia \(6 \varepsilon \lvert b \rvert + 2 t\),
  donde \(t\) es una tolerancia positiva.
  Supone que \(f(a)\) y \(f(b)\) tienen signo opuesto.
  \lstinputlisting[float,
                   language = {Algol},
                   caption = {El algoritmo de Brent},
                   label = lst:Brent]{code/Brent.alg}
  En un paso típico,
  tenemos tres puntos  \(a\), \(b\) y \(c\),
  tales que \(f(b) f(c) < 0\),
  \(\lvert f(b) \rvert \le \lvert f(c) \rvert\),
  y \(a\) puede coincidir con \(c\).
  \(b\) es la mejor aproximación hasta el momento al cero \(x^*\),
  \(a\) es el valor previo de \(b\)
  y \(x^*\) está entre \(b\) y \(c\).
  Inicialmente \(a = c\).

  Si \(f(b) = 0\),
  estamos listos.
  Esto puede ocurrir por calcular \(f\) en forma aproximada.

  Si \(f(b) \ne 0\),
  sea \(m = \frac{1}{2} (c - b)\).
  No se retorna \(\frac{1}{2} (b + c)\)
  en cuanto \(\lvert m \rvert \le 2 \delta\),
  ya que si tenemos convergencia superlineal
  probablemente \(b\) sea mucho mejor aproximación a \(x^*\).
  Si \(\lvert m \rvert \le \delta\),
  retornamos \(b\)
  (sabemos que el error es a lo más \(\delta\) en tal caso).

  Si no se dan las anteriores,
  interpolamos (o extrapolamos) linealmente entre \(a\) y \(b\),
  dando un nuevo punto \(i\).
  Más adelante discutimos la interpolación cuadrática.
  Para evitar problemas de rebalse o división por cero,
  buscamos números \(p\) y \(q\) tales que \(i = b + p / q\),
  y omitimos la división si \(2 \lvert p \rvert \ge 3 \lvert m q \rvert\)
  (no se requiere en tal caso).
  Como \(0 < \lvert f(b) \rvert \le \lvert f(a) \rvert\)
  (vea más adelante),
  podemos calcular \(s = f(a) / f(b)\),
  \(p \pm (a - b) s\) y \(q \mp (1 - s)\)
  sin meternos en problemas.
  Ahora definimos:
  \begin{align*}
    b''
      &= \begin{cases}
           i	 & \text{si \(i\) está entre \(b\) y  \(b + m\)
                          (interpolación)} \\
           b + m	 & \text{caso contrario
                          (bisección)}
         \end{cases} \\
    b'
      &= \begin{cases}
           b''		      & \lvert b - b'' \rvert > \delta \\
           b + \delta \sgn(m)  & \text{caso contrario
                                       (paso \(\delta\))}
         \end{cases}
  \end{align*}
  Aún evitando pasos \(\delta\) la convergencia puede ser muy lenta,
  el algoritmo fuerza ocasionales bisecciones:
  sea \(e\) el valor de \(p / q\) el penúltimo paso,
  si \(\lvert e \rvert < \delta\)
  o \(\lvert p / q \rvert \ge \frac{1}{2} \delta\),
  haga una bisección.
  O sea,
  \(e\) se reduce al menos en un factor de \num{2} cada segundo paso,
  y si \(\lvert e \rvert < \delta\) damos un paso de bisección.
  Luego de una bisección el avance es \(e = m\).
  Experimentos mostraron que el criterio más simple
  de usar \(p / q\) del último paso
  frena la convergencia para funciones de buen comportamiento.

  Si los puntos \(a\), \(b\) y \(c\) son diferentes,
  podemos recurrir a interpolación cuadrática inversa,
  que debiera dar una mejor estimación de \(x^*\).
  En este proceso hay que tener cuidado
  de no caer en rebalse o división por cero.
  Como \(b\) es la aproximación más reciente
  y \(a\) es el valor previo de \(b\),
  si \(\lvert f(b) \rvert \ge \lvert f(a) \rvert\)
  damos un paso de bisección.
  En caso contrario,
  es \(\lvert f(b) \rvert < \lvert f(a) \rvert \le \lvert f(c) \rvert\),
  una forma segura de hallar \(i\) es calcular:
  \begin{align*}
    r_1
      &= \frac{f(a)}{f(c)} \\
    r_2
      &= \frac{f(b)}{f(c)} \\
    r_3
      &= \frac{f(b)}{f(a)} \\
    p
      &= \pm r_3 ((c - b) r_1 (r_1 - r_2) - (b - a) (r_2 - 1)) \\
    q
      &= \mp (r_1 - 1) (r_2 - 1) (r_3 - 1)
  \end{align*}
  con lo que \(i = b + p / q\),
  pero
  (igual que antes)
  no efectuamos la división a menos que resulte útil.
  Pero si entre \((b, f(b))\) y \((c, f(c))\)
  no estamos en la misma rama de la parábola,
  esta no es una buena aproximación a \(f\),
  aceptamos \(i\) solo si está entre \(b\) y \(c\),
  y hasta tres cuartas partes del camino de \(b\) a \(c\)
  (considere el caso extremo de \(f(b) = - f(c)\),
   con la parábola con una tangente vertical en \((c, f(c))\)).
  O sea,
  rechazamos \(i\) si \(2 \lvert p \rvert \ge 3 \lvert m q \rvert\).

\section*{Ejercicios}
\label{sec:ejercicios-other-zeros}

  \begin{enumerate}
  \item
    Una forma alternativa
    de derivar la iteración de Halley~\eqref{eq:Halley-iteration}
    es aplicar el método de Newton a la función:
    \begin{equation*}
      \mu(x)
        = \frac{f(x)}{\sqrt{\lvert f'(x) \rvert}}
    \end{equation*}
    lo que es válido siempre que el cero sea simple
    (\(f'(x^*) \ne 0\)).
  \item
    Analice la convergencia de los métodos cúbicos para ceros múltiples.
  \end{enumerate}

\bibliography{../referencias}

%%% Local Variables:
%%% mode: latex
%%% TeX-master: "../INF-221_notas"
%%% ispell-local-dictionary: "spanish"
%%% End:

% LocalWords:  cuadráticamente superlineal Interpolación bisección eq
% LocalWords:  interpolación Algol reindentado bisecciones modified
% LocalWords:  limit Pade iteration inverse interpolation latin falsi

\bibliographystyle{babplain-fl}

\chapter{Sistemas de ecuaciones no lineales}
\label{cha:sistemas-ecuaciones}

  Hay situaciones en que estamos interesados
  en resolver sistemas de ecuaciones,
  tanto lineales como no lineales.
  La solución de sistemas lineales ya se vio en el colegio,
  y se profundizará bastante en ramos posteriores.
  Hay una muy linda teoría al respecto
  (el álgebra lineal,
   ver por ejemplo a Treil~%
     \cite{treil17:_linear_algeb_done_wrong}),
  y un conjunto fascinante de algoritmos al efecto.
  Acá nos interesan sistemas no lineales,
  en los cuales al menos una variable aparece en una dependencia no lineal.

\section{Iteración de punto fijo}
\label{sec:FPI-multiple}

  Nuestro problema es el siguiente:
  hallar el vector \(\mathbf{x}\) tal que se cumple:
  \begin{equation}
    \label{eq:nonlinear-system}
    \mathbf{f}(\mathbf{x})
      = \mathbf{0}
  \end{equation}
  Claramente,
  las dimensiones de \(\mathbf{x}\) y de \(\mathbf{f}\)
  deben ser iguales.
  En forma explícita nuestro problema es:
  \begin{align*}
      f_1(x_1, x_2, \dotsc, x_n)
        &= 0 \\
      f_2(x_1, x_2, \dotsc, x_n)
        &= 0 \\
        &\vdots \\
      f_n(x_1, x_2, \dotsc, x_n)
        &= 0
  \end{align*}
  Formalmente la ecuación~\eqref{eq:nonlinear-system}
  es idéntica al problema tratado en la sección~\ref{sec:FPI},
  nuevamente podremos escribir:
  \begin{equation*}
    \mathbf{x}
      = \mathbf{g}(\mathbf{x})
  \end{equation*}
  (siempre funciona la transformación
    \(\mathbf{g}(\mathbf{x})
         = \mathbf{x} + \mathbf{A}\mathbf{f}(\mathbf{x})\)
   para una matriz \(\mathbf{A}\) no singular)
  y nos interesa un punto fijo \(\mathbf{x^*}\) de \(\mathbf{g}\).

\section{Métodos de Newton y cuasi-Newton}
\label{sec:cuasi-Newton}

  Sistemas de ecuaciones pueden expresarse:
  \begin{equation*}
    \mathbf{f}(\mathbf{x})
      = \mathbf{0}
  \end{equation*}
  Muchas veces las funciones no están dadas explícitamente,
  pueden ser el resultado de un complicado cálculo
  o incluso ser el resultado de un experimento.

  Usando la versión multivariable del teorema de Taylor,
  podemos aproximar:
  \begin{equation}
    \label{eq:Taylor-multi}
    \mathbf{f}(\mathbf{x})
      = \mathbf{f}(\mathbf{x}^*)
          + \mathbf{f}'(\mathbf{x}^*) (\mathbf{x} - \mathbf{x}^*)
          + \dotsb
  \end{equation}
  Acá \(\mathbf{f}'(\mathbf{x})\) es la matriz jacobiana:
  \begin{equation}
    \label{eq:Newton-multi}
    \mathbf{f}'(\mathbf{x})
      = \begin{pmatrix}
          \frac{\partial f_i}{\partial x_j}
        \end{pmatrix}
  \end{equation}
  Como el método de Newton en una dimensión,
  esto sugiere la iteración:
  \begin{equation*}
    \mathbf{x}_{k + 1}
      = \mathbf{x}_k
          - (\mathbf{f}'(\mathbf{x}_k))^{-1} \mathbf{f}(\mathbf{x}_k)
  \end{equation*}
  En forma muy similar al caso en una dimensión,
  puede demostrarse que el método converge en forma cuadrática:
  \begin{equation*}
    \lVert \mathbf{x}_{k + 1} - \mathbf{x}^* \rVert
      \le \alpha \lVert \mathbf{x}_k - \mathbf{x}^* \rVert^2
  \end{equation*}
  Lo malo es que~\eqref{eq:Newton-multi} en cada iteración
  requiere calcular \(n^2\) derivadas
  si \(\mathbf{f}\) y \(\mathbf{x}\) son de dimensión \(n\),
  y resolver el sistema de ecuaciones implícito en esto
  requiere \(O(n^3)\) operaciones.
  Nos interesa deducir un método similar al de la secante,
  menos costoso.
  Hay una variedad de técnicas,
  que en su conjunto se conocen como \emph{métodos cuasi-Newton}.
  Denis y Moré~%
   \cite{dennis77:_quasi_newton_methods}
  y Martínez~%
    \cite{martinez00:_pract_quasi_newton_method_solving_nonlin_system}
  discuten la teoría y exploran variantes.
  Una técnica reciente en esta línea es la de Klement~%
    \cite{klement14:_using_quasi_newton_algor}.
  Acá consideraremos el primero y más simple de ellos,
  el método de Broyden~%
    \cite{broyden65:_solving_nonlin_simul_equat}.

  La idea base
  es aproximar la matriz jacobiana \(\mathbf{f}'(\mathbf{x}_k)\)
  mediante una matriz \(\mathbf{B}_k\),
  calculada
  usando \(\mathbf{f}(\mathbf{x}_k)\) y \(\mathbf{f}(\mathbf{x}_{k - 1})\)
  junto con \(\mathbf{B}_{k - 1}\).
  En vista de~\eqref{eq:Taylor-multi},
  es razonable exigir:
  \begin{equation*}
    \mathbf{f}(\mathbf{x}_k)
      = \mathbf{f}(\mathbf{x}_{k - 1})
          + \mathbf{B}_k (\mathbf{x}_k - \mathbf{x}_{k - 1})
  \end{equation*}
  Para simplificar notación,
  llamamos:
  \begin{align*}
    \mathbf{s}_k
      &= \mathbf{x}_k - \mathbf{x}_{k - 1} \\
    \mathbf{y}_k
      &= \mathbf{f}(\mathbf{x}_k) - \mathbf{f}(\mathbf{x}_{k - 1})
  \end{align*}
  y escribimos:
  \begin{equation}
    \label{eq:quasi-Newton-condition}
    \mathbf{B}_k \mathbf{s}_k
      = \mathbf{y}_k
  \end{equation}
  Si \(n = 1\)
  (el número de ecuaciones,
   y la dimensión de las matrices \(\mathbf{B}_k\)),
  la ecuación~\eqref{eq:quasi-Newton-condition}
  define \(\mathbf{B}_k\) en forma única,
  es el método de secante.
  Si \(n > 1\),
  podemos argüir que solo conocemos la variación de \(\mathbf{f}\)
  a lo largo de \(\mathbf{s}_k\);
  si tenemos una aproximación previa \(\mathbf{B}_{k - 1}\),
  el cambio no aporta nueva información
  en direcciones ortogonales a \(\mathbf{s}_k\),
  vale decir debiéramos exigir:
  \begin{equation}
    \label{eq:Broyden-condition}
    \mathbf{B}_k \mathbf{z}
      = \mathbf{B}_{k - 1} \mathbf{z}
      \quad\text{si \(\langle \mathbf{s}_k, \mathbf{z} \rangle = 0\)}
  \end{equation}
  Resulta que~\eqref{eq:quasi-Newton-condition} y~\eqref{eq:Broyden-condition}
  determinan \(\mathbf{B}_k\) en forma única.
  Llamemos \(\mathbf{B}_k = \mathbf{B}_{k - 1} + \mathbf{X}\),
  en esos términos~\eqref{eq:Broyden-condition}
  dice que cada fila de \(\mathbf{X}\) es un múltiplo de \(\mathbf{s}_k^T\).
  Para simplificar notación,
  temporalmente anotemos:
  \begin{align*}
    \overline{\mathbf{B}}
      &= \mathbf{B}_k \\
    \mathbf{B}
      &= \mathbf{B}_{k - 1} \\
    \mathbf{s}
      &= \mathbf{s}_k \\
    \mathbf{y}
      &= \mathbf{y}_k
  \end{align*}
  La observación anterior dice que:
  \begin{equation*}
    \mathbf{X}
      = \mathbf{v} \, \mathbf{s}^T
  \end{equation*}
  o sea:
  \begin{align*}
    \overline{\mathbf{B}} \, \mathbf{s}
      &= \mathbf{y} \\
    (\mathbf{B} + \mathbf{v} \, \mathbf{s}^T) \mathbf{s}
      &= \mathbf{y} \\
    (\mathbf{v} \, \mathbf{s}^T) \, \mathbf{s}
      &= \mathbf{y} - \mathbf{B} \, \mathbf{s} \\
    \mathbf{v}
      &= \frac{\mathbf{y} - \mathbf{B} \, \mathbf{s}}
              {\mathbf{s}^T \, \mathbf{s}}
  \end{align*}
  Como \(\mathbf{s}^T \, \mathbf{s} = \langle \mathbf{s}, \mathbf{s} \rangle\),
  en términos de la notación original:
  \begin{equation}
    \label{eq:quasi-Newton-equation}
    \mathbf{B}_k
      = \mathbf{B}_{k - 1}
          + \frac{
              (\mathbf{y}_k
                 - \mathbf{B}_{k - 1} \mathbf{s}_k) \mathbf{s}_k^T
            }
            {
              \langle \mathbf{s}_k, \mathbf{s}_k \rangle
            }
  \end{equation}
  Sigue pendiente el problema de resolver:
  \begin{equation*}
    \mathbf{x}_{k + 1}
      = \mathbf{x}_k - \mathbf{B}_k^{-1} \mathbf{f}(\mathbf{x}_k)
  \end{equation*}
  Un lema muestra cómo hacerlo con mínimo costo.
  El resultado se atribuye a Sherman y Morrison~%
    \cite{sherman49:_adjust_inverse_matrix_col_row,
          sherman50:_adjust_inverse_matrix_change},
  lo que citamos es la forma de Bartlett~%
    \cite{bartlett51:_inverse_matrix_adjustment}.
  Como es tradicional,
  Hager~%
    \cite{hager89:_updating_inverse_matrix}
  al discutir la historia y aplicaciones halla que el resultado es anterior.
  \begin{lemma}[Fórmula de Sherman-Morrison]
    \label{lem:Sherman-Morrison}
    Sea \(\mathbf{A}\) una matriz no singular,
    y sean \(\mathbf{u}\), \(\mathbf{v}\) vectores.
    Entonces \(\mathbf{A} + \mathbf{u} \, \mathbf{v}^T\) es no singular
    si y solo si
      \(\sigma
          = 1 + \langle \mathbf{v}, \mathbf{A}^{-1} \mathbf{u} \rangle \ne 0\).
    Si \(\sigma \ne 0\),
    es:
    \begin{equation*}
      \label{eq:Sherman-Morrison}
      (\mathbf{A} + \mathbf{u} \, \mathbf{v}^T)^{-1}
        = \mathbf{A}^{-1}
            - \frac{1}{\sigma}
                 \mathbf{A}^{-1} \mathbf{u} \, \mathbf{v}^T \mathbf{A}^{-1}
    \end{equation*}
  \end{lemma}
  \begin{proof}
    Siempre que \(\mathbf{A}\) no sea singular,
    podemos expresar:
    \begin{equation*}
      \mathbf{A} + \mathbf{u} \mathbf{v}^T
        = \mathbf{A} (\mathbf{I} + \mathbf{A}^{-1} \mathbf{u} \, \mathbf{v}^T)
    \end{equation*}
    Consideremos entonces la matriz siguiente
    para vectores \(\mathbf{x}\) e \(\mathbf{y}\):
    \begin{equation*}
      \mathbf{I} + \mathbf{x} \, \mathbf{y}^T
    \end{equation*}
    Sospechamos que su inversa es de la forma siguiente,
    para un escalar \(\alpha\):
    \begin{equation*}
      \mathbf{I} + \alpha \mathbf{x} \, \mathbf{y}^T
    \end{equation*}
    Multiplicando por la inversa propuesta,
    como \(\mathbf{y} \, \mathbf{x} = \langle \mathbf{x}, \mathbf{y} \rangle\)
    es un escalar:
    \begin{align*}
      (\mathbf{I} + \mathbf{x} \, \mathbf{y}^T)
        \cdot (\mathbf{I} + \alpha \mathbf{x} \, \mathbf{y}^T)
        &= \mathbf{I}
             + \mathbf{x} \, \mathbf{y}^T
             + \alpha \mathbf{x} \, \mathbf{y}^T
             + \alpha \mathbf{x} \, \mathbf{y}^T \mathbf{x} \, \mathbf{y}^T \\
         &= \mathbf{I}
              + (1 + \alpha + \alpha \langle \mathbf{x}, \mathbf{y} \rangle)
                  \mathbf{x} \, \mathbf{y}^T
    \end{align*}
    Esto debe ser \(\mathbf{I}\),
    el escalar del segundo término es \num{0},
    de donde despejamos:
    \begin{equation*}
      \alpha
        = - \frac{1}{1 + \langle \mathbf{x}, \mathbf{y} \rangle}
    \end{equation*}
    De la misma forma vemos que:
    \begin{equation*}
       (\mathbf{I} + \alpha \mathbf{x} \, \mathbf{y}^T)
          \cdot (\mathbf{I} + \mathbf{x} \, \mathbf{y}^T)
          = \mathbf{I}
    \end{equation*}
    y realmente es inversa.

    Con \(\mathbf{x} = \mathbf{A}^{-1} \mathbf{u}\),
    \(\mathbf{y} = \mathbf{v}\)
    tenemos así la inversa para:
    \begin{align*}
      (\mathbf{A} + \mathbf{u} \, \mathbf{v}^T)^{-1}
        &= (\mathbf{A}
              (\mathbf{I}
                 + \mathbf{A}^{-1} \mathbf{u} \, \mathbf{v}^T))^{-1} \\
        &= (\mathbf{I} + \mathbf{A}^{-1} \mathbf{u} \, \mathbf{v}^T)^{-1}
              \mathbf{A}^{-1} \\
        &= (\mathbf{I}
              - \frac{1}{\sigma} \mathbf{A}^{-1} \mathbf{u} \, \mathbf{v}^T)
              \mathbf{A}^{-1} \\
        &= \mathbf{A}^{-1}
             - \frac{1}{\sigma}
                  \mathbf{A}^{-1} \mathbf{u} \, \mathbf{v}^T \mathbf{A}^{-1}
    \end{align*}
    Es claro que si \(\sigma = 0\),
    no hay inversa.
  \end{proof}
  El algoritmo parte entonces con una estimación inicial \(\mathbf{x}_0\),
  y una estimación inicial  \(\widetilde{\mathbf{J}}\)
  de la matriz jacobiana en ese punto
  (por ejemplo,
   estimando las derivadas mediante diferencias en cada dimensión)
  y calculamos una aproximación inicial
    \(\mathbf{H}_0 = \widetilde{\mathbf{J}}^{-1}\).
  De allí la iteración procede usando la fórmula de Sherman-Morrison
  para actualizar la inversa.
  Usamos las abreviaturas
  \(\Delta \mathbf{x}_k = \mathbf{x}_{k + 1} - \mathbf{x}_k\)
  y \(\Delta \mathbf{f}(\mathbf{x}_k)
        = \mathbf{f}(\mathbf{x}_{k + 1}) - \mathbf{f}(\mathbf{x}_k)\):
  \begin{align*}
    \mathbf{x}_{n + 1}
      &= \mathbf{x}_n - \mathbf{H}_n \mathbf{f}(\mathbf{x}_n) \\
    \mathbf{H}_{n + 1}
      &= \mathbf{H}_n
           + \frac{\Delta \mathbf{x}_{n + 1}
                     - \mathbf{H}_n \Delta \mathbf{f}(\mathbf{x}_{n + 1})}
                  {\Delta \mathbf{x}_{n + 1}^T
                     \mathbf{H}_n \Delta \mathbf{f}(\mathbf{x}_{n + 1})}
                \Delta \mathbf{f}^T(\mathbf{x}_{n + 1})
  \end{align*}
  Condiciones de término apropiadas
  pueden ser \(\lVert \mathbf{x}_{n + 1} - \mathbf{x}_n \rVert \le \varepsilon\),
  o una variante relativa de esta,
  como
  \(\lVert \mathbf{x}_{n + 1} - \mathbf{x}_n \rVert
      / \lVert \mathbf{x}_{n + 1} \rVert \le \varepsilon\);
  o \(\lVert \mathbf{f}(\mathbf{x}_{n + 1}) \rVert \le \varepsilon\).

\section*{Ejercicios}
\label{sec:ejercicios-03-seq}

  \begin{enumerate}
  \item
    \label{ex:03-seq:Broyden}
    En su publicación,
    Broyden~%
      \cite{broyden65:_solving_nonlin_simul_equat}
    prueba su método con varios sistemas de ecuaciones.
    Uno es:
    \begin{align*}
      f_1
        &= -(3 + \alpha x_1) x_1 + 2 x_2 - \beta \\
      f_i
        &= x_{i - 1} - (3 + \alpha x_i) x_i - 2 x_{i + 1} - \beta
          && i = 2, 3, \dotsc, n - 1 \\
      f_n
        &= x_n - (3 + \alpha x_n) x_n - \beta
    \end{align*}
    Valores de los parámetros usados
    están dados en el cuadro~\ref{tab:ex:03-seq:Broyden},
    valores iniciales son siempre \(x_i = 1,0\).
    \begin{table}[ht]
      \centering
      \begin{tabular}{>{\(}r<{\)}D{.}{,}{1}D{.}{,}{1}}
        \multicolumn{1}{c}{\boldmath\(n\)\unboldmath} &
          \multicolumn{1}{c}{\boldmath\(\alpha\)\unboldmath} &
          \multicolumn{1}{c}{\boldmath\(\beta\)\unboldmath} \\
        \hline
         5 & -0,1 & 1,0 \\
         5 & -0,5 & 1,0 \\
        10 & -0,5 & 1,0 \\
        20 & -0,5 & 1,0 \\
      \end{tabular}
      \caption{Parámetros para ejercicio~\ref{ex:03-seq:Broyden}}
      \label{tab:ex:03-seq:Broyden}
    \end{table}
    Experimente con el método de Newton y el de Broyden con algunas de estas,
    o con parámetros diferentes.
    Compare número de evaluaciones de las funciones y el tiempo total
    para obtener las soluciones con cinco cifras.
  \item
    Una propuesta para evitar el cálculo inicial de la matriz jacobiana
    en el método de Broyden
    es partir con una matriz inicial \(\mathbf{H}\) arbitraria,
    por ejemplo \(\mathbf{I}\).
    Experimente con esto en el ejercicio~\ref{ex:03-seq:Broyden}.
  \item
    Otra opción sería ordenar \(\mathbf{f}\)
    de forma que la derivada de \(f_i\) respecto de \(x_i\) inicial sea máxima,
    y aproximar la matriz jacobiana inicial
    por la matriz diagonal con estas entradas.
  \end{enumerate}

\bibliography{../referencias}

%%% Local Variables:
%%% mode: latex
%%% TeX-master: "../INF-221_notas"
%%% ispell-local-dictionary: "spanish"
%%% End:

% LocalWords:  eq nonlinear system cuasi jacobiana multi quasi
% LocalWords:  condition

\bibliographystyle{babplain-fl}

\chapter{Interpolación}
\label{cha:interpolacion}

  Hay situaciones en las cuales se conoce el valor de una función
  solo en algunos puntos dados,
  y queremos calcular valores en puntos intermedios.
  Por ejemplo,
  contamos solo con algunos valores medidos,
  o hay una tabla de valores numéricos.
  Este es el problema de interpolación.

  Nos dan los valores exactos de una función desconocida
  en \(n + 1\) puntos \(f(x_0), \ldots, f(x_n)\),
  queremos hallar una función que tome esos valores
  (para calcular valores intermedios).
  El caso más común es utilizar \emph{polinomios}.
  Para esta ocasión veremos
  que hay exactamente \emph{un} polinomio de grado a lo más \(n\)
  que pasa por \(n + 1\) puntos.

  Entre las cosas que podemos hacer
  para hallar la interpolación de \(f(x)\) tenemos:
  \begin{itemize}
  \item
    Obtener los coeficientes
    del polinomio en forma canónica de un sistema de ecuaciones.
  \item
    De forma implícita,
    dando el polinomio en forma no canónica.
  \end{itemize}

\section{Por sistema de ecuaciones}

  Suponiendo \(p(x) = a_0 + a_1 x + \dotsb + a_n x^n\),
  creamos un sistema de ecuaciones de la forma:
  \begin{align*}
    p(x_0)
       = f(x_0)
      &= a_0 + a_1 x_0 + a_2 x_0^2 + \dotsb a_n x_0^n \\
    p(x_1)
       = f(x_1)
      &= a_0 + a_1 x_1 + a_2 x_1^2 + \dotsb a_n x_1^n \\
      &\vdots \\
    p(x_n)
       = f(x_n)
      &= a_0 + a_1 x_n + a_2 x_n^2 + \dotsb a_n x_n^n
  \end{align*}
  que en forma de matriz se escribe como:
  \begin{equation}
    \label{04::Sistema:de:ecuaciones:1}
    \begin{pmatrix}
      1 & x_0 & x_0^2 & \cdots & x_0^n \\
      1 & x_1 & x_1^2 & \cdots & x_1^n \\
      \vdots & \vdots & \vdots & \ddots & \vdots & \\
      1 & x_n & x_n^2 & \cdots & x_n^n
    \end{pmatrix}
    \begin{pmatrix}
      a_0 \\
      a_1 \\
      \vdots \\
      a_n
    \end{pmatrix} =
    \begin{pmatrix}
      f(x_0) \\
      f(x_1) \\
      \vdots \\
      f(x_n)
    \end{pmatrix}
  \end{equation}
  Con igual cantidad de ecuaciones e incógnitas,
  el sistema de ecuaciones~\eqref{04::Sistema:de:ecuaciones:1}
  tiene solución única si y solo si
  el determinante es distinto de cero
  (ver por ejemplo Treil~%
    \cite{treil17:_linear_algeb_done_wrong}):
  \begin{equation}
    \label{04::Determinante:de:Vandermonde = 0}
    \left\lvert
      \begin{matrix}
        1 & \cdots & x_0^n\\
        \vdots & \ddots & \vdots \\
        1 & \cdots & x_n^n
      \end{matrix}
    \right\rvert
      \ne 0
  \end{equation}

  El determinante de la ecuación \eqref{04::Determinante:de:Vandermonde = 0}
  resulta ser el \emph{determinante de Vandermonde}.
  \begin{theorem}
    El determinante de Vandermonde vale:
    \begin{equation*}
      \begin{vmatrix}
        1 & a_1 & a_1^2 & \cdots & a_1^{n - 1} \\
        1 & a_2 & a_2^2 & \cdots & a_2^{n - 1} \\
        \vdots & \vdots & \vdots & \ddots & \vdots \\
        1 & a_n & a_n^2 & \cdots & a_n^{n - 1}
      \end{vmatrix}
        = \prod_{1 \le i < j \le n} (a_j - a_i)
    \end{equation*}
  \end{theorem}
  \begin{proof}
    Por inducción sobre \(n\).
    Para abreviar,
    llamaremos:
    \begin{equation*}
      V_n
        = \begin{vmatrix}
            1 & a_1 & a_1^2 & \cdots & a_1^{n - 1} \\
            1 & a_2 & a_2^2 & \cdots & a_2^{n - 1} \\
            \vdots & \vdots & \vdots & \ddots & \vdots \\
            1 & a_n & a_n^2 & \cdots & a_n^{n - 1}
          \end{vmatrix}
    \end{equation*}
    \begin{description}
    \item[Base:]
      El caso \(n = 1\) es trivial,
      es simplemente:
      \begin{equation*}
        \begin{vmatrix}
          1
        \end{vmatrix}
          = 1
      \end{equation*}
    \item[Inducción:]
      Supongamos que vale para \(k\),
      y consideremos el determinante:
      \begin{equation*}
        \begin{vmatrix}
          1 & a_1 & a_1^2 & \cdots & a_1^k \\
          1 & a_2 & a_2^2 & \cdots & a_2^k \\
          \vdots & \vdots & \vdots & \ddots & \vdots \\
          1 & a_k & a_k^2 & \cdots & a_k^k \\
          1 & x	  & x^2	  & \cdots & x^k \\
        \end{vmatrix}
      \end{equation*}
      Expandiendo por la última fila,
      vemos que es un polinomio en \(x\) de grado a lo más \(k\),
      llamémosle \(f(x)\).
      Como determinantes con filas iguales son cero,
      sabemos que \(f(a_i) = 0\) para \(1 \le i \le k\),
      o sea:
      \begin{align*}
        f(x)
          &= c (x - a_1) (x - a_2) \dotsm (x - a_k) \\
          &= c \prod_{1 \le i \le k} (x - a_i)
      \end{align*}
      Como el grado de \(f\) es a lo más \(k\),
      y tenemos \(k\) factores lineales en \(x\),
      \(c\) es independiente de \(x\).

      La expansión por la última fila,
      nuevamente,
      muestra que el coeficiente de \(x^k\) es \(V_k\),
      con lo que \(c = V_k\).
      Reemplazando \(x \mapsto a_{k + 1}\)
      con nuestra hipótesis de inducción entrega:
      \begin{align*}
        V_{k + 1}
          &= V_k \cdot \prod_{1 \le i \le k} (a_{k + 1} - a_i) \\
          &= \prod_{1 \le i < j \le k} (a_j - a_i)
               \cdot \prod_{1 \le i \le k} (a_{k + 1} - a_i) \\
          &= \prod_{1 \le i < j \le k + 1} (a_j - a_i)
      \end{align*}
    \end{description}
    Por inducción, vale para todo \(n \in \mathbb{N}\).
  \end{proof}

\section{De forma implícita}

  Para encontrar la interpolación de \(f\) de manera implícita
  simplemente inventamos un polinomio que pase por los puntos.

\subsection{Forma de Lagrange}

  Consiste en usar el polinomio:
  \begin{align}
    p(x)
      &= f(x_0) \frac{(x - x_1) (x - x_2) \dotsm (x - x_n)}
                     {(x_0 - x_1) (x_0 - x_2) \dotsm (x_0 - x_n)}
           + f(x_1) \frac{(x -x_0) (x - x_2) \dotsm (x - x_n)}
                         {(x_1 - x_0) (x_1 - x_2) \dotsm (x_1 - x_n)}
           + \dotsb \notag \\
      & \qquad {} + f(x_n) \frac{(x - x_0) \dotsm (x - x_{n - 1})}
                               {(x_n - x_0) \dotsm (x_n - x_{n - 1})}
              \notag \\
     &= \sum_{0 \le k \le n} f(x_k)
          \prod_{\substack{0 \le j \le n \\ j \ne k}}
            \frac{x - x_j}{x_k - x_j}
              \label{04::Lagrange:2}
  \end{align}
  como interpolación de \(f\).
  Es claro que si evalúa \(p\)
  en alguno de los \(x_k\) con \(k \in \{ 0, 1, \dotsc, n\}\) que nos entregan,
  se tiene que \(p(x_k) = f(x_k)\),
  y el polinomio es de grado \(n\).

  Para referencia futura,
  llamaremos:
  \begin{equation}
    \label{eq:Lagrange-bases}
    \ell_i(x)
      = \prod_{\substack{0 \le j \le n \\ j \ne i}}
            \frac{x - x_j}{x_i - x_j}
  \end{equation}
  Estos polinomios de grado \(n\),
  los \emph{polinomios base de Lagrange},
  dependen de los puntos \(x_i\),
  y tienen la particularidad que:
  \begin{equation*}
    \ell_i(x_j)
      = [i = j]
  \end{equation*}
  Acá usamos la convención de Iverson:
  \begin{equation*}
    [\text{condición}]
      = \begin{cases}
          0 & \text{si la condición es falsa} \\
          1 & \text{si la condición es verdadera}
        \end{cases}
  \end{equation*}
  En resumen,
  son linealmente independientes siempre que los \(x_i\) son todos distintos
  (básicamente,
   el determinante de Vandermonde es cero solo si algún \(x_i\) se repite).

  En la práctica,
  la fórmula~\eqref{04::Lagrange:2} es poco deseable,
  requiere \(O(n^2)\) computación para calcular \(p(x)\).
  Una forma alternativa es la \emph{fórmula baricéntrica}
  que recomienda Winrich~%
    \cite{winrich69:_compar_evaluat_schem_interp_polyn}
  al comparar varias alternativas en términos de números de operaciones:
  \begin{equation}
    \label{eq:formula-baricentrica}
    p(x)
      = \frac{\sum_{0 \le j \le n} \frac{w_j f(x_j)}{x - x_j}}
             {\sum_{0 \le j \le n} \frac{w_j}{x - x_j}}
  \end{equation}
  donde:
  \begin{equation}
    \label{eq:wj-Lagrange}
    w_j
      = \left(
          \prod_{\substack{0 \le k \le n \\ k \ne j}} (x_j -x_k)
        \right)^{-1}
  \end{equation}
  (claro que la expresión~\eqref{eq:formula-baricentrica}
   es indefinida cuando \(x = x_i\),
   hay que manejar ese caso en forma separada).
  Berrut y Trefethen~%
    \cite{berrut04:_bary_lagrange_interp} discuten esta fórmula en detalle.
  Higham~%
    \cite{higham04:_numer_stab_baryc_lagrange_interp}
  analiza esta técnica,
  concluyendo que es numéricamente estable
  y recomendable para uso general.
  Por ejemplo,
  es la técnica que emplea SciPy~%
    \cite{2020SciPy-NMeth}
  para interpolación por polinomios.

\subsection{Forma de Newton}
\label{sec:Newton-interpolation}

  Podemos escribir:
  \begin{alignat}{2}
    a_0
     &= f(x_0)
     &\qquad&
     Q_0(x)
       = a_0 \notag \\
  \intertext{Luego para \(k > 0\):}
     a_k
      &= \frac{f(x_k) - Q_{k - 1}(x_k)}
              {\prod_{0 \le i \le k - 1}(x_k - x_i)}
      &&
      Q_k(x)
        = Q_{k - 1}(x) + a_k \prod_{0 \le i \le k - 1} (x - x_i)
           \label{eq:interpolacion-Newton}
  \end{alignat}
  donde \(Q_k\) corresponde a la interpolación de grado \(k\).

\subsubsection{Diferencias divididas}
\label{divided-differences}

  La forma de Newton es engorrosa.
  Una alternativa se obtiene con \emph{diferencias divididas},
  que para los puntos \(x_0, \dotsc, x_n\) se definen mediante:
  \begin{align*}
    f[x_0]
      &= f(x_0) \\
    f[x_0, \dotsc, x_j]
      &= \frac{f(x_j) - Q_{j - 1}(x_j)}{\prod_{0 \le k \le j - 1} (x_j - x_k)}
  \end{align*}
  Con esta notación~\eqref{eq:interpolacion-Newton}
  se escribe:
  \begin{equation}
    \label{eq:interpolacion-diferencias-divididas}
    Q_n(x)
      = f[x_0]
          + f[x_0, x_1] (x - x_0)
          + \dotsb
          + f[x_0, x_1, \dotsc, x_n] \prod_{0 \le k < n}(x - x_k)
  \end{equation}
  La utilidad de la forma~\eqref{eq:interpolacion-diferencias-divididas}
  es por el siguiente resultado:
  \begin{lemma}
    Las diferencias divididas cumplen:
    \begin{equation}
      \label{eq:divided-differences-recurrence}
      f[x_0, \dotsc, x_n]
        = \frac{f[x_1, \dotsc, x_n] - f[x_0, \dotsc, x_{n - 1}]}
               {x_n - x_0}
    \end{equation}
  \end{lemma}
  \begin{proof}
    Para cualquier \(k\),
    sea \(Q_k\) el polinomio de grado a lo más \(k\)
    que interpola \(f\) en los puntos \(x_0, \dotsc, x_k\),
    o sea:
    \begin{equation*}
      Q_k(x_j)
        = f(x_j)
           \quad \text{para \(0 \le j \le k\)}
    \end{equation*}
    Consideremos el polinomio \(P(x)\) único de grado a lo más \(n - 1\)
    que interpola \(f\) en los puntos \(x_1, \dotsc, x_n\).
    Es simple verificar que:
    \begin{equation}
      \label{eq:QvsP}
      Q_n(x)
        = P(x) + \frac{x - x_n}{x_n - x_0} (P(x) - Q_{n - 1}(x))
    \end{equation}
    En~\eqref{eq:QvsP} el coeficiente de \(x^n\) al lado izquierdo
    es \(f[x_0, \dotsc, x_n]\).
    El coeficiente de \(x^{n - 1}\) en \(P(x)\)
    es \(f[x_1, \dotsc, x_n]\),
    y en \(Q_{n - 1}(x)\)
    es \(f[x_0, \dotsc, x_{n - 1}]\).
    O sea,
    el coeficiente de \(x^n\) al lado derecho de~\eqref{eq:QvsP}
    es:
    \begin{equation*}
      \frac{f[x_1, \dotsc, x_n] - f[x_0, \dotsc, x_{n - 1}]}
           {x_n - x_0}
    \end{equation*}
    como aseveramos.
  \end{proof}

\paragraph{Nota:}

  Algunos textos definen:
  \begin{equation*}
      f[x_0, \dotsc, x_n]
        = f[x_1, \dotsc, x_n] - f[x_0, \dotsc, x_{n - 1}]
  \end{equation*}
  en cuyo caso las fórmulas deben ajustarse
  para compensar por los denominadores faltantes.

  Un ejemplo
  (la función es la de Runge,
   ecuación~\eqref{eq:Runge-function})
  es la tabla del cuadro~\ref{tab:interpolacion}.
% Table created by divided-differences script
  \begin{table}[ht]
    \centering
    \begin{tabular}{*{ 8 }{>{\(}l<{\)}}}
      \multicolumn{1}{c}{\boldmath\(x\)\unboldmath} &
         \multicolumn{1}{c}{\boldmath\(y\)\unboldmath} &
         \multicolumn{ 6 }{c}{\textbf{Diferencias divididas}} \\
      \hline
      0,10 & 0,8000 \\
           &	    & -2,4615 \\
      0,30 & 0,3077 &	      & 4,3140 \\
           &	    & -0,9516 &	       & -5,6441 \\
      0,45 & 0,1649 &	      & 2,0564 &	 & 4,7074 \\
           &	    & -0,5403 &	       & -3,5258 &	  & -0,4531 \\
      0,50 & 0,1379 &	      & 1,1749 &	 & 4,4355 &	    & -4,6852\\
           &	    & -0,4229 &	       & -1,7516 &	  & -4,2013 \\
      0,55 & 0,1168 &	      & 0,7371 &	 & 1,9148 \\
           &	    & -0,2754 &	       & -0,8899 \\
      0,70 & 0,0755 &	      & 0,3811 \\
           &	    & -0,1421 \\
      0,90 & 0,0471
    \end{tabular}
    \caption{Tabla para interpolación}
    \label{tab:interpolacion}
  \end{table}
  Las entradas
  (diferencias divididas)
  se obtienen restando los elementos inmediatamente a la izquierda
  divididas por la diferencia de los valores de \(x\) en las diagonales.
  Para interpolar,
  se usa la primera fila diagonal.
  Por ejemplo,
  calculemos las aproximaciones a \(y\) para \(x = 0,47\)
  de los datos del cuadro~\ref{tab:interpolacion}.
  \begin{align*}
    Q_0(0,47)
      &= 0,80000 \\
    Q_1(0,47)
      &= Q_0(0,47) - 2,4315 (0,47 - 0,1) \\
      &= -0,11077 \\
    Q_2(0,47)
      &= Q_1(0,47) + 4,3140 (0,47 - 0,1) (0,47 - 0,3) \\
      &= 0,16058 \\
    Q_3(0,47)
      &= Q_2(0,47) - 5,6441 (0,47 - 0,1) (0,47 - 0,3) (0,47 - 0.45) \\
      &= 0,15348 \\
    Q_4(0,47)
      &= Q_3(0,47)
           + 4,7074 (0,47 - 0,1) (0,47 - 0,3) (0,47 - 0.45) (0,47 - 0,50) \\
      &= 0,15331 \\
    Q_5(0,47)
      &= 0,15330 \\
    Q_6(0,47)
      &= 0,15331
  \end{align*}
  El valor correcto es \num{0,15446}.

\section*{Ejercicios}
\label{sec:ejercicios-04}

  \begin{enumerate}
  \item
    Demuestre la fórmula baricéntrica~\eqref{eq:formula-baricentrica}.
    Defina:
    \begin{equation*}
      \ell(x)
        = \prod_{0 \le j \le n} (x - x_j)
    \end{equation*}
    escriba~\eqref{04::Lagrange:2} como:
    \begin{equation*}
      p(x)
        = \ell(x) \sum_{0 \le j \le n} \frac{w_j f(x_j)}{x - x_j}
    \end{equation*}
    Use esta forma de~\eqref{04::Lagrange:2} para interpolar la función \num{1},
    divida y simplifique.
  \item
    Plantee algoritmos eficientes
    para evaluar el polinomio interpolador en \(x\)
    dados los puntos \(x_0, \dotsc, x_n\)
    usando las fórmulas~\eqref{04::Lagrange:2},
    \eqref{eq:formula-baricentrica}
    y~\eqref{eq:interpolacion-Newton}.
    Separe inicialización que depende de los \(x_j\)
    de cálculos que dependen de \(x\)
    (o sea,
     maneje el caso en que dados los \(x_j\)
     interesa evaluar para varios \(x\)).
    Calcule el número de operaciones de punto flotante
    (\emph{flops})
    para evaluar el polinomio interpolador en \(x\) en cada caso.
  \end{enumerate}

\bibliography{../referencias}

%%% Local Variables:
%%% mode: latex
%%% TeX-master: "../INF-221_notas"
%%% ispell-local-dictionary: "spanish"
%%% End:

% LocalWords:  Interpolación interpolación llamémosle baricéntrica c
% LocalWords:  SciPy flops eq baricentrica sec interpolation QvsP
% LocalWords:  interpolacion function

\bibliographystyle{babplain-fl}

\chapter{Error de Interpolación}
\label{cha:interpolacion-error}

  La clase pasada vimos cómo obtener la interpolación
  dado los pares de puntos \((x_k, f(x_k))\) con \(k \in \{0, \dotsc, n \}\)
  que nos daban.
  Nuestro tema de interés ahora es obtener
  el error de esas interpolaciones
  (figura \ref{05::ErrorEjemplo}).
  En particular,
  nos interesa ver cómo depende el error de los puntos elegidos.
  Usaremos algunos resultados del análisis real,
  refiérase a sus apuntes,
  al apéndice~\ref{apx:analysis}
  o a textos como el de Chen~%
    \cite{chen08:_fundam_analy}
  para demostraciones de los teoremas del caso.
  \begin{figure}[ht]
    \centering
    \begin{tikzpicture}
      \begin{axis}[
                    axis lines = center,
                    xlabel = {\(x\)},
                    ylabel = {\(y\)},
                    every axis y
                      label/.style = {at = (current axis.above origin),
                        anchor = south},
                    every axis x
                      label/.style = {at = (current axis.right of origin),
                        anchor = west},
                    xmin = 0,	 xmax = 3,
                    ymin = -0.6, ymax = 2.3,
                    xtick = {0},
                    xticklabels = {},
                    ytick = {0},
                    yticklabels = {},
                    height = 6cm, width = 9cm
                  ]

        \addplot [black!30!blue!90, samples = 100, domain = 0.15:2.4]
          {ln(2 * x + 0.8) + 0.2} node [xshift = 10] (f) {\(f(x)\)};
        \addplot [black!10!red!90, smooth, tension = 0.8]
          coordinates {(0.2, 0.2) (0.55, 1.3) (0.9, 0.6) (1.2, 1.7)
                       (1.5, 0.8) (1.9, 1.9)}
          node [yshift = 5pt, xshift = 4pt] (p) {\(Q_n(x)\)};

        \draw [fill = black!30] (axis cs:0.257, 0.47) circle (0.7mm);
        \draw [fill = black!30] (axis cs:0.72, 1) circle (0.7mm);
        \draw [fill = black!30] (axis cs:1.315, 1.43) circle (0.7mm);
        \draw [fill = black!30] (axis cs:1.85, 1.7) circle (0.7mm);
        \draw [fill = black!30] (axis cs:1.85, 1.7) circle (0.7mm);

        \draw [latex'-latex', thick, green!60!black!90]
          (axis cs:0.888, 1.15) -- (axis cs:0.888, 0.595);
        \draw [fill = black!30] (axis cs:0.888, 0.595) circle (0.7mm);
      \end{axis}
    \end{tikzpicture}
    \caption{El error es la diferencia
             entre \(Q_n(x)\) y \(f(x)\) en cada punto (flecha verde).}
    \label{05::ErrorEjemplo}
  \end{figure}
  \begin{theorem}
    \label{theo:interpolation-error}
    Sea \(f \in C^{n+1}[a, b]\)
    (vale decir,
     \(f\) tiene \(n + 1\) derivadas continuas en el intervalo).
    Sea \(Q_n(x)\) el polinomio de grado \(n\)
    que interpola \(f\)
    en los puntos distintos \(x_0, \dotsc, x_n \in[a, b]\).
    Entonces para todo \(x \in [a, b]\)
    hay \(\zeta \in [a, b]\) tal que
    \begin{equation}
      \label{05::Teo2}
      f(x) - Q_n(x)
        = \underbrace{\frac{1}{(n + 1)!} f^{(n+1)}(\zeta)
              \prod_{0 \le j \le n}(x - x_j)}_{\text{Error}}
    \end{equation}
  \end{theorem}
  \begin{proof}
    Fijemos \(x \in [a, b]\).
    Si \(x = x_j\) para algún \(j\),
    el lado izquierdo y derecho de~\eqref{05::Teo2} se anulan
    para todo \(\zeta\)
    y estamos listos.

    En caso contrario,
    sea
    \begin{equation}
      \omega(x)
        = \prod_{0\leq j\leq n}(x-x_j)
    \end{equation}
    Notamos para referencia futura
    que \(\omega(x)\) es mónico%
      \footnote{Para un polinomio de grado \(n\),
                el coeficiente de \(x^n\) es \num{1}.},
    con lo que \(\omega^{(n+1)}(x) = (n + 1)!\).

    Definamos:
    \begin{equation}
      F(y)
        = f(y)-Q_n(y) - \lambda \omega(y)
    \end{equation}
    donde elegimos \(\lambda\) tal que \(F(x) = 0\).

    La función \(F(y)\) está en \(C^{n+1}[a, b]\),
    y tiene \(n + 2\) ceros distintos en \([a, b]\),
    a saber \(x, x_0, \dotsc, x_n\).
    Por el teorema de Rolle extendido
    (teorema~\ref{theo:Rolle-extended})
    \(F^{(n + 1)}(y)\) tiene un cero en \([a, b]\),
    llamémosle \(\zeta\).
    O sea:
    \begin{align*}
      F^{(n + 1)}(\zeta)
        &= f^{(n + 1)}(\zeta)
             - \cancelto{\scriptstyle 0}{Q_n^{(n + 1)}(\zeta)}
             - \lambda
          \quad \cancelto{\scriptstyle(n + 1)!}{\omega^{(n + 1)}(\zeta)}
         = 0 \\
    \intertext{Vale decir:}
      f^{(n + 1)}(\zeta)
        &= \lambda(n + 1)! \\
    \intertext{Con esto:}
      \lambda
        &= \frac{f^{(n + 1)}(\zeta)}{(n + 1)!} \\
      F(y)
        &= f(y) - Q_n(y) - \frac{f^{(n + 1)}(\zeta)}{(n + 1)!} \omega(y)
    \end{align*}
    El error en el punto \(x\) es:
    \begin{equation*}
      f(x) - Q_n(x)
        = \frac{f^{(n + 1)}(\zeta)}{(n + 1)!} \omega(x)
    \end{equation*}
  \end{proof}
  Acá \(n\) lo elegimos nosotros
  y corresponde al grado de la interpolación.
  Es claro que mientras mayor sea el grado de nuestra interpolación,
  en general menor será el error
  (vea el \((n + 1)!\) como denominador).
  Claro que intervienen las características de la función
  y los puntos elegidos también.

\section{El fenómeno de Runge}
\label{sec:Runge-phenomenon}

  En 1901 Runge~%
    \cite{runge01:_ueber_aequidistante_interpolation}
  observó el fenómeno que lleva su nombre.
  Lo ilustramos con la función que el mismo usó de ejemplo:
  \begin{equation}
    \label{eq:Runge-function}
    f(x)
      = \frac{1}{1 + 25 x^2}
  \end{equation}
  Se grafica~\eqref{eq:Runge-function}
  en el rango \([-1, 1]\) en la figura~\ref{fig:Runge-function}.
  \begin{figure}[ht]
    \centering
    \pgfimage[width = 0.8\textwidth]{images/runge}
    \caption{La función de Runge}
    \label{fig:Runge-function}
  \end{figure}
  Se aprecia que parece ser bastante mansa,
  pero las figuras~\ref{fig:Runge-function-equiespaciados}
  muestran lo que ocurre al interpolar
  con 5, 7, 9 y~11 puntos igualmente espaciados
  entre \(-1\) y~\num{1}.
  \begin{figure}[ht]
    \centering
    \subfloat[\num{5} puntos]{
      \pgfimage[width = 0.475\textwidth]{images/runge-4}
      \label{fig:Runge-function-4}
    }
    \subfloat[\num{7} puntos]{
      \pgfimage[width = 0.475\textwidth]{images/runge-6}
      \label{fig:Runge-function-6}
    } \\
    \subfloat[\num{9} puntos]{
      \pgfimage[width = 0.475\textwidth]{images/runge-8}
      \label{fig:Runge-function-8}
    }
    \subfloat[\num{11} puntos]{
      \pgfimage[width = 0.475\textwidth]{images/runge-10}
      \label{fig:Runge-function-10}
    }
    \caption{Interpolando la función de Runge, puntos equiespaciados}
    \label{fig:Runge-function-equiespaciados}
  \end{figure}
  Se aprecia que alrededor de \num{0},
  el polinomio interpolante se acerca a la función,
  como esperábamos;
  sin embargo,
  en los extremos el error
  (la diferencia absoluta máxima entre la función y el polinomio)
  aumenta.
  La figura~\ref{fig:Ruge-error} muestra cómo evoluciona el error máximo
  con el número de puntos igualmente espaciados.
  \begin{figure}[ht]
    \centering
    \pgfimage[width = 0.8\textwidth]{images/runge-error}
    \caption{Error al interpolar la función de Runge}
    \label{fig:Ruge-error}
  \end{figure}
  Una explicación clara del fenómeno,
  aunque un tanto simplificada para efectos didácticos,
  es la de Epperson~%
    \cite{epperson87:_runge_example}.
  Resulta que depende de las singularidades de la función
  en el plano complejo,
  que no se aprecian en la línea real.

\section{Puntos de Chebyshev}
\label{sec:puntos-de-chebyshev}

  Nace entonces la pregunta de elegir los puntos
  de manera de evitar este comportamiento indeseable.
  Del teorema~\ref{theo:interpolation-error}
  el error de interpolación para los puntos distintos
  \(x_0, \dotsc, x_n \in[a, b]\)
  cumple:
  \begin{equation}
    \label{eq:interpolation-error}
          f(x) - Q_n(x)
        = \frac{1}{(n + 1)!} f^{(n+1)}(\zeta)
            \prod_{0 \le j \le n}(x - x_j)
  \end{equation}
  donde \(a < \zeta < b\).
  Debe destacarse que \(\zeta\)
  depende de los puntos \(x_0, \dotsc, x_n\) además de \(x\),
  minimizar el máximo valor absoluto de esta expresión
  en algún intervalo no es fácil.
  Nos contentaremos con minimizar el polinomio dado por la productoria.
  Para ello emplearemos una secuencia de polinomios especiales.

  Los \emph{polinomios de Chebyshev}
  se definen mediante:
  \begin{equation}
    \label{eq:Chebyshev-recurrence}
    \begin{split}
      T_0(x)
        &= 1 \\
      T_1(x)
        &= x \\
      T_{n + 1}(x)
        &= 2 x T_n(x) - T_{n - 1}(x)
          \quad n \ge 1
    \end{split}
  \end{equation}

  En vez de la recurrencia~\eqref{eq:Chebyshev-recurrence}
  es posible derivar una fórmula explícita para \(T_n(x)\).
  \begin{lemma}
    \label{lem:Chebyshev-explicit}
    Para \(x \in [-1, 1]\):
    \begin{equation}
      \label{eq:Chebyshev-explicit}
      T_n(x)
        = \cos(n \arccos x)
    \end{equation}
  \end{lemma}
  \begin{proof}
    La identidad para el coseno de suma de ángulos da:
    \begin{align*}
      \cos (n + 1) \theta
        &= \cos \theta \cos n \theta - \sin \theta \sin n \theta \\
      \cos (n - 1) \theta
        &= \cos \theta \cos n \theta + \sin \theta \sin n \theta
    \end{align*}
    de donde:
    \begin{equation*}
      \cos (n + 1) \theta
        = 2 \cos \theta \cos n \theta - \cos (n - 1) \theta
    \end{equation*}
    Sea \(\theta = \arccos x\),
    vale decir,
    \(x = \cos \theta\).
    Si definimos:
    \begin{equation*}
      t_n(x)
        = \cos (n \arccos x)
        = \cos n \theta
    \end{equation*}
    Resulta así:
    \begin{align*}
      t_0(x)
        &= 1 \\
      t_1(x)
        &= x \\
      t_{n + 1}(x)
        &= 2 x t_n(x) - t_{n - 1}(x)
          \quad n \ge 1
    \end{align*}
    Esta es exactamente la recurrencia~\eqref{eq:Chebyshev-recurrence}.
  \end{proof}
  De la ecuación~\eqref{eq:Chebyshev-recurrence}
  vemos que:
  \begin{equation*}
    T_n(x)
      = 2^{n - 1} x^n + \dotsb
  \end{equation*}
  Conocemos los ceros de \(T_n(x)\) por el lema~\ref{lem:Chebyshev-explicit}:
  \begin{equation*}
    x_k
      = \cos \frac{(2 k - 1) \pi}{2 n}, k = 1, \dotsc, n
  \end{equation*}
  así que:
  \begin{equation*}
    T_n(x)
      = 2^{n - 1} \prod_{1 \le k \le n} (x - x_k)
  \end{equation*}

  ¿Qué tienen de especial estos polinomios?
  Veremos primero un resultado general sobre polinomios mónicos.
  \begin{theorem}
    \label{theo:maxima-monic-polynomial}
    Si \(p_n(x)\) es un polinomio mónico de grado \(n\),
    entonces:
    \begin{equation}
      \label{eq:maxima-monic-polynomial}
      \max_{-1 \le x \le 1} \lvert p_n(x) \rvert
        \ge 2^{1 - n}
    \end{equation}
  \end{theorem}
  \begin{proof}
    Demostramos~\eqref{eq:maxima-monic-polynomial} por contradicción.
    Supongamos que para el polinomio mónico \(p_n\) de grado \(n\)
    y para todo \(\lvert x \rvert \le 1\) siempre es:
    \begin{equation*}
      \lvert p_n(x) \rvert
        < 2^{1 - n}
    \end{equation*}
    Sea
    \begin{equation*}
      q_n(x)
        = 2^{1 - n} T_n(x)
    \end{equation*}
    con lo que \(q_n\) es mónico,
    y sean \(u_j\) los siguientes \(n + 1\) puntos:
    \begin{equation*}
      u_j
        = \cos \frac{j \pi}{n}
          \quad 0 \le j \le n
    \end{equation*}
    Como por el lema~\ref{lem:Chebyshev-explicit}:
    \begin{align*}
      T_n \left( \cos \frac{j \pi}{n} \right)
        &= \cos \left( n \arccos \left( \cos \frac{j \pi}{n} \right) \right) \\
        &= \cos \left( n \cdot \frac{j \pi}{n} \right) \\
        &= \cos j \pi \\
        &= (-1)^j
    \end{align*}
    tenemos que:
    \begin{equation*}
      (-1)^j q_n(u_j)
        = 2^{1 - n}
    \end{equation*}
    Por lo tanto:
    \begin{equation*}
      (-1)^j p_n(u_j)
        \le \lvert p_n(u_j) \rvert
        < 2^{1 - n}
        = (-1)^j q_n(u_j)
    \end{equation*}
    Concluimos que:
    \begin{equation*}
      (-1)^j (q_n(u_j) - p_n(u_j)) > 0
        \quad 0 \le j \le n
    \end{equation*}
    lo que significa que el polinomio \(q_n(x) - p_n(x)\)
    cambia de signo \(n + 1\) veces en el intervalo.
    Vale decir,
    tiene al menos \(n\) ceros en el intervalo.
    Pero \(p_n\) y \(q_n\) son mónicos,
    por lo que \(q_n(x) - p_n(x)\) tiene grado a lo más \(n - 1\),
    y no puede tener más de \(n - 1\) ceros
    a menos que sea el polinomio cero,
    o sea,
    \(p_n = q_n\).
    Como el máximo de \(p_n\) es menor al máximo de \(q_n\),
    son polinomios distintos y esto es imposible.
  \end{proof}
  Volvamos a nuestro problema de interpolación.
  Queremos hallar los puntos \(x_j\) que minimicen:
  \begin{equation*}
    \max_{-1 \le x \le 1}
      \left\lvert \prod_{0 \le j \le n} (x - x_j) \right\rvert
  \end{equation*}
  Notamos que este polinomio es mónico de grado \(n + 1\),
  por lo que por el teorema~\ref{theo:maxima-monic-polynomial}:
  \begin{equation*}
    \max_{-1 \le x \le 1}
      \left\lvert \prod_{0 \le j \le n} (x - x_j) \right\rvert
      \ge 2^{- n}
  \end{equation*}
  y podemos obtener el mínimo de esto
  si el polinomio es \(2^{-n} T_{n + 1}(x)\),
  o sea,
  los \(x_j\) son los ceros del polinomio de Chebyshev \(T_{n + 1}(x)\).
  Por el lema~\ref{lem:Chebyshev-explicit}
  vemos que:
  \begin{equation}
    \label{eq:Chebyshev-points}
    x_j
      = \cos \frac{2 j + 1}{2 n + 2} \pi
        \quad 0 \le j \le n
  \end{equation}
  Para estos puntos nuestra cota~\eqref{05::Teo2} para el error
  da:
  \begin{align}
    \lvert f(x) - p_n(x) \rvert
      &=   \left\lvert
             2^{-n} T_{n + 1}(x) \frac{f^{(n + 1)}(\zeta(x))}{(n + 1)!}
           \right\rvert \notag \\
      &\le \frac{B_{n + 1}}{2^n (n + 1)!}
             \label{eq.Chebyshev-error}
  \end{align}
  donde:
  \begin{equation}
    \label{eq:Chebyshev-error-coeff}
    B_{n + 1}
      = \max_{x \in [-1, 1]} \lvert f^{(n + 1)}(x) \rvert
  \end{equation}
  Es claro que mediante una transformación lineal
  podemos cambiar el rango de integración.

  Si repetimos el ejercicio de interpolación anterior,
  pero ahora interpolando la función de Runge en los puntos de Chebyshev
  resultan las figuras~\ref{fig:Chebyshev-Runge-function}.
  \begin{figure}[ht]
    \centering
    \subfloat[\num{5} puntos]{
      \pgfimage[width = 0.475\textwidth]{images/chebyshev-4}
      \label{fig:Chebyshev-Runge-function-4}
    }
    \subfloat[\num{7} puntos]{
      \pgfimage[width = 0.475\textwidth]{images/chebyshev-6}
      \label{fig:Chebyshev-Runge-function-6}
    } \\
    \subfloat[\num{9} puntos]{
      \pgfimage[width = 0.475\textwidth]{images/chebyshev-8}
      \label{fig:Chebyshev-Runge-function-8}
    }
    \subfloat[\num{11} puntos]{
      \pgfimage[width = 0.475\textwidth]{images/chebyshev-10}
      \label{fig:Chebyshev-Runge-function-10}
    }
    \caption{Interpolando la función de Runge, puntos de Chebyshev}
    \label{fig:Chebyshev-Runge-function}
  \end{figure}
  La figura~\ref{fig:Chebyshev-error} muestra cómo evoluciona el error máximo
  con el número de puntos de Chebyshev.
  Se ve que las máximas magnitudes de los errores son bastante parejos
  (cabe esperar que lo sean,
   por la forma sinusoidal de \(T_n(x)\) en nuestro rango,
   y siendo el error una función
   --que esperamos varíe lentamente con \(x\)--
   multiplicada por un polinomio de Chebyshev).
  \begin{figure}[ht]
    \centering
    \pgfimage[width = 0.8\textwidth]{images/chebyshev-error}
    \caption{Error al interpolar la función de Runge
             en puntos de Chebyshev}
    \label{fig:Chebyshev-error}
  \end{figure}

\bibliography{../referencias}

%%% Local Variables:
%%% mode: latex
%%% TeX-master: "../INF-221_notas"
%%% ispell-local-dictionary: "spanish"
%%% End:

% LocalWords:  grafica eq function width images runge equiespaciados
% LocalWords:  recurrence maxima monic polynomial

\bibliographystyle{babplain-fl}

\chapter{Cuadratura}
\label{cha:cuadratura}

  Interesa desarrollar técnicas numéricas para calcular integrales.
  A estas técnicas se les llama \emph{cuadratura}
  porque son hallar áreas,
  para los geómetras griegos esto era construir un cuadrado de la misma área
  que la figura planteada.
  La triste realidad es que,
  a pesar del entusiasmo de los colegas de matemáticas
  por técnicas de integración,
  la minoría de las integrales que debemos calcular en la práctica
  tienen una forma cerrada,
  e incluso si la tienen puede ser poco manejable.

  Queremos evaluar:
  \begin{equation}
    \label{05::Integral}
    \int_a^b f(x) \mathrm{d} x
  \end{equation}
  Nuevamente,
  no consideramos errores de redondeo,
  esos aspectos deberán considerarse aparte.
  Supongamos \(f\) dado en \(x_0, \ldots, x_n\)
  (los \textit{puntos de cuadratura}).
  Para encontrar el valor de~\eqref{05::Integral}
  simplemente interpolamos \(f\)
  dando el polinomio \(p\),
  e integramos el polinomio interpolante.
  Si usamos la fórmula de Lagrange~\eqref{04::Lagrange:2}
  para el polinomio interpolador:
  \begin{align}
    \int_a^b p(x) \mathrm{d} x
      &= \int_a^b \left( \sum_{0 \le j \le n} f(x_j) l_j(x) \right)
                           \mathrm{d} x \notag \\
      &= \sum_{0 \le j \le n} f(x_j) \int_a^b l_j(x) \mathrm{d} x
                 \notag \\
      &= \sum_{0 \le j \le n} f(x_j) A_j
                  \label{eq:quadrature-general}
  \end{align}
  De~\eqref{eq:quadrature-general} obtenemos la fórmula general
  para los coeficientes:
  \begin{equation}
    \label{eq:quadrature-coefs}
    A_j
      = \int_a^b l_j(x) \mathrm{d} x
  \end{equation}
  Claro que las integrales~\eqref{eq:quadrature-coefs} son engorrosas,
  alternativamente podemos plantear un sistema de ecuaciones para los \(A_j\)
  suponiendo que la fórmula es exacta para polinomios elegidos.

  El error está dado por:
  \begin{equation}
    \label{eq:quadrature-error}
    E
      = \int_a^b f(x) \mathrm{d} x
          - \sum_{0 \le j \le n} A_j f(x_j)
  \end{equation}
  Suponiendo que la fórmula es exacta para polinomios hasta grado \(m\)
  (no siempre coincidirá con \(n\)),
  expandimos mediante serie de Taylor alrededor de \(a\)
  hasta el término en \((x - a)^{m + 1}\)
  y simplificamos.
  Si son muchos puntos,
  esto es bastante engorroso.
  Sabemos,
  eso sí,
  que los términos hasta grado \(m\) se cancelan
  (dado que la fórmula es exacta hasta ese grado).
  En los desarrollos que siguen mostraremos algunos trucos
  que simplifican los cálculos para situaciones consideradas.

\section{Caso más simple: Polinomio de grado 0}
\label{sec:cuadratura-0}

  Corresponde a una aproximación con rectángulos
  (figura \ref{05::CasoSimple:Cuadratura})
  \begin{figure}[ht]
    \centering
    \begin{tikzpicture}
      \begin{axis}[
                    axis lines = center,
                    xlabel = {\(x\)},
                    ylabel = {\(y\)},
                    every axis y
                      label/.style = {at = (current axis.above origin),
                        anchor = south},
                    every axis x
                      label/.style = {at = (current axis.right of origin),
                        anchor = west},
                      xmin = 0,	   xmax = 3,
                      ymin = -0.6, ymax = 2.3,
                      xtick = {0.5, 2.0},
                      xticklabels = {\(a\), \(b\)},
                      ytick = {0},
                      yticklabels = {},
                      height = 6cm, width = 9cm
                    ]

        \addplot [black!10!red!90, samples = 100, domain = 0.15:2.4]
          {ln(2 * x + 0.8) + 0.2} node [xshift = 10] (f) {\(f(x)\)};

        \draw [blue!60!black!90]
          (axis cs:0.5, 0) -- (axis cs:0.5, 0.79) -- (axis cs: 0.8, 0.79)
            -- (axis cs: 0.8, 0);
        \draw [blue!60!black!90]
          (axis cs:0.8, 0) -- (axis cs:0.8, 1.07) -- (axis cs: 1.1, 1.07)
            -- (axis cs: 1.1, 0);
        \draw [blue!60!black!90]
          (axis cs:1.1, 0) -- (axis cs: 1.1, 1.3) -- (axis cs: 1.4, 1.3)
            -- (axis cs: 1.4, 0);
        \draw [blue!60!black!90]
          (axis cs:1.4, 0) -- (axis cs: 1.4, 1.48) -- (axis cs: 1.7, 1.48)
            -- (axis cs: 1.7, 0);
        \draw [blue!60!black!90]
          (axis cs:1.7, 0) -- (axis cs: 1.7, 1.63) -- (axis cs: 2.0, 1.63)
            -- (axis cs: 2.0, 0);
      \end{axis}
    \end{tikzpicture}
    \caption{Aproximar el área bajo una curva usando rectangulitos}
    \label{05::CasoSimple:Cuadratura}
  \end{figure}
  Para ello,
  usamos la fórmula:
  \begin{align*}
    \int_a^b f(x) \mathrm{d} x
      \approx \sum_{0 \le j < n} f(x_j)(x_{j + 1} - x_j)
        \qquad\qquad a = x_0, b = x_n
  \end{align*}
  Si son igualmente espaciados,
  se tiene que \(x_{j + 1} - x_j = h\) para \(0 \le j < n\):
  \begin{align*}
    \int_a^b f(x) \mathrm{d} x
      &\approx h \sum_{0 \le j < n} f(x_j) \\
      &=       h \sum_{0 \le j < n} f(x_0 + j h)
  \end{align*}

  Interesa obtener un valor suficientemente preciso de la integral,
  el número de subintervalos determina el trabajo a efectuar en el cálculo.
  Veamos un único intervalo \([x_j, x_{j + 1}]\)
  de largo \(h\) para derivar el error.
  Consideremos la anti-derivada%
    \footnote{Es claro suponer que la anti-derivada existe,
              de lo contrario este cuento no tiene chiste.}:
  \begin{equation*}
    F(x)
      = \int_{x_j}^{x} f(t) \mathrm{d} t
  \end{equation*}
  Expandiendo \(F(x)\) en serie de Taylor alrededor de \(x_j\)
  y usando el teorema fundamental del cálculo:
  \begin{align*}
    F(x)
      &= F(x_j)
          + F'(x_j) (x - x_j)
          + \frac{1}{2!} F''(\xi) (x - x_j)^2 \\
      &= 0 + f(x_j) (x - x_j) + \frac{1}{2!} f'(\xi) (x - x_j)^2
  \end{align*}
  donde \(x_j \le \xi \le x\).
  (Como siempre,
   el determinar dónde cortar la serie es un arte,
   no una ciencia.
   Acá sospechamos que solo el término lineal incide.
   Si resultara que los términos relevantes se cancelan,
   habría que incluir términos adicionales;
   si sobraran términos irrelevantes,
   conviene cortar antes.)

  Recordando \(h = x_{j + 1} - x_j\)
  esto es:
  \begin{align}
    \int_{x_j}^{x_{j + 1}} f(x) \mathrm{d} x
      &= f(x_j) h
           + \frac{f'(\xi)}{2} h^2 \notag \\
    E
      &= \int_{x_j}^{x_{j + 1}} f(x) \mathrm{d} x - f(x_j) h \notag \\
      &= \frac{f'(\xi)}{2} h^2
           \label{eq:E-initial-point}
  \end{align}
  El error es cuadrático en \(h\)
  para cada intervalo,
  si son \(n\) intervalos es:
  \begin{align*}
    n O(h^2)
      &= O\left( n \frac{(b - a)^2}{n^2} \right) \\
      &= O\left( \frac{(b - a)^2}{n} \right) \\
      &= O(h)
  \end{align*}
  La misma observación es válida para otras reglas compuestas.

\subsection{Teorema del Valor Intermedio Ad Hoc}

  Considerando el error obtenido en la ecuación~\eqref{eq:E-initial-point}:
  \begin{equation}
    E
      = \frac{1}{2} f'(\xi) h^2
          \qquad \text{con \(x_j \le \xi \le x_{j + 1}\)}
  \end{equation}
  Suponiendo intervalos \(a = x_0, x_1, \dotsc, x_n = b\)
  con \(x_{j + 1} - x_j = h\),
  tenemos para cada uno:
  \begin{align*}
    E_j
      = \frac{1}{2} f'(\xi_j) h^2, \qquad x_j \le \xi _j \le x_{j + 1}
  \end{align*}
  Con \(m\) el mínimo y \(M\) el máximo de \(f'\) en \([a, b]\),
  tenemos \(m \le f'(x) \le M\):
  \begin{equation}
    \label{eq:E-rectangles}
    E
      = \sum_j E_j
      = \frac{h^2}{2} \sum_j f'(\xi _j)
  \end{equation}
  Ahora:
  \begin{align*}
    n m
      &\le \sum_j f'(\xi_j) \le n M \\
    m
      &\le \frac{1}{n} \sum_j f'(\xi_j) \le M
  \end{align*}
  Como suponemos \(f'\) continua en el intervalo,
  esto nos dice que hay \(\xi \in [a, b]\) tal que:
  \begin{equation*}
    \frac{1}{n} \sum_j f'(\xi_j)
      = f'(\xi)
  \end{equation*}
  Con la fórmula~\eqref{eq:E-rectangles},
  recordando que \(n h = b - a\),
  una constante,
  tenemos para la regla compuesta:
  \begin{equation}
    \label{eq:E-rectangles-composite}
    E
      = \frac{n h^2}{2} f'(\xi)
      = \frac{(b - a) h}{2} f'(\xi)
  \end{equation}
  Un desarrollo similar es aplicable a los otros métodos.

\section{Variante del caso simple: punto medio}

  En lugar de considerar una cota
  como se hizo en el caso de la figura~\ref{05::CasoSimple:Cuadratura},
  el punto de evaluación de \(f(x)\) será el punto medio
  de la base del rectángulo
  (figura~\ref{05::CasoVariado:Cuadratura}).
  \begin{figure}[ht]
    \centering
    \begin{tikzpicture}
      \begin{axis}[
                    axis lines = center,
                    xlabel = {\(x\)},
                    ylabel = {\(y\)},
                    every axis y
                      label/.style = {at = (current axis.above origin),
                        anchor = south},
                    every axis x
                      label/.style = {at = (current axis.right of origin),
                        anchor = west},
                    xmin = 0,	 xmax = 3,
                    ymin = -0.6, ymax = 2.3,
                    xtick = {0.5, 0.65, 0.95, 1.25, 1.55, 1.85, 2.0},
                    xticklabels = {\(a\), {}, {}, {}, {}, {}, \(b\)},
                    ytick = {0},
                    yticklabels = {},
                    height = 6cm, width = 9cm
                  ]

        \addplot [black!10!red!90, samples = 100, domain = 0.15:2.4]
          {ln(2 * x + 0.8) + 0.2} node [xshift = 10] (f) {\(f(x)\)};

        \draw [blue!60!black!90]
          (axis cs:0.5, 0) -- (axis cs:0.5, 0.94) -- (axis cs: 0.8, 0.94)
            -- (axis cs: 0.8, 0);
        \draw [blue!60!black!90]
          (axis cs:0.8, 0) -- (axis cs:0.8, 1.2) -- (axis cs: 1.1, 1.2)
            -- (axis cs: 1.1, 0);
        \draw [blue!60!black!90]
          (axis cs:1.1, 0) -- (axis cs: 1.1, 1.39) -- (axis cs: 1.4, 1.39)
            -- (axis cs: 1.4, 0);
        \draw [blue!60!black!90]
          (axis cs:1.4, 0) -- (axis cs: 1.4, 1.56) -- (axis cs: 1.7, 1.56)
            -- (axis cs: 1.7, 0);
        \draw [blue!60!black!90]
          (axis cs:1.7, 0) -- (axis cs: 1.7, 1.7) -- (axis cs: 2.0, 1.7)
            -- (axis cs: 2.0, 0);
      \end{axis}
    \end{tikzpicture}
    \caption{El excedente de \textquote{triangulitos} por sobre \(f\)
             compensan la falta de estos que están bajo \(f\).}
    \label{05::CasoVariado:Cuadratura}
  \end{figure}
  Suponemos que \(f\) es integrable,
  y que además tiene \textquote{suficientes} derivadas continuas.

  Para este caso se tiene que:
  \begin{align*}
    \int_a^b f(x) \mathrm{d} x
      \approx (b - a) f \left( \frac{a + b}{2} \right)
  \end{align*}
  Expandimos usando series de Taylor:
  \begin{align*}
    F(a + h)
      &= F(a) + F'(a) h
           + \frac{1}{2} F''(a) h^2 + \frac{1}{6} F'''(a) h^3
           + O(h^4) \\
      &= f(a) h + \frac{1}{2} f'(a) h^2 + \frac{1}{6} f''(a) h^3
           + O(h^4)
  \end{align*}
  Si \(b = a+h\),
  tenemos
  (expandiendo \(f(a + h / 2)\))
  para el error:
  \begin{align}
    E
      &= \int_a^{a + h} f(x) \mathrm{d} x - h f\left( a + \frac{h}{2} \right)
              \notag \\
      &= h f(a) + \frac{h^2}{2} f'(a)
           + \frac{h^3}{6} f''(a)
           + O(h^4)
        - h \left(
              f(a) + \frac{h}{2} f'(a) + \frac{h^2}{8} f''(a) + O(h^3)
            \right)
              \notag \\
      &= h f(a) + \frac{h^2}{2} f'(a) + \frac{h^3}{6} f''(a)
           + O(h^4)
        - \left(
            h f(a) + \frac{h^2}{2} f'(a) + \frac{h^3}{8} f''(a)
              + O(h^4)
          \right)
              \notag \\
      &= \frac{1}{24} f''(a) h^3
           + O(h^4)
              \label{05::ErrrObtenido}
  \end{align}
  Sorprendentemente,
  el error es de carácter cúbico,
  cabía esperar un error cuadrático como en el caso anterior
  (sección~\ref{sec:cuadratura-0}).
  Si tenemos una cota para la segunda derivada de \(f\) en \(a\),
  estamos listos.

\subsection{Teorema del Valor Intermedio Ad Hoc}
\label{sec:mean-value-ad-hoc}

  Considerando el error obtenido en la ecuación~\eqref{05::ErrrObtenido}:
  \begin{equation}
    E
      = \frac{1}{24} f''(\xi) h^3
          \qquad \text{con \(a \le \xi \le a + h\)}
  \end{equation}
  Suponiendo intervalos \(a = x_0, x_1, \dotsc, x_n = b\)
  con \(x_{j + 1} - x_j = h\),
  tenemos para cada uno:
  \begin{align*}
    E_j
      = \frac{1}{24} f''(\xi_j) h^3, \qquad x_j \le \xi _j \le x_{j + 1}
  \end{align*}
  Si \(m \le f''(x) \le M\) en \([a, b]\):
  \begin{equation*}
    E
      = \sum_j E_j
      = \frac{h^3}{24} \sum_j f''(\xi _j)
      = \frac{n h^3}{24} f''(\xi)
  \end{equation*}
  porque:
  \begin{align*}
    n m
      &\le \sum_j f''(\xi_j) \le n M \\
    m
      &\le \frac{1}{n} \sum_j f''(\xi_j) \le M
  \end{align*}
  Lo que nos dice que hay \(\xi \in [a, b]\) tal que:
  \begin{equation*}
    \frac{1}{n} \sum_j f''(\xi_j)
      = f''(\xi)
  \end{equation*}

\section{Polinomio de grado 1 (trapezoides)}
\label{sec:trapezoides}

  La siguiente idea más simple es aproximar \(f(x)\)
  mediante la recta que pasa por \((a, f(a))\) y \((b, f(b))\),
  ver la figura~\ref{fig:trapezoid}.
  \begin{figure}[ht]
    \centering
    \begin{tikzpicture}
      \begin{axis}[
                    axis lines = center,
                    xlabel = {\(x\)},
                    ylabel = {\(y\)},
                    every axis y
                      label/.style = {at = (current axis.above origin),
                        anchor = south},
                    every axis x
                      label/.style = {at = (current axis.right of origin),
                        anchor = west},
                      xmin = 0,	   xmax = 3,
                      ymin = -0.6, ymax = 2.3,
                      xtick = {0.5, 2.0},
                      xticklabels = {\(a\), \(b\)},
                      ytick = {0},
                      yticklabels = {},
                      height = 6cm, width = 9cm
                    ]

        \addplot [black!10!red!90, samples = 100, domain = 0.15:2.4]
          {ln(2 * x + 0.8) + 0.2} node [xshift = 10] (f) {\(f(x)\)};

        \draw [blue!60!black!90]
          (axis cs:0.5, 0) -- (axis cs:0.5, 0.79) -- (axis cs: 2.0, 1.77)
            -- (axis cs: 2.0, 0);
      \end{axis}
    \end{tikzpicture}
    \caption{Aproximar el área bajo una curva usando un trapezoide}
    \label{fig:trapezoid}
  \end{figure}
  Esto lleva a la aproximación:
  \begin{equation*}
    \int_a^b f(x) \mathrm{d} x
      = \frac{1}{2} (f(a) + f(b)) (b - a)
  \end{equation*}
  Si tenemos múltiples intervalos,
  digamos \(n\) con
  \(x_0, \ldots, x_n\) donde \(x_{i + 1} - x_i = h\),
  donde \(a = x_0\) y \(b = x_n\),
  la regla se traduce en:
  \begin{equation*}
    \int_a^b f(x) \mathrm{d} x
      = \left(
          \frac{1}{2} (f(x_0) + f(x_n)) + \sum_{0 < i < n} f(x_i)
        \right) \frac{b - a}{n}
  \end{equation*}

  El análisis tradicional es relativamente complejo,
  pero Cruz-Uribe y~Neugebauer~%
    \cite{cruz-uribe03:_elemen_proof_error_estim_trapez_rule}
  proponen la técnica que usaremos.
  Consideremos un intervalo,
  tenemos para el error:
  \begin{equation}
    \label{eq:def-error-trapezoide}
    E_j
      = \frac{x_{j + 1} - x_j}{2} (f(x_j) + f(x_{j + 1}))
          - \int_{x_j}^{x_{j + 1}} f(x) \mathrm{d} x
  \end{equation}
  Sea \(c\) el centro del intervalo:
  \begin{equation*}
    c
      = \frac{x_j + x_{j + 1}}{2}
  \end{equation*}
  con lo que:
  \begin{equation*}
    x_{j + 1} - c
      = c - x_j
      = \frac{x_{j + 1} - x_j}{2}
  \end{equation*}
  De~\eqref{eq:def-error-trapezoide}
  vemos que corresponde usar integración por partes \textquote{en reversa}:
  \begin{equation*}
    E_j
      = \int_{x_j}^{x_{j + 1}} (x - c) f'(x) \mathrm{d} x
  \end{equation*}
  Nuevamente integrando por partes,
  como \(x_{j + 1} - c = c - x_j = (x_{j + 1} - x_j) / 2\):
  \begin{align*}
    E_j
      &= \left. \frac{1}{2}(x - c)^2 f'(x) \right|_{x_j}^{x_{j + 1}}
           - \frac{1}{2} \int_{x_j}^{x_{j + 1}}
               (x - c)^2 f''(x) \, \mathrm{d} x \\
      &= \frac{(x_{j + 1} - x_j)^2}{4} (f(x_{j + 1}) - f(x_j))
           - \frac{1}{2} \int_{x_j}^{x_{j + 1}}
               (x - c)^2 f''(x) \, \mathrm{d} x \\
  \intertext{El primer término
             es una constante por la integral de \(f''(x)\):}
      &= \frac{1}{2}
           \int_{x_j}^{x_{j + 1}}
                    \left(
                      \left(
                        \frac{x_{j + 1} - x_j}{2}
                      \right)^2
                        - (x - c)^2
                    \right) f''(x) \mathrm{d} x
  \end{align*}
  El polinomio bajo la integral no cambia de signo en \([x_j, x_{j + 1}]\),
  suponiendo \(f''\) continua en el intervalo
  podemos aplicar el teorema del valor intermedio de la integral,
  que nos asegura que hay \(\xi_j \in [x_j, x_{j + 1}]\) tal que:
  \begin{align*}
    E_j
      &= \frac{1}{2} f''(\xi_j)
                   \int_{x_j}^{x_{j + 1}}
                      \left(
                        \left(
                          \frac{x_{j + 1} - x_j}{2}
                        \right)^2
                          - (x - c)^2
                    \right) \mathrm{d} x \\
      &= f''(\xi_j) \frac{(x_{j + 1} - x_j)^3}{12} \\
      &= \frac{f''(\xi_j)}{12} h^3
  \end{align*}

  Tomemos un rango completo ahora.
  El error de la integral para el rango \([a, b]\)
  dividido en \(n\)~intervalos es:
  \begin{align*}
    E
      &= \sum_{0 \le j < n} E_j \\
      &= \sum_{0 \le j < n} \frac{f''(\xi_j)}{6} h^3 \\
      &= \frac{h^3}{6} \sum_{0 \le j < n} f''(\xi_j) \\
      &= n \frac{h^3}{6} \cdot \frac{1}{n} \sum_{0 \le j < n} f''(\xi_j)
  \end{align*}
  Suponiendo \(f''\) continua en \([a, b]\),
  como el promedio de los valores debe estar entre el máximo y el mínimo,
  hay \(\xi \in [a, b]\) tal que:
  \begin{equation*}
    \frac{1}{n} \sum_{0 \le j < n} f''(\xi_j)
      = f''(\xi)
  \end{equation*}
  con lo que:
  \begin{align*}
    E
      &= \frac{n h^3}{12} f''(\xi) \\
      &= \frac{(b - a) h^2 f''(\xi)}{12}
  \end{align*}

  Consideremos la función de Runge~\eqref{eq:Runge-function}
  y su integral:
  \begin{equation*}
    \int_{-1}^1 \frac{\mathrm{d} x}{1 + 25 x^2}
      = \frac{2 \arctan 5}{5}
  \end{equation*}
  El cuadro~\ref{tab:Runge-trapezoids}
  resume los resultados de estimar la integral por trapezoides.
  Se da el valor de largo del subintervalo,
  \(h\),
  el error
  y una estimación de \(C\) en \(E = C h^2\).
  \begin{table}[ht]
    \centering
    \begin{tabular}{>{\(}c<{\)}*{2}{>{\(}r<{\)}}}
      \multicolumn{1}{c}{\boldmath\(h\)\unboldmath} &
        \multicolumn{1}{c}{\textbf{Error}} &
        \multicolumn{1}{c}{\boldmath\(C\)\unboldmath} \\
      \hline
      2 / 2^1 & \num{-4.8910e-1} & \num{-4.8910e-1} \\
      2 / 2^2 & \num{-1.0780e-1} & \num{-4.3121e-1} \\
      2 / 2^3 & \num{-7.5376e-3} & \num{-1.2060e-1} \\
      2 / 2^4 &	 \num{1.3798e-4} &  \num{8.8309e-3} \\
      2 / 2^5 &	 \num{4.8118e-5} &  \num{1.2318e-2} \\
      2 / 2^6 &	 \num{1.2036e-5} &  \num{1.2325e-2} \\
      2 / 2^7 &	 \num{3.0095e-6} &  \num{1.2327e-2} \\
      2 / 2^8 &	 \num{7.5240e-7} &  \num{1.2327e-2} \\
      2 / 2^9 &	 \num{1.8810e-7} &  \num{1.2327e-2}
    \end{tabular}
    \caption{Aproximando la integral de la función de Runge por trapezoides}
    \label{tab:Runge-trapezoids}
  \end{table}

\section{Fórmula general para el error}
\label{sec:general-quadrature-error}

  Para el caso general de reglas de la forma:
  \begin{equation*}
    \int_a^b f(x) \, \mathrm{d} x
       \approx \sum_{0 \le k \le n} w_k f(x_k)
  \end{equation*}
  obtenidas integrando el polinomio interpolador
  podemos acotar el error en forma simple:
  \begin{align*}
    E
      &=   \int_a^b f(x) \, \mathrm{d} x
             - \sum_{0 \le k \le n} w_k f(x_k) \\
      &=   \int_a^b (f(x) - Q_n(x)) \, \mathrm{d} x \\
    \lvert E \rvert
      &\le \int_a^b \lvert f(x) - Q_n(x) \rvert \, \mathrm{d} x
  \end{align*}
  La fórmula~\ref{05::Teo2} para el error de interpolación nos dice:
  \begin{align*}
    f(x) - Q_n(x)
       &=   \frac{1}{(n + 1)!} f^{(n+1)}(\zeta)
              \prod_{0 \le j \le n}(x - x_j) \\
       &\le \frac{M_{n + 1}}{(n + 1)!}
              \left\lvert \prod_{0 \le j \le n}(x - x_j) \right\rvert
  \end{align*}
  Acá usamos una cota \(\lvert f^{(n + 1})(x) \rvert \le M_{n + 1}\)
  para \(a \le x \le b\).
  Esto da la cota genérica:
  \begin{equation}
    \label{eq:general-quadrature-error}
    \lvert E \rvert
      = \frac{M_{n + 1}}{(n + 1)!}
          \int_a^b \lvert \omega_{n + 1}(x) \rvert \, \mathrm{d} x
  \end{equation}
  Claro que esta cota no suele ser ajustada.

\section{Regla de Simpson}
\label{sec:regla-de-simpson}

  El siguiente paso es interpolar con una parábola
  (polinomio cuadrático).
  Hay varias maneras,
  más o menos complicadas,
  para deducir la fórmula del caso.
  Es recomendable el artículo de Talvila y Wiersma~%
    \cite{talvila12:_simpl_deriv_basic_quadr_formul},
  que resume derivaciones sencillas de los métodos más comunes.
  Acá tomaremos un camino distinto.

  Integrar el polinomio interpolador cuadrático significa \num{3} puntos,
  que para simplificar consideramos igualmente espaciados,
  en \(a\), \((a + b) / 2\) y \(b\).
  Una transformación lineal transforma esto en los puntos \(-1, 0, 1\),
  aún más cómodos.
  Lo que buscamos es una fórmula de la forma:
  \begin{equation*}
    \int_{-1}^1 f(x) \mathrm{d} x
      = A_{-1} f(-1) + A_0 f(0) + A_1 f(1)
  \end{equation*}
  Si esto es exacto para polinomios hasta de grado \num{2},
  quiere decir que:
  \begin{align*}
    \int_{-1}^1 \mathrm{d} x
      &= 2
       = A_{-1} + A_0 + A_1 \\
    \int_{-1}^1 x \mathrm{d} x
      &= 0
       = - A_{-1} + A_1 \\
    \int_{-1}^1 x^2 \mathrm{d} x
      &= \frac{2}{3}
       = A_{-1} + A_1
  \end{align*}
  Notamos que también:
  \begin{equation*}
    \int_{-1}^1 x^3 \mathrm{d} x
      = 0
      = - A_{-1} + A_1
  \end{equation*}
  que simplemente repite la ecuación que tenemos para \(x\),
  inesperadamente el método es exacto hasta grado~\num{3}.
  Lamentablemente,
  para grado~\num{4} no coincide:
  \begin{equation*}
    \int_{-1}^1 x^4 \mathrm{d} x
      = \frac{2}{5}
      = A_{-1} + A_1
  \end{equation*}
  contradiciendo la ecuación para grado~\num{2}.

  La solución a nuestro sistema de ecuaciones es
  \(A_{-1} = A_1 = 1/3\), \(A_0 = 4/3\).

  Lamentablemente,
  el resultado general~\eqref{eq:general-quadrature-error}
  sobreestima el error,
  en particular,
  no considera que la regla es exacta para polinomios cúbicos.
  Tenemos el siguiente resultado para la regla de Simpson:
  \begin{theorem}
    Sean \(f \in C^4([a, b])\),
    \(h = (b - a) / 2\),
    y llamemos \(x_0 = a\), \(x_1 = x_0 + h\), \(x_2 = b\).
    Entonces hay \(\xi \in [a, b]\) tal que:
    \begin{equation*}
      E
        = \int_a^b f(x) \mathrm{d} x
            - \frac{h}{3} (f(x_0) + 4 f(x_1) + f(x_2))
        = - \frac{(b - a)^5}{2880} f^{(4)}(\xi)
    \end{equation*}
  \end{theorem}
  \begin{proof}
    Esta demostración viene de Süli y Mayers~%
      \cite[capítulo 7]{sueli03:_intro_numerical_analysis}.
    Considere el cambio de variable:
    \begin{equation*}
      x(t)
        = x_1 + h t
        \quad t \in [-1, 1]
    \end{equation*}
    Defina \(F(t) = f(x(t))\),
    con lo que \(\mathrm{d} x = h \mathrm{d} t\),
    y nuestra integral es:
    \begin{equation*}
      \int_{x_0}^{x_2} f(x) \mathrm{d} x
        = h \int_{-1}^1 F(\tau) \mathrm{d} \tau
    \end{equation*}
    y el error es:
    \begin{equation*}
      E
        = \int_a^b f(x) \mathrm{d} x
            -  \frac{h}{3} (f(x_0) + 4 f(x_1) + f(x_2))
        = h \left(
              \int_{-1}^1 F(\tau) \mathrm{d} \tau
                - \frac{1}{3} (F(-1) + 4 F(0) + F(1))
            \right)
    \end{equation*}
    Para \(t \in [-1, 1]\) definamos la función:
    \begin{equation*}
      G(t)
        = \int_{-t}^t F(\tau) \mathrm{d} \tau
                - \frac{t}{3} (F(-t) + 4 F(0) + F(t))
    \end{equation*}
    En particular,
    el error de integración que nos interesa estimar es \(E = h G(1)\).
    Consideremos ahora:
    \begin{equation*}
      H(t)
        = G(t) - t^5 G(1)
    \end{equation*}
    Vemos que \(H(0) = H(1) = 0\),
    con lo que por el teorema de Rolle hay \(\xi_1 \in (0, 1)\)
    tal que \(H'(\xi_1) = 0\).
    Como a su vez \(H'(0) = 0\),
    por el teorema de Rolle hay \(\xi_2 \in (0, \xi_1) \subset (0, 1)\)
    tal que \(H''(\xi_2) = 0\).
    El mismo argumento muestra que hay \(\xi_3 \in (0, 1)\)
    con \(H'''(\xi_3) = 0\).
    Note que la tercera derivada de \(G\) es:
    \begin{equation*}
      G'''(t)
        = -  \frac{t}{3} (F'''(t) - F'''(-t))
    \end{equation*}
    de donde:
    \begin{equation*}
      H'''(\xi_3)
        = - \frac{\xi_3}{3} (F'''(\xi_3) - F'''(- \xi_3)) - 60 \xi_3^2 G(1)
        = 0
    \end{equation*}
    O sea,
    dividiendo la última igualdad por \(2 \xi_3^2 = \xi_3^2 - (-\xi_3^2)\),
    lo que es válido ya que \(\xi_3 \ne 0\),
    y reorganizando:
    \begin{equation*}
      - \frac{F'''(\xi_3) - F'''(-\xi_3)}{\xi_3 - (-\xi_3)}
        = 90 G(1)
    \end{equation*}
    Por el teorema del valor intermedio de la derivada,
    sabemos que hay \(\xi_4 \in (-\xi_3, \xi_3) \subset (-1, 1)\)
    tal que:
    \begin{equation*}
      90 G(1)
        = - F^{(4)}(\xi_4)
    \end{equation*}
    de donde tenemos para el error:
    \begin{align*}
      E
        &= - \frac{h}{90} F^{(4)}(\xi_4) \\
        &= - \frac{h^5}{90} f^{(4)}(x_1 + h \xi_4) \\
        &= - \frac{h^5}{90} f^{(4)}(\xi)
    \end{align*}
    donde \(\xi = x_1 + h \xi_4 \in (a, b)\).
  \end{proof}
  Esto explica el \(t^5\) mágico que empleamos antes:
  sabíamos que el error resultaría \(O(h^4)\),
  necesitábamos uno más.

\subsection{Fórmula compuesta}
\label{sec:composite-Simpson}

  Del desarrollo anterior para un par de intervalos \(x_i\) a \(x_{i + 2}\)
  tenemos que hay \(\xi_i \in (x_i, x_{i + 2})\)
  tal que el error para ese intervalo cumple:
  \begin{equation*}
    E_i
      = - \frac{h^5}{90} f^{(4)}(\xi_i)
  \end{equation*}
  Para el rango completo \([a, b]\) con \(n\) intervalos
  (sabemos que \(n\) debe ser par):
  \begin{align*}
    E
      &= \sum_{0 \le i < n / 2} E_i \\
      &= - \frac{h^5}{90} \sum_{0 \le i < n / 2} f^{(4)}(\xi_i) \\
      &= - \frac{n h^5}{2 \cdot 90}
             \frac{1}{n / 2} \sum_{0 \le i < n / 2} f^{(4)}(\xi_i) \\
      &= - \frac{(b - a) h^4 f^{(4)}(\xi)}{180}
  \end{align*}
  La existencia del valor \(\xi \in (a, b)\)
  es consecuencia del teorema del valor medio.

\section{Fórmulas de mayor grado}
\label{sec:cuadratura-alto-grado}

  Por la simetría del sistema de ecuaciones subyacente,
  si tenemos \(2 n + 1\) puntos \(-n, \dotsc, n\)
  es claro que debe ser \(A_{-k} = A_k\)
  y tendremos:
  \begin{align*}
    \sum_{-n \le k \le n} A_k k^{2 n + 1}
      &= A_0 0^{2 n + 1}
          + 2 \sum_{1 \le k \le n} A_k
                \left( (-k)^{2 n + 1} + k^{2 n + 1} \right) \\
      &= 0
  \end{align*}
  Vale decir,
  si el número \(n\) de puntos igualmente espaciados es impar
  (en realidad,
   basta que estén distribuidos simétricamente alrededor del \num{0}),
  la regla de cuadratura es exacta
  hasta para polinomios de grado~\(n\).
  No ocurre lo mismo si \(n\) es par,
  en cuyo caso es exacta para polinomios hasta grado~\(n - 1\),
  como plantea el sistema de ecuaciones,
  pero para grado~\(n\)
  (que es par)
  falla.

  Vimos el fenómeno de Runge,
  usar polinomios interpolantes de alto grado con puntos igualmente espaciados
  puede llevar al desastre.
  Por esa razón en la práctica se usan fórmulas compuestas
  con pocos puntos en cada intervalo,
  como la regla de Simpson.
  Una opción para disminuir el error es usar puntos de Chebyshev,
  pero eso complica las fórmulas.
  No nos detendremos en esto,
  veremos reglas mejores en el capítulo~\ref{cha:cuadratura-gauss}.

\section{Aceleración de Richardson}
\label{sec:richardson-acc}

  Una técnica general de aceleración de secuencias es la de Richardson~%
    \cite{richardson11:_approx_arith_solut_finit_differ,
          richardson27:_deferred_approach_limit}.
  Supongamos que tenemos una aproximación a una cantidad \(A^*\)
  que cumple:
  \begin{equation}
    \label{eq:richardson-start}
    A(h)
      = A^* + C h^n + O(h^{n + 1})
  \end{equation}
  Si despreciamos el término de orden mayor,
  y calculamos aproximaciones para dos valores de \(h\),
  tenemos dos ecuaciones en las incógnitas \(A^*\) y \(C\):
  \begin{align*}
    A(h)
      &= A^* + C h^n \\
    A(h/t)
      &= A^* + C (h/t)^n
  \end{align*}
  Obtenemos una nueva aproximación \(A^+\):
  \begin{equation*}
    A^+
      = \frac{t^n A(h/t) - A(h)}{t^n - 1}
  \end{equation*}
  Vemos que:
  \begin{align*}
    A^+
      &= \frac{t^n (A^* + C (h/t)^n + O(h^{n + 1}))
                 - (A^* + C h^n + O(h^{n + 1}))}
              {t^n - 1} \\
      &= A^* + O(h^{n + 1})
  \end{align*}
  Calculando con dos pasos distintos aumentamos el orden del método.

  Una aplicación es integración de Romberg~%
    \cite{romberg55:_vereinfachte_integration}:
  aproxima la integral mediante trapezoides
  o usando la regla del punto medio
  para \(h\) y \(h/2\),
  y usa extrapolación de Richardson
  con \(t = 2\) para obtener una aproximación mejor.
  Como pueden reutilizarse nodos,
  el trabajo requerido es solo aproximadamente el para el paso menor.
  Si la estimación del error resultante es demasiado grande,
  podemos volver a disminuir el paso a la mitad
  (calculando la función para puntos intermedios).

  Esto puede extenderse a \emph{cuadratura adaptativa}:
  vamos monitorizando el error en cada subintervalo,
  deteniendo el refinamiento en este
  cuando hemos alcanzado la precisión requerida para él.
  Esto es aplicable a funciones con comportamientos distintos
  en el intervalo de interés.

\section*{Ejercicios}
\label{sec:ejercicios-06-previa}

\begin{enumerate}
\item
  Derive algunas de las fórmulas de cuadratura
  integrando los polinomios interpolantes.
  Compare con nuestras derivaciones.
\item
  Derive fórmulas para el error de algunas de las fórmulas de cuadratura
  usando la técnica tradicional.
  Compare con nuestras derivaciones.
\item
  Hay una segunda regla de Simpson,
  que parte de interpolación cúbica.
  Derive esa fórmula mediante nuestra técnica
  de plantear un sistema de ecuaciones,
  junto con la estimación del error
  (conviene elegir \num{4} puntos igualmente espaciados,
   centrados en \num{0}).
\item
  Use un sistema de álgebra simbólica
  para obtener los polinomios interpolantes con
  \numlist{3; 5; 7; 9; 11}  puntos igualmente espaciados
  de la función de Runge,
  y compare la integral exacta con la integral de los polinomios.
\item
  Use un sistema de álgebra simbólica
  para obtener los polinomios interpolantes
  con \numlist{3; 5; 7; 9; 11} puntos de Chebyshev
  de la función de Runge,
  y compare la integral exacta con la integral de los polinomios.
\end{enumerate}

\bibliography{../referencias}

%%% Local Variables:
%%% mode: latex
%%% TeX-master: "../INF-221_notas"
%%% ispell-local-dictionary: "spanish"
%%% End:

% LocalWords:  eq quadrature coefs rectangulitos subintervalos anti
% LocalWords:  Ad Hoc initial point rectangles triangulitos def
% LocalWords:  function subintervalo sobreestima

\bibliographystyle{babplain-fl}

\chapter{Cuadratura Gaussiana}
\label{cha:cuadratura-gauss}

  Estamos interesados en investigar la posibilidad de escribir cuadraturas
  (cálculo de la integral definida)
  más precisas sin incrementar el número de \emph{puntos de cuadratura}
  (los llamaremos \(x_0, x_1, \dotsc, x_n\)).
  Esto puede ser posible si nos tomamos la libertad de escoger estos puntos.
  Por lo tanto,
  el problema de cuadratura se transforma
  en un problema de escoger los puntos de cuadratura
  en adición a determinar los respectivos coeficientes
  tal que la cuadratura es exacta para los polinomios de grado máximo.
  Las cuadraturas que son obtenidas con este método
  se conocen como \emph{cuadratura gaussianas}.

  \begin{ejemplo}[Cuadratura gaussiana con dos puntos]
    \label{06::CuadGauss:n2}
    Supongamos que queremos encontrar dos puntos de cuadratura de la ecuación:
    \begin{equation}
      \label{06::Asterisco}
      \int_{-1}^1 f(x) \mathrm{d} x
        \approx A_0 f(x_0) + A_1 f(x_1)
      \end{equation}
      Como queremos encontrar los valores de \(A_0, A_1, x_0\) y \(x_1\),
      esperamos que la ecuación~\eqref{06::Asterisco}
      sea exacta para polinomios hasta de grado \(2 \cdot 2 - 1 = 3\).
      O sea,
      es exacta para \(1, x, x^2, x^3\)
      (puede reemplazar \(f(x)\)
       por cualquier otro conjunto de polinomios linealmente independientes
       de grado hasta \num{3},
       claro,
       si es que busca complicarse la existencia\ldots).
      Entonces,
      reemplazamos \(f(x) = x^k\) con \(k \in \{ 0, 1, 2, 3 \}\)
      para la \(k\)-ésima ecuación,
      y con ello,
      formamos el siguiente sistema de ecuaciones:
      \begin{equation}
        \label{06::SistemaEcuaciones}
        \int_{-1}^{1} x^k \mathrm{d} x
          = A_0 x_0^k + A_1 x_1^k
              \qquad\qquad k \in \{ 0, 1, 2, 3 \}
      \end{equation}
      Resolvemos la integral:
      \begin{align*}
        \int_{-1}^1 x^k \mathrm{d} x
          &= \frac{x^{k + 1}}{k + 1} \Bigg |_{-1}^1
           = \begin{cases}
               0,		&\text{si \(k\) es impar} \\
               \frac{2}{k + 1}, &\text{si \(k\) es par}
             \end{cases}
      \end{align*}
      y reemplazamos en el sistema de ecuaciones~\eqref{06::SistemaEcuaciones}:
      \begin{equation}
        \label{eq:Gauss-2}
        \begin{split}
          2
            &= A_0 + A_1 \\
          0
            &= A_0 x_0 + A_1 x_1 \\
          \frac{2}{3}
            &= A_0 x_0^2 + A_1 x_1^2 \\
          0
            &= A_0 x_0^3 + A_1 x_1^3
        \end{split}
      \end{equation}
      Al resolver el sistema de ecuaciones~\eqref{eq:Gauss-2} se obtiene:
      \begin{equation*}
        A_0 = 1,\qquad A_1 = 1,\qquad
        x_0 = - \frac{1}{\sqrt{3}},\qquad x_1 = \frac{1}{\sqrt{3}}
      \end{equation*}
    \end{ejemplo}
    Una observación es que los coeficientes
    de los valores extremos \(A_0\) y \(A_1\) son iguales:
    \begin{equation*}
      A_0 = A_1
    \end{equation*}
    y que los puntos de cuadratura extremos \(x_0\) y \(x_1\) son opuestos,
    vale decir:
    \begin{equation*}
      x_0 = - x_1
    \end{equation*}
    Lo anterior es extensible para \(n\) puntos de cuadratura,
    donde para \(0 \le i < n\):
    \begin{align*}
      A_i
        &= A_{n - i - 1} \\
      x_i
        &= - x_{n - i - 1}
    \end{align*}

  \begin{ejemplo}[Cuadratura gaussiana con tres puntos]
    Supongamos que queremos encontrar tres puntos de cuadratura,
    entonces usamos la ecuación:
    \begin{align*}
      \int_{-1}^{1} f(x) \mathrm{d} x
        \approx A_0 f(x_0) + A_1 f(x_1) + A_2 f(x_2)
    \end{align*}
    Sospechamos que:
    \begin{align*}
      x_0
        &= - x_2 \\
      x_1
        &= 0 \\
      A_0
        &= A_2
    \end{align*}
    Siguiendo los pasos del ejemplo~\ref{06::CuadGauss:n2},
    el cuadro~\ref{06::SistemaEcuaciones:Cuadro}
    exhibe el sistema de ecuaciones a resolver.
    \begin{table}[ht]
      \centering
      \begin{tabular}{>{\(}c<{\)}|>{\(}c<{\)}@{\quad = \quad}>{\(}c<{\)}}
        k & \mathlarger{\int}_{-1}^1 x^k \mathrm{d} x
          & A_0x_0^k + A_1 x_1^k + A_2 x_2^k \\[7pt]
        \hline
        0 & 2		& A_0 \phantom{x_0}
                            + A_1 \phantom{x_1}
                            + A_2 \phantom{x_2} \\
        1 & 0		& A_0 x_0   + A_1 x_1	+ A_2 x_2 \\
        2 & 2/3		& A_0 x_0^2 + A_1 x_1^2 + A_2 x_2^2 \\
        3 & 0		& A_0 x_0^3 + A_1 x_1^3 + A_2 x_2^3 \\
        4 & 2/5		& A_0 x_0^4 + A_1 x_1^4 + A_2 x_2^4 \\
        5 & 0		& A_0 x_0^5 + A_1 x_1^5 + A_2 x_2^5 \\
      \end{tabular}
      \caption{Comprobamos nuestras sospechas.}
      \label{06::SistemaEcuaciones:Cuadro}
    \end{table}

    Comenzamos con la ecuación para \(k = 2\),
    la ecuación para \(k = 0\) no es demasiado informativa en este punto
    y nuestra sospecha sobre los \(x_j\)
    hace que se cumplan las ecuaciones para valores impares de \(k\):
    \begin{align}
      \frac{2}{3}
        &= 2 A_0 x_2^2
             \notag \\
      x_2^2
        &= \frac{1}{3 A_0}
             \label{06::Ej2:x2:inicio}
    \end{align}
    Continuamos con \(k = 4\):
    \begin{align*}
      \frac{2}{5}
        & = 2 A_0 x_2^4 \\
      A_0
        &= \frac{5}{9}
    \end{align*}
    Luego,
    reemplazamos \(A_0\) en~\eqref{06::Ej2:x2:inicio}:
    \begin{align*}
      x_2^2
        &= \frac{1}{3\cdot \dfrac{5}{9}} \\
      x_2
        &= \sqrt{\frac{3}{5}}
    \end{align*}
    Además:
    \begin{align*}
      A_1
        &= 2 - 2 A_0 \\
        &= \frac{8}{9}
    \end{align*}
    Basta reemplazar nuestra solución en el sistema de ecuaciones
    para comprobarla.
  \end{ejemplo}

\section{Teoría de cuadraturas gaussianas}
\label{sec:cuadratura-gauss-teo}

  Muy bonito todo lo anterior\ldots{}
  pero no es una técnica particularmente elegante,
  y no da luces
  sobre el comportamiento de las reglas de cuadratura resultantes.
  Para tratar ese tema,
  se requiere un desvío.
  Lo siguiente es básicamente de Levy~%
    \cite[capítulos~4 y~6]{levy10:_introd_numer_analy},
  sazonado con un poco de Treil~%
    \cite[capítulo~5]{treil17:_linear_algeb_done_wrong}.
  La teoría requerida sobre espacios normados
  se reseña en el apéndice~\ref{apx:espacios-normados}.

\subsection{Polinomios ortogonales}
\label{sec:polinomios-ortogonales}

  Para simplificar notación,
  llamaremos \(\Pi_n\)
  al conjunto de polinomios con coeficientes reales
  de grado menor o igual a \(n\).
  Notamos que \(\Pi_n\) es un espacio vectorial de dimensión \(n + 1\)
  (una base es el conjunto \(\{1, x, x^2, \dotsc, x^n\}\))
  y que \(\Pi_m\) es un subespacio de \(\Pi_n\) si \(m < n\).

  \begin{definition}
    \label{def:producto-interno}
    Consideremos un intervalo \([a, b]\),
    y una función peso \(w \colon [a, b] \to \mathbb{R}\),
    continua y positiva.
    Dadas dos funciones \(f\) y \(g\),
    continuas sobre el intervalo,
    definimos su \emph{producto interno}
    (sobre \([a, b]\) respecto de \(w\)) por:
    \begin{equation*}
      \langle f, g \rangle_w
        = \int_a^b w(x) f(x) g(x) \mathrm{d} x
    \end{equation*}
    Definimos la \emph{norma}
    (sobre \([a, b]\) respecto de \(w\)) por:
    \begin{equation*}
      \lVert f \rVert_w
        = \sqrt{\langle f, f \rangle_w}
    \end{equation*}

    Decimos que \(f\) y \(g\) son \emph{ortogonales}
    (sobre \([a, b]\) con peso \(w\))
    si \(\langle f, g \rangle_w = 0\).
    Decimos que \(f\) está \emph{normalizada}
    (sobre \([a, b]\) con peso \(w\))
    si \(\lVert f \rVert_w = 1\)
  \end{definition}
  Es claro de la definición que la norma nunca es negativa.
  Comúnmente omitimos el subíndice,
  la función peso y el intervalo se subentienden.

  \begin{definition}
    \label{def:polinomios-ortogonales}
    Sea \(w\) un peso sobre \([a, b]\).
    Diremos que la secuencia de polinomios \(p_n(x)\)
    con \(\deg(p_n) = n\)
    son \emph{\(w\)\nobreakdash-ortogonales}
    (o simplemente \emph{ortogonales},
     si \(w\) se subentiende)
    si \(\langle p_i, p_j \rangle_w = 0\) para todo \(i \ne j\).
    Los llamaremos \emph{\(w\)\nobreakdash-orto\-nor\-ma\-les}
    (o simplemente \emph{ortonormales})
    si además \(\lVert p_i \rVert_w = 1\).
  \end{definition}
  Dado un conjunto de vectores linealmente independientes
  (en nuestro caso,
   \(\{ 1, x, x^2, \dotsc \}\)),
  el proceso de Gram-Schmidt
  (ver cualquier texto de álgebra lineal,
   recomendamos el de Treil~%
     \cite{treil17:_linear_algeb_done_wrong},
   o refiérase al apéndice~\ref{apx:espacios-normados})
  permite construir un conjunto ortogonal.

  \begin{theorem}
    \label{theo:polinomios-ortogonales-ceros}
    Sea \(p_n(x)\) un polinomio ortogonal de grado \(n\)
    sobre \([a, b]\) con peso \(w\).
    Entonces \(p_n\) tiene \(n\) ceros reales simples en \([a, b]\).
  \end{theorem}
  \begin{proof}
    Sean \(x_1, \dotsc, x_r\) los ceros de \(p_n\) en \([a, b]\)
    (pueden repetirse),
    y consideremos:
    \begin{equation*}
      q(x)
        = (x - x_1) (x - x_2) \dotsm (x - x_r)
    \end{equation*}
    Claramente \(\deg(q) = r \le n\).
    En \([a, b]\),
    \(p_n(x) q(x)\) tiene un único signo
    (todos los ceros restantes de \(p_n\) caen fuera de \([a, b]\)),
    por lo que:
    \begin{equation*}
      \int_a^b w(x) p_n(x) q(x) \mathrm{d} x
        \ne 0
    \end{equation*}
    Pero \(p_n\) es ortogonal a todo \(\Pi_{n - 1}\),
    por lo que \(\deg(q) = n\).

    Supongamos que \(x_1\) es un cero múltiple de \(p_n\),
    y analicemos:
    \begin{equation*}
      p_n(x)
        = (x - x_1)^2 g(x)
    \end{equation*}
    O sea:
    \begin{equation*}
      p_n(x) g(x)
        =   (x - x_1)^2 g^2(x)
        =   \left( \frac{p_n(x)}{x - x_1} \right)^2
        \ge 0
    \end{equation*}
    Por tanto:
    \begin{equation*}
      \int_a^b w(x) p_n(x) g(x) \mathrm{d} x
        > 0
    \end{equation*}
    Esto se contradice con \(g \in \Pi_{n - 2}\),
    con los que \(p_n\) es ortogonal.
  \end{proof}

\subsection{Cuadratura de Gauß}
\label{sec:cuadratura-Gauss}

  Buscamos reglas de cuadratura de la forma:
  \begin{equation}
    \label{eq:regla-cuadratura}
    \int_a^b w(x) f(x) \mathrm{d} x
      = \sum_{0 \le i \le n} A_i f(x_i)
  \end{equation}
  La ecuación~\eqref{eq:regla-cuadratura} es exacta para todo \(f \in \Pi_n\)
  si y solo si
  (esto es básicamente por la forma de Lagrange del polinomio interpolador):
  \begin{equation}
    \label{eq:regla-cuadratura-Ai}
    A_i
      = \int_a^b w(x)
          \prod_{\substack{0 \le j \le n \\
                           j \ne i}} \frac{x - x_j}{x_i - x_j} \mathrm{d} x
  \end{equation}
  En~\eqref{eq:regla-cuadratura} tenemos \(2 n + 2\) grados de libertad,
  (\(x_0, \dotsc, x_n\) y \(A_0, \dotsc, A_n\)),
  aspiramos a una regla que sea exacta en \(\Pi_{2 n + 1}\).
  \begin{theorem}
    \label{theo:regla-gaussiana-exacta-2n+1}
    Sea \(q\) un polinomio de grado \(n + 1\),
    ortogonal a \(\Pi_n\),
    o sea para todo \(p \in \Pi_n\):
    \begin{equation*}
      \int_a^b w(x) p(x) q(x) \mathrm{d} x
        = 0
    \end{equation*}
    Si \(x_i\) para \(0 \le i \le n\) son los ceros de \(q\),
    entonces la regla de cuadratura~\eqref{eq:regla-cuadratura}
    con los coeficientes~\eqref{eq:regla-cuadratura-Ai}
    es exacta para todo \(f \in \Pi_{2 n + 1}\).
  \end{theorem}
  \begin{proof}
    Por construcción,
    nuestra fórmula es exacta para polinomios hasta grado \(n\).
    Sea \(f \in \Pi_{2 n + 1}\),
    escribamos por el algoritmo de división:
    \begin{equation*}
      f(x)
        = q(x) p(x) + r(x)
    \end{equation*}
    Acá \(\deg(r) \le n\),
    con lo que	notamos que \(p, r \in \Pi_n\).
    En los ceros de \(q\) tenemos:
    \begin{equation*}
      f(x_i)
        = r(x_i)
    \end{equation*}
    Por lo tanto:
    \begin{align*}
      \int_a^b w(x) f(x) \mathrm{d} x
        &= \int_a^b w(x) (q(x) p(x) + r(x)) \mathrm{d} x \\
        &= \int_a^b w(x) r(x) \mathrm{d} x \\
        &= \sum_{0 \le i \le n} A_i r(x_i) \\
        &= \sum_{0 \le i \le n} A_i f(x_i)
    \end{align*}
    y la regla es exacta para \(f\).
  \end{proof}
  Otro punto interesante es el siguiente:
  \begin{lemma}
    \label{lem:cuadratura-gaussiana-suma-coeficientes}
    En una regla de cuadratura gaussiana,
    los coeficientes son positivos
    y su suma es:
    \begin{equation*}
      \sum_{0 \le i \le n} A_i
        = \int_a^b w(x) \mathrm{d} x
    \end{equation*}
  \end{lemma}
  \begin{proof}
    Fijemos \(n\),
    y sea \(q \in \Pi_{n + 1}\) \(w\)\nobreakdash-ortogonal a \(\Pi_n\)
    (con esto \(\deg(q) = n + 1\)),
    donde \(q(x_i) = 0\)
    en los puntos de cuadratura \(\{ x_i \}_{0 \le i \le n}\):
    \begin{equation*}
      \int_a^b w(x) f(x) \, \mathrm{d} x
        \approx \sum_{0 \le i \le n} A_i f(x_i)
    \end{equation*}
    Fijemos \(0 \le j \le n\),
    y sea dado por:
    \begin{equation*}
      p(x)
        = \frac{q(x)}{x - x_j}
    \end{equation*}
    Siendo \(x_j\) un cero de \(q\),
    \(p\) es un polinomio de grado \(n\),
    con lo que \(\deg(p^2) = 2 n\),
    así que la siguiente es exacta:
    \begin{equation*}
      0
        < \int_a^b w(x) p^2(x) \mathrm{d} x
        = \sum_{0 \le i \le n} A_i p^2(x_i)
        = A_j p^2(x_j)
    \end{equation*}
    Concluimos que \(A_j > 0\).

    Por el otro lado,
    la regla de cuadratura es exacta para \(1 \in \Pi_{2 n + 1}\):
    \begin{equation*}
      \int_a^b w(x) \mathrm{d} x
        = \sum_{0 \le i \le n} A_i
    \end{equation*}
  \end{proof}

\section{Más sobre Cuadratura Gaussiana}
\label{sec:mas-cuadratura-gauss}

  Interesan métodos de cálculo mejores para los polinomios ortogonales,
  y derivar fórmulas para el error de reglas de cuadratura gaussianas.
  Antes de entrar en el tema,
  daremos un desvío.

\subsection{Interpolación de Hermite}
\label{sec:interpolacion-Hermite}

  Un problema de interpolación interesante se presenta
  si se dan los valores de la función
  y adicionalmente se especifican valores para derivadas.
  Obtener un polinomio que cumple estas condiciones
  se conoce como \emph{interpolación de Hermite}.
  Mayores detalles se hallan por ejemplo en el texto de Levy~%
    \cite{levy10:_introd_numer_analy},
  acá requeriremos solo un caso muy particular.

  Supongamos que tenemos los valores de las función \(f\)
  y los de su primera derivada en los puntos \(x_i\)
  para \(0 \le i \le n\).
  Buscamos un polinomio que coincida con ellos.
  Claramente,
  tal polinomio será de grado \(2 n + 1\)
  (tenemos \(2 n + 2\) condiciones).
  Podemos escribirlo:
  \begin{equation}
    \label{eq:Hermite-A-B}
    p(x)
      = \sum_{0 \le i \le n} f(x_i) A_i(x)
          + \sum_{0 \le i \le n} f'(x_i) B_i(x)
  \end{equation}
  para polinomios \(A_i, B_i\) de grado \(2 n + 1\),
  donde requerimos:
  \begin{equation}
    \label{eq:A-B-requirements}
    \begin{array}{ll}
      A_i(x_j) = [i = j] & A_i'(x_j) = 0 \\
      B_i(x_j) = 0	 & B_i'(x_j) = [i = j]
    \end{array}
  \end{equation}
  Notamos que los polinomios \(l_i\) definidos por~\eqref{eq:Lagrange-bases}
  para los puntos \(x_0, \dotsc, x_n\) cumplen:
  \begin{equation*}
    l_i(x_j)
      = [i = j]
        \quad \text{para \(0 \le j \le n\)}
  \end{equation*}
  El cuadrado de \(l_i\) tiene ceros dobles en \(x_j\) para \(j \ne i\):
  \begin{equation*}
    l_i^2(x_j) = 0
      \quad \left( l_i^2(x_j) \right)' = 0
  \end{equation*}
  Como \(l_i\) es de grado \(n\),
  el grado de \(l_i^2\) es \(2 n\).
  Esto sugiere que los polinomios \(A_i\) y \(B_i\) pueden escribirse:
  \begin{align*}
    A_i(x)
      &= r_i(x) l_i^2(x) \\
    B_i(x)
      &= s_i(x) l_i^2(x)
  \end{align*}
  con polinomios lineales \(r_i, s_i\).
  Ahora bien,
  por~\eqref{eq:A-B-requirements}:
  \begin{equation*}
    [i = j]
      = A_i(x_j)
      = r_i(x_j) l_i^2(x_j)
      = r_i(x_j) [i = j]
  \end{equation*}
  por lo que \(r_i(x_i) = 1\).
  Por el otro lado requerimos:
  \begin{equation*}
    0
      = A_i'(x_j)
      = r_i'(x_j)l_i^2(x_j) + 2 r_i(x_j) l_i(x) l_i'(x)
      = r_i'(x_j) [i = j] + 2 r_i(x_j) [i = j] l_i'(x_j)
  \end{equation*}
  Esto se resume en:
  \begin{equation*}
    r_i'(x_i) + 2 l_i'(x_i)
      = 0
  \end{equation*}
  Suponiendo \(r_i\) lineal,
  resulta:
  \begin{equation*}
    r_i(x)
      = - 2 l_i'(x_i) x + (1 + 2 l_i'(x_i) x_i)
  \end{equation*}
  que lleva a:
  \begin{equation*}
    A_i(x)
      = \left( 1 - 2 l_i'(x_i) (x - x_i) \right) l_i^2(x)
  \end{equation*}
  De forma similar,
  de~\eqref{eq:A-B-requirements}:
  \begin{align*}
    0
      &= B_i(x_j)
       = s_i(x_j) l_i^2(x_j) \\
    [i = j]
      &= B_i'(x_j)
       = s_i'(x_j) l_i^2(x_j) + 2 s_i(x_j) \left( l_i^2(x_j) \right)'
  \end{align*}
  Concluimos que:
  \begin{equation*}
    s_i(x)
      = x - x_i
  \end{equation*}
  con lo que:
  \begin{equation*}
    B_i(x)
      = (x - x_i) l_i^2(x)
  \end{equation*}
  y finalmente el polinomio interpolador de Hermite
  toma la forma:
  \begin{equation}
    \label{eq:p-Hermite}
    p(x)
      = \sum_{0 \le i \le n} \left( 1 - 2 l_i'(x_i) (x - x_i) \right)
                             l_i^2(x) f(x_i)
          + \sum_{0 \le i \le n} (x - x_i) l_i^2(x) f'(x_i)
  \end{equation}
  Para el error en~\eqref{eq:p-Hermite}
  tenemos:
  \begin{theorem}
    \label{theo:Hermite-interpolation-error}
    Sean \(x_0, \dotsc, x_n\) puntos distintos en \([a, b]\)
    y \(f \in C^{2 n + 2}[a, b]\).
    Si \(p \in \Pi_{2 n + 1}\) es tal que para \(0 \le i \le n\):
    \begin{equation*}
      p(x_i) = f(x_i)
        \qquad
        p'(x_i) = f'(x_i)
    \end{equation*}
    entonces hay \(\xi \in (a, b)\) tal que:
    \begin{equation}
    \label{eq:Hermite-interpolation-error}
      f(x) - p(x)
        = \frac{f^{(2 n + 2)}(\xi)}{(2 n + 2)!} \prod_{0 \le i \le n} (x - x_i)^2
    \end{equation}
  \end{theorem}
  \begin{proof}
    La técnica es similar a la empleada
    para demostrar el teorema~\ref{theo:interpolation-error}.
    Sea \(x\) un punto fijo en	\([a, b]\),
    distinto de los \(x_i\),
    y definimos:
    \begin{equation*}
      w(y)
        = \prod_{0 \le i \le n} (y - x_i)^2
    \end{equation*}
    Definimos también:
    \begin{equation*}
      \phi(y)
        = f(y) - p(y) - \lambda w(y)
    \end{equation*}
    donde elegimos \(\lambda\) de manera que \(\phi(x) = 0\),
    o sea:
    \begin{equation*}
      \lambda
        = \frac{f(x) - p(x)}{w(x)}
    \end{equation*}
    Entonces \(\phi\) tiene al menos \(n + 2\) ceros en \([a, b]\),
    a saber \(x, x_0, \dotsc, x_n\).
    Por el teorema de Rolle,
    \(\phi'\) tiene (al menos) \(n + 1\) ceros distintos de los anteriores,
    y sabemos que \(\phi'\) se anula en \(x_0, \dotsc, x_n\),
    con lo que \(\phi'\) tiene al menos \(2 n + 2\) ceros en \((a, b)\).
    Finalmente \(\phi^{(2 n + 2)}\) tiene al menos un cero en \((a,  b)\),
    llamémosle \(\xi\):
    \begin{equation*}
      0
        = \phi^{(2 n + 2)}(\xi)
        = f^{(2 n + 2)}(\xi) - p^{(2 n + 2)}(\xi) - \lambda w^{(2 n + 2)}(\xi)
    \end{equation*}
    Como \(p\) es un polinomio de grado \(2 n + 1\)
    y \(w\) un polinomio mónico de grado \(2 n + 2\),
    resulta lo indicado.
  \end{proof}

\subsection{Error de reglas de cuadratura gaussiana}
\label{sec:error-integracion-gaussiana}

  Estamos en condiciones de obtener el error
  de la regla de cuadratura gaussiana.
  \begin{theorem}
    \label{theo:gauss-quadrature-error}
    Sea \(f \in C^{2 n + 2}[a, b]\) y sea \(w\) una función de peso
    en ese intervalo.
    Considere la cuadratura gaussiana:
    \begin{equation*}
      \int_a^b w(x) f(x) \mathrm{d} x
        \approx \sum_{0 \le i \le n} A_i f(x_i)
    \end{equation*}
    Entonces existe \(\zeta \in (a, b)\) tal que
    \begin{equation*}
      \int_a^b w(x) f(x) \mathrm{d} x - \sum_{0 \le i \le n} A_i f(x_i)
        = \frac{f^{(2 n + 2)}(\zeta)}{(2 n + 2)!}
            \int_a^b w(x) \prod_{0 \le i \le n} (x - x_i)^2 \, \mathrm{d} x
    \end{equation*}
  \end{theorem}
  \begin{proof}
    Consideremos interpolación de Hermite en puntos \(x_0, \dotsc, x_n\),
    cuyo error por el teorema~\ref{theo:Hermite-interpolation-error}
    para algún \(\xi \in (a, b)\) cumple:
    \begin{equation*}
      f(x) - p(x)
        = \frac{f^{(2 n + 2)}(\xi)}{(2 n + 2)!} \prod_{0 \le i \le n} (x - x_i)^2
    \end{equation*}
    Tenemos la fórmula~\eqref{eq:p-Hermite} para \(p\):
    \begin{equation}
      p(x)
        = \sum_{0 \le i \le n} \left( 1 - 2 l_i'(x_i) (x - x_i) \right)
                               l_i^2(x) f(x_i)
            + \sum_{0 \le i \le n} (x - x_i) l_i^2(x) f'(x_i)
    \end{equation}
    Vemos que:
    \begin{equation*}
      (x - x_i) l_i^2(x)
        = l_i(x) \prod_{0 \le k \le n} (x - x_k)
    \end{equation*}
    Para una cuadratura gaussiana
    \(x_0, \dotsc, x_n\)
    son los ceros del polinomio ortogonal de grado \(n + 1\),
    el producto es un múltiplo de ese polinomio,
    ortogonal a \(l_i\) de grado \(n\).
    Las integrales de los términos
    que involucran las derivadas \(f'(x_i)\) se anulan.

    En consecuencia,
    tenemos
    (recuerde que \(\xi\) depende de \(x\)):
    \begin{align*}
      \int_a^b w(x) f(x) \, \mathrm{d} x
        - \sum_{0 \le i \le n} A_i f(x_i)
        &= \int_a^b w(x) f(x) \, \mathrm{d} x
             - \int_a^b w(x) p(x) \, \mathrm{d} x \\
        &=  \int_a^b w(x) \frac{f^{(2 n + 2)}(\xi)}{(2 n + 2)!}
              \prod_{0 \le i \le n} (x - x_i)^2 \, \mathrm{d} x
    \end{align*}
    Por el teorema del valor intermedio de la integral,
    teorema~\ref{theo:mean-value-integral},
    hay \(\zeta\) tal que la última integral puede expresarse:
    \begin{equation*}
      \frac{f^{(2 n + 2)}(\zeta)}{(2 n + 2)!}
         \int_a^b w(x) \prod_{0 \le i \le n} (x - x_i)^2 \, \mathrm{d} x
    \end{equation*}
    que es lo que asevera el teorema.
  \end{proof}

\section{La recurrencia para polinomios ortogonales}
\label{sec:recurrencia-polinomios-ortogonales}

  Los polinomios ortogonales tienen bastantes propiedades notables.
  Comentaremos una de ellas.
  \begin{theorem}
    \label{theo:triple-recursion-relation}
    La secuencia de polinomios ortogonales \(P_n\)
    cumple una recurrencia de la forma:
    \begin{equation}
      \label{eq:triple-recursion-relation}
      P_{n + 1}(x)
        = (A_n x + B_n) P_n(x) + C_n P_{n - 1}(x)
    \end{equation}
    Si \(P_n(x) = a_n x^n + b_n x^{n - 1} + \dotsb\)
    entonces:
    \begin{equation}
      \label{eq:triple-recursion-relation-coefs}
      A_n
        = \frac{a_{n + 1}}{a_n}
        \quad
      B_n
        = \frac{a_{n + 1}}{a_n}
            \left( \frac{b_{n + 1}}{a_{n + 1}} - \frac{b_n}{a_n} \right)
        \quad
      C_n
        = \frac{a_{n + 1} a_{n - 1}}{a_n^2}
    \end{equation}
  \end{theorem}
  \begin{proof}
    Es claro que el polinomio:
    \begin{equation*}
      Q_n(x)
        = P_{n + 1}(x) - A_n x P_n(x)
    \end{equation*}
    es de grado a lo más \(n\),
    por lo que podemos escribirlo como:
    \begin{equation*}
      Q_n(x)
        = \sum_{0 \le i \le n} \alpha_i P_i(x)
    \end{equation*}
    Los coeficientes pueden expresarse usando el producto interno:
    \begin{equation*}
      \alpha_i
        = \frac{\langle Q_n, P_i \rangle}{\langle P_i, P_i \rangle}
        = \frac{\langle P_{n + 1} - A_n x P_n, P_i \rangle}
               {\langle P_i, P_i \rangle}
        = \frac{\langle P_{n + 1}, P_i \rangle
                  - \langle A_n x P_n, P_i \rangle}
               {\langle P_i, P_i \rangle}
        = - A_n \frac{\langle x P_n, P_i \rangle}
                     {\langle P_i, P_i \rangle}
    \end{equation*}
    Por la definición del producto interno:
    \begin{equation*}
      \langle f, g \rangle_w
        = \int_a^b w(x) f(x) g(x) \mathrm{d} x
    \end{equation*}
    vemos que:
    \begin{equation*}
      \langle x P_n, P_i \rangle
        = \langle P_n, x P_i \rangle
    \end{equation*}
    y como \(P_n\) es ortogonal a todo \(\Pi_{n - 1}\):
    \begin{equation*}
      \langle P_n, x P_i \rangle
        = 0
        \qquad 0 \le i \le n - 2
    \end{equation*}
    O sea:
    \begin{equation*}
      P_{n + 1}(x)
        = A_n x P_n(x) + \alpha_n P_n + \alpha_{n - 1} P_{n - 1}(x)
    \end{equation*}
    Reordenando
    y comparando coeficientes de \(x^n\) y \(x^{n - 1}\) a ambos lados
    completa los valores indicados para \(B_n\) y \(C_n\) citados.
  \end{proof}

% To do:
% - Table of polynomials and points?
% - Program(s) to find them (SciPy, SymPy, ...)
% - Sturm-Liouville theory (!)

\section*{Ejercicios}
\label{sec:ejercicios-06}

  \begin{enumerate}
  \item
    Demostrar que \(\langle f, g \rangle_w\)
    es un producto interno.
  \item
    Demostrar que los coeficientes \(A_i\)
    están dados por la ecuación~\eqref{eq:regla-cuadratura-Ai}.
  \end{enumerate}

\bibliography{../referencias}

%%% Local Variables:
%%% mode: latex
%%% TeX-master: "../INF-221_notas"
%%% ispell-local-dictionary: "spanish"
%%% End:

% LocalWords:  Gaussiana gaussianas ésima eq pt subespacio nor ma Ai
% LocalWords:  ortonormales gaussiana ll requirements


\part{Algoritmos Combinatorios}
\thispagestyle{empty}

\bibliographystyle{babplain-fl}

\chapter{Algoritmos discretos}
\label{cha:discrete-algorithms}

  Nos interesan algoritmos que operan con objetos \emph{discretos}
  (a diferencia de los objetos continuos que hemos discutido hasta acá).
  Nos interesa determinar cuántos recursos
  se requieren para resolver un problema
  (cotas inferiores),
  inventar buenos algoritmos,
  evaluar el rendimiento de los algoritmos para compararlos entre sí
  y para compararlos con cotas teóricas.
  Hay dos maneras en que se organiza la enorme cantidad de material relevante:
  una es dar un catálogo de problemas ordenados por área
  y proponer algoritmos para ellos
  (es fundamentalmente lo que hace el clásico texto de Cormen y otros~%
     \cite{cormen09:_CLRS},
   es la organización del texto de Knuth~%
     \cite{knuth97:_fundam_algor,
           knuth97:_semin_algor,
           knuth98:_sortin_searc,
           knuth11:_combin_alg_1})
  o centrarse en técnicas de diseño y análisis,
  usando problemas en distintas áreas como ilustración
  (como lo hacen Sedgewick y Wayne~%
     \cite{sedgewick11:_algorithms},
   Skiena~%
     \cite{skiena08:_algor_desig_manual}
   y Erickson~%
     \cite{erickson19:_algorithms})
  otros autores combinan ambas formas de abordarlo
  (es el caso de Goodrich y Tamassia~%
     \cite{goodrich01:_algorithm_design}).
  El presente texto se concentra en técnicas de diseño de algoritmos.

  Cuidado,
  mucho del \textquote{análisis tradicional}
  (su indiscutible máximo exponente es Donald Knuth)
  supone que la memoria es rápida
  y acceso a ella es básicamente uniforme.
  La realidad actual
  (desde hace unos 40 años o así)
  es muy diferente,
  como demuestra Kamp~%
    \cite{kamp10:_doing_it_wrong}.
  Si se manejan enormes cantidades de datos,
  la organización de las estructuras de datos puede ser crítica.
  Asimismo,
  el análisis tradicionalmente se concentra en el caso de único procesador,
  máquinas actuales tienen varios de ellos y usarlos efectivamente
  es un tema importante.
  Tarjetas de video actuales son muchísimos procesadores,
  a ser usados en forma especializada
  (básicamente efectuar la misma operación en paralelo sobre datos distintos),
  aprovechar esas facilidades presenta desafíos interesantes.

  Para estimar el tiempo de ejecución de un algoritmo,
  podemos determinar cuántas veces se ejecuta cada instrucción
  (esto generalmente depende de la construcción exacta de los datos,
   con lo que resultan relevantes estadísticas de los datos de entrada,
   que lleva a estadísticas de los números de veces que se ejecuta),
  sabiendo cuánto se demora cada instrucción
  podemos determinar tiempos de ejecución.
  Pero esto significa partir esencialmente con el programa completo,
  ya sea en lenguaje de máquina
  (como ejemplifica Knuth~%
     \cite{knuth97:_fundam_algor, knuth97:_semin_algor,knuth98:_sortin_searc})
  o medir tiempos de ejecución promedio de operaciones básicas del lenguaje,
  como muestra Bentley~%
    \cite{bentley99:_programming_pearls}
  para C
  (pero hay que considerar que compiladores modernos
   efectúan extensas reorganizaciones del código,
   con lo que esto puede dar resultados engañosos).
  Técnicas similares pueden aplicarse para espacio de memoria.
  Nuevamente,
  Bentley~%
    \cite{bentley99:_programming_pearls}
  da código para medir el costo de estructuras en C.

  Estimaciones más sencillas se obtienen definiendo operaciones clave,
  tales que por cada operación clave hay un número acotado de otras operaciones
  (generalmente operaciones centrales o especialmente costosas).
  Contabilizar el número de operaciones clave es menos trabajo,
  muchas veces puede hacerse
  contando solo con una descripción general del algoritmo,
  no un programa detallado.
  Claramente esto nos dará solo cotas de tiempo asintóticas.

  Nos interesan algoritmos eficientes,
  específicamente en tiempo.
  Esto significa tener una medida del tiempo requerido
  para resolver el problema entre manos,
  típicamente medido en término de operaciones clave como descritas antes,
  independiente del algoritmo empleado.
  Por otro lado,
  tenemos algún algoritmo concreto y su análisis.
  En el caso ideal,
  los costos mínimos ideales y del algoritmo propuesto son iguales;
  generalmente serán diferentes,
  e interesa afinar el modelo teórico,
  hallar un algoritmo mejor,
  o ambos.
  Hay que tener cuidado,
  las cotas asintóticas valen cuando el tamaño del problema es muy grande,
  lo que puede significar tamaños fuera del alcance práctico,
  y esconden constantes de proporcionalidad,
  que pueden ser muy grandes.
  El análisis teórico debe refinarse y contrastarse con mediciones
  para situaciones de gran importancia.

  Para mostrar algo de los problemas y técnicas,
  ilustrando el tipo de resultados que buscamos,
  exploraremos un par de problemas muy sencillos:
  dado un arreglo no ordenado de \(n\) enteros diferentes,
  hallar los dos mayores
  y encontrar el menor y el mayor elemento.
  La discusión siguiente se adapta de~%
    \cite{OpenDSA16:_senior_algorithms}.

\section{Hallar el máximo de un arreglo}
\label{sec:max-array}

  Un problema simple es hallar el máximo elemento de un arreglo.
  Es claro que podemos obtener el máximo con \(n - 1\) comparaciones,
  y no puede hacerse mejor
  (con menos de \(n - 1\) comparaciones
   habrá un elemento que no se compara con ningún otro,
   no tenemos garantía de tener el mayor).
  Este razonamiento es correcto,
  pero informal.
  \begin{proposition}
    Para identificar el máximo de \(n\) elementos
    debemos efectuar al menos \(n - 1\) comparaciones.
  \end{proposition}
  \begin{proof}
    Cada vez que se comparan dos elementos,
    uno resulta perdedor
    (es menor que el otro).
    Como cada comparación produce (a lo más) un (nuevo) perdedor,
    y debemos identificar \(n - 1\) perdedores,
    debemos hacer al menos \(n - 1\) comparaciones.
  \end{proof}
  Una demostración alternativa,
  cuya maquinaria nos servirá
  para construir un buen algoritmo más adelante,
  usa el concepto de \emph{poset}
  (de \emph{\foreignlanguage{english}{partially ordered set}},
   conjunto parcialmente ordenado;
   un conjunto con un \emph{orden parcial},
   una relación reflexiva,
   antisimétrica y transitiva).
  En el orden parcial \(\le\) sobre \(\mathscr{U}\)
  dos elementos pueden ser comparables
  (es \(a \le b\) o \(b \le a\))
  o no.
  Por ejemplo,
  dado un conjunto \(\mathscr{A}\),
  la relación subconjunto define un orden parcial sobre \(2^{\mathscr{A}}\);
  no todos los subconjuntos de \(\mathscr{A}\) son comparables.
  El orden parcial puede describirse como un DAG:
  hay un arco de \(a\) a \(b\) si \(a \le b\),
  al ser una relación transitiva no pueden haber ciclos.
  Podemos considerar que nuestro algoritmo va agregando arcos
  según efectúa comparaciones
  (en el fondo,
   construimos el DAG de las relaciones conocidas entre elementos
   hasta explorar lo suficiente para determinar el máximo).
  \begin{proof}
    Iniciamos el algoritmo con \(n\) DAGs de un único vértice
    (no conocemos nada),
    y debemos construir un DAG que contiene todos los elementos
    (y nos permite identificar el máximo).
    El mínimo número de arcos para conectar los \(n\) vértices es \(n - 1\).
  \end{proof}
  El listado~\ref{lst:maximo-a} da el algoritmo obvio.
  \lstinputlisting[float,
                   language=C,
                   firstline = 6,
                   xleftmargin=3em, numbers=left,
                   caption={Hallar el máximo},
                   label=lst:maximo-a]
                   {code/maximum.c}
  Es obvio que efectúa \(n - 1\) comparaciones,
  y nuestro algoritmo es óptimo.

  Es claro que el número de veces
  que se asigna a~\lstinline[language = C]!max!
  es \(O(n)\);
  es claro que el máximo es \(n\)
  (si \lstinline[language = C]!a! viene ordenado de menor a mayor),
  el mínimo es \num{1}
  (si \lstinline[language = C]!a[0]! es el mayor),
  pero interesa el número promedio de veces.
  Calcular el promedio presenta otro conjunto de problemas:
  ¿Cómo vienen distribuidos los datos?
  Un modelo simple
  (¡que en casos importantes deberá verificarse que se cumple!)
  es suponer que los datos son todos distintos
  y que todos los órdenes de los datos de entrada
  son igualmente probables.
  Se asigna a~\lstinline[language = C]!max!
  en la iteración~\lstinline[language = C]!i!
  si y solo si~\lstinline[language = C]!a[i]!
  es el máximo de los primeros~\lstinline[language = C]!i + 1! elementos.
  Suponiendo que el orden es aleatorio,
  la probabilidad de esto es \(1 / i\).
  Sumando,
  el número promedio de asignaciones es:
  \begin{equation*}
    \sum_{0 \le i \le n - 1} \frac{1}{i + 1}
      = \sum_{1 \le i \le n} \frac{1}{i}
      = H_n
  \end{equation*}
  Sabemos que
  (vea por ejemplo el apéndice~\ref{apx:asymptotics}):
  \begin{equation*}
    \ln n \le H_n < \ln n + 1
  \end{equation*}
  Esto muestra también que nuestra intuición suele engañarnos:
  antes de ver este desarrollo
  creía que se asignaba aproximadamente la mitad de las veces.
  O sea,
  calculaba que si se buscaba el máximo de un millón de elementos,
  se harían unas 500~mil comparaciones.
  En promedio son en realidad \(H_{1\,000\,000} = 14,393\)~comparaciones..
  También muestra que puede resultar relevante
  usar diferentes medidas de comportamiento,
  en este caso número de comparaciones o de asignaciones.

\section{Dos mayores elementos}
\label{sec:dos-mayores}

  Habiendo ubicado el mayor,
  podemos identificar al segundo con \(n - 2\) comparaciones adicionales,
  para un total de \(2 n - 3\).
  Pero esto no considera la información
  que entregan las comparaciones para obtener el máximo.
  Note que los candidatos a segundo lugar
  son los que perdieron al comparar con el máximo,
  queremos el máximo de estos.
  Obtenemos el máximo de información
  al comparar dos elementos que sabemos mayores que el mismo número de valores.
  Esto lleva a comparar pares,
  comparar los máximos de los pares,
  y así sucesivamente.
  Consideremos \emph{árboles binomiales},
  el árbol binomial \(B_0\) es un único vértice,
  y \(B_k\) se construye uniendo dos árboles binomiales \(B_{k - 1}\),
  haciendo que la raíz de uno sea hijo de la raíz del otro
  (ver la figura~\ref{fig:binomial-trees-a}).
  El árbol \(B_k\) tiene \(2^k\) vértices,
  la raíz de \(B_k\) tiene \(k\)~hijos.
  \begin{figure}[ht]
    \centering
    \begin{tikzpicture}[every node/.style = {shape = circle, fill, draw},
                        scale = 0.75]
      \node (a0) at (0, 4) {};
      \node [draw = none, fill = none, below] at (0, -0.2) {\(B_0\)};

      \node (b0) at (1.5, 3) {};
      \node (b1) at (1.5, 4) {};
      \draw (b0) -- (b1);
      \node [draw = none, fill = none, below] at (1.5, -0.2) {\(B_1\)};

      \node (c0) at (3.0, 2) {};
      \node (c1) at (3.0, 3) {};
      \node (c2) at (4.0, 3) {};
      \node (c3) at (4.0, 4) {};
      \draw (c0) -- (c1)
            (c1) -- (c3)
            (c2) -- (c3);
      \node [draw = none, fill = none, below] at (3.5, -0.2) {\(B_2\)};

      \node (d0) at (5.5, 1) {};
      \node (d1) at (5.5, 2) {};
      \node (d2) at (6.5, 2) {};
      \node (d3) at (6.5, 3) {};
      \node (d4) at (7.5, 2) {};
      \node (d5) at (7.5, 3) {};
      \node (d6) at (8.5, 3) {};
      \node (d7) at (8.5, 4) {};
      \draw (d0) -- (d1)
            (d1) -- (d3)
            (d2) -- (d3)
            (d4) -- (d5)
            (d6) -- (d7)
            (d3) -- (d7)
            (d5) -- (d7);
      \node [draw = none, fill = none, below] at (7, -0.2) {\(B_3\)};

      \node (e0)  at (10, 0) {};
      \node (e1)  at (10, 1) {};
      \node (e2)  at (11, 1) {};
      \node (e3)  at (11, 2) {};
      \node (e4)  at (12, 1) {};
      \node (e5)  at (12, 2) {};
      \node (e6)  at (13, 2) {};
      \node (e7)  at (13, 3) {};
      \node (e8)  at (14, 1) {};
      \node (e9)  at (14, 2) {};
      \node (e10) at (15, 2) {};
      \node (e11) at (15, 3) {};
      \node (e12) at (16, 2) {};
      \node (e13) at (16, 3) {};
      \node (e14) at (17, 3) {};
      \node (e15) at (17, 4) {};
      \draw (e0) -- (e1)
            (e1) -- (e3)
            (e2) -- (e3)
            (e4) -- (e5)
            (e6) -- (e7)
            (e3) -- (e7)
            (e5) -- (e7)
            (e8) -- (e9)
            (e9) -- (e11)
            (e10) -- (e11)
            (e12) -- (e13)
            (e13) -- (e15)
            (e14) -- (e15)
            (e11) -- (e15)
            (e7) -- (e15);
      \node [draw = none, fill = none, below] at (13.5, -0.2) {\(B_4\)};
    \end{tikzpicture}
    \caption{Árboles binomiales}
    \label{fig:binomial-trees-a}
  \end{figure}
  Nuestra estrategia es entonces comparar pares,
  comparar los ganadores de los pares,
  y continuar comparando solo ganadores;
  estamos construyendo y uniendo
  árboles binomiales.
  El árbol final es parte de un árbol binomial,
  en determinar el máximo efectuamos \(n - 1\) comparaciones,
  si la raíz
  (el máximo)
  tiene \(k\)~descendientes,
  debemos obtener el máximo de estos para determinar el segundo,
  con un costo de \(k - 1\)~comparaciones adicionales.
  El total de comparaciones es \(n + k - 2\).
  Pero el árbol tiene entre \(2^{k - 1}\) y \(2^k\) vértices en total,
  con lo que \(k = \lceil \log_2 n \rceil\),
  con lo que el número de comparaciones es \(n + \lceil \log_2 n \rceil - 2\).

  Esto claramente es mucho mejor que la idea original
  (pero más complejo).
  ¿Es el mejor posible?
  Una técnica para demostrar que un algoritmo es óptimo
  es imaginar un adversario,
  que en cada paso responde lo peor posible
  (consistente con sus respuestas anteriores,
   claro está).
  Nuevamente consideramos el modelo de construir un DAG
  que nos entregue el máximo y el segundo mayor elemento,
  sin restringirnos ahora a un árbol binomial.
  Antes de ir a la demostración misma,
  observamos que los primeros \(n - 1\)~elementos
  tienen que haber perdido al menos una comparación,
  se requieren al menos \(n - 1\) comparaciones.
  Adicionalmente,
  al menos \(k - 1\)~elementos perdieron contra el segundo mayor
  (los \(k\) perdedores de las comparaciones con el máximo),
  haciendo un total de \(n + k - 2\) comparaciones.
  La pregunta entonces es,
  ¿qué tan pequeño podemos hacer \(k\)?
  Llamemos \emph{poder} de~\lstinline[language = C]!a[i]!
  al número de elementos que sabemos son menores a él.
  Si~\lstinline[language = C]!a[i]! tiene poder~\(a\)
  y~\lstinline[language = C]!a[j]! tiene poder~\(b\),
  el ganador de la comparación entre ellos tiene poder~\(a + b + 1\).
  Nuestro algoritmo conoce los poderes (actuales) de cada elemento,
  y debe decidir cuáles de ellos comparar a continuación.
  El adversario decide el resultado de la comparación,
  y busca entregar la mínima información con el.
  Esto corresponde a minimizar el aumento de poder de cualquier elemento,
  o sea siempre dar por ganador al elemento con mayor poder.
  Esto es construir el peor caso de datos al algoritmo,
  no es injusto.

  Queremos minimizar el efecto del peor caso,
  esto es maximizar el aumento mínimo de poder
  balanceando los poderes de los contendores.
  El mejor caso se da si el poder del ganador se duplica,
  vale decir,
  si el poder de ambos es el mismo.
  Como los poderes comienzan en cero,
  el ganador tiene que efectuar al menos \(k\)~comparaciones,
  donde \(2^{k - 1} < n \le 2^k\).
  O sea,
  \(k = \lceil \log_2 n \rceil\),
  nuestro algoritmo es óptimo.

\section{Máximo y mínimo}
\label{sec:maximo-minimo}

  Nuestro siguiente problema,
  hallar el máximo y el mínimo,
  es superficialmente similar al anterior,
  pero requiere métodos de análisis diferentes.
  Nuevamente consideremos solo el número de comparaciones.
  Una primera idea es recorrer el arreglo,
  recordando el máximo y el mínimo vistos hasta el momento,
  vea el listado~\ref{lst:max-min}.
  \lstinputlisting[float,
                   language=C,
                   firstline = 8,
                   xleftmargin=3em, numbers=left,
                   caption={Hallar el máximo y el mínimo},
                   label=lst:max-min]
                   {code/max-min.c}
  Es claro que esto es el doble de trabajo que solo obtener el máximo,
  listado~\ref{lst:maximo-a}.
  No estamos aprovechando bien el trabajo hecho al comparar,
  analicemos alternativas.

  Una opción es dividir el conjunto en dos,
  hallar (recursivamente) máximo y mínimo en ambas partes,
  y comparar los máximos y los mínimos de las partes.
  El listado~\ref{lst:max-min-d&c} da detalles.
  \lstinputlisting[float,
                   language=C,
                   firstline = 8,
                   xleftmargin=3em, numbers=left,
                   caption={Hallar el máximo y el mínimo, dividiendo en dos},
                   label=lst:max-min-d&c]
                   {code/max-min-dc.c}
  Si dividimos lo más equitativamente posible,
  el número de comparaciones para \(n\)~elementos,
  \(T(n)\),
  cumple:
  \begin{equation}
    \label{eq:max-min-recursion-1}
    T(n)
      = \begin{cases}
          0							& n = 1	  \\
          1							& n = 2	  \\
          T(\lfloor n / 2 \rfloor) + T(\lceil n / 2 \rceil) + 2 & n \ge 2
        \end{cases}
  \end{equation}
  Esta recurrencia tiene una solución complicada.
  Pero vemos directamente que si \(n = 2^k\) o \(n = 2^k \pm 1\),
  la solución es \(T(n) = 3 n / 2 - 2\);
  cuando \(n = 3 \cdot 2^k\) es \(T(n) = 5 n / 3 - 2\).

  En realidad,
  nada obliga a dividir aproximadamente en dos partes iguales,
  la recurrencia real debiera ser:
  \begin{equation}
    \label{eq:max-min-recursion-2}
    T(n)
      = \begin{cases}
          0							& n = 1	  \\
          1							& n = 2	  \\
          \min_{1 \le k < n} \{ T(k) + T(n - k) \} + 2		& n \ge 2
        \end{cases}
  \end{equation}
  Experimentando con esta nueva versión,
  vemos que siempre obtenemos el mínimo con \(k = 2\),
  lo que lleva a la recurrencia:
  \begin{equation}
    \label{eq:max-min-recursion}
    T(n)
      = \begin{cases}
          0			& n = 1	  \\
          1			& n = 2	  \\
          T(n - 2) + 3		& n \ge 2
        \end{cases}
  \end{equation}
  La solución es \(T(n) = \lceil 3 n / 2 \rceil - 2\)
  (resuelva pares e impares por separado,
   combine las soluciones).
  El código sugerido por esta recurrencia
  (traducido de recursión a iteración)
  es el dado por el listado~\ref{lst:max-min-opt}.
  \lstinputlisting[float,
                   language=C,
                   firstline = 8,
                   xleftmargin=3em, numbers=left,
                   caption={Hallar el máximo y el mínimo, versión mejorada},
                   label=lst:max-min-opt]
                   {code/max-min-opt.c}
  La pregunta obvia es si esto es óptimo.
  Para demostrarlo,
  agregamos otra herramienta a nuestro arsenal,
  usar un espacio de estados.

  En cada momento de la ejecución de cualquier algoritmo
  podemos representar lo que \textquote{sabe}
  registrando el número de \emph{ganadores} \(W\),
  elementos que han sido comparados con otros y nunca han perdido;
  \emph{perdedores} \(L\),
  elementos que han sido comparados con otros y nunca han ganado;
  elementos \emph{no comparados} \(U\);
  y elementos \emph{medios} \(M\),
  han sido comparados y han ganado y perdido.
  La situación inicial es \((U, W, L, M) = (n, 0, 0, 0)\),
  la situación final es \((0, 1, 1, n - 2)\).
  Con cuatro tipos de elementos,
  hay diez tipos de comparaciones posibles.
  Comparar \(M\) con otros no tiene sentido,
  puede cambiar a lo más la clasificación del otro elemento.
  Quedan seis comparaciones de interés,
  cuyos efectos sobre el estado \((U, W, L, M) = (i, j, k, l)\)
  resume el cuadro~\ref{tab:comparison-efect}.
  \begin{table}[ht]
    \centering
    \begin{tabular}{>{\(}c<{\)}|*{4}{>{\(}l<{\)}}}
      & \multicolumn{1}{c}{\boldmath\(U\)\unboldmath} &
        \multicolumn{1}{c}{\boldmath\(W\)\unboldmath} &
        \multicolumn{1}{c}{\boldmath\(L\)\unboldmath} &
        \multicolumn{1}{c}{\boldmath\(M\)\unboldmath} \\
      \hline
      U \colon U & i - 2 & j + 1 & k + 1 & l	 \\
      \hline
      W \colon W & i	 & j - 1 & k	 & l + 1 \\
      \hline
      L \colon L & i	 & j	 & k - 1 & l + 1 \\
      \hline
      \multirow{2}{*}{\(L \colon U\)}
                & i - 1 & j + 1 & k	 & l	 \\
                & i - 1 & j	 & k	 & l + 1 \\
      \hline
      \multirow{2}{*}{\(W \colon U\)}
                 & i - 1 & j	 & k + 1 & l	 \\
                 & i - 1 & j	 & k	 & l + 1 \\
      \hline
      \multirow{2}{*}{\(W \colon L\)}
                 & i	 & j	 & k	 & l	 \\
                 & i	 & j - 1 & k - 1 & l + 2 \\
      \hline
    \end{tabular}
    \caption{Comparaciones y sus efectos}
    \label{tab:comparison-efect}
  \end{table}
  Consideremos nuevamente un adversario,
  que dicta los resultados de las comparaciones
  de manera que el algoritmo trabaje lo más posible,
  vale decir,
  elige la alternativa que menos acerca al resultado deseado.
  Por ejemplo,
  comparar un ganador con un perdedor en el peor caso no da información nueva
  (el perdedor pierde y el ganador gana).
  Eliminado las alternativas que en el peor caso no aportan información nueva,
  quedan las del cuadro~\ref{tab:comparison-efect-select}.
  \begin{table}[ht]
    \centering
    \begin{tabular}{>{\(}c<{\)}|*{4}{>{\(}l<{\)}}}
      & \multicolumn{1}{c}{\boldmath\(U\)\unboldmath} &
        \multicolumn{1}{c}{\boldmath\(W\)\unboldmath} &
        \multicolumn{1}{c}{\boldmath\(L\)\unboldmath} &
        \multicolumn{1}{c}{\boldmath\(M\)\unboldmath} \\
      \hline
      U \colon U & i - 2 & j + 1 & k + 1 & l	 \\
      L \colon U & i - 1 & j + 1 & k	 & l	 \\
      W \colon U & i - 1 & j	 & k + 1 & l	 \\
      \hline
      W \colon W & i	 & j - 1 & k	 & l + 1 \\
      L \colon L & i	 & j	 & k - 1 & l + 1 \\
      \hline
    \end{tabular}
    \caption{Comparaciones interesantes y sus efectos}
    \label{tab:comparison-efect-select}
  \end{table}
  Solo las últimas dos transiciones
  del cuadro~\ref{tab:comparison-efect-select}
  aumentan el número de \(M\),
  debe haber un mínimo de \(n - 2\) de estas.
  El número de elementos sin comparar debe llegar a cero,
  y la primera transición es la más eficiente en esto,
  se requieren \(\lfloor n / 2 \rfloor\) de estas,
  junto a una comparación \(L \colon U\) o \(W \colon U\)
  si \(n\) es impar,
  para hacer cero \(U\),
  o sea \(\rceil n / 2 \lceil\)~comparaciones
  En total,
  se necesitan al menos \(n - 2 + \lceil n / 2 \rceil\)~comparaciones.
  Nuestro algoritmo es óptimo.

\section*{Ejercicios}
\label{sec:exercises-07-pre-previa}

  \begin{enumerate}
  \item
    Demuestre que \(k = 2\) siempre da un mínimo
    en la recurrencia~\eqref{eq:max-min-recursion-2}.
  \item
    Escriba un programa que halle el máximo y segundo elemento de un arreglo
    usando el algoritmo óptimo esbozado en el texto.
  \item
    Evalúe el número promedio de asignaciones de elementos en el programa
    del listado~\ref{lst:max-min}.
    Es claro que la segunda comparación fallará si tiene éxito la primera,
    evalúe el efecto de considerar esto.
  \item
    Considere el problema de hallar los tres mayores elementos.
    ¿Cuántas comparaciones requiere?
  \end{enumerate}

\bibliography{../referencias}

%%% Local Variables:
%%% mode: latex
%%% TeX-master: "../INF-221_notas"
%%% ispell-local-dictionary: "spanish"
%%% End:

% LocalWords:  poset english partially ordered set antisimétrica DAG
% LocalWords:  DAGs contendores eq max min recursion

\bibliographystyle{babplain-fl}

\chapter{Rendimiento de programas}
\label{cha:rendimiento}

  Requerimos estimar el uso de recursos de los programas que escribimos.
  La forma clásica de obtener el tiempo de ejecución,
  cuyo pionero es Knuth~%
    \cite{knuth97:_fundam_algor,
          knuth97:_semin_algor,
          knuth98:_sortin_searc},
  es contar en detalle las operaciones que ejecuta un programa.
  El problema es que esto es mucho trabajo,
  y el resultado depende del programa exacto
  y de los detalles del ambiente de ejecución.
  Para programas muy importantes,
  esto tiene sentido.
  Modificar programas para hacerlos más eficientes
  es el tema del libro de Bentley~%
    \cite{bentley82:_writing_efficient_programs},
  que lamentablemente hace tiempo no se imprime.
  Una de las conclusiones centrales es que elegir la arquitectura del sistema
  y los algoritmos con cuidado es vital.

  En general,
  bastan estimaciones bastante burdas
  para elegir entre algoritmos alternativos.
  Suele ser suficiente elegir alguna \emph{operación clave},
  tal que signifique el mayor costo del algoritmo,
  y contabilizar éstas como medida de rendimiento.
  Por ejemplo,
  al analizar ejemplos de algoritmos discretos
  en el capítulo~\ref{cha:discrete-algorithms}
  nos fijamos únicamente en el número de comparaciones efectuadas.
  La justificación es que esa operación es mucho más costosa que las demás,
  o que las demás operaciones
  tienen un costo aproximadamente proporcional
  al de las operaciones contabilizadas.
  Obtenemos entonces el costo del algoritmo en forma de \(\Theta(f(n))\),
  ojalá para una función \(f\) simple.
  Una ventaja de este proceder
  es que no requiere detalles del programa ni de su ambiente,
  en muchos casos es posible obtener estimaciones útiles
  de descripciones de alto nivel del algoritmo.

\section{Algunas reglas}
\label{sec:rules-performance}

  Un análisis más detallado del tema
  se halla en el apéndice~\ref{apx:rendimiento-programas}.
  Acá nos remitiremos a dar algunas reglas básicas.
  En ocasiones solo podremos dar cotas asintóticas,
  en cuyo caso en las reglas deberán considerarse las manipulaciones del caso,
  como detalladas en el apéndice~\ref{apx:asymptotics}.
  \begin{description}
  \item[Secuencia de acciones:]
    El costo de la secuencia
    es simplemente la suma de los costos de cada acción.
  \item[Alternancia:]
    En caso de varias opciones,
    habrá que analizar las alternativas por separado.
    Finalmente,
    habrá que determinar cuántas veces se efectúa cada alternativa.
    Si esto no resulta practicable,
    podemos dar una cota superior mediante el costo máximo.
  \item[Ciclos:]
    Se debe evaluar el costo de cada iteración,
    y sumar.
    El caso más simple es el de iteraciones definidas,
    en cuyo caso el costo generalmente
    será el número de iteraciones por el costo de su cuerpo.
    En iteraciones indefinidas deberá estimarse el número de iteraciones,
    o al menos acotarlo.
  \item[Llamadas a procedimientos:]
    Se considera el costo del cuerpo respectivo.
    Generalmente este costo dependerá de algún argumento
    (por ejemplo,
     el costo de ordenar un arreglo dependerá de su número de elementos).
    Claramente llamadas recursivas naturalmente llevarán a recurrencias,
    discutiremos algunos casos comunes más adelante
    (capítulo~\ref{cha:dividir-conquistar}
     y apéndice~\ref{apx:recurrencias}).
  \end{description}

\section{Máximo común divisor}
\label{sec:GCD}

  El algoritmo clásico para obtener el máximo común divisor
  es el de Euclides,
  algoritmo~\ref{alg:Euclides}.
  El algoritmo de Euclides de interés histórico también,
  es el algoritmo más antiguo que involucra ciclos,
  y fue el primer algoritmo cuyo rendimiento se analizó matemáticamente
  (por Gabriel Lamé~%
    \cite{lame44:_gcd}
   en 1844).
  \begin{algorithm}[htbp]
    \DontPrintSemicolon\Indp

    \Function{\(\mathrm{gcd}(a, b)\)}{
      \While{\(b > 0\)}{
        \((a, \; b) \gets (b, \; a \bmod b)\) \;
      }
      \Return \(a\) \;
    }
    \caption{Algoritmo de Euclides para calcular \(\gcd(a, b)\)}
    \label{alg:Euclides}
  \end{algorithm}
  Este algoritmo es aplicable a números enteros,
  pero también por ejemplo a polinomios.
  Analizaremos ambos casos.

\subsection{Algoritmo de Euclides en enteros}
\label{sec:GCD-integer}

  Tomando como medida de costo el número de veces que se calcula un residuo
  (probablemente sea la operación más costosa;
   si se efectúan cálculos a mano o con números muy grandes,
   habrá que considerar el número de dígitos en el costo),
  vemos que cada iteración tiene costo uno.
  Falta estimar el número de iteraciones.
  El cálculo preciso es fascinante,
  ver por ejemplo Knuth~%
    \cite[sección~4.5.3]{knuth97:_semin_algor}.
  Acá nos conformaremos con una cota superior,
  suponiendo que las operaciones tienen costo fijo.

  Sean \(r_i\) los restos en cada paso del algoritmo,
  con el entendido que \(r_0 = a\) y \(r_1 = b\),
  y que \(a > b\)
  (en caso contrario,
   lo único que hace la primera iteración es intercambiarlos).
  Estamos calculando:
  \begin{equation*}
    r_{i + 2} = r_i \bmod r_{i + 1}
  \end{equation*}
  O sea,
  para una secuencia de \(q_i\)
  tenemos las relaciones:
  \begin{equation*}
    r_{i + 2} = r_i - q_i r_{i + 1}
  \end{equation*}
  donde \(r_k \ne 0\) y \(r_{k + 1} = 0\)
  si hay \(k\) iteraciones.
  El peor caso se da cuando \(q_i = 1\) siempre,
  ya que en tal caso
  los \(r_i\) disminuyen lo más lentamente posible.
  Además,
  el caso en que \(\gcd(a, b) = 1\)
  es el en el cual más terreno se debe recorrer.
  Podemos dar vuelta esto,
  y preguntarnos qué tan lejos del final estamos,
  y calcular desde allí:
  \begin{equation*}
    r_{i + 2}
      = r_{i + 1} + r_i
      \qquad r_0 = 0, r_1 = 1
  \end{equation*}
  Reconocemos la recurrencia de los números de Fibonacci,
  \(F_n\):
  \begin{equation}
    \label{eq:Fibonacci}
    F_{k + 2} = F_{k + 1} + F_k \qquad F_0 = 0, F_1 = 1
  \end{equation}
  La secuencia comienza:
  \begin{equation*}
    0, 1, 1, 2, 3, 5, 8, 13, 21, 34, 55, \dotsc
  \end{equation*}
  El resultado es entonces
  que si el algoritmo efectúa \(k\) iteraciones
  entonces \(b \ge F_k\).
  Sabemos que:
  \begin{align*}
    F_n
      &=       \frac{(1 + \sqrt{5})^n - (1 - \sqrt{5})^n}{2^n \sqrt{5}} \\
      &\approx \frac{\tau^n}{\sqrt{5}}
  \end{align*}
  donde:
  \begin{equation*}
    \tau
      = \frac{1 + \sqrt{5}}{2}
  \end{equation*}
  Concluimos que el número de iteraciones
  está acotado por \(\log_\tau b\),
  que es aproximadamente cinco veces el número de dígitos decimales de \(b\).

\subsection{Algoritmo de Euclides en polinomios}
\label{sec:gcd-polynomial}

  Este caso es más simple por un lado
  (al calcular \(r = a \bmod b\) obtenemos que \(\deg(r) < \deg(b)\);
   como no hay una cota mejor,
   el número de iteraciones está acotado simplemente por \(\deg(b)\)),
  pero más complejo por el otro
  (es natural considerar como operaciones elementales
   las operaciones entre coeficientes de los polinomios,
   el costo de una iteración ya no es fijo).

  Suponiendo \(\deg(a) = \deg(b) + 1\)
  (en general llegaremos a este caso a la primera iteración),
  calcular \(r = a \bmod b\)
  es restar un múltiplo constante de \(x b\) de \(a\),
  esto corresponde a \(2 \deg(b) + 1\) operaciones
  (considera el cálculo de la constante,
   multiplicar cada coeficiente de \(b\) por la constante
   y la resta).
  Vale decir,
  el costo será:
  \begin{equation*}
    \sum_{1 \le k \le \deg(b)} (2 k + 1)
      = \deg^2(b) + 2 \deg(b)
  \end{equation*}

\section{Resolver sistemas lineales}
\label{sec:Gauss-elimination}

  Consideremos el problema de resolver un sistema lineal de \(n\)~ecuaciones
  con \(n\)~incógnitas.
  No tomamos en cuenta el efecto de errores de redondeo,
  son un tema delicado en esta aplicación.
  Una forma de hacerlo es despejar una incógnita de una de las ecuaciones,
  y usarla para eliminar esa incógnita de las demás ecuaciones.
  Esto nos deja con un sistema de \(n - 1\)~ecuaciones
  en \(n - 1\)~incógnitas.
  Aplicando esto repetidas veces queda una ecuación en una incógnita,
  que podemos resolver directamente.
  Con el valor de esta incógnita,
  de la ecuación en dos incógnitas obtenemos una segunda incógnita,
  y así sucesivamente hasta completar el trabajo.

  Si expresamos el sistema como:
  \begin{equation*}
    \sum_{1 \le k \le n} a_{i k} x_k
      = b_i
  \end{equation*}
  Resulta cómodo trabajar con la \emph{matriz extendida},
  la matriz de coeficientes a la que se adosa el vector \(\mathbf{b}\)
  como una columna adicional.
  Suponiendo que usamos la ecuación número~\num{1}
  para eliminar \(x_1\) de las demás ecuaciones,
  esto se expresa para la ecuación~\(j\) como:
  \begin{equation*}
    a'_{i j}
      = a_{i j} - \frac{a_{i 1}}{a_1 1} \cdot a_{1 j}
  \end{equation*}
  Estas son \(3 n (n - 1)\)~operaciones
  (una división, una multiplicación y una resta
   para cada uno de \(n\)~coeficientes de una ecuación,
   son \(n - 1\)~ecuaciones).
  Falta el costo de despejar la incógnita \(x_j\),
  para la cual conocemos las demás:
  \begin{equation*}
    x_j
      = b_j - \sum_{j + 1 \le k \le n} a_j k x_k
  \end{equation*}
  Esto cuesta \(n - (j + 1) + 1 + 1 = n - j + 1\)~operaciones.
  Sumando todas las contribuciones,
  el costo de resolver el sistema de ecuaciones
  en número de operaciones entre coeficientes es:
  \begin{align*}
    \sum_{1 \le k \le n} 3 k (k - 1)
      + \sum_{1 \le k \le n} (n - k + 1)
      &=    n (n^2 - n + 3) \\
      &\sim n^3
  \end{align*}

\section*{Ejercicios}
\label{sec:ejercicios-07-post-previa}

  \begin{enumerate}
  \item
    Plantee un algoritmo similar al de eliminación gaussiana
    para invertir matrices
    (esto corresponde aproximadamente a resolver un sistema de~\(n \times n\)
     para cada una de las \(n\) columnas).
    Evalúe el costo de resolver
    el sistema \(\mathbf{A} \cdot \mathbf{x} = \mathbf{b}\) de \(n \times n\)
    mediante la ecuación \(\mathbf{x} = \mathbf{A}^{-1} \cdot \mathbf{b}\).
  \item
    La técnica descrita como eliminación gaussiana no es satisfactoria,
    no considera el caso que algún coeficiente sea cero.
    Más en general,
    al calcular \(a'_{i j}\) conviene que lo que se resta sea lo menor posible
    en valor absoluto
    (el error absoluto del redondeo de las restas es básicamente fijo,
     convienen resultados grandes para disminuir el error relativo).
    Esto sugiere elegir el máximo coeficiente de la columna
    como ecuación a utilizar.
    Pero nada obliga a considerar las incógnitas en orden,
    conviene usar el coeficiente máximo en valor absoluto
    para seleccionar el \emph{pivote}.
    Analice el número de operaciones entre coeficientes
    de este algoritmo
    (\emph{eliminación de Gauß con pivote completo}).
    Considere que comparar dos coeficientes en valor absoluto
    tiene el mismo costo que las otras operaciones entre coeficientes.
  \end{enumerate}

\bibliography{../referencias}

%%% Local Variables:
%%% mode: latex
%%% TeX-master: "../INF-221_notas"
%%% ispell-local-dictionary: "spanish"
%%% End:

% LocalWords:  gaussiana

\bibliographystyle{babplain-fl}

\chapter{Argumentos de adversario}
\label{cha:adversary}

  Una herramienta poderosa para demostrar cotas inferiores
  al rendimiento de algoritmos
  es imaginar un adversario malévolo que genera los datos pedidos
  de forma de forzar que trabaje el máximo posible
  (consistente con sus respuestas anteriores,
   claro está).
  El material presente se adapta de las notas de Jeff Erickson~%
    \cite{erickson18:_adversary_arguments}
  y Goldman y Goldman~%
    \cite{goldman07:_adversary_lower_bound}

\section{Subpalabras en secuencias binarias}
\label{sec:subwords}

  Consideremos primeramente el problema de determinar
  si una palabra binaria de largo~\(n\)
  contiene algún \(1\).
  Intuitivamente,
  es necesario revisar \(n\)~bits,
  nuestro objetivo es demostrar esta cota inferior.
  Imaginamos un adversario que retorna los bits solicitados por el algoritmo,
  retornando siempre \(0\) y registrando las respuestas dadas.
  Después de una secuencia de consultas,
  podrán haber varias palabras consistentes con ellas.
  Si hay posiciones que no se han consultado,
  el algoritmo no tiene cómo distinguir palabras
  con \(0\) y \(1\) en esas posiciones,
  no puede responder correctamente en todos los casos
  si no consulta todas las posiciones.

  Diremos que una propiedad de una estructura es \emph{elusiva}
  si para determinar si la estructura cumple es necesario revisarla completa.
  Hemos demostrado que la propiedad \textquote{la palabra contiene un \(1\)}
  es elusiva.

  La siguiente pregunta es si la palabra contiene \(01\).
  Nuevamente,
  podemos responder si contiene \(01\) revisando todos los bits.
  La pregunta es si es posible hacerlo revisando menos.

  Si el largo \(n\) de la palabra es impar,
  podemos revisar sus posiciones pares,
  \(B[2], B[4], \dotsc, B[n - 1]\).
  Si para algún \(i < j\) vemos \(B[i] = 0\) y \(B[j] = 1\),
  podemos parar:
  sabemos que hay \(01\) entremedio.
  Si hay solo \(1\) seguidos por \(0\),
  no tiene sentido revisar el bit entre el último \(1\) y el primer \(0\),
  los demás son posibles candidatos a completar \(01\).
  Si hay solo \(0\),
  debemos revisar si hay \(1\) en las posiciones \(3, 5, \dotsc, n\)
  (un \(1\) en la primera posición iría seguido por \(0\));
  si hay solo \(1\),
  debemos revisar si hay \(0\) las posiciones \(1, 3, \dotsc, n - 2\)
  (un \(0\) en la última posición seguiría un \(1\)).
  O sea,
  basta revisar \(n - 1\) bits.

  Si la palabra es de largo par,
  palabras sin \(01\) están formadas por \(1\) seguidos de \(0\).
  una estrategia de adversario demuestra que debemos revisar todos los bits.
  Crearemos un adversario que mantiene la incógnita de \(01\)
  consistente con las consultas,
  en una palabra iniciada con \(1\) y terminada en \(0\),
  con el rango entre \(r + 1\) y \(s - 1\) indeciso.
  El adversario mantiene índices \(r\) y \(s\)
  para el último \(1\) y el primer \(0\) revelado,
  respectivamente.
  Inicialmente \(r = 0\) y \(s = n + 1\)
  (no se ha revelado nada).
  El adversario mantiene el invariante que \(s - r\) es par,
  forzando al algoritmo
  a solicitar todos los bit para verificar que no hay un par \(01\):
  Si no revisa algún bit a la izquierda de \(r\) o
  a la derecha de \(s\),
  el adversario tiene libertad de darle el valor \(0\) o \(1\).
  Antes de revisar todos los bits
  hay al menos dos bits sin revisar entre \(r\) y \(s\)
  (porque \(s - r\) se mantiene par),
  que el adversario puede reemplazar por \(01\).
  El algoritmo~\ref{alg:hide01} describe la lógica del adversario.
  \begin{algorithm}
    \DontPrintSemicolon

    \(r \gets 0; s \leftarrow n + 1\) \;
    \Function{\(\mathrm{Hide01}(i)\)}{
      \uIf{\(i \le r\)}{
        \(B[i] \gets 1\) \;
      }
      \uElseIf{\(i \ge s\)}{
        \(B[i] \gets 0\) \;
      }
      \uElseIf{\(i - r\) is even}{
        \(B[i] \gets 0\) \;
        \(s \gets i\) \;
      }
      \Else{
        \(B[i] \gets 1\) \;
        \(r \gets i\) \;
      }
      \Return \(B[i]\) \;
    }
    \caption{El algoritmo \(\mathrm{Hide01}\) del adversario}
    \label{alg:hide01}
  \end{algorithm}

  O sea,
  el patrón \(01\)
  (y simétricamente \(10\))
  es elusivo solo si el largo de la palabra es par.
  Resulta que los únicos patrones elusivos para todos los largos
  son los patrones de un símbolo \(0\) y \(1\).

\section{Ordenar por comparaciones}
\label{sec:comparison-sort-bound}

  Suponga que nos dan \(n\)~elementos distintos
  (por ejemplo en forma de un arreglo \(\mathbf{a}\)),
  y piden ordenarlos de menor a mayor
  usando únicamente comparaciones entre elementos.
  Primero notamos que hay \(n!\)~permutaciones de \(n\)~elementos
  (y por tanto posibles soluciones).
  El adversario mantiene el conjunto \(L\)
  de las permutaciones de los elementos
  compatibles con las respuestas dadas hasta el momento,
  inicialmente \(L\) son las \(n!\) permutaciones posibles.
  La estrategia del adversario
  al responder a la pregunta \textquote{es \(a[i] < a[j]\)?} es como sigue:
  sea \(L_{\text{Si}}\)
  el subconjunto de permutaciones consistentes con la respuesta \textquote{Sí},
  sea \(L_{\text{No}}\)
  el subconjunto de permutaciones consistentes con la respuesta \textquote{No}
  (es claro que siempre \(L = L_{\text{Si}} \cup L_{\text{No}}\)).
  Para maximizar el número de comparaciones,
  el adversario responde \textquote{Sí})
  si \(\lvert L_{\text{Si}} \rvert > \lvert L_{\text{No}} \rvert\).
  Vale decir,
  cada comparación reduce el número de candidatos a lo más a la mitad,
  con lo que el número de comparaciones necesario cumple:
  \begin{equation*}
    C(n)
      \ge \lceil \log_2 n! \rceil
  \end{equation*}
  Para estimar el logaritmo:
  \begin{align*}
    \ln n!
      &=   \sum_{1 \le k \le n} \ln k \\
      &\ge \sum_{n / 2 \le k \le n} \ln k \\
      &\ge \sum_{n / 2 \le k \le n} \ln \frac{n}{2} \\
      &= \frac{n}{2} \ln \frac{n}{2} \\
      &= \frac{n}{2} \ln n - \frac{n}{2} \ln 2 \\
    \ln n!
      &= \Omega(n \log n)
  \end{align*}
  Incidentalmente,
  una cota superior es igual de simple de obtener:
  \begin{align*}
    \ln n!
      &= \sum_{1 \le k \le n} \ln k \\
      &\le \sum_{1 \le k \le n} \ln n \\
      &= n \ln n \\
    \ln n!
      &= O(n \log n)
  \end{align*}
  con lo que resulta:
  \begin{equation*}
    \ln n!
      = \Theta(n \log n)
  \end{equation*}
  Concluimos que al ordenar \(n\) elementos
  se requieren \(\Omega(n \log n)\) comparaciones.

  Note que en este razonamiento no restringe en absoluto
  las comparaciones hechas por el algoritmo de ordenamiento.
  Tampoco especificamos
  cómo el adversario construye los conjuntos de permutaciones candidatos.
  En la forma descrita,
  es claro que el adversario
  debe efectuar mucho trabajo para cada comparación,
  pero esto tampoco es relevante.
  Para familias de métodos de ordenamiento específicos
  es posible construir adversarios eficientes,
  ver por ejemplo el de McIllroy para Quicksort~%
    \cite{mcillroy99:_killer_adver_quicksort}.

\section{Propiedades de grafos}
\label{sec:graph-properties}

  Estrategias de adversario proveen cotas ajustadas
  para algunas propiedades de grafos,
  cuando el grafo se representa por una matriz de adyacencia.
  Acá llamaremos elusiva una propiedad
  si un algoritmo
  debe revisar todas las \(\binom{n}{2}\) entradas de la matriz de adyacencia
  para decidirla.

  Un ejemplo obvio es preguntar si el grafo es vacío
  (no tiene arcos).
  Dado el conjunto de vértices \(V\),
  el adversario mantiene el grafo \(G\)
  (esencialmente lo no revisado).
  Inicialmente \(G\) es el grafo completo sobre \(V\).
  Cada vez que el algoritmo consulta si \(u v\) es un arco del grafo,
  el adversario responde \textquote{No} y elimina \(u v\) de \(G\).
  Si el algoritmo no consulta por algún posible arco,
  el grafo vacío \(E = (V, \varnothing)\)
  y el grafo no vacío \(G\) que mantiene el adversario
  son ambos consistentes con las respuestas dadas,
  el algoritmo no puede responder correctamente.

  Un ejemplo más complejo es conectividad.
  Acá el adversario mantiene dos grafos,
  \(S\) y \(T\)
  (por \textquote{Sí} y \textquote{Tal vez}).
  El grafo \(S\) contiene los arcos que el algoritmo sabe pertenecen al grafo,
  \(T\) además contiene los arcos aún no considerados.
  Inicialmente \(S = (V, \varnothing)\) es el grafo vacío
  y \(T\) es el grafo completo sobre \(V\).
  Supondremos que el algoritmo no hace consultas redundantes,
  si consulta sobre el arco \(e\) es que no ha consultado sobre él antes
  (o sea,
   \(e\) pertenece a \(T \smallsetminus S\)).
  El algoritmo del adversario es~\ref{alg:HideConnected}.
  \begin{algorithm}
    \DontPrintSemicolon

    \(S \gets (V, \varnothing); T \leftarrow (V, V \times V)\) \;
    \Function{\(\mathrm{HideConnected}(e)\)}{
      \If{\(T \smallsetminus \{e\}\) is connected}{
        \(T \gets T \smallsetminus \{ e \}\) \;
        \Return \(\mathrm{False}\) \;
      }
      {
        \(S \gets S \cup \{ e \}\) \;
        \Return \(\mathrm{True}\) \;
      }
    }
    \caption{El algoritmo \(\mathrm{HideConnected}\)}
    \label{alg:HideConnected}
  \end{algorithm}
  Mantiene algunos invariantes:
  \begin{description}
  \item[\boldmath\(S\) es subgrafo de \(T\)\unboldmath:]
    Esto es obvio.
  \item[\boldmath\(T\) es conexo\unboldmath:]
    También es obvio.
  \item[\boldmath Si \(T\) tiene un ciclo,
        ninguno de sus arcos está en \(S\)\unboldmath:]
    Si el algoritmo hubiese consultado por uno de los arcos del ciclo,
    el adversario lo habría eliminado de \(T\) y no lo agregaría a \(S\).
  \item[\boldmath\(S\) es acíclico\unboldmath:]
    Esto es inmediato del invariante anterior.
  \item[\boldmath Si \(S \ne T\), \(S\) no es conexo\unboldmath:]
    Sabemos que los grafos conexos acíclicos son árboles.
    Supongamos \(S\) es un árbol y \(e\) es un arco en \(T\) pero no en \(S\).
    Entonces \(S \cup \{ e \}\) tiene un ciclo,
    que está también en \(T\).
    Esto viola el tercer invariante.
  \end{description}
  Si el algoritmo no revisa todos los arcos,
  \(S \ne T\),
  ambos consistentes con las respuestas dadas,
  pero uno es conexo y el otro no.
  La conectividad es elusiva.

\section{Máximo y mínimo}
\label{sec:max-min}

  Nuestro problema ahora es,
  dada una colección \(X\) de \(n\) elementos,
  halle el mayor y el menor de todos ellos
  usando comparaciones entre elementos.

  Una cota superior es fácil:
  con \(n - 1\) comparaciones hallamos el máximo,
  con \(n - 2\) comparaciones entre los demás identificamos el mínimo,
  para un total de \(2 n - 3\) comparaciones.

  Podemos mejorar la cota superior mediante el siguiente algoritmo.
  Divida el conjunto en pares \(x_{2 k - 1}, x_{2 k}\)
  (si \(n\) es impar,
   aparte el ultimo elemento),
  compare los elementos de cada par
  dejando los menores en el conjunto \(L\) y los mayores en el conjunto \(H\).
  Si \(n\) es impar,
  deposite \(x_n\) en ambos conjuntos.
  Resulta \(\lvert L \rvert = \lvert H \rvert = \lceil n / 2 \rceil\).
  Esto consume \(\lfloor n / 2 \rfloor\) comparaciones.
  Podemos obtener el mínimo de \(L\)
  (el mínimo de \(X\))
  con \(\lceil n / 2 \rceil - 1\) comparaciones,
  asimismo e máximo de \(H\)
  (el máximo de \(X\))
  con \(\lceil n / 2 \rceil - 1\) comparaciones.
  El total de comparaciones es:
  \begin{equation*}
    \left\lfloor \frac{n}{2} \right\rfloor
      + \left\lceil \frac{n}{2} \right\rceil - 1
      + \left\lceil \frac{n}{2} \right\rceil - 1
      = \left\lceil \frac{3 n}{2} \right\rceil - 2
  \end{equation*}

  Para una cota inferior,
  usamos un argumento de adversario.
  El adversario mantiene marcas en los elementos,
  \(+\) si puede ser el máximo y \(-\) si puede ser el mínimo.
  Inicialmente todos los elementos llevan marcas \(+\) y \(-\),
  un total de \(2 n\) marcas.
  Al final debe quedar exactamente un elemento marcado \(+\)
  y uno marcado \(-\)
  (el adversario puede declarar cualquier elemento marcado \(+\) como máximo
   y cualquier elemento marcado \(-\) como mínimo).
  Al comparar dos elementos con ambas marcas
  (marca \(\pm\)),
  el adversario declara uno mayor y el otro menor,
  eliminando las marcas \(-\) y \(+\) de los dos elementos,
  respectivamente.
  En todos los otros casos
  el adversario puede responder
  de manera de eliminar a lo más una de las marcas.
  Por ejemplo,
  al comparar un elemento marcado \(\pm\) con uno marcado \(-\),
  declara que el segundo es menor y deja las marcas \(+\)
  (el marcado \(\pm\) ya no es el mínimo)
  y \(-\)
  (sigue pudiendo ser el menor).
  El caso de marcas \(\pm\) y \(+\) es similar.
  Comparar dos elementos ambos marcados \(+\) o \(-\)
  simplemente elimina la marca de uno de ellos y declara al otro mayor
  (respectivamente menor),
  comparar uno marcado \(+\) con uno marcado \(-\) no requiere cambios
  (el primero es mayor).
  Ahora bien,
  un algoritmo correcto debe remover \(2 n - 2\) marcas.
  A lo más \(\lfloor n / 2 \rfloor\) comparaciones eliminan dos marcas,
  otras comparaciones eliminan a lo más una.
  El número de comparaciones requeridas es a lo menos:
  \begin{equation*}
    2 n - 2 - 2 \left\lfloor \frac{n}{2} \right\rfloor
      = \left\lceil \frac{3 n}{2} \right\rceil - 2
  \end{equation*}

  Como tenemos una cota superior igual a la inferior,
  este es el número exacto de comparaciones requerido
  (y el algoritmo esbozado es óptimo).

\section*{Ejercicios}
\label{sec:ex-adversary}

  \begin{enumerate}
  \item
    Demuestre que el patrón \(1 1\) es elusivo
    solo si el largo de la palabra es un múltiplo de \num{3}.
  \item
    Demuestre que hallar el máximo de una colección de \(n\) elementos
    requiere \(n - 1\) comparaciones
    usando un argumento de adversario.
  \item
    Un \emph{escorpión} es un grafo con un vértice de grado \num{1}
    (el aguijón),
    conectado únicamente a un vértice de grado \num{2}
    (la cola),
    la cola está conectada solo al aguijón y a un vértice adicional
    (el cuerpo).
    Los demás vértices del grafo
    (las patas,
     antenas,
     alas,
     ojos,
     ruedas,
     \ldots)
    están conectadas al cuerpo
    y arbitrariamente entre sí.
    ¿Es elusiva la propiedad de ser escorpión?
  \item
    Demuestre que la propiedad de ser acíclico es elusiva
    si el grafo se representa por una matriz de adyacencia.
  \end{enumerate}
\bibliography{../referencias}

%%% Local Variables:
%%% mode: latex
%%% TeX-master: "../INF-221_notas"
%%% ispell-local-dictionary: "spanish"
%%% End:

% LocalWords:  Jeff Goldman Subpalabras is even connected subgrafo

\bibliographystyle{babplain-fl}

\chapter{Recursión}
\label{cha:recursion}

% To do: Compare with Smid

  Nuestra herramienta principal
  es \emph{reducir} problemas a problemas más \textquote{simples}.
  Por ejemplo:
  de una expresión regular obtener
  un programa eficiente para reconocer un patrón.
  Para resolver este problema usamos:
  \begin{itemize}
  \item
    Expresión Regular a NFA (por ejemplo Thompson)
  \item
    NFA a DFA (algoritmo de subconjuntos)
  \item
    DFA a DFA mínimo (hay varias opciones)
  \item Interpretar el DFA o traducirlo a código.
  \end{itemize}
  Es la estrategia que usamos en Informática Teórica
  (INF-155),
  pero allí lo hicimos para demostrar que hay problemas difíciles.

  Al resolver un problema,
  lo dividimos en subproblemas diversos y combinamos resultados.
  Notar que al hacer esto
  (por ejemplo,
   invocar una rutina de biblioteca,
   como \texttt{printf(3)} en C)
  confiamos en que la solución al subproblema
  hace su trabajo correctamente.
  \emph{Confiamos} en terceros.
  De la misma manera,
  al invocar una función que nosotros escribimos,
  \emph{confiamos} en que hace su trabajo correctamente.
  La consideramos una caja negra,
  cuyo funcionamiento no nos interesa,
  en realidad ni siquiera nos incumbe.
  Podemos incluso cambiarla por una versión más eficiente,
  sin cambiar el programa.
  Desde el otro ángulo,
  quien escribe la solución al subproblema no considera dónde se usará,
  se concentra en cumplir con las especificaciones de su tarea.

  Usar recursión es lo mismo,
  solo que la función se invoca a sí misma
  (directa o indirectamente).

  Recursión es inducción en forma de programa.
  Igual que en inducción requerimos casos base y paso de inducción.

  La belleza de la recursión es que nos debemos preocupar
  solo del problema entre manos,
  subproblemas se resuelven automáticamente.
  Una extensa discusión,
  introduciendo el tema gradualmente,
  ofrece Roberts~%
    \cite{roberts86:_thinking_recursively}.
  Lúcida es la exposición de Hetland~%
    \cite[capítulo~4]{hetland14:_python_algorithms};
  otra opción es Smid~%
    \cite[capítulo~4]{smid19:_discr_struct_comput_scien}
  quien discute varios ejemplos de definiciones por inducción
  que llevan a programas recursivos nada obvios.

  Para escribir un programa recursivo hay varios pasos clave
  a considerar:
  \begin{itemize}
  \item
    Describir la solución al problema como la combinación
    de resultados de problemas similares,
    con datos distintos.
  \item
    Determinar situaciones en las cuales la solución es tan simple
    que puede resolverse directamente.
  \end{itemize}
  Para verificar que el programa resultante es correcto,
  debemos asegurarnos que la descomposición es correcta,
  que las soluciones de los casos base
  (no recursivos)
  son correctos
  y que cada vez las llamadas son a \textquote{casos más cercanos}
  a los casos base.
  Note que solo nos preocupamos de detallar el primer paso de la solución,
  los demás pasos son automáticos.
  Es lo que Erickson~%
    \cite{erickson19:_algorithms}
  llama \emph{\foreignlanguage{english}{recursion fairy}},
  el hada de la recursión,
  haciéndose cargo.
  Esto hace atractiva la recursión como técnica de programación,
  es frecuente que dé soluciones simples,
  transparentes.

\section{Factoriales}
\label{sec:factoriales}

  La definición de factoriales es recursiva:
  \begin{equation*}
    n!
      = \begin{cases}
          1		    & n = 0 \\
          n \cdot (n - 1)!  & n > 0
        \end{cases}
  \end{equation*}
  De acá es obvia la versión recursiva
  del listado~\ref{lst:factorial-recursive}.
  \lstinputlisting[float, floatplacement = h,
                   language = C,
                   firstline = 5, lastline = 11,
                   caption = {Factoriales, versión recursiva},
                   label = lst:factorial-recursive]
                  {code/factorial.c}
  Este es un ejemplo un tanto tonto,
  ya que es fácil ver que eventualmente llegaremos al caso base \(n = 0\),
  retornando \num{1},
  luego
  (trepando las llamadas recursivas)
  vemos que en cada paso multiplicamos por el siguiente valor.
  Esto lleva a la versión iterativa~\ref{lst:factorial-iterative}.
  \lstinputlisting[float, floatplacement = h,
                   language = C,
                   firstline = 15,
                   caption = {Factoriales, versión iterativa},
                   label = lst:factorial-iterative]
                  {code/factorial.c}
  En este caso simple la relación entre las llamadas
  es simple de dilucidar,
  dando una solución iterativa simple.

\section{Recorrer árbol binario}
\label{sec:binary-tree-traversal}

  La estructura de árbol binario tiene definición recursiva.
  Un árbol binario es:
  \begin{itemize}
  \item
    Vacío
  \item
    Consta de un nodo \emph{raíz},
    conectado a un subárbol binario izquierdo
    y un subárbol binario derecho.
  \end{itemize}
  Con esta definición recursiva,
  procesar estas estructuras
  (por ejemplo,
   visitar cada uno de sus nodos)
  tiene la obvia expresión recursiva
  del listado~\ref{lst:binary-tree-traversal}.
  \lstinputlisting[float, floatplacement = h,
                   language = C,
                   firstline = 5,
                   caption = {Recorrer un árbol binario},
                   label = lst:binary-tree-traversal]
                  {code/binary-tree-traversal.c}
  Esto también puede reducirse a una versión iterativa,
  pero será necesario manejar explícitamente las tareas pendientes
  mediante una pila explícita.
  En tales casos es preferible que el lenguaje de programación
  se haga cargo de estos detalles burocráticos.

\section{Torres de Hanoi}

  Según la leyenda
  (inventada por Édouard Lucas en el siglo~XIX),
  en un monasterio en Hanoi
  hay \num{64} placas redondas de oro
  en tres agujas de diamante.
  Una profecía dice que se crearon cuando se creó el mundo,
  con las placas ordenadas de mayor a menor en una de las agujas,
  los monjes del monasterio tienen la tarea de mover las placas a otra aguja.
  Solo se permite mover una placa a la vez
  y nunca se debe ubicar una placa sobre una menor.
  La figura~\ref{13::TorreHanoi1}
  muestra la situación inicial de una versión simplificada,
  con menos placas,
  común como acertijo.

  Lo que buscan los monjes es mover todas las placas de \(A\) a \(C\),
  usando \(B\) como auxiliar.
  \begin{figure}[ht]
    \centering
    \newcommand{\hanoiDisk}[4]
                  {\draw [fill = #4] (#1 - #3/2, #2)
                    -- ++ (#3, 0) -- ++ (0, 0.15)
                    -- ++ (-#3, 0) -- cycle} %(x, y, width, color)
    \begin{tikzpicture}[scale = 0.75]
      % Dibujo de la primera plataforma
      \hanoiDisk{0}{0}{2.5}{green!60!black!80};
      \hanoiDisk{2.5}{0}{2.5}{blue!60!black!60};
      \hanoiDisk{5}{0}{2.5}{green!60!black!80};
      \draw (0, 0.15) -- ++ (0, 2);
      \draw (2.5, 0.15) -- ++ (0, 2);
      \draw (5, 0.15) -- ++ (0, 2);

      % Dibujo de los discos de la primera plataforma
      \hanoiDisk{0}{0.15}{2}{gray!40};
      \hanoiDisk{0}{0.15 + 0.15}{1.8}{gray!20};
      \hanoiDisk{0}{0.15 + 2 * 0.15}{1.6}{gray!40};
      \hanoiDisk{0}{0.15 + 3 * 0.15}{1.4}{gray!20};
      \hanoiDisk{0}{0.15 + 4 * 0.15}{1.2}{gray!40};
      \hanoiDisk{0}{0.15 + 5 * 0.15}{1.0}{gray!20};
      \hanoiDisk{0}{0.15 + 6 * 0.15}{0.8}{gray!40};
      \hanoiDisk{0}{0.15 + 7 * 0.15}{0.6}{gray!20};
      \hanoiDisk{0}{0.15 + 8 * 0.15}{0.4}{gray!40};
      \node at (0, -0.25) (A) {\(A\)};
      \node at (2.5, -0.25) (B) {\(B\)};
      \node at (5, -0.25) (C) {\(C\)};

      % Dibujo de la segunda plataforma
      \hanoiDisk{0   + 8.5}{0}{2.5}{green!60!black!80};
      \hanoiDisk{2.5 + 8.5}{0}{2.5}{blue!60!black!60};
      \hanoiDisk{5   + 8.5}{0}{2.5}{green!60!black!80};
      \draw (0	 + 8.5, 0.15) -- ++ (0, 2);
      \draw (2.5 + 8.5, 0.15) -- ++ (0, 2);
      \draw (5	 + 8.5, 0.15) -- ++ (0, 2);

      % Dibujo de los discos de la segunda plataforma
      \hanoiDisk{5 + 8.5}{0.15}{2}{gray!40};
      \hanoiDisk{5 + 8.5}{0.15 + 0.15}{1.8}{gray!20};
      \hanoiDisk{5 + 8.5}{0.15 + 2 * 0.15}{1.6}{gray!40};
      \hanoiDisk{5 + 8.5}{0.15 + 3 * 0.15}{1.4}{gray!20};
      \hanoiDisk{5 + 8.5}{0.15 + 4 * 0.15}{1.2}{gray!40};
      \hanoiDisk{5 + 8.5}{0.15 + 5 * 0.15}{1.0}{gray!20};
      \hanoiDisk{5 + 8.5}{0.15 + 6 * 0.15}{0.8}{gray!40};
      \hanoiDisk{5 + 8.5}{0.15 + 7 * 0.15}{0.6}{gray!20};
      \hanoiDisk{5 + 8.5}{0.15 + 8 * 0.15}{0.4}{gray!40};
      \node at (0   + 8.5, -0.25) (A2) {\(A\)};
      \node at (2.5 + 8.5, -0.25) (B2) {\(B\)};
      \node at (5   + 8.5, -0.25) (C2) {\(C\)};

      \draw [-latex', thick, red!60!black!80] (6.35, 1.25) -- ++ (0.75, 0);
    \end{tikzpicture}
    \caption{Mover las placas desde la plataforma \(A\) a la \(C\).}
    \label{13::TorreHanoi1}
  \end{figure}

\subsection{Solución (recursiva)}

  Una solución
  es como se muestra en la figura~\ref{13::TorreHanoi2}.
  \begin{figure}[ht]
    \centering
    \newcommand{\hanoiDisk}[4]
         {\draw [fill = #4] (#1 - #3/2, #2) -- ++ (#3, 0)
           -- ++ (0, 0.15) -- ++ (-#3, 0) -- cycle} %(x, y, width, color)
    \begin{tikzpicture}[scale = 0.75]
      % Dibujo de la primera plataforma
      \hanoiDisk{0}{0}{2.5}{green!60!black!80};
      \hanoiDisk{2.5}{0}{2.5}{blue!60!black!60};
      \hanoiDisk{5}{0}{2.5}{green!60!black!80};
      \draw (0, 0.15) -- ++ (0, 2);
      \draw (2.5, 0.15) -- ++ (0, 2);
      \draw (5, 0.15) -- ++ (0, 2);

      % Dibujo de los discos de la primera plataforma
      \hanoiDisk{0}{0.15}{2}{gray!40};
      \hanoiDisk{2.5}{0.15 + 0 * 0.15}{1.8}{gray!20};
      \hanoiDisk{2.5}{0.15 + 1 * 0.15}{1.6}{gray!40};
      \hanoiDisk{2.5}{0.15 + 2 * 0.15}{1.4}{gray!20};
      \hanoiDisk{2.5}{0.15 + 3 * 0.15}{1.2}{gray!40};
      \hanoiDisk{2.5}{0.15 + 4 * 0.15}{1.0}{gray!20};
      \hanoiDisk{2.5}{0.15 + 5 * 0.15}{0.8}{gray!40};
      \hanoiDisk{2.5}{0.15 + 6 * 0.15}{0.6}{gray!20};
      \hanoiDisk{2.5}{0.15 + 7 * 0.15}{0.4}{gray!40};
      \node at (0, -0.25) (A) {\(A\)};
      \node at (2.5, -0.25) (B) {\(B\)};
      \node at (5, -0.25) (C) {\(C\)};

      % Dibujo de la segunda plataforma
      \hanoiDisk{0 + 8.5}{0}{2.5}{green!60!black!80};
      \hanoiDisk{2.5 + 8.5}{0}{2.5}{blue!60!black!60};
      \hanoiDisk{5 + 8.5}{0}{2.5}{green!60!black!80};
      \draw (0	 + 8.5, 0.15) -- ++ (0, 2);
      \draw (2.5 + 8.5, 0.15) -- ++ (0, 2);
      \draw (5	 + 8.5, 0.15) -- ++ (0, 2);

      % Dibujo de los discos de la segunda plataforma
      \hanoiDisk{5 + 8.5}{0.15}{2}{gray!40};
      \hanoiDisk{2.5 + 8.5}{0.15 + 0 * 0.15}{1.8}{gray!20};
      \hanoiDisk{2.5 + 8.5}{0.15 + 1 * 0.15}{1.6}{gray!40};
      \hanoiDisk{2.5 + 8.5}{0.15 + 2 * 0.15}{1.4}{gray!20};
      \hanoiDisk{2.5 + 8.5}{0.15 + 3 * 0.15}{1.2}{gray!40};
      \hanoiDisk{2.5 + 8.5}{0.15 + 4 * 0.15}{1.0}{gray!20};
      \hanoiDisk{2.5 + 8.5}{0.15 + 5 * 0.15}{0.8}{gray!40};
      \hanoiDisk{2.5 + 8.5}{0.15 + 6 * 0.15}{0.6}{gray!20};
      \hanoiDisk{2.5 + 8.5}{0.15 + 7 * 0.15}{0.4}{gray!40};
      \node at (0   + 8.5, -0.25) (A2) {\(A\)};
      \node at (2.5 + 8.5, -0.25) (B2) {\(B\)};
      \node at (5   + 8.5, -0.25) (C2) {\(C\)};

      \draw [-latex', thick, red!60!black!80] (6.35, 1.25) -- ++ (0.75, 0);
    \end{tikzpicture}
    \caption{Solo nos queda mover el último disco (más grande)
             de \(A\) a \(C\) directamente.}
    \label{13::TorreHanoi2}
  \end{figure}
  Esto es solución porque traduce el problema de mover \(n\) piezas
  a mover \(n - 1\) recursivamente,
  luego mover \num{1}
  (trivial, figura~\ref{13::TorreHanoi2}),
  luego mover \(n - 1\) recursivamente.
  Es critico en este diseño no considerar más que el disco mayor,
  en el que nos concentramos.
  Dejamos el completar la tarea al hada de recursión,
  la llamada recursiva es simplemente una caja negra
  cuyo funcionamiento interno no es de nuestra incumbencia.

\paragraph{Problema:}

  Mover \(n\) piezas de \(A\) a \(C\).

  \begin{proof}
    \begin{description}
    \item[Base:]
      Si \(n = 0\),
      no hay que hacer nada.
    \item[Inducción:]
      Supongamos que sabemos mover \(k\) piezas de \(i\) a \(j\)
      (con \(i \ne j\)).
      Para mover \(k + 1\) piezas de \(A\) a \(C\)
      (con \(B\) de \textquote{apoyo}):
      \begin{itemize}
      \item
        Movemos las \(k\) piezas superiores de \(A\) a \(B\).
      \item
        Movemos la pieza inferior de \(A\) a \(C\).
      \item
        Movemos las \(k\) piezas de \(B\) a \(C\)
        (con \(A\) de \textquote{apoyo}).
      \end{itemize}
    \end{description}
  \end{proof}
  Pseudocódigo es el algoritmo~\ref{alg:Hanoi}.
  Usamos las variables \(\mathnormal{src}\) para el origen,
  \(\mathnormal{dst}\) para el destino
  y \(\mathnormal{tmp}\) para el auxiliar
  (libre).
  Estas variables valen \(A\), \(B\) o \(C\),
  según sea.
  La solución al problema está dada por la llamada
  \(\mathop{hanoi}(64, A, C, B)\)
  (mueva \num{64} platos de \(A\) a \(C\),
   usando \(B\) de intermediario).
  \begin{algorithm}[ht]
    \DontPrintSemicolon\Indp

    \Procedure{\(\mathrm{hanoi}(n,
                     \mathnormal{src}, \mathnormal{dst}, \mathnormal{tmp})\)}{
      \If{\(n > 0\)}{
        \(\mathrm{hanoi}(n - 1,
              \mathnormal{src}, \mathnormal{tmp}, \mathnormal{dst})\) \;
        Move a disk from \(\mathnormal{src}\) to \(\mathnormal{dst}\) \;
        \(\mathrm{hanoi}(n - 1,
              \mathnormal{tmp}, \mathnormal{dst}, \mathnormal{src})\) \;
      }
    }
    \caption{Solución recursiva a las torres de Hanoi}
    \label{alg:Hanoi}
  \end{algorithm}
  Por la estructura de la solución,
  siempre cumplimos con las restricciones.

  Una pregunta obvia es el número de movidas que hace nuestro algoritmo.
  \begin{proposition}
    \label{pro:Hanoi-recursive}
    El algoritmo recursivo~\ref{alg:Hanoi}
    ejecuta \(2^n - 1\)~movidas para transferir \(n\) placas de \(A\) a \(C\).
  \end{proposition}
  \begin{proof}
    Sea \(T(n)\) el número de movidas para transferir \(n\) platos
    usando el algoritmo~\ref{alg:Hanoi}.
    La recursión lleva a la recurrencia:
    \begin{align*}
      T(0)
        &= 0 \\
      T(n + 1)
        &= 2 T(n) + 1,\qquad n \ge 0
    \end{align*}
    Técnicas estándar de solución de recurrencias entregan:
    \begin{equation*}
      T(n)
        = 2^n - 1
    \end{equation*}
    \qedhere
  \end{proof}
  \begin{proposition}
    \label{pro:Hanoi-optimal}
    En el problema de las torres de Hanoi
    se requieren al menos \(2^n - 1\) movidas para transferir \(n\) placas.
  \end{proposition}
  \begin{proof}
    La demostración es por inducción.
    Llamemos \(R(n)\) el número de movidas requeridas
    para transferir \(n\) placas.
    \begin{description}
    \item[Base:]
      Si no hay placas,
      no hay movidas,
      o sea,
      \(R(0) = 0 \ge 2^0 - 1\).
    \item[Inducción:]
      Supongamos que para mover \(k\)
      discos debemos hacer \(R(k) \ge 2^k - 1\) movidas,
      y consideremos mover \(k + 1\) discos.
      El disco mayor no puede ayudar en el proceso,
      siempre está abajo.
      Para poder moverlo,
      deberemos mover \(k\) discos de \(A\) a \(B\),
      luego mover el disco mayor de \(A\) a \(C\),
      finalmente mover los que están en \(B\) a \(C\).
      Esto suma:
      \begin{align*}
        R(k + 1)
          &\ge R(k) + R(k) + 1 \\
          &\ge 2 (2^k - 1) + 1 \\
          &=   2^{k + 1} - 1
      \end{align*}
    \end{description}
    Por inducción,
    se requieren \(R(n) \ge 2^n - 1\) movidas para todo \(n \in \mathbb{N}\).
  \end{proof}
  Como la cota inferior es exactamente lo que da nuestro algoritmo,
  este es óptimo.

\section{Interpretar expresiones regulares}
\label{sec:regex-interpreter}

  Un excelente ejemplo de recursión es el intérprete de
  (una versión muy restringida)
  de expresiones regulares escrito por Pike~
    descrito y comentado por Kernighan~%
    \cite{kernighan07:_regex_matcher}
  (también publicado en el texto de Kernighan y Pike~%
     \cite{kernighan99:_practice_programming}).
  Es una muy instructiva ilustración de recursión.
  Reconoce solo las siguientes construcciones:

  \begin{tabular}[h]{cl}
    c			   & Calza el caracter 'c'
                             (salvo los especiales a continuación) \\
    .			   & Calza cualquier caracter \\
    \textasciicircum	   & Calza el comienzo
                             del \emph{\foreignlanguage{english}{string}} \\
    \textdollar		   & Calza el final
                             del \emph{\foreignlanguage{english}{string}} \\
    \textasteriskcentered  & Cero o más ocurrencias del caracter anterior
  \end{tabular}

  El código
  (ligeramente editado para usanza moderna en C)
  es el listado~\ref{lst:Pike-regex},
  que usa el encabezado~\ref{lst:Pike-regex-h}.
  \lstinputlisting[language = C,
                   firstline = 7,
                   caption = {Declaraciones para la rutina de Pike},
                   label = lst:Pike-regex-h]{code/match.h}
  \lstinputlisting[language = C,
                   firstline = 9,
                   caption = {La rutina de Pike},
                   label = lst:Pike-regex]{code/match.c}
  Usa el simple expediente de intentar calzar en cada punto de la línea actual,
  llamando a \lstinline[language = C]!matchhere! en cada posición.
  La función \lstinline[language = C]!matchhere! es el corazón del código,
  ve si la expresión calza en la posición actual.
  Según si el primer caracter que queda por manejar de la expresión
  calza en la posición actual del código,
  recursivamente considera el resto de la expresión y del texto;
  si el caracter viene seguido por un asterisco
  invoca a \lstinline[language = C]!matchstar!,
  que intenta repetir el caracter inicial
  (al que se aplica asterisco)
  cero o más veces,
  llamando recursivamente a \lstinline[language = C]!matchhere!
  con el resto de la expresión y del texto.

\section{Mergesort}

  Queremos ordenar \(A[1, \ldots, n]\).
  Nuestro pseudocódigo para mergesort es el del algoritmo~\ref{alg:mergesort}.
  \begin{algorithm}[ht]
    \DontPrintSemicolon\Indp

    \Procedure{\(\mathrm{mergesort}(A[1, \dotsc, n])\)}{
      \If{\(n > 1\)}{
        \(\mathrm{mergesort}
             (A[1, \dotsc,
                 \left\lfloor \frac{n}{2} \right\rfloor])\) \;
        \(\mathrm{mergesort}
             (A[\left\lfloor \frac{n}{2} \right\rfloor + 1,
                 n])\) \;
        \(\mathrm{merge}
             (A[1, \dotsc,
                 \left\lfloor \frac{n}{2} \right\rfloor],
              A[\left\lfloor \frac{n}{2} \right\rfloor + 1,
                 n])\) \;
      }
    }
    \caption{Mergesort}
    \label{alg:mergesort}
  \end{algorithm}
  De la misma forma que lo hicimos con las torres de Hanoi:
  interesa el número de operaciones.
  Para simplificar el análisis,
  contabilizamos el número de copias de datos,
  por cada copia hay un número acotado de operaciones adicionales.
  Sea \(M(n)\) el número de copias para completar el ordenamiento.
  Es claro que:
  \begin{align}
    M(0)
      &= M(1) \notag \\
    M(n)
      &= M\left( \left\lfloor \frac{n}{2} \right\rfloor \right)
           + M\left( \left\lceil \frac{n}{2} \right\rceil \right)
           + m\left( \left\lfloor \frac{n}{2} \right\rfloor,
                     \left\lceil \frac{n}{2} \right\rceil \right)
                 \label{13::M1}
  \end{align}
  Acá \(m(a, b)\)
  es el costo de intercalar dos grupos de tamaños \(a\) y \(b\).
  Si consideramos el costo simplemente en términos de elementos a mover,
  vemos que \(m(a, b) = a + b\),
  todos los elementos se mueven de las áreas respectivas a la de salida,
  y el último término de~\eqref{13::M1} es simplemente \(n\).
  Discutiremos recurrencias como esta
  en el capítulo~\ref{cha:dividir-conquistar},
  donde concluiremos que:
  \begin{equation*}
    M(n)
      = O(n \log n)
  \end{equation*}
  Dentro de un factor constante,
  mergesort es óptimo
  (vimos en el apunte de \emph{Fundamentos de Informática}~%
     \cite{brand17:_fundamentos_informatica}
   que si solo se comparan elementos
   para ordenar \(n\)~elementos
   se requieren~\(\Omega(n \log n)\) comparaciones).

\section*{Ejercicios}
\label{sec:ejercicios-13}

  \begin{enumerate}
  \item
    Para plantear soluciones alternativas
    a problemas naturalmente expresados en forma recursiva
    sí debemos considerar las actividades de las llamadas recursivas,
    generalmente llegando al caso base y trabajar de allí en reversa.
    Para verificar que las soluciones son correctas,
    deben demostrarse propiedades de la solución,
    como por ejemplo en el caso de torres de Hanoi,
    suponiendo que las agujas están en los vértices de un triángulo equilátero,
    demostrar que:
    \begin{itemize}
    \item
      Cada disco se mueve siempre en la misma dirección
    \item
      Mover un disco pequeño siempre alterna con la única movida legal.
    \end{itemize}
  \item
    Una estrategia alternativa para las torres de Hanoi
    es considerar que las agujas
    están ubicadas en los vértices de un triángulo equilátero.
    Repita lo siguiente hasta completar la solución:
    \begin{itemize}
    \item
      Mueva el disco más pequeño una posición a la izquierda
    \item
      Haga la movida legal que no involucra al disco más pequeño.
    \end{itemize}
    Esta estrategia es más fácil de recordar para resolver el acertijo a mano,
    pero demostrar que es correcta es mucho más trabajo
    que para la solución recursiva del texto.
  \item
    Una solución no recursiva al problema de torres de Hanoi
    es como sigue.
    Nombre las posiciones como \(A\), \(B\) y \(C\),
    con los discos originalmente en \(A\) y deben quedar en \(C\).
    Si el número de discos es par,
    repita lo siguiente hasta completar la tarea:
    \begin{itemize}
    \item
      Haga la movida legal entre \(A\) y \(B\)
      (en cualquier dirección)
    \item
      Haga la movida legal entre \(A\) y \(C\)
      (en cualquier dirección)
    \item
      Haga la movida legal entre \(B\) y \(C\)
      (en cualquier dirección)
    \end{itemize}
    Si el número de discos es impar,
    repita lo siguiente hasta completar la tarea:
    \begin{itemize}
    \item
      Haga la movida legal entre \(A\) y \(C\)
      (en cualquier dirección)
    \item
      Haga la movida legal entre \(A\) y \(B\)
      (en cualquier dirección)
    \item
      Haga la movida legal entre \(B\) y \(C\)
      (en cualquier dirección)
    \end{itemize}
    \begin{enumerate}
    \item
      Determine el número de movidas para traspasar \(n\) discos
      de \(A\) a \(C\).
    \item
      Demuestre que esta estrategia resuelve el problema.
    \end{enumerate}
  \end{enumerate}

% To do:
% - Exercises (e.g. Smid)

\bibliography{../referencias}

%%% Local Variables:
%%% mode: latex
%%% TeX-master: "../INF-221_notas"
%%% ispell-local-dictionary: "spanish"
%%% End:

% LocalWords:  Recursión NFA DFA INF subproblemas printf C recursión
% LocalWords:  subproblema english recursion fairy Hanoi XIX ht c to
% LocalWords:  recursivamente Pseudocódigo recurrencia recurrencias
% LocalWords:  caracter string Mergesort pseudocódigo mergesort fill
% LocalWords:  equilátero Édouard cycle Move disk from cl

\bibliographystyle{babplain-fl}

\chapter{Recorrido de Grafos}
\label{cha:recorrido-grafos}

  Un grafo describe las conexiones entre objetos
  (vértices).
  Como tal,
  es una estructura cómoda para describir una gran variedad de situaciones,
  manipular estas estructuras es una tarea común.
  Hay una variedad de situaciones en las que se debe revisar completo
  algún grafo.
  Ejemplos típicos son los recorridos estándar
  (preorden, postorden, inorden)
  de árboles binarios
  (que en realidad no son grafos).

  Tradicionalmente se representan grafos
  mediante \emph{matrices de adyacencia}
  (una matriz de booleanos indexada por vértices,
   verdadero indica que los vértices son vecinos)
  o \emph{listas de adyacencia}
  (se asocia una lista de vecinos a cada vértice).
  Otra opción
  (particularmente para digrafos,
   en los cuales los arcos tienen una dirección)
  es representar vértices por nodos en una estructura enlazada
  y los arcos como punteros.
  En muchos casos de interés,
  el grafo a recorrer no se conoce explícitamente,
  dado un vértice en él se generan descendientes (vecinos) bajo demanda.

  Como es un área de múltiples aplicaciones,
  hay \emph{\foreignlanguage{english}{software}} genérico disponible
  para muchas tareas comunes,
  como el descrito por Knuth~%
    \cite{knuth09:_sgb}.

  Para uniformidad,
  supondremos un grafo \(G = (V, E)\)
  de vértices \(V\) y arcos \(E\).
  Para referirnos a los vértices de \(G\) anotamos \(V_G\),
  para sus arcos usaremos \(E_G\).
  El vecindario
  (\emph{\foreignlanguage{english}{neighborhood}})
  de \(v \in V_G\) lo anotaremos:
  \begin{equation*}
    N_G(v)
      = \{ u \in V_G \colon u v \in E_G \}
  \end{equation*}

  En nuestros algoritmos
  usaremos conjuntos (\emph{\foreignlanguage{english}{set}}),
  pilas (\emph{\foreignlanguage{english}{stack}}) y
  colas (\emph{\foreignlanguage{english}{queue}})
  como estructuras de datos.
  Bibliotecas proveen estas estructuras básicas para muchos lenguajes,
  es fácil armar al menos versiones rudimentarias
  (ineficientes o limitadas)
  si no están disponibles.
  Los algoritmos que discutiremos
  son aplicables con modificaciones menores a grafos dirigidos
  (digrafos)
  y a estructuras que no son grafos
  (como árboles binarios o árboles ordenados).

  Muchos algoritmos usarán un conjunto \(C\),
  comúnmente llamado \emph{cerrado}
  (\emph{\foreignlanguage{english}{closed}} en inglés)
  para registrar los vértices ya considerados
  de forma de evitar caer en ciclos.
  Si sabemos que la estructura es acíclica,
  podemos obviar el conjunto de vértices ya visitados.

  Generalmente no basta con dar con un vértice particular,
  se requiere el camino que lleva a él.
  La técnica general para acomodar esto
  es almacenar los vértices visitados,
  y a cada uno de ellos asociar el vértice padre
  (desde el cual llegamos a él).
  Recorriendo estas listas obtenemos el camino a la meta
  en reversa.

\section{Búsqueda en profundidad}
\label{seq:DFS}

  En inglés,
  se le conoce como \emph{\foreignlanguage{english}{depth first search}},
  abreviado DFS.
  Es recorrer un camino del grafo hasta el final
  (llegar a un vértice que no tenga vecinos no visitados aún),
  para devolverse al vértice anterior e continuar desde allí.

\subsection{Búsqueda en profundidad -- versión recursiva}
\label{sec:DFS-recursive}

  La forma más directa de describir esto es mediante un programa recursivo,
  vea el algoritmo~\ref{alg:DFS-recursive}.
  \begin{algorithm}[H]
    \DontPrintSemicolon

    \(C \gets \varnothing\) \;

    \Procedure{\(\mathrm{DFS}(G, s)\)}{
        \(C \gets C \cup \{ s \}\) \;
        \(\mathrm{process}(s)\) \;
        \ForEach{\(v \in N_G(s)\)}{
          \If{\(v \notin C\)}{
            \(\mathrm{DFS}(G, v)\) \;
        }
      }
    }
    \caption{Búsqueda en profundidad -- versión recursiva}
    \label{alg:DFS-recursive}
  \end{algorithm}

\subsection{Clasificar arcos}
\label{sec:clasify-arcs}

  Es claro que los arcos que sigue búsqueda en profundidad
  forman un árbol recubridor de \(G\) si este es conexo.
  A sus arcos los llamamos \emph{arcos de árbol}
  (en inglés \emph{\foreignlanguage{english}{tree edges}})
  o \emph{arcos hacia adelante}
  (en inglés \emph{\foreignlanguage{english}{forward edges}}).
  A los demás arcos de \(G\)
  (que llevan a vértices ya visitados)
  los llamamos \emph{arcos reversos}
  (en inglés \emph{\foreignlanguage{english}{back edges}}).

  Note que esta clasificación depende de las elecciones del algoritmo,
  no son realmente propiedades del grafo.

\subsection{Búsqueda en profundidad -- versión iterativa}
\label{sec:DFS-iterative}

  Notando que lo que hacemos es dejar pendientes los vértices no visitados
  para profundizar la búsqueda desde el actual,
  podemos evitar la recursión explícita
  manejando la pila de vértices pendientes manualmente.
  Vea el algoritmo~\ref{alg:DFS-iterative}.
  \begin{algorithm}[ht]
    \DontPrintSemicolon

    \Procedure{\(\mathrm{DFS}(G, s)\)}{
      \(S \gets \mathrm{Stack}\);
      \(\mathrm{push}(S, s)\);
      \(C \gets \varnothing\) \;
      \While{\(\neg \mathrm{empty}(S)\)}{
        \(s \gets \mathrm{pop}(S)\) \;
        \If{\(s \notin C\)}{
          \(C \gets C \cup \{ s \}\) \;
          \(\mathrm{process}(s)\) \;
          \ForEach{\(v \in N_G(s)\)}{
            \If{\(v \notin C\)}{
              \(\mathrm{push}(S, v)\) \;
            }
          }
        }
      }
    }
    \caption{Búsqueda en profundidad -- versión iterativa}
    \label{alg:DFS-iterative}
  \end{algorithm}

\section{Búsqueda a lo ancho}
\label{sec:BFS}

  En inglés,
  se le conoce como \emph{\foreignlanguage{english}{breadth first search}},
  abreviado BFS.

  La idea acá es visitar los vecinos de \(s\),
  luego de haberlos visitado todos
  comenzamos a revisar los vecinos de los vecinos de \(s\),
  y así sucesivamente.

  Sorprendentemente es un cambio menor
  respecto de la búsqueda en profundidad iterativa,
  cambiar la pila por una cola de vértices pendientes,
  vea el algoritmo~\ref{alg:BFS}.
  \begin{algorithm}[ht]
    \DontPrintSemicolon

    \Procedure{\(\mathrm{BFS}(G, s)\)}{
      \(Q \gets \mathrm{Queue}\);
      \(\mathrm{enqueue}(Q, s)\);
      \(C \gets \varnothing\) \;
      \While{\(\neg \mathrm{empty}(Q)\)}{
        \(s \gets \mathrm{dequeue}(Q)\) \;
        \If{\(s \notin C\)}{
          \(C \gets C \cup \{ s \}\) \;
          \(\mathrm{process}(s)\) \;
          \ForEach{\(v \in N_G(s)\)}{
            \If{\(v \notin C\)}{
              \(\mathrm{enqueue}(Q, v)\) \;
            }
          }
        }
      }
    }
    \caption{Búsqueda a lo ancho}
    \label{alg:BFS}
  \end{algorithm}

\section{Comentarios}
\label{sec:comments-graph-traversal}

  El programa más simple es búsqueda en profundidad recursiva.
  Búsqueda en profundidad tiene la virtud de solo usar espacio
  (en la pila explícita en la versión iterativa,
   o en la pila implícita usada para recursión)
  proporcional a la profundidad máxima.

  Búsqueda a lo ancho garantiza hallar el vértice más cercano,
  al ir desarrollando el grafo en oleadas.
  Requiere almacenar gran parte del árbol sobre los nodos considerados.
  Al no revisar los vértices en orden de hacer/deshacer,
  para aplicaciones de revisión sistemática
  no basta con una estructura global.

\section{Aplicaciones}
\label{sec:graph-traversal-uses}

  Un uso simple es determinar si un grafo es conexo,
  o identificar sus componentes conexos.

  Determinar si un grafo es bipartito es tomar un vértice,
  asignarle un color y darle el color contrario a sus vecinos;
  esto se repite hasta colorear todos los vértices o hallar una contradicción.

  Muchos de los algoritmos sobre grafos que veremos
  son variantes de búsqueda en su corazón.

\bibliography{../referencias}

%%% Local Variables:
%%% mode: latex
%%% TeX-master: "../INF-221_notas"
%%% ispell-local-dictionary: "spanish"
%%% End:


\bibliographystyle{babplain-fl}

\chapter{Búsqueda en Grafos}
\label{cha:busqueda-en-grafos}

  Hay una variedad de situaciones en las que se debe explorar algún grafo
  en busca de un vértice (nodo) meta.
  Antes se han discutido técnicas básicas,
  como búsqueda/recorrido de árboles,
  búsqueda en profundidad y a lo ancho en grafos,
  y algoritmos como los de Dijkstra y Bellman-Ford.
  Pero estos tienen la desventaja de considerar todos los nodos,
  interesan técnicas que elijan inteligentemente los nodos a revisar
  para minimizar el trabajo.

  Las técnicas son variantes de \emph{\foreignlanguage{english}{backtracking}}
  (ver el capítulo~\ref{cha:backtracking}).
  Exploraremos algunas de ellas.
  En los casos de interés,
  el grafo a recorrer no se conoce explícitamente,
  dado un vértice en él se generan descendientes (vecinos) bajo demanda.
  La estructura general es como describen Dechter y Pearl~%
    \cite{dechter85:_gen_best_first_search}.

\section{Branch and Bound}
\label{seq:branch-and-bound}

  Bajo este rótulo se agrupan muchas técnicas diferentes
  para buscar un nodo óptimo en el grafo,
  basadas en la idea de \emph{generar} nuevas opciones
  (esto es \emph{\foreignlanguage{english}{branch}})
  que se evalúan para \emph{podar} ramas que nacen de opciones
  que se puede demostrar no llevan a la meta
  (la evaluación provee la cota,
   \emph{\foreignlanguage{english}{bound}}).

  Formalmente,
  buscamos el nodo en el grafo \(G = (V, E)\) que maximiza la función
  \(f \colon V \to \mathbb{R}\).
  Suponemos que hay una función \(\Gamma \colon V \to 2^V\)
  que,
  dado un nodo entrega los vecinos relevantes.
  Contamos con una función cota \(b \colon V \to \mathbb{R}\)
  tal que \(b(x) \ge \max_{y \text{\ alcanzable de\ } x} f(y)\).

  Si se revisa el algoritmo~\ref{alg:branch-and-bound},
  es una versión modificada de las rutinas no recursivas
  de recorrido de grafos.
  Considera una cola \(Q\)
  (puede ser una cola de prioridad,
   donde puede tener sentido ordenar por el valor de \(g\)
   u otra función que refleja la probabilidad
   de hallar la solución desde ese nodo)
  y una cota \(B\),
  el valor de la mejor solución hallada hasta el momento.
  En la práctica se agregará la restricción
  que \(\Gamma\) no genere nodos anteriores
  para que no entre en un ciclo
  (por ejemplo,
   el grafo es un árbol o al menos un grafo dirigido acíclico,
   DAG).
   En el algoritmo usamos un \emph{\foreignlanguage{english}{bag}}
   de vértices
   (literalmente,
    una \textquote{bolsa} o \textquote{saco};
    una estructura en la que podemos poner elementos
    a ser sacados luego,
    no especifica orden).
  \begin{algorithm}
    \DontPrintSemicolon\Indp

    \Function{\(\mathrm{BB}\)}{
      \(x \gets \text{\ initial guess}\) \;
      \(\mathrm{B} \gets f(x)\) \;
      \(\mathrm{insert}(Q, x)\) \;
      \BlankLine \;
      \While{\(\neg \mathrm{empty}(Q)\)}{
        \(x \gets \mathrm{extract}(Q)\) \;
        \If{\(f(x) < B\)}{
          \(B \gets f(x)\) \;
        }
        \ForEach{\(v \in \Gamma(x)\)}{
          \If{\(b(v) \ge B\)}{
            \(\mathrm{enBag}(Q, v)\) \;
          }
        }
      }
      \Return \(B\) \;
    }
    \caption{Esquema de \emph{Branch and Bound}}
    \label{alg:branch-and-bound}
  \end{algorithm}
  Es claro que la misma idea puede aplicarse para minimizar,
  cambiando el sentido de las comparaciones.
  
  Una aplicación es resolver el problema del vendedor viajero
  (\textsc{TSP},
   un conocido problema \NP\nobreakdash-completo).
  El problema contempla un grafo \(G = (V, E)\)
  y un peso \(w \colon E \to \mathbb{R}\),
  y busca un ciclo hamiltoniano de \(G\)
  (un camino simple que visita cada vértice exactamente una vez)
  de peso mínimo.
  Modela el caso de un vendedor viajero que debe visitar a sus clientes,
  partiendo de la oficina y volviendo a ella,
  con mínimo costo.

  La idea es partir con un vértice cualquiera,
  y a partir de allí ir extendiendo el camino un arco a la vez.
  Analizando el algoritmo~\ref{alg:branch-and-bound}
  vemos que \(f(x)\) debe representar un valor que puede lograrse
  (en nuestro ejemplo,
   el costo del camino que se considera hasta \(x\)
   y un camino posible a través del resto del grafo,
   como el que resulta del algoritmo voraz
   que visita cada vez el vértice más cercano),
  Asimismo,
  \(g(x)\) debe ser una cota inferior al costo del camino
  que lleva del último vértice en el camino parcial al vértice inicial.
  Podemos usar para \(b(x)\) el costo del árbol recubridor mínimo
  del grafo que resulta de eliminar los vértices intermedios ya visitados
  (el camino óptimo es un árbol recubridor de este grafo,
   con la particularidad que es un camino simple,
   su costo no puede ser menor al del árbol recubridor mínimo).

\section{El algoritmo \(A^*\)}
\label{seq:A-star}

  Un algoritmo genérico de búsqueda es \(A^*\),
  desarrollado inicialmente para planificar rutas de robots
  moviéndose en un ambiente con obstáculos.
  En el algoritmo,
  vértices son posibles posiciones del robot,
  que comienza su viaje en \(s\),
  y debe llegar a la más cercana de las posiciones en \(T\)
  (pueden haber varios destinos alternativos).
  Desde la posición \(r\)
  pueden alcanzarse directamente las posiciones en \(\Gamma(r)\);
  si \(u \in \Gamma(r)\),
  conocemos el costo \(c(r, u)\) para moverse directamente de~\(r\) a~\(u\).
  Los proponentes de \(A^*\),
  Hart, Nilsson y Raphael~%
    \cite{hart68:_A-star}
  demuestran que es óptimo en el sentido que discutiremos.
  Dechter y Pearl~%
    \cite{dechter85:_gen_best_first_search}
  discuten esquemas de búsqueda en situaciones generalizadas,
  que incluyen el nuestro como caso muy particular,
  y discuten optimalidad de \(A^*\).

  Suponemos un grafo dirigido \(G = (V, E)\),
  con una función de costo de los arcos \(c \colon E \to \mathbb{R}\),
  donde se cumple que para una constante \(\delta > 0\)
  siempre es \(c(e) \ge \delta\)
  (esto evita caminos de largo infinito con costo finito),
  nos dan un conjunto de \emph{fuentes} \(S \subset V\),
  un conjunto de \emph{metas} \(T \subset V\)
  y un operador \emph{sucesor} \(\Gamma \colon V \to 2^V\)
  (vale decir,
   el grafo está dado en forma implícita solamente;
   suponemos además que cuando \(\Gamma\)
   nos entrega los descendientes de \(v\)
   simultáneamente entrega los costos
   desde el nodo \(v\) a cada uno de los vecinos).
  Note que \emph{no} estamos suponiendo que \(G\) es finito,
  suponemos eso sí que el número de nodos vecinos (descendientes)
  es siempre finito
  (a un grafo con esta propiedad le llaman \emph{localmente finito}).
  El subgrafo \(G_v\) es el nodo \(v\) junto con todos sus descendientes.
  Dado un nodo fuente \(s \in S\)
  nos interesa hallar en \(G_s\) el nodo \(t \in T\)
  que minimiza el costo del camino
  (la suma de los costos de los arcos)
  de \(s\) a \(t\).
  Al costo mínimo de un camino de \(u\) a \(v\) lo anotaremos \(h(u, v)\),
  para abreviar escribiremos \(h(v)\) para \(\min_{t \in T} \{ h(v, t) \}\)
  (\(h(v)\) es el costo del camino óptimo desde \(v\) a un destino).

  Podemos imaginar muchos algoritmos que expanden vértices
  y exploran los caminos que nacen de ellos,
  podando la búsqueda.
  Diremos que un algoritmo es \emph{admisible}
  si garantiza hallar un camino óptimo de \(s\) a una meta
  para todo grafo como descrito.
  Algoritmos admisibles podrán expandir diferentes nodos,
  o hacerlo en distinto orden.
  Interesa que el algoritmo expanda el mínimo número de nodos.
  Expandir nodos que se sabe que no pueden estar en un camino óptimo
  es desperdiciar esfuerzo,
  mientras ignorar nodos que están en un camino óptimo
  puede hacer que no lo encuentre y no ser admisible.
  Nos interesan algoritmos admisibles y eficientes.

  Supondremos una \emph{función de evaluación}
  \(\widehat{f} \colon V \to \mathbb{R}\),
  de manera de expandir a continuación
  el nodo de mínimo valor de \(\widehat{f}\).
  Esto sugiere el algoritmo~\ref{alg:A-star},
  que contempla una cola de prioridad \(Q\).
  Diremos que nodos en \(Q\)
  (al igual que nodos aún no considerados)
  están \emph{abiertos},
  y marcaremos ciertos nodos como \emph{cerrados}
  para no considerarlos nuevamente.
  \begin{algorithm}
    \DontPrintSemicolon\Indp

    \Function{\(A^* (G, s, T)\)}{
      \(\mathrm{Insert}(Q, s, \widehat{f}(s))\) \;
      \While{\(\not \mathrm{empty}(Q)\)}{
        \(v \gets \mathrm{DeleteMin}(Q)\) \;
        Mark \(v\) closed \;
        \If{\(v \in T\)}{
          \Return \(v\) \;
        }
        \ForEach{\(u \in \Gamma(v)\)}{
          \uIf{\(u \text{\ is not closed}\)}{
            \(\mathrm{Insert}(Q, u, \widehat{f}(u))\) \;
          }
          \ElseIf{\(\text{new \(\widehat{f}(u)\) less than old value}\)}{
            Remove closed mark from \(u\) \;
            \(\mathrm{Insert}(Q, u, \widehat{f}(u))\) \;
          }
        }
      }
    }
    \caption{El algoritmo \(A^*\)}
    \label{alg:A-star}
  \end{algorithm}
  En realidad nos interesa el camino de \(s\) a \(T\),
  la modificación obvia
  es registrar el nodo padre de \(v\) cuando lo marcamos cerrado
  (y corregirlo al volverlo a cerrar),
  finalmente seguimos la lista desde el nodo meta alcanzado hacia atrás
  para reconstruir el camino buscado.

  El algoritmo~\ref{alg:A-star}
  supone una cola de prioridad que permite modificar prioridades.
  La mayoría de las versiones disponibles en bibliotecas
  supone prioridades inmutables.
  En el caso que el algoritmo solicita modificar prioridades,
  podemos simplemente insertar el nodo con la nueva prioridad.
  Cuando nuevamente encontremos el nodo,
  será con prioridad más alta y lo descartaremos.
  Ensucia la cola con datos inútiles,
  pero parece no tener un impacto demasiado alto en el rendimiento,
  como muestran experimentos de Rintala y Valmari~%
   \cite{rintala15:_prio_queue_class}.

\subsection{La función de evaluación}
\label{sec:A-star-evaluacion}

  Para el subgrafo \(G_s\) sea \(f(v)\) el costo óptimo
  de un camino de \(s\) a \(T\),
  con la restricción que el camino pase por \(v\).
  Note que \(f(s) = h(s)\),
  que \(f(v) = f(s)\) para todo \(v\) en un camino óptimo,
  y que \(f(v) > f(s)\) si \(v\) no está en un camino óptimo.
  No conocemos \(f\)
  (determinar su valor es precisamente el objetivo del ejercicio),
  pero es razonable usar una estimación de \(f\)
  como función de evaluación \(\widehat{f}\)
  en el algoritmo~\ref{alg:A-star}.

  Podemos escribir \(f\) como una suma:
  \begin{equation}
    \label{eq:A-star:f=g+h}
    f(v)
      = g(v) + h(v)
  \end{equation}
  donde \(g(v)\) es el costo óptimo de un camino de \(s\) a \(v\)
  mientras \(h(v)\) es el costo óptimo de un camino de \(v\) a \(T\).
  Dadas estimaciones de \(g\) y \(h\)
  podemos calcular una aproximación a \(f\).
  Sea \(\widehat{g}\) una estimación de \(g\),
  un valor obvio es el costo del camino más corto hallado entre \(s\) y \(v\)
  hasta el momento,
  lo que implica \(\widehat{g}(v) \ge g(v)\).
  El siguiente punto es una estimación de \(h\),
  que llamaremos \(\widehat{h}\).
  Dependiendo del problema,
  definimos funciones \(\widehat{h}\) apropiadas,
  por el momento demostramos que si \(\widehat{h}(v) \le h(v)\),
  el algoritmo~\ref{alg:A-star} es admisible.
  \begin{lemma}
    \label{lem:A-star:1}
    Para un nodo no cerrado \(v\) y un camino óptimo \(P\) de \(s\) a \(v\),
    hay un nodo abierto \(v'\) en \(P\)
    con \(\widehat{g}(v') = g(v')\).
  \end{lemma}
  \begin{proof}
    Sea \(P = \langle s = v_0, \dotsc, v_k = v \rangle\).
    Si \(s\) está abierto
    (no se ha completado ninguna iteración),
    tome \(s = v'\),
    con lo que \(\widehat{g}(s) = g(s) = 0\),
    y el lema se cumple trivialmente.
    Supongamos ahora que \(s\) está cerrado,
    sea \(\Delta\) el conjunto de nodos cerrados \(v_i\) en \(P\)
    para los que \(\widehat{g}(v_i) = g(v_i)\).
    Sabemos que \(\Delta \ne \varnothing\),
    ya que \(s \in \Delta\).
    Sea \(v^*\) el elemento de \(\Delta\) con máximo índice
    (el último nodo cerrado de \(P\)),
    donde \(v^* \ne v\) porque \(v\) está abierto.
    Sea \(v'\) el sucesor de \(v^*\) en \(P\).
    Entonces:
    \begin{align*}
      \widehat{g}(v')
        &\le \widehat{g}(v^*) + c(v^* v')
          && \text{por definición de \(\widehat{g}\)} \\
      \widehat{g}(v^*)
        &=   g(v^*)
          && \text{porque \(v^* \in \Delta\)} \\
      g(v')
        &=   g(v^*) + c(v^* v')
          && \text{dado que \(P\) es óptimo}
    \end{align*}
    Concluimos que \(\widehat{g}(v') \le g(v')\),
    como \(\widehat{g}(v') \ge g(v')\) resulta \(\widehat{g}(v') = g(v')\),
    y por la definición de \(\Delta\),
    \(v'\) está abierto.
  \end{proof}
  \begin{corollary}
    \label{cor:A-star:1}
    Suponga que para todo \(v\) es \(\widehat{h}(v) \le h(v)\),
    y que \(A^*\) no ha terminado.
    Entonces para todo camino óptimo \(P\) de \(s\) a \(T\)
    hay un nodo abierto \(v' \in P\) con \(\widehat{f}(v') \le f(s)\).
  \end{corollary}
  \begin{proof}
    Por el lema~\ref{lem:A-star:1},
    hay un nodo abierto \(v' \in P\) con \(\widehat{g}(v') = g(v')\),
    con lo que por la definición de \(\widehat{f}\):
    \begin{align*}
      \widehat{f}(v')
        &=   \widehat{g}(v') + \widehat{h}(v') \\
        &=   g(v') + \widehat{h}(v') \\
        &\le g(v') + h(v') \\
        &=   f(v')
    \end{align*}
    Como \(P\) es óptimo,
    \(f(v') = f(s)\) para todo \(v' \in P\).
  \end{proof}
  Estamos en condiciones de demostrar:
  \begin{theorem}
    \label{theo:A-star:admisible}
    Si para todo \(v \in V\) es \(\widehat{h}(v) \le h(v)\),
    \(A^*\) es admisible.
  \end{theorem}
  \begin{proof}
    La demostración es por contradicción.
    Hay dos casos a considerar:
    \begin{description}
    \item[No termina:]
      Sea \(t \in T\),
      alcanzable desde \(s\) en un número finito de pasos
      con costo mínimo \(f(s)\).
      Como el costo de cada arco es a lo menos \(\delta\),
      se alcanza \(t\) en a lo más \(M = f(s) / \delta\) pasos,
      y para todos los vértices \(v\)
      más lejos de \(s\) que \(M\) es:
      \begin{equation*}
        \widehat{f}(v)
          \ge \widehat{g}(v)
          \ge g(v)
          \ge M \delta
      \end{equation*}
      O sea,
      ningún nodo a distancia mayor a \(M\) de \(s\) se expande,
      ya que por el corolario~\ref{cor:A-star:1}
      habrá un nodo abierto \(v'\)en un camino óptimo
      tal que \(\widehat{f}(v') \le f(s) < f(v)\).
      El algoritmo elegirá \(v'\) en vez de \(v\).
      Hay un número finito de nodos a distancia a lo más \(M\),
      cada uno de ellos puede ser reabierto solo un número finito de veces,
      ya que hay un número finito de caminos que pasan por él,
      y se reabre solo si calculamos un \(\widehat{g}(v)\) menor.
    \item[Entrega un camino no óptimo:]
      Supongamos que \(A^*\) termina en el nodo \(t\)
      con \(\widehat{f}(t) = \widehat{g}(t) > f(s)\).
      Por el corolario~\ref{cor:A-star:1}
      había un nodo abierto \(v'\) en un camino óptimo
      con \(\widehat{f}(v') \le f(s) < \widehat{f}(t)\).
      Se habría elegido \(v'\) para ser expandido en vez de \(t\),
      con lo que \(A^*\) no habría terminado.
    \end{description}
    \qedhere
  \end{proof}

\subsection{Optimalidad de \(A^*\)}
\label{sec:A-star:optimo}

  Hemos demostrado que si \(\widehat{h}(v) \le h(v)\),
  \(A^*\) es admisible.
  Una cota inferior obvia es \(\widehat{h}(v) = 0\),
  con lo que \(A^*\) es ciego
  (el resultado es esencialmente el algoritmo de Dijkstra).
  Muchos problemas ofrecen cotas inferiores mejores,
  que restringen los nodos a ser considerados.
  Por ejemplo,
  en el problema original de movimiento de un robot en un área con obstáculos,
  una cota inferior a la distancia a recorrer
  es la distancia entre dos puntos,
  obviando los obstáculos.
  En general,
  si omitimos algunas de las restricciones del problema,
  obtendremos un costo no mayor,
  o sea un valor admisible de \(\widehat{h}(v)\).

  Resulta que \(A^*\) es óptimo,
  en el sentido que expande el mínimo número de nodos
  entre todos los algoritmos que usan la misma información
  (la misma cota \(\widehat{h}\)).
  Esto porque un algoritmo que \emph{no} expanda
  todos los nodos con \(\widehat{f}(v) < f(s)\) para la meta \(s\)
  puede omitir el camino óptimo.

  Diremos que la estimación \(\widehat{h}\)
  cumple la condición de \emph{monotonía} si:
  \begin{equation}
    \label{eq:A-star:monotonia}
    h(u, v) + \widehat{h}(u)
      \ge \widehat{h}(v)
  \end{equation}
  La condición~\eqref{eq:A-star:monotonia}
  expresa que la estimación \(\widehat{h}(v)\)
  no puede mejorarse usando datos correspondientes de otros nodos.

  Resulta que si \(\widehat{h}\) cumple monotonía,
  nunca se reconsideran nodos.
  \begin{lemma}
    \label{lem:A-star:2}
    Suponga que se cumple
    la condición de monotonía~\eqref{eq:A-star:monotonia},
    y que \(A^*\) cerró el nodo \(v\).
    Entonces \(\widehat{g}(v) = g(v)\).
  \end{lemma}
  \begin{proof}
    Por contradicción.
    Considere el subgrafo \(G_s\) justo antes de cerrar \(v\),
    y suponga que \(\widehat{g}(v) > g(v)\).
    Sea \(P\) un camino óptimo de \(s\) a \(v\),
    como \(\widehat{g}(v) > g(v)\) el algoritmo no lo encontró.
    Por el lema~\ref{lem:A-star:1},
    hay un nodo abierto \(v' \in P\) con \(\widehat{g}(v') = g(v')\).
    Por suposición,
    \(v \ne v'\),
    con lo que:
    \begin{align*}
      g(v)
        &= g(v') + h(v', v) \\
        &= \widehat{g}(v') + h(v', v) \\
    \intertext{Vale decir:}
      \widehat{g}(v)
        &> \widehat{g}(v') + h(v', v) \\
    \intertext{Sumando \(\widehat{h}\) a ambos lados:}
      \widehat{g}(v) + \widehat{h}(v)
        &> \widehat{g}(v') + h(v', v) + \widehat{h}(v') \\
    \intertext{Aplicando~\eqref{eq:A-star:monotonia} al lado derecho:}
      \widehat{g}(v) + \widehat{h}(v)
        &> \widehat{g}(v') + \widehat{h}(v') \\
    \intertext{Por la definición de \(\widehat{f}\):}
      \widehat{f}(v)
        &> \widehat{f}(v')
    \end{align*}
    Pero en tal caso \(A^*\) hubiese expandido \(v'\),
    que estaba disponible,
    en vez de \(v\).
  \end{proof}

\section{Juegos}
\label{seq:juegos}

  Consideremos un juego en que compiten dos jugadores,
  \(A\) y \(B\),
  que juegan alternadamente.
  A cada posición
  (o estado)
  del juego se le asigna un valor,
  que indica qué tan buena es para el jugador.
  Claramente,
  cada jugador hará la movida
  que maximice el valor mínimo
  resultando de las posibles movidas siguientes del oponente.
  Una posibilidad es asignar el valor \(+1\)
  si es una posición en que \(A\) gana inmediatamente,
  \(-1\) si gana \(B\),
  y \num{0} si es empate.
  En este sentido,
  \(A\) busca maximizar,
  \(B\) busca minimizar.

\subsection{Min-Max}
\label{seq:minmax}

  Generalmente no es posible explorar el árbol completo,
  y evaluamos posiciones mediante alguna función heurística
  al llegar a una profundidad máxima.
  Un posible algoritmo es~\ref{alg:minmax},
  que se invoca
  como \(\operatorname{minmax}(\mathrm{inicio},
                               \mathrm{depth}, A)\)
  si \(A\) es quien abre el juego
  y queremos explorar hasta \(\mathrm{depth}\).
  Es simple registrar además la movida que da lugar al mejor valor
  (es la jugada a hacer).
  \begin{algorithm}
    \DontPrintSemicolon\Indp

    \Function{\(\operatorname{minmax}(
                   \mathrm{node},
                   \mathrm{depth},
                   \mathrm{turn})\)}{
      \If{\(\mathrm{depth} = 0
              \vee \mathrm{node} \text{\ is terminal}\)}{
        \Return heuristic value of \(\mathrm{node}\) \;
      }

      \eIf{\(\mathrm{turn} = A\)}{
        \(\mathrm{best} \gets -1\) \;
        \ForEach{\(\mathrm{child} \text{\ of\ } \mathrm{node}\)}{
          \(\mathrm{v}
              \gets \operatorname{minmax}(\mathrm{child},
                                          \mathrm{depth} - 1,
                                          B)\) \;
          \(\mathrm{best}
              \gets \max(\operatorname{best},
                         \mathrm{v})\) \;
        }
      }{
        \(\mathrm{best} \gets +1\) \;
        \ForEach{\(\mathrm{child} \text{\ of\ } \operatorname{node}\)}{
          \(\mathrm{v}
              \gets \operatorname{minmax}(\mathrm{child},
                                          \mathrm{depth} - 1,
                                          A)\) \;
          \(\mathrm{best}
              \gets \min(\mathrm{best},
                         \mathrm{v})\) \;
        }
      }
      \Return \(\mathrm{best}\) \;
    }
    \caption{Algoritmo MinMax}
    \label{alg:minmax}
  \end{algorithm}

\subsection{Alpha-Beta}
\label{seq:alphabeta}

  Supongamos que es el turno de \(A\)
  (busca maximizar),
  analizando posibles jugadas de \(B\)
  (busca minimizar).
  Si ya conocemos una cota \(\alpha\)
  (hemos visto una movida que garantiza ese valor para \(A\))
  no tiene sentido continuar explorando un camino
  si lo mejor que podemos lograr en él es peor,
  con consideraciones simétricas para \(B\).

  Nuestro algoritmo~\ref{alg:alphabeta}
  mantiene valores \(\alpha\)
  (el mínimo que ya tiene garantizado \(A\) que puede obtener)
  y \(\beta\)
  (el máximo que ya tiene garantizado \(B\) que puede obtener),
  y los usa para cortar la exploración tempranamente.
  Se invoca inicialmente
  como \(\mathrm{alphabeta}(\mathnormal{inicio},
           -1, +1, \mathrm{depth}, A)\)
  (la única garantía que tiene \(A\) es que puede perder,
   simétricamente \(B\) gana).
  Es obvio registrar con \(\alpha\)
  (respectivamente \(\beta\))
  la movida que da lugar a ese valor.
  \begin{algorithm}
    \DontPrintSemicolon\Indp

    \Function{\(\operatorname{alphabeta}(\mathrm{node},
                                         \mathrm{depth},
                                         \alpha, \beta,
                                         \mathrm{turn})\)}{
      \If{\(\mathrm{depth} = 0
              \vee \mathrm{node} \text{\ is terminal}\)}{
        \Return heuristic value of \(\mathrm{node}\) \;
      }

      \eIf{\(\mathrm{turn} = A\)}{
        \(\mathrm{best} \gets -1\) \;
        \ForEach{\(\mathrm{child} \text{\ of\ } \mathrm{node}\)}{
          \(\mathrm{v}
              \gets \operatorname{alphabeta}(\mathrm{child},
                                             \mathrm{depth} - 1,
                                             \alpha, \beta,
                                             B)\) \;
          \(\mathrm{best}
               \gets \max(\mathrm{best}, \mathrm{v})\)\;
          \(\alpha
               \gets \max(\alpha, \mathrm{best})\) \;
          \If{\(\beta \le \alpha\)}{
            \Break \;
          }
        }
      }{
        \(\mathrm{best} \gets +1\) \;
        \ForEach{\(\mathrm{child} \text{\ of\ } \operatorname{node}\)}{
          \(\mathrm{v}
              \gets \operatorname{alphabeta}(\mathrm{child},
                                             \mathrm{depth} - 1,
                                             \alpha, \beta,
                                              B)\) \;
          \(\mathrm{best}
               \gets \min(\mathrm{best}, \mathrm{v})\)\;
          \(\beta
              \gets \min(\beta, \mathrm{best})\) \;
          \If{\(\beta \le \alpha\)}{
            \Break \;
          }
        }
      }
      \Return \(\mathrm{best}\) \;
    }
    \caption{Algoritmo Alpha-Beta}
    \label{alg:alphabeta}
  \end{algorithm}
  Note que el algoritmo~\ref{alg:alphabeta}
  no especifica el orden en que se exploran los hijos de un nodo,
  claramente conviene explorar de forma que \(\alpha\) aumente rápidamente
  (\(\beta\) disminuya),
  porque eso limita las búsquedas.
  O sea,
  conviene explorar primero las mejores movidas.
  En la práctica se usa alguna evaluación heurística
  para ordenarlos adecuadamente.
  Knuth y Moore~%
    \cite{knuth75:_alpha_beta_pruning}
  discuten la historia del algoritmo,
  arguyendo que muchas de las variantes tempranas que se discuten como tal
  en realidad son algoritmos similares,
  bastante más limitados;
  dan una de las primeras descripciones precisas
  y un análisis de su rendimiento.
  Pearl~%
    \cite{pearl82:_branch_factor_alpha_beta}
  demuestra que es óptimo.

\bibliography{../referencias}

%%% Local Variables:
%%% mode: latex
%%% TeX-master: "../INF-221_notas"
%%% ispell-local-dictionary: "spanish"
%%% End:

% LocalWords:  english backtracking Branch and Bound acíclico DAG bag
% LocalWords:  TSP hamiltoniano recubridor optimalidad Mark closed of
% LocalWords:  less than old value Remove mark from Min Max heuristic
% LocalWords:  heurística MinMax Alpha branch bound subgrafo

\bibliographystyle{babplain-fl}

\setchessboard{
  showmover = false,
  label = false
}

\chapter{Backtracking}
\label{cha:backtracking}

  Una idea al resolver problemas complejos
  es ir construyendo la solución incrementalmente,
  explorando distintas ramas
  y volviendo atrás
  (\emph{\foreignlanguage{english}{backtrack}} en inglés)
  si resulta que un camino es sin salida.
  Es aplicable a problemas de búsqueda
  (estamos buscando un objeto con ciertas características,
   debemos contabilizar cuántos hay,
   nos interesa el \textquote{mejor} de ellos).
  En términos generales,
  buscamos construir el objeto buscado por etapas,
  en cada etapa interesa eliminar de consideraciones futuras
  objetos parcialmente construidos que no pueden completarse a lo buscado.
  La manera natural de organizar esto es mediante una estructura global,
  que representa un objeto parcialmente construido,
  en la cual se registra el trabajo de construcción.
  Generalmente habrán varias opciones de paso siguiente,
  intentamos cada una de ellas por turno,
  deshaciendo el cambio si no lleva al destino
  (o debemos explorarlas todas porque interesa hallarlos todos).
  Es básicamente un recorrido parcial
  de un árbol de objetos construidos a medias,
  en el cual tenemos un único objeto parcial activo.
  Esto corresponde a búsqueda en profundidad,
  efectuada naturalmente mediante recursión
  (aunque podríamos usar búsqueda en profundidad no recursiva
   de ser necesario).

  El esquema general es el dado por el algoritmo~\ref{alg:backtrack-schema}.
  \begin{algorithm}
    \DontPrintSemicolon

    \tcc*[l]{Set up basic object} \;

    \Function{\(\mathrm{backtrack}()\)}{
      \eIf{\(\mathrm{object}\) is complete}{
        Process \(\mathrm{object}\) \;
      }
      {
        \(S \gets \text{set of possible next steps}\) \;
        \ForEach{\(s \in S\)}{
          Do step \(s\) on \(\mathrm{object}\) \;
          \(\mathrm{backtrack}()\) \;
          Undo step \(s\) on \(\mathrm{object}\) \;
        }
      }
    }
    \caption{Esquema básico de \emph{\foreignlanguage{english}{backtracking}}}
    \label{alg:backtrack-schema}
  \end{algorithm}

  Resolver un problema usando \emph{\foreignlanguage{english}{backtracking}}
  significa efectuar los siguientes pasos:
  \begin{enumerate}[font = \textbf, label = {(\alph*)}]
  \item
    Definir los pasos de la construcción del objeto buscado.
  \item
    Diseñar la estructura de datos
    que representa objetos parcialmente construidos.
  \item
    Definir cómo determinar si se ha llegado a destino.
  \item
    Diseñar posibles estructuras auxiliares
    que ayuden a determinar pasos siguientes.
  \item
    Definir cómo hacer y deshacer modificaciones.
  \item
    Definir criterios que permitan abortar la búsqueda
    porque el objeto parcial no puede completarse a una solución.
    Esto puede requerir estructuras globales auxiliares.
    Esto es parte clave de la determinación de los pasos candidatos.
  \item
    Definir qué hacer con objetos completos.
    La opción de terminar la búsqueda es poco natural
    en la versión recursiva,
    considere basar su programa en búsqueda en profundidad no recursiva
    si no se requiere exploración completa.
  \end{enumerate}
  El esbozo anterior supone que todos los datos son globales.
  Puede tener sentido que la función \(\mathrm{backtrack}()\)
  tome como argumento el número del paso a tomar,
  o incluso la lista de decisiones que llevan a la situación presente.

  El esbozo no puede explicitar el orden en que se intentan pasos candidatos,
  Una elaboración obvia si solo interesa hallar un objeto
  es intentar los candidatos
  en algún orden que prometa hallar pronto un destino.

  \begin{ejemplo}[Un clásico]
    En el ajedrez la reina es la pieza más poderosa.
    Amenaza los casilleros en su fila y columna,
    y los ubicados en diagonal.
    La figura~\ref{fig:reina-amenaza} muestra los casilleros
    que amenaza una reina en el ajedrez.
    \begin{figure}[ht]
      \centering
      \setchessboard{setpieces = {Qc5}}
      \chessboard[pgfstyle = {[fill]circle},
                  padding = -1ex,
                  backfields  = {a5, b5, d5, e5, f5, g5, h5,
                                 c1, c2, c3, c4, c6, c7, c8,
                                 a3, b4, d6, e7, f8,
                                 a7, b6, d4, e3, f2, g1}
                 ]
      \caption{Los casilleros amenazados por una reina}
      \label{fig:reina-amenaza}
    \end{figure}
    Un problema clásico
    (propuesto por Max Bezzel en 1848)
    es determinar
    si se pueden ubicar ocho reinas en el tablero
    de manera que ninguna pueda amenazar a otra.
    Claramente no pueden ser más de ocho,
    puede haber a lo más una reina por columna.
    Resolver este problema
    mediante \emph{\foreignlanguage{english}{backtracking}}
    fue un ejemplo de programación estructurada de Dijkstra~%
      \cite{dijkstra72:_structured_programming}.

    Siguiendo los pasos:
    \begin{enumerate}[font = \textbf, label = {(\alph*)}]
    \item
      \textbf{Pasos de la construcción:}
      Agregamos reinas una a una al tablero.
      Un orden simple útil
      (porque elimina posiciones imposibles)
      es agregar reinas por columnas en orden.
    \item
      \textbf{Objetos parciales:}
      Basta registrar la fila en que se ubica la reina de cada columna
    \item
      \textbf{Estamos en el destino:}
      Ubicamos la reina en la última columna.
    \item
      \textbf{Estructuras auxiliares:}
      Es útil registrar qué filas y diagonales están libres.
    \item
      \textbf{Hacer/deshacer:}
      A cada columna le corresponde la posición de la reina,
      hacer y deshacer es actualizar esto.
      Se deben marcar libres filas y diagonales de la reina que se remueve,
      y marcar ocupadas las de la reina que se ubica.
    \item
      \textbf{Criterios para abortar:}
      No intentamos posicionar reinas en la columna en filas ya ocupadas.
    \item
      \textbf{Objetos completos:}
      Solo los contabilizamos
      o escribimos la solución,
      según sea el caso.
    \end{enumerate}
    En este caso:
    \begin{itemize}
    \item
      Ubicar reina en columna 1, 2, \ldots
    \item
      Registrar filas libres
      (para omitir ocupadas al ubicar la siguiente reina)
    \item
      Registrar diagonales libres.

      Reina en \(r, c\):
      \begin{align*}
        y - c
          &= 1 \cdot (x - r) \\
        r - c
          &= x - y \text{\ es constante} \\
        y - c
          &= -1 \cdot (x - r) \\
        c - r
          &= x + y \text{\ es constante}
      \end{align*}
    \end{itemize}
    Por ejemplo,
    luego de ubicadas las primeras tres reinas
    en las filas \num{1}, \num{3} y \num{5} la configuración resultante
    es la de la figura~\ref{fig:tres-reinas}.
    \begin{figure}[ht]
      \centering
      \setchessboard{setpieces = {Qa1, Qb3, Qc5}}
      \chessboard[pgfstyle = {[fill]circle},
                  padding = -1ex,
                  backfields  = {a2, a3, a4, a5, a6, a7, a8,
                                   b1, c1, d1, e1, f1, g1, h1,
                                   b2, c3, d4, e5, f6, g7, h8,
                                 b1, b2, b4, b5, b6, b7, b8,
                                   a3, c3, d3, e3, f3, g3, h3,
                                   a2, c4, d5, e6, f7, g8,
                                   a4, c2, d1,
                                 c1, c2, c3, c4, c6, c7, c8,
                                    a5, b5, d5, e5, f5, g5, h5,
                                    a7, b6, d4, e3, f2, g1,
                                    a3, b4, d6, e7, f8
                                }
                 ]
      \caption{Configuración con tres reinas}
      \label{fig:tres-reinas}
    \end{figure}
    Vemos que las filas y diagonales amenazadas por estas tres
    restringen muchísimo las posiciones viables
    para la cuarta y siguientes.
    Con estas tres reinas,
    para la cuarta reina quedan solo \num{3} posibilidades.

    Elegimos Python (!),
    en Python los arreglos tienen índices desde \num{0},
    rango de las variables \(r\), \(c\) es de \num{0} a~\num{7}.
    Interesan los rangos de las expresiones:
    \begin{description}
    \item[\boldmath\(r - c\)\unboldmath:]
      Rango es \(-7, \dotsc, 7\),
      sumar \num{7} para llevar al rango \(0, \dotsc, 14\).
    \item[\boldmath\(r + c\)\unboldmath:]
      Rango es \(0, \dotsc, 14\)
    \end{description}
    El programa final es el del listado~\ref{lst:8queens}.
    \lstinputlisting[float,
                     language = Python,
                     firstline = 3,
                     caption = {Ocho reinas en Python},
                     label = lst:8queens]
                    {code/8queens}
    La figura~\ref{fig:8reinas} muestra una de las \num{92}~soluciones.
    \begin{figure}[ht]
      \centering
      \setchessboard{setpieces = {Qa1, Qb5, Qc8, Qd6,
                                  Qe3, Qf7, Qg2, Qh4}}
      \chessboard
      \caption{Una solución para el problema de 8 reinas.}
      \label{fig:8reinas}
    \end{figure}
  \end{ejemplo}

\section{El algoritmo de Warnsdorf}
\label{sec:Warnsdorf-algo}

  Un problema también relacionado al ajedrez
  es determinar si un caballo puede visitar cada casillero del tablero
  exactamente una vez.
  La figura~\ref{fig:knight-moves} muestra las posibles movidas de un caballo.
  \begin{figure}[ht]
    \centering
    \setchessboard{setpieces = {Nd5}}
    \chessboard[pgfstyle = {[fill]circle},
                padding = -1ex,
                backfields  = {b6, c7, e7, f6, f4, e3, c3, b4}
               ]
    \caption{Movidas posibles de un caballo}
    \label{fig:knight-moves}
  \end{figure}
  Warnsdorf~%
    \cite{warnsdorf23:_roesselsprung_loesung}
  halló un algoritmo simple para resolver este problema
  sin tener que reconsiderar movidas:
  en cada paso elija la posición siguiente como aquella
  que da menos movidas posibles a continuación.

  En forma abstracta,
  este problema corresponde a determinar
  si el grafo con vértices los casilleros
  y arcos los posibles saltos entre ellos
  tiene un camino o ciclo hamiltonianos
  (que visitan cada vértice exactamente una vez).
  En el caso de grafos generales esto no evita reconsideraciones,
  pero resulta un criterio eficiente para hallar soluciones.
  La idea de considerar a continuación
  la configuración con menos vecinos es una heurística útil en muchos casos.

\section{Sudoku}
\label{sec:sudoku}

  En Sudoku se plantea una grilla de \(9 \times 9\) casillas
  a ser llenadas con los dígitos \num{1} a \num{9}
  de forma que cada fila, columna y subcuadrado de \(3 \times 3\)
  contenga cada dígito exactamente una vez.
  Un ejemplo de Sudoku
  es el dado en la figura~\ref{fig:Sudoku-dificil}.
  \begin{figure}[ht]
    \centering
    \subfloat[Problema]{
      \begin{tabular}{|*{3}{c@{\;\,}c@{\;\,}c|}}
        \hline
           &   &   &
           &   &   &
           & 1 & 2 \\

           &   &   &
           & 3 & 5 &
           &   &   \\

           &   &   &
         6 &   &   &
           & 7 &   \\
        \hline
         7 &   &   &
           &   &   &
         3 &   &   \\

           &   &   &
         4 &   &   &
         8 &   &   \\

         1 &   &   &
           &   &   &
           &   &   \\
        \hline
           &   &   &
         1 & 2 &   &
           &   &   \\

           & 8 &   &
           &   &   &
           & 4 &   \\

           & 5 &   &
           &   &   &
         6 &   &   \\
        \hline
      \end{tabular}
    }
    \hspace*{3em}
    \subfloat[Solución]{
      \begin{tabular}{|*{3}{c@{\;\,}c@{\;\,}c|}}
        \hline
         6 & 7 & 3 &
         8 & 9 & 4 &
         5 & 1 & 2 \\

         9 & 1 & 2 &
         7 & 3 & 5 &
         4 & 8 & 6 \\

         8 & 4 & 5 &
         6 & 1 & 2 &
         9 & 7 & 3 \\
        \hline
         7 & 9 & 8 &
         2 & 6 & 1 &
         3 & 5 & 4 \\

         5 & 2 & 6 &
         4 & 7 & 3 &
         8 & 9 & 1 \\

         1 & 3 & 4 &
         5 & 8 & 9 &
         2 & 6 & 7 \\
        \hline
         4 & 6 & 9 &
         1 & 2 & 8 &
         7 & 3 & 5 \\

         2 & 8 & 7 &
         3 & 5 & 6 &
         1 & 4 & 9 \\

         3 & 5 & 1 &
         9 & 4 & 7 &
         6 & 2 & 8 \\
        \hline
        \end{tabular}
    }
    \caption{Sudoku muy difícil}
    \label{fig:Sudoku-dificil}
  \end{figure}
  Estrategias para elegir siguiente casillero a llenar:
  \begin{itemize}
  \item
    Al azar/primero libre.
  \item
    Más restringido.
  \end{itemize}
  Estrategias para podar:
  \begin{itemize}
  \item
    Cuenta local
    (revisar si quedan opciones fila/columna/cuadrante)
  \item
    \emph{\foreignlanguage{english}{Look ahead}}
    (revisar si quedan casilleros sin opciones)
  \end{itemize}
  Norvig~%
    \cite{norvig:_sudoku}
  discute algunas estrategias y da un programa que resuelve Sudoku,
  reportando resultados.
  Skiena~%
    \cite{skiena08:_algor_desig_manual}
  reporta resultados del cuadro~\ref{tab:backtracking-Sudoku}
  para distintos problemas.
  \begin{table}[ht]
    \centering
    \begin{tabular}{l|l|r|r|r}
      \multicolumn{1}{c|}{\textbf{Combinaciones}}
         & \textbf{Criterio de Poda}
         & \multicolumn{1}{c|}{\textbf{Simple}}
         & \multicolumn{1}{c|}{\textbf{Mediano}}
         & \multicolumn{1}{c}{\textbf{Difícil}} \\
      \hline
      Azar & Local & 1\,904\,832 & 863\,305 & No \ldots\\
      Azar & Look ahead & 127 & 142 & 12\,507\,212\\
      Restringida & Local & 48 & 84 & 1\,243\,838\\
      Restringida & Look ahead & 48 & 65 & 10\,374
    \end{tabular}
    \caption{Rendimiento de variantes
             de \emph{\foreignlanguage{english}{backtracking}} en Sudoku}
    \label{tab:backtracking-Sudoku}
  \end{table}
  El simple es de los que típicamente se dan para principiantes,
  el mediano se planteó en un campeonato
  (y ninguno de los participantes pudo resolverlo),
  el difícil es el de la figura~\ref{fig:Sudoku-dificil}
  (tiene 17 pistas,
   es de los con menos pistas que tiene una única solución).
  El programa que obtiene estos valores es~%
    \cite{skiena06:_backtr_progr_solve_sudoku}.
  Las estrategias aplicadas por Skiena no son las únicas posibles,
  todo jugador serio aplica un menú de estrategias adicionales,
  como las que discute Norvig~%
    \cite{norvig:_sudoku}.
  Un buen ejemplo es el texto de Zambon~%
    \cite{zambon15:_sudoku_programming_c},
  quien desarrolla programas alrededor de Sudoku,
  incluyendo una gran colección de estrategias,
  usando \emph{\foreignlanguage{english}{backtracking}}
  solo como último recurso.
  En 2012 McGuire, Tugemann y Civario~%
    \cite{mcguire12:_no_16-clue_sudoku}
  demostraron mediante una búsqueda exhaustiva
  que tomó \num{7,1}~millones de horas-núcleo en un supercomputador
  que no hay problemas con solo \num{16}~pistas con solución única.

\section*{Ejercicios}
\label{sec:ejercicios-backtracking}

  \begin{enumerate}
  \item
    Podemos generalizar el problema de las 8~reinas en forma obvia
    al problema de \(n\)~reinas.
    Nuestra estrategia bastante ingenua funciona bien en el caso \(n = 8\),
    pero rápidamente se meterá en problemas si \(n\) es mayor.
    En particular,
    si solo interesa determinar si hay o no solución,
    basta revisar una parte de las opciones.
    Experimente con otras estrategias,
    como elegir la columna que tiene menos filas libres,
    definir un criterio para elegir dentro de la columna
    de manera de restringir al máximo hacia adelante
    (considerar por ejemplo las columnas más cercanas aún libres).
  \item
    Una \emph{rotulación graciosa} de un grafo \(G = (V, E)\)
    con \(n\) vértices asigna los rótulos \num{1} a \(n\) a los vértices,
    tal que al rotular los arcos
    con el valor absoluto de la diferencia entre los rótulos de sus vértices
    los arcos están rotulados con los números de \num{1} a \(n - 1\),
    cada uno exactamente una vez.
    Es claro que esto solo se puede hacer en árboles.

    Escriba un programa para determinar
    cuántas rotulaciones graciosas tiene el grafo \(P_n\).
  \item
    Determine si es posible que un caballo del ajedrez
    visite exactamente una vez cada una de las casillas del tablero.
  \item
    Nuestra solución al problema de 8~reinas
    considera las columnas estrictamente en orden.
    Una variante es considerar a continuación
    la columna con menos casilleros libres.
    Experimente con esta variante,
    contabilizando el número de posiciones de reinas que se intentan
    y también el tiempo total de solución.
  \end{enumerate}

\bibliography{../referencias}

%%% Local Variables:
%%% mode: latex
%%% TeX-master: "../INF-221_notas"
%%% ispell-local-dictionary: "spanish"
%%% End:

% LocalWords:  showmover label Backtracking incrementalmente english
% LocalWords:  backtrack recursión setpieces Qc pgfstyle fill circle
% LocalWords:  padding backfields b f g h c Subproblemas Qa Qb Python
% LocalWords:  Qd Qe Qf Qg Qh backtracking ht subproblemas reusarse
% LocalWords:  subcuadrado Look ahead Zambon supercomputador false ex
% LocalWords:  rotulación rotulaciones Steven is solution Process Set
% LocalWords:  Register move Undo Sudoku up basic object step on Max
% LocalWords:  backtracling Nd hamiltonianos

\bibliographystyle{babplain-fl}

\chapter{Dividir y Conquistar}
\label{cha:dividir-conquistar}

  Una de las mejores estrategias para diseñar algoritmos.
  Muchos de los algoritmos importantes se basan en esto,
  y su análisis presenta problemas matemáticos interesantes.

  La idea general es
  dado un problema grande,
  reducirlo a varios problemas menores del mismo tipo,
  y combinar resultados.

  Consideremos algún método de ordenamiento tonto
  (que llamaremos \(\mathrm{DumbSort}\))
  que siempre hace \(n^2\) comparaciones al ordenar
  un arreglo de \(n\)~elementos.
  Podemos crear un método mejor mediante la siguiente estrategia:
  \begin{itemize}
  \item
    Dado el arreglo, divídalo en dos mitades
    (o casi)
    de \(\lfloor n / 2 \rfloor\) y \(\lceil n / 2 \rceil\) elementos
  \item
    Ordene las mitades mediante \(\mathrm{DumbSort}\)
  \item
    Intercale los arreglos ordenados,
    lo que en el peor caso toma \(\lfloor n / 2 \rfloor\)~comparaciones.
  \end{itemize}
  El costo total
  (número de comparaciones)
  de este método para \(2 n\)~elementos es:
  \begin{equation*}
    2 n^2 + n
     < (2 n)^2
     = 4 n^2
  \end{equation*}
  Para \(n\) grande,
  esto es poco más que la mitad de lo que demanda \(\mathrm{DumbSort}\)
  para ordenar ese arreglo.
  Obviamente,
  podemos repetir el ejercicio para las mitades,
  recursivamente,
  con lo que se componen los ahorros.
  En el límite esto lleva al siguiente algoritmo de ordenamiento:
  \begin{example}[Merge Sort]
    Ya discutido en el capítulo~\ref{cha:recursion}.
    Para ordenar \(N\) elementos:
    \begin{itemize}
    \item
      Dividir en \textquote{mitades}
      de \(\left\lfloor \frac{N}{2} \right\rfloor\)
      y \(\left\lceil \frac{N}{2}\right\rceil\) elementos.
    \item
      Ordenarlas recursivamente.
    \item
      Intercalar resultados.
    \end{itemize}
  \end{example}
  \begin{example}[Potencias enteras]
    Es claro que si \(n \ge 0\) es entero,
    podemos escribir:
    \begin{equation}
      \label{eq:2}
      a^n
        = \begin{cases}
            1  & n = 0 \\
            \left( a^{\lfloor n / 2 \rfloor} \right)^2 \cdot a^{[2 \nmid n]}
               & n \ge 1
          \end{cases}
    \end{equation}
    Esto lleva al algoritmo~\ref{alg:integer-power}.
    \begin{algorithm}[ht]
      \DontPrintSemicolon\Indp

      \Function{\(\operatorname{power}(a, n)\)}{
        \If{\(n = 0\)}{
          \Return \num{1} \;
        }
        \(r \gets \operatorname{power}(a, \lfloor n / 2 \rfloor)\) \;
        \(r \gets r \cdot r\) \;
        \If{\(2 \nmid n\)}{
          \(r \gets r \cdot a\) \;
        }
        \Return \(r\) \;
      }
      \caption{Potencias enteras}
      \label{alg:integer-power}
    \end{algorithm}
    Es claro que el algoritmo~\ref{alg:integer-power}
    es válido siempre que la operación del caso sea asociativa.
  \end{example}
  \begin{example}[Búsqueda binaria]
    Un arreglo ordenado de \(N\) elementos,
    y una clave a buscar.
    Obtener el elemento
    en la posición \(\left\lfloor \frac{N}{2} \right\rfloor\),
    buscar en la mitad que tiene que contener la clave.
  \end{example}
  \begin{example}[Multiplicación de Karatsuba]
    Para multiplicar números de \(2 n\) dígitos,
    dividimos ambos en mitades~%
      \cite{karatsuba62:_multiplication}:
    \begin{align*}
      A
        &= 10^n a + b \\
      B
        &= 10^n c + d
    \end{align*}
    con \(0 \le a, b, c, d < 10^n\).

    Además:
    \begin{equation}\label{16::Primero}
      A \cdot B
        = 10^{2 n} a c + 10^n (a d + b c) + b d
    \end{equation}
    Notando que:
    \begin{equation*}
      (a + b) \cdot (c + d)
        = a c + (a d + b c) + b d
    \end{equation*}
    Podemos calcular los coeficientes de~\eqref{16::Primero}
    con \num{3} (no \num{4}) multiplicaciones
    (este truco se le atribuye a Gauß,
     quien lo empleaba para multiplicar números complejos):
    \begin{align*}
      &a c \\
      &b d \\
      &(a + b) (c + d) - a c - b d
    \end{align*}
    El programa~\ref{lst:karatsuba} muestra el funcionamiento del algoritmo.
    Obviamente,
    en uso real no se puede depender de la suma nativa
    (se emplea para números muy grandes,
     un \textquote{dígito} será una palabra de la arquitectura subyacente).
    Tampoco se emplearían operaciones de división y módulo
    para separar dígitos.
    \lstinputlisting[language = Python,
                     firstline = 3, lastline = 18,
                     caption = {Esbozo de multiplicación de Karatsuba},
                     label = lst:karatsuba]
                    {divide-and-conquer/karatsuba.py}
  \end{example}
  \begin{example}
    Otra aplicación de esta estrategia es el algoritmo de Strassen~%
      \cite{strassen69:_matrix_multiplication}
    para multiplicar matrices.
    Consideremos primeramente el producto de
    dos matrices de \(2 \times 2\):
    \begin{equation*}
      \begin{pmatrix}
        c_{1 1} & c_{1 2} \\
        c_{2 1} & c_{2 2}
      \end{pmatrix}
        = \begin{pmatrix}
            a_{1 1} & a_{1 2} \\
            a_{2 1} & a_{2 2}
          \end{pmatrix}
            \cdot
              \begin{pmatrix}
                b_{1 1} & b_{1 2} \\
                b_{2 1} & b_{2 2}
              \end{pmatrix}
    \end{equation*}
    Sabemos que:
    \begin{equation*}
      \begin{array}{l@{\qquad}l}
        c_{1 1}
          = a_{1 1} b_{1 1} + a_{1 2} b_{2 1} &
        c_{1 2}
          = a_{1 1} b_{1 2} + a_{1 2} b_{2 2} \\
        c_{2 1}
          = a_{2 1} b_{1 1} + a_{2 2} b_{2 1} &
        c_{2 2}
          = a_{2 1} b_{1 2} + a_{2 2} b_{2 2}
      \end{array}
    \end{equation*}
    Esto corresponde a \num{8} multiplicaciones.
    Definamos los siguientes productos:
    \begin{equation*}
      \begin{array}{l@{\qquad}l}
        m_1
          = (a_{1 1} + a_{2 2}) \, (b_{1 1} + b_{2 2}) &
        m_2
          = (a_{2 1} + a_{2 2}) \, b_{1 1} \\
        m_3
          = a_{1 1} \, (b_{1 2} - b_{2 2}) &
        m_4
          = a_{2 2} \, (b_{2 1} - b_{1 1}) \\
        m_5
          = (a_{1 1} + a_{1 2}) \, b_{2 2} &
        m_6
          = (a_{2 1} - a_{1 1}) \, (b_{1 1} + b_{1 2}) \\
        m_7
          = (a_{1 2} - a_{2 2}) \, (b_{2 1} + b_{2 2})
      \end{array}
    \end{equation*}
    Entonces podemos expresar:
    \begin{align*}
      \begin{array}{l@{\qquad}l}
        c_{1 1}
          = m_1 + m_4 - m_5 + m_7 &
        c_{1 2}
          = m_3 + m_5 \\
        c_{2 1}
          = m_2 + m_4 &
        c_{2 2}
          = m_1 - m_2 + m_3 + m_6
      \end{array}
    \end{align*}
    Con estas fórmulas se usan \num{7} multiplicaciones
    para evaluar el producto de dos matrices.
    Cabe hacer notar que estas fórmulas no hacen uso de conmutatividad,
    por lo que son aplicables también
    para multiplicar matrices de \(2 \times 2\)
    cuyos elementos son a su vez matrices.
    Podemos usar esta fórmula recursivamente
    para multiplicar matrices de \(2^n \times 2^n\).
  \end{example}

\section{La transformada rápida de Fourier}
\label{sec:FFT}

  Un algoritmo importantísimo basado en dividir y conquistar
  es la transformada rápida de Fourier,
  abreviada FFT
  (por el nombre en inglés,
   \emph{\foreignlanguage{english}{Fast Fourier Transform}}).
  Es relevante no solo por sus aplicaciones directas
  (particularmente en procesamiento de señales),
  también es la base para algunos otros algoritmos importantes
  y es el corazón de los algoritmos asintóticamente más rápidos conocidos
  para algunos problemas.
  La historia del algoritmo~%
    \cite{heideman84:_gauss_history_FFT }
  es compleja,
  ya Gauß por 1805 empleó una variante de él,
  muchos métodos similares hallaron uso en el intertanto.
  La versión moderna y su popularización se atribuye a Cooley y Tukey~%
    \cite{cooley65:_FFT}.
  La motivación siguiente se adapta de Dasgupta, Papadimitrou y Vazirani~%
    \cite{dasgupta06:_algorithms}.

  Vimos que dividir y conquistar ayuda al multiplicar números y matrices,
  ahora consideraremos polinomios.
  Sabemos que un polinomio de grado \(n\)
  queda determinado por sus valores en \(n + 1\) puntos distintos,
  lo que nos da una representación alternativa a la secuencia de coeficientes.
  Podemos dar el polinomio \(A(x)\) como:
  \begin{itemize}
  \item
    Su secuencia de coeficientes,
    \(a_0, a_1, \dotsc, a_n\)
  \item
    Sus valores en \(n + 1\) puntos distintos,
    \(A(x_0), A(x_1), \dotsc, A(x_n)\)
  \end{itemize}
  La segunda representación es muy atractiva
  a la hora de multiplicar polinomios,
  \(A(x) \cdot B(x)\) es simplemente
  \(A(x_0) \cdot B(x_0), A(x_1) \cdot B(x_1), \dotsc, A(x_n) \cdot B(x_n)\),
  son \(n + 1\) multiplicaciones,
  no \((n + 1) (n + 2) / 2\) como en el esquema tradicional:
  \begin{equation*}
    [x^k] A(x) B(x)
      = \sum_{0 \le j \le k} a_j b_{k - j}
  \end{equation*}
  Esto lleva a la idea de partir con los coeficientes
  de dos polinomios de grado \(d\);
  evaluar los polinomios en \(n\)
  puntos \(x_0, x_1, \dotsc, x_{n - 1}\) elegidos,
  donde \(n \ge 2 d + 1\);
  multiplicar los valores;
  y extraer los coeficientes del producto.

  Una primera idea es evaluar cada polinomio en pares positivo/negativo,
  vale decir:
  \begin{equation*}
    \pm x_0, \pm x_1, \dotsc, \pm x_{n / 2 - 1}
  \end{equation*}
  porque de esa forma
  las computaciones para calcular \(A(x_i)\) y \(A(-x_i)\)
  traslapan en gran medida,
  ya que las potencias pares de \(x_i\) coinciden con las de \(-x_i\).
  Separando potencias pares e impares podemos escribir:
  \begin{equation*}
    A(x)
      = A_e(x^2) + x A_o(x^2)
  \end{equation*}
  donde \(A_e\) y \(A_o\) tienen la mitad del grado de \(A\),
  con lo que:
  \begin{align*}
    A(x)
      &= A_e(x^2) + x A_e(x^2) \\
    A(-x)
      &= A_e(x^2) - x A_e(x^2)
  \end{align*}
  y basta entonces evaluar \(A_e\) y \(A_o\) en los \(n / 2\) puntos
  \(x_0^2, x_1^2, \dotsc, x_{n/2 - 1}^2\) y un poco de trabajo adicional.
  Lo malo es que este truco positivo/negativo
  solo puede aplicarse una vez,
  no podemos aplicarlo recursivamente
  \emph{a menos que usemos números complejos}.

  La pregunta ahora es,
  ¿qué números complejos elegir?
  Ingeniería reversa del proceso
  indica que partimos con un número de puntos que es una potencia de \num{2},
  que en cada recursión se dividen en dos
  hasta terminar con un único valor final,
  que podemos arbitrariamente fijar como \num{1}.
  O sea,
  en cada nivel tenemos las raíces cuadradas de los puntos del nivel previo,
  tenemos las \(n\)\nobreakdash-ésimas raíces complejas de \num{1},
  para \(n\) una potencia de \num{2}.
  Recordamos que podemos escribirlas
  \(1, \omega, \omega^2, \dotsc, \omega^{n - 1}\),
  donde \(\omega = \mathrm{e}^{2 \pi \mathrm{i} / 2}\).
  Si \(n\) es par,
  están pareadas positivo/negativo,
  o sea \(\omega^{n/2 + j} = - \omega^j\),
  y sus cuadrados son las \((n/2)\)\nobreakdash-ésimas raíces de \num{1}.
  Si partimos con las \(n\)\nobreakdash-ésimas raíces de \num{1},
  con \(n\) una potencia de \num{2},
  en cada nivel tendremos las raíces \((n / 2^k)\)\nobreakdash-ésimas
  para \(k = 0, 1, \dotsc\).
  Sucesivos niveles de la recursión funcionan perfectamente.
  El algoritmo~\ref{alg:FFT} resultante es la transformada rápida de Fourier.
  \begin{algorithm}[ht]
    \DontPrintSemicolon\Indp

    \Function{\(\mathrm{FFT}(A, \omega)\)}{
      \If{\(\omega = 1\)}{
        \Return \(A(1)\) \;
      }
      Write \(A(x) = A_e(x^2) + x A_o(x^2)\) \;
      Call \(\mathrm{FFT}(A_e, \omega^2)\) to evaluate \(A_e\)
        on even powers of \(\omega\) \;
      Call \(\mathrm{FFT}(A_o, \omega^2)\) to evaluate \(A_o\)
        on even powers of \(\omega\) \;
      \For{\(j \gets 0\) \KwTo \(n - 1\)}{
        Compute \(A(\omega^j)
                   = A_e(\omega^{2 j}) + \omega^j A_o(\omega^{2 j})\) \;
      }
      \Return \(A(\omega^0), A(\omega^1), \dotsc, A(\omega^{n - 1})\) \;
    }
    \caption{Transformada rápida de Fourier}
    \label{alg:FFT}
  \end{algorithm}
  Tenemos así un algoritmo \(O(n \log n)\)
  para evaluar un polinomio en los \(n\)~puntos \(\omega^j\),
  podemos multiplicar polinomios expresados como valores en tiempo \(O(n)\),
  falta obtener los coeficientes del producto,
  la operación inversa.
  Sorprendentemente,
  si:
  \begin{equation*}
    \langle \mathrm{valores} \rangle
      = \mathrm{FFT}(\langle \mathrm{coeficientes} \rangle, \omega)
  \end{equation*}
  es:
  \begin{equation*}
    \langle \mathrm{coeficientes} \rangle
      = \frac{1}{n}
          \mathrm{FFT}(\langle \mathrm{valores} \rangle, \omega^{-1})
  \end{equation*}
  Esto porque si:
  \begin{equation*}
    b_r
      = \sum_{0 \le k \le n} a_k \omega^{k r}
  \end{equation*}
  tenemos que:
  \begin{align*}
    \sum_{0 \le k < n} b_k \omega^{-k r}
      &= \sum_{0 \le k < n}
           \left(
             \sum_{0 \le j < n} a_j \omega^{j k}
           \right) \omega^{-k r} \\
      &= \sum_{0 \le j < n}
           a_j \sum_{0 \le k < n} \omega^{(j - r) k} \\
  \end{align*}
  Ahora bien,
  sabemos que:
  \begin{equation*}
    \sum_{0 \le k < n} \omega^{k m}
      = \begin{cases}
           n & n \mid m \\
           0 & n \centernot\mid m
        \end{cases}
  \end{equation*}
  Esto porque la suma es una serie geométrica.
  Si \(m \mid n\),
  es \(\omega^{k m} = 1\),
  dando el primer caso;
  cuando \(n \centernot\mid m\),
  tenemos que \(\omega^m \ne 1\),
  por definición es \(\omega^n = 1\) y:
  \begin{align*}
    \sum_{0 \le k < n} \omega^{k m}
      &= \frac{1 - \omega^{m n}}{1 - \omega^m} \\
      &= 0
  \end{align*}
  En nuestra suma solo cuando \(j = r\) es \(n \mid (j - r)\),
  dando el resultado anunciado:
  \begin{equation*}
    \sum_{0 \le k < n} b_k \omega^{-k r}
      = a_r
  \end{equation*}
  Esto completa un elegante algoritmo \(O(n \log n)\)
  para multiplicar polinomios.

  Multiplicar enteros,
  por ejemplo en notación decimal,
  es multiplicar dos polinomios evaluados en la base del caso.
  Puede adaptarse el algoritmo de multiplicación de polinomios
  para obtener un algoritmo \(O(n \log n)\) para números de \(n\) dígitos.

\section{El teorema maestro}
\label{sec:master-theorem}

  Un problema de tamaño \(n\) se reduce a \(a\) problemas de tamaño \(n / b\),
  que se resuelven recursivamente
  y las soluciones se combinan.
  El siguiente desarrollo toma de Bentley, Haken y Saxe~%
   \cite{bentley80:_master_theorem}
  y de CLRS~%
     \cite{cormen09:_CLRS}.
  Este último texto fue el que bautizó como \emph{teorema maestro}
  el tipo de resultados que discutimos.

  Si el trabajo para resolver una instancia de tamaño \(n\)
  la llamamos \(T(n)\),
  omitiendo pisos y cielos
  (cosa que justificaremos luego,
   esto restringe \(n\) a potencias de \(b\))
  y el trabajo para reducir y combinar soluciones lo llamamos \(f(n)\),
  resulta la recurrencia:
  \begin{equation}
    \label{eq:divide-and-conquer}
    T(n)
      = \begin{cases}
          a T(n / b) + f(n) & n > 1 \\
          T_1		    & n = 1
        \end{cases}
  \end{equation}
  La situación que estamos analizando
  indica que \(a \ge 1\), \(b > 1\), \(f(n) > 0\).
  Para llevar a una recurrencia lineal,
  hacemos el cambio de variable:
  \begin{equation*}
    W(r)
      = T(b^r)
  \end{equation*}
  y obtenemos:
  \begin{equation*}
    W(r)
      = \begin{cases}
          a W(r - 1) + f(b^r)  & r > 0 \\
          T_1		       & r = 0
        \end{cases}
  \end{equation*}
  La solución de esta recurrencia lineal de primer orden es:
  \begin{align*}
    W(r)
      &= T_1 a^r + \sum_{0 \le k < r} a^{r - 1 - k} f(b^{k + 1}) \\
      &= T_1 a^r + a^r \sum_{0 \le k < r} a^{- k - 1} f(b^{k + 1}) \\
      &= a^r
           \cdot \left(
                   T_1 + \sum_{1 \le k \le r} a^{-k} f(b^k)
                 \right)
  \end{align*}
  En términos de las variables originales,
  como:
  \begin{equation*}
    r
      = \log_b n
  \end{equation*}
  resulta:
  \begin{align*}
    a^r
      &= a^{\log_b n} \\
      &= \left( b^{\log_b a} \right)^{\log_b n} \\
      &= b^{\log_b a \cdot \log_b n} \\
      &= b^{\log_b n \cdot \log_b a} \\
      &= \left( b^{\log_b n} \right)^{\log_b a} \\
      &= n^{\log_b a}
  \end{align*}
  Definimos \(\alpha = \log_b a\),
  ya que es una constante que aparece frecuentemente en lo que sigue.
  Resulta:
  \begin{equation*}
    T(n)
      = n^\alpha
         \cdot \left(
                 T_1 + \sum_{1 \le r \le \log_b n} a^{-r} f(b^r)
               \right)
  \end{equation*}
  Si la suma converge cuando \(n\) tiende a infinito,
  vale decir,
  si \(f(n) = O(n^c)\) para algún \(c < \alpha\),
  es determinante el factor \(n^\alpha\):
  \begin{equation*}
    T(n)
      = \Theta(n^\alpha)
  \end{equation*}

  En caso que la suma no converja,
  dominará el segundo término.
  Analicemos primeramente el caso en que \(f(n) = \Omega(n^c)\),
  donde \(c > \alpha\).
  Es central la suma:
  \begin{equation*}
    \sum_{1 \le r \le \log_b n} a^{-r} f(b^r)
  \end{equation*}
  Los términos son positivos y crecen,
  siendo el último el mayor.
  O sea,
  \(T(n) = \Omega(f(n))\).
  Si suponemos además que \(a f(n / b) \le k f(n)\)
  para alguna constante \(k < 1\)
  la suma es acotada por una serie geométrica convergente,
  que podemos acotar por su límite:
  \begin{align*}
    \sum_{1 \le r \le \log_b n} a^{-r} f(b^r)
      &=   \frac{1}{a^{\log_b n}} \sum_{1 \le r \le \log_b n} a^r f(n / b^r) \\
      &\le \frac{1}{a n^\alpha} \sum_{1 \le r \le \log_b n} k^r f(n) \\
      &<   \frac{1}{a n^\alpha} \frac{f(n)}{1 - k}
  \end{align*}
  Esto con la cota inferior anterior se resume en:
  \begin{equation*}
    T(n)
      = \Theta(f(n))
  \end{equation*}

  El caso intermedio de mayor interés es
  \(f(n) = \Theta(n^\alpha \log^\beta n)\):
  \begin{align*}
    \sum_{1 \le r \le \log_b n} a^{-r} f(b^r)
      &= \sum_{1 \le r \le \log_b n} a^{-r} \Theta(a^r r^\beta) \\
      &= \sum_{1 \le r \le \log_b n} \Theta(r^\beta) \\
      &= \Theta \left(
                  \sum_{1 \le r \le \log_b n} r^\beta
                \right)
  \end{align*}
  Lo último es válido al ser una suma finita.
  Esta suma a su vez converge siempre que \(\beta < -1\),
  si \(\beta = -1\) es una suma harmónica
  (sabemos que \(H_n \sim \ln n + \gamma\))
  y si \(\beta > -1\) podemos acotar por una integral.
  Esto resulta en:
  \begin{equation*}
    T(n)
      = \begin{cases}
          \Theta(n^\alpha)		      & \text{si \(\beta < -1\)} \\
          \Theta(n^\alpha \log \log n)	      & \text{si \(\beta = -1\)} \\
          \Theta(n^\alpha \log^{\beta + 1} n) & \text{si \(\beta > -1\)}
        \end{cases}
  \end{equation*}

  Uniendo todas las piezas,
  tenemos:
  \begin{theorem}[Teorema Maestro]
    \label{theo:master-theorem}
    La recurrencia:
    \begin{equation}
    T(n)
      = \begin{cases}
          a T(n / b) + f(n) & n > 1 \\
          T_1		    & n = 1
        \end{cases}
    \end{equation}
    con constantes \(a \ge 1\), \(b > 1\),
    la función \(f(n) > 0\),
    con la constante \(\alpha = \log_b a\)
    tiene solución:
    \begin{equation*}
      T(n)
        = \begin{cases}
            \Theta(n^{\alpha})
               & \text{\(f(n) = O(n^c)\) para \(c < \alpha\)} \\
            \Theta(n^{\alpha})
               & \text{\(f(n) = \Theta(n^{\alpha} \log^\beta n)\)
                       con \(\beta < -1\)} \\
            \Theta(n^{\alpha} \log \log n)
               & \text{\(f(n) = \Theta(n^{\alpha} \log^\beta n)\)
                       con \(\beta = -1\)} \\
            \Theta(n^{\alpha} \log^{\beta + 1} n)
               & \text{\(f(n) = \Theta(n^{\alpha} \log^\beta n)\)
                       con \(\beta > -1\)} \\
            \Theta(f(n))
              & \text{\(f(n) = \Omega(n^c)\) con \(c > \alpha\)
                      y \(a f(n / b) < k f(n)\) para \(n\) grande
                      con \(k < 1\)}
          \end{cases}
    \end{equation*}
  \end{theorem}

\subsection{Recurrencias exactas y el teorema maestro}
\label{sec:exact-recurrence-master-theorem}

  En la práctica no podemos dividir arbitrariamente,
  tendremos recurrencias
  como la que describe realmente a mergesort:
  \begin{equation*}
    M(n)
      = M(\lceil n / 2 \rceil) + M(\lfloor n / 2 \rfloor) + c n
  \end{equation*}
  porque al dividir por ejemplo \num{101}~elementos en dos grupos
  tendremos un grupo de~\num{51} y otro de~\num{50}.
  Pero resulta que las soluciones de tales recurrencias
  tienen el mismo comportamiento asintótico que analizamos.

  Consideremos la recurrencia genérica:
  \begin{equation}
    \label{eq:recurrence-generic}
    T(n)
      = \sum_{1 \le j \le m} b_j T(\lfloor p_j n + \delta_j \rfloor)
          + \sum_{1 \le j \le m} b_j' T(\lceil p_j' n + \delta_j' \rceil)
          + f(n)
  \end{equation}
  con valores iniciales:
  \begin{equation}
    \label{eq:recurrence-generic-initial}
    T(n)
      = g(n) \qquad 0 \le n \le k
  \end{equation}
  Para que la recurrencia haga referencia a casos anteriores,
  debe ser \(0 < p_j < 1\) y \(0 < p_j' < 1\).
  En un algoritmo,
  cada término es un costo,
  serán positivos \(b_j\) y \(b_j'\).
  Es natural suponer que \(f(n)\)
  (en nuestros casos,
   el costo de subdividir y combinar)
  es monótona creciente,
  al igual que \(g\)
  (costos de casos pequeños).
  Es claro que bajo estas condiciones \(T(n)\) es monótona creciente
  (por inducción;
   para \(T(n + 1)\) cada término es mayor o igual al término respectivo
   en \(T(n)\)).
  Si consideramos las recurrencias similares
  que dan cotas inferiores y superiores a \(T\):
  \begin{align*}
    c(n)
      &= \sum_{1 \le j \le m} b_j c(\lfloor p_j n + \delta_j \rfloor)
           + \sum_{1 \le j \le m} b_j' c(\lfloor p_j' n + \delta_j' \rfloor)
           + f(n) \\
    c(n)
      &= g(n) \qquad 0 \le n \le k \\
    C(n)
      &= \sum_{1 \le j \le m} b_j C(\lceil p_j n + \delta_j \rceil)
           + \sum_{1 \le j \le m} b_j' C(\lceil p_j' n + \delta_j' \rceil)
           + f(n) \\
    C(n)
      &= g(n) \qquad 0 \le n \le k
  \end{align*}
  Nuevamente,
  por inducción es claro que:
  \begin{equation*}
    c(n) \le T(n) \le C(n)
  \end{equation*}
  Si hay una secuencia de \(n\) para los cuales todos los pisos y techos
  de~\eqref{eq:recurrence-generic} toman argumentos enteros,
  es entonces claro que para esos valores de \(n\):
  \begin{equation*}
    c(n) = T(n) = C(n)
  \end{equation*}
  Pero esta es exactamente la situación que resolvimos para el teorema maestro.
  Para el caso típico de \(n = b^k\),
  obtenemos:
  \begin{equation*}
    T(b^{\lfloor \log_b n \rfloor}) = c(b^{\lfloor \log_b n \rfloor})
      \le T(n)
      \le C(b^{\lceil \log_b n \rceil}) = T(b^{\lceil \log_b n \rceil})
  \end{equation*}
  Viendo el teorema maestro,
  teorema~\ref{theo:master-theorem},
  en todos los casos salvo el último
  esto significa una diferencia en un factor de a lo más \(b\) entre las cotas,
  las condiciones sobre el último caso dan un factor de \(k\).

\subsection{El teorema de Akra-Bazzi}
\label{sec:Akra-Bazzi}

  Una variante del teorema maestro es el teorema de Akra-Bazzi~%
       \cite{akra98:_solution_linear_recurrennce_equations},
  del que reportamos la versión de Leighton~%
    \cite{leighton96:_notes_better_master_theo},
  como simplificada en el texto de Lehman, Leighton y Meyer~%
    \cite{lehman18:_mathem_comput_scien}.
  \begin{theorem}[Akra-Bazzi]
    \label{theo:Akra-Bazzi}
    Sea una recurrencia de la forma:
    \begin{equation*}
      T(z)
        = g(z) + \sum_{1 \le k \le n} a_k T(b_k z + h_k(z))
           \quad \text{para \(z \ge z_0\)}
    \end{equation*}
    donde \(z_0\), \(a_k\) y \(b_k\) son constantes,
    sujeta a las siguientes condiciones:
    \begin{itemize}
    \item
      La recurrencia está bien definida para todo \(z \ge z_0\).
    \item
      Hay suficientes casos base.
    \item
      Para todo \(k\) se cumplen \(a_k > 0\) y \(0 < b_k < 1\).
    \item
      La función \(g(z)\) es no negativa,
      tal que \(\lvert g'(z) \rvert\) está acotado por un polinomio
    \item
      Para todo \(k\),
      \(\lvert h_k(z) \rvert = O(z/\log^2 z)\).
    \end{itemize}
    Entonces,
    si \(p\) es el único real tal que:
    \begin{equation*}
      \sum_{1 \le k \le n} a_k b_k^p
        = 1
    \end{equation*}
    la solución a la recurrencia cumple:
    \begin{equation*}
      T(z)
        = \Theta
            \left(
              z^p \left(
                     1 + \int_1^z \frac{g(u)}{u^{p + 1}}
                           \, \mathrm{d} u
                  \right)
            \right)
    \end{equation*}
  \end{theorem}
  Frente a nuestro tratamiento tiene la ventaja
  de manejar divisiones desiguales
  (\(b_k\) diferentes),
  y explícitamente
  considera pequeñas perturbaciones en los términos,
  como lo son aplicar pisos o techos,
  a través de los \(h_k(z)\).
  Diferencias con pisos y techos están acotados por una constante
  (la diferencia entre \(n / b\)
   y \(\lfloor n / b \rfloor\) o \(\lceil n / b \rceil\)
   es a lo más \((b - 1) / b < 1\)),
  mientras la cota del teorema permite que crezcan.

\subsection{Otras variantes}
\label{sec:master-theorem-variants}

  Variantes útiles del teorema maestro,
  pero de expresión bastante engorrosa,
  presenta Yap~%
    \cite{yap11:_elemen_approach_master_recurrences}.
  También discute en detalle una variedad de resultados afines.

\section{Ejemplos de dividir y conquistar}
\label{sec:divide-and-conquer-examples}

  Nuestros ejemplos se resumen en el cuadro~\ref{tab:complejidades}.
  En los casos límite
  (mergesort y búsqueda binaria)
  tenemos \(\alpha = 0 > -1\),
  que es lejos lo más común en la práctica.
  \begin{table}[ht]
    \centering
    \begin{tabular}{l*{3}{>{\(}c<{\)}}>{\(}l<{\)}}
      \multicolumn{1}{c}{\textbf{Algoritmo}}
                & \multicolumn{1}{c}{\boldmath\(a\)\unboldmath}
                & \multicolumn{1}{c}{\boldmath\(b\)\unboldmath}
                & \multicolumn{1}{c}{\boldmath\(f(n)\)\unboldmath}
                & \multicolumn{1}{c}{\boldmath\(t(n)\)\unboldmath} \\
      \hline
      Potencias
                & 1 & 2 & 1   & \Theta(\log n) \\
      Mergesort
                & 2 & 2 & n   & \Theta(n \log n) \\
      FFT
                & 2 & 2 & n   & \Theta(n \log n) \\
      Búsqueda binaria
                & 1 & 2 & 1   & \Theta(\log n) \\
      Karatsuba
                & 3 & 2 & n   & \Theta(n^{\log_2 3}) \\
      Strassen
                & 7 & 2 & n^2 & \Theta(n^{\log_2 7})
    \end{tabular}
    \caption{Complejidad de nuestros ejemplos}
    \label{tab:complejidades}
  \end{table}

  La recurrencia correcta para el número de comparaciones
  en ordenamiento por intercalación es:
  \begin{equation*}
    M(n)
      = M(\lfloor n / 2 \rfloor) + M(\lceil n / 2 \rceil)
          + \lfloor n / 2 \rfloor
  \end{equation*}
  El teorema de Akra-Bazzi es aplicable.
  La recurrencia es:
  \begin{equation*}
    M(n)
      = M(n / 2 + h_{+}(n)) + M(n / 2 + h_{-}(n)) + n - 1
  \end{equation*}
  Acá \(\lvert h_{\pm}(n) \rvert \le 1/2\),
  además \(a_{\pm} = 1\) y \(b_{\pm} = 1/2\).
  Tenemos \(\lvert g'(z) \rvert = 1 = O(1)\).
  Estos cumplen las condiciones del teorema,
  de:
  \begin{equation*}
    \sum_{1 \le k \le 2} a_k b_k^p = 1
  \end{equation*}
  resulta \(p = 1\),
  y tenemos la cota:
  \begin{equation*}
    M(n)
      = \Theta
          \left(
            z \left(
          1 + \int_1^n \frac{u - 1}{u^2} \, \mathrm{d} u
              \right)
          \right)
      = \Theta
          \left(
            n \ln n + 1
          \right)
      = \Theta(n \log n)
  \end{equation*}
  Otro ejemplo son los árboles de búsqueda aleatorizados
  (\emph{\foreignlanguage{english}{Randomized Search Trees}},
   ver por ejemplo Aragon y Seidel~%
     \cite{aragon89:_random_search_tree},
   Martínez y Roura~%
     \cite{martinez98:_random_binar_searc_trees}
   y Seidel y Aragon~%
     \cite{seidel96:_random_search_trees})
  en uno de ellos de tamaño~\(n\)
  una búsqueda toma tiempo aproximado:
  \begin{equation*}
    T(n)
      = \frac{1}{4} \, T(n / 4) + \frac{3}{4} \, T(3 n / 4) + 1
  \end{equation*}
  Nuevamente es aplicable el teorema~\ref{theo:Akra-Bazzi},
  de:
  \begin{equation*}
    \frac{1}{4} \, \left(\frac{1}{4}\right)^p
        + \frac{3}{4} \left(\frac{3}{4}\right)^p
      = 1
  \end{equation*}
  obtenemos \(p = 0\),
  y por tanto la cota
  \begin{equation*}
    T(z)
      = \Theta \left(
          z^0 \left( 1 + \int_1^z \frac{\mathrm{d} u}{u} \right)
        \right)
      = \Theta ( \log z )
  \end{equation*}


\section*{Ejercicios}
\label{sec:ejercicios-16}

  \begin{enumerate}
  \item % 20162c2p1
    Para cierto problema
    cuenta con tres algoritmos alternativos:
    \begin{description}
    \item[Algoritmo A:]
      Resuelve el problema dividiéndolo en cinco problemas
      de la mitad del tamaño,
      los resuelve recursivamente
      y combina las soluciones en tiempo lineal.
    \item[Algoritmo B:]
      Resuelve un problema de tamaño \(n\)
      resolviendo recursivamente dos problemas de tamaño \(n - 1\)
      y combina las soluciones en tiempo constante.
    \item[Algoritmo C:]
      Divide el problema de tamaño \(n\)
      en nueve problemas de tamaño \(n / 3\),
      resuelve los problemas recursivamente
      y combina las soluciones en tiempo \(O(n^2)\).
    \end{description}
    ¿Cuál elije si \(n\) es grande,
    y porqué?
  \item % 20162c2p3
    En un arreglo \(a\) se dice que la posición \(i\) es un \emph{mínimo local}
    si \(a[i]\) es menor a sus vecinos,
    o sea \(a[i - 1] > a[i]\) y \(a[i] < a[i + 1]\).
    Decimos además que \num{0} es un mínimo local si \(a[0] < a[1]\),
    y que lo es \(n - 1\) si \(a[n - 2] > a[n - 1]\)
    (los extremos tienen un único vecino).
    Dado un arreglo \(a\) de \(n\) números distintos,
    diseñe un algoritmo eficiente basado en dividir y conquistar
    para hallar un mínimo local
    (pueden haber varios).
    Justifique su algoritmo,
    y derive su complejidad aproximada.
  \item % 20162cRp4
    Se dan \(k\) listas ordenadas de \(n\) elementos,
    y se pide intercalarlas para obtener una única lista ordenada.
    Considere que el costo de intercalar es proporcional al largo del resultado.
    Analice las siguientes alternativas:
    \begin{enumerate}
    \item % 20162cRp4a
      Intercalar las primeras dos listas,
      intercalar el resultado con la tercera,
      y así sucesivamente hasta terminar el trabajo.
    \item % 20162cRp4b
      Usar un esquema de dividir y conquistar.
    \end{enumerate}
% TODO: Swizzle exercises from Stein, Bogart, Drysdale (chapter 5)
  \end{enumerate}

\bibliography{../referencias}

%%% Local Variables:
%%% mode: latex
%%% TeX-master: "../INF-221_notas"
%%% ispell-local-dictionary: "spanish"
%%% End:

% LocalWords:  conmutatividad FFT english Fast Transform intertanto
% LocalWords:  asintóticamente ésimas Write Call to evaluate on even
% LocalWords:  powers of CLRS mergesort eq recurrence generic Search
% LocalWords:  aleatorizados Randomized Trees

\bibliographystyle{babplain-fl}

\chapter{Dividir y Conquistar -- Soluciones Exactas}
\label{cha:dividir-conquistar-solucion}

  Lo discutido en el capítulo~\ref{cha:dividir-conquistar}
  da soluciones aproximadas a las recurrencias resultantes
  de algoritmos de dividir y conquistar.
  Las soluciones de las recurrencias exactas
  (con pisos y cielos)
  tienen soluciones muy extrañas.
  Como ejemplo,
  resolveremos exactamente la recurrencia
  para el número de comparaciones de mergesort:
  \begin{equation}
    \label{eq:mergesort-exact}
    M(n)
      = M(\lfloor n / 2 \rfloor)
          + M(\lceil n / 2 \rceil)
          + n
    \quad
    M(0)
      = M(1)
      = 0
  \end{equation}
  Seguimos a Sedgewick y Flajolet~%
    \cite{sedgewick13:_introd_anal_algor}
  para resolver la recurrencia.
  Si definimos diferencias:
  \begin{equation*}
    d(n)
      = M(n + 1) - M(n)
  \end{equation*}
  considerando por separado los casos \(n\) par e impar y simplificando
  concluimos que vale la recurrencia auxiliar:
  \begin{equation*}
    d(n)
      = d(\lfloor n / 2 \rfloor) + 1
    \quad
    d(1)
      = 1
  \end{equation*}
  Iterando esta recurrencia vemos que \(d(n) = \lfloor \log_2 n \rfloor + 1\)
  con lo que la solución es:
  \begin{equation}
    \label{eq:mergesort-interim}
    M(n)
      = n - 1 + \sum_{1 \le k < n} \lfloor \log_2 k \rfloor
  \end{equation}
  Hay varias formas de expresar la suma en forma cerrada.
  Consideremos primero la suma:
  \begin{equation*}
    \sum_{1 \le k < 2^r} \lfloor \log_2 k \rfloor
  \end{equation*}
  Vemos que en el rango \(2^s \le k < 2^{s + 1}\)
  es \(\lfloor \log_2 k \rfloor = s\),
  en ese rango hay \(2^s\) valores.
  O sea:
  \begin{align*}
    \sum_{1 \le k < 2^r} \lfloor \log_2 k \rfloor
      &= \sum_{0 \le s < r} s \cdot 2^s \\
      &= (r - 2) \cdot 2^r + 2
  \end{align*}
  La suma original es esto más algunos términos
  después de la última potencia de dos:
  \begin{align*}
    \sum_{1 \le k < n} \lfloor \log_2 k \rfloor
      &= \sum_{1 \le k < 2^{\lfloor \log_2 n \rfloor}} \lfloor \log_2 k \rfloor
           + \sum_{2^{\lfloor \log_2 n \rfloor} \le k < n}
               \lfloor \log_2 k \rfloor \\
      &= (\lfloor \log_2 n \rfloor - 2) \cdot 2^{\lfloor \log_2 n \rfloor} + 2
           + \sum_{2^{\lfloor \log_2 n \rfloor} \le k < n}
               \lfloor \log_2 n \rfloor \\
      &= (\lfloor \log_2 n \rfloor - 2) \cdot 2^{\lfloor \log_2 n \rfloor} + 2
           + (n - 2^{\lfloor \log_2 n \rfloor})
               \lfloor \log_2 n \rfloor \\
      &= n \cdot \lfloor \log_2 n \rfloor
           - 2^{\lfloor \log_2 n \rfloor + 1}
           + 2
  \end{align*}
  Con esto,
  de~\ref{eq:mergesort-interim}
  obtenemos:
  \begin{equation*}
    \label{eq:mergesort-solution}
    M(n)
      = n \cdot \lfloor \log_2 n \rfloor
          + 2 n
          - 2 {\lfloor \log_2 n \rfloor + 1}
  \end{equation*}
  Usando la parte fraccionaria:
  \begin{equation*}
    \{ x \}
      = x - \lfloor x \rfloor
  \end{equation*}
  podemos expresar esto como:
  \begin{align*}
    M(n)
      &= n \log_2 n - n \{ \log_2 n \}
          + 2 n
          - n \cdot 2^{1 - \{ \log_2 n \}} \\
      &= n \log_2 n
          + n \left(
                1 + 1 - \{ \log_2 n \} - 2^{1 - \{ \log_2 n \}}
              \right) \\
      &= n \log_2 n + n \phi(1 - \{ \log_2 n \})
  \end{align*}
  Acá es:
  \begin{equation}
    \label{eq:phi}
    \phi(x)
      = 1 + x - 2^x
  \end{equation}
  Interesa el máximo de esto en el rango \(0 < x \le 1\)
  (es \(\phi(0) = \phi(1) = 0\);
   esta función es convexa,
   su segunda derivada es negativa),
  que se da para \(x^* = - \ln \ln 2 / \ln 2 \approx 0,5288\),
  donde vale:
  \begin{equation*}
    \phi(x^*)
      = 1 - \frac{\ln \ln 2}{\ln 2} - \frac{1}{\ln 2}
      \approx 0,0861
  \end{equation*}
  O sea,
  tenemos \(0 \le \phi(x) \le 0,0861\).
  Nuestra solución aproximada es muy cercana a la solución exacta.

  Los primeros resultados generales sobre este tipo de recurrencias
  son los de Erdős, Hildebrand, Odlyzko, Pudaite y Reznik~%
    \cite{erdos87:_asymptotic_behavior_family_sequences},
  quienes demuestran
  (usando técnicas muy diferentes)
  que la recurrencia:
  \begin{equation*}
    a(n)
      = \sum_{1 \le k \le s} r_k a(\lfloor n / m_k \rfloor)
      \qquad a(0) = 1
  \end{equation*}
  tiene soluciones diferentes dependiendo si hay enteros \(d, u_k\)
  tales que \(m_k = d^{u_k}\) o no
  (elegimos el valor máximo de \(d\) en este caso,
   o sea,
   los \(u_k\) relativamente primos).
  Sea \(\tau\) la única solución real a:
  \begin{equation*}
    \sum_{1 \le k \le s} \frac{r_k}{m_k^\tau}
      = 1
  \end{equation*}
  Si hay tales enteros,
  llaman el caso \emph{entramado}
  (en inglés, \emph{\foreignlanguage{english}{lattice}}),
  en caso contrario \emph{ordinario}.
  En el caso ordinario,
  la solución cumple:
  \begin{equation*}
    \lim_{n \to \infty} \frac{a(n)}{n^\tau}
      = c_o
  \end{equation*}
  para una constante explícita \(c_o\) fácil de calcular;
  en el caso entramado ese límite no existe,
  pero existe:
  \begin{equation*}
    \lim_{k \to \infty} \frac{a(d^k)}{f^{k \tau}}
      = c_l
  \end{equation*}
  para otra constante fácil de calcular.

  Hwang, Janson y Tsai~%
    \cite{hwang17:_exact_asymp_solut_divid_conquer}
  dan soluciones exactas a recurrencias de la forma
  \begin{equation*}
    f(n)
      = f(\lfloor n / 2 \rfloor) + f(\lceil n / 2 \rceil) + g(n)
  \end{equation*}
  que resultan ser de la forma:
  \begin{equation*}
    f(n)
      = F(n) + n P( \log_2 n) - Q(n)
  \end{equation*}
  donde \(F(n) = 0\) o más que lineal,
  \(P\) es una función con período~\num{1}
  y \(Q(n) = o(n)\).
  Muestran cómo expresar estas funciones explícitamente.

  Drmota y Szpankowski~%
    \cite{drmota13:_master_theorem_discrete}
  discuten recurrencias discretas de la forma
  \begin{equation*}
    T(n)
      = a_n + \sum_{1 \le j \le m} b_j T(\rfloor p_j n + \delta_j \lfloor)
  \end{equation*}
  donde la secuencia \(a_n\) es conocida,
  \(b_j\) y \(\delta_j\) son constantes dadas.

\bibliography{../referencias}

%%% Local Variables:
%%% mode: latex
%%% TeX-master: "../INF-221_notas"
%%% ispell-local-dictionary: "spanish"
%%% End:

% LocalWords:  mergesort english lattice

\bibliographystyle{babplain-fl}

\chapter{Matrimonios Estables}
\label{cha:gale-shapley}

  Nuestro interés principal es algoritmos que operan con objetos discretos,
  de los que estudia la combinatoria.
  Un primer ejemplo muestra que situaciones a primera vista simples
  pueden tener profundidades insospechadas.

\section{El problema}
\label{sec:stable-marriage-problem}

  El problema a resolver es el de matrimonios estables
  (\emph{\foreignlanguage{english}{stable marriage problem}}):
  Dados un conjunto de mujeres \(\mathscr{X}\)
  y otro de hombres \(\mathscr{Y}\),
  donde \(\lvert \mathscr{X} \rvert = \lvert \mathscr{Y} \rvert\),
  cada mujer define un orden de deseabilidad para los hombres,
  y similarmente los hombres con las mujeres,
  donde suponemos que no hay empates.
  \begin{definition}
    Un conjunto de matrimonios se dice \emph{inestable}
    si incluye parejas \(u v\) y \(x y\),
    tales que \(u\) prefiere a \(y\) frente a \(v\)
    y \(y\) prefiere a \(u\) frente a \(x\).
  \end{definition}
  Si el conjunto es inestable,
  \(u\) y \(x\) pueden mejorar sus elecciones divorciándose
  y volviéndose a casar.
  Buscamos matrimonios estables.
  Este problema y variantes aparece en un amplio rango de situaciones,
  resumidas por Iwama y Miyazaki~%
    \cite{iwama08:_stable_marriage_probl_survey}.

  Hallar un conjunto estable de matrimonios parece ser simplemente
  \textquote{cumpla las preferencias},
  pero reflexión más profunda muestra que ni siquiera es obvio
  que tal conjunto existe.
  Es claro que no todos pueden obtener su mejor opción,
  si hay un hombre que es el preferido de dos o más mujeres,
  una de ellas al menos deberá conformarse con otro.

  Una posibilidad es asignar parejas al azar,
  y en caso de que cambiando parejas se pueda mejorar,
  permitir divorcios y nuevos matrimonios.
  Sin embargo,
  el siguiente ejemplo
  (adaptado de Knuth~%
     \cite{knuth96:_stable_marriage})
  muestra que esto no siempre termina.
  Considere las preferencias del cuadro~\ref{tab:sm-divorces-remarriages},
  donde simplemente damos las mujeres en orden de preferencia para cada hombre
  y viceversa.
  \begin{table}[ht]
    \centering
    \subfloat[Hombres]{
      \begin{tabular}{>{\(}c<{\):}*{3}{>{\(}c<{\)}}}
        1 & 2 & 1 & 3 \\
        2 & \multicolumn{3}{c}{arbitrario} \\
        3 & 1 & 2 & 3
      \end{tabular}
      \label{subtab:sm-dr-men}
    }
    \hspace*{3em}
    \subfloat[Mujeres]{
      \begin{tabular}{>{\(}c<{\):}*{3}{>{\(}c<{\)}}}
        1 & 1 & 3 & 2 \\
        2 & 3 & 1 & 2 \\
        3 & \multicolumn{3}{c}{arbitrario}
      \end{tabular}
      \label{subtab:sm-dr-women}
    }
    \caption{Contraejemplo para divorcios y matrimonios}
    \label{tab:sm-divorces-remarriages}
  \end{table}
  Una secuencia de divorcios y matrimonios entra en un ciclo;
  pero hay soluciones estables,
  como \(((x_1, y_2), (x_2, y_3), (x_3, y_1))\)
  o \(((x_1, y_1), (x_2, y_3), (x_3, y_2))\).

  Una solución puede hallarse mediante un algoritmo bastante sencillo,
  y este algoritmo muestra incidentalmente
  que los matrimonios estables siempre existen.
  El algoritmo fue
  discutido formalmente por primera vez por Gale y Shapley~%
    \cite{gale62:_stable_marriage}.
  \begin{theorem}
    \label{theo:stable-marriage-exists}
    Siempre hay un conjunto de matrimonios estable.
  \end{theorem}
  \begin{proof}
    Consideremos el modelo tradicional,
    en el cual los hombres proponen matrimonio a las mujeres,
    y estas aceptan o no.
    Efectuamos varias rondas,
    en cada ronda los hombres sin pareja proponen matrimonio
    a la mujer más alta en su preferencia que no lo ha rechazado aún,
    cada mujer elige entre las propuestas que recibe
    al hombre más alto en sus preferencias
    y se compromete provisoriamente.
    Si una mujer provisoriamente comprometida recibe una mejor oferta,
    disuelve el compromiso
    y se compromete con el nuevo pretendiente.

    Note que una vez que una mujer recibe una propuesta,
    nunca más queda libre
    (puede cambiar de novio,
     claro).
    En cada ronda un hombre propone a mujeres
    hasta hallar una que lo acepte,
    el número de mujeres no comprometidas disminuye.
    Los hombres pueden proponer siempre en orden de preferencia decreciente
    (las mujeres que ya lo rechazaron solo pueden mejorar su pretendiente,
     no lo aceptarán después).
    El número total de propuestas es a lo más \(n^2 - 2 n + 2\) propuestas,
    si \(\lvert \mathscr{X} \rvert = \lvert \mathscr{Y} \rvert = n\).
    Una vez que todas las mujeres han recibido propuestas
    el noviazgo se declara terminado
    y los compromisos se formalizan.

    El resultado es estable,
    cosa que demostramos por contradicción.
    Supongamos que \(x\) tiene a \(y\) de pareja,
    pero prefiere a \(y'\),
    quien a su vez prefiere a \(x\) sobre su marido \(x'\).
    Entonces \(x\) propuso matrimonio a \(y'\) antes que a \(y\),
    y fue rechazado a cambio de alguien a quien \(y'\) prefiere a \(x\).
    Si \(y'\) cambió su compromiso en el intertanto,
    fue por alguien a quien prefiere aún más que a \(x\).
    O sea,
    \(y'\) prefiere a \(x'\),
    no hay inestabilidad.
  \end{proof}
  Podemos extender trivialmente al caso
  \(\lvert \mathscr{X} \rvert \ne \lvert \mathscr{Y} \rvert\),
  simplemente sobrarán hombres o mujeres que no encuentran pareja,
  y por el mismo razonamiento los matrimonios acordados son estables.

  Una pregunta obvia es si la solución es única,
  y la relación entre esta
  y la que da el algoritmo simétrico en que proponen las mujeres.
  Resulta que pueden haber muchas soluciones,
  como demuestra Knuth~%
    \cite{knuth96:_stable_marriage}
  con el ejemplo del cuadro~\ref{tab:sm-example-many}.
  Se muestran solo las dos primeras preferencias de los hombres
  y las últimas de las mujeres
  (las demás no importan,
   no entran en soluciones estables).
  \begin{table}[ht]
    \centering
    \subfloat[Hombres]{
      \begin{tabular}{>{\(}r<{\)}@{:\;}*{2}{>{\(}c<{\)}}@{\;\({} \cdots {}\)}}
        1     & 1     & 2    \\
        2     & 2     & 1    \\
        \multicolumn{1}{r}{\(\vdots\)\phantom{:}} &
          \multicolumn{1}{c}{\(\vdots\)} &
          \multicolumn{1}{c}{\(\vdots\)} \\
        n - 1 & n - 1 & n     \\
        n     & n     & n - 1
      \end{tabular}
      \label{subtab:sm-ex-men}
    }
    \hspace*{5em}
    \subfloat[Mujeres]{
      \begin{tabular}{>{\(}r<{\)}@{:\;\({} \cdots {}\)\;}>{\(}c<{\)}}
        1     & 1    \\
        2     & 2    \\
        \multicolumn{1}{r}{\(\vdots\)\phantom{:\;\({} \cdots {}\)}} &
          \multicolumn{1}{c}{\(\vdots\)} \\
        n - 1 & n - 1 \\
        n     & n
      \end{tabular}
      \label{subtab:sm-ex-women}
    }
    \caption{Preferencias para muchas soluciones,
             \(n\) par}
    \label{tab:sm-example-many}
  \end{table}
  Vemos que el hombre \num{1} puede formar parejas estables
  con las mujeres \num{1} o \num{2},
  mientras el hombre \num{2} queda con \num{2} o \num{1};
  el hombre \num{3} con \num{3} o \num{4},
  dejando la otra para \num{4};
  y así sucesivamente.
  Ambas posibilidades son estables,
  si dos hombres se conforman con sus segundas opciones,
  son la última preferencia para su primera opción
  y ella nunca lo preferirá.
  Esto da \(2^{n/2}\) soluciones posibles.

  Incidentalmente,
  si se permiten preferencias incompletas
  (\textquote{prefiero muerto que casado con\ldots}),
  puede no haber solución estable.
  Nuevamente Knuth~%
    \cite{knuth96:_stable_marriage}
  plantea un ejemplo.
  \begin{table}[ht]
    \centering
    \subfloat[Hombres]{
      \begin{tabular}{>{\(}c<{\)}@{:\;}*{3}{>{\(}c<{\)}}}
        1 & 1 &	&     \\
        2 & 3 & 2 & 1 \\
        3 & 3 & 1 &
      \end{tabular}
      \label{subtab:sm-inc-men}
    }
    \hspace*{3em}
    \subfloat[Mujeres]{
      \begin{tabular}{>{\(}c<{\)}@{:\;}*{3}{>{\(}c<{\)}}}
        1 & 3 & 1 & 2 \\
        2 & 2 & 1 & 3 \\
        3 & 1 & 2 & 3
      \end{tabular}
      \label{subtab:sm-inc-women}
    }
    \caption{Preferencias incompletas}
    \label{tab:sm-incomplete}
  \end{table}
  En el cuadro~\ref{tab:sm-incomplete} la única posibilidad es
  \(((x_1, y_1), (x_2, y_2), (x_3, y_3))\),
  pero esta es inestable por \(x_2\) e \(y_3\).

  En realidad,
  se da:
  \begin{theorem}
    \label{theo:sm-optimal}
    La solución dada por el algoritmo de Gale-Shapley
    es la mejor posible para los hombres.
  \end{theorem}
  \begin{proof}
    Llame a una mujer \emph{posible} para \(x\)
    si hay una solución estable que la da como pareja para \(x\).
    Demostraremos que cada hombre se casa
    con su mujer posible más alta en sus preferencias
    por inducción.
    Supongamos que en un cierto punto del algoritmo
    ningún hombre ha sido rechazado por una mujer posible.
    Al inicio esto se cumple vacuamente;
    si siempre se cumple durante la ejecución del algoritmo
    se cumple al final,
    y cada hombre se casa con su pareja posible preferida.
    Suponga que en este punto \(y\),
    habiendo recibido una propuesta mejor,
    rechaza a \(x\).
    Debemos demostrar que \(y\) es imposible para \(x\).
    Si \(y\) elige a \(x'\),
    es porque prefiere a \(x'\) sobre \(x\);
    y \(x'\) se propuso a \(y\) porque todas sus mejores opciones
    lo rechazaron.
    Por inducción,
    \(x'\) es imposible para esas otras opciones.
    Si se casara \(x\) con \(y\),
    \(x'\) deberá conformarse con \(y'\),
    que considera menos deseable.
    Pero esta configuración es inestable,
    \(x'\) e \(y\) estarían dispuestos a intercambiar parejas.
    En resumen,
    \(x\) es imposible para \(y\).
  \end{proof}
  Solo si la solución es única
  los resultados de propuestas de hombres y propuestas de mujeres coinciden.
  Gusfield~%
    \cite{gusfield87:_three_fast_algor_four_probl_stabl_marriag}
  da algoritmos eficientes para hallar todas las combinaciones estables
  (¡pueden ser muchas!)
  y otros problemas afines.

\section{Postulación a carreras}
\label{sec:postulacion-carreras}

  Una extensión es considerar estudiantes \(\mathscr{A}\)
  que postulan a universidades \(\mathscr{U}\),
  donde la universidad \(u \in \mathscr{U}\) ofrece \(q_u\) vacantes.
  Nuevamente,
  cada estudiante tiene una lista de prioridades de las universidades
  y cada universidad una lista de preferencias de postulantes.
  Todos postulan a su universidad preferida
  entre las que aún no lo han rechazado,
  y la universidad con \(q\) cupos elige los \(q\) mejores
  entre los que tiene en la lista actualmente y los nuevos llegados
  (posiblemente rechazando a quienes no cumplen requisitos mínimos,
   con lo que la lista podría tener menos de \(q\) postulantes).
  El proceso termina cuando todos los estudiantes
  o están en una lista de espera
  o han sido rechazados por todas las universidades
  a las que pueden postular.
  En forma similar al teorema~\ref{theo:stable-marriage-exists}
  se demuestra que el resultado es estable
  (ningún estudiante se cambiaría de universidad con otro),
  como en el teorema~\ref{theo:sm-optimal}
  los postulantes obtienen sus mejores cupos posibles.

\section*{Ejercicios}
\label{sec:ejercicios-07-previa}

  \begin{enumerate}
  \item
    Encuentre las soluciones que entrega el algoritmo
    orientado a hombres y a mujeres
    para las preferencias del cuadro~\ref{tab:sm-exercise}.
    \begin{table}[ht]
      \centering
      \subfloat[Hombres]{
        \begin{tabular}{>{\(}c<{\)}@{:}*{8}{@{\;\;}>{\(}c<{\)}}}
          1 & 5 & 7 & 1 & 2 & 6 & 8 & 4 & 3 \\
          2 & 2 & 3 & 7 & 5 & 4 & 1 & 8 & 6 \\
          3 & 8 & 5 & 1 & 4 & 6 & 2 & 3 & 7 \\
          4 & 3 & 2 & 7 & 5 & 1 & 6 & 8 & 4 \\
          5 & 7 & 2 & 5 & 1 & 3 & 6 & 8 & 4 \\
          6 & 1 & 6 & 7 & 5 & 8 & 4 & 2 & 3 \\
          7 & 2 & 5 & 7 & 6 & 3 & 4 & 8 & 1 \\
          8 & 3 & 8 & 4 & 5 & 7 & 2 & 6 & 1
        \end{tabular}
        \label{subtab:sm-exer-men}
      }
      \hspace*{2em}
      \subfloat[Mujeres]{
        \begin{tabular}{>{\(}c<{\)}@{:}*{8}{@{\;\;}>{\(}c<{\)}}}
          1 & 5 & 3 & 7 & 6 & 1 & 2 & 8 & 4 \\
          2 & 8 & 6 & 3 & 5 & 7 & 2 & 1 & 4 \\
          3 & 1 & 5 & 6 & 2 & 4 & 8 & 7 & 3 \\
          4 & 8 & 7 & 3 & 2 & 4 & 1 & 5 & 6 \\
          5 & 6 & 4 & 7 & 3 & 8 & 1 & 2 & 5 \\
          6 & 2 & 8 & 5 & 3 & 4 & 6 & 7 & 1 \\
          7 & 7 & 5 & 2 & 1 & 8 & 6 & 4 & 3 \\
          8 & 7 & 4 & 1 & 5 & 2 & 3 & 6 & 8
        \end{tabular}
        \label{subtab:sm-exer-women}
      }
      \caption{Preferencias para ejercicio}
      \label{tab:sm-exercise}
    \end{table}
  \item
    Acote el número de propuestas de matrimonio aceptadas,
    como menciona la demostración
    del teorema~\ref{theo:stable-marriage-exists}.
  \item
    Demuestre que a lo más un hombre
    recibe su última elección con el algoritmo dado.
    En consecuencia,
    si hay una asignación estable en la cual varios hombres
    se deben conformar con sus últimas preferencias,
    hay varias soluciones.
  \item
    El algoritmo esbozado en el teorema~\ref{theo:stable-marriage-exists}
    no especifica el orden en que los hombres se proponen.
    Demuestre que cualquiera sea este orden,
    la asignación resultante es la misma.
  \item
    Demuestre que el algoritmo
    como esbozado en teorema~\ref{theo:stable-marriage-exists}
    a cada mujer le asigna la peor de sus preferencias
    entre todas las soluciones estables.
  \item
    Demuestre que el algoritmo de llenado de cupos en universidades
    cumple lo enunciado.
  \end{enumerate}

% To do:
% - Related problems: Roommate
% - Other algorithms cited by Knuth
% - Complete bibliography

\bibliography{../referencias}

%%% Local Variables:
%%% mode: latex
%%% TeX-master: ../INF-221_clases
%%% ispell-local-dictionary: "spanish"
%%% End:

% LocalWords:  english stable marriage problem deseabilidad
% LocalWords:  intertanto

\bibliographystyle{babplain-fl}

\chapter{Algoritmos voraces}
\label{cha:greedy-algorithm}

  Muchos problemas involucran optimización combinatoria:
  buscamos una configuración óptima de algún objeto discreto.
  En este capítulo plantearemos una técnica general simple y muy atractiva,
  elegir \textquote{la mejor opción local}
  y nunca reconsiderar elecciones previas.
  Esto se conoce como \emph{algoritmos voraces}
  (en inglés,
   \emph{\foreignlanguage{english}{greedy algorithms}};
   una traducción más precisa sería \emph{algoritmos ávidos}
   o \emph{algoritmos codiciosos},
   pero no suena tan bien;
   posiblemente una mejor descripción sería \emph{algoritmos oportunistas}).

  Un ejemplo es un problema de programación de tareas
  (\emph{\foreignlanguage{english}{scheduling}} en inglés).
  \begin{example}
    \label{07::EjemploAlma}
    Supongamos que usted está a cargo de programar observaciones en ALMA.
    Para justificar el gasto de este enorme recurso,
    su misión es programar el máximo número de observaciones.
    Las observaciones tienen instante de inicio y duración,
    y no pueden traslapar.

    Formalmente,
    tenemos un conjunto \(P = \{p_1, p_2, \ldots, p_n\}\)
    de observaciones propuestas,
    la observación \(i\) tiene duración
    el intervalo abierto \(\left[ s_i, s_i+\ell _i \right)\).
    Se pide elegir el subconjunto \(\Pi \subseteq P\)
    tales que los elementos de \(\Pi\) sean disjuntos
    y el número de elementos de \(\Pi\) sea máximo
    (figura~\ref{07::Tareas1}).
    \newcommand{\lineaTarea}[4]{ % (x, y, l, label)
                  \draw (#1, 0.15+#2) -- (#1, -0.15+#2) -- (#1, #2)
                          -- (#1+#3, #2) -- (#1+#3, 0.15+#2)
                          -- (#1+#3, -0.15+#2);
                  \node at (#1 - 0.3, #2) () {\(#4\)};
               }
    \begin{figure}[ht]
      \centering
      \begin{tikzpicture}
        \lineaTarea{0}{0}{2}{p_1};
        \lineaTarea{1.8}{-0.5}{0.9}{p_2};
        \lineaTarea{2.6}{-1}{2}{p_3};
      \end{tikzpicture}
      \caption{Intervalos de tiempo que dura cada proyecto/tarea.}
      \label{07::Tareas1}
    \end{figure}
    Básicamente,
    estamos suponiendo que el observatorio
    se arrienda por observación y no por tiempo.
    ¿Cómo hacerlo?
    Algunas posibilidades:
    \begin{description}
    \item[Sugerencia 1:]
      Repetidamente elegir la tarea más corta
      que no entra en conflicto con las ya programadas.
      La figura~\ref{07::Tareas2}
      \begin{figure}[ht]
        \centering
        \begin{tikzpicture}
          \lineaTarea{0}{0}{2}{p_1};
          \lineaTarea{1.9}{-0.5}{0.8}{p_2};
          \lineaTarea{2.6}{-1}{2}{p_3};
        \end{tikzpicture}
        \caption{Empezando con \(p_2\),
                 efectuamos \num{1} tarea.
                 Con \(p_1\) completamos \num{2}.}
        \label{07::Tareas2}
      \end{figure}
      muestra un contra-ejemplo.
      Hay tres tareas,
      la segunda,
      corta,
      comienza luego de comenzada la primera y antes que ésta termine,
      y a su vez comienza la tercera antes que termine la segunda,
      pero luego del fin de la primera.
      Usando este criterio
      elegiríamos la tarea \(p_2\),
      dejando fuera
      (por conflictos)
      a \(p_1\) y \(p_3\).
      La solución óptima es elegir \(p_1, p_3\).
      Por lo tanto,
      esta sugerencia no siempre da un óptimo.
    \item[Sugerencia 2:]
      Elegir la tarea con inicio más temprano
      que no crea conflicto con las ya programadas.
      La figura~\ref{07::Tareas3}
      \begin{figure}[ht]
        \centering
        \begin{tikzpicture}
          \lineaTarea{0}{0}{5}{p_1};
          \lineaTarea{1}{-0.5}{1}{p_2};
          \lineaTarea{3}{-0.5}{1}{p_3};
        \end{tikzpicture}
        \caption{Empezando con \(p_1\), completa \num{1} tarea.
                 Con \(p_2\) y \(p_3\) hacemos \num{2}.}
        \label{07::Tareas3}
      \end{figure}
      muestra un contra-ejemplo,
      con una tarea larga que comienza temprano
      que traslapa con varias tareas cortas independientes.
      elegiríamos la tarea \(p_1\),
      sin embargo,
      el óptimo es \(p_2, p_3\).
      Nuevamente,
      esta sugerencia no siempre da un óptimo.
    \item[Sugerencia 3:]
      Marcar cada proyecto con el número de proyectos
      con que entra en conflicto,
      programar en orden creciente de conflictos.
      La figura~\ref{07::Tareas4}
      \begin{figure}[ht]
        \centering
        \begin{tikzpicture}
          \lineaTarea{0}{0}{1.7}{p_1};
          \lineaTarea{2.5}{0}{1.7}{p_2};
          \lineaTarea{5}{0}{1.7}{p_3};
          \lineaTarea{7.5}{0}{1.7}{p_4};
          \lineaTarea{1.5}{-0.5}{1.5}{p_5};
          \lineaTarea{1.5}{-2*0.5}{1.5}{p_6};
          \lineaTarea{1.5}{-3*0.5}{1.5}{p_7};
          \lineaTarea{4}{-0.5}{1.5}{p_8};
          \lineaTarea{6.5}{-0.5}{1.5}{p_9};
          \lineaTarea{6.5}{-2*0.5}{1.5}{p_{10}};
          \lineaTarea{6.5}{-3*0.5}{1.5}{p_{11}};
        \end{tikzpicture}
        \caption{Elije en orden \(p_8\), \(p_1\)
                 y \(p_4\), un total de \num{3} tareas;
                 pero \(p_1, p_2, p_3, p_4\) son \num{4}.
                 Me echaron a perder el día.}
        \label{07::Tareas4}
      \end{figure}
      muestra un contra-ejemplo,
      con cuatro tareas que se pueden ejecutar en secuencia,
      pero que traslapan con muchas otras tareas menores
      (\(p_1\) y \(p_2\) traslapan con \(p_5\) a \(p_7\),
       \(p_2\) y \(p_3\) traslapan con \(p_8\),
       \(p_3\) y \(p_4\) traslapan con \(p_9\) a \(p_{11}\)).
      El óptimo es cuatro tareas,
      \(p_1, p_2, p_3, p_4\);
      esta estrategia elige tres tareas
      (\(p_8\),
       que solo traslapa con \(p_2\) y \(p_3\),
       y dos que no interfieren con ella).
    \end{description}
  \end{example}
  ¿Estructura común de las propuestas?
  \begin{itemize}
  \item
    Elija elementos sucesivamente hasta que no queden opciones viables.
  \item
    Entre las opciones visibles en cada paso,
    elija la que minimiza (maximiza) alguna propiedad.
  \end{itemize}
  Es importante destacar que no siempre hay un algoritmo voraz
  que encuentra una solución óptima del problema,
  pero sí pueden ofrecer una aproximación bastante buena.

\section{Comprobar que un algoritmo da un óptimo}

  Volvamos al ejemplo~\ref{07::EjemploAlma}.
  En ese caso,
  un criterio a usar para encontrar una solución óptima
  consta en elegir la tarea que \emph{finaliza} más temprano
  y no entra en conflictos con las ya elegidas.
  Para el proyecto \(i\),
  el instante de \emph{fin} es
  \begin{equation*}
    f_i
      = s_i + \ell _i
  \end{equation*}
  ¿Es óptima la programación que esto construye?

   Llamemos \(P\) a un conjunto de tareas,
  una solución \(\Pi\) es un subconjunto de \(P\),
  la solución \(\Pi\) es viable si no incluye tareas que se traslapan.
  Buscamos \(\Pi\) viable de tamaño máximo.
  Llamaremos \(p\) a una tarea.

  Una forma de comprobar que el algoritmo voraz retorna un óptimo
  es demostrar las siguientes propiedades:
  \begin{description}
  \item[Elección Voraz
        (\emph{\foreignlanguage{english}{greedy choice}}):]
    Para toda instancia \(P\), hay una solución óptima
    que incluye el primer elemento \(\widehat p\) elegido.
  \item[Estructura Inductiva
        (\emph{\foreignlanguage{english}{inductive structure}}):]
    Dada la elección voraz \(\widehat{p}\),
    queda un subproblema menor \(P'\)
    tal que si \(\Pi'\) es solución viable de \(P'\),
    \(\{ \widehat{p} \} \cup \Pi'\) es solución viable de \(P\)
    (\(P'\) no tiene \textquote{restricciones externas},
     parte de lo que se hace al definir \(P'\)
     es precisamente asegurar esto).
  \item[Subestructura Óptima
        (\emph{\foreignlanguage{english}{optimal substructure}}):]
    Si \(P'\) queda de \(P\) al sacar \(\widehat p\),
    y  \(\Pi'\) es óptima para \(P'\),
    \(\Pi' \cup \{ \widehat{p} \}\) es óptima para \(P\).
  \end{description}
  Con estos tres podemos demostrar
  que la secuencia de elecciones de \(\widehat{p}\)
  da una solución óptima por inducción sobre los pasos.
  El esqueleto de la demostración es el siguiente:
  \begin{theorem}
    \label{theo:esquema-voraz}
    Si un algoritmo cumple con Elección Voraz, Estructura Inductiva
    y Subestructura Óptima,
    entrega una solución óptima al problema.
  \end{theorem}
  \begin{proof}
    Por inducción sobre el tamaño del problema.
    \begin{description}
    \item[Base:]
      Si \(\lvert P \rvert = 1\),
      por Elección Voraz se elige el único \(p\) posible,
      que claramente es óptimo.
    \item[Inducción:]
      Supongamos que el algoritmo voraz da una solución óptima
      para todos los problemas hasta tamaño \(k\),
      y consideremos la instancia \(P\) de tamaño \(k + 1\).
      Elegimos \(\widehat{p}\) por el criterio voraz,
      sabemos que hay una solución óptima que incluye \(\widehat{p}\)
      por Elección Voraz.
      Sea \(P'\) el problema
      que resulta al eliminar \(\widehat{p}\) de \(P\),
      junto con sus dependencias.
      Es claro que \(\lvert P' \rvert \le k\),
      sea \(\Pi'\) la solución dada por el algoritmo voraz a \(P'\).
      Por inducción,
      \(\Pi'\) es óptima para \(P'\).
      Por Estructura Inductiva,
      \(\Pi' \cup \{ \widehat{p} \}\) es viable para \(P\);
      por Subestructura Óptima,
      \(\Pi' \cup \{ \widehat{p} \}\) es óptima para \(P\).
    \end{description}
  \end{proof}

\subsection{Demostrando que un algoritmo voraz da un óptimo}

  Volvamos al ejemplo~\ref{07::EjemploAlma}.
  Si quisiéramos demostrar que la sugerencia entrega un óptimo
  solo tenemos que demostrar que cumple las propiedades enunciadas:
  \begin{description}
  \item[Elección Voraz:]
    Hay una solución óptima que incluye la elección voraz \(\widehat{p}\).
    \begin{proof}
      Sea \(\widehat{p}\) la primera observación elegida
      y \(\Pi^*\) una solución óptima para \(P\).
      Si \(\widehat{p} \in \Pi^*\),
      estamos listos.
      En caso contrario,
      sea \(\Pi'\) la solución obtenida
      reemplazando la observación más temprana de \(\Pi^*\) por \(\widehat p\).
      Esto no produce nuevos conflictos,
      ya que la primera observación de \(P^*\)
      no termina antes de \(\widehat{p}\)
      por cómo fue elegida esta,
      y \(\lvert \Pi^* \rvert = \lvert \Pi' \rvert\),
      ambos son óptimos.
    \end{proof}
  \item[Estructura Inductiva:]
    El elegir \(\widehat p\) nos deja un problema \(P'\)
    sin restricciones externas.
    \begin{proof}
      El problema \(P \smallsetminus \{ \widehat{p} \}\)
      incluye observaciones en conflicto con \(\widehat{p}\),
      hay soluciones viables para este que no son viables con \(\widehat{p}\).
      Hay restricciones externas.

      Eliminar también las observaciones con conflictos con \(\widehat{p}\)
      deja un problema \(P'\) sin restricciones externas.
      Toda solución viable para el problema resultante
      puede combinarse con \(\widehat{p}\).

      Como comentamos,
      estructura inductiva realmente indica cómo debemos construir
      subproblemas que resultan de elegir \(\widehat{p}\).
    \end{proof}
   \item[Subestructura Óptima:]
     Si \(P'\) queda después de elegir \(\widehat p\),
     y \(\Pi'\) es óptima para \(P'\),
     entonces \(\Pi'\cup \{\widehat p\}\) es óptima para \(P\).
     \begin{proof}
       Sea \(\Pi'\) como dado.
       Entonces \(\Pi' \cup \{\widehat p\}\) es viable para \(P\)
       (por Estructura Inductiva),
       y \(\lvert \Pi' \cup \{ \widehat p\ \} \rvert
              = \lvert \Pi' \rvert + 1\).
       Sea \(\Pi^*\) una solución óptima para \(P\)
       que contiene \(\widehat{p}\)
       (existe por Elección Voraz).
       Entonces \(\Pi^* \smallsetminus \{ \widehat{p} \}\)
       es una solución óptima para \(P'\)
       (si hubiese una mayor,
        combinada con \(\widehat{p}\) daría una solución mayor que \(\Pi^*\)
        para \(P\)).
       Pero entonces:
       \begin{align*}
         \lvert \Pi' \rvert
           &= \lvert \Pi^* \smallsetminus \{ \widehat{p} \} \rvert \\
           &= \lvert \Pi^* \rvert - 1 \\
        \intertext{O sea:}
         \lvert \Pi'\cup \{\widehat{p} \} \rvert
           &= \lvert \Pi^* \rvert
       \end{align*}
       y \(\Pi' \cup \{\widehat{p}\}\) es óptima.
     \end{proof}
   \end{description}

\section{Problema de Asignación de Tareas}

  Demostraremos formalmente que nuestro algoritmo voraz
  entrega una solución óptima al problema de asignación de tareas,
  usando las tres propiedades
  (Elección Voraz,
   Estructura Inductiva
   y Subestructura Óptima).
  \begin{theorem}
    Para el problema de programación de tareas,
    la estrategia de elegir en cada paso la tarea sin conflicto
    con fin más temprano entrega una solución óptima.
  \end{theorem}
  \begin{proof}
    Por inducción sobre \(\lvert P \rvert\),
    el número de tareas.
    \begin{description}
    \item[Base:]
      Si hay una única tarea,
      la estrategia la programa.
      Esto es óptimo.
    \item[Inducción:]
      Supongamos que obtiene una solución óptima para a lo más \(k\) tareas.
      Sea \(P\) una instancia con \(\lvert P \rvert = k + 1\).
      Elegimos \(\widehat{p}\) según criterio voraz.
      Por Elección Voraz hay una solución óptima que lo incluye;
      al eliminar la tarea \(\widehat{p}\)
      con las tareas con las que interfiere
      queda un problema \(P'\),
      claramente \(\lvert P' \rvert \le k\).

      Por inducción,
      obtenemos una solución óptima \(\Pi'\) de \(P'\).
      Por Estructura Inductiva \(\Pi'\) junto a \(\widehat p\)
      es una solución viable para \(P\)
      Esta es una solución óptima para \(P\) por Subestructura Óptima.
    \end{description}
  \end{proof}
  La demostración no depende realmente de este problema,
  lo que demuestra que siempre que se cumplan las tres propiedades
  obtendremos una solución óptima.
  De ahora en adelante nos contentaremos con demostrar las tres propiedades,
  sabiendo que podemos usar el mismo esquema para completar la demostración
  de que el algoritmo voraz entrega un óptimo.

\section{Knapsack (mochila)}
\label{sec:fractional-knapsack}

  Hay una mochila de capacidad \(M\),
  y un conjunto de \(n\) tipos de objetos,
  del objeto tipo \(i\) hay disponible \(p_i\) en total,
  de valor total \(v_i\).
  Se pueden incluir fracciones de objetos
  (es café, azúcar, arroz, \ldots)

  Estrategia:
  \begin{itemize}
  \item
    Ordenar los objetos por
    \begin{equation*}
      \frac{v_i}{p_i}
    \end{equation*}
    decreciente.
  \item
    Echar en la mochila sucesivamente todo lo que se pueda del objeto \(i\),
    en el orden anterior.
  \end{itemize}

  Falta demostrar que esta estrategia da una solución óptima,
  lo que quedará de ejercicio.

  Una variante obvia es objetos discretos:
  el objeto \(i\) se agrega completo o no
  (no fracciones).
  En este caso la estrategia voraz \emph{no} da óptimo.
  Incluso vimos que este problema es \NP\nobreakdash-completo.
  Construir un contraejemplo para la estrategia indicada queda de ejercicio.

\section{Árbol recubridor mínimo}
\label{sec:MST}

  Dado un grafo \(G = (V, E)\),
  con arcos rotulados \(c \colon E \to \mathbb{R}^+\),
  se busca el árbol recubridor
  (o sea,
   el que une todos los vértices)
  de costo mínimo
  (suma de los \(c\) sobre sus arcos).
  En inglés se le conoce
  como \emph{\foreignlanguage{english}{minimal spanning tree}},
  y se abrevia MST.
  Este problema tiene una distinguida historia.
  Nešetřil y Nešetřilová~%
    \cite{nesetril12:_origin_mst_algorithms}
  discuten las primeras soluciones,
  Mareš~%
    \cite{mares08:_saga_mst}
  discute desarrollos más recientes.

  Para el grafo \(G = (V, E)\)
  usamos la notación \(V = G_V\), \(E = G_E\).
  En nuestros algoritmos a continuación al iterar sobre un conjunto,
  lo haremos en el orden dado antes.
  Dos algoritmos alternativos para resolver este problema
  son el algoritmo de Prim
  (algoritmo~\ref{alg:prim},
   en realidad de Jarník~%
     \cite{jarnik30:_MST},
   redescubierto por Prim~%
     \cite{prim57:_MST}
   y Dijkstra~%
     \cite{dijkstra59:_MST}).
  \begin{algorithm}
    \DontPrintSemicolon\Indp

    \Procedure{\(\mathrm{Prim}(G)\)}{
      Sort \(G_E\) by increasing \(c(e)\) \;
      \BlankLine \;
      Select a vertex \(u \in G_V\) \;
      \(T \gets (\{ u \}, \varnothing)\) \;
      \For{\(u v \in G_E\) such that \(u \in T_V\), \(v \notin T_V\)}{
        \(T \gets (T_V \cup \{ v \}, T_E \cup \{ u v \})\) \;
      }
      \Return \(T\) \;
    }
    \caption{Algoritmo de Prim}
    \label{alg:prim}
  \end{algorithm}
  y el algoritmo de Kruskal~%
   \cite{kruskal56:_MST}
  (algoritmo~\ref{alg:kruskal}).
  \begin{algorithm}
    \DontPrintSemicolon\Indp

    \Procedure{\(\mathrm{Kruskal}(G)\)}{
      Sort \(G_E\) by increasing \(c(e)\) \;
      \BlankLine \;
      \(T \gets (\varnothing, \varnothing)\) \;
      \For{\(u v \in G_E\)}{
        \If{\(u v\) doesn't form a cycle in \(T\)}{
          \(T \gets (T_V \cup \{ u, v \},
                          T_E \cup \{ u v \})\) \;
        }
      }
      \Return \(T\) \;
    }
    \caption{Algoritmo de Kruskal}
    \label{alg:kruskal}
  \end{algorithm}
  Ambos son algoritmos voraces,
  como puede apreciarse,
  aunque usan criterios diferentes
  para seleccionar el siguiente arco a incluir.

  La demostración de ambos se basa en:
  \begin{proposition}
    \label{prop:base-MST}
    Sea \(G = (V, E)\) un grafo como indicado,
    y sea \(V_1, V_2\) una partición de \(V\).
    Si el arco \(v_1 v_2\) tiene costo mínimo entre los arcos
    entre \(V_1\) y \(V_2\),
    hay un árbol recubridor mínimo de \(G\) que incluye \(v_1 v_2\).
  \end{proposition}
  \begin{proof}
    Sea el grafo \(G = (V, E)\) con costos \(c \colon E \to \mathbb{R}\).
    Sea \(V_1, V_2\) una partición de \(V\),
    y \(e = v_1 v_2\) un arco de costo mínimo
    con \(v_1 \in V_1\) y \(v_2 \in V_2\).
    Consideremos un árbol recubridor de costo mínimo \(T\),
    cuyo costo anotamos \(c(T)\).
    Si \(e \in T\),
    estamos listos.
    En caso contrario,
    el grafo \(T \cup e\) tiene un único ciclo,
    que debe incluir otros arcos entre \(V_1\) y \(V_2\).
    Sea \(e' = v_1' v_2'\) con \(v_1' \in V_1\) y \(v_2' \in V_2\)
    un arco del ciclo de costo mínimo.
    Como \(e\) es mínimo entre \(V_1\) y \(V_2\),
    \(c(e') \ge c(e)\).
    Intercambiando \(e\) con \(e'\),
    obtenemos un árbol recubridor \(T'\)
    de costo total:
    \begin{equation*}
      c(T') = c(T) - c(e') + c(e) \ge c(T)
    \end{equation*}
    Como \(T\) es de costo mínimo,
    esto debe cumplirse con igualdad,
    esto a su vez significa \(c(e) = c(e')\),
    el árbol recubridor~\(T'\) que incluye a~\(e'\) también es mínimo.
  \end{proof}

  Este es un problema muy importante,
  hay una variedad de algoritmos mejores que los planteados,
  como el de Chazelle~%
   \cite{chazelle00:_soft_heap,chazelle00:_minimal_spanning_tree}.
  Karger, Klein y Tarjan~%
    \cite{karger95:_random_linear_time_MST}
  describen un algoritmo aleatorizado
  (ver el capítulo~\ref{cha:randomized-algorithms})
  con tiempo de ejecución esperado lineal.
  Curiosamente,
  todos ellos usan de alguna forma
  el primer algoritmo publicado para resolver este problema,
  el de Borůvka~%
    \cite{boruvka26:_problemu_minimalnim}.
  Una revisión tutorial relativamente reciente es la de Eisner~%
    \cite{eisner97:_state_art_algor_minim_spann_trees},
  Mareš~%
    \cite{mares08:_graph_algorithms}
  discute los algoritmos en detalle.

\section{Programar tareas con plazo fatal}
\label{sec:tareas-plazo-fatal}

  Hay una colección de tareas,
  cada una de las cuales requiere ejecutarse
  por una unidad de tiempo en la única máquina disponible.
  La tarea~\(i\) trae ganancia \(g_i\)
  si se completa antes de su plazo fatal~\(d_i\),
  en caso contrario no aporta nada.
  Se busca la secuencia de tareas a programar
  de forma de obtener la máxima ganancia.

  Como cada tarea demanda una unidad de tiempo,
  podemos considerar el tiempo dividido en ranuras,
  que pueden estar libres u ocupadas.
  Es claro que debemos ver de programar las tareas que más ganancia traen,
  pero de forma que interfieran lo menos posible con otras tareas
  de menor ganancia.
  Vale decir,
  programar las tareas en la ranura libre más tardía
  antes de su plazo fatal.
  Esto sugiere ordenar las tareas por ganancia decreciente,
  y si hay empate en ganancia por plazo fatal decreciente;
  luego asignar cada tarea a la ranura libre más tardía
  en la cual aún cumple su plazo fatal o descartarla.
  Si la programación resultante tiene tiempos muertos,
  podemos compactar al final
  adelantando tareas.
  El tiempo del algoritmo resultante está dominado por el ordenamiento,
  si hay \(n\)~tareas es~\(O(n \log n)\).

  Para demostrar que esto da una solución óptima,
  recurrimos a nuestras tres propiedades:
  \begin{description}
  \item[Greedy Choice:]
    Nuestro algoritmo elige la tarea \(\hat{t}\) que más ganancia da
    de entre las que aún pueden cumplir su plazo fatal.
    Demostramos por contradicción
    que hay una solución óptima que incluye la tarea \(\hat{t}\)
    así elegida.

    Tomemos una solución óptima.
    Es claro que las ranuras antes del plazo fatal de \(\hat{t}\)
    están todas ocupadas,
    ya que en caso contrario podemos programar \(\hat{t}\) en una libre,
    contradiciendo que la solución es óptima.

    Si la solución óptima incluye a \(\hat{t}\),
    estamos listos.
    Si no la incluye,
    podemos tomar una tarea
    que termina antes del plazo fatal de \(\hat{t}\).
    Si su ganancia es menor que la ganancia de \(\hat{t}\),
    intercambiándola con \(\hat{t}\) mejoramos la ganancia total,
    contradicción con que la solución sea óptima.
    La única posibilidad es que tenga la misma ganancia de \(\hat{t}\),
    podemos intercambiarlas obteniendo una solución óptima
    que incluye a \(\hat{t}\).
  \item[Inductive Substructure:]
    Sea \(P\) el problema original,
    \(\hat{t}\) la tarea elegida por el criterio voraz,
    y el problema \(P'\) lo que queda
    al asignar \(\hat{t}\) a la última ranura libre
    antes de su plazo fatal.
    Una solución viable a \(P'\),
    o sea,
    una colección de tareas a programar entre las restantes,
    nunca puede entrar en conflicto con la programación de \(\hat{t}\);
    podemos combinar una solución a \(P'\) con \(\hat{t}\)
    para dar una solución viable a \(P\).

    En realidad,
    en este caso no hay restricciones \textquote{cruzadas},
    esto se cumple automáticamente al eliminar tareas
    que ya no pueden cumplir sus plazos fatales.
  \item[Optimal Substructure:]
    Consideremos una solución óptima \(\Pi^*\) al problema \(P\).
    Para simplificar notación,
    llamemos \(\lvert \Pi \rvert\)
    a la ganancia de la solución viable \(\Pi\)
    al problema \(P\),
    y similarmente \(\lvert t \rvert\)
    la ganancia que reporta la tarea \(t\).

    Por \textbf{Greedy Choice},
    podemos suponer sin pérdida de generalidad
    que \(\Pi^*\) incluye la elección voraz \(\hat{t}\).
    Sea \(P'\) el problema que queda al eliminar \(\hat{t}\)
    y las tareas que ya no pueden completarse.
    Sea \(\Pi'\) una solución óptima para \(P'\),
    por \textbf{Inductive Substructure} es compatible con \(\hat{t}\);
    la ganancia de esa solución es:
    \begin{equation*}
      \lvert \Pi' \rvert
        \le \lvert \Pi^* \rvert
               - \lvert \hat{t} \rvert
    \end{equation*}
    (si fuera mayor,
     junto con \(\hat{t}\) daría una solución mejor que la óptima).
    Pero la solución \(\Pi^* \smallsetminus \{ \hat{t} \}\)
    da el valor \(\lvert \Pi^* \rvert - \lvert \hat{t} \rvert\),
    y la combinación \(\Pi'\) con \(\hat{t}\) es óptima.
  \end{description}
  Como se cumplen las tres propiedades,
  el algoritmo da una solución óptima.

\section{Otras técnicas para demostrar correctitud}
\label{sec:greedy-otras-tecnicas}

  La técnica expuesta para demostrar que el algoritmo voraz entrega un óptimo
  (basada en las tres propiedades)
  es bastante general,
  pero no siempre es aplicable
  ni la manera más natural de enfrentar el problema.

\subsection{Demostración por contradicción}
\label{sec:demostr-por-contradiccion}

  Notar las diferencias entre este caso
  y la demostración por contradicción
  para demostrar que se cumple la propiedad de elección voraz.

  Consideremos el problema de ordenar archivos en forma óptima en una cinta,
  donde el largo del archivo \(i\) es \(l_i\).
  Los usuarios solicitan el archivo \(i\) con probabilidad \(p_i\),
  y el costo de extraer un archivo
  (que llamaremos \(L_i\))
  es proporcional a la suma de los largos de los archivos que lo preceden
  y de ese mismo.
  Como el tiempo es proporcional al largo leído,
  usaremos largos como medidas de tiempo.
  Interesa minimizar el valor esperado del tiempo para extraer archivos,
  determinando el orden de los archivos en la cinta:
  \begin{equation*}
    T
      = \sum_i p_i L_i
  \end{equation*}

  Usamos el algoritmo voraz de ordenar los archivos en la cinta
  en orden creciente de \(l_i / p_i\).
  Para \(n\) archivos
  este orden se puede determinar en tiempo \(O(n \log n)\),
  el costo de ordenar domina.

  Demostramos que esto es óptimo por contradicción.
  Supongamos que un orden diferente da una solución mejor.
  Eso quiere decir que hay archivos vecinos \((a, b)\)
  tales que:
  \begin{equation*}
    \frac{l_a}{p_a}
      > \frac{l_b}{p_b}
  \end{equation*}
  pero \(a\) se almacena antes de \(b\)
  (si no se cumpliera,
   estarían en el orden que da el algoritmo voraz).
  Demostraremos que intercambiándolos mejora \(T\),
  este orden no puede ser óptimo.

  Como \(a\) y \(b\) son vecinos,
  intercambiarlos no afecta el tiempo de extracción de ningún otro archivo,
  por lo que \(T\) mejora en:
  \begin{align*}
    p_a l_a + p_b (l_a + l_b)
      - (p_b l_b + p_a (l_a + l_b))
      &= p_b l_a - p_a l_b \\
      &= p_a p_b \left( \frac{l_a}{p_a} - \frac{l_b}{p_b} \right) \\
      &> 0
  \end{align*}
  Vea la figura~\ref{fig:file-reordering},
  donde marcamos con asterisco los supuestos óptimos.
  \begin{figure}[ht]
    \centering
    \begin{tikzpicture}
      % Before swap
      \draw[thick] (0, 4) rectangle (10, 4.5);
      \draw[thick] (4,	 4) rectangle (5.5, 4.5)
         node [pos = 0.5] {\(a\)};
      \draw[thick] (5.5, 4) rectangle (6.5, 4.5)
         node [pos = 0.5] {\(b\)};

      \draw (0,	  3.0 - 0.125) -- (0,	4);
      \draw (5.5, 3.5 - 0.125) -- (5.5, 4);
      \draw (6.5, 3.0 - 0.125) -- (6.5, 4);

      \draw[latex'-latex'] (0, 3.5) -- (5.5, 3.5)
        node [pos = 0.5, fill = white] {\(L_a^*\)};
      \draw[latex'-latex'] (0, 3.0) -- (6.5, 3.0)
        node [pos = 0.5, fill = white] {\(L_b^*\)};

      % After swap
      \draw[thick] (0, 1.5) rectangle (10, 2.0);
      \draw[thick] (4, 1.5) rectangle (5.0, 2)
         node [pos = 0.5] {\(b\)};
      \draw[thick] (5, 1.5) rectangle (6.5, 2)
         node [pos = 0.5] {\(a\)};

      \draw (0,	  0.5 - 0.125) -- (0,	1.5);
      \draw (5.0, 1.0 - 0.125) -- (5.0, 1.5);
      \draw (6.5, 0.5 - 0.125) -- (6.5, 1.5);

      \draw[latex'-latex'] (0, 1) -- (5.0, 1)
        node [pos = 0.5, fill = white] {\(L_b\)};
      \draw[latex'-latex'] (0, 0.5) -- (6.5, 0.5)
        node [pos = 0.5, fill = white] {\(L_a\)};
    \end{tikzpicture}
    \caption{Resultado de intercambiar los archivos \(a\) y \(b\)}
    \label{fig:file-reordering}
  \end{figure}
  Pero \(p_a p_b > 0\),
  ya que son probabilidades;
  y supusimos que \(l_a / p_a > l_b / p_b\).
  Al intercambiarlos,
  disminuye el tiempo promedio.
  Esto contradice el que haya sido óptimo antes del cambio.

  Un caso particular importante de este problema
  se da al programar tareas que deben ejecutarse una tras otra,
  donde nos interesa minimizar el tiempo promedio de término de las tareas.
  Es el mismo problema anterior,
  si consideramos que las probabilidades de acceso son todas iguales.
  Nuestro resultado indica ejecutar las tareas en orden de duración creciente
  (lo que llaman \emph{\foreignlanguage{english}{Shortest Job First}},
   o SJF).

\subsection{Usando un invariante}
\label{sec:greedy-invariante}

  Un caso claro donde el esquema general discutido no funciona
  es el problema de hallar los caminos más cortos
  a todos los vértices de un grafo
  desde un vértice dado.
  Nuestro problema es un grafo \(G = (V, E)\)
  con arcos rotulados \(w(u,v) \colon V \times V \to \mathbb{R}^+\),
  y el costo del camino \(p\) es:
  \begin{equation*}
    w(p)
      = \sum_{(u, v) \in p} w(u, v)
  \end{equation*}
  Dado un vértice origen \(s\),
  nos interesan los costos mínimos de los caminos
  a cada vértice \(v \in V \smallsetminus \{s\}\).
  Llamaremos \(\delta(v)\) al costo de tal camino.

  El algoritmo de Dijkstra~%
    \cite{dijkstra59:_MST}
  es un algoritmo voraz
  que resuelve el problema si no hay arcos de peso negativo.
  Funciona como sigue:
  mantiene una partición de los vértices,
  \(S\) y \(V \smallsetminus S\).
  En cada instante,
  \(S\) es el conjunto de vértices
  a los cuales ya se conocen los caminos más cortos.
  Inicialmente \(S = \varnothing\).
  Para cada vértice \(v \in V\) tenemos una variable
  \(d[v]\) que es el largo del mejor camino desde \(s\) a \(v\)
  que se ha hallado.
  Los vértices se ubican en una cola de prioridad \(Q\),
  con prioridad \(d[v]\).
  El algoritmo~\ref{alg:Dijkstra} describe esto informalmente.
  \begin{algorithm}[ht]
    \DontPrintSemicolon\Indp

    \ForEach{\(v \in V\)}{
      \(d[v] \gets \infty\) \;
    }
    \(d[s] \gets 0\) \;
    Initialize \(Q\) empty \;
    \ForEach{\(v \in V\)}{
      Insert \(v\) in \(Q\) with priority \(d[v]\) \;
    }
    \(S \gets \varnothing\) \;
    \BlankLine
    \While{\(Q\) not empty}{
      \(u \gets \mathrm{DeleteMin}(Q)\) \;
      \(S \gets S \cup \{ u \}\) \;
      \ForEach{\(v in N(u)\)}{
        \If{\(d[v] > d[u] + w(u, v)\)}{
          \(d[v] \gets d[u] + w(u, v)\) \;
          Update key of \(v\) in \(Q\) to \(d[v]\) \;
        }
      }
    }
    \caption{Algoritmo de Dijkstra}
    \label{alg:Dijkstra}
  \end{algorithm}
  Demostramos que el algoritmo de Dijkstra es correcto
  demostrando por inducción sobre \(S\)
  que se cumple el invariante:
  \begin{equation}
    \label{eq:Dijkstra-invariante}
    \forall v \in S, d[v] = \delta[v]
  \end{equation}
  \begin{description}
  \item[Base:]
    Luego de la primera iteración tenemos \(S = \{ s \}\),
    donde es \(d[s] = \delta(s) = 0\),
    el invariante vale.
  \item[Inducción:]
    Supongamos que cuando \(\lvert S \rvert = k\)
    el invariante se cumple.
    Sea \(v\) el siguiente vértice extraído de \(Q\)
    (y colocado en \(S\)),
    y sea \(p\) un camino de \(s\) a \(v\) de costo \(d[v]\)
    (claramente existe,
     y el algoritmo en un uso real registrará tal camino con el vértice).
    Sea \(u\) el vértice inmediatamente predecesor de \(v\) en \(p\).
    Entonces \(u \in S\),
    y \(d[u] = \delta[u]\) por inducción.

    Demostraremos por contradicción
    que \(p\) es un camino de costo mínimo de \(s\) a \(v\).
    Supongamos que hay un camino \(p^*\) de \(s\) a \(v\)
    tal que \(w(p^*) = \delta(v) < w(p)\).
    Como \(p^*\) conecta al vértice \(s \in S\)
    con el vértice \(v \in V \smallsetminus S\),
    debe haber un primer arco \(a b \in p^*\)
    con \(a \in S\) y \(b \in V \smallsetminus S\).
    Podemos dividir el camino en \(p_1, p_2\),
    con \(p_1\) de \(s\) a \(a\) y \(p_2\) de \(b\) a \(v\).
    Por inducción,
    \(d[a] = \delta[a]\).
    Como \(p^*\) es un camino más corto,
    \(p_1, b\) es un camino más corto de \(s\) a \(b\)
    (si hubiera uno más corto,
     \(p^*\) no sería óptimo).
    Después de agregar \(a\) a \(S\)
    se consideró el arco \(a b\),
    con lo que después de actualizarlo \(d[b] = \delta[b]\).
    Como \(v\) se agregó a \(S\) mientras \(b\) estaba en \(Q\),
    es \(\delta[v] \le d[b]\).
    Como los pesos son no negativos,
    \(\delta[v] = w(p^*) \ge d[b]\).
    En conjunto con \(d[v] \le d[b]\)
    resulta \(w(p^*) \ge d[v] = w(p)\),
    contradiciendo que \(w(p^*) < w(p)\).
  \end{description}
  Como el invariante se cumple al principio del algoritmo,
  y se cumple luego de cada paso,
  se cumple al terminar.
  Pero al terminar el algoritmo todos los vértices están en \(S\).
  Como en cada iteración se agrega un vértice a \(S\),
  el algoritmo siempre termina.

  Para acotar su tiempo de ejecución,
  observamos que cada vértice se extrae de la cola \(Q\) una sola vez
  (así se agrega \(v\) a \(S\),
   ya se conoce \(\delta[v]\)),
  para cada vértice se consideran todos sus vecinos.
  O sea,
  el tiempo de ejecución es \(O(\lvert V \rvert \Delta_G)\),
  donde \(\Delta_G\) es el máximo grado de los vértices de \(G\).

\section*{Ejercicios}
\label{sec:ejercicios-08}

  \begin{enumerate}
  \item
    Demuestre en detalle que el algoritmo voraz
    da una solución óptima al problema de la mochila,
    demostrando las tres propiedades.
  \item
    Demuestre la proposición~\ref{prop:base-MST}.
  \item
    Demuestre que el algoritmo de Prim da un árbol recubridor mínimo.
  \item
    Demuestre que el algoritmo de Kruskal da un árbol recubridor mínimo.
  \end{enumerate}

\bibliography{../referencias}

%%% Local Variables:
%%% mode: latex
%%% TeX-master: "../INF-221_notas"
%%% ispell-local-dictionary: "spanish"
%%% End:

% LocalWords:  english greedy algorithms scheduling at choice optimal
% LocalWords:  inductive structure substructure Knapsack recubridor
% LocalWords:  spanning tree MST Sort by increasing Select vertex in
% LocalWords:  such that doesn form cycle aleatorizado tutorial First
% LocalWords:  correctitud Shortest SJF Initialize empty Insert with
% LocalWords:  priority not Update key of to

\bibliographystyle{babplain-fl}

\chapter{Código Huffman}
\label{cha:Huffman}

  El código Huffman~%
    \cite{huffman52:_method_const_minim_redun_codes}
  es una aplicación muy importante de algoritmo voraz.
  Lo desarrolló como estudiante de pregrado,
  cuando se presentó la alternativa de dar un examen final
  o escribir un trabajo sobre la optimalidad de ciertos códigos
  en su ramo de teoría de comunicaciones.
  Vio que no podría resolver el problema planteado,
  y estaba a punto de abandonar cuando se le ocurrió la idea de este algoritmo,
  demostrando que es óptimo.
  En esto le ganó a sus profesores,
  que estaban desarrollando el código Shannon-Fano que resulta no ser óptimo.

  Dado un texto,
  formado por símbolos,
  buscamos codificarlo eficientemente.
  Si cada símbolo se codifica en \(k\) bits
  (total \(2^k\) símbolos posibles),
  de texto de largo \(n\) usa \(n k\) bits.
  En texto,
  las frecuencias son \emph{muy} desiguales.
  Por ejemplo,
  en la novela Moby Dick
  aparece~\num{117\,194} veces la letra~'e',
  y~\num{640} veces la~'z'.
  Nuestro principal objetivo es asignarle códigos cortos
  a los símbolos que más se repiten,
  a costa de codificaciones largas para símbolos poco frecuentes.

  Pero hay que tener cuidado:
  \begin{align*}
    a &\mapsto 0 \\
    b &\mapsto 1 \\
    c &\mapsto 01
  \end{align*}
  Con esta codificación escribimos:
  \begin{equation}
    \label{09::PrimerCodigo}
    a b a b c \rightsquigarrow 010101
  \end{equation}
  Pero también:
  \begin{equation}
    \label{09::SegundoCodigo}
    c c c \rightsquigarrow 010101
  \end{equation}
  ¡Se produce ambigüedad entre~\eqref{09::PrimerCodigo}
  y~\eqref{09::SegundoCodigo}!

  Obviamente nos interesan códigos que tengan decodificación única.
  Condición suficiente para evitar ambigüedades es que
  ningún código sea prefijo de otro
  (\textquote{\emph{\foreignlanguage{english}{prefix-free code}}}
   o \textquote{\emph{\foreignlanguage{english}{prefix code}}}).
  La desigualdad de Kraft\nobreakdash-McMillan~%
    \cite{kraft49:_coding, mcmillan56:_two_inequalities}
  muestra que si un código puede decodificarse en forma única,
  hay un código prefijo con códigos del mismo largo para cada símbolo.
  Esto es importante porque hace eficiente el decodificar.

  Podemos describir un código prefijo binario como un árbol binario,
  las hojas son los símbolos y el camino desde la raíz a la hoja es el código
  (por ejemplo,
   ir al hijo izquierdo es 0,
   ir al derecho 1).
  Dadas las frecuencias con que aparecen los símbolos en las hojas,
  podemos asignar frecuencias a nodos internos del árbol
  como la suma de las frecuencias de todas las hojas descendientes
  (o sea,
   la suma de la frecuencia de los hijos
   en los nodos internos).
  Un ejemplo es la figura~\ref{09::EjemploArbolHuffman},
  el código representado es el del cuadro~\ref{tab:EjemploArbolHuffman}.
  \begin{figure}[ht]
    \centering
    \begin{tikzpicture}[scale = 0.75]
      \coordinate (r)	 at (3, 3);
      \coordinate (0)	 at (0, 2);
      \coordinate (1)	 at (6, 2);
      \coordinate (10)	 at (4, 1);
      \coordinate (100)	 at (3, 0);
      \coordinate (101)	 at (5, 0);
      \coordinate (11)	 at (8, 1);

      \draw (r)	 -- node [above] {\num{0}} (0);
      \draw (r)	 -- node [above] {\num{1}} (1);
      \draw (1)	 -- node [above] {\num{0}} (10);
      \draw (1)	 -- node [above] {\num{1}} (11);
      \draw (10) -- node [above left] {\num{0}} (100);
      \draw (10) -- node [above right] {\num{1}} (101);

      \draw [fill] (r)	 circle [radius = 3pt];
      \draw [fill] (1)	 circle [radius = 3pt];
      \draw [fill] (10)	 circle [radius = 3pt];

      \node [below = 0.1 of 0]	 {\(a\)};
      \node [below = 0.1 of 100] {\(b\)};
      \node [below = 0.1 of 101] {\(d\)};
      \node [below = 0.1 of 11]	 {\(c\)};
    \end{tikzpicture}
    \caption{Ejemplo de código prefijo como árbol}
    \label{09::EjemploArbolHuffman}
  \end{figure}
  \begin{table}[ht]
    \centering
    \begin{tabular}{>{\(}c<{\)}>{\(}l<{\)}}
      \multicolumn{1}{c}{\textbf{Símbolo}} &
        \multicolumn{1}{c}{\textbf{Código}} \\
      \hline
        a & 0	\\
        b & 100 \\
        c & 11	\\
        d & 101
    \end{tabular}
    \caption{Código representado en la figura~\ref{09::EjemploArbolHuffman}}
    \label{tab:EjemploArbolHuffman}
  \end{table}
  Queremos nodos caros
  (alta frecuencia)
  cerca de la raíz,
  nodos baratos
  (bajas frecuencias)
  lejos.
  Una idea es agrupar desde las hojas,
  partiendo con los nodos más baratos,
  considerando igualmente nodos internos resultantes en iteraciones sucesivas.

\section{Descripción del problema}

  Para atacar el problema,
  y finalmente demostrar que el algoritmo esbozado da un óptimo,
  hay que describirlo formalmente.

  Dada una secuencia \(T\) sobre \(\Sigma = \{x_1, \ldots, x_n\}\),
  donde \(x_i\) aparece con frecuencia \(f_i\),
  construir una función de codificación
    \(C \colon \Sigma \rightarrow \text{cadenas de bits}\),
  tal que \(C\) es un código prefijo
  y el número total de bits para representar \(T\) se minimiza.
  \begin{definition}
    La \emph{profundidad} de la hoja \(\ell _i\),
    anotada \(d(\ell _i)\),
    es el largo del camino de la raíz a esa hoja.
  \end{definition}
  En el código descrito por el árbol \(R\)
  el símbolo \(x_i\) queda codificado por \(d(x_i)\) bits,
  el texto completo queda representado por el siguiente número de bits:
  \begin{equation*}
    B(R)
      = \sum_i f_i d(x_i)
  \end{equation*}
  Observamanos que:
  \begin{itemize}
  \item
    Si \(R\) es óptimo,
    todo nodo interno tiene dos hijos.

    Si hubiese un nodo interno con un único hijo,
    podríamos acortar los caminos
    (códigos)
    de descendientes de su hijo
    haciéndolos depender directamente del nodo.
  \item
    Hay dos hojas \(x_a\), \(x_b\)
    a la profundidad máxima que son hermanos.

    Por el punto anterior,
    no pueden haber nodos internos con un único hijo,
    toda hoja tiene un hermano.
  \end{itemize}
  Intuitivamente,
  buscamos letras poco frecuentes a altas profundidades,
  frecuentes a profundidades bajas.
  Lo que hace el algoritmo de Huffman
  es asignar desde los símbolos menos frecuentes,
  agrupando colecciones de símbolos.
  Sea \(L = (\ell_1, \ldots, \ell_n)\) el conjunto de hojas
  para todos los símbolos,
  y sea \(f_i\) la frecuencia de la letra \(x_i\).
  Hallar las dos letras de frecuencia mínima,
  digamos \(x_a\) y \(x_b\) con frecuencias \(f_a\) y \(f_b\).
  Unir sus hojas en la hoja \(\ell_{a }\) con frecuencia \(f_a+f_b\)
  dando un árbol \(R_{a b}\)
  (figura~\ref{09::RabTree}):
  \begin{figure}[ht]
    \centering
    \begin{tikzpicture}
          [every state/.style = {draw = black,
                                 fill = black,
                                 minimum size = 2mm}
          ]
      \node [state] (d0) {};
      \node [above of = d0, yshift = -42] (dummy:ce) {\(x_{ab}\)};
      \node [below left of = d0] (xa) {\(x_a\)};
      \node [below right of = d0] (xb) {\(x_b\)};
      \path (d0) edge node [above left] {\num{0}} (xa)
      edge node [above right] {\num{1}} (xb);
    \end{tikzpicture}
    \caption{El nodo \(x_{a b}\) es la unión entre \(x_a\) y \(x_b\).}
    \label{09::RabTree}
  \end{figure}
  Recursivamente resolver el problema con:
  \begin{equation}
    L
      = \{ \ell _1, \ldots, \ell _n \}
           \smallsetminus \{ \ell _a, \ell _b \} \cup \{ \ell _{a b}\}
  \end{equation}
  y frecuencias ajustadas (\(\ell_{a b}\rightsquigarrow f_a+f_b\))
  \begin{ejemplo}
    Considere el cuadro~\ref{09::CuadroEjemplo1}.
    \begin{table}[ht]
      \centering
      \begin{tabular}{>{\(}c<{\)}|c}
        \multicolumn{1}{c|}{\textbf{Símbolo}}
            & \textbf{Frecuencia} \\
        \hline
        a & \phantom{0}9 \\
        b & \phantom{0}4 \\
        c & \phantom{0}2 \\
        d & 15 \\
        e & \phantom{0}3 \\
        f & 17
      \end{tabular}
      \caption{Frecuencias de los símbolos \(a, b, c, d, e, f\).}
      \label{09::CuadroEjemplo1}
    \end{table}
    El algoritmo de Huffman
    encuentra los dos símbolos de menor frecuencia
    y crea un subárbol con ellos.
    En el cuadro~\ref{09::CuadroEjemplo1},
    se tiene que los símbolos \(c\) y \(e\)
    son los de menor frecuencia.
    Luego,
    formamos un subárbol con ellos
    (figura~\ref{09::EjemploArbol1}).
    \begin{figure}[ht]
      \centering
      \begin{tikzpicture}
          [every state/.style = {draw = black,
                                 fill = black,
                                 minimum size = 2mm}
          ]
        \node [state] (ce) {};
        \node [above of = ce, yshift = -42] (dummy:ce) {\(c e\)};
        \node [below left of = ce] (c) {\(c\)};
        \node [below right of = ce] (e) {\(e\)};
        \path (ce) edge node [above left] {\num{0}} (c)
           edge node [above right] {\num{1}} (e);
      \end{tikzpicture}
      \caption{Hojas son los símbolos que menos se repiten.}
      \label{09::EjemploArbol1}
    \end{figure}
    Agregamos este \textquote{nodo conjunto} al cuadro~\ref{09::CuadroEjemplo1},
    cuya frecuencia es equivalente
    al peso del árbol de la figura~\ref{09::EjemploArbol1}
    (suma de las frecuencias de \(c\) y \(e\)).
    El resultado se aprecia en el cuadro~\ref{09::CuadroEjemplo2}.
    \begin{table}[ht]
      \centering
      \begin{tabular}{>{\(}c<{\)}|c}
        \multicolumn{1}{c|}{\textbf{Símbolo}} & \textbf{Frecuencia} \\
        \hline
          a   & \phantom{0}9 \\
          b   & \phantom{0}4 \\
          d   & 15 \\
          f   & 17 \\
          c e & \phantom{0}5
      \end{tabular}
      \caption{Nodo conjunto \(c e\)
               con frecuencia la suma de las de \(c\) y \(e\).}
      \label{09::CuadroEjemplo2}
    \end{table}
    Repetimos el proceso,
    es decir,
    escogemos dos símbolos del cuadro~\ref{09::CuadroEjemplo2}
    que tienen menor frecuencia
    y creamos un nuevo subárbol.
    Estos símbolos son \(c e\) y \(b\).
    La figura~\ref{09::EjemploArbol2} muestra el árbol resultante.
    \begin{figure}[ht]
      \centering
      \begin{tikzpicture}
          [every state/.style = {draw = black,
                                 fill = black,
                                 minimum size = 2mm}
          ]
        \node [state] (ce) {};
        \node [above of = ce, yshift = -42] (dummy:ce) {\(c e\)};
        \node [below left of = ce] (c) {\(c\)};
        \node [below right of = ce] (e) {\(e\)};
        \path (ce) edge node [above left] {\num{0}} (c)
          edge node [above right] {\num{1}} (e);

        \node [above left of = ce, state] (bce) {};
        \node [below left of = bce] (b) {\(b\)};
        \node [above of = bce, yshift = -42] (dummy:bce) {\(b c e\)};
        \path (bce) edge node [above left] {\num{0}} (b)
          edge node [above right] {\num{1}} (ce);
      \end{tikzpicture}
      \caption{Hojas son los símbolos que menos se repiten.}
      \label{09::EjemploArbol2}
    \end{figure}
    Reemplazamos los símbolos \(b\) y \(c e\)
    del cuadro~\ref{09::CuadroEjemplo2}
    con \(b c e\) de frecuencia \(f_{bce} = 9\).
    El resultado queda en el cuadro~\ref{09::CuadroEjemplo3}.
    \begin{table}[ht]
      \centering
      \begin{tabular}{>{\(}c<{\)}|c}
        \multicolumn{1}{c|}{\textbf{Símbolo}} & \textbf{Frecuencia} \\
        \hline
          a	& \phantom{0}9 \\
          d	& 15 \\
          f	& 17 \\
          b c e & \phantom{0}9
      \end{tabular}
      \caption{Reemplazando los símbolos \(b\) y \(c e\)
               del cuadro~\ref{09::CuadroEjemplo2}.}
      \label{09::CuadroEjemplo3}
    \end{table}
    Iteramos nuevamente.
    En el cuadro~\ref{09::CuadroEjemplo3}
    se tiene que los dos símbolos con menor frecuencia
    son \(b c e\) y \(a\).
    El árbol resultante es el de la figura~\ref{09::EjemploArbol3}.
    \begin{figure}[ht]
      \centering
      \begin{tikzpicture}
          [every state/.style = {draw = black,
                                 fill = black,
                                 minimum size = 2mm}
          ]
        \node [state] (ce) {};
        \node [above of = ce, yshift = -42] (dummy:ce) {\(c e\)};
        \node [below left of = ce] (c) {\(c\)};
        \node [below right of = ce] (e) {\(e\)};
        \path (ce) edge node [above left] {\num{0}} (c)
          edge node [above right] {\num{1}} (e);

        \node [above left of = ce, state] (bce) {};
        \node [below left of = bce] (b) {\(b\)};
        \node [above of = bce, yshift = -42, xshift = 5] (dummy:bce)
          {\(b c e\)};
        \path (bce) edge node [above left] {\num{0}} (b)
          edge node [above right] {\num{1}} (ce);

        \node [state, above left of = bce] (abce) {};
        \node [below left of = abce] (a) {\(a\)};
        \node [above of = abce, yshift = -42] (dummy:abce)
          {\(a b c e\)};

        \path (abce) edge node [above right] {\num{1}} (bce)
          edge node [above left] {\num{0}} (a);
      \end{tikzpicture}
      \caption{Árbol con peso de \(f_{bce}+f_a = 18\).}
      \label{09::EjemploArbol3}
    \end{figure}
    Quitamos estos símbolos del cuadro~\ref{09::CuadroEjemplo3}
    y los reemplazamos por \(a b c e\).
    El resultado es el cuadro~\ref{09::CuadroEjemplo4}.
    \begin{table}[ht]
      \centering
      \begin{tabular}{>{\(}c<{\)}|c}
        \multicolumn{1}{c|}{\textbf{Símbolo}} & \textbf{Frecuencia} \\
        \hline
        d	& 15 \\
        f	& 17 \\
        a b c e & 18
      \end{tabular}
      \caption{Reemplazando los símbolos \(b c e\) y \(a\)
               del cuadro~\ref{09::CuadroEjemplo3}.}
      \label{09::CuadroEjemplo4}
    \end{table}
    En el cuadro~\ref{09::CuadroEjemplo4}
    vemos que los símbolos con menor frecuencia son \(d\) y \(f\).
    Tomamos estos dos símbolos
    y creamos un nuevo árbol que los tenga como hojas
    (figura~\ref{09::EjemploArbol4}).
    \begin{figure}[ht]
      \centering
      \begin{tikzpicture}
          [every state/.style = {draw = black,
                                 fill = black,
                                 minimum size = 2mm}
          ]
        \node [state] (df) {};
        \node [below left of = df] (d) {\(d\)};
        \node [below right of = df] (f) {\(f\)};
        \node [above of = df, yshift = -42] (dummy:df) {\(df\)};
        \path (df) edge node [above left] {\num{0}} (d)
          edge node [above right] {\num{1}} (f);
      \end{tikzpicture}
      \caption{Hojas son los símbolos de menor frecuencia
               del cuadro~\ref{09::CuadroEjemplo4}.}
      \label{09::EjemploArbol4}
    \end{figure}
    Sacamos esos símbolos
    y los reemplazamos por \(d f\),
    dando el cuadro~\ref{09::CuadroEjemplo5}.
    \begin{table}[ht]
      \centering
      \begin{tabular}{c|c}
        \multicolumn{1}{c|}{\textbf{Símbolo}} & \textbf{Frecuencia} \\
        \hline
          d f	  & 32\\
          a b c e & 18
      \end{tabular}
      \caption{Reemplazando \(d\) y \(f\)
               del cuadro~\ref{09::CuadroEjemplo4}.}
      \label{09::CuadroEjemplo5}
    \end{table}
    Tomamos los dos últimos símbolos
    y creamos el árbol final
    de la figura~\ref{09::EjemploArbol5}.
    \begin{figure}[ht]
      \centering
      \begin{tikzpicture}
          [every state/.style = {draw = black,
                                 fill = black,
                                 minimum size = 2mm}
          ]
        \node [state] (ce) {};
        \node [above of = ce, yshift = -42] (dummy:ce) {\(c e\)};
        \node [below left of = ce] (c) {\(c\)};
        \node [below right of = ce] (e) {\(e\)};
        \path (ce) edge node [above left] {\num{0}} (c)
          edge node [above right] {\num{1}} (e);

        \node [above left of = ce, state] (bce) {};
        \node [below left of = bce] (b) {\(b\)};
        \node [above of = bce, yshift = -42, xshift = 5] (dummy:bce)
          {\(b c e\)};
        \path (bce) edge node [above left] {\num{0}} (b)
          edge node [above right] {\num{1}} (ce);

        \node [state, above left of = bce] (abce) {};
        \node [below left of = abce] (a) {\(a\)};
        \node [above of = abce, yshift = -42, xshift = 10] (dummy:abce)
          {\(a b c e\)};

        \path (abce) edge node [above right] {\num{1}} (bce)
          edge node [above left] {\num{0}} (a);

        \node [state, above left of = abce, xshift = -20, yshift = -10]
          (final) {};

        \node [state, below left of = final, yshift = 10, xshift = -10] (df) {};
        \node [below left of = df] (d) {\(d\)};
        \node [below right of = df] (f) {\(f\)};
        \node [above of = df, yshift = -42, xshift = -5] (dummy:df) {\(df\)};
        \path (df) edge node [above left] {\num{0}} (d)
          edge node [above right] {\num{1}} (f);

        \path (final) edge node [above left] {\num{0}} (df)
          edge node [above right] {\num{1}} (abce);

      \end{tikzpicture}
      \caption{Este árbol tiene un peso de \(f_{bce} + f_a = 18\).}
      \label{09::EjemploArbol5}
    \end{figure}
    La codificación resultante
    se lee directamente del árbol,
    es la del cuadro~\ref{tab:huffman-example-code}.
    \begin{table}[ht]
      \centering
      \begin{tabular}{>{\(}c<{\)}|r|l}
        \multicolumn{1}{c|}{\textbf{Símbolo}}
          & \multicolumn{1}{c|}{\textbf{Freq}}
          & \multicolumn{1}{c}{\textbf{Código}} \\
        \hline
        a &  9 & 10   \\
        b &  4 & 110  \\
        c &  2 & 1110 \\
        d & 15 & 00   \\
        e &  3 & 1111 \\
        f & 17 & 01
        \end{tabular}
      \caption{Codificación del ejemplo}
      \label{tab:huffman-example-code}
    \end{table}
    Si usáramos un código de largo fijo,
    requeriríamos \(\lceil \log_2 6 \rceil = 3\) bits por símbolo.
    Nuestro código da \num{2,28}~bits en promedio.
  \end{ejemplo}

\section{Algoritmo}

  Sucesivamente:
  \begin{enumerate}
  \item
    Tome los dos símbolos con menos frecuencia de su tabla
    y reemplácelos por un nuevo símbolo que representa a ambos.
    Supongamos que estos símbolos son \(x_a\) y \(x_b\),
    entonces el nuevo símbolo es \(x_{a b}\).
    La frecuencia de este símbolo conjunto
    será la suma de la frecuencia de \(x_a\) y \(x_b\).
  \item
    Cree un árbol que tenga como raíz
    al símbolo conjunto \(x_{a b}\) con \(x_{a}\) y \(x_{b}\) como hijos.
  \item
    Volver al paso 1 hasta que nuestra tabla esté formada
    por solo un símbolo conjunto,
    que representará a todos los símbolos de \(\Sigma\).
  \end{enumerate}
  Llegamos a la parte entretenida:
  demostrar que el algoritmo de Huffman halla un árbol óptimo.
  \begin{proof}
    Para demostrar que da un óptimo,
    usamos el teorema~\ref{theo:esquema-voraz}.
    \begin{description}
    \item[Elección Voraz:]
      Sea \(L\) la instancia original
      (o sea, el texto completo,
      con la frecuencia de cada símbolo respectivo),
      sean \(\ell_a\) y \(\ell_b\) las hojas menos frecuentes.
      Entonces hay un árbol óptimo que incluye \(R_{a b}\).

      Sea \(R\) un árbol óptimo para \(L\).
      Si \(R_{a b}\) es parte de \(R\),
      salimos a carretear.
      Si el árbol \(R_{a b}\) \emph{no} es parte de \(R\),
      sean \(\ell_x\), \(\ell_y\) dos hojas en \(R\) con padre común
      (hermanos),
      con \(\delta = d(\ell _x) = d(\ell_y)\) máximo.

      Claramente,
      \(a\) o \(b\) pueden coincidir con \(x\) o \(y\).
      Consideraremos el caso en que son diferentes,
      la situación en que alguno coincide es similar.

      Obtenga \(R^*\)
      intercambiando \(x\leftrightarrow a\), \(y\leftrightarrow b\),
      \(R^*\) contiene \(R_{a b}\).
      Sea \(B(R)\) el número de bits usados por el árbol \(R\)
      (la profundidad \(d\) hace referencia al árbol original \(R\)).
       En el árbol \(R^*\):
       \begin{align*}
         B(R^*)
           &= B(R) - (f_x + f_y) \delta
                   - f_a d(\ell_a)
                   - f_b d(\ell_b)
                   + (f_a + f_b) \delta
                   + f_x d(\ell_a) + f_y d(\ell_b) \\
           &= B(R) - \underbrace{(f_x-f_a)}_{> 0}
                     \underbrace{(\delta +d(\ell_a)}_{> 0}
                   - \underbrace{(f_y - f_b)}_{> 0}
                     \underbrace{(\delta + d(\ell_b))}_{> 0}
       \end{align*}
       Pero \(R\) es óptimo.
       Hemos llegado a una contradicción.
    \item[Estructura inductiva:]
      Elegir un (sub)árbol no interfiere con los demás.
    \item[Subestructura óptima:]
      Sean \(x, y\) los símbolos menos frecuentes,
      con frecuencias \(f_x\) y \(f_y\),
      respectivamente.
      Sea \(R'\) el árbol óptimo para el problema con el \textquote{símbolo}
      \(x y\) con frecuencia \(f_x + f_y\).
      Debemos demostrar que el árbol \(R\),
      con \(x y\) reemplazado por el subárbol respectivo con \(x\) e \(y\)
      es óptimo.
      Usaremos primas para distinguir valores en \(R'\) de valores en \(R\).
      Primeramente:
      \begin{align*}
        B(R)
          &= \sum_{s \in \Sigma} f_s d(s) \\
          &= \sum_{s \in \Sigma \smallsetminus \{ x, y \}} f_s d(s)
               + f_x d(x) + f_y d(y) \\
          &= \sum_{s \in \Sigma \smallsetminus \{ x, y \}} f_s d(s)
               + (f_x + f_y) (d'(x y) + 1) \\
          &= \sum_{s \in \Sigma \smallsetminus \{ x, y \}} f_s d(s)
               + f'_{x y} d'(x y) + f_x + f_y \\
          &= B(R') + f_x + f_y
      \end{align*}
      Para llegar a una contradicción,
      suponga que \(R\) no es óptimo,
      y sea \(T\) un árbol óptimo con \(x\) e \(y\) de hojas hermanas
      (sabemos que existe por lo anterior).
      Sea \(T'\) el árbol que resulta de eliminar \(x\) e \(y\).
      Podemos ver \(T'\) como un árbol
      para el alfabeto \(\Sigma \smallsetminus \{ x, y \} \cup \{ x y \}\)
      (agrupa \(x\) con \(y\)).
      Podemos repetir el cálculo anterior para obtener
      \(B(T) = B(T') + f_x + f_y\).
      O sea:
      \begin{align*}
        B(R')
          &= B(R) - f_x - f_y \\
          &> B(T) - f_x - f_y \\
          &= B(T')
      \end{align*}
      Pero supusimos que \(R'\) era óptimo.
    \end{description}
  \end{proof}
% To do:
% - Own (Python) programs
% - Apply to a (Spanish?) text, compute savings

\section{Comentarios finales}
\label{sec:Huffman-comments}

  Este esquema es óptimo bajo el supuesto que los símbolos aparecen al azar.
  Pero sabemos que esto no es así,
  por ejemplo,
  ciertas combinaciones están prohibidas en castellano.
  En un texto particular habrán palabras que se repiten con frecuencia,
  aprovechar estas particularidades es otro problema.
  Los algoritmos de compresión de datos que se usan hoy aprovechan esto,
  pero en algún nivel usan código Huffman por ser simple y eficiente.
  Discusión del ubicuo formato Zip
  y un programa ejemplo
  (que muestra cómo la teoría precedente
   --- junto con otras técnicas de compresión de datos ---
   se puede llevar a código práctico)
  ofrece Wennborg~%
    \cite{wennborg20:_zip_files}.

\section{La desigualdad de Kraft-McMillan}
\label{sec:Kraft-McMillan}

  La desigualdad de Kraft~%
    \cite{kraft49:_coding}
  limita los largos de los códigos en un código prefijo.
  A su vez,
  McMillan~%
    \cite{mcmillan56:_two_inequalities}
  demostró que si un código no cumple la desigualdad de Kraft
  no puede decodificarse en forma única.

  Demostraremos ambas.
  \begin{theorem}
    Sean los símbolos del alfabeto \(\Sigma = \{a_1, a_2, \dotsc, a_n\}\)
    codificados mediante un código decodificable de forma única
    sobre un alfabeto de tamaño \(r\),
    con palabras de código respectivamente \(\ell_1, \ell_2, \ldots, \ell_n\).
    Entonces:
    \begin{equation}
      \label{eq:Kraft-inequality}
      \sum_{1 \le k \le n} r^{\ell_k}
        \le 1
    \end{equation}
    Por el contrario,
    para un conjunto de números naturales \(\ell_1, \ell_2, \ldots, \ell_n\)
    que cumplen la desigualdad indicada hay un código con  palabras de código
    de los respectivos largos que es decodificable en forma única.
  \end{theorem}
  \begin{proof}
    Primero demostraremos que la desigualdad de Kraft~\ref{eq:Kraft-inequality}
    se cumple para códigos prefijo
    y que hay un código prefijo con esos largos si se cumple la desigualdad.
    Sin pérdida de generalidad,
    suponemos \(\ell_1 \le \ell_2  \le \dotsb \le \ell_n\).
    Sea \(A\) el árbol \(r\)\nobreakdash-ario completo de altura \(\ell_n\),
    con lo que podemos considerar las palabras del código prefijo
    como nodos de \(A\).
    Sea \(A_k\) el conjunto de hojas del subárbol que tiene el nodo \(v_k\)
    (correspondiente al código de \(a_k\))
    de \(A\) como raíz.
    Como la altura del subárbol con raíz \(v_k\) es \(\ell_n - \ell_k\),
    tiene \(\lvert A_k \rvert = r^{\ell_n - \ell_k}\) hojas.
    Como es un código prefijo,
    sabemos que \(A_r \cap A_s = \varnothing\) si \(r \ne s\).
    Como el total de hojas de \(A\) es \(r^{\ell_n}\),
    concluimos que:
    \begin{equation*}
      \left\lvert \bigcup_{1 \le k \le n} A_k \right\rvert
        =   \sum_{1 \le k \le n} \lvert A_k \rvert
        =   \sum_{1 \le k \le n} r^{\ell_n - \ell_k}
        \le r^{\ell_n}
    \end{equation*}
    de donde sigue la desigualdad~\ref{eq:Kraft-inequality}.

    Para el recíproco,
    dada una secuencia de enteros \(\ell_1 \le \ell_2 \le \dotsb \le \ell_n\)
    que cumplen la desigualdad~\ref{eq:Kraft-inequality},
    podemos construir un código prefijo con códigos de esos largos.
    La idea es elegir arbitrariamente
    un código de largo \(\ell_1\) para \(a_1\).
    En el árbol \(A\)
    esto corresponde a eliminar como posibles códigos
    los prefijos de ese código.
    También elimina los descendientes del vértice \(v_i\),
    en particular las hojas \(A_1\),
    que son \(r^{\ell_n - \ell_1}\).
    Podemos elegir del resto del árbol algún código de largo \(\ell_2\),
    lo que elimina \(r^{\ell_n - \ell_2}\) hojas adicionales.

    Siguiendo de esta forma,
    si se cumple la desigualdad vemos que esto es posible en cada paso,
    construimos un código prefijo con los largos prescritos.
    Como \(\ell_{k + 1} \ge \ell_k\),
    por la forma de los subárboles
    no podemos quedar con un conjunto de hojas que no tienen ascendente común,
    hay cómo elegir cada uno de los códigos.

    Ahora demostraremos que si el código se puede decodificar en forma única,
    entonces cumple la desigualdad de Kraft.
    El recíproco de esto ya lo demostramos,
    un código prefijo obviamente se puede decodificar en forma única.

    Sea:
    \begin{equation*}
      C
        = \sum_{1 \le k \le n} r^{-\ell_k}
    \end{equation*}
    La idea es acotar \(C^m\) para \(m \in \mathbb{N}\)
    y demostrar que la cota solo es posible para todo \(m\) si \(C \le 1\).
    Escribimos:
    \begin{align*}
      C^m
        &= \left( \sum_{1 \le k \le n} r^{-\ell_k} \right)^m \\
        &= \sum_{1 \le r_1 \le n}
             \sum_{1 \le r_2 \le n} \cdots \sum_{1 \le r_n \le n}
               r^{-\ell_{r_1} - \ell_{r_2} - \dotsm - \ell_{r_n}}
    \end{align*}
    Considere ahora todas las palabras en \(\Sigma^m\),
    si las codificamos con el código supuesto todos los códigos serán distintos.
    Esto se traduce en que cada término de la suma
    corresponde a una palabra en \(\Sigma^m\).
    Si llamamos \(q_\ell\)
    el número de codificaciones de \(\Sigma^m\) de largo \(\ell\),
    resultando:
    \begin{equation*}
      C^m
        = \sum_{\ell \ge 1} q_\ell r^{-\ell}
    \end{equation*}
    Para un alfabeto de \(r\) símbolos
    hay solo \(r^\ell\) códigos de largo \(\ell\),
    con lo que tenemos la cota simple \(q_\ell \le r^\ell\).
    Si \(\ell_{\mathtt{max}}\)
    es el largo máximo de un código
    esto da la cota:
    \begin{align*}
      C^m
        &\le \sum_{1 \le \ell \le m \ell_{\mathtt{max}}} r^{\ell} r^{-\ell} \\
        &=   m \ell_{\mathtt{max}} \\
      C
        &\le \left( m \ell_{\mathtt{max}} \right)^{1 / m}
    \end{align*}
    Esto debe cumpĺirse para todo \(m\).
    Pero:
    \begin{equation*}
      \lim_{m \to \infty} \left( m \ell_{\mathtt{max}} \right)^{1 / m}
        = 1
    \end{equation*}
    O sea,
    debe ser \(C \le 1\),
    de lo contrario la desigualdad
    se violaría para \(m\) suficientemente grande.
  \end{proof}

%%% Local Variables:
%%% mode: latex
%%% TeX-master: "../INF-221_notas"
%%% End:

% LocalWords:  decodificable max
    

\section*{Ejercicios}
\label{sec:ejercicios-09}

  \begin{enumerate}
  \item
    Escriba un programa que lea las frecuencias de un conjunto de símbolos
    (para simplificar,
     considere caracteres ASCII únicamente)
    con sus frecuencias,
    y retorne una codificación de Huffman para ellos.
  \end{enumerate}

\bibliography{../referencias}

%%% Local Variables:
%%% mode: latex
%%% TeX-master: "../INF-221_notas"
%%% ispell-local-dictionary: "spanish"
%%% End:

% LocalWords:  pregrado optimalidad Moby Dick english prefix free sub
% LocalWords:  code Observamanos Freq Zip

\bibliographystyle{babplain-fl}

\chapter{Sistemas de subconjuntos y matroides}
\label{cha:matroids}

  Para muchos problemas hay una respuesta
  a la pregunta de por qué funcionan los algoritmos voraces.
  \begin{definition}
    Un \emph{sistema de subconjuntos} es un conjunto \(\mathscr{I}\)
    de subconjuntos de un conjunto \(\mathscr{E}\)
    (el \emph{conjunto base})
    de modo que \(\mathscr{I}\) es cerrado bajo inclusión.
  \end{definition}
  O sea,
  si \(\mathscr{B} \in \mathscr{I}\) y \(\mathscr{A} \subseteq \mathscr{B}\)
  entonces \(\mathscr{A} \in \mathscr{I}\).
  Note en particular que si \(\mathscr{I} \ne \varnothing\),
  siempre es \(\varnothing \in \mathscr{I}\).
  Cunningham~%
    \cite{cunningham12:_coming_matroids}
  reseña la curiosa historia de estas estructuras,
  y su creciente relevancia en optimización combinatoria.
  Oxley~%
    \cite{oxley03:_matroids,oxley14:_matroids}
  da una visión general de las matemáticas relevantes
  (la segunda referencia es una versión expandida).

  El \emph{problema de optimización} para un sistema de subconjuntos
  asigna un peso positivo a cada elemento de \(\mathscr{E}\),
  y busca un conjunto \(\mathscr{X} \in \mathscr{I}\)
  cuyo peso sea máximo en todos los conjuntos de \(\mathscr{I}\).
  Es claro que podemos definir en forma afín la búsqueda de un mínimo.
  Retendremos esta definición
  para no complicar innecesariamente la discusión general.
  Parte de la discusión y los ejemplos que siguen vienen de Erickson~%
    \cite{erickson19:_algorithms}.

  Algunos ejemplos:
  \begin{itemize}
  \item
    Sea \(\mathscr{E}\) un conjunto cualquiera de vectores
    de un espacio vectorial \(V\),
    y sea \(\mathscr{I}\) el conjunto de subconjuntos de \(\mathscr{E}\)
    que son linealmente independientes.
  \item
    Considere un grafo \(G = (V, E)\).
    Considere subconjuntos de los vértices entre los que no hay arcos.
    Determinar si hay tal conjunto de \(k\) vértices
    es el problema \foreignlanguage{english}{\textsc{Independent~Set}}.
  \item
    Considere nuevamente un grafo \(G = (V, E)\).
    Considere subconjuntos de los vértices
    que están conectados todos entre sí
    (un grafo \(K_k\) como subgrafo de \(G\).
    Determinar si hay tal conjunto de \(k\) vértices
    es el problema \foreignlanguage{english}{\textsc{Clique}}.
  \item
    Considere un conjunto de tareas que usan recursos comunes
    con instantes de inicio y fin fijos.
    Conjuntos de tareas que pueden ejecutarse sin interferencia
    forman un sistema de subconjuntos.
  \item
    Considere:
    \begin{align*}
      \mathscr{E}
        &= \{ a, b, c \} \\
      \mathscr{I}
        &= \{ \varnothing, \{ a \}, \{ b \}, \{ a, b \}, \{ b, c \} \}
    \end{align*}
    Podemos verificar directamente que es cerrado bajo inclusión.
    Una manera alternativa de describir \(\mathscr{I}\)
    es como todos los conjuntos que no contienen a \(a\) y \(c\).
  \item
    Sea \(\mathscr{E}\) el conjunto de arcos de un grafo,
    y sean \(\mathscr{I}\) los conjuntos de arcos que no comparten vértices
    (se les llama \emph{\foreignlanguage{english}{matching}}
     del grafo).
  \end{itemize}

\section{Algoritmo voraz genérico}
\label{sec:greedy-generic}

  Dado un sistema finito de subconjuntos \((\mathscr{E}, \mathscr{I})\)
  hallamos un conjunto en \(\mathscr{I}\)
  mediante el algoritmo~\ref{alg:greedy-generic}.
  \begin{algorithm}[ht]
    \DontPrintSemicolon\Indp

    \Function{\(\mathrm{Greedy}(\mathscr{E}, \mathscr{I})\)}{
      \(\mathscr{X} \gets \varnothing\) \;
      Sort the elements of \(\mathscr{E}\) by decreasin weight \;
      \For{\(x \in \mathscr{E}\)}{
        \If{\(\mathscr{X} \cup \{ x \} \in \mathscr{I}\)}{
          \(\mathscr{X} \gets \mathscr{X} \cup \{ x \}\) \;
        }
      }
      \Return \(\mathscr{X}\) \;
    }
    \caption{Algoritmo voraz genérico}
    \label{alg:greedy-generic}
  \end{algorithm}
  El algoritmo retorna un conjunto \emph{maximal}
  (no se le pueden agregar elementos de \(\mathscr{E}\)
   sin salir de \(\mathscr{I}\)),
  pero no necesariamente \emph{máximo}
  (no hay elementos de \(\mathscr{I}\) de mayor peso).
  Nuestro problema de optimización pide un conjunto máximo.

  Nuestro resultado es que hay una propiedad
  de sistemas de subconjuntos finitos
  que garantiza que el algoritmo voraz~\ref{alg:greedy-generic}
  da un conjunto máximo para todas las funciones de peso.

  En nuestros ejemplos previos:
  \begin{itemize}
  \item
    Si consideramos los subconjuntos de \(\{ a, b, c \}\)
    que no tienen \(\{ a, c \}\) como subconjunto,
    y asignamos peso a los elementos,
    agregaremos \(b\) y aquél de \(\{ a, c \}\) de mayor peso.
  \item
    En los conjuntos de arcos que no forman ciclos,
    lo que tenemos
    es el problema \emph{\foreignlanguage{english}{Maximal Weight Forest}}
    (hallar el bosque de mayor peso que es subgrafo de \(G\),
     abreviado MWF).
    Este problema es equivalente a MST,
    si el máximo peso de un arco en MST es \(m\),
    asígnele peso \(2 m - w(e)\) al arco \(e\).
    El algoritmo~\ref{alg:greedy-generic} para MWF aplicado a esto
    entrega un MST
    (es el algoritmo de Kruskal~\ref{alg:kruskal} disfrazado).
  \end{itemize}

\section{Matroides y algoritmos voraces}
\label{sec:matroid-greedy}

  Diremos que un sistema de subconjuntos \((\mathscr{E}, \mathscr{I})\)
  tiene la \emph{propiedad de intercambio}
  si:
  \begin{equation}
    \label{eq:subset-system-exchange}
    \forall \mathscr{A}, \mathscr{B} \in \mathscr{I},
      (\lvert \mathscr{A} \rvert < \lvert\mathscr{B} \rvert)
        \Longrightarrow
          (\exists e \in \mathscr{B} \smallsetminus \mathscr{A}
             \text{\ tal que\ } \mathscr{A} \cup \{ e \} \in \mathscr{I})
  \end{equation}
  En estos términos:
  \begin{definition}
    Un \emph{matroide} es un sistema de subconjuntos
    \(M = (\mathscr{E}, \mathscr{I})\)
    con la propiedad de intercambio.
    Llamamos \emph{conjuntos independientes}
    a los elementos de \(\mathscr{I}\).
  \end{definition}
  Se les llama \emph{conjuntos dependientes}
  (¡Sorpresa!)
  a los subconjuntos de \(\mathscr{E}\) que no pertenecen a \(\mathscr{I}\).
  El \emph{rango} de \(\mathscr{X} \subseteq \mathscr{E}\)
  es la cardinalidad de su máximo subconjunto independiente.
  Un conjunto independiente de dice que es una \emph{base} de \(M\)
  si no es subconjunto de ningún conjunto independiente
  (es un conjunto independiente maximal).
  A un conjunto dependiente cuyos subconjuntos propios
  son todos independientes se le llama \emph{circuito}.

  Algunos ejemplos,
  varios de los cuales aparecerán nuevamente luego.
  Que son matroides quedará como ejercicio:
  \begin{description}
  \item[Matroide uniforme \boldmath\(U_{k, n}\)\unboldmath:]
    Un subconjunto \(X \subseteq \{ 1, 2, \dotsc, n \}\)
    es independiente si y solo si \(\lvert X \rvert \le k\).

    Todo subconjunto de \(k\) elementos es una base;
    todo conjunto de \(k + 1\) o más elementos es un circuito.
  \item[Matroide gráfico \boldmath\(\mathscr{M}(G)\)\unboldmath:]
    Sea \(G = (V, E)\) un grafo.
    Un subconjunto de \(E\) es independiente si no contiene ciclos.

    Una base del matroide es un árbol recubridor de \(G\);
    un circuito es un ciclo en \(G\).
  \item[Matroide cográfico \boldmath\(\mathscr{M}^*(G)\)\unboldmath:]
    Sea \(G = (V, E)\) un grafo.
    Un subconjunto \(I \subseteq E\) es independiente
    si el subgrafo complementario \((V, E \smallsetminus I)\) es conexo.

    Una base del matroide es el complemento de un árbol recubridor de \(G\);
    un circuito es un \emph{cociclo} de \(G\),
    un conjunto mínimo de arcos que desconecta a \(G\).
  \item[Matroide de correspondencias:]
    Sea \(G = (V, E)\) un grafo.
    Un conjunto \(I \subseteq V\) es independiente si y solo si
    hay un \emph{\foreignlanguage{english}{matching}}
    (conjunto de arcos que no tienen vértices en común)
    que los cubre.
  \item[Caminos disjuntos:]
    Sea \(G = (V, E)\) un grafo dirigido,
    y sea \(s\) un vértice fijo de \(G\).
    Un subconjunto \(I \subseteq V\) es independiente si y solo si
    hay caminos que no comparten arcos desde \(s\) a cada elemento de \(I\).
  \end{description}

  El resultado general es de Radó y Edmonds~%
    \cite{edmonds71:_matroid_greed_algor}:
  \begin{theorem}[Radó-Edmonds]
    \label{theo:greedy-matroid}
    Dado un sistema de subconjuntos \((\mathscr{E}, \mathscr{I})\),
    las siguientes son equivalentes:
    \begin{enumerate}
    \item
      El algoritmo voraz~\ref{alg:greedy-generic}
      entrega una solución óptima
      (mínima o máxima)
      para toda función de peso
    \item
      El sistema de subconjuntos es un matroide
    \end{enumerate}
  \end{theorem}
  \begin{proof}
    Demostramos implicancia en ambas direcciones.

    Usamos contradicción para demostrar
    que si \(M= (\mathscr{E}, \mathscr{I})\) es un matroide
    entonces el algoritmo voraz entrega un óptimo
    (máximo en la demostración,
     demostrar el caso de mínimo es simétrico).
    Sea \(\mathscr{A} = \{ a_1, a_2, \dotsc, a_k \}\)
    la solución entregada por el algoritmo voraz,
    y sea \(\mathscr{B} = \{ b_1, b_2, \cdots, b_{k'} \}\) una solución óptima,
    donde suponemos \(w(\mathscr{B}) > w(\mathscr{A})\).
    Primero,
    \(k = k'\),
    ya que si fuera \(k' \ne k\) por la propiedad de intercambio
    podríamos agregar un elemento del conjunto mayor al otro.
    Esto o contradice la optimalidad de \(\mathscr{B}\)
    o contradice el que el algoritmo terminó con \(\mathscr{A}\).
    Luego,
    podemos suponer que los elementos de \(\mathscr{A}\) y \(\mathscr{B}\)
    se listan en orden de mayor a menor
    (en \(\mathscr{A}\) es el orden en que los incluyó nuestro algoritmo),
    y consideremos el mínimo \(s\) tal que \(w(b_s) > w(a_s)\).
    Sean los subconjuntos:
    \begin{align*}
      \alpha
        &= \{ a_1, \cdots, a_{s - 1} \} \\
      \beta
        &= \{ b_1, \cdots, b_s \}
    \end{align*}
    Por ser parte de \(\mathscr{I}\)
    los conjuntos \(\mathscr{A}\) y \(\mathscr{B}\),
    también son parte de \(\mathscr{I}\) los conjuntos \(\alpha\) y \(\beta\).
    Por la propiedad de intercambio,
    hay \(t\) con \(1 \le t \le s\)
    tal que \(b_t \in \beta \smallsetminus \alpha\)
    y \(\alpha \cup \{ b_t \} \in \mathscr{I}\).
    Pero \(w(b_t) \ge w(b_s) > w(a_s)\),
    y nuestro algoritmo hubiese preferido \(b_t\) a \(a_s\).

    Al revés,
    si el algoritmo voraz siempre entrega un óptimo
    entonces \((\mathscr{E}, \mathscr{I})\) es un matroide.
    Para esto basta demostrar que el algoritmo voraz
    puede no entregar un óptimo
    si \((\mathscr{E}, \mathscr{I})\) no es un matroide.
    Si \((\mathscr{E}, \mathscr{I})\) no es un matroide,
    entonces:
    \begin{equation*}
      \exists \mathscr{A}, \mathscr{B} \in \mathscr{I},
        (\lvert \mathscr{A} \rvert < \lvert\mathscr{B} \rvert)
          \wedge
            (\not\exists e \in \mathscr{B} \smallsetminus \mathscr{A}
               \text{\ tal que\ } \mathscr{A} \cup \{ e \} \in \mathscr{I})
    \end{equation*}
    Sean \(m = \lvert \mathscr{A} \rvert\)
    y \(n = \lvert \mathscr{E} \rvert\).
    Defina:
    \begin{equation*}
      w(e)
        = \begin{cases}
             m + 2	 & e \in \mathscr{A} \\
             m + 1	 & e \in \mathscr{B} \smallsetminus \mathscr{A} \\
             1 / (2 n)	 & \text{caso contrario}
          \end{cases}
    \end{equation*}
    El algoritmo voraz retorna \(\mathscr{A}\),
    con peso a lo más \(m (m + 2) + 1 / 2 = m^2 + 2 m + 1 / 2\);
    una solución mejor es \(\mathscr{B}\)
    con peso al menos \((m + 1)^2 = m^2 + 2 m + 1\).
  \end{proof}
  Otra propiedad interesante resulta de lo siguiente:
  \begin{definition}
    Un sistema de subconjuntos \((\mathscr{E}, \mathscr{I})\)
    tiene la \emph{propiedad de cardinalidad}
    si:
    \begin{equation}
      \label{eq:propiedad-cardinalidad}
      \forall \mathscr{E}' \subseteq \mathscr{E},
         (A, B \in \mathscr{I}
            \text{\ subconjuntos maximales de \(\mathscr{E'}\)})
           \implies ( \lvert A \rvert = \lvert B \rvert )
    \end{equation}
  \end{definition}
  Decimos que \(A \in \mathscr{I}\)
  es \emph{subconjunto maximal} de \(\mathscr{E}'\)
  si \(A \subseteq \mathscr{E}'\) y no hay \(a \in \mathscr{E}'\)
  tal que \(A \cup \{ a \} \in \mathscr{I}\).
  Con esto tenemos:
  \begin{theorem}[Propiedad de cardinalidad]
    Sea un sistema de subconjuntos \((\mathscr{E}, \mathscr{I})\).
    Entonces \((\mathscr{E}, \mathscr{I})\) es un matroide
    si y solo si cumple la propiedad de cardinalidad.
  \end{theorem}
  \begin{proof}
    Es un si y solo si,
    demostramos implicancia en ambas direcciones.

    Sea \(\mathscr{E}\) un matroide,
    \(A, B\) subconjuntos maximales de \(\mathscr{E}' \subseteq \mathscr{E}\).
    Debemos demostrar \(\lvert A \rvert = \lvert B \rvert\),
    cosa que haremos por contradicción.
    Supongamos \(\lvert A \rvert < \lvert B \rvert\),
    por la propiedad de intercambio:
    \begin{equation*}
      \exists e \in B \smallsetminus A,
        (A \cup \{ e \} \in \mathscr{I})
    \end{equation*}
    Note que \(A \cup \{ e \} \in \mathscr{E}'\)
    ya que \(e \in B \subseteq \mathscr{E}'\),
    o sea,
    \(A\) no sería maximal.
    El caso \(\lvert A \rvert > \lvert B \rvert\) es simétrico.

    Al revés,
    debemos demostrar que
    si \((\mathscr{E}, \mathscr{I})\) no es matroide
    entonces hay \(\mathscr{E}'\) y \(A, B \in \mathscr{I}\)
    con \(A, B\) maximales en \(\mathscr{E}'\)
    con \(\lvert A \rvert \ne \lvert B \rvert\).
    Si \((\mathscr{E}, \mathscr{I})\) no es matroide:
    \begin{equation*}
      \exists A, C \in \mathscr{I},
        \lvert A \rvert < \lvert C \rvert \wedge
          \not\exists e \in C \smallsetminus A
          \text{\ con\ } A \cup \{ e \} \in \mathscr{I}
    \end{equation*}
    Defina \(\mathscr{E}' = A \cup C\),
    note que \(A\) es maximal en \(\mathscr{E}'\).
    Hay \(B \in \mathscr{I}\) tal que \(C \subseteq B\)
    y \(B\) es maximal en \(\mathscr{E}'\).
    Pero \(\lvert B \rvert \ge \lvert A \rvert + 1\),
    como debíamos demostrar.
  \end{proof}
  Tenemos dos propiedades diferentes
  (intercambio y cardinalidad)
  que describen los matroides.

  Note que nuestro primer ejemplo de algoritmo voraz
  (programar observaciones de ALMA)
  tiene un sistema de subconjuntos natural asociado:
  dos tareas son independientes si no traslapan
  (conjuntos independientes son los \textsc{Independent~Set}
   del grafo en el cual cada observación es un vértice,
   y dos vértices están unidos si traslapan).
  Esto \emph{no} es un matroide,
  pueden haber conjuntos independientes maximales de tamaños distintos.
  Nuestro algoritmo voraz entrega un óptimo por la elección
  de función de peso
  (todos iguales),
  para otras funciones de peso falla.

  Algunos ejemplos de matroides
  y los algoritmos voraces correspondientes:
  \begin{itemize}
  \item
    Los subconjuntos de un conjunto finito \(\mathscr{U}\)
    de cardinalidad a lo más \(k\).

    Podemos hallar el subconjunto más pesado usando el algoritmo voraz.
  \item
    Sea \(G = (V, E)\) un grafo,
    \(s \in V\) un vértice cualquiera.
    Los arcos de caminos desde \(s\) en el grafo son un matroide.
  \item
    Matroide de columnas.
    Sea \(\mathbf{A}\) una matriz,
    \(\mathscr{E}
        = \{ \mathbf{x}
               \colon \mathbf{x} \text{\ es columna de \(\mathbf{A}\)} \}\),
    y sea \(\mathscr{I}\) los conjuntos de columnas linealmente independientes.

    Podemos hallar la base más pesada para las columnas de \(\mathbf{A}\)
    usando el algoritmo voraz.
  \end{itemize}
  Nuestro último ejemplo de sistema de subconjuntos
  (arcos independientes en un grafo)
  no es un matroide,
  y el algoritmo voraz no siempre da un máximo.
  Considere el grafo de la figura~\ref{fig:edge-labelled-graph}.
  \begin{figure}
    \centering
    \tikzstyle{vertex} = [circle, fill = black!75, draw = black]
    \begin{tikzpicture}
       \node[vertex] at (0, {2 * sqrt(3)})		      (A)   {};
       \node[vertex] at (2, {2 * sqrt(3)})		      (B)   {};
       \node[vertex] at (4, {2 * sqrt(3)})		      (C)   {};

       \node[vertex] at (1, {sqrt(3)})		      (D)   {};
       \node[vertex] at (3, {sqrt(3)})		      (E)   {};

       \node[vertex] at (2, 0)			      (F)   {};

       \path (A) edge node[above] {\num{3}} (B)
             (A) edge node[above] {\num{2}} (D)
             (B) edge node[above] {\num{2}} (C)
             (B) edge node[above] {\num{2}} (E)
             (C) edge node[below] {\num{2}} (E)
             (D) edge node[above] {\num{3}} (E)
             (D) edge node[above] {\num{2}} (F)
             (E) edge node[below] {\num{2}} (F);
    \end{tikzpicture}
    \caption{Un grafo con arcos rotulados}
    \label{fig:edge-labelled-graph}
  \end{figure}
  Al buscar un conjunto de arcos de máximo peso
  que no comparten vértices,
  el algoritmo voraz da el conjunto maximal
  que consta de los arcos de peso~\num{3},
  para un total de~\num{6}
  (ver~\ref{fig:edge-labelled-graph-greedy});
  \begin{figure}
    \centering
    \tikzstyle{vertex} = [circle, fill = black!75, draw = black]
    \subfloat[Resultado del algoritmo voraz]{
      \begin{tikzpicture}
         \node[vertex] at (0, {2 * sqrt(3)})		      (A)   {};
         \node[vertex] at (2, {2 * sqrt(3)})		      (B)   {};
         \node[vertex] at (4, {2 * sqrt(3)})		      (C)   {};

         \node[vertex] at (1, {sqrt(3)})		      (D)   {};
         \node[vertex] at (3, {sqrt(3)})		      (E)   {};

         \node[vertex] at (2, 0)			      (F)   {};

         \draw[very thick] (A) edge node[above] {\num{3}}	      (B)
                           (D) edge node[above] {\num{3}}	      (E);

         \draw[thin, gray] (D) edge node[above] {\num{2}}	      (F)
                           (E) edge node[below] {\num{2}}	      (F)
                           (A) edge node[above] {\num{2}}	      (D)
                           (B) edge node[above] {\num{2}}	      (C)
                           (B) edge node[above] {\num{2}}	      (E)
                           (C) edge node[below] {\num{2}}	      (E);
      \end{tikzpicture}
    \label{fig:edge-labelled-graph-greedy}
  }
  \qquad
  \subfloat[Un óptimo]{
    \begin{tikzpicture}
       \node[vertex] at (0, {2 * sqrt(3)})		      (A)   {};
       \node[vertex] at (2, {2 * sqrt(3)})		      (B)   {};
       \node[vertex] at (4, {2 * sqrt(3)})		      (C)   {};

       \node[vertex] at (1, {sqrt(3)})		      (D)   {};
       \node[vertex] at (3, {sqrt(3)})		      (E)   {};

       \node[vertex] at (2, 0)			      (F)   {};

       \draw[very thick] (A) edge node[above] {\num{3}} (B)
                         (C) edge node[below] {\num{2}} (E)
                         (D) edge node[above] {\num{2}} (F);

       \draw[thin, gray] (A) edge node[above] {\num{2}} (D)
                         (B) edge node[above] {\num{2}} (C)
                         (B) edge node[above] {\num{2}} (E)
                         (D) edge node[above] {\num{3}} (E)
                         (E) edge node[below] {\num{2}} (F);
    \end{tikzpicture}
    \label{fig:edge-labelled-graph-maximum}
  }
  \end{figure}
  pero el máximo es~\num{7}
  (ver~\ref{fig:edge-labelled-graph-maximum}).

  La teoría de matroides nació de consideraciones
  sobre conjuntos linealmente independientes de vectores,
  y se vio aplicable a teoría de grafos.
  Rápidamente se extendió a otras áreas,
  un ejemplo es el texto de Lawler~%
    \cite{lawler76:_combinatorial_optimization_matroids}
  sobre optimización combinatoria y matroides.

\section{Programación con plazos fatales}
\label{sec:deadline-scheduling}

  Suponga que debe completar \(n\) tareas en \(n\) días,
  donde cada tarea demanda un día completo de dedicación.
  Cada tarea tiene un plazo fatal,
  si se completa después de su plazo fatal hay un costo a pagar.
  Se busca el orden en el cual ejecutar las tareas
  de forma de pagar el mínimo costo.

  Formalmente,
  numeramos las tareas de \num{1} a \(n\),
  dados un arreglo de plazos fatales \(D\)
  (un entero entre \num{1} y \(n\))
  y uno de penalizaciones \(P\)
  (números reales no negativos).
  Un programa \(\pi\) es una permutación de \(\{ 1, 2, \dotsc, n \}\).
  Buscamos un programa \(\pi\) que minimice:
  \begin{equation*}
    \sum_{1 \le i \le n} P[i] \cdot [ \pi(i) > D[i] ]
  \end{equation*}

  No parece para nada un ejemplo de optimización de matroide,
  allí solicitan un subconjunto y buscamos una permutación.
  Sorprendentemente,
  hay un matroide disfrazado en él.
  Para el programa \(\pi\),
  diga que las tareas para las cuales \(\pi(i) > D[i]\)
  están \emph{atrasadas},
  las demás \emph{a tiempo}.
  La observación trivial de que el costo de una programación
  queda determinada por sus tareas a tiempo lleva a revelar el matroide.

  Llame \emph{realista} a un conjunto de tareas \(X\)
  tal que hay una programación \(\pi\) en la que todas las tareas de \(X\)
  se completan a tiempo.
  Podemos caracterizar los subconjuntos realistas de la siguiente forma.
  Sea \(X(t)\) el conjunto de tareas en \(X\)
  con plazo fatal en o antes de \(t\):
  \begin{equation*}
    X(t)
      = \{ i \in X \colon D[i] \le t \}
  \end{equation*}
  En particular,
  \(X(0) = \varnothing\) y \(X(n) = X\).
  \begin{proposition}
    \label{prop:schedule-realistic}
    Sea \(X \subseteq \{ 1, 2, \dotsc, n \}\) un conjunto arbitrario de tareas.
    Entonces \(X\) es realista si y solo si \(\lvert X(t) \rvert \le t\)
    para todo \(t\).
  \end{proposition}
  \begin{proof}
    Es un si y solo si,
    demostramos implicancia en ambas direcciones.

    Sea \(\pi\) una programación
    en la que todas las tareas de \(X\) están a tiempo.
    Sea \(i_t\) la \(t\)\nobreakdash-ésima de \(X\) a completar.
    Por un lado,
    \(\pi(i_t) \ge t\),
    ya que completamos \(t - 1\) tareas antes;
    por el otro,
    \(\pi(i_t) \le D[i_t]\) ya que \(i_t\) está a tiempo.
    Concluimos \(D[i_t] \ge t\),
    por lo que \(\lvert X(t) \rvert \le t\).

    Suponga ahora que \(\lvert X(t) \rvert \le t\) para todo \(t\).
    Si ejecutamos las tareas de \(X\) en orden de plazo fatal,
    completamos las tareas con plazo fatal a más tardar \(t\)
    el día \(t\).
    Para todo \(i \in X\) estamos completando \(i\) antes de \(D[i]\),
    \(X\) es realista.
  \end{proof}
  Llamemos \emph{canónica} una programación para el conjunto de tareas \(X\)
  en la cual se ejecutan las tareas de \(X\) en orden de plazo fatal creciente,
  y las demás tareas en orden arbitrario.
  La proposición~\ref{prop:schedule-realistic}
  nos dice que \(X\) es realista
  si y solo si todas sus tareas se completan a tiempo
  en su programación canónica.
  O sea,
  nuestro problema se puede reformular como
  hallar un subconjunto realista \(X\) que maximice:
  \begin{equation*}
    \sum_{i \in X} P[i]
  \end{equation*}
  Estamos buscando un subconjunto óptimo.
  \begin{proposition}
    \label{prop:realistic=matroid}
    La colección de conjuntos realistas es un matroide.
  \end{proposition}
  \begin{proof}
    El conjunto vacío es realista
    (vacuamente),
    todo subconjunto de un conjunto realista es obviamente realista.
    Resta demostrar que se cumple la propiedad de intercambio.
    Sean entonces \(X\) e \(Y\) conjuntos realistas,
    con \(\lvert X \rvert > \lvert Y \rvert\).

    Sea \(t^*\) el máximo entero
    tal que \(\lvert X(t^*) \rvert \le \lvert Y(t^*) \rvert\).
    Debe existir,
    ya que \(\lvert X(0) \rvert = 0 \le 0 = \lvert Y(0) \rvert\)
    mientras
    \(\lvert X(n) \rvert
        = \lvert X \rvert
        > \lvert Y \rvert
        = \lvert Y(n) \rvert\).
    Por la definición de \(t^*\),
    hay más tareas con plazo fatal \(t^* + 1\) que \(t^*\)
    en \(X\) que en \(Y\).
    O sea,
    podemos elegir \(j \in X \smallsetminus Y\) con plazo fatal \(t^* + 1\).
    Llamemos \(Z = Y \cup \{ j \}\).

    Sea \(t\) arbitrario.
    Si \(t \le t^*\),
    entonces \(\lvert Z(t) \rvert = \lvert Y(t) \rvert \le t\),
    ya que \(Y\) es realista.
    Por otro lado,
    si \(t > t^*\),
    \(\lvert Z(t) \rvert = \lvert Y(t) \rvert + 1 \le \lvert X(t) \rvert\)
    por la definición de \(t^*\) y dado que \(X\) es realista.
    Por la proposición~\ref{prop:schedule-realistic},
    \(Z\) es realista.
    Se cumple la propiedad de intercambio.
  \end{proof}
  Por la proposición~\ref{prop:realistic=matroid},
  nuestro problema de hallar la programación óptima
  es una optimización de matroide,
  el algoritmo voraz~\ref{alg:greedy-schedule} da detalles.
  \begin{algorithm}
    \DontPrintSemicolon\Indp

    Sort \(P\) in decreasing order
      and sort \(D\) the same \;
    \(j \gets 0\) \;
    \For{\(i \gets 1\) \KwTo \(n\)}{
      \(X[j + 1] \gets i\) \;
      \If{\(X[1 .. j + 1]\) is realistic}{
        \(j \gets j + 1\) \;
      }
    }
    \Return the canonical program for \(X[1 .. j]\) \;

    \caption{Algoritmo voraz para programar tareas}
    \label{alg:greedy-schedule}
  \end{algorithm}
  Falta determinar si un conjunto de tareas es realista.
  La proposición~\ref{prop:schedule-realistic}
  da una pista cómo hacerlo,
  dando el algoritmo~\ref{alg:realistic-subset}.
  Esto supone que \(X\) viene ordenado por plazo fatal:
  \begin{equation*}
    i \le j
      \implies D[X[i]] \le D[X[j]]
  \end{equation*}
  \begin{algorithm}
    \DontPrintSemicolon\Indp

    \Function{\(\mathrm{realistic}(X, D)\)}{
      \(N \gets 0\) \;
      \(j \gets 0\) \;
      \For{\(t \gets 1\) \KwTo \(n\)}{
        \If{\(D[X[j]] = t\)}{
          \(N \gets N + 1; j \leftarrow j + 1\) \;
          \If{\(N > t\)}{
            \Return \(F\) \;
          }
        }
      }
      \Return \(T\) \;
    }
    \caption{Determinar si un conjunto es realista}
    \label{alg:realistic-subset}
  \end{algorithm}
  El resultado se ejecuta en tiempo \(O(n^2)\),
  usando estructuras de datos apropiadas esto se reduce a \(O(n \log n)\).
  Detalles quedan de ejercicio.


\section*{Ejercicios}
\label{sec:ejercicios-matroides}

  \begin{enumerate}
  \item
    Demuestre que los ejemplos de sistemas de subconjuntos citados
    realmente lo son.
    ¿Son matroides?
  \item
    Demuestre que el matroide gráfico \(\mathscr{M}(G)\) es un matroide
    para todo grafo \(G\).
    Describa sus bases y circuitos.
  \item
    Demuestre que para todo grafo \(G\)
    el matroide cográfico \(\mathscr{M}^*(G)\) es un matroide.
  \item
    Demuestre que para todo grafo \(G\) el matroide de correspondencias
    es un matroide.
    \\ \textbf{Pista:}
       ¿Qué es la diferencia simétrica de dos correspondencias?
  \item
    Indique qué entrega el algoritmo voraz en cada ejemplo de matroide citado.
  \item
    Sea \(G\) un grafo.
    Un conjunto de ciclos \(\{ c_1, c_2, \dotsc, c_k \}\) de \(G\)
    se llama \emph{redundante} si cada arco de \(G\)
    aparece en un número par de \(c_i\).
    Un conjunto de ciclos es \emph{independiente}
    si no contiene subconjuntos redundantes.
    Un conjunto maximal de ciclos independientes es una \emph{base de ciclos}
    de \(G\).
    \begin{enumerate}
    \item
      Sea \(C\) una base de ciclos de \(G\).
      Demuestre que para cada ciclo \(\gamma\) de \(G\),
      hay un subconjunto \(A \subseteq C\) tal que \(A \cap \{ \gamma \}\)
      es redundante.
      O sea,
      \(\gamma\) es el \textquote{o exclusivo} de los ciclos de \(A\).
    \item
      Demuestre que la colección de conjuntos de ciclos independientes
      es un matroide.
    \item
      Suponga ahora que cada arco de \(G\) tiene un peso,
      que el peso de un ciclo es la suma de los pesos de sus arcos,
      y que el peso de un conjunto de ciclos
      es la suma de los pesos de los ciclos
      (note que arcos que se repiten se cuentan cada vez que aparecen).
      Describa y analice un algoritmo eficiente
      para hallar una base de ciclos de mínimo peso.
      (Esto no es sencillo,
       no es inmediato obtener los ciclos de \(G\)).
    \end{enumerate}
  \item
    Demuestre que el sistema de subconjuntos
    del problema de programar observaciones de ALMA no es un matroide.
    Dé un ejemplo con una función de peso
    (retorno de la observación)
    tal que el algoritmo voraz no entregue una programación óptima.
  \item
    Demuestre cómo programar tareas con plazo fatal en tiempo \(O(n \log n)\).

    \textbf{Pista:} Use una estructura
      que permita determinar si \(X \cup \{ i \}\) es realista
      y agregar \(i\) a \(X\) en tiempo \(O(\log n)\) cada operación.
  \end{enumerate}

\bibliography{../referencias}

%%% Local Variables:
%%% mode: latex
%%% TeX-master: "../INF-221_notas"
%%% ispell-local-dictionary: "spanish"
%%% End:

% LocalWords:  english Independent Set subgrafo Clique matching Sort
% LocalWords:  the elements of by decreasin weight Forest MWF MST in
% LocalWords:  recubridor cográfico cociclo optimalidad vertex circle
% LocalWords:  fill black draw ésima reformular decreasing order and
% LocalWords:  sort same is realistic canonical program for

\bibliographystyle{babplain-fl}

\chapter{Programación Dinámica}
\label{cha:programacion-dinamica}

  Hay situaciones que naturalmente se enfrentarían por búsquedas recursivas,
  pero en las cuales la búsqueda obvia termina resolviendo una y otra vez
  los mismos subproblemas.
  Hay dos opciones:
  la más fácil es registrar los subproblemas resueltos con sus soluciones,
  y revisar si un subproblema ya se resolvió antes de emprender su solución.
  Esta idea,
  atribuida a Michie,
  se conoce como \emph{memoización}~%
    \cite{michie68:_memoization}.
  La otra es \emph{programación dinámica},
  que consiste en calcular sistemáticamente las soluciones a subproblemas,
  de manera que cuando una de ellas se requiera ya esté calculada de antemano.
  La ventaja de programación dinámica
  es que no requiere una compleja estructura extra,
  nos ahorramos su administración y las búsquedas en ella.
  La exposición siguiente se organiza en parte
  siguiendo las sugerencias de Forišek~%
    \cite{forisek15:_towards_better_way_teach_dynam_progr}
  y usa la estructura dada por Erickson~%
    \cite{erickson19:_algorithms}
  para desarrollar algoritmos.

  Requisitos para la aplicabilidad de programación dinámica
  son similares a las de algoritmos voraces
  (ver el capítulo~\ref{cha:greedy-algorithm}).
  Suponemos un problema \(P\),
  que se resuelve en etapas,
  eligiendo parte de la solución \(p\) en cada una de ellas.
  \begin{description}
  \item[Estructura Inductiva:]
    Dada la elección \(\widehat{p}\),
    queda un subproblema menor \(P'\)
    tal que si \(\Pi'\) es solución viable de \(P'\),
    \(\{\widehat{p}\} \cup \Pi'\) es solución viable de \(P\)
    (\(P'\) no tiene \textquote{restricciones externas}).
  \item[Subestructura Óptima:]
    Si \(P'\) queda de \(P\) al sacar \(\hat p\),
    y  \(\Pi'\) es óptima para \(P'\),
    \(\Pi' \cup \{ \widehat{p} \}\) es óptima para \(P\).
  \item[Elección Completa:]
    Elegimos aquel \(\widehat{p}\) que da el mejor resultado
    combinado con \(\Pi'\),
    una solución óptima para el problema resultante
    de \(P \smallsetminus \{ \widehat{p} \}\).
  \end{description}
  A diferencia de un algoritmo voraz,
  no conocemos un criterio \textquote{local}
  que nos permita elegir \(\widehat{p}\),
  debemos considerar varias opciones.
  Esto lleva naturalmente a una recursión:
  resuelva los subproblemas recursivamente,
  y elija aquella combinación que da la solución global.

  Lo anterior está planteado en términos de buscar un óptimo,
  pero puede adaptarse para determinar si hay o no soluciones.

\section{Un primer ejemplo}
\label{sec:primer-ejemplo-programacion-dinamica}

  Consideremos los números de Fibonacci,
  definidos por la recurrencia:
  \begin{equation}
    \label{eq:Fibonacci-recurrence}
    F_{n + 2}
      = F_{n + 1} + F_n
      \qquad F_0 = 0, F_1 = 1
  \end{equation}
  Esto lleva a la obvia función recursiva
  del listado~\ref{lst:Fibonacci-straight}.
  \lstinputlisting[float,
		   language = Python,
		   caption = {Cálculo de número de Fibonacci,
			      recursión obvia},
		   label = lst:Fibonacci-straight,
		   firstline = 3, lastline = 7]
		  {code/fibonacci-straight}
  Si consideramos como medida de costo el número de llamadas,
  llamando \(C_n\) al número de llamadas para calcular \(F_n\)
  resulta la recurrencia:
  \begin{equation}
    \label{eq:Fibonacci-straight-cost-recurrence}
    C_{n + 2}
      = 1 + C_{n + 1} + C_n
      \qquad C_0 = C_1 = 1
  \end{equation}
  Técnicas tradicionales de solución de recurrencias
  (o simplemente verificando la solución)
  dan:
  \begin{equation}
    \label{eq:Fibonacci-straight-cost}
    C_n
      = 2 F_{n + 1} - 1
  \end{equation}
  Sabemos que \(F_n \sim \tau^n / \sqrt{5}\),
  donde \(\tau = (1 + \sqrt{5}) / 2\).
  Esto crece en forma exponencial.

  Aplicar memoización es simple en Python,
  usamos un diccionario para registrar los valores ya calculados.
  Ver el listado~\ref{lst:Fibonacci-memoized}.
  \lstinputlisting[float,
		   language = Python,
		   caption = {Cálculo de número de Fibonacci,
			      memoizado},
		   label = lst:Fibonacci-memoized,
		   firstline = 3, lastline = 8]
		   {code/fibonacci-memoized}
  Esto incluso puede automatizarse,
  dado que Python tiene funciones como objetos de primera clase
  y es un lenguaje dinámico.
  El listado~\ref{lst:Fibonacci-memoized-Klein}
  (tomado del curso de Klein~%
     \cite{klein20:_python_course})
  muestra una forma simple de hacerlo.
  La decoración \lstinline[language = Python]!@memoize!
  aplica la función \lstinline[language = Python]!memoize!
  a la función \lstinline[language = Python]!fibonacci!
  que se define a continuación.
  \lstinputlisting[float,
		   language = Python,
		   caption = {Cálculo de número de Fibonacci,
			      memoizado automático},
		   label = lst:Fibonacci-memoized-Klein,
		   firstline = 3]
		   {code/fibonacci-memoized-auto}
  Resultan \(2 n - 1\) llamadas.

  La idea de programación dinámica
  es calcular sistemáticamente los valores,
  de forma de tenerlos disponibles cuando se requieren.
  Esto lleva directamente al listado~\ref{lst:Fibonacci-array}.
  \lstinputlisting[float,
		   language = Python,
		   caption = {Cálculo de número de Fibonacci,
			      programación dinámica},
		   label = lst:Fibonacci-array,
		   firstline = 3, lastline = 7]
		  {code/fibonacci-array}
  Si reconocemos además que solo se necesitan los dos últimos valores,
  no requerimos almacenar los valores anteriores,
  llegamos al listado~\ref{lst:Fibonacci-last}.
  \lstinputlisting[float,
		   language = Python,
		   caption = {Cálculo de número de Fibonacci,
			      guardando solo los últimos dos valores},
		   label = lst:Fibonacci-last,
		   firstline = 3, lastline = 9]
		  {code/fibonacci-last}
  Aún mejor,
  podemos simplificar si partimos la iteración con el valor \(F_{-1} = 1\)
  (perfectamente consistente con la recurrencia),
  dando el programa~\ref{lst:Fibonacci-short}.
  \lstinputlisting[float,
		   language = Python,
		   caption = {Cálculo simplificado de número de Fibonacci},
		   label = lst:Fibonacci-short,
		   firstline = 3, lastline = 7]
		  {code/fibonacci-short}

  Es claro que todas las formulaciones alternativas toman tiempo lineal.

  Este ejemplo muestra las características salientes
  de la programación dinámica:
  tenemos un problema con solución recursiva obvia;
  tenemos las opciones de memoizar
  o usar programación dinámica;
  de usar programación dinámica
  puede ser suficiente retener solo parte de los valores calculados.

\section{Tome lo que más pueda}
\label{sec:tomar-maximo}

  Hay \(n\) botellas en línea,
  numeradas de \num{0} a~\(n - 1\)
  (¡somos computines!),
  la botella \(i\) contiene \(v_i\)~de cerveza eslovaca.
  Después de un difícil semestre,
  queremos beber el máximo posible,
  pero nos ponen la restricción que no podemos beber dos botellas vecinas.

  Este problema tiene una sencilla solución recursiva:
  debemos decidir si bebemos la última botella
  (en cuyo caso debemos omitir la penúltima,
   es claro que beberemos el máximo de las botellas \num{0} a~\(n - 3\))
  o dejamos la última botella
  (en cuyo caso bebemos el máximo de las botellas \num{0} a~\(n - 2\)).
  Obtenemos la recurrencia para \(M_n\),
  el máximo posible de beber si hay \(n\) botellas:
  \begin{align*}
    M_0
      &= 0 \\
    M_1
      &= v_0 \\
    M_{n + 2}
      &= \max \{ M_{n + 1}, M_n + v_{n - 1} \}
  \end{align*}
  Esto nos lleva al programa~\ref{lst:beer-recursive}.
  \lstinputlisting[float,
		   language = Python,
		   caption = {Beber cerveza recursivamente},
		   label = lst:beer-recursive,
		   firstline = 3, lastline = 11]
		  {code/beer-recursive}
  Es fácil ver que el costo de este algoritmo
  (número de llamadas a \lstinline[language = Python]!solve!)
  para \(n\) es \(2 F_{n + 1} - 1\)
  (igual que el cálculo ingenuo de números de Fibonacci),
  que sabemos es exponencial en \(n\).

  Usar explícitamente un arreglo \lstinline[language = Python]!M!
  lleva al programa~\ref{lst:beer-dp}.
  \lstinputlisting[float,
		   language = Python,
		   caption = {Beber cerveza por programación dinámica},
		   label = lst:beer-dp,
		   firstline = 3, lastline = 9]
		  {code/beer-dp}
  Notamos que solo se usan los últimos dos valores
  de \lstinline[language = Python]!M!,
  no se requiere el arreglo completo.
  Esto lleva al programa final~\ref{lst:beer-dp-final}.
  \lstinputlisting[float,
		   language = Python,
		   caption = {Beber cerveza por programación dinámica,
			      versión final},
		   label = lst:beer-dp-final,
		   firstline = 3, lastline = 9]
		  {code/beer-dp-final}

\section{Proyectos de plantas}
\label{sec:proyectos-plantas}

  Una corporación tiene US\$\,5~millones a invertir este año,
  y planea expandir tres de sus plantas.
  Cada planta ha entregado a lo más tres propuestas,
  con sus costos y retornos estimados.
  Las diferentes propuestas de cada planta son excluyentes,
  vale decir,
  de las tres se puede ejecutar solo una.
  Además,
  los proyectos o se hacen
  (gastando el presupuesto completo)
  o no se hacen,
  no se pueden efectuar parcialmente.
  El cuadro~\ref{tab:proyectos}
  resume los costos de las propuestas y sus retornos.
  \begin{table}[ht]
    \centering
    \begin{tabular}{|>{\(}c<{\)}||*{3}{>{\(}r<{\)}>{\(}r<{\)}|}}
      \hline
	& \multicolumn{2}{c|}{\textbf{Planta 1}}
	& \multicolumn{2}{c|}{\textbf{Planta 2}}
	& \multicolumn{2}{c|}{\textbf{Planta 3}} \\
      \multicolumn{1}{|c||}{\textbf{Propuesta}}
	& \multicolumn{1}{c}{\boldmath\(c_1\)\unboldmath}
	& \multicolumn{1}{c|}{\boldmath\(r_1\)\unboldmath}
	& \multicolumn{1}{c}{\boldmath\(c_2\)\unboldmath}
	& \multicolumn{1}{c|}{\boldmath\(r_2\)\unboldmath}
	& \multicolumn{1}{c}{\boldmath\(c_3\)\unboldmath}
	& \multicolumn{1}{c|}{\boldmath\(r_3\)\unboldmath} \\
      \hline\hline
	0 & 0 &	 0 & 0 &  0 & 0 &  0 \\
	1 & 1 &	 5 & 2 &  8 & 1 &  4 \\
	2 & 2 &	 6 & 3 &  9 &	&    \\
	3 &   &	   & 4 & 12 &	&    \\
      \hline
    \end{tabular}
    \caption{Propuestas, sus costos y retornos}
    \label{tab:proyectos}
  \end{table}
  Algunas plantas no completaron las tres propuestas,
  y en todos los casos se agrega la propuesta de \textquote{no hacer nada}.
  El objetivo es maximizar los retornos asignando los \num{5}~millones.
  Se asume que si no se invierten todos,
  el resto se \textquote{pierde}
  (no genera retornos).
  Un ejercicio interesante es considerar opciones más realistas.

  Una forma directa de resolver esto
  es considerar las \(3 \cdot 4 \cdot 2 = 24\)~posibilidades,
  y elegir la mejor.
  Claro que con más plantas y más proyectos,
  esto rápidamente se hace inmanejable.

  Una manera de obtener la solución es la siguiente:
  dividamos el problema en tres \emph{etapas}
  (cada etapa representa la asignación a una planta).
  Imponemos un orden artificial a las etapas,
  considerando las plantas en orden de número.
  Cada etapa la dividimos en \emph{estados},
  que recogen la información para ir a la etapa siguiente.
  En nuestro caso,
  los estados de la etapa~\num{1} son \(\{0, 1, 2, 3, 4, 5\}\),
  correspondientes a invertir esas cantidades en la planta~1

  Cada estado tiene un retorno asociado.
  Note que para decidir cuánto conviene asignar a la planta~3
  (cual de los proyectos financiar)
  basta saber cuánto queda por asignar
  luego de financiar los proyectos de las plantas~1 y~2.
  Los proyectos aprobados no interesan para esto.
  Note que nos interesa que \(x = 5\)
  (queremos invertir todo,
   o la mayor parte posible).

  Calculemos los retornos asociados a cada estado.
  Esto es simple en la etapa~\num{1}.
  El cuadro~\ref{tab:proyecto-etapa-1} resume los resultados.
  \begin{table}[ht]
    \centering
    \begin{tabular}{|>{\(}c<{\)}||*{2}{>{\(}c<{\)}}|}
      \hline
      \multicolumn{1}{|c||}{\textbf{Capital}}
	& \multicolumn{1}{c}{\textbf{Propuesta}}
	& \multicolumn{1}{c|}{\textbf{Retorno}} \\
      \multicolumn{1}{|c||}{\boldmath\(x\)\unboldmath}
	& \multicolumn{1}{c}{\textbf{óptima}}
	& \multicolumn{1}{c|}{\textbf{\boldmath\num{1}\unboldmath}} \\
      \hline
       0 &  0 &	 0 \\
       1 &  1 &	 5 \\
       2 &  2 &	 6 \\
       3 &  2 &	 6 \\
       4 &  2 &	 6 \\
       5 &  2 &	 6 \\
      \hline
    \end{tabular}
    \caption{Cómputo de la etapa~\num{1}}
    \label{tab:proyecto-etapa-1}
  \end{table}
  Estamos en condiciones de atacar la etapa~\num{2},
  la mejor combinación para las plantas~1 y~2.
  Dada cierta cantidad total \(x\) a invertir,
  consideramos cada propuesta para la planta~2 en turno,
  y sumamos su retorno con lo que rendiría lo que reste
  al invertir de la mejor manera en la planta~1
  (como da el cuadro~\ref{tab:proyecto-etapa-1}).
  Por ejemplo,
  con el capital total de~\num{5}
  si en la planta~2 elegimos la propuesta~3,
  tenemos un retorno de \num{9}
  y nos queda \num{2} para la planta~1,
  que da retorno~\num{6},
  para un total de~\num{15}.
  El cuadro
  resume esto para las distintas opciones.
  \begin{table}[ht]
    \centering
    \begin{tabular}{|>{\(}c<{\)}||*{2}{>{\(}c<{\)}}|}
      \hline
      \multicolumn{1}{|c||}{\textbf{Capital}}
	& \multicolumn{1}{c}{\textbf{Propuesta}}
	& \multicolumn{1}{c|}{\textbf{Retorno}} \\
      \multicolumn{1}{|c||}{\boldmath\(x\)\unboldmath}
	& \multicolumn{1}{c}{\textbf{óptima}}
	& \multicolumn{1}{c|}{\textbf{\boldmath\num{1} y \num{2}\unboldmath}} \\
      \hline
       0 &  0 &	 0 \\
       1 &  1 &	 5 \\
       2 &  1 &	 8 \\
       3 &  1 & 13 \\
       4 &  1 & 14 \\
       5 &  3 & 17 \\
      \hline
    \end{tabular}
    \caption{Cómputo de la etapa~\num{2}}
    \label{tab:proyecto-etapa-2}
  \end{table}
  Vamos por la etapa~\num{3},
  con la misma idea tenemos el cuadro~\ref{tab:proyecto-etapa-3}.
  \begin{table}[ht]
    \centering
    \begin{tabular}{|>{\(}c<{\)}||*{2}{>{\(}c<{\)}}|}
      \hline
      \multicolumn{1}{|c||}{\textbf{Capital}}
	& \multicolumn{1}{c}{\textbf{Propuesta}}
	& \multicolumn{1}{c|}{\textbf{Retorno}} \\
      \multicolumn{1}{|c||}{\boldmath\(x\)\unboldmath}
	& \multicolumn{1}{c}{\textbf{óptima}}
	& \multicolumn{1}{c|}
	     {\textbf{\boldmath\num{1}, \num{2} y \num{3}\unboldmath}} \\
      \hline
       0 &  0 &	 0 \\
       1 &  0 &	 5 \\
       2 &  1 &	 9 \\
       3 &  0 & 13 \\
       4 &  1 & 17 \\
       5 &  1 & 18 \\
      \hline
    \end{tabular}
    \caption{Cómputo de la etapa~\num{3}}
    \label{tab:proyecto-etapa-3}
  \end{table}
  La entrada para \(x = 5\)
  dice que el mejor retorno posible es~\num{18}.
  Nos indica que la mejor opción para la planta~3
  es su proyecto~\num{1},
  lo que deja~\(5 - 1 = 4\) para las otras;
  del cuadro~\ref{tab:proyecto-etapa-2}
  vemos que la mejor opción para la planta~2 es la~\num{1};
  queda~\(4 - 2 = 2\),
  con lo que del cuadro~\ref{tab:proyecto-etapa-1}
  vemos que la mejor opción para la planta~1 es la~\num{2}.

  Podemos generalizar lo anterior.
  Sea \(r_{j k}\) el retorno para la propuesta \(k\) en la etapa \(j\),
  y sea \(c_{j k}\) el costo de esa propuesta.
  Sea \(f_j(x)\) la ganancia total en la etapa \(j\)
  si el capital disponible en ella es \(x\).
  Entonces,
  incluyendo siempre la opción
  \textquote{no haga nada y no gaste en esta planta}:
  \begin{align*}
    f_1(x)
      &= \max_{k \colon c_{1 k} \le x} \{ r_{1 k} \} \\
    f_j(x)
      &= \max_{k \colon c_{j k} \le x} \{ r_{j k} + f_{j - 1}(x - c_{j k}) \}
  \end{align*}
  y si son \(n\) etapas y tenemos \(x\) capital disponible
  nos interesa \(f_n(x)\).
  Lo que hicimos arriba es evaluar esta recursión.

  Estamos desarrollando nuestra recurrencia \textquote{mirando hacia atrás},
  consideramos la etapa \(j\) suponiendo
  que tenemos resuelto el problema hasta \(j - 1\),
  el cálculo resultante \textquote{camina hacia adelante},
  aumentando \(j\).
  Claramente es igualmente válido \textquote{mirar hacia adelante}
  resultando una recurrencia \textquote{marcha atrás},
  las soluciones serán equivalentes.
  Cuál de las opciones es más natural depende del problema
  (y de las inclinaciones del programador).

\section{Estructura general}
\label{sec:dp-estructura-general}

  El desarrollo de los ejemplos anteriores
  sigue el siguiente esquema:
  \begin{enumerate}[font = \textbf, label = {(\alph*)}]
  \item \textbf{Plantear la recurrencia:}
    Esta es la parte crítica,
    depende íntimamente del problema.
    Debemos identificar los subproblemas relevantes,
    cómo se combinan para una solución,
    identificar todas las alternativas relevantes
    y cómo elegir la mejor de las alternativas.
  \item \textbf{Escribir un programa recursivo:}
    Generalmente es una traducción mecánica de la recurrencia.
  \item \textbf{Identificar subproblemas:}
    \label{step:dp:identify}
    Vea todas las formas en las que el programa recursivo se llama a sí mismo.
    Determine el conjunto de posibles argumentos.
  \item \textbf{Defina una estructura de datos:}
    Requerimos almacenar resultados para todas las combinaciones posibles
    de argumentos.
    Esto generalmente lleva a alguna clase de tabla,
    pero perfectamente puede ser apropiada una estructura diferente.
  \item \textbf{Identifique dependencias:}
    Debemos organizar los cálculos de forma que valores requeridos
    ya se hayan calculado antes.
    Por ejemplo,
    considere un valor genérico,
    y dibuje los valores de los que depende.
    Formalice esto.
  \item \textbf{Determine un buen orden de cálculo:}
    El orden en que se obtienen los resultados
    en el programa recursivo es una guía,
    pero no es necesario seguirlo estrictamente.
    Interesa definir un orden que sea simple de programar.
    Las dependencias descubiertas en el paso anterior
    definen un orden parcial entre subproblemas,
    buscamos una extensión lineal.
    Esto es crítico,
    tenga cuidado.
  \item \textbf{Analice requerimientos de tiempo y espacio:}
    Depende fundamentalmente del número de subproblemas
    y lo que se debe hacer para cada uno de ellos.
    Incluso es posible de obtener directamente
    luego del paso~\ref{step:dp:identify}.
  \item \textbf{Escriba el algoritmo:}
    Esto es inmediato si se desarrollaron cuidadosamente los pasos anteriores.
  \end{enumerate}

\subsection{Un ejemplo detallado}
\label{sec:dp-example}

  Para el entero no negativo \(n\) se define su valor recursivo
  como el máximo entre \(n\)
  y la suma de los valores recursivos
  de \(\lfloor n / 2 \rfloor\),
  \(\lfloor n / 3 \rfloor\) y \(\lfloor n / 4 \rfloor\).
  Es claro que habrán muchos cálculos repetidos al efectuar esto recursivamente,
  y éstos son independientes entre sí.
  Es la situación ideal para considerar programación dinámica.
  Aplicando nuestra estrategia,
  llamando \(m(n)\) al valor buscado:
  \begin{enumerate}[label = {(\alph*)}]
  \item \textbf{Plantear la recurrencia:}
    \label{item:dp-example-recurrence}
    La recurrencia está básicamente dada por el enunciado:
    \begin{equation*}
      m(n)
	= \max \{
		   n,
		   m(\lfloor n / 2 \rfloor)
		     + m(\lfloor n / 3 \rfloor)
		     + m(\lfloor n / 4 \rfloor)
	       \}
    \end{equation*}
    Es claro que \(m(0) = 0, m(1) = 1\).
  \item \textbf{Escribir un programa recursivo:}
    \label{item:dp-example-recursive-program}
    El programa recursivo es bastante obvio,
    es una traducción inmediata
    de la recurrencia de~\ref{item:dp-example-recurrence}.
    Vea el listado~\ref{lst:dp-example-recursive}
    \lstinputlisting[language = Python,
		     linerange = {3-8},
		     caption = {Programa recursivo},
		     label = lst:dp-example-recursive]
		    {code/max_sum_rec.py}
  \item \textbf{Identificar subproblemas:}
    \label{item:dp-example-subproblems}
    De la recurrencia es obvio que subproblemas inmediatos para \(n\)
    son los para \(\lfloor n / 2 \rfloor\),
     \(\lfloor n / 3 \rfloor\) y \(\lfloor n / 4 \rfloor\).
  \item \textbf{Defina una estructura de datos:}
    \label{item:dp-example-data-structure}
    Se requiere únicamente el valor de \(m\) para valores selectos de \(n\),
    un arreglo
    (lista en Python)
    es suficiente.
    Bastan \(\lfloor n / 2 \rfloor\) elementos,
    elementos posteriores no se usan.
  \item \textbf{Identifique dependencias:}
    \label{item:dp-example-dependencies}
    Ya lo vimos, \(m(n)\)
    depende de los valores para \(\lfloor n / 2 \rfloor\),
     \(\lfloor n / 3 \rfloor\) y \(\lfloor n / 4 \rfloor\).
  \item \textbf{Determine un buen orden de cálculo:}
    \label{item:dp-example-order}
    Como \(m(n)\)
    depende únicamente de valores de \(m\) para argumentos menores,
    un orden de cálculo obvio es llenar el arreglo secuencialmente.
  \item \textbf{Analice requerimientos de tiempo y espacio:}
    \label{item:dp-example-analysis}
    Es claro que la memoria extra usada
    es el arreglo para guardar valores de \(m\)
    (espacio \(\Theta(n)\))
    y el cálculo de cada elemento hace referencia a tres elementos adicionales,
    es fijo
    (tiempo \(\Theta(n)\)).
  \item \textbf{Escriba el algoritmo:}
    Escribir el programa final
    es juntar lo discutido en los puntos~\ref{item:dp-example-data-structure}
    a~\ref{item:dp-example-order},
    vea el listado~\ref{lst:dp-example-dp}.
    \label{item:dp-example-algorithm}
    \lstinputlisting[language = Python,
		     linerange = {3-10},
		     caption = {Programación dinámica},
		     label = lst:dp-example-dp]
		    {code/max_sum_dp.py}
    Es útil tener a la mano
    el programa recursivo~\ref{lst:dp-example-recursive}
    para contrastar resultados.
  \end{enumerate}
  Los ejemplos siguientes no serán desarrollados en forma tan detallada.

\section{Subset Sum}
\label{sec:SubsetSum}

  La idea de programación dinámica
  puede emplearse siempre que en una propuesta recursiva de solución
  hayan subproblemas que se repiten.
  Comúnmente se usa para problemas de optimización,
  pero no es el único uso.

  Dado un conjunto de enteros positivos
  \(\mathscr{A} = \{ a_1, a_2, \dotsc, a_n \}\),
  se pide determinar si hay un subconjunto de \(\mathscr{A}\) que suma \(s\).

  Definimos \(p[i, t]\) como verdadero
  si entre \(\{a_1, \dotsc, a_i\}\) hay un subconjunto que suma \(t\).

  La recurrencia sobre \(i\) es bastante obvia.
  Al agregar \(a_{i + 1}\) a los elementos considerados,
  podemos simplemente no usarlo
  (es posible obtener \(t\) con \(\{a_1, \dotsc, a_i\}\))
  o lo incluimos
  (debemos obtener \(t - a_{i + 1}\) con \(\{a_1, \dotsc, a_i\}\)).
  O sea:
  \begin{equation*}
    p[i + 1, t]
      = p[i, t] \vee p[i, t - a_{i + 1}]
  \end{equation*}
  Tenemos condiciones iniciales:
  \begin{equation*}
    p[i, t]
      = \begin{cases}
	  F & \text{si \(t > 0\) y \(i = 0\)} \\
	  T & \text{si \(t = 0\)}
	\end{cases}
  \end{equation*}
  Es simple llevar esto al algoritmo~\ref{alg:SubsetSum-r} recursivo,
  donde hemos aprovechado cálculo en corto circuito de la expresión.
  Debemos tener cuidado de no considerar incluir \(a_i\) si \(a_i > t\),
  como se muestra.
  Obtenemos el resultado como \(\mathrm{\textsc{SubsetSum}}(n, S)\).
  \begin{algorithm}[htbp]
    \DontPrintSemicolon\Indp

    \Function{\(\mathrm{\textsc{SubsetSum}}(i, S)\)}{
      \If{\(S = 0\)}{
	\Return \(T\) \;
      }
      \If{\(i = 0\)}{
	\Return \(F\) \;
      }
      \If{\(\mathrm{\textsc{SubsetSum}}(i - 1, S)\)}{
	  \Return \(T\) \;
      }
      \eIf{\(S \ge a_i\)}{
	\Return \(\mathrm{\textsc{SubsetSum}}(i - 1, S - a_i)\) \;
      }
      {
	\Return \(F\) \;
      }
    }
    \caption{Hay subconjunto de \(\{a_1, \dotsc, a_i\}\)
	     con la suma \(S\) dada}
    \label{alg:SubsetSum-r}
  \end{algorithm}
  En el peor caso,
  este algoritmo considera los \(2^n\) subconjuntos de los \(a_i\).

  Una solución por programación dinámica llena el arreglo
  desde \(t = 0\),
  para cada \(t = 0, 1, \dotsc\) calculamos \(p[i, t]\)
  sistemáticamente,
  con lo que los valores requeridos los habremos calculado antes.
  Ver el algoritmo~\ref{alg:SubsetSum-pd}.
  \begin{algorithm}[htb]
    \DontPrintSemicolon\Indp

    \Function{\(\mathrm{\textsc{SubsetSum}}(\mathbf{a}, S)\)}{
      \For{\(t \gets 1\) \KwTo \(S\)}{
	\(p[0, t] = F\) \;
      }
      \For{\(i \gets 0\) \KwTo \(\mathrm{len}(a)\)}{
	\(p[i, 0] \gets T\) \;
      }
      \For{\(t \gets 1\) \KwTo \(S\)}{
	\For{\(i \gets 1\) \KwTo \(\mathrm{len}(a)\)}{
	  \eIf{\(a_i > t\)}{
	    \(p[i, t] \gets p[i - 1, t]\) \;
	  }
	  {
	    \(p[i, t]
		\gets p[i - 1, t] \vee p[i - 1, t - a_i] \)\;
	  }
	}
      }
      \Return \(p[\mathrm{len}(a), S]\) \;
    }
    \caption{Subconjunto de \(\{a_1, \dotsc, a_n\}\)
	     que suma \(S\),
	     programación dinámica}
    \label{alg:SubsetSum-pd}
  \end{algorithm}
  Luego de ejecutar este algoritmo,
  \(p[n, S]\) nos dice si es posible o no la suma dada.
  Si quisiéramos además obtener un subconjunto que logra la suma pedida,
  habría que registrar con los \(p[t, i]\) verdaderos
  cuál fue la opción que dio verdadero,
  y revisar hacia atrás desde el resultado final.

  La complejidad de este algoritmo es \(O(n s \log s)\),
  estamos calculando esencialmente \(n s\) elementos,
  cada uno de los cuales significa algunas operaciones
  entre palabras de \(\log_2 s\) bits.
  Note que si \(s\) es substancialmente mayor a \(2^n\),
  el algoritmo recursivo es menos costoso.
  Vimos en \emph{Informática Teórica} (INF-155) que este problema
  (\textsc{SubsetSum})
  es \NP\nobreakdash-completo.
  En términos del largo de la representación en binario de los datos
  (los \(n\) elementos \(a_i\) y \(s\);
   si los \(a_i\) son de la misma magnitud que \(s\)
   los datos de entrada tienen tamaño \(O(n \log s)\))
  este algoritmo no es polinomial,
  pero sí lo es en términos de los valores de los datos de entrada
  A tales algoritmos se les llama \emph{pseudopolinomiales}.

\section{Subsecuencia creciente más larga}
\label{sec:LIS}

  Dado una secuencia de \(n\) elementos,
  nos interesa determinar el largo de la subsecuencia creciente más larga.
  Por ejemplo,
  una subsecuencia creciente más larga está marcada con negrilla
  en la siguiente secuencia:
  \begin{equation*}
    80, \mathbf{10}, \mathbf{22}, 9, \mathbf{33}, 21, \mathbf{50},
    41, \mathbf{60}, 23, 7
  \end{equation*}

  Sea \(a_i\) el \(i\)\nobreakdash-ésimo elemento de la secuencia
  (\(0 \le i < n\)),
  y sea \(L_i\) el largo de la secuencia creciente más larga
  \emph{que termina en \(a_i\)}.
  Nos interesa el valor máximo de \(L_i\).

  Note que nuestros subproblemas no son del mismo tipo que el problema inicial,
  estamos poniendo la condición de que el último elemento de la secuencia
  sea parte de la secuencia creciente,
  y el resultado no es simplemente el valor registrado en una posición fija.

  Pensando en la composición de la subsecuencia creciente más larga
  que termina con \(a_i\),
  antes de \(a_i\) hay una subsecuencia creciente más larga
  que termina en \(a_j\),
  con \(a_j < a_i\).
  Si no hay un \(a_j \le a_i\) con \(j < i\),
  la subsecuencia más larga que incluye \(a_i\) es simplemente este elemento.
  Esto hace plantear la recurrencia:
  \begin{align}
    L_i
      = \begin{cases}
	   1 + \max_{\substack{ 0 < j < i \\
				a_j \le a_i }} \{ L_j \}
		& \text{si tal \(j\) existe} \\
	   1
		& \text{caso contrario}
	\end{cases}
  \end{align}
  Como mencionamos,
  el valor buscado es \(\max_{0 \le i < n} \{ L_i \}\).

  Hay subproblemas repetidos,
  vale la pena pensar en programación dinámica.
  En nuestro caso,
  calcular los \(L_i\) sistemáticamente.
  El programa Python del listado~\ref{lst:LIS} da detalles.
  \lstinputlisting[float,
		   language = Python,
		   firstline = 7, lastline = 23,
		   caption = {Subsecuencia creciente más larga},
		   label = lst:LIS]
		  {code/lis}
  Esto entrega el largo de la subsecuencia más larga,
  obtener una subsecuencia más larga queda de ejercicio.

\section{Producto de matrices}
\label{sec:producto-de-matrices}

  Queremos calcular el producto de \(n\) matrices,
  \(\mathbf{A}_1 \cdot \mathbf{A}_2 \dotsm \mathbf{A}_n\),
  donde \(\mathbf{A}_i\) es \(n_i \times n_{i + 1}\)
  (para que sea posible el producto \(\mathbf{A}_i \cdot \mathbf{A}_{i + 1}\)).

  La técnica tradicional
  de multiplicar una matriz de \(r \times s\) por otra \(s \times t\)
  toma \(r s t\) multiplicaciones,
  y usaremos esto como medida de costo.
  Nuestro resultado no depende realmente del detalle de esto.

  Sabemos que la multiplicación de matrices es asociativa:
  \begin{equation*}
    \mathbf{A} \cdot (\mathbf{B} \cdot \mathbf{C})
      = (\mathbf{A} \cdot \mathbf{B}) \cdot \mathbf{C}
  \end{equation*}
  El trabajo total depende del orden.
  Por ejemplo,
  si tenemos matrices:
  \begin{align*}
    \mathbf{A} &\colon \phantom{0}2 \times 12 \\
    \mathbf{B} &\colon 12 \times 3  \\
    \mathbf{C} &\colon \phantom{0}3 \times 4
  \end{align*}
  Entonces,
  el costo de calcular \((\mathbf{A} \cdot \mathbf{B}) \cdot \mathbf{C}\) es:
  \begin{equation*}
    2 \cdot 12 \cdot 3 + 2 \cdot 3 \cdot 4
      = 96
  \end{equation*}
  El primer término corresponde al producto \(\mathbf{A} \cdot \mathbf{B}\),
  el segundo a multiplicar esto por \(\mathbf{C}\).
  Continuamos con el cálculo del costo
  de obtener \(\mathbf{A} \cdot (\mathbf{B} \cdot \mathbf{C})\):
  \begin{equation*}
    2 \cdot 12 \cdot 4 + 12 \cdot 3 \cdot 4
      = 240
  \end{equation*}
  Vemos que hasta en este caso mínimo el orden hace una gran diferencia.

  Para obtener el óptimo en el caso general,
  consideremos \textquote{de afuera adentro}.
  Si el último producto es:
  \begin{equation}
    \label{11::ExplicacionOptimo}
    \underbrace{
      (\mathbf{A}_1 \cdot \mathbf{A}_2 \dotsm \mathbf{A}_i)
    }_{\text{óptimo}}
       \underbrace{(\mathbf{A}_{i + 1} \dotsm \mathbf{A}_n)}_{\text{óptimo}}
  \end{equation}
  Note que \((\mathbf{A}_1 \cdot \mathbf{A}_2 \dotsm \mathbf{A}_i)\)
  y \((\mathbf{A}_{i + 1} \dotsm \mathbf{A}_n)\)
  también deben calcularse de forma óptima,
  de lo contrario el cuento no sirve.
  En consecuencia,
  tendremos que dividir nuevamente lo que está entre paréntesis
  sucesivamente hasta obtener un producto de dos matrices.
  Es importante destacar que no conocemos \(i\),
  que tendremos que probar todas las opciones.
  Ojo,
  muchos subproblemas se repiten.

  Idea de programación no recursiva:
  \begin{itemize}
  \item \(T[i, j]\):
    costo de calcular el producto
      \(\mathbf{A}_i\cdot \ldots \cdot \mathbf{A}_j\).
  \end{itemize}
  Inicialmente:
  \begin{equation*}
    T[i, i] = 0
  \end{equation*}
  Sabemos que:
  \begin{equation}
    \label{11::MinimoIteracion}
    T[i, j]
      = \min_{i \le k < j}
	  \{ T[i, k] + T[k+1, j]
	      + \underbrace{
		  n_i \cdot n_{k + 1} \cdot n_{j + 1}
		}_{\text{costo del producto}}\hspace{-0.2mm}
	  \}
  \end{equation}
  Acá \(k\) corresponde a dónde dividimos con los paréntesis
  (en~\eqref{11::ExplicacionOptimo}
   es \(k = i\))
  que dio el valor mínimo de~\eqref{11::MinimoIteracion}.
  Además,
  \(n\) corresponde al valor izquierdo de la dimensión de la matriz.
  Por ejemplo,
  si nuestra \(\mathbf{A}_i\) tiene dimensión \(a \times b\),
  \(\mathbf{A}_{k + 1}: c \times d\) y \(\mathbf{A}_{j + 1}: e \times f\),
  se tiene que \(n_i = a\),
  \(n_{k + 1} = c\) y \(n_{j+1} = e\) respectivamente.

  Nos interesa: \(T[1, n]\).
  Calculamos:
  \begin{align*}
    &T[i, i] \\
    &T[i, i+1] \\
    &\quad\vdots
  \end{align*}
  Esta da sólo el costo.
  Hay que registrar con cada \(T[i, j]\) cuál fue el \(k\)
  que dio el mínimo.
  Siguiendo esos desde \(T[1, n]\) da el orden óptimo.
  \begin{ejemplo}
    Supongamos que queremos calcular el producto de matrices:
    \begin{equation*}
      \mathbf{A}
	\cdot \mathbf{B}
	\cdot \mathbf{C}
	\cdot \mathbf{D}
	\cdot \mathbf{E}
	\cdot \mathbf{F}
	\cdot \mathbf{G}
	\cdot \mathbf{H}
    \end{equation*}
    donde las dimensiones son:
    % Given in dimensiones.txt
    \begin{itemize}
    \item \(\mathbf{A} = \mathbf{A}_1 \colon 2 \times 3\)
    \item \(\mathbf{B} = \mathbf{A}_2 \colon 3 \times 4\)
    \item \(\mathbf{C} = \mathbf{A}_3 \colon 4 \times 1\)
    \item \(\mathbf{D} = \mathbf{A}_4 \colon 1 \times 9\)
    \item \(\mathbf{E} = \mathbf{A}_5 \colon 9 \times 3\)
    \item \(\mathbf{F} = \mathbf{A}_6 \colon 3 \times 7\)
    \item \(\mathbf{G} = \mathbf{A}_7 \colon 7 \times 2\)
    \item \(\mathbf{H} = \mathbf{A}_8 \colon 2 \times 8\)
    \end{itemize}
    Para la primera iteración,
    es decir,
    \(T[i, i]\)
    es claro que intentamos calcular la cantidad de productos
    que son necesarios para hacer la multiplicación \(\mathbf{A}_i\).
    Como no lo estamos multiplicando con nada más,
    se tiene que la cantidad de multiplicaciones necesarias
    es \(T[i, i] = 0\) para cualquier \(i\).
    Agregamos esta información al cuadro~\ref{11::Iteracion1}.
    \newcolumntype{M}{>{\(}r<{\)}}
    \begin{table}[ht]
      \centering
      \begin{tabular}{*{8}{M|}M|}
	 & \multicolumn{1}{c|}{1}
	     & \multicolumn{1}{c|}{2}
	     & \multicolumn{1}{c|}{3}
	     & \multicolumn{1}{c|}{4}
	     & \multicolumn{1}{c|}{5}
	     & \multicolumn{1}{c|}{6}
	     & \multicolumn{1}{c|}{7}
	     & \multicolumn{1}{c|}{8} \\
	\hline
	\multirow{2}{*}{1} & 0 &   &   &   &   &   &   & \phantom{000} \\
			   & 1 &   &   &   &   &   &   &   \\
	\hline
	\multirow{2}{*}{2} &   & 0 &   &   &   &   &   &   \\
			   &   & 2 &   &   &   &   &   &   \\
	\hline
	\multirow{2}{*}{3} &   &   & 0 &   &   &   &   &   \\
			   &   &   & 3 &   &   &   &   &   \\
	\hline
	\multirow{2}{*}{4} &   &   &   & 0 &   &   &   &   \\
			   &   &   &   & 4 &   &   &   &   \\
	\hline
	\multirow{2}{*}{5} &   &   &   &   & 0 &   &   &   \\
			   &   &   &   &   & 5 &   &   &   \\
	\hline
	\multirow{2}{*}{6} &   &   &   &   &   & 0 &   &   \\
			   &   &   &   &   &   & 6 &   &   \\
	\hline
	\multirow{2}{*}{7} &   &   &   &   &   &   & 0 &   \\
			   &   &   &   &   &   &   & 7 &   \\
	\hline
	\multirow{2}{*}{8} & \phantom{000}
			   & \phantom{000}
			   & \phantom{000}
			   & \phantom{000}
			   & \phantom{000}
			   & \phantom{000}
			   & \phantom{000}
			   & 0 \\
			   &   &    &	 &    &	   &	 &    &	  8 \\
	\hline
      \end{tabular}
      \caption{Para la primera iteración
	       no necesitamos realizar multiplicaciones.}
      \label{11::Iteracion1}
    \end{table}
    Note que cada casilla del cuadro~\ref{11::Iteracion1}
    contiene \(T[i, j]\) y el \(k\) que dio el mínimo.
    Para la segunda iteración,
    tenemos que calcular todos los \(T[i, i + 1]\),
    es decir,
    la mínima cantidad de multiplicaciones
    para obtener \(\mathbf{A}_i \cdot \mathbf{A}_{i + 1}\).
    Como solo tenemos dos matrices involucradas,
    es bastante fácil realizar este cálculo:
    \begin{itemize}
    \item \(T[1, 2] = 2 \cdot 3 \cdot 4 = \phantom{0}24\)
    \item \(T[2, 3] = 3 \cdot 4 \cdot 1 = \phantom{0}12\)
    \item \(T[3, 4] = 4 \cdot 1 \cdot 9 = \phantom{0}36\)
    \item \(T[4, 5] = 1 \cdot 9 \cdot 3 = \phantom{0}27\)
    \item \(T[5, 6] = 9 \cdot 3 \cdot 7 = 189\)
    \item \(T[6, 7] = 3 \cdot 7 \cdot 2 = \phantom{0}42\)
    \item \(T[7, 8] = 7 \cdot 2 \cdot 8 = 112\)
    \end{itemize}
    Luego,
    agregamos estos valores al cuadro~\ref{11::Iteracion1}.
    Los cambios se pueden apreciar en el cuadro~\ref{11::Iteracion2}.
    \begin{table}[ht]
      \centering
      \begin{tabular}{*{8}{M|}M|}
	 & \multicolumn{1}{c|}{1}
	     & \multicolumn{1}{c|}{2}
	     & \multicolumn{1}{c|}{3}
	     & \multicolumn{1}{c|}{4}
	     & \multicolumn{1}{c|}{5}
	     & \multicolumn{1}{c|}{6}
	     & \multicolumn{1}{c|}{7}
	     & \multicolumn{1}{c|}{8} \\
	\hline
	\multirow{2}{*}{1} & 0 & 24 &	 &    &	   &	 &    &	    \\
			   & 1 &  1 &	 &    &	   &	 &    &	    \\
	\hline
	\multirow{2}{*}{2} &   &  0 & 12 &    &	   &	 &    &	    \\
			   &   &  2 &  2 &    &	   &	 &    &	    \\
	\hline
	\multirow{2}{*}{3} &   &    &  0 & 36 &	   &	 &    &	    \\
			   &   &    &  3 &  3 &	   &	 &    &	    \\
	\hline
	\multirow{2}{*}{4} &   &    &	 &  0 & 27 &	 &    &	    \\
			   &   &    &	 &  4 &	 4 &	 &    &	    \\
	\hline
	\multirow{2}{*}{5} &   &    &	 &    &	 0 & 189 &    &	    \\
			   &   &    &	 &    &	 5 &   5 &    &	    \\
	\hline
	\multirow{2}{*}{6} &   &    &	 &    &	   &   0 & 42 &	    \\
			   &   &    &	 &    &	   &   6 &  6 &	    \\
	\hline
	\multirow{2}{*}{7} &   &    &	 &    &	   &	 &  0 & 112 \\
			   &   &    &	 &    &	   &	 &  7 &	  7 \\
	\hline
	\multirow{2}{*}{8} & \phantom{000}
			   & \phantom{000}
			   & \phantom{000}
			   & \phantom{000}
			   & \phantom{000}
			   & \phantom{000}
			   & \phantom{000}
			   & 0 \\
			   &   &    &	 &    &	   &	 &    &	  8 \\
	\hline
      \end{tabular}
      \caption{Segunda iteración.}
      \label{11::Iteracion2}
    \end{table}
    Seguimos con la tercera iteración,
    completando la diagonal siguiente.
    En esta ocasión,
    tendremos que calcular los \(T[i, i + 2]\),
    es decir,
    la cantidad de multiplicaciones mínima
    para obtener
      \(\mathbf{A}_i \cdot \mathbf{A}_{i + 1} \cdot \mathbf{A}_{i + 2}\).
    Para ello,
    comenzamos calculando \(T[1, 3]\),
    es decir:
    \begin{equation}
      \label{11::PrimerMinimo}
      T[1, 3]
	= \min_{1 \le k < 3} \{ T[1, k] + T[k + 1, 3]
				 + n_1 \cdot n_{k + 1} \cdot n_4 \}
    \end{equation}
    Vamos por partes:
    \begin{itemize}
    \item
      Para \(k = 1\):
      \begin{equation*}
	T[1, 1] + T[2, 3] + n_1 \cdot n_2 \cdot n_4
	  = 0 + 12 + 2 \cdot 3 \cdot 1
	  = 18
      \end{equation*}
    \item
      Para \(k = 2\):
      \begin{equation*}
	T[1, 2] + T[3, 3]
	       + n_1 \cdot n_3 \cdot n_4
	  = 24 + 0 + 2 \cdot 4 \cdot 1
	  = 32
      \end{equation*}
    \end{itemize}
    Por lo tanto,
    de acuerdo a lo anterior la ecuación~\eqref{11::PrimerMinimo}
    obtiene el mínimo \num{18} cuando hacemos el corte en \(k = 1\).
    Es decir,
    obtenemos el mínimo de productos
    para \(\mathbf{A}_1 \cdot \mathbf{A}_2 \cdot \mathbf{A}_3\)
    si hacemos un corte
    \begin{equation}
      \mathbf{A}_1 \cdot (\mathbf{A}_2\cdot \mathbf{A}_3)
	= \mathbf{A} \cdot (\mathbf{B} \cdot \mathbf{C})
    \end{equation}
    Agregamos estos datos a la tabla.

    Para dejar más en claro cómo calcular~\eqref{11::MinimoIteracion}
    repetiremos los pasos anteriores,
    pero para calcular \(T[4, 7]\),
    el que se puede obtener a través de:
    \begin{equation*}
      T[4, 7]
	= \min_{4 \le k < 7} \{ T[4, k] + T[k + 1, 7]
		      + n_4 \cdot n_{k + 1} \cdot n_7 \}
    \end{equation*}
    Vamos viendo:
    \begin{align*}
      k = 4 &\colon
	T[4, 4] + T[5, 7] + 1 \cdot 9 \cdot 2
	      = \phantom{0}0 + 96 + 18
	      = 114 \\
      k = 5 &\colon
	T[4, 5] + T[6, 7] + 1 \cdot 3 \cdot 2
	      = 27 + 42 + \phantom{0}6
	      = \phantom{0}75 \\
      k = 6 &\colon
	T[4, 6] + T[7, 7] + 1 \cdot 7 \cdot 2
	      = 48 + \phantom{0}0 + 14
	      = \phantom{0}62
    \end{align*}
    El mínimo se obtiene para \(k = 6\),
    de costo \num{62},
    es decir:
    \begin{equation}
      (\mathbf{A}_4 \cdot \mathbf{A}_5 \cdot \mathbf{A}_6) \cdot \mathbf{A}_7
	= (\mathbf{D} \cdot \mathbf{E} \cdot \mathbf{F}) \cdot \mathbf{G}
    \end{equation}
    Note que calculando sistemáticamente las diagonales,
    cuando se requieran valores ya estarán calculados de antes.

    El objetivo de este ejemplo
    es mostrar cómo funciona el algoritmo,
    por lo que no es necesario explicar paso a paso
    cómo obtener el resto de las casillas de la tabla.
    El resultado final es el dado en el cuadro~\ref{tab:matriz-final}.
    \begin{table}[ht]
      \centering
      \begin{tabular}{*{8}{M|}M|}
	 & \multicolumn{1}{c|}{1}
	     & \multicolumn{1}{c|}{2}
	     & \multicolumn{1}{c|}{3}
	     & \multicolumn{1}{c|}{4}
	     & \multicolumn{1}{c|}{5}
	     & \multicolumn{1}{c|}{6}
	     & \multicolumn{1}{c|}{7}
	     & \multicolumn{1}{c|}{8} \\
	\hline
	\multirow{2}{*}{1}
	   & 0 & 24 & 18 & 36 & 51 &  80 & 84 & 112 \\
	   & 1 &  1 &  1 &  3 &	 3 &   3 &  3 &	  3 \\
	\hline
	\multirow{2}{*}{2}
	   &   &  0 & 12 & 39 & 48 &  81 & 80 & 114 \\
	   &   &  2 &  2 &  3 &	 3 &   3 &  3 &	  3 \\
	\hline
	\multirow{2}{*}{3}
	   &   &    &  0 & 36 & 39 &  76 & 70 & 110 \\
	   &   &    &  3 &  3 &	 3 &   3 &  3 &	  3 \\
	\hline
	\multirow{2}{*}{4}
	   &   &    &	 &  0 & 27 &  48 & 62 &	 78 \\
	   &   &    &	 &  4 &	 4 &   5 &  6 &	  7 \\
	\hline
	\multirow{2}{*}{5}
	   &   &    &	 &    &	 0 & 189 & 96 & 240 \\
	   &   &    &	 &    &	 5 &   5 &  5 &	  7 \\
	\hline
	\multirow{2}{*}{6}
	   &   &    &	 &    &	   &   0 & 42 &	 90 \\
	   &   &    &	 &    &	   &   6 &  6 &	  7 \\
	\hline
	\multirow{2}{*}{7}
	   &   &    &	 &    &	   &	 &  0 & 112 \\
	   &   &    &	 &    &	   &	 &  7 &	  7 \\
	\hline
	\multirow{2}{*}{8} & \phantom{000}
			   & \phantom{000}
			   & \phantom{000}
			   & \phantom{000}
			   & \phantom{000}
			   & \phantom{000}
			   & \phantom{000}
			   & 0 \\
			   &   &    &	 &    &	   &	 &    &	  8 \\
	\hline
      \end{tabular}
      \caption{Tabla final}
      \label{tab:matriz-final}
    \end{table}
  \end{ejemplo}
  \begin{proof}
    Como siempre,
    demostramos que cumple con:
    \begin{description}
    \item[Estructura inductiva:]
      Dada la selección \(k\) (última) se subdivide en problemas,
      cuyas soluciones viables junto con \(k\)
      dan una solución viable para todo.
    \item[Subestructura óptima:]
      Con soluciones óptimas
      para \(1 \cdots k\) y \(k + 1 \cdots n\)
      obtenemos la solución óptima para \(1 \cdots n\),
      \emph{suponiendo \(k\)}.
    \item[Elección completa:]
      Elegimos aquel \(k\) que da el mejor \textquote{último paso},
      revisando todas las posibilidades.
    \end{description}
  \end{proof}
  Por el momento solo nos interesaba obtener el valor del costo mínimo.
  Con los valores de \(k\) registrados en la tabla
  podemos calcular el orden óptimo.
  La entrada \((1, 8)\) nos dice que el mejor corte final
  está luego de \(\mathbf{A}_3 = C\),
  o sea:
  \begin{equation*}
    (\mathbf{A} \cdot \mathbf{B} \cdot \mathbf{C})
      \cdot (\mathbf{D} \cdot \mathbf{E}
		\cdot \mathbf{F} \cdot \mathbf{G} \cdot \mathbf{H})
  \end{equation*}
  Seguimos viendo la subdivisión óptima
  para \(\mathbf{A} \cdot \mathbf{B} \cdot \mathbf{C}\) en \num{1, 3},
  que resulta ser:
  \begin{equation*}
    \mathbf{A} \cdot (\mathbf{B} \cdot \mathbf{C})
  \end{equation*}
  Similarmente,
  \num{4, 8} nos indica:
  \begin{equation*}
    (\mathbf{D} \cdot \mathbf{E}
		\cdot \mathbf{F} \cdot \mathbf{G}) \cdot \mathbf{H}
  \end{equation*}
  Continuamos de la misma forma,
  obteniendo finalmente:
  \begin{equation*}
    (\mathbf{A} \cdot (\mathbf{B} \cdot \mathbf{C}))
       \cdot ((((\mathbf{D} \cdot \mathbf{E}) \cdot \mathbf{F})
		   \cdot \mathbf{G}) \cdot \mathbf{H})
  \end{equation*}
  Es sencillo escribir un programa recursivo que recorra la tabla
  para extraer la subdivisión óptima.

\section{Subsecuencia común más larga}
\label{sec:LCS}

  En inglés conocido
  como \emph{\foreignlanguage{english}{Longest Common Subsequence}},
  LCS.
  Dadas dos palabras \(X\) e \(Y\),
  de símbolos \(x_1, x_2, \dotsc, x_m\)
  e \(y_1, y_2, \dotsc, y_n\) respectivamente,
  buscamos la secuencia más larga de símbolos \(x_{i_k} = y_{j_k}\)
  tales que \(i_k\) y \(j_k\) son ambas crecientes.

  Esta es la base del comando Unix \texttt{diff(1)},
  que compara dos archivos \(X\) e \(Y\)
  (se consideran las líneas como \textquote{símbolos}),
  y entrega las líneas eliminadas e insertadas para cambiar \(X\) a \(Y\):
  \begin{itemize}
  \item
    Hallar la subsecuencia común más larga (LCS) entre ellos.
  \item
    Marcar líneas agregadas/borradas
  \end{itemize}
  Supongamos que tenemos dos archivos:
  \(X\) e \(Y\),
  cuyo contenido se muestra en el cuadro~\ref{12::ContenidoArchivos}.
  \newcolumntype{T}{>{\ttfamily}l}
  \begin{table}[ht]
    \centering
    \begin{tabular}{T|T}
      \multicolumn{1}{c|}{\(X\)}&\multicolumn{1}{c}{\(Y\)}\\
      \hline
      foo	& bar \\
      bar	& xyzzy \\
      baz	& plugh \\
      quux	& baz \\
      windows & foo \\
	      & quux \\
	      & linux
    \end{tabular}
    \caption{Los archivos \(X\) e \(Y\).
	     Cada fila de la tabla
	     es una línea del archivo.}
    \label{12::ContenidoArchivos}
  \end{table}
  En consecuencia,
  si ejecutamos \lstinline[language = sh]!diff X Y!
  obtenemos como resultado,
  la columna \textquote{Resultado} del cuadro~\ref{12::DiffXY}.
  \begin{table}[ht]
    \centering
    \begin{tabular}{c|T|T|TT}
      Línea & \multicolumn{1}{c|}{\(X\)}
	    & \multicolumn{1}{c|}{\(Y\)}
	    & \multicolumn{2}{c}{\textrm{Resultado}} \\
      \hline
      1 & foo	    & bar   & - & foo \\
      2 & bar	    & xyzzy &	& bar \\
      3 & baz	    & plugh & + & xyzzy \\
      4 & quux	  & baz	  & + & plugh \\
      5 & windows & foo	  &	& baz \\
      6 &	    & quux  & + & foo \\
      7 &	    & linux &	& quux \\
      8 &	    &	    & - & windows \\
      9 &	    &	    & + & linux
    \end{tabular}
    \caption{La columna \textquote{Resultado}
	     resume las operaciones.}
    \label{12::DiffXY}
  \end{table}
  Esta operación es crítica en sistemas de control de versiones
  (para ahorrar espacio almacenan solo las diferencias,
   que suelen ser pequeñas entre versiones sucesivas),
  y es fundamental para mostrar diferencias entre versiones.
  Mucho de la biología molecular
  es determinar diferencias
  entre secuencias de aminoácidos de proteínas
  o de genes similares

\section{Aspectos formales}

  Nos dan arreglos \(X[1, \ldots, n]\), \(Y[1, \ldots, m]\),
  palabras sobre un alfabeto (\textquote{símbolo} es una línea).
  Hallar la secuencia de pares de índices
  (números de línea)
  \((x_1, y_1)\), \((x_2, y_2)\), \ldots, \((x_q, y_q)\) tales que:
  \begin{align*}
    x_1	   &< x_2 < \dotsb < x_q \qquad	 \forall x_i \in \mathbb{N} \\
    y_1	   &< y_2 < \dotsb < y_q \qquad	 \forall y_i \in \mathbb{N} \\
    X[x_i] &= Y[y_i] \qquad \text{para \(1 \le i \le q\)}
  \end{align*}
  Interesa la secuencia más larga
  (máximo \(q\),
   correspondiente a la cantidad de coincidencias).
  Para ello,
  consideremos \(X[n]\), \(Y[m]\).
  Tres opciones:
  \begin{enumerate}
  \item
    \(X[n]\) queda fuera de la subsecuencia.
    En consecuencia, lo marcamos con \texttt{(-)}
  \item
    \(Y[m]\) queda fuera de la subsecuencia.
    Para efectos de \texttt{diff} marcamos con \texttt{(+)}
  \item
    Si \(X[n] = Y[m]\), hacer \(x_q = n\), \(y_q = m\).
  \end{enumerate}
  Pueden quedar fuera ambos,
  pero eso resulta automáticamente de eliminar uno y luego el otro.
  Notar que si \(X[n] \ne Y[m]\) no pueden pertenecer ambos a la LCS.

  En resumen,
  hay tres opciones
  que dan lugar a los siguientes subproblemas:
  \begin{enumerate}
  \item
    \(X[1, \dotsc, n - 1]\), \(Y[1, \dotsc, m]\)
  \item
    \(X[1, \dotsc, n]\), \(Y[1, \dotsc, m - 1]\)
  \item
    Solo si \(X[n] = Y[m]\)
    considerar \(X[1, \dotsc, n - 1]\), \(Y[1, \dotsc, m - 1]\)
    (y contabilizar una coincidencia)
  \end{enumerate}

  Sea \(\mathrm{LCS}(A, B)\)
  la subsecuencia común más larga entre \(A\) y \(B\).
  Entonces
  (hasta el momento solo nos interesa el largo de la subsecuencia óptima,
   después veremos como encontrar la secuencia):
  \begin{align*}
    \lvert \mathrm{LCS}(X, Y) \rvert
      = \max\{\lvert \mathrm{LCS}(X[1, \dotsc, n - 1), Y \rvert] + 0,
	      \lvert \mathrm{LCS}(X, Y[1, \dotsc, m - 1) \rvert] + 0, \\
	      \lvert \mathrm{LCS}(X[1, \dotsc, n - 1],
					Y[1, \dotsc, m - 1]) \rvert
			  + 1 \}
  \end{align*}
  Esto sugiere un arreglo \(L[i, j]\):
  \begin{equation}
    L[i, j]
      = \lvert \mathrm{LCS}(X[1, \ldots, i), Y[1, \ldots, j \rvert]
  \end{equation}
  Sabemos \(L[0, j] = L[i, 0] = 0\).
  Para calcular \(L[i, j]\)
  necesitamos \(L[i-1, j]\), \(L[i, j-1]\), \(L[i-1, j-1]\) (posiblemente).
  Llenar el arreglo,
  calculando \(L[i, k]\) para \(i\) de \num{1} a \(n\),
  llenando los \(j\) de \num{1} a \(m\).
  Vemos que el costo total es \(O(n\cdot m)\).

\section{Árboles binarios óptimos}
\label{sec:optimal-BST}

  Nos dan un arreglo ordenado de claves \(A[1, \dotsc, n]\)
  con sus respectivas frecuencias de búsqueda \(f[1, \dotsc, n]\).
  Buscamos crear un árbol binario
  para el cual el costo total de las búsquedas sea mínimo.

  Requerimos una forma de plantear la cantidad a optimizar.
  Fijemos un árbol binario de búsqueda \(T\) de las claves \(A\),
  con la clave \(A[i]\) en el nodo \(v_i\),
  el costo total de las búsquedas en \(T\) es proporcional a:
  \begin{equation*}
    C(T, f)
      = \sum_i f[i] \cdot \text{número de ancestros de \(v_i\) en \(T\)}
  \end{equation*}
  Podemos particionar \(C\) según subárboles del vértice raíz \(v_r\):
  \begin{equation*}
    C(T, f)
      = \sum_i f[i]
	  + \sum_{1 \le i < r}
	      f[i] \cdot \text{\# ancestros de \(v_i\)
		 en \(\mathrm{left}(T)\)}
	  + \sum_{r < i \le n}
	      f[i] \cdot \text{\# ancestros de \(v_i\)
		 en \(\mathrm{right}(T)\)}
  \end{equation*}
  Las dos sumas son de la misma forma de nuestro original,
  obtenemos la recursión:
  \begin{equation*}
    C(T, f)
      = \sum_i f[i] + C(\operatorname{left}(T)) + C(\operatorname{right}(T))
  \end{equation*}
  El caso base es para \(n = 0\),
  buscar en el árbol vacío tiene costo cero.

  Es claro que las claves no inciden,
  dada la raíz \(A[r]\)
  el subárbol derecho y el izquierdo deben ser óptimos para las claves
  en sus respectivos rangos.
  Fijemos el arreglo de frecuencias,
  y sea \(\mathrm{Opt}(i, k)\) el costo total
  de buscar en el árbol óptimo para las claves \(A[i, \dotsc, k]\).
  Nuestra recurrencia se transforma en:
  \begin{equation*}
    \mathrm{Opt}(i, k)
      = \begin{cases}
	   0  & i > k \\
	   \sum_{i \le j \le k} f[j]
	      + \min_{i \le r \le k} \{ \mathrm{Opt}(i, r - 1)
					   + \mathrm{Opt}(r + 1, k) \}
	      & \text{caso contrario}
	 \end{cases}
  \end{equation*}
  El programa resulta más simple
  (y más eficiente)
  si precalculamos el primer término:
  \begin{equation*}
    F[i, k]
      = \sum_{i \le j \le k} f[j]
  \end{equation*}
  Podemos calcular los valores requeridos en tiempo \(O(n^2)\) mediante
  (¡sorpresa!)
  programación dinámica,
  algoritmo~\ref{alg:F[i,k]}.
  \begin{algorithm}
    \DontPrintSemicolon\Indp

    \Procedure{\(\mathrm{InitF}(f)\)}{
      \For{\(i \gets 1\) \KwTo \(n\)}{
	\(F[i, i - 1] \gets 0\) \;
	\For{\(k \gets i\) \KwTo \(n\)}{
	  \(F[i, k] \gets F[i, k - 1] + f[k]\) \;
	}
      }
    }
    \caption{Calcular \(F[i, k]\)}
    \label{alg:F[i,k]}
  \end{algorithm}
  (Basta calcular \(\tilde{F}[k] = \sum_{1 \le j \le k} f[j]\),
   y obtener \(F[i, k] = \tilde{F}(k) - \tilde{F}[i - 1]\),
   ahorrando espacio en el proceso;
   pero esto es irrelevante en este caso.
   Por lo demás,
   no es el momento de preocuparse de optimizaciones.)
   Nuestra recurrencia se simplifica a:
  \begin{equation*}
    \mathrm{Opt}(i, k)
      = \begin{cases}
	   0  & i > k \\
	   F[i, k]
	      + \min_{i \le r \le k} \{ \mathrm{Opt}(i, r - 1)
					   + \mathrm{Opt}(r + 1, k) \}
	      & \text{caso contrario}
	 \end{cases}
  \end{equation*}
  Estamos listos para seguir nuestra estrategia.
  \begin{description}
  \item[Subproblemas:]
    Cada subproblema queda descrito por dos enteros,
    \(1 \le i \le n + 1\) y \(0 \le k \le n\).
  \item[Estructura de datos:]
    Se requiere un arreglo \(\mathrm{Opt}[1 .. n + 1, 0 .. n]\)
    para registrar los valores
    (solo se usan las entradas \(\mathrm{Opt}[i, j]\)
     con \(j \ge i - 1\),
     pero da lo mismo;
     si realmente se requiriera ahorrar espacio
     se puede almacenar solo la mitad usada del arreglo).
   \item[Dependencias:]
     Cada entrada \(\mathrm{Opt}[i, k]\)
     depende de las entradas \(\mathrm{Opt}[i, j - 1]\)
     y \(\mathrm{Opt}[j + 1, k]\)
     para \(j\) tales que \(i \le j \le k\).
     Vale decir,
     las que están directamente a la izquierda o debajo.
     El algoritmo~\ref{alg:compute-OPT} calcula \(\mathrm{Opt}[i, k]\).
     \begin{algorithm}
       \DontPrintSemicolon\Indp

       \Procedure{\(\mathrm{ComputeOpt}(i, k)\)}{
	 \(\mathrm{Opt}[i, k] \gets \infty\) \;
	 \For{\(r \gets i\) \KwTo \(n\)}{
	   \(\mathrm{tmp}
	       \gets \mathrm{Opt}[i, r - 1]
			     + \mathrm{Opt}[r + 1, k]\) \;
	   \If{\(\mathrm{Opt}[i, k] > \mathrm{tmp}\)}{
	     \(\mathrm{Opt}[i, k] \gets \mathrm{tmp}\) \;
	   }
	 }
	 \(\mathrm{Opt}[i, k]
	     \gets \mathrm{Opt}[i, k] + F[i, k]\) \;
       }
       \caption{Cómputo de \(\mathrm{Opt}[i, k]\)}
       \label{alg:compute-OPT}
     \end{algorithm}
   \item[Orden de evaluación:]
     Hay varias opciones.
     Una es llenar el arreglo una diagonal a la vez,
     partiendo con los valores triviales \(\mathrm{Opt}[i, i - 1]\)
     y llegando a \(\mathrm{Opt}[1, n]\),
     algoritmo~\ref{alg:opt-BST-1}.
     \begin{algorithm}
       \DontPrintSemicolon\Indp

       \Function{\(\mathrm{OptimalBST}(f)\)}{
	 \(\mathrm{InitF}(f)\) \;
	 \For{\(i \gets 1\) \KwTo \(n + 1\)}{
	   \(\mathrm{Opt}[i, i - 1] \gets 0\) \;
	 }
	 \For{\(d \gets 0\) \KwTo \(n - 1\)}{
	   \For{\(i \gets 1\) \KwTo \(n - d\)}{
	     \(\mathrm{ComputeOpt}(i, i + d)\) \;
	   }
	 }
	 \Return \(\mathrm{Opt}[1, n]\) \;
       }

       \caption{Llenar en diagonal}
       \label{alg:opt-BST-1}
     \end{algorithm}
     Otras opciones son llenar por filas de abajo arriba,
     algoritmo~\ref{alg:opt-BST-2};
     o llenar por columnas de izquierda a derecha,
     llenando cada columna de abajo arriba,
     algoritmo~\ref{alg:opt-BST-2}.
     \begin{algorithm}
       \DontPrintSemicolon\Indp

       \Function{\(\mathrm{OptimalBST}(f)\)}{
	 \(\mathrm{InitF}(f)\) \;
	 \For{\(i \gets n + 1\) \Downto \num{1}}{
	   \(\mathrm{Opt}[i, i - 1] \gets 0\) \;
	   \For{\(j \gets i\) \KwTo \(n\)}{
	     \(\mathrm{ComputeOpt}(i, j)\) \;
	   }
	 }
	 \Return \(\mathrm{Opt}[1, n]\) \;
       }

       \caption{Llenar por filas}
       \label{alg:opt-BST-2}
     \end{algorithm}
     \begin{algorithm}
       \DontPrintSemicolon\Indp

       \Function{\(\mathrm{OptimalBST}(f)\)}{
	 \(\mathrm{InitF}(f)\) \;
	 \For{\(j \gets 0\) \KwTo \(n + 1\)}{
	   \(\mathrm{Opt}[j + 1, j] \gets 0\) \;
	   \For{\(i \gets j\) \Downto \num{1}}{
	     \(\mathrm{ComputeOpt}(i, j)\) \;
	   }
	 }
	 \Return \(\mathrm{Opt}[1, n]\) \;
       }

       \caption{Llenar por columnas}
       \label{alg:opt-BST-3}
     \end{algorithm}
   \end{description}
   Es claro que el algoritmo resultante toma tiempo \(O(n^3)\)
   y usa espacio \(O(n^2)\).

\section{Insertar caracteres para palíndromo}
\label{sec:insert-palindrome}

  Dado un \emph{\foreignlanguage{english}{string}}
  \(x_{1 \ldots n}\),
  hallar el mínimo número de caracteres a insertar para obtener un palíndromo
  (se lee igual de adelante hacia atrás que de atrás hacia adelante).
  Por ejemplo,
  para \texttt{Ab3bd} son \num{2}
  (obtenemos \texttt{dAb3bAd} o \texttt{Adb3dbA}
   insertando '\texttt{d}' y '\texttt{A}').

  Una forma de obtener la respuesta es obtener \(x^{\mathrm{R}}\),
  el reverso de \(x\),
  y calcular
    \(\lvert x \rvert - \mathrm{\operatorname{LCS}}(x, x^{\mathrm{R}})\).

  Otra forma de resolverlo es planteando directamente
  programación dinámica.
  Siguiendo nuestra estrategia:

\subsection{Definir subproblemas}
\label{sec:definir-subproblemas}

  Sea \(D_{i,j}\) el mínimo número de caracteres a insertar
  para transformar \(x_{i \ldots j}\) en un palíndromo.

\subsection{Hallar la recurrencia}
\label{sec:hallar-recurrencia}

  Consideremos el palíndromo más corto \(y_{1 \ldots k}\)
  que contiene \(x_{i \ldots j}\).
  Entonces debe ser \(y_1 = x_i\) o \(y_k = x_j\)
  (¿porqué?).
  Esto significa que \(y_{2 \ldots k}\)
  es óptimo para \(x_{i + 1 \ldots j}\)
  o para \(x_{i \ldots j - 1}\)
  o para \(x_{i + 1 \ldots j - 1}\)
  (este caso solo es posible si \(y_1 = y_k = x_i = x_j\)).

  Casos base son \(D_{i, i} = D_{i, i - 1} = 0\) para todo \(i\).

  La recurrencia es:
  \begin{equation*}
    D_{i, j}
      = \begin{cases}
	  1 + \min\{ D_{i + 1, j}, D_{i, j - 1}\} & x_i \ne x_j \\
	  D_{i + 1, j - 1}			  & x_i = x_j
	\end{cases}
  \end{equation*}

\subsection{Programa}
\label{sec:programa}

  Los \(D_{i, j}\) deben calcularse en orden de \(j - i\) creciente.
  El esqueleto del programa en C es el listado~\ref{lst:palindrome-skel}.
  \lstinputlisting[float,
		   caption = {Esqueleto de obtener palíndromo},
		   label = lst:palindrome-skel]
		  {dynamic-programming/palindrome-skeleton.c}
  El detalle queda de ejercicio.

\section{Máximo conjunto independiente de un árbol}
\label{sec:IndependentSet-tree}

  Definimos \emph{conjunto independiente} en un grafo
  como un conjunto de vértices que no están conectados entre sí,
  y vimos que determinar si un grafo cualquiera
  tiene un conjunto independiente de tamaño \(k\)
  (el problema \textsc{Independent~Set})
  es \NP\nobreakdash-completo.
  Sin embargo,
  en el caso de árboles hay una solución eficiente al problema de búsqueda.

  Sin pérdida de generalidad,
  supongamos un árbol \(T\) con raíz \(r\).
  Observamos que el tamaño del máximo conjunto independiente de \(T\)
  puede incluir la raíz \(r\)
  (en cuyo caso no incluye a ninguno de sus hijos)
  o la excluye
  (en cuyo caso los hijos de \(r\) pueden ser miembros.
  Esto lleva a considerar como subproblemas
  determinar el máximo conjunto independiente de árboles,
  incluyendo y excluyendo a la raíz.
  Si llamamos \(I(T)\) al tamaño del máximo conjunto independiente
  que incluye la raíz de \(T\),
  y \(E(T)\) al tamaño del máximo conjunto independiente
  que no incluye la raíz de \(T\),
  tenemos que el tamaño del máximo conjunto independiente de \(T\)
  es \(\max \{ I(T), E(T) \}\),
  y podemos escribir las recurrencias:
  \begin{align*}
    I(T)
      & = \begin{cases}
	    0	 & \text{si \(T\) es vacío} \\
	    1 + \sum_{T' \mathop{\text{subárbol de}} T} \, E(T')
		\phantom{\max \{, E \}}	 % Works by trial and error...
		 & \text{caso contrario}
	  \end{cases} \\
    E(T)
      & = \begin{cases}
	    0	 & \text{si \(T\) es vacío} \\
	    \sum_{T' \mathop{\text{subárbol de}} T} \,
		     \max \{ I(T'), E(T') \}
		 & \text{caso contrario}
	  \end{cases}
  \end{align*}
  En este caso,
  para cada vértice debemos almacenar dos valores
  (\(I(T)\) y \(E(T)\)),
  la estructura natural para hacerlo es el mismo árbol.
  Un orden de cálculo obvio de los valores según las recurrencias
  es en postorden.
  El tiempo claramente es lineal en el número de vértices del árbol.

\section*{Ejercicios}
\label{sec:ejercicios-programacion-dinamica}

  \begin{enumerate}
  \item
    Derive la complejidad del algoritmo
    para determinar el orden óptimo de multiplicación de matrices,
    suponiendo que le dan a multiplicar \(n\) matrices.
    Compare con evaluar todos los órdenes posibles.
  \item
    Considere los valores \(\langle a_n \rangle\) y \(\langle b_n \rangle\)
    definidos mediante:
    \begin{align*}
      a_0 &= a_1 = 1 \\
      b_0 &= b_1 = 2 \\
      a_n
	&= a_{n - 2} + b_{n - 1} \\
      b_n
	&= a_{n - 1} + b_{n - 2}
    \end{align*}
    Podemos calcularlos mediante las funciones recursivas
    dadas en el algoritmo~\ref{alg:seq-ab}.
    \begin{algorithm}
      \DontPrintSemicolon\Indp

      \Function{\(\mathrm{CalculeA}(n)\)}{
	\eIf{\(n < 2\)}{
	  \Return \num{1} \;
	}{
	  \Return \(\mathrm{CalculeA}(n - 2)
		      + \mathrm{CalculeB}(n - 1)\) \;
	}
      }

      \Function{\(\mathrm{CalculeB}(n)\)}{
	\eIf{\(n < 2\)}{
	  \Return \num{2} \;
	}{
	  \Return \(\mathrm{CalculeA}(n - 1)
		      + \mathrm{CalculeB}(n - 2)\) \;
	}
      }
      \caption{Cálculo obvio de \(a_n\) y \(b_n\)}
      \label{alg:seq-ab}
    \end{algorithm}
    \begin{enumerate}
    \item
      Demuestre que el tiempo que demanda~\ref{alg:seq-ab}
      para calcular \(a_n\) es exponencial.
    \item
      Describa un algoritmo más eficiente para calcular \(a_n\),
      y derive su complejidad.
    \item
      Escriba una versión memoizada del algoritmo.
    \item
      Escriba una versión basada en programación dinámica.
      ¿Qué datos deben retenerse?
    \end{enumerate}
  \item
    Escriba un programa que resuelva
    el caso general de asignación de proyectos a plantas,
    para un número arbitrario de plantas y proyectos.
    Entregue no solo el mejor retorno,
    sino también los proyectos a ser realizados.
    ¿Qué modificaciones deben hacerse a las recurrencias
    para acomodar el caso en que la opción de no hacer nada
    no se muestre explícitamente?
  \item
    ¿Cuál es la complejidad del algoritmo
    para resolver el problema de subsecuencia creciente más larga,
    programa~\ref{lst:LIS}?
  \item
    Extienda el programa~\ref{lst:LIS}
    para entregar una secuencia creciente más larga.
  \item
    Escriba un programa que construya el árbol binario de búsqueda óptimo.
  \item
    Decida una representación como estructura de datos de árboles con raíz,
    y dada esa estructura escriba un programa que dé un conjunto independiente
    de tamaño máximo para su árbol.
    ¿Cuál es la complejidad de su algoritmo?
  \item
    Un \emph{palíndromo} es una palabra
    que se lee igual de adelante o de atrás,
    como \emph{arenera} o \emph{reconocer}.
    Cualquier palabra puede verse como una secuencia de palíndromos,
    considerando una única letra como un palíndromo de largo \num{1}.
    Interesa obtener la división de una palabra
    en el mínimo número de palíndromos,
    \(\mathrm{MinPal}(\sigma)\).
    \begin{enumerate}
    \item
      Describa la recurrencia para \(\mathrm{MinPal}(\sigma)\)
      en términos de subpalabras de \(\sigma\).
    \item
      Describa un algoritmo que toma tiempo \(O(\lvert \sigma \rvert^3)\)
      para hallar \(\mathrm{MinPal}(\sigma)\).
    \end{enumerate}
  \item
    Considere una hoja rectangular de papel cuadriculado,
    algunos de cuyos cuadritos están marcados con X.
    Nos interesa determinar para cada cuadradito
    el largo de la secuencia de X de largo máximo que pasa por él,
    en vertical, horizontal o diagonal.
  \item
    Nos encargan escribir frases
    usando un conjunto de azulejos predefinidos
    \(\{A_0, \dotsc, A_{m - 1}\}\) con secuencias de letras
    (y espacios).
    Debemos determinar si se puede escribir la frase \(S\)
    con los azulejos dados,
    o sea si:
    \begin{equation*}
      S
	= A_{i_1} A_{i_2} \ldots A_{i_n}
    \end{equation*}
    para alguna secuencia \(\langle i_1, i_2, \dotsc, i_n \rangle\).
    Los azulejos pueden repetirse,
    suponemos que hay suficientes de cada uno de ellos.
  \item
    Sean \(x = x_1 x_2 \ldots x_m\),
    \(y = y_1 y_2 \ldots y_n\),
    \(z = z_1 z_2 \ldots z_{m + n}\)
    dos palabras
    (los \(x_i\), \(y_i\) y \(z_i\) son símbolos individuales).
    Un \emph{barajamiento} de \(x\) e \(y\)
    es una palabra de largo \(\lvert x \rvert + \lvert y \rvert\)
    en la cual aparecen \(x\) e \(y\) como subsecuencias no traslapantes.
    La pregunta es si \(z\) es un barajamiento de \(x\) e \(y\),
    y dar una posible división de \(z\) en subsecuencias
    (note que pueden haber varias posibilidades).
  \item
    Considere una gramática de contexto libre \(G\)
    en forma normal de Chomsky
    (ver por ejemplo Hopcroft, Motwani y Ullman~%
      \cite[capítulo~7]{hopcroft07:_introd_autom_theor_languag_comput})
    y una palabra \(\sigma = a_1 a_2 \ldots a_n\),
    donde queremos determinar si \(\sigma \in \mathscr{L}(G)\).
    Esto puede hacerse mediante programación dinámica,
    registrando para el rango \(a_i .. a_j\) el conjunto de no-terminales
    que generan esa palabra,
    si el símbolo de partida pertenece al conjunto de no-terminales
    que generan \(a_1 .. a_n = \sigma\),
    la respuesta es si
    (y podemos extraer una derivación).
    \begin{enumerate}
    \item
      Detalle este algoritmo
      (se le conoce como Cocke-Younger-Kasami,
       o CYK,
       por sus inventores,
       que independientemente plantearon esencialmente la misma idea~%
       \cite{cocke70:_prog_lang_compilers,
	     kasami65:_effic_recog_syntax_analy_algor,
	     younger67:_recognizing_context_free_language}).
    \item
      ¿Cuál es la complejidad de su algoritmo?
    \end{enumerate}
  \item
    Considere el problema de asignación de tareas,
    pero ahora cada tarea tiene un valor,
    y nos interesa la colección de tareas de máximo valor.
    Formalmente,
    la tarea \(i\) comienza en el instante \(s_i\) y tiene duración \(\ell_i\)
    (está activa en el intervalo abierto \([s_i, s_i + \ell_i)\)),
    y tiene valor \(v_i\).
    Buscamos una colección de tareas que no traslapan
    tal que la suma de los valores es máxima.
  \item
    Convénzase que la técnica basada en LCS
    para el número de caracteres a insertar para lograr un palíndromo
    (sección~\ref{sec:insert-palindrome})
    es correcta.
  \item
    Complete el programa C
    que calcula el número de caracteres a insertar para lograr un palíndromo
    (sección~\ref{sec:programa}).
    ¿Qué datos adicionales debe almacenar
    para poder construir la secuencia de inserciones que da el palíndromo?
  \end{enumerate}

% To do:
% - Many exercises/problems at
% prismoskills.appspot.com/lessons/Dynamic_Programming/Chapter_01_Introduction.jsp
% www.geeksforgeeks.org/category/dynamic-programming
% https://courses.engr.illinois.edu/cs374/sp2018/A/notes/03-dynprog.pdf,
%  https://courses.engr.illinois.edu/cs374/sp2018/A/labs/lab7bis.pdf,
%  https://courses.engr.illinois.edu/cs374/sp2018/A/labs/lab8.pdf,
%  https://courses.engr.illinois.edu/cs374/sp2018/A/labs/lab8bis.pdf
% Google CodeJam
% web.stanford.edu/class/cs97si/04-dynamic-programming.pdf
% https://www.topcoder.com/community/competitive-programming/tutorials/dynamic-programming-from-novice-to-advanced/
% https://www.hackerearth.com/practice/algorithms/dynamic-programming/introduction-to-dynamic-programming-1/tutorial/
% https://www.sanfoundry.com/dynamic-programming-problems-solutions/
% https://brilliant.org/wiki/problem-solving-dynamic-programming/?subtopic=algorithms&chapter=dynamic-programming

\bibliography{../referencias}

%%% Local Variables:
%%% mode: latex
%%% TeX-master: "../INF-221_notas"
%%% ispell-local-dictionary: "spanish"
%%% End:

% LocalWords:  memoización memoizar computines US Subset Sum INF LCS
% LocalWords:  SubsetSum pseudopolinomiales Subsecuencia subsecuencia
% LocalWords:  ésimo english Longest Common Subsequence diff foo baz
% LocalWords:  xyzzy plugh quux windows linux TT particionar left Opt
% LocalWords:  right string Ab bd dAb bAd Adb dbA Independent Set CYK
% LocalWords:  postorden memoizada subpalabras cuadritos cuadradito
% LocalWords:  barajamiento subsecuencias traslapantes

\bibliographystyle{babplain-fl}

\chapter{Más de programación dinámica}
\label{cha:programacion-dinamica-cont}

  Hasta el momento hemos tratado casos
  en los cuales la aplicación de las recurrencias es directo.
  Mostraremos un par de ejemplos en los cuales se requiere trabajo previo.
  Vienen en parte de las notas de Ignjatović~\cite{ignjatovic16:_DP}.

\section{Caminos más cortos en un grafo}
\label{sec:shortest-paths}

  Vimos cómo calcular los caminos más cortos
  desde un vértice de un grafo a los demás
  (algoritmo de Dijkstra~\ref{alg:Dijkstra}).
  Buscamos ahora hallar los largos de los caminos más cortos
  entre cada par de vértices
  El algoritmo resultante se conoce bajo el nombre de Floyd-Warshall~%
    \cite{floyd62:_shortest_path, warshall62:_theo_boolean_matrices},
  por quienes plantearon las ideas fundamentales.

  Para aplicar programación dinámica al problema de hallar el camino más corto
  entre un par dado de vértices \(u, v)\),
  es claro que si el camino óptimo pasa por el vértice \(x\) intermedio,
  los caminos \(u, x\) y \(x, v\) son óptimos.
  El problema es que esto no reduce para nada el problema entre manos.
  Pero podemos restringir arbitrariamente los nodos intermedios permitidos,
  para obtener subproblemas manejables.
  Calculemos caminos óptimos que incluyen un subconjunto de los vértices
  como pasos intermedios,
  incrementando el conjunto permitido paso a paso.
  Sea \(G = (V, E)\) un grafo,
  \(w \colon E \to \mathbb{R}\) una función de costos
  (podemos permitir costos negativos,
   siempre que no hayan ciclos de costo negativo
   el camino más corto entre un par de vértices está bien definido).
  Podemos plantear la recursión
  para un conjunto de vértices \(U \subseteq V\)
  y un vértice cualquiera \(x \notin U\).
  El camino de costo mínimo entre \(u\) y \(v\)
  que pasa solo por los vértices en \(U \cup \{x\}\)
  es el mínimo entre el camino mínimo que solo pasa por vértices en \(U\)
  y el camino mínimo \(u \cdots x \cdots v\) que pasa por \(x\),
  que a su vez está compuesto por el camino óptimo entre \(u\) y \(x\)
  y el camino óptimo entre \(x\) y \(v\),
  ambos solo pasando por vértices de \(U\).
  Esto da la recursión requerida.

  Numeremos los vértices en orden arbitrario,
  con lo que podemos usar vértices
  como índices en un arreglo bidimensional \(d\)
  de distancias entre vértices.
  Definimos \(d_{i i} = 0\),
  \(d_{i j} = w(i j)\) si \(i j\) es un arco
  y \(d_{i j} = \infty\) si \(i j\) no es un arco.
  Llamemos \(d^{(k)}_{i j}\)
  al costo del camino de mínimo costo entre \(i\) y \(j\)
  que pasa únicamente por vértices \(1, \dotsc, k\) entremedio.
  Es claro que:
  \begin{align*}
    d^{(0)}_{i j}
      &= d_{i j} \\
    d^{(k + 1)}_{i j}
      &= \min\{ d^{(k)}_{i j}, d^{(k)}_{i k} + d^{(k)}_{k j}
  \end{align*}
  La idea es entonces almacenar \(d^{(k)}_{i j}\) en un arreglo tridimensional.
  Pero podemos ahorrar espacio reconociendo
  que podemos usar el mismo arreglo bidimensional,
  ya que los valores relevantes en la fila y columna \(k\)
  no cambian en la iteración \(k\)
  (esto es intuitivamente obvio,
   caminos que comienzan o terminan en \(k\) no cambian al incluir \(k\)
   como paso intermedio permitido).
  Formalmente:
  \begin{align*}
    d^{(k)}_{i k}
      &= \min\{d^{(k - 1)}_{i k}, d^{(k - 1)}_{i k} + d^{(k - 1)}_{k k}\} \\
      &= \min\{d^{(k - 1)}_{i k}, d^{(k - 1)}_{i k} + 0\} \\
      &= d^{(k - 1)}_{i k} \\
    d^{(k)}_{k j}
      &= \min\{d^{(k - 1)}_{k j}, d^{(k - 1)}_{k k} + d^{(k - 1)}_{k j}\} \\
      &= \min\{d^{(k - 1)}_{k j}, 0 + d^{(k - 1)}_{k j}\} \\
      &= d^{(k - 1)}_{k j}
  \end{align*}
  Nuestro algoritmo final es~\ref{alg:Floyd-Warshall}.
  \begin{algorithm}[htbp]
    \DontPrintSemicolon\Indp

    \Function{\(\mathrm{FloydWarshall}(G = (V, E), w)\)}{
      \(d_{u v} =
          \begin{cases}
            0		 & u = v \\
            \infty	 & u v \notin E \\
            w(u v)	 & u v \in E
          \end{cases}\) \;
       \For{\(k \gets 1 \KwTo \lvert V \rvert\)}{
         \(d_{u v}
             \gets \min\{d_{u v}, d_{u k} + d_{k v}\}\) \;
       }
       \Return \(\mathbf{d}\) \;
    }
    \caption{Camino más corto entre cada par de vértices de \(G\)}
    \label{alg:Floyd-Warshall}
  \end{algorithm}
  Es claro que el tiempo demandado es \(O(\lvert V \rvert^3\),
  cosa bastante sorprendente habiendo \(O(\lvert V \rvert^2\)
  posibles arcos en \(G\).

  El algoritmo de Floyd-Warshall tiene la ventaja de acomodar costos negativos
  (siempre que no haya ciclos de costo negativo).
  Tiene la ventaja adicional que es fácil de paralelizar,
  los cálculos para cada vértice
  pueden hacerse en paralelo sin interferencias.
  Por esta razón alguna variante de este algoritmo
  es popular en aplicaciones de ruteo en redes.

\section{Torre de Tortugas}
\label{sec:torre-de-tortugas}

  Nos dan \(n\) tortugas,
  para cada una se da su peso y su resistencia.
  La resistencia de una tortuga es el peso máximo
  que es capaz de soportar sin romper su caparazón.
  Se busca el máximo número de tortugas que se pueden apilar
  sin romper sus caparazones.

  Llamemos \(T_1, \dotsc, T_n\) a las tortugas
  (en orden arbitrario),
  donde el peso de \(T_i\) es \(W(T_i)\) y su fuerza \(S(T_i)\).
  Diremos que  una torre de tortugas es \emph{legítima}
  si la fuerza de cada tortuga es mayor o igual
  al peso de las tortugas sobre ella.
  Ordenamos las torres desde la punta a la base.

  La programación dinámica
  consiste en construir recursivamente una solución al problema
  de soluciones a subproblemas.
  Podemos plantear por ejemplo para \(1 \le j \le n\)
  el subproblema \(P(j)\) de hallar el máximo número de tortugas
  del conjunto \(\{ T_1, \dotsc, T_j \}\) que pueden apilarse.
  Lamentablemente,
  este planteo no permite recursión.
  Nos interesaría hallar una solución a \(P(j)\)
  vía soluciones a todos los problemas \(P(i)\) para \(1 \le i < j\).
  Pero la cadena más larga construida
  con tortugas de \(\{ T_1, \dotsc, T_j \}\)
  puede que incluya a \(T_j\),
  pero no en la última posición.
  Por tanto la solución óptima a \(P(j)\)
  no siempre es una simple extensión de una solución óptima
  a algún \(P(i)\) anterior.

  Debemos hallar un ordenamiento adecuado
  junto con un subconjunto de subproblemas que permitan recurrencia.
  Hallar tal ordenamiento no es simple.

  \begin{proposition}
    \label{prop:W+S}
    Si hay una torre legítima de altura \(k\),
    hay una torre legítima de altura \(k\)
    en orden no-decreciente de peso más fuerza.
  \end{proposition}
  \begin{proof}
    Demostraremos que cualquier torre legítima
    puede reordenarse para dar otra torre legítima
    en el orden indicado.
    Sea \(\langle t_1, \dotsc, t_m \rangle\) una torre legítima,
    basta demostrar que si dos tortugas consecutivas cumplen:
    \begin{equation*}
      W(t_{i + 1}) + S(t_{i + 1})
        < W(t_i) + S(t_i)
    \end{equation*}
    podemos intercambiar esas tortugas obteniendo otra torre legítima.
    Con esto,
    podemos usar la idea del método de burbuja
    para ordenar las tortugas en orden creciente de \(W + S\),
    manteniendo siempre la legitimidad.

    Sea \(\tau\) la torre original y \(\tau^*\) la obtenida al intercambiar.
    O sea:
    \begin{align*}
      \tau
        &= \langle t_1, \dotsc, t_{i - 1},
                   t_i, t_{i + 1},
                   \dotsc, t_m \rangle \\
      \tau^*
        &= \langle t_1, \dotsc, t_{i - 1},
                   t_{i + 1}, t_i,
                   \dotsc, t_m \rangle \\
    \end{align*}
    La única tortuga con más carga en su espalda en \(\tau^*\) es \(t_i\),
    debemos demostrar que no sobrepasa su fuerza,
    o sea que:
    \begin{equation*}
      \sum_{1 \le r \le i - 1} W(t_r) + W(t_{i + 1})
        < S(t_i)
    \end{equation*}
    Como la torre original era legítima:
    \begin{equation*}
      \sum_{1 \le r \le i - 1} W(t_r) + W(t_i)
        \le S(t_{i + 1})
    \end{equation*}
    Sumando \(W(t_{i + 1})\) a esta desigualdad tenemos:
    \begin{equation*}
      \sum_{1 \le r \le i - 1} W(t_r) + W(t_i) + W(t_{i + 1})
        \le W(t_{i + 1}) + S(t_{i + 1})
    \end{equation*}
    Pero supusimos \(W(t_{i + 1}) + W(t_{i + 1}) < W(t_i) + S(t_i)\),
    con lo que:
    \begin{align*}
      \sum_{1 \le r \le i - 1} W(t_r) + W(t_i) + W(t_{i + 1})
        &< W(t_i) + S(t_i) \\
      \sum_{1 \le r \le i - 1} W(t_r) + W(t_{i + 1})
        &< S(t_i)
    \end{align*}
    Esto era lo que había que demostrar.
  \end{proof}
  La proposición~\ref{prop:W+S} permite restringirnos a torres
  no-decrecientes en \(W + S\) y aún así obtener una solución óptima.
  Supondremos entonces que las tortugas están numeradas en este orden.

  Pero hay un problema:
  consideremos la torre legítima más alta que termina en \(T_i\):
  \begin{equation*}
    \langle t_1, \dotsc, t_m, T_i \rangle
  \end{equation*}
  donde
    \(\{ t_1, \dotsc, t_{m - 1}, t_m \} \subseteq \{ T_1, \dotsc, T_{i - 1} \}\).
  Desafortunadamente,
  la torre \(\langle t_1, \dotsc, t_{m - 1}, t_m \rangle\)
  podría no ser la torre más alta con \(t_m\) en la base;
  puede haber una torre legítima con al menos \(m + 1\) tortugas
  \(\langle t_1^*, \dotsc, t_m^*, t_m \rangle\)
  pero demasiado pesada para la tortuga \(T_i\).
  Esta formulación no cumple con subestructura óptima,
  debemos generalizar nuestro problema con mayor cuidado.

  Construiremos una secuencia de las torres más livianas de cada altura.
  O sea,
  resolvemos los siguientes subproblemas para \(j \le n\):
  \(P'(j)\) para cada \(r < j\) para el que hay una torre de tortugas
  de altura \(r\) de tortugas del conjunto \(\{ T_1, \dotsc, T_j \}\)
  (no necesariamente incluyendo a \(T_j\))
  encuentre la más liviana.
  Con esto la recursión funciona:
  resuelto el problema \(P'(i - 1)\),
  buscamos la torre más liviana \(\theta_k^i\) de largo \(k\)
  incluyendo solo tortugas \(\{ T_1, \dotsc, T_i\} \).
  Para ello consideramos las torres más livianas \(\theta_k^{i - 1}\)
  y \(\theta_{k - 1}^{i - 1}\),
  y vemos si podemos extender legítimamente la última con \(T_i\).
  Esto da el óptimo,
  si sobre \(T_i\) podemos poner una torre de largo \(m\),
  ciertamente podemos poner la torre más liviana de largo \(m\) sobre ella.
  Si la torre más alta construida con \(\{ T_1, \dotsc, T_{i - 1} \}\)
  tiene altura \(m\) y \(T_i\) puede extenderla,
  obtenemos la primera torre de altura \(m + 1\)
  compuesta con \(\{ T_1, \dotsc, T_i \}\).
  Note que nuestro problema se hizo bidimensional en el proceso.

\section{Variación mínima}
\label{sec:variacion-minima}

  Definimos la \emph{variación total} de una secuencia
  \(s = \langle x_1, \dotsc, x_n \rangle\)
  como:
  \begin{equation*}
    V(s)
      = \sum_{1 \le i \le n - 1} \lvert x_{i + 1} - x_i \lvert
  \end{equation*}
  Dan una secuencia de números \(a_1, \dotsc, a_n\).
  Divídala en dos subsecuencias
  (manteniendo el orden original)
  de manera que la suma de las variaciones totales de las subsecuencias
  sea la menor posible,
  o sea,
  halle:
  \begin{align*}
    s_1
      &= \langle a_{i_1}, \dotsc, a_{i_k} \rangle
        & i_1 < i_2 < \dotsm < i_k \\
    s_2
      &= \langle a_{j_1}, \dotsc, a_{j_k} \rangle
        & j_1 < j_2 < \dotsm < j_{n -k}
  \end{align*}
  y \(\{ i_1, i_2, \dotsc, i_k \} \cup \{ j_1, j_2, \dotsc, j_{n - k} \}
          = \{1, 2, \dotsc, n \}\)
  tal que \(V(s_1) + V(s_2)\) es mínimo.

  Esta también tiene su truco.
  Uno se ve tentado a resolver subproblemas \(P(j)\) para todo \(m \le n\),
  donde \(P(j)\) es dividir \(\langle a_1, \dotsc, a_m \rangle\)
  en subsecuencias con mínima variación.
  Extendemos las subsecuencias
  \(\langle x_1, \dotsc, x_r \rangle\) donde \(r \le m\)
  y \(\langle y_1, \dotsc, y_s \rangle\) donde \(s \le m\)
  considerando el menor
  de \(\lvert x_r - a_{m + 1} \rvert\) y \(\lvert y_s - a_{m + 1} \rvert\)
  para agregar \(a_{m + 1}\) a una o la otra.
  Desafortunadamente,
  puede haber una división no óptima de \(\langle a_1, \dotsc, a_m \rangle\)
  en \(\langle u_1, \dotsc, u_{r'} \rangle\) y \(\langle v_1, \dotsc, v_{s'} \rangle\)
  tal que:
  \begin{equation*}
    \sum_i \lvert u_{i + 1} - u_i \rvert
        + \sum_j \lvert v_{j + 1} - v_j \rvert
      > \sum_i \lvert x_{i + 1} - x_i \rvert
          + \sum_j \lvert y_{j + 1} - y_j \rvert
  \end{equation*}
  pero tal que \(\lvert v_{s'} - a_{m + 1} \rvert\)
  es mucho menor que \(\lvert x_r - a_{m + 1} \rvert\)
  y \(\lvert y_s - a_{m + 1} \rvert\),
  de manera que:
  \begin{multline*}
    \sum_i \lvert u_{i + 1} - u_i  \rvert
        + \sum_j \lvert v_{j + 1} - v_j	 \rvert
        + \lvert v_s - a_{m + 1} \rvert \\
      < \sum_i \lvert x_{i + 1} - x_i \rvert
          + \sum_j \lvert y_{j + 1} - y_j \rvert
          + \min \{ \lvert x_r - a_{m + 1} \rvert,
                    \lvert y_s - a_{m + 1} \rvert \}
  \end{multline*}

  Para resolver esto,
  planteamos el siguiente problema bidimensional:
  \(P(r, s)\) es dividir la secuencia en secuencias
  que terminan en \(a_r\) y \(a_s\)
  de forma que la suma de sus variaciones totales se minimice.
  Para la solución del subproblema \(P(r, s)\)
  consideramos varios casos:
  \begin{enumerate}
  \item
    Si \(r < s - 1\),
    extendemos la solución óptima para \(P(r, s - 1)\)
    agregando \(a_s\) a la secuencia que termina en \(a_{s - 1}\),
    ya que la otra termina en \(a_r\).
  \item
    Si \(r = s - 1\),
    consideramos soluciones para todos los subproblemas \(P(t, s - 1)\)
    con \(t < s - 1\),
    extendiendo la subsecuencia que termina en \(a_t\)
    y eligiendo aquella con la mínima variación total.
    Esto lo comparamos
    con las subsecuencias \(\lvert a_1, \dotsc, a_{s - 1} \rvert\)
    y \(\lvert a_s \rvert\),
    reteniendo la menor.
  \end{enumerate}

\section{Ahorrar espacio}
\label{sec:ahorrar-espacio}

  Consideramos el problema de máxima subsecuencia común
  (sección~\ref{sec:LCS}).
  Vimos que el tiempo requerido por programación dinámica
  es \(O(m n)\) al comparar secuencias de largos \(m\) y \(n\),
  y que el espacio es también \(O(m n)\).
  Es rutinario querer comparar secuencias de muchos miles de líneas,
  el espacio requerido se puede hacer prohibitivo.
  Si se analiza el algoritmo esbozado,
  solo se requieren algunas entradas del arreglo,
  no se necesita el arreglo completo.
  Basándose en esta observación,
  Hirschberg~%
    \cite{hirschberg75:_linear_space_algor_comput_maxim_common_subseq}
  construye un algoritmo que requiere espacio lineal.
  Suponemos palabras \(X\) e \(Y\),
  de largos \(m\) y \(n\),
  respectivamente.
  Partimos con el algoritmo de programación dinámica directo,
  algoritmo~\ref{alg:LCS-A}.
  \begin{algorithm}[htbp]
    \DontPrintSemicolon\Indp

    \Procedure{\(A(m, n, X, Y, L)\)}{
      \(L_{i, 0} \gets 0 \quad [ i = 0, \dotsc, m ]\) \;
      \(L_{0, j} \gets 0 \quad [ j = 0, \dotsc, n ]\) \;
      \For{\(i \gets 1\) \KwTo \(m\)}{
        \For{\(j \gets 1\) \KwTo \(n\)}{
          \eIf{\(X_i = Y_j\)}{
            \(L_{i, j} \gets L_{i - 1, j - 1} + 1\) \;
          }
          {
            \(L_{i, j} \gets
               \max \{ L_{i, j - 1}, L_{i - 1, j} \}\) \;
          }
        }
      }
    }

    \caption{Máxima común subsecuencia por programación dinámica directa}
    \label{alg:LCS-A}
  \end{algorithm}
  Observamos que el algoritmo~\ref{alg:LCS-A}
  para calcular la fila \(i\) del arreglo \(\mathbf{L}\)
  solo hace referencia a la fila \(i - 1\).
  Una pequeña modificación da el algoritmo~\ref{alg:LCS-B},
  que calcula el vector \(\mathbf{\tilde{L}}\),
  donde \(\tilde{L}_j = L_{m, j}\).
  Básicamente mantenemos en la matriz \(\mathbf{K}\)
  los valores requeridos.
  \begin{algorithm}[htbp]
    \DontPrintSemicolon\Indp

    \Procedure{\(B(m, n, X, Y, \tilde{L})\)}{
      \(K_{1, j} \gets 0 \quad [ j = 0, \dotsc, n ]\) \;
      \For{\(i \gets 1\) \KwTo \(n\)}{
        \(K_{0, j} \gets K_{1, j} \quad [ j = 0, \dotsc, n ]\) \;
        \For{\(j \gets 1\) \KwTo \(n\)}{
          \eIf{\(X_i = Y_j\)}{
            \(K_{1, j} \gets K_{0, j - 1} + 1\) \;
          }
          {
            \(K_{1, j} \gets
               \max \{ K_{1, j - 1}, K_{0, j} \}\) \;
          }
        }
        \(\tilde{L}_j \gets K_{1, j} \quad [ i = 0, \dotsc, n ]\) \;
      }
    }

    \caption{Largo de máxima común subsecuencia por programación dinámica
             ahorrando espacio}
    \label{alg:LCS-B}
  \end{algorithm}
  Lo malo es que el algoritmo~\ref{alg:LCS-B} solo entrega el largo,
  no tenemos cómo recuperar la subsecuencia máxima.
  Veremos cómo usar el algoritmo~\ref{alg:LCS-B} sobre subpalabras
  para recuperar la máxima común subsecuencia en espacio lineal.

  Llamemos \(X_{r s}\) a la subsecuencia \(x_r, x_{r + 1}, \dotsc, x_s\).
  Para explicitar que estamos marchando en reversa,
  anotamos \(\widehat{X}_{r s}\) con \(r > s\)
  para \(x_s, x_{s + 1}, \dotsc, x_r\).
  Sea \(L^*_{i, j}\) el largo de la subsecuencia máxima
  entre \(X_{i + 1, m}\) e \(Y_{j + 1, n}\).
  Notamos que \(L_{i j}\) para \(j = 0, \dotsc, n\)
  son los largos máximos de subsecuencias comunes de \(X_{1, i}\)
  y prefijos de \(Y\).
  Podemos interpretarlos igualmente en términos de las palabras reversas
  y sufijos en \(\widehat{X}\) e \(\widehat{Y}\).
  Definamos:
  \begin{equation}
    \label{eq:LCS-M}
    M_i
      = \max_{0 \le j \le n} \{ L_{i, j} + L^*_{i, j} \}
  \end{equation}
  Haremos uso del siguiente teorema:
  \begin{theorem}
    \label{theo:LCS}
    Para \(1 \le i \le m\) es \(M_i = L_{m, n}\).
  \end{theorem}
  \begin{proof}
    Sea \(j\) tal que \(M_i = L_{i, j} + L^*_{i, j}\).
    Sea también \(S_{i j}\) una subsecuencia máxima común
    de \(X_{1 i}\) e \(Y_{1 j}\);
    y sea \(S^*_{i j}\) una subsecuencia máxima común
    de \(X_{i + 1, m}\) e \(Y_{j + 1, n}\).
    Entonces \(Z = S_{i j} \cdot S^*_{i j}\)
    es una subsecuencia común de \(X\) e \(Y\),
    y su largo es \(M_i\).
    O sea,
    \(L_{m n} \ge M_i\).

    Por otro lado,
    sea \(S_{m n}\) cualquier subsecuencia común más larga
    entre \(X\) e \(Y\).
    Podemos escribir \(S_{m n} = S_1 \cdot S_2\),
    donde \(S_1\) es subsecuencia de \(X_{1 i}\) para algún \(i\)
    y \(S_2\) es subsecuencia de \(X_{i + 1, m}\).
    Hay un \(j\) tal que \(S_1\) es subsecuencia de \(Y_{1 j}\)
    y \(S_2\) es subsecuencia de \(Y_{j + 1, n}\).
    Por las definiciones de \(L\) y \(L^*\),
    \(\lvert S_1 \rvert \le L_{i j}\)
    y \(\lvert S_2 \rvert \le L^*_{i j}\).
    O sea:
    \begin{align*}
      L_{m n}
        &=   \lvert S_{m n} \rvert \\
        &=   \lvert S_1 \rvert + \lvert S_2 \rvert \\
        &\le L_{i j} + L^*_{i j} \\
        &\le M_i
    \end{align*}
    Concluimos \(L_{m n} = M_i\).
  \end{proof}
  Usamos el teorema~\ref{theo:LCS} recursivamente
  para dividir el problema original en subproblemas similares
  hasta obtener problemas triviales
  (ver también el capítulo~\ref{cha:dividir-conquistar}).
  Nuestro algoritmo final~\ref{alg:LCS-C}
  construye la palabra \(Z\) que es la subsecuencia común más larga
  de \(X\) e \(Y\).
  \begin{algorithm}
    \DontPrintSemicolon\Indp

    \Procedure{\(C(m, n, X, Y, Z)\)}{
      \uIf{\(n = 0\)}{
        \(Z \gets \varepsilon\) \;
      }
      \uElseIf{\(m = 1\)}{
        \eIf{\(\exists j \le n, X_1 = Y_j\)}{
          \(Z \gets A_1\) \;
        }
        {
          \(Z \gets \varepsilon\) \;
        }
      }
      \Else{
        \(i \gets \lfloor m / 2 \rfloor\) \;
        Compute \(L_{i, j}\) and \(L^*_{i, j}\)
          for \(0 \le j \le n\) \;
        \(B(i, n, X_{1 i}, Y_{1, n}, L')\) \;
        \(B(m - 1, n, \widehat{X}_{n, i + 1}, \widehat{Y}_{n, 1}, L'')\) \;
        Find \(j\) such that \(L_{i j} + L^*_{i j} = L_{m n}\)
          using theorem~\ref{theo:LCS}: \;
        {
          \(M \gets \max_{0 \le j \le n} \{ L'_j + L''_{n - j} \}\) \;
          \(k \gets j \text{\ tal que\ } M = L'_j + L''_{n - j}\) \;
        }
        \(C(i, k, X_{1 i}, Y_{1 k}, Z')\) \;
        \(C(m - 1, n - k, X_{i + 1, m}, Y_{k + 1, n}, Z'')\) \;

        \Return \(Z' \cdot Z''\) \;
      }
    }

    \caption{Máxima común subsecuencia por programación dinámica
             ahorrando espacio}
    \label{alg:LCS-C}
  \end{algorithm}
  El algoritmo \(B\) toma tiempo \(O(m n)\)
  y usa espacio \(O(m)\)
  (suponemos que \(X\) e \(Y\) y sus subpalabras
   se manejan con índices a principio y fin de espacio común).

  El algoritmo \(C\) se ejecuta a lo más \(2 m - 1\) veces,
  por inducción:
  Sea \(m \le 2^r\).
  Si \(r = 0\),
  es \(m = 1\) y hay \(2^0 = 1\) llamada a \(C\).
  Suponga ahora que para \(m \le 2^r = M\) hay \(2 m - 1\) llamadas a \(C\).
  Para \(m' \le 2^{r + 1} = 2 M\),
  \(i\) a lo más toma el valor \(M\),
  hay dos llamadas a \(C\) con \(m_1\) y \(m_2\) tales que \(m_1 + m_2 = m'\)
  y con \(m_1\) y \(m_2\) ambos menores a \(M\).
  Cada cual ejecutará \(2 m_1 - 1\) y \(2 m_2 - 1\) llamadas a \(C\)
  por inducción,
  agregando la llamada original a \(C\)
  da un total de \(2 m_1 - 1 + 2 m_2 + 1 = 2 m' - 1\),
  como se quería demostrar.

\section*{Ejercicios}
\label{sec:ejercicios-mas-programacion-dinamica}

  \begin{enumerate}
  \item
    ¿Cómo detectar en el algoritmo de Floyd-Warshall
    si hay un ciclo de costo negativo?
  \item
    Como planteado,
    el algoritmo de Floyd-Warshall solo calcula el costo mínimo
    del camino entre vértices.
    ¿Qué información adicional se requiere registrar
    para poder reconstruir los caminos del caso?
  \item
    Complete la discusión sobre la torre de tortugas,
    desarrollando un programa que resuelva el problema.
    ¿Cuál es su complejidad?
  \item
    Use un razonamiento similar a la torre de tortugas
    para hallar la subsecuencia creciente más larga
    de una secuencia de \(n\) números
    en tiempo \(O(n \log n)\).
  \item
    Reduzca el problema de variación mínima a una única dimensión
    considerando los subproblemas \(P(s - 1, s)\) únicamente.
  \item
    Escriba un programa que resuelva el problema de variación mínima.
    ¿Cuál es su complejidad?
  \item
    Halle,
    módulo \(10^{16}\),
    el número de subconjuntos no vacíos
    de \(\{ 1^1, 2^2, 3^3, \dotsc, 250250^{250250} \}\)
    cuya suma es divisible por \num{250}.
  \item
    Una \emph{partición} de \(n \in \mathbb{N}\)
    es un conjunto \(\{ p_1, \dotsc, p_k \}\)
    (las \emph{partes},
     \(p_i \in \mathbb{N}\))
    tal que \(p_1 + \dotsb + p_k = n\).
    Dé un algoritmo para obtener el número de particiones de \(n\),
    y dé su complejidad.
  \item
    El \emph{problema del vendedor viajero}
    es un famoso problema \NP\nobreakdash-completo.
    Plantea un grafo \(G = (V, E)\)
    con costos de arcos \(w(e)\) para \(e \in E\).
    Muestre cómo resolverlo,
    eligiendo \(u \in V\) arbitrario para comenzar la gira,
    sean \(u \ne v \in S \subseteq V\),
    y sea \(d[v][S]\) el costo mínimo
    de un viaje por todos los vértices de \(S\),
    comenzando en \(u\) y terminando en \(v\).
    Plantee una recurrencia para \(d\) considerando el último arco del viaje.
    ¿Cuál es la complejidad de su algoritmo?
  \end{enumerate}

% To do:
% - Space saving methods
% - More stuff from Jeffe

\bibliography{../referencias}

%%% Local Variables:
%%% mode: latex
%%% TeX-master: "../INF-221_notas"
%%% ispell-local-dictionary: "spanish"
%%% End:

% LocalWords:  ruteo subsecuencias subsecuencia subpalabras and for
% LocalWords:  Find such that using theorem

\bibliographystyle{babplain-fl}

\chapter{Máxima subsecuencia común, otra mirada}
\label{cha:LCS-practical}

  Discutimos el problema de máxima subsecuencia común
  (LCS,
   por el inglés \emph{\foreignlanguage{english}{Longest Common Subsequence}})
  dando algoritmos basados en programación dinámica
  (sección~\ref{sec:LCS},
   el algoritmo de Wagner-Fischer~%
     \cite{wagner74:_string_string_correc_probl},
   aunque Navarro~%
     \cite{navarro01:_guided_tour_approx_strin_match}
   halla múltiples autores independientes de la misma idea)
  y las variantes de Hirschberg~%
    \cite{hirschberg75:_linear_space_algor_comput_maxim_common_subseq}
  para ahorrar espacio
  (discutido en la sección~\ref{sec:ahorrar-espacio}).
  Otros algoritmos para este problema incluyen el de Hunt-Szymanski~%
    \cite{hunt77:_fast_algor_comput_lcs}.
  Una visión distinta es la de Heckel~%
    \cite{heckel78:_techn_isolat_differ_between_files},
  quien intenta reconstruir diferencias fijándose en líneas únicas
  en ambos archivos,
  con el objetivo de obtener diferencias intuitivamente relevantes.
  Una discusión bastante completa de diferentes algoritmos
  es la de Hirschberg~%
    \cite{hirschberg97:_serial_comp_distances}.
  Aho, Hirschberg y Ullman~%
    \cite{aho76:_bound_compl_longes_common_subseq_probl}
  derivan cotas inferiores para el problema general,
  concluyen que en caso de solo comparar símbolos por igualdad
  y alfabeto ilimitado
  (la situación de comparar archivos,
   donde una línea representada por un \emph{\foreignlanguage{english}{hash}}
   es un símbolo)
  en el caso general comparar dos secuencias de largo \(N\)
  demanda tiempo \(\Omega(N^2)\).

  El problema halla aplicación práctica en muchas áreas,
  prominente en las cuales es comparar archivos
  (como describen Hunt y McIllroy~%
    \cite{hunt76:_algor_differ_file_compar}
   y Miller y Myers~%
    \cite{miller85:_file_compar_progr}).
  Los sistemas de control de versiones
  deben mostrar diferencias entre archivos de diferentes versiones,
  y muchos usan internamente las diferencias entre versiones
  para ahorrar espacio al almacenar la historia.
  Ejemplos tempranos son SCCS~%
    \cite{rochkind75:_sccs}
  y RCS~%
    \cite{tichy85:_rcs}.
  Claro que en esta aplicación
  (y al comprimir datos)
  se suele usar además la idea de hacer referencia a copias anteriores
  de los mismos datos.
  El algoritmo básico es de Bentley y McIllroy~%
    \cite{bentley99:_data_compr_using_long_common_strin},
  se estandarizó en el formato VCDIFF~%
    \cite{rfc3284}
  y hay herramientas código abierto,
  como \lstinline[language =]{xdelta}~%
    \cite{macdonald16:_xdelta}.

  Para aplicaciones prácticas
  (archivos de muchos miles de líneas,
   secuencias de genes de muchos millones de pares de bases)
  los algoritmos presentados no son adecuados.
  Hunt y McIllroy~%
    \cite{hunt76:_algor_differ_file_compar}
  comparan algunas alternativas tempranas,
  y plantean técnicas para mejorar el rendimiento,
  como usar un \emph{\foreignlanguage{english}{hash}}
  (ver el capítulo~\ref{cha:hashing} para la teoría del caso)
  en vez de la línea completa para acelerar las comparaciones y ahorrar espacio
  (esto es usar huellas digitales,
   tema que discutiremos en el capítulo~\ref{cha:randomized-algorithms}).
  Concluyen que los algoritmos disponibles en ese entonces
  no hacen diferencia para los casos a su alcance
  (archivos de \num{3\,500}~líneas),
  el tiempo de ejecución está dominado por la lectura de los datos
  y manipulación de caracteres individuales.
  Advertimos,
  eso sí,
  que los distintos algoritmos tienen comportamientos diferentes,
  no siempre este es el más adecuado.
  Barabucci et al~%
    \cite{barabucci16:_measur_qualit_diff_algor}
  discuten varios escenarios e intentan formalizar \textquote{calidad}
  aplicada específicamente a diferencias entre documentos en XML.

  Hoy el algoritmo empleado con mayor frecuencia
  es el de Myers~%
    \cite{myers86:_O_n_d_differ_algor_its_variat},
  cuya variante debida a Wu, Manber, Myers y Miller~%
    \cite{wu90:_sequence_comparison_algorithm}
  discutiremos a continuación.
  Bergroth, Hakonen y Raita~%
    \cite{bergroth00:_survey_longes_common_subseq_algor}
  comparan varios algoritmos,
  para situaciones en las cuales las secuencias son similares
  este algoritmo parece ser el mejor.

  Suponemos dos textos,
  \(X\) e \(Y\),
  de largos \(M = \lvert X \rvert\)
  y \(N = \lvert Y \rvert\),
  donde sin pérdida de generalidad asumimos \(N \ge M\),
  con subsecuencia común máxima de largo \(L\).
  La diferencia entre los largos es \(\Delta = N - M\).
  Sea \(D = M + N - 2 L\),
  el largo de la secuencia de edición
  (operaciones agregar/eliminar)
  más corta
  (de \(X\) debemos eliminar \(M - L\)~símbolos,
   luego hay que insertar \(N - L\) para crear \(Y\)).
  A este valor se le llama \emph{distancia de Levenshtein}~%
    \cite{levenshtein66:_binar_codes_capab_correc_delet_inser_rever}.
  Una medida relacionada,
  que llamaremos \(P\),
  es el número de símbolos eliminados de \(X\):
  \begin{align*}
    P
      &= M - L \\
      &= M - \frac{M + N - D}{2} \\
      &= \frac{D - \Delta}{2}
  \end{align*}
  El algoritmo planteado tiene tiempo de ejecución \(O(N P)\),
  lo que lo hace aplicable
  en situaciones comunes en que las palabras a comparar
  son muy similares.
  Se basa en una formulación intuitiva de grafo de edición,
  es un algoritmo de programación dinámica
  que aprovecha un criterio voraz para limitar las opciones a considerar.

\section{Grafo de edición}
\label{sec:grafo-edicion}

  Sean palabras \(X = x_1 x_2 \ldots x_M\)
  e \(Y = y_1 y_2 \ldots y_N\),
  de largos respectivos \(M\) y \(N\) con \(N \ge M\).
  El \emph{grafo de edición}
  para \(X\) e \(Y\) tiene nodos en los puntos de la cuadrícula
  \((i, j)\) para \(0 \le i \le M\) y \(0 \le j \le N\).
  Los nodos se conectan mediante arcos dirigidos
  verticales, horizontales y diagonales,
  formando un grafo dirigido acíclico.
  Arcos horizontales conectan \((i - 1, j)\) con \((i, j)\),
  arcos verticales conectan \((i, j - 1)\) con \((i, j)\),
  hay un arco diagonal \((i - 1, j - 1)\) a \((i, j)\)
  siempre que \(x_i = y_j\).
  Nuestro problema es llegar desde la \emph{fuente} \((0, 0)\)
  al \emph{sumidero} \((M, N)\) en este grafo.
  Un camino más corto usará el máximo posible de pasos diagonales,
  nos da la subsecuencia común más larga.
  Nuestra medida de distancia es el número de eliminaciones en \(X\),
  que corresponden a pasos verticales.
  La figura~\ref{fig:grafo-edicion} ilustra este grafo
  para \(X = a c b d e a c b e d\)
  e \(Y = a c e b d a b b a b e d\),
  ejemplo de~%
    \cite{wu90:_sequence_comparison_algorithm}.
  \begin{figure}[ht]
    \centering
    \tikzstyle{vertex} = [circle, inner sep = 0pt, minimum size = 0.7mm, fill]
    \begin{tikzpicture}[scale = 0.7]
      \foreach \i in {0, 1, ..., 12}
      {
        \foreach \j in {0, 1, ..., 10}
        {
          \node[vertex] at (\i, \j) {};
        }
      }
      \foreach \i in {0, 1, ..., 12}
      {
        \draw[dotted] (\i, 0) -- (\i, 10);
      }
      \foreach \j in {0, 1, ..., 10}
      {
        \draw[dotted] (0, \j) -- (12, \j);
      }

      % Y = acebdabbabed
      \node at ( 1, 10) [above] {\(a\)};
      \node at ( 2, 10) [above] {\(c\)};
      \node at ( 3, 10) [above] {\(e\)};
      \node at ( 4, 10) [above] {\(b\)};
      \node at ( 5, 10) [above] {\(d\)};
      \node at ( 6, 10) [above] {\(a\)};
      \node at ( 7, 10) [above] {\(b\)};
      \node at ( 8, 10) [above] {\(b\)};
      \node at ( 9, 10) [above] {\(a\)};
      \node at (10, 10) [above] {\(b\)};
      \node at (11, 10) [above] {\(e\)};
      \node at (12, 10) [above] {\(d\)};
      % X = acbdeacbed
      \node at (0, 9) [left] {\(a\)};
      \node at (0, 8) [left] {\(c\)};
      \node at (0, 7) [left] {\(b\)};
      \node at (0, 6) [left] {\(d\)};
      \node at (0, 5) [left] {\(e\)};
      \node at (0, 4) [left] {\(a\)};
      \node at (0, 3) [left] {\(c\)};
      \node at (0, 2) [left] {\(b\)};
      \node at (0, 1) [left] {\(e\)};
      \node at (0, 0) [left] {\(d\)};
      % Diagonals
      \draw[thin] ( 0,	5) -- ( 2, 3);
      \draw[thin] ( 2,	6) -- ( 3, 5);
      \draw[thin] ( 2,	2) -- ( 3, 1);
      \draw[thin] ( 3,	3) -- ( 4, 2);
      \draw[thin] ( 4,	1) -- ( 5, 0);
      \draw[thin] ( 5, 10) -- ( 6, 9);
      \draw[thin] ( 6,	8) -- ( 7, 7);
      \draw[thin] ( 7,	8) -- ( 8, 7);
      \draw[thin] ( 7,	3) -- ( 8, 2);
      \draw[thin] ( 8, 10) -- ( 9, 9);
      \draw[thin] ( 8,	5) -- ( 9, 4);
      \draw[thin] ( 9,	8) -- (10, 7);
      \draw[thin] ( 9,	3) -- (10, 2);
      \draw[thin] (10,	6) -- (11, 5);
      \draw[thin] (11,	7) -- (12, 6);
      % A shortest path
      \draw[thick, gray] (0, 10) -- (2, 8) -- (3, 8) -- (5, 6) -- (5, 5)
                           -- (6, 4) -- (6, 3) -- (7, 2) -- (10, 2) -- (12, 0);
    \end{tikzpicture}
    \caption{Un grafo de edición y un camino óptimo}
    \label{fig:grafo-edicion}
  \end{figure}
  En la figura~\ref{fig:grafo-edicion}
  se remarca un camino de \((0, 0)\) a \((12, 10)\),
  correspondiente a
  \(a c b d a b e d
      = x_1 x_2 x_4 x_5 x_6 x_7 x_{11} x_{12}
      = y_1 y_2 y_3 y_4 y_6 y_8 y_9 y_{10}\).
  Una movida vertical corresponde a eliminar un símbolo de \(X\),
  una horizontal inserta un símbolo de \(Y\),
  un paso diagonal es retener el símbolo de ambos
  (parte del camino común más largo).
  En este ejemplo hay \(P = 2\) eliminaciones de \(X\)
  y un total de \(D = 6\) ediciones,
  la secuencia común más larga es de \(L = 8\).
  Hay caminos más cortos alternativos,
  como se ven en la figura~\ref{fig:grafo-edicion}.

\section{Preliminares}
\label{sec:LCS-WMMM:preliminares}

  Considere el grafo de edición dibujado como grilla.
  Sea la diagonal \(k\)
  los puntos \((i, j)\) tales que \(i - j = k\).
  La fuente está sobre la diagonal \num{0},
  el sumidero sobre la diagonal \(\Delta = N - M\).
  Tenemos diagonales numeradas \(-M\) a \(N\).
  El algoritmo presente considera los nodos en la franja de diagonales
  entre \(-P\) y \(\Delta + P\).
  Esto porque todo camino que salga de esta franja
  incluye más de \(P\) pasos verticales:
  si pasa por un vértice bajo \(-P\),
  para llegar a él desde la fuente da más de \(P\) pasos verticales;
  si pasa por un vértice sobre \(\Delta + P\)
  requiere más de \(P\) pasos para llegar al sumidero.

  La \emph{distancia de edición} a \((i, j)\),
  anotada \(D(i, j)\),
  es el costo
  (número total de inserciones y eliminaciones)
  de un camino óptimo de \((0, 0)\) a \((i, j)\)
  sobre la diagonal \(k = i - j\).
  Si tal camino contiene \(v\) pasos verticales y \(h\) horizontales,
  el número de pasos no diagonales es \(v + h = D(i, j)\),
  y debe terminar en la diagonal \(h - v = k\).
  El número de pasos verticales,
  \(V(i, j)\),
  en este camino es una cantidad bien definida:
  es \(V(i, j) = (D(i, j) + k) / 2\).
  Sea la \emph{distancia comprimida} como definida a continuación:
  \begin{equation}
    \label{eq:LCS-WMMM-P}
    P(i, j)
      = \begin{cases}
          V(i, j)		 & \text{si \((i, j)\)
                                         está bajo la diagonal \(\Delta\)} \\
          V(i, j) + (k - \Delta) & \text{caso contrario}
        \end{cases}
  \end{equation}
  La distancia comprimida
  es la distancia vertical \(V(i, j)\) más una cota inferior
  al número de pasos verticales para llegar de \((i, j)\) al sumidero.
  Bajo la diagonal no se requieren pasos verticales,
  sobre la diagonal son al menos \(k - \Delta\).

  El algoritmo se centra en calcular los vértices más lejanos del origen
  en orden de distancia hasta hallar el sumidero.
  Sea el \emph{\(d\)\nobreakdash-punto más lejano sobre la diagonal \(k\)}
  el vértice sobre esa diagonal con valor de \(D(i, j) = d\) y máximo \(i\)
  (o \(j\)).
  Llamaremos \(\mathrm{fd}(k, d)\) a la coordenada \(j\):
  \begin{equation}
    \label{eq:LCS-WMMM-fd}
    \mathrm{fd}(k, d)
      = \max \{ j \colon D(j - k, j) = d \}.
  \end{equation}
  El conjunto \(\mathrm{FD}(d)\) es la frontera de los vértices
  con distancia \(d\).
  Usamos distancias comprimidas,
  con frontera \(\mathrm{FP}(p)\)
  con definición análoga,
  para \(\mathrm{fp}(k, p) = \max \{ j \colon P(j - k, j) = p \}\):
  \begin{equation}
    \label{eq:LCS-WMMM-FD}
    \mathrm{FP}(p)
      = \{ (j - k, j)
             \colon y = \mathrm{fp}(k, p) \wedge -p \le k \le p + \Delta
        \}
  \end{equation}

\section[El algoritmo \(O(N P)\)]
        {El algoritmo \boldmath\(O(N P)\)\unboldmath}
\label{sec:LCS-WMMM}

  Calculamos el conjunto \(\mathrm{FP}(p)\)
  desde \(\mathrm{FP}(p - 1)\)
  hasta que \((M, N) \in \mathrm{FP}(p)\),
  con lo que conocemos \(P\) y también \(D = \Delta + 2 P\).
  Daremos primero una descripción informal,
  para luego formalizarlo en una recurrencia
  que demostramos correcta.
  Sea \(q_k\) el \((p - 1)\)\nobreakdash-punto más lejano en la diagonal \(k\)
  (vale decir,
   el punto \((j - k, j)\) tal que \(j = \mathrm{fp}(k, p - 1)\))
  y sea \(g_k\) el \(p\)\nobreakdash-punto más lejano en la diagonal \(k\).
  Suponga que ya conocemos
    \(\mathrm{FP}(p - 1)
        = \{ q_{-(p - 1)}, q_{-(p - 2)}, \dotsc, q_{\Delta + (p - 1)} \}\).
  Primero calculamos \(g_{-p}, g_{-(p - 1)}, \dotsc, g_{\Delta - 1}\)
  en este orden.
  Para calcular \(g_k\) de \(g_{k - 1}\) y \(q_{k + 1}\) procedemos como sigue.
  Sea \(a\) el vértice inmediatamente a la derecha de \(g_{k - 1}\)
  y \(b\) el vértice inmediatamente debajo de \(q_{k + 1}\)
  (ver la figura~\ref{fig:LCS-WMMM-extend-FP}).
  \begin{figure}[ht]
    \centering
    \tikzstyle{vertex} = [circle, inner sep = 0pt, minimum size = 0.7mm, fill]
    \begin{tikzpicture}
      % Diagonals
      \foreach \i in {0, 1.5, 3, 6, 7.5, 9}
      {
        \draw[dotted, thin] (\i,   5) -- ( 5 + \i,   0);
      }
      \foreach \i in {1.75, 3.25, 7.75, 9.25}
      {
        \draw (\i, 4.75) -- (\i + 0.75, 4) node[vertex] {};
      }
      \foreach \i in {2, 8}
      {
        \draw (\i, 3) -- (\i + 1.5, 1.5) node[vertex] {};
        \draw[dashed, -latex'] (\i + 1.5, 1.5) -- (\i + 3, 1.5);
      }
      \foreach \i in {4, 10}
      {
        \draw[dashed, -latex'] (\i, 4) -- (\i, 2.5) node[vertex] {};
      }
      \foreach \i in {5, 11}
      {
        \draw (\i, 1.5) node[vertex] {} -- (\i + 1, 0.5) node[vertex] {};
      }
      \foreach \i in {5, 11}
      {
        \node at (0.1 + \i, -0.2) [below right, centered] {\small\(k - 1\)};
        \node at (1.6 + \i, -0.2) [below right, centered] {\small\(k\)};
        \node at (3.1 + \i, -0.2) [below right, centered] {\small\(k + 1\)};
      }
      \node at (2.5, 4	) [below left]	{\small\(q_k^{p - 1}\)};
      \node at (3.5, 1.5) [below left]	{\small\(q_{k - 1}^p\)};
      \node at (4,   2.5) [below left]	{\small\(b\)};
      \node at (4,   4	) [above right] {\small\(q_{k + 1}^{p - 1}\)};
      \node at (5,   1.5) [below left]	{\small\(a\)};
      \node at (6,   0.5) [above right] {\small\(q_k^p\)};

      \node at ( 8.5, 4	 ) [below left]	 {\small\(q_k^{p - 1}\)};
      \node at ( 9.5, 1.5) [below left]	 {\small\(q_{k - 1}^{p - 1}\)};
      \node at (10,   2.5) [below left]	 {\small\(b\)};
      \node at (10,   4	 ) [above right] {\small\(q_{k + 1}^p\)};
      \node at (11,   1.5) [below left]	 {\small\(a\)};
      \node at (12,   0.5) [above right] {\small\(q_k^p\)};

      \node at ( 4, -0.5) [below] {\((a)\)};
      \node at (10, -0.5) [below] {\((b)\)};
    \end{tikzpicture}
    \caption{Obtener \(\mathrm{FP}(p)\)
             desde \(\mathrm{FP}(p - 1)\)}
    \label{fig:LCS-WMMM-extend-FP}
  \end{figure}
  Ambos están sobre la diagonal \(k\).
  Del vértice con máxima coordenada \(j\) seguimos arcos diagonales
  hasta un vértice sin arco diagonal saliente
  (o llegamos al borde inferior de la grilla).
  Este es el vértice \(g_k\),
  cosa que demostraremos en el lema~\ref{lem:LCS-WMMM-1}.
  Luego se obtienen
    \(g_{\Delta + p}, g_{\Delta + (p - 1)}, \dotsc, g_{\Delta + 1}\),
  esta vez usando \(q_{k - 1}\) y \(g_{k + 1}\) para calcular \(g_k\).
  Finalmente se calcula \(g_\Delta\)
  de \(g_{\Delta - 1}\) y \(g_{\Delta + 1}\).

  El cálculo de \(\mathrm{FP}(p)\)
  a partir de \(\mathrm{FP}(p - 1)\)
  se formaliza en el lema~\ref{lem:LCS-WMMM-1},
  que da una recurrencia para \(\mathrm{fp}(k, p)\)
  en términos de la coordenada \(j\)
  de puntos más lejanos calculados previamente.
  Sea \(\mathrm{snake}(k, j)\) la coordenada \(j\) del punto más lejano
  sobre la diagonal \(k\) que puede alcanzarse desde \((j - k, j)\)
  atravesando arcos diagonales.
  Formalmente:
  \begin{equation*}
    \mathrm{snake}(k, j)
      = \max \{ r \colon x_{j + 1 - k} \dots x_{r - k} = y_{j + 1} \dots y_r \}
  \end{equation*}
  Que la recurrencia es correcta depende del manejo de casos límite:
  \(p = 0\), \(k = - p\) y \(k = \Delta + p\).
  Estas se resuelven limpiamente definiendo
  \(\mathrm{fp}(k, p) = -1\)
  siempre que \(p < 0\) o \(k \notin [-p, \Delta + p]\).
  \begin{lemma}
    \label{lem:LCS-WMMM-1}
    \begin{equation*}
      \mathrm{fp}(k, p)
        = \begin{cases}
            \mathrm{snake}(k,
               \max \{ \mathrm{fp}(k - 1, p) + 1,
                       \mathrm{fp}(k + 1, p - 1) \})
              & k \in [-p, \Delta - 1] \\
            \mathrm{snake}(k,
               \max \{ \mathrm{fp}(k - 1, p) + 1,
                       \mathrm{fp}(k + 1, p) \})
              & k = \Delta\\
            \mathrm{snake}(k,
               \max \{ \mathrm{fp}(k - 1, p - 1) + 1,
                       \mathrm{fp}(k + 1, p) \})
              & k \in [\Delta + 1, \Delta + p]
          \end{cases}
    \end{equation*}
  \end{lemma}
  \begin{proof}
    Demostramos el caso \(k < \Delta\),
    los demás casos son similares.
    Sea \(g\) el \(p\)\nobreakdash-punto más lejano en la diagonal \(k - 1\),
    y sea \(q\) el \((p - 1)\)\nobreakdash-punto más lejano
    en la diagonal \(k + 1\).
    Sean \(a\) el vértice inmediatamente a la derecha de \(g\),
    \(b\) el vértice inmediatamente inferior a \(q\),
    y \(d\) el vértice más lejano alcanzable
    desde el más lejano de \(a\) y \(b\) por arcos diagonales.
    La coordenada \(j\) de \(a\) es \(\mathrm{fp}(k - 1, p) + 1\),
    la de \(b\) es \(\mathrm{fp}(k + 1, p - 1)\),
    la de \(d\) es la dada por el primer caso del lema.

    La figura~\ref{fig:LCS-WMMM-lemma-1}
    ilustra los casos en que \(a\) está arriba de \(b\)
    (vale decir,
     \(\mathrm{fp}(k + 1, p) + 1 \le \operatorname{fp}(k, p - 1)\))
    y el caso en que \(b\) está sobre \(a\).
    \begin{figure}[ht]
      \centering
      \tikzstyle{vertex}
          = [circle, inner sep = 0pt, minimum size = 0.7mm, fill]
      \begin{tikzpicture}
        % Diagonals
        \foreach \i in {0, 1.5, 3, 6, 7.5, 9}
        {
          \draw[dotted, thin] (\i,   5) -- ( 5 + \i,   0);
        }
        \foreach \i in {5, 11}
        {
          \node at (0.1 + \i, -0.2) [below right, centered] {\small\(k - 1\)};
          \node at (1.6 + \i, -0.2) [below right, centered] {\small\(k\)};
          \node at (3.1 + \i, -0.2) [below right, centered] {\small\(k + 1\)};
        }
        % Subfigure (a)
        \draw (0, 5)	-- (1.5, 3.5) node[vertex] {};
        \draw[dashed] (1.5, 3.5) -- (3, 3.5) node[vertex] {};
        \draw (3, 3.5)	-- (3, 3.5) node[vertex] {};
        \draw (3, 5)	-- (4.5, 3.5) node[vertex] {};
        \draw[dashed] (4.5, 3.5) -- (4.5, 2) node[vertex] {};
        \draw ((4.5, 2) -- (6, 0.5) node[vertex] {};

        \node[vertex] at (7.5, 0.5) {};
        \node at (1.5, 3.5) [below left]  {\(g\)};
        \node at (3,   3.5) [above right] {\(a\)};
        \node at (4.5, 3.5) [above right] {\(q\)};
        \node at (4.5, 2)   [above right] {\(b\)};
        \node at (6,  0.5)  [above right] {\(d\)};
        \node at (7.5,0.5)  [above right] {\(c\)};

        % Subfigure (b)
        \draw (6, 5)	-- (9.5, 1.5) node[vertex] {};
        \draw[dashed] (9.5, 1.5) -- (11, 1.5)  node[vertex] {};
        \draw (11, 1.5) -- (12, 0.5)  node[vertex] {};
        \draw (9, 5)	-- (9.5, 4.5) node[vertex] {};
        \draw[dashed] (9.5, 4.5) -- (9.5, 3)   node[vertex] {};

        \node at (9.5, 4.5) [above right] {\(q\)};
        \node at (9.5, 3)   [above right] {\(b\)};
        \node at (9.5, 1.5) [below left]  {\(g\)};
        \node at (11, 1.5)  [above right] {\(a\)};
        \node at (12, 0.5)  [above right] {\(d\)};

        \node at ( 4, -0.5) [below] {\((a)\)};
        \node at (10, -0.5) [below] {\((b)\)};
      \end{tikzpicture}
      \caption{Los dos casos del lema~\ref{lem:LCS-WMMM-1}}
      \label{fig:LCS-WMMM-lemma-1}
    \end{figure}
    Trataremos solo el primer caso,
    el otro es similar.
    El valor \(P\) de \(d\) tiene que ser \(p\)
    dado que hay un camino a \(d\) con distancia comprimida \(p\)
    (el que pasa por \(q\) y \(b\)),
    si hubiese uno más corto,
    el vértice \(c\) de la figura tendría un valor de \(P\) menor a \(p - 1\),
    contradiciendo la elección de \(q\).
    Debemos demostrar además que \(d\) es el más lejano posible.
    Un camino de distancia \(p\) más largo no puede pasar por \(d\),
    eso contradiría la elección de \(d\).
    Eso quiere decir que pasa por un vértice a distancia \(p - 1\)
    en la diagonal \(k - 1\) bajo \(g\)
    o uno a distancia \(p - 1\) sobre la diagonal \(k\) bajo  \(q\),
    contradiciendo las elecciones de \(g\) y \(q\),
    respectivamente.
    O sea,
    tal camino no existe y \(d\) es el más lejano.
  \end{proof}
  El algoritmo~\ref{alg:WMMM} viene directamente del lema~\ref{lem:LCS-WMMM-1}.
  Note que para \(k \in [-p, \Delta - 1]\)
  se usan solo los valores \(\mathrm{fp}[k + 1, p - 1]\)
  y \(\mathrm{fp}[k - 1, p]\)
  al calcular \(\mathrm{fp}[k, p]\).
  Podemos almacenar \(\mathrm{fp}\) en un único arreglo con índice \(k\)
  si calculamos en orden de \(k\) creciente en este rango.
  Análogamente,
  si \(k \in [\Delta + 1, \Delta + p]\)
  conviene calcular en orden de \(k\) decreciente.
  \begin{algorithm}[htbp]
    \DontPrintSemicolon\Indp

    \Function{\(\mathrm{LCS}(X, Y, M, N)\)}{
       \(\mathrm{fp}[-(M + 1), \dotsc, (N + 1)] \gets -1\) \;
       \(\Delta \gets N - M\) \;
       \(p \gets -1\) \;
       \Repeat{\(\mathrm{fp}[\Delta] = N\)}{
         \(p \gets p + 1\) \;
         \For{\(k \gets -p\) \KwTo \(\Delta - 1\)}{
           \(\mathrm{fp}[k]
               \gets \mathrm{snake}(k,
                 \max \{ \mathrm{fp}[k - 1] + 1,
                         \mathrm{fp}[k + 1] \})\) \;
         }
         \For{\(k \gets \Delta + p\) \Downto \(\Delta + 1\)}{
           \(\mathrm{fp}[k]
               \gets \mathrm{snake}(k,
                 \max \{ \mathrm{fp}[k - 1] + 1,
                         \mathrm{fp}[k + 1] \})\) \;
         }
         \(\mathrm{fp}[\Delta]
             \gets	 \mathrm{snake}(\Delta,
                 \max \{ \mathrm{fp}[\Delta - 1] + 1,
                         \mathrm{fp}[\Delta + 1] \})\) \;
       }
       \Return \(M - p\) \;
    }
    \;
    \Function{\(\mathrm{snake}(k, j)\)}{
      \(i \gets j - k\) \;
      \While{\(j < M \wedge i < N \wedge X[j + 1] = Y[i + 1]\)}{
        \(i \gets i + 1; \quad j \leftarrow j + 1\) \;
      }
      \Return \(j\) \;
    }
    \caption{El algoritmo de Wu-Manber-Myers-Miller}
    \label{alg:WMMM}
  \end{algorithm}
  El algoritmo~\ref{alg:WMMM} toma tiempo \(O((M + N) P)\):
  el ciclo externo se ejecuta \(P\) veces,
  y es claro que para cada diagonal
  (de largo acotado por \(M + N\))
  cada nodo se considera a lo más una vez.

\section{Obtener la subsecuencia común más larga}
\label{sec:LCS-lineal}

  Nuevamente nuestro algoritmo solo da el largo buscado,
  no la secuencia común que buscamos
  (o las operaciones de edición que transforman \(X\) en \(Y\)).
  Registrar los valores necesarios
  para poder reconstruir directamente el camino óptimo
  requiere espacio \(O(N P)\),
  en el peor caso \(O(N M)\).
  Esto es inaceptable.
  Siguiendo la pista de Hirschberg~%
    \cite{hirschberg75:_linear_space_algor_comput_maxim_common_subseq},
  podemos reducir el espacio requerido a \(O(M + N)\)
  a costa de procesamiento adicional.

  La idea es aplicar el algoritmo~\ref{alg:WMMM}
  desde ambos extremos.
  Identificamos así la serpiente central de una subsecuencia común más larga,
  lo que divide el problema
  en determinar recursivamente subsecuencias comunes más largas
  llegando a ambos extremos de la serpiente identificada.
  O sea,
  usamos dividir y conquistar
  (ver el capítulo~\ref{cha:dividir-conquistar}).

  Es claro que la estrategia planteada deberá hallar una serpiente central
  en aproximadamente \(P/2\) pasos.
  Podemos plantear la recurrencia para el tiempo \(T(R, P)\) que toma
  el algoritmo para hallar las serpientes.
  Acá \(R = M + N\) y \(P\) es el número de pasos verticales.
  Para constantes \(\alpha\), \(\beta\) apropiadas,
  y \(R_1 + R_2 \le R\) tenemos:
  \begin{equation}
    \label{eq:recurrencia-WMMM}
    T(R, P)
      \le \begin{cases}
            \alpha R P
              + T(R_1, \lceil  P / 2 \rceil)
              + T(R_2, \lfloor P / 2 \rfloor) & P >   1 \\
            \beta R			      & P \le 1
          \end{cases}
  \end{equation}
  Como \(\lceil P / 2 \rceil \le 2 P / 3\) siempre que \(P \ge 2\),
  una simple inducción demuestra que
  \(T(R, P) \le 3 \alpha R P + \beta R\).
  O sea,
  esta división recursiva sigue tomando tiempo \(O(N P)\).
  Se requieren dos arreglos de tamaño \(M + N\),
  uno en cada dirección.
  Pero el espacio se requiere antes de la recursión,
  puede reutilizarse.

\bibliography{../referencias}

%%% Local Variables:
%%% mode: latex
%%% TeX-master: "../INF-221_notas"
%%% ispell-local-dictionary: "spanish"
%%% End:

% LocalWords:  subsecuencia LCS english Longest Common Subsequence et
% LocalWords:  hash SCCS RCS XML vertex circle inner sep pt minimum
% LocalWords:  size fill fd FP fp snake subsecuencias

\bibliographystyle{babplain-fl}

\chapter{Diseño de Algoritmos}
\label{cha:diseno-de-algoritmos}

  Para demostrar cómo se aplican las técnicas de diseño descritas,
  discutiremos un problema planteado por Bentley~%
    \cite{bentley84:_algorithm_design_techn}.
  Dado el arreglo \(a[n]\), hallar la máxima suma de un rango:
  \begin{equation}
    \label{eq:problema}
    \max_{i, j} \left\{ \sum_{i \le k \le j} a[k] \right\}
  \end{equation}
  Si todos los valores son positivos,
  la respuesta es obvia:
  la suma de todos los elementos del arreglo.
  El punto está si hay elementos negativos:
  ¿incluimos uno de ellos
  en la esperanza que los elementos positivos que lo rodean
  más que lo compensen?
  Finalmente,
  acordamos que la suma de un rango vacío es cero,
  y que en un arreglo de elementos negativos la suma máxima es cero.

\section{Algoritmo ingenuo}
\label{sec:algoritmo-1}

  La solución obvia,
  traducción directa de la especificación
  dada por la ecuación~\eqref{eq:problema},
  es la mostrada en el programa C del listado~\ref{lst:algoritmo-1}.
  \lstinputlisting[float,
                   language = C,
                   firstline = 8,
                   caption = {Algoritmo 1: Versión ingenua},
                   label = lst:algoritmo-1]
                  {code/max-sum-1.c}
  La complejidad del algoritmo~1 es \(O(n^3)\).
  Lo que buscamos es mejorarlo.

\section{No recalcular sumas}
\label{sec:no-recalcular-sumas}

  Hay dos ideas sencillas para evitar recalcular sumas.

\subsection{Extender sumas}
\label{sec:algoritmo-2}

  En vez de calcular la suma del rango cada vez,
  extendemos la suma anterior.
  Esto da el programa del listado~\ref{lst:algoritmo-2}.
  \lstinputlisting[float,
                   language = C,
                   firstline = 8,
                   caption = {Algoritmo 2: Evitar recalcular sumas},
                   label = lst:algoritmo-2]
                  {code/max-sum-2.c}
  La complejidad del algoritmo~2 es \(O(n^2)\).

\subsection{Sumas cumulativas}
\label{sec:algoritmo-3}

  Una manera de manejar rangos es usar sumas cumulativas,
  y obtener el valor para el rango restando.
  Esta idea da el listado~\ref{lst:algoritmo-3}.
  \lstinputlisting[float,
                   language = C,
                   firstline = 8,
                   caption = {Algoritmo 3: Usar arreglo cumulativo},
                   label = lst:algoritmo-3]
                  {code/max-sum-3.c}
  La complejidad del algoritmo~3 es \(O(n^2)\).
  Comparado a nuestro algoritmo original resulta una mejora,
  pero no respecto a la segunda variante.

\section{Dividir y Conquistar}
\label{sec:algoritmo-4}

  Aplicar la estrategia discutida en el capítulo~\ref{cha:dividir-conquistar}
  lleva a la figura~\ref{subfig:Algoritmo-4-AB}.
  Pero debemos también considerar que el rango con máxima suma
  esté a horcajadas,
  cruzando el punto central,
  como en la figura~\ref{subfig:Algoritmo-4-C}.
  \begin{figure}[ht]
    \centering
    \subfloat[Máximos en las mitades]{
      \begin{tikzpicture}[scale = 0.65]
        \draw[thick] (0, 0) rectangle (20, 1);
        \draw[thick] (10, 0) -- (10, 1);

        \draw[fill = lightgray, thick] (2, 0) rectangle (5, 1);
        \node at (3.5, 0) [below] {\(M_A\)};

        \draw[fill = lightgray, thick] (15, 0) rectangle (19, 1);
        \node at (17, 0) [below] {\(M_B\)};
      \end{tikzpicture}
      \label{subfig:Algoritmo-4-AB}
    } \\
    \subfloat[Máximo al medio]{
      \begin{tikzpicture}[scale = 0.65]
        \draw[fill = lightgray, thick] (7, 0) rectangle (11, 1);
        \draw[thick] (0, 0) rectangle (20, 1);
        \draw[thick] (10, 0) -- (10, 1);

        \node at (9, 0) [below] {\(M_C\)};
      \end{tikzpicture}
      \label{subfig:Algoritmo-4-C}
    }
    \caption{Dividir y conquistar}
    \label{fig:algoritmo-4}
  \end{figure}
  El algoritmo es el dado en el listado~\ref{lst:algoritmo-4}.
  \lstinputlisting[float,
                   language = C,
                   firstline = 8,
                   caption = {Algoritmo 4: Dividir y conquistar},
                   label = lst:algoritmo-4]
                  {code/max-sum-4.c}
  Usando el teorema maestro
  (teorema~\ref{theo:master-theorem}),
  para el algoritmo~4 son \(a = 2\), \(b = 2\) y  \(f(n) = O(n)\),
  por lo tanto la complejidad es \(O(n \log n)\).

\section{Un algoritmo lineal}
\label{sec:algoritmo-5}

  Otro algoritmo resulta de la idea,
  común al procesar arreglos,
  de tener una solución parcial hasta \(a[i]\),
  y analizar cómo extenderla para cubrir hasta \(a[i + 1]\).
  En nuestro caso,
  esto significa considerar la máxima suma que llega hasta \(a[i]\),
  y recordar la máxima suma vista hasta ahora,
  ver la figura~\ref{fig:Algoritmo-5}.
  Gries~%
    \cite{gries82:_note_strategy_loop_inv}
  deriva el algoritmo sistemáticamente
  y demuestra su correctitud.
  Esto da el algoritmo~5,
  del listado~\ref{lst:algoritmo-5}
  \begin{figure}[ht]
    \centering
    \begin{tikzpicture}[scale = 0.65]
      \draw[thick] (0, 0) rectangle (20, 1);

      \draw[fill = lightgray, thick] (2, 0) rectangle (8, 1);
      \node at (5, -0.1) [below] {\textsf{MaxSoFar}};

      \draw[fill = lightgray, thick] (11, 0) rectangle (16, 1);
      \node at (13.5, -0.1) [below] {\textsf{MaxToHere}};

      \draw[thick] (16, 1.1) -- (16, -0.1) node [below] {\textsf{i}};
    \end{tikzpicture}
    \caption{Extender la solución}
    \label{fig:Algoritmo-5}
  \end{figure}

  \lstinputlisting[float,
                   language = C,
                   firstline = 8,
                   caption = {Algoritmo 5: Ir extendiendo resultado parcial},
                   label = lst:algoritmo-5]
                  {code/max-sum-5.c}
  La complejidad del algoritmo~5 es \(O(n)\).
  Claramente es imposible tener una complejidad menor que \(n\),
  dado que es necesario revisar cada elemento del arreglo.

  \begin{table}[ht]
    \centering
    \begin{tabular}{r|*{5}{|c}}
      \multicolumn{1}{c||}{\textbf{Algoritmo}}
         & \textbf{1} & \textbf{2} & \textbf{4} & \textbf{5} \\
      \hline
      \multicolumn{1}{l||}{Líneas de C} & 8 & 7 & 14 & 7 \\
      \hline
      \multicolumn{1}{l||}{Tiempo en \([\mu \mathrm{s}]\)}
            & \(3.4 n^3\) & \(13 n^2\) & \(46 n \log n\) & \(33 n\) \\
      \hline
      Tiempo para \(n = {}\)
        \(10^2\) & \(3.4 [\mathrm{s}]\)	 & \(130 [\mathrm{ms}]\)
                 & \(30 [\mathrm{ms}]\)	 & \(3,3 [\mathrm{ms}]\) \\
        \(10^3\) & \(0,94 [\mathrm{h}]\) & \(14 [\mathrm{s}]\)
                 & \(0,45 [\mathrm{s}]\) & \(33 [\mathrm{ms}]\) \\
        \(10^4\) & \(39 \text{ días}\)	 & \(22 [\mathrm{min}]\)
                 & \(6,1 [\mathrm{s}]\)	 & \(0.33 [\mathrm{s}]\) \\
        \(10^5\) & \(108 \text{ años}\)	 & \(1,5 \text{ días}\)
                 & \(1,3[\mathrm{min}]\) & \(3,3 [\mathrm{s}]\) \\
        \(10^6\) & \(108 \text{ millones de años}\) & \(5 \text{ meses}\)
                 & \(15 [\mathrm{min}]\) & \(33 [\mathrm{s}]\)
    \end{tabular}
    \caption{Comparativa de Bentley~%
               \cite{bentley84:_algorithm_design_techn}
             entre las variantes}
    \label{tab:comparativa-algoritmos}
  \end{table}
  Reportar la complejidad de un algoritmo en términos de \(O(\cdot)\)
  es incompleto,
  pero el cuadro~\ref{tab:comparativa-algoritmos} muestra su relevancia.
  La ventaja es que la complejidad en estos términos es sencilla de obtener,
  en nuestros casos simples
  (algoritmos 1, 2, 3 y 5)
  por inspección,
  el teorema maestro da la complejidad para el algoritmo 4 directamente.

\section{Mayoría de una secuencia}
\label{sec:mayoria-secuencia}

  Se dice que un elemento de una secuencia de \(n\) elementos
  es \emph{mayoría} si es más de \(\lfloor n / 2 \rfloor\) elementos de ella.
  Nuestro problema es,
  dada una secuencia que sabemos tiene una mayoría,
  determinar cuál es ese elemento.

  Algoritmos obvios para resolver este problema son ordenar la secuencia
  y ver el elemento del medio
  (es claro que si cierto elemento
   es al menos la mitad de la secuencia ordenada,
   estará al medio,
   independiente de dónde comience la repetición).
  Esto lo podemos mejorar viendo que solo interesa el elemento del medio,
  basta determinar la mediana.
  O podemos ir contabilizando repeticiones de cada elemento,
  almacenándolos por ejemplo en un árbol
  o una tabla \emph{\foreignlanguage{english}{hash}}.

  La primera idea demora \(O(n \log n)\),
  y exige tener la secuencia completa a la mano.
  Es posible determinar la mediana en tiempo \(O(n)\),
  pero los algoritmos con engorrosos.
  La tercera puede procesar la secuencia conforme llega,
  usando tablas \emph{\foreignlanguage{english}{hash}},
  demora \(O(n)\) en promedio,
  pero requiere \(O(n)\) espacio adicional.

  Boyer y Moore~%
    \cite{boyer91:_mjrty_fast_majority_vote_algor}
  propusieron un algoritmo que lee la secuencia una sola vez
  y usa una cantidad mínima de espacio adicional.
  El algoritmo es el~\ref{alg:MJRTY}.
  \begin{algorithm}[ht]
    \DontPrintSemicolon\Indp

    \Function{\(mathrm{MJRTY}(S)\)}{
      \(\mathrm{count} \gets 0\) \;
      \ForEach{\(x \in S\)}{
        \uIf{\(\mathrm{count} = 0\)}{
          \(m \gets x\) \;
          \(\mathrm{count} \gets 1\) \;
        }
        \uElseIf{\(m = x\)}{
          \(\mathrm{count} \gets \mathrm{count} + 1\) \;
        }
        \Else{
          \(\mathrm{count} \gets \mathrm{count} - 1\) \;
        }
        \Return \(m\) \;
      }
    }
    \caption{Algoritmo MJRTY de Boyer-Moore}
    \label{alg:MJRTY}
  \end{algorithm}
  La idea es ir contabilizando elementos iguales al candidato a mayoría,
  si aparecen más elementos diferentes que iguales a la mayoría propuesta,
  cambiamos de candidato.

  La demostración de que el misterioso algoritmo~\ref{alg:MJRTY} es correcto
  es como sigue:
  sea \(c\) una variable fantasma cuyo valor es el de \(\mathrm{count}\)
  si \(m\) es la mayoría,
  \(- \mathrm{count}\) en caso contrario.
  Cada vez que el algoritmo se encuentra con un valor igual a la mayoría,
  \(c\) aumenta
  (si \(m\) es la mayoría,
   aumenta \(\mathrm{count}\) en uno;
   si no es la mayoría,
   \(\mathrm{count}\) disminuye en uno);
  cada vez que encuentra un valor diferente a la mayoría,
  \(c\) puede aumentar o disminuir en uno.
  Habrán más aumentos que disminuciones de \(c\),
  con lo que \(c\) al final del algoritmo es positivo.
  Pero esto solo es posible si \(\mathrm{count}\) es positivo,
  y esto a su vez es solo si \(m\) es la mayoría.

  Está claro que este algoritmo siempre retorna un valor,
  no detecta si la secuencia no tiene mayoría.
  Verificar esto requiere una segunda pasada por la secuencia.

\bibliography{../referencias}

%%% Local Variables:
%%% mode: latex
%%% TeX-master: "../INF-221_notas"
%%% ispell-local-dictionary: "spanish"
%%% End:

% LocalWords:  eq cumulativas correctitud english hash MJRTY

\bibliographystyle{babplain-fl}

\chapter{Programación lineal}
\label{cha:programacion-lineal}

  Muchos de los problemas que queremos resolver
  son problemas de \emph{optimización}:
  hallar el camino \emph{más corto},
  un árbol recubridor de \emph{costo mínimo},
  la subsecuencia común \emph{más larga}.
  Tenemos ciertas reglas:
  hay que cumplir \emph{restricciones}
  (solo usar arcos del grafo,
   los objetos deben ir en el orden de las secuencias)
  y debe ser \emph{mejor posible} según un criterio bien definido.

  Un conjunto amplio de problemas de optimización
  es la \emph{programación lineal}.
  Como nota al margen,
  en esto \textquote{programación}
  se usa en el sentido de \textquote{planificación} o \textquote{asignar recursos},
  no \textquote{elaborar programas para computadora}.
  Tenemos una colección de restricciones lineales a cumplir,
  para ellas nos interesa obtener el óptimo de una función objetivo
  también lineal.
  Muchísimos problemas pueden plantearse en esta forma,
  su solución es un problema muy importante
  que ha dado lugar a extensas investigaciones y extensiones.
  Mucho más detalle del que podremos dar en este limitado espacio
  dan Ferguson~%
    \cite{ferguson15:_linear_programming},
  Dasgupta, Papadimitriou y Vazirani~%
    \cite[capítulo~7]{dasgupta06:_algorithms}
  y también Erickson~%
    \cite{erickson19:_algorithms},
  quien reseña un poco de la historia del problema.
  Sabemos que la variante del problema con variables
  enteras es \NP\nobreakdash-completo,
  nos interesa el caso de variables reales.

\section{Solución gráfica}
\label{sec:solucion-grafica}

  Como un primer ejemplo,
  consideremos el problema de hallar \(x_1\) y \(x_2\)
  tales que \(x_1 \ge 0\), \(x_2 \ge 0\) y:
  \begin{equation}
    \label{eq:lp-restricciones}
    \sysdelim..
    \systeme{
         x_1 +	 x_2 \leq \phantom{0}9,
         x_1 + 3 x_2 \leq 12,
      \- x_1 + 2 x_2 \leq \phantom{0}2
    }
  \end{equation}
  tal que sea máximo \(x_1 + 2 x_2\).
  Lo relevante en este ejemplo es que las restricciones
  y la función objetivo son todas lineales.
  Gráficamente podemos visualizar este problema
  como en la figura~\ref{fig:lp-example}.
  \begin{figure}[ht]
    \centering
      \begin{tikzpicture}
        \coordinate (p0) at (0, 0);
        \coordinate (p1)
          at (intersection of {0, 0 -- 0, 11} and {0, 1 -- 10, 6});
        \coordinate (p2)
          at (intersection of {0, 4 -- 10, 1} and {0, 1 -- 10, 6});
        \coordinate (p3)
          at (intersection of {0, 9 -- 9, 0} and {0, 4 -- 10, 1});
        \coordinate (p4)
          at (intersection of {0, 9 -- 9, 0} and {0, 0 -- 11, 0});

        \fill[yellow] (p0) -- (p1) -- (p2) -- (p3) -- (p4) -- (p0);

        \draw[latex'-latex'] (0, 11) node [below left] {\(x_2\)}
                                -- (0, 0)
                                -- (11,	 0) node [below left] {\(x_1\)};

        \draw (0,  9) -- ( 9, 0)
                node [above, pos = 0.2, sloped] {\(x_1 + x_2 = 9\)};
        \draw (0,  4) -- (10, 1)
                node [above, pos = 0.2, sloped] {\(x_1 + 3 x_2 = 12\)};
        \draw (0,  1) -- (10, 6)
                node [above, pos = 0.8, sloped] {\(-x_1 + 2 x_2 = 2\)};

        \draw[gray, dashed]  (0,  2) -- ( 4, 0)
                node [black, above, pos = 0.8, sloped] {\(x_1 + 2 x_2 = 2\)};
        \draw[gray, dashed]  (0,  4) -- ( 8, 0)
                node [black, above, pos = 0.8, sloped] {\(x_1 + 2 x_2 = 4\)};
        \draw[gray, dashed]  (0,  6) -- (10, 1)
                node [black, above, pos = 0.8, sloped] {\(x_1 + 2 x_2 = 6\)};

      \end{tikzpicture}
    \caption{Un problema de programación lineal}
    \label{fig:lp-example}
  \end{figure}
  El área en amarillo es el área factible,
  donde se cumplen las restricciones.
  La geometría de la situación nos indica que el máximo
  de la función objetivo se encuentra en uno de los vértices del área factible,
  en este caso en la intersección
  entre \(x_1 + 3 x_2 = 12\)
  y \(x_1 + x_2 = 9\),
  que es \(x_1 = 15/2\) y \(x_2 = 3/2\),
  donde vale \(21/2\).

  El área factible está limitada por las restricciones
  (cada restricción define un semiplano,
   el área factible es la intersección entre ellos),
  puede ser un polígono convexo
  (y hay soluciones factibles),
  puede no haber soluciones factibles
  (la intersección entre los semiplanos es vacía),
  o el área puede ser ilimitada,
  en cuyo caso puede o no haber óptimo.

\section{Problemas estándar}
\label{sec:problemas-estandar}

  En nuestro ejemplo,
  tenemos restricciones de que las variables son no negativas,
  que ciertas expresiones lineales en las variables
  son menores o iguales que una constante,
  y buscamos el máximo de una expresión lineal.
  Nuestro ejemplo es un \emph{problema estándar de máximo},
  si son solo restricciones de mayores o iguales
  y minimizamos la función objetivo
  se le llama \emph{problema estándar de mínimo}.
  Todo problema de programación lineal puede llevarse a una de estas formas
  usando las siguientes:
  Si la variable \(x\) no tiene restricciones,
  podemos reemplazarla por \(x^+\) y \(x^-\),
  ambas no negativas,
  y la reemplazamos por \(x^+ - x^-\).
  Si tenemos restricciones lineales de mayor o igual,
  basta multiplicarlas por \(-1\).
  Si hay igualdades,
  podemos despejar alguna de las variables de ellas
  y reemplazar en las demás restricciones para eliminarlas.
  Si buscamos un mínimo,
  nuevamente basta multiplicar por \(-1\) la función objetivo.

  Para simplificar notación,
  para vectores \(\mathbf{x}\) e \(\mathbf{y}\)
  escribiremos por ejemplo \(\mathbf{x} \ge \mathbf{y}\)
  si cada componente de \(\mathbf{x}\)
  es mayor o igual al elemento correspondiente de \(\mathbf{y}\).
  Así el problema máximo estándar puede expresarse como:
  \begin{equation}
    \label{eq:std-max-problem}
    \begin{aligned}
      &\text{maximizar \(\mathbf{c}^T \mathbf{x}\)} \\
      &\text{sujeto a las restricciones
               \(\mathbf{A} \mathbf{x} \le \mathbf{b}\)
               y \(\mathbf{x} \ge \mathbf{0}\)}
    \end{aligned}
  \end{equation}
  Una manera alternativa de ver el problema~\ref{eq:std-max-problem}
  es buscar una cota superior al valor buscado.
  Esto lleva a considerar combinaciones lineales de las restricciones,
  sujetas a la condición
  que los coeficientes de las variables
  no pueden sobrepasar los coeficientes respectivos en la función objetivo,
  y buscamos la combinación que nos da el mínimo valor posible.
  O sea,
  tenemos un sistema para el vector \(\mathbf{y}\):
  \begin{equation*}
    (\text{restricción 1}) y_1 + (\text{restricción 2}) y_2
      + \dotsb + (\text{restricción m}) y_m
  \end{equation*}
  Decir que este valor es mínimo dados los lados derechos de las restricciones
  se traduce en:
  \begin{equation*}
    \mathbf{y}^T \mathbf{c} \text{\ mínimo}
  \end{equation*}
  Las restricciones de no sobrepasar los coeficientes en la función objetivo
  se traducen en:
  \begin{equation*}
    \mathbf{A}^T \mathbf{y} \le \mathbf{b}
  \end{equation*}
  Obtuvimos un problema estándar de mínimo,
  el \emph{dual} del problema~\eqref{eq:std-max-problem}:
  \begin{equation}
    \label{eq:std-max-problem-dual}
    \begin{aligned}
      &\text{minimizar \(\mathbf{y}^T \mathbf{b}\)} \\
      &\text{sujeto a las restricciones
               \(\mathbf{y}^T \mathbf{A} \ge \mathbf{c}^T\)
               e \(\mathbf{y} \ge \mathbf{0}\)}
    \end{aligned}
  \end{equation}
  Los vectores \(b\) y \(c\) cambiaron de posición,
  y la matriz se traspone.
  Repetir el ejercicio para el sistema resultante
  nos lleva de vuelta al problema original.
  El punto interesante es que los valores óptimos de ambos problemas
  están relacionados.
  Se dice que~\eqref{eq:std-max-problem} y~\eqref{eq:std-max-problem-dual}
  son \emph{duales}.
  El problema original suele llamarse \emph{primal}
  para distinguirlo del \emph{dual}.

  La relación entre problemas duales está dada por los siguientes resultados.
  \begin{theorem}
    \label{theo:lp-feasible-dual}
    Si \(\mathbf{x}\) es factible
    para el problema estándar de maximización~\eqref{eq:std-max-problem}
    y es factible \(\mathbf{y}\) para su dual~\eqref{eq:std-max-problem-dual},
    entonces:
    \begin{equation}
      \label{eq:lp-feasible-dual}
      \mathbf{c}^T \mathbf{x}
        \le \mathbf{y}^T \mathbf{b}
    \end{equation}
  \end{theorem}
  \begin{proof}
    \begin{equation*}
      \mathbf{c}^T \mathbf{x}
        \le \mathbf{y}^T \mathbf{A} \mathbf{x}
        \le \mathbf{y}^T \mathbf{b}
    \end{equation*}
    La primera desigualdad
    es por \(\mathbf{c}^T \le \mathbf{y}^T \mathbf{A}\)
    con \(\mathbf{x} \ge \mathbf{0}\),
    la segunda de \(\mathbf{A} \mathbf{x} \le \mathbf{b}\)
    con \(\mathbf{y} \ge \mathbf{0}\).
  \end{proof}
  \begin{corollary}
    \label{cor:lp-feasible-dual-1}
    Si un problema estándar y su dual son ambos factibles,
    ambos son factibles acotados.
  \end{corollary}
  \begin{proof}
    Si \(\mathbf{y}\) es factible para el problema mínimo,
    por~\eqref{eq:lp-feasible-dual}
    sabemos que \(\mathbf{y}^T \mathbf{b}\) es una cota superior
    para los valores de \(\mathbf{c}^T \mathbf{x}\) de un \(\mathbf{x}\)
    factible del problema máximo.
    El converso es similar.
  \end{proof}
  \begin{corollary}
    \label{cor:lp-feasible-dual-2}
    Si hay vectores factibles \(\mathbf{x^*}\)
    para el problema estándar máximo~\eqref{eq:std-max-problem}
    e \(\mathbf{y^*}\)
    para el problema dual~\eqref{eq:std-max-problem-dual}
    tales que \(\mathbf{c}^T \mathbf{x^*} = \mathbf{y^*}^T \mathbf{b}\),
    ambos son óptimos.
  \end{corollary}
  \begin{proof}
    Si \(\mathbf{x}\) es una solución factible de~\eqref{eq:std-max-problem},
    entonces
    \(\mathbf{c}^T \mathbf{x}
        \le \mathbf{y^*}^T \mathbf{b}
        = \mathbf{c}^T \mathbf{x^*}\),
    por lo que \(\mathbf{x^*}\) es óptimo.
    Un argumento simétrico se aplica a \(\mathbf{y^*}\).
  \end{proof}
  Demostraremos el resultado fundamental siguiente usando el método Simplex
  para resolver problemas de programación lineal más adelante.
  En resumen,
  si el problema o su dual es factible acotado,
  se cumplen las condiciones del corolario~\ref{cor:lp-feasible-dual-2}.
  \begin{theorem}[Dualidad]
    \label{theo:lp-duality}
    Si un problema estándar es factible acotado,
    también lo es su dual,
    sus óptimos son iguales
    y hay vectores óptimos para ambos.
  \end{theorem}
  Básicamente,
  de las nueve combinaciones posibles de factible acotado,
  factible sin cota
  y no factible para el problema y su dual
  por el corolario~\ref{cor:lp-feasible-dual-1} tres no son posibles
  (si el problema y su dual son factibles,
   ambos son factibles acotados),
  mientras el teorema~\ref{theo:lp-duality} elimina dos más.
  Las combinaciones restantes
  (ambos factibles no acotados,
   uno factible no acotado y el otro no factible
   y ambos no factibles)
  son posibles.

  Como corolario del teorema de dualidad,
  tenemos:
  \begin{theorem}[Equilibrio]
    Sean \(\mathbf{x^*}\) e \(\mathbf{y^*}\) factibles
    para el problema~\eqref{eq:std-max-problem}
    y su dual~\eqref{eq:std-max-problem-dual},
    respectivamente.
    Entonces \(\mathbf{x^*}\) e \(\mathbf{y^*}\) son óptimos si y solo si:
    \begin{alignat}{4}
      y_i^* &= 0
         &\quad& \text{para todo \(i\) para los cuales
                 \(\sum_{j} a_{i j} x_j^* < b_i\)}
         \label{eq:lp-equilibrium-y} \\
      x_j^* &= 0
         && \text{para todo \(j\) para los cuales
                 \(\sum_{i}  y_i^* a_{i j} > c_j\)}
         \label{eq:lp-equilibrium-x}
    \end{alignat}
  \end{theorem}
  \begin{proof}
    Demostramos implicancia en ambas direcciones.
    Primero,
    solo si para el índice \(i\) es \(\sum_{j} a_{i j} x_j^* = b_i\)
    puede ser \(y_i^* \ne 0\),
    y simétricamente
    solo si para \(j\) es \(\sum_{i} y_i^* a_{i j} = c_j\)
    es posible \(x_j^* \ne 0\),
    con lo que:
    \begin{equation*}
      \sum_i y_i^* b_i
        = \sum_{i, j} y_i^* a_{i j} x_j^*
        = \sum_j c_j x_j^*
    \end{equation*}
    Pero eso es \(\mathbf{y^*}^T \mathbf{b} = \mathbf{c}^T \mathbf{x^*}\),
    por el corolario~\ref{cor:lp-feasible-dual-2} ambos son óptimos.

    En la dirección contraria,
    por el corolario~\ref{cor:lp-feasible-dual-2}
    si \(\mathbf{x^*}\) e \(\mathbf{y^*}\) son óptimos,
    es \(\mathbf{y^*}^T \mathbf{b} = \mathbf{c}^T \mathbf{x^*}\).
    Como \(\mathbf{x^*} \ge \mathbf{0}\)
    e \(\mathbf{y^*} \ge \mathbf{0}\),
    esta igualdad es imposible
    a menos que se cumplan~\eqref{eq:lp-equilibrium-y}
    y~\eqref{eq:lp-equilibrium-x}.
  \end{proof}

  Esto hace preguntarse el significado del vector \(\mathbf{y^*}\)
  óptimo del problema dual.
  Consideremos para ello el tipo de problema de programación lineal
  conocido como \emph{problema de dieta}:
  tenemos ciertas necesidades diarias de componentes de dieta
  (carbohidratos,
   proteínas,
   diversas vitaminas,
   minerales,
   ácidos grasos y aminoácidos esenciales).
  Sea~\(b_j\) la cantidad mínima diaria
  que debemos ingerir del componente~\(j\).
  Cada alimento provee ciertas cantidades de cada componente,
  el alimento~\(i\) provee \(a_{i j}\) del nutriente \(j\) por gramo,
  su costo es \(c_i\) por gramo.
  Si nos interesa la dieta de mínimo costo
  que cubre nuestras necesidades alimenticias,
  este es un problema de mínimo estándar,
  definido por la matriz \(\mathbf{A}\),
  el vector de restricciones \(\mathbf{b}\)
  y el vector de costos \(\mathbf{c}\).
  El vector óptimo \(\mathbf{x^*}\)
  expresa la cantidad de cada alimento a ingerir diariamente
  para cubrir las necesidades alimenticias a mínimo costo,
  el valor correspondiente de la función objetivo \(\mathbf{c}^T \mathbf{x^*}\)
  es el costo diario de la dieta más barata.
  La función objetivo del problema dual en su óptimo,
  \(\mathbf{y^*}^T \mathbf{b}\) tiene el mismo valor
  (por el corolario~\ref{cor:lp-feasible-dual-2}),
  los \(y_j^*\) representan
  lo que estamos pagando por unidad del componente~\(j\)
  en la dieta óptima.
  El hecho que los \(y_j^*\) se anulan cuando hay holgura en una restricción
  significa que adquirir una unidad adicional de \(j\) sobre la necesaria
  es gratis.

\section{Geometría de programación lineal}
\label{sec:geometria-programacion-lineal}

  Muchos problemas de asignación de recursos
  son naturalmente programas lineales.
  Hay muchos otros problemas que se pueden traducir en programación lineal.
  Es importante contar con algoritmos eficientes
  para resolver esta clase de problemas.
  El método Simplex~%
    \cite{dantzig47:_Simplex}
  es eficiente en la práctica
  (aunque es necesario tener algunas precauciones que discutiremos luego).
  Claro que Klee y Minty~%
    \cite{klee69:_how_good_simplex_algo}
  hallaron una familia infinita de modelos
  (hipercubos en \(d\)~dimensiones)
  en los cuales Simplex
  usando la regla de pivote de Dantzig requiere \(2^d\)~iteraciones
  para llegar al óptimo.
  En la práctica es mucho mejor,
  se ha demostrado que en hipercubos de \(d\)~dimensiones,
  partiendo de un vértice al azar en promedio son \(d\)~iteraciones,
  ver por ejemplo a Schrijver~%
    \cite{schrijver98:_theo_linear_integ_progr}
  o a Borgwardt~%
    \cite{borgwardt87:_simplex_method}.
  Sigue siendo un problema abierto si hay una regla de elección de pivotes
  que garantiza tiempo polinomial.
  Recién en 1979
  Khachiyan~%
    \cite{khachiyan79:_polynomial_algo_linear_programming}
  dio un algoritmo polinomial,
  aunque en la práctica es muy lento.
  Karmarkar~%
    \cite{karmarkar84:_new_polynomial_algorithm_linear_progr}
  luego dio un algoritmo polinomial que en algunos casos es competitivo
  en la práctica.

  Discusión detallada de Simplex provee Reveliotis~%
    \cite{reveliotis97:_intro_linear_progr_simplex_method},
  parte de lo que sigue se adapta de allí.

  Si tenemos \(n\) variables,
  estamos operando en un espacio con \(n\)~dimensiones.
  Las restricciones corresponden a hiperplanos,
  cada una define un semi-espacio en el cual la restricción se cumple.
  La intersección de tales semi-espacios
  (donde se cumplen las restricciones)
  define lo que se denomina \emph{politopo},
  un politopo acotado se llama \emph{poliedro}.

  Dados puntos \(\mathbf{x}_1 = (x_{1 1}, x_{1 2}, \dotsc, x_{1 n})^T\)
  y \(\mathbf{x}_2 = (x_{2 1}, x_{2 2}, \dotsc, x_{2 n})^T\)
  la línea recta que pasa por ellos puede expresarse como:
  \begin{equation*}
    \mathbf{x}_1 + t (\mathbf{x}_2 - \mathbf{x_1}),
    t \in \mathbb{R}
  \end{equation*}
  El segmento de línea recta entre \(\mathbf{x}_1\) y \(\mathbf{x}_2\)
  es el conjunto:
  \begin{equation*}
    \mathbf{x}_1 + t (\mathbf{x}_2 - \mathbf{x_1}),
    t \in [0, 1]
  \end{equation*}
  Equivalentemente:
  \begin{equation*}
    \mu \mathbf{x}_1 + \kappa \mathbf{x}_2,
    \mu + \kappa = 1,
    \mu, \kappa \ge 0
  \end{equation*}
  Decimos que esto define
  una \emph{combinación convexa} de \(\mathbf{x}_1\) y \(\mathbf{x}_2\).
  Se dice que un conjunto de puntos es \emph{convexo}
  si el segmento que conecta dos puntos del conjunto
  está enteramente en su interior.
  Vale decir,
  si \(\mathbf{x}_1, \mathbf{x}_2 \in S\),
  entonces \((1 - t) \mathbf{x}_1 + t \mathbf{x}_2 \in S\)
  para todo \(0 \le t \le 1\).
  Es claro que el semi-espacio definido por una restricción es convexo,
  y por tanto lo es todo politopo resultante de sus intersecciones.
  Como último concepto,
  un \emph{punto extremo} de un conjunto convexo \(S\)
  es un punto \(\mathbf{x}_0\)
  tal que todo segmento que está enteramente en \(S\)
  y que contiene \(\mathbf{x}_0\)
  tiene a \(\mathbf{x}_0\) como uno de sus extremos.
  O sea,
  si \(\mathbf{x}_1, \mathbf{x}_2 \in S\),
  y para \(\mathbf{x}_0 \in S\) podemos escribir
     \(\mathbf{x}_0 = (1 - \kappa) \mathbf{x}_1 + \kappa \mathbf{x}_2\),
  entonces \(\mathbf{x}_0 = \mathbf{x}_1\) o \(\mathbf{x}_0 = \mathbf{x}_2\).
  En el caso particular de politopos convexos
  se habla de \emph{vértices} para referirse a los puntos extremos.
  Esto nos lleva a:
  \begin{theorem}[Fundamental de la programación lineal]
    Si un programa lineal tiene una solución óptima acotada,
    existe un punto extremo de la región factible que es óptimo.
  \end{theorem}
  Note que pueden haber varios vértices óptimos
  si resulta que un arista o una cara del politopo
  es paralela al plano que define la función objetivo.

\subsection{El método Simplex}
\label{sec:simplex}

  El método Simplex revisa distintos puntos extremos
  (vértices)
  del politopo definido por las restricciones.
  El nombre del método
  viene de que esos puntos extremos normalmente coinciden con vértices
  de poliedros con el mínimo número de caras
  si se omiten restricciones no involucradas
  (en dos dimensiones,
   un triángulo;
   en tres es tetraedro;
   en general,
   en \(n\) dimensiones tiene \(n + 1\) caras).
  A tales poliedros se les llama \emph{simplex}.

  Volvamos a nuestro primer ejemplo,
  maximizar \(x_1 + 2 x_2\) sujeto a \(x_i \ge 0\) y:
  \begin{equation}
    \label{eq:lp-restricciones-2}
    \sysdelim..
    \systeme{
         x_1 +	 x_2 \leq \phantom{0}9,
         x_1 + 3 x_2 \leq 12,
      \- x_1 + 2 x_2 \leq \phantom{0}2
    }
  \end{equation}
  Trabajar con desigualdades es incómodo,
  introducimos variables holgura
  (en inglés,
   \emph{\foreignlanguage{english}{slack variables}})
  para cada desigualdad,
  que también cumplirán la restricción \(s_i \ge 0\).
  Agregamos una igualdad adicional con holgura \(z\) para la función objetivo:
  \begin{equation}
    \label{eq:lp-igualdades}
    \sysdelim..
    \syssubstitute{{y_1}{s_1}{y_2}{s_2}{y_3}{s_3}}
    \systeme{
         x_1 +	 x_2 + y_1     = \phantom{0}9,
         x_1 + 3 x_2 + y_2     = 12,
      \- x_1 + 2 x_2 + y_3     = \phantom{0}2,
      \- x_1 - 2 x_2	   + z = \phantom{0}0
    }
  \end{equation}
  De~\eqref{eq:lp-igualdades} una solución factible al problema es obvia:
  haga \(x_1 = x_2 = 0\),
  y \(s_1 = 9\), \(s_2 = 12\), \(s_3 = 2\).
  Se dice que \(s_1\) a \(s_3\) están en la base
  las demás variables no
  (estamos trabajando en el espacio vectorial definido por las variables,
   las variables en la base están siendo usadas como base para el subespacio
   definido por las ecuaciones).
  Claro que esto nos da \(z = 0\) para la función objetivo.
  De la última fila vemos que podemos aumentar \(z\)
  aumentando \(x_1\) o \(x_2\)
  (aparecen con coeficientes negativos).
  Analicemos \(x_2\).
  De la primera ecuación vemos que podemos aumentar \(x_2\) hasta \num{9},
  dejando \(s_1 = 0\);
  la segunda indica que \(x_2 = 4\) deja \(s_2 = 0\);
  la tercera dicta \(x_2 = 1\) para \(s_3 = 0\).
  Debemos elegir el mínimo de estos
  (no se permiten holguras negativas).
  Usamos la tercera ecuación para eliminar \(x_2\) de las demás ecuaciones:
  \begin{equation}
    \label{eq:lp-pivot-x2}
    \sysdelim..
    \syssubstitute{{y_1}{s_1}{y_2}{s_2}{y_3}{s_3}}
    \systeme{
         \frac{3}{2} x_1	      +	 y_1	   - \frac{1}{2} y_3	 = 8,
         \frac{5}{2} x_1		     + y_2 - \frac{3}{2} y_3	 = 9,
      \- \frac{1}{2} x_1 + x_2			   + \frac{1}{2} y_3	 = 1,
      \- 2	     x_1			   +		 y_3 + z = 2
    }
   \end{equation}
   El sistema resultante tiene la misma forma que el original,
   claro que las variables independientes
   (que aparecen en una ecuación solamente)
   ahora son \(x_2, s_1, s_2\).
   A esta operación le llaman \emph{pivotear}
   (alrededor de \((x_2, s_3)\)),
   \(x_2\) entra a la base mientras \(s_3\) sale.
   El resultado corresponde
   a la solución factible \(s_1 = 8, s_2 = 9, x_2 = 1\)
   con \(x_1 = s_3 = 0\);
   la función objetivo es \(z = 2\).
   Ahora la única variable con coeficiente negativo
   en la última fila es \(x_1\).
   Debemos elegir entre \(16/3\) y \(18/5\),
   es la segunda ecuación:
  \begin{equation}
    \label{eq:lp-pivot-x1}
    \sysdelim..
    \syssubstitute{{y_1}{s_1}{y_2}{s_2}{y_3}{s_3}}
    \systeme{
                                 y_1 - \frac{3}{5} y_2 + \frac{2}{5} y_3
           = \frac{13}{5},
                    x_1		     + \frac{2}{5} y_2 - \frac{3}{5} y_3
           = \frac{18}{5},
                        + x_2	     + \frac{1}{5} y_2 + \frac{1}{5} y_3
           = \frac{14}{5},
                                       \frac{4}{5} y_2 - \frac{1}{5} y_3
                + z
           = \frac{46}{5}
    }
   \end{equation}
   Esto corresponde a \(x_1 = 18/5\), \(x_2 = 14/5\), \(s_1 = 13/5\)
   y función objetivo \(z = 46/5\).
   Coeficiente negativo en la última fila tiene \(s_3\);
   como antes elegimos el menor,
   en este caso \(13/2\) de la primera ecuación,
   resultando:
  \begin{equation}
    \label{eq:lp-pivot-s3}
    \sysdelim..
    \syssubstitute{{y_1}{s_1}{y_2}{s_2}{y_3}{s_3}}
    \systeme{
    % \frac{1}{1} x_1 + \frac{1}{2} x_2
    %	 + \frac{1}{3} s_1 + \frac{1}{4} s_2 + \frac{1}{5} s_3 + z
    %	     = \frac{x}{y},
      % No x_1, x_2
            \frac{5}{2} y_1 - \frac{3}{2} y_2 +		   y_3
             = \frac{13}{2},
                  x_1
          + \frac{3}{2} y_1 - \frac{1}{2} y_2
             = \frac{15}{2},
                                    x_2
          - \frac{1}{2} y_1 + \frac{1}{2} y_2
             = \frac{3}{2},
      % No x_1, x_2
           \frac{1}{2} y_1 + \frac{1}{2} y_2		       + z
             = \frac{21}{2}
    }
  \end{equation}
  No hay coeficientes negativos en la última fila,
  es óptimo.
  Es \(x_1 = 15/2, x_2 = 3/2, s_3 = 13/2\)
  con función objetivo \(z = 21/2\).
  Esto es lo mismo que habíamos obtenido de nuestra solución gráfica.
  Si analizamos el camino seguido por nuestra técnica,
  caminamos de un vértice a un vecino hasta alcanzar el óptimo.
  En cada paso aumenta la función objetivo.

  Es obvio abreviar esto registrando solo los coeficientes en una matriz.
  Vemos también que en la matriz aparecen muchos ceros,
  y hay columnas con un único \num{1}.
  En vez de representar esto explícitamente,
  podemos indicar para cada fila cuál es la columna (variable)
  que tiene coeficiente uno
  y cuyo coeficiente es cero en el resto de la columna.
  Podemos reutilizar el espacio que se abre al eliminar una variable,
  llenándolo con los coeficientes de la nueva columna.
  Para ello debemos indicar a qué variable corresponde cada columna,
  y qué variable es la que aparece como independiente en la fila.
  A esta representación se llama un \emph{\foreignlanguage{french}{tableau}}
  (plural del francés es \emph{\foreignlanguage{french}{tableaux}}).
  El efecto de la operación
  de pivote alrededor de \(p\) sobre elementos en su misma fila y columna
  y otros elementos se describe mediante:
  \begin{equation}
    \label{eq:pivote}
    \boxed{
      \begin{array}{cc}
        p & a \\
        b & c
      \end{array}
    }
    \rightsquigarrow
    \boxed{
      \begin{array}{cc}
          1 / p & a / p \\
        - c / p & c - a b / p
      \end{array}
    }
  \end{equation}

  Ilustramos el proceso de trabajar
  con \emph{\foreignlanguage{french}{tableaux}}
  con el ejemplo
  de maximizar \(5 x_1 + 2 x_2 + x_3\) con las restricciones:
  \begin{equation}
    \label{eq:lp-example}
    \sysdelim..
    \systeme{
        x_1 + 3 x_2 - x_3 \leq 6,
                x_2 + x_3 \leq 4,
      3 x_1 +	x_2	  \leq 7
    }
  \end{equation}
  El \emph{\foreignlanguage{french}{tableau}} inicial es:
  \begin{center}
    \begin{tabular}[ht]{>{\(}c<{\)}|*{3}{>{\(}c<{\)}}|>{\(}c<{\)}}
            & x_1 & x_2 & x_3 & \\
      \hline
        s_1 &	1 &   3 &  -1 & 6 \\
        s_2 &	0 &   1 &   1 & 4 \\
        s_3 &	3 &   1 &   0 & 7 \\
      \hline
            &  -5 &  -2 &  -1 & 0
    \end{tabular}
  \end{center}
  Pivoteando alrededor de \(s_3, x_1\)
  (intercambia el papel de estas variables)
  da:
  \begin{center}
    \begin{tabular}[ht]{>{\(}c<{\)}|*{3}{>{\(}c<{\)}}|>{\(}c<{\)}}
            & s_3  & x_2 & x_3 & \\
      \hline
        s_1 & -1/3 &  8/3 &  -1 & 11/3 \\
        s_2 &	0  &  1	  &   1 &  4   \\
        x_1 &  1/3 &  1/3 &   0 &  7/3 \\
      \hline
            &  5/3 & -1/3 &  -1 & 35/3
    \end{tabular}
  \end{center}
  Ahora pivotear \((s_2, x_3)\) entrega:
  \begin{center}
    \begin{tabular}[ht]{>{\(}c<{\)}|*{3}{>{\(}c<{\)}}|>{\(}c<{\)}}
            & s_3  & x_2 & s_2 & \\
      \hline
        s_1 & -1/3 & 11/3 &   1 & 23/3 \\
        x_3 &	0  &  1	  &   1 &  4   \\
        x_1 &  1/3 &  1/3 &   0 &  7/3 \\
      \hline
            &  5/3 &  2/3 &   1 & 47/3
    \end{tabular}
  \end{center}
  Como no hay coeficientes negativos en la última fila,
  es óptimo.
  Leemos \(x_1 = 7/3\), \(x_2 = 0\) y \(x_3 = 4\);
  la función objetivo es \(47/3\).

  Si recordamos el problema dual,
  su solución se obtiene de leer por columnas y no por filas.
  El óptimo para este es \(y_1 = 0\), \(y_2 = 1\), \(y_3 = 5/3\),
  con el mismo valor de la función objetivo.

\subsection{Reglas de pivote}
\label{sec:reglas-de-pivote}

  Sistematizaremos la elección de pivotes.
  Hay varios casos a considerar.
  Supongamos que después de pivotear un rato
  tenemos el \emph{\foreignlanguage{french}{tableau}}:
  \begin{center}
    \begin{tabular}[ht]{*{2}{>{\(}c<{\)}|}>{\(}c<{\)}}
                   & \mathbf{y}	 &	      \\
      \hline
        \mathbf{x} & \mathbf{A}	 & \mathbf{b} \\
      \hline
                   & \mathbf{-c} & v
    \end{tabular}
  \end{center}
  Si \(\mathbf{b} \ge \mathbf{0}\),
  una solución factible para el problema máximo del que partimos
  es \(\mathbf{x} = \mathbf{b}\), \(\mathbf{y} = \mathbf{0}\),
  y tenemos el valor \(v\) de la función objetivo.
  Si \(-\mathbf{c} \ge \mathbf{0}\),
  una solución factible para el problema dual está dada por
  \(\mathbf{y} = - \mathbf{c}\) y \(\mathbf{x} = \mathbf{0}\),
  también con valor \(v\).
  Sabemos que si \(\mathbf{b} \ge \mathbf{0}\)
  y \(-\mathbf{c} \ge \mathbf{0}\)
  tenemos los óptimos.

  Consideraremos primero el caso en que ya tenemos un punto factible.
  \begin{description}
  \item[Caso 1:]
    Si \(\mathbf{b} \ge \mathbf{0}\),
    tome cualquier columna \(s\)
    que contenga un elemento negativo en la última fila,
    \(-c_s < 0\).
    En esa columna elija \(r\)
    tal que \(a_{r s} > 0\) y \(b_r / a_{r s}\) sea mínimo,
    eligiendo cualquiera en caso de empate.
    Pivotee alrededor de \(x_r, y_s\).

    Si esta regla no es aplicable,
    puede ser porque \(-\mathbf{c} \ge \mathbf{0}\),
    en cuyo caso tenemos el óptimo;
    o en la columna \(s\) todos los \(a_{i s} \le 0\).
    En este último caso,
    el problema de maximización es factible no acotado.
    Para verlo,
    considere un vector \(\mathbf{x} \ge \mathbf{0}\)
    con \(x_s > 0\) y \(x_j = 0\) para \(j \ne s\).
    Entonces \(\mathbf{x}\) es factible para el problema de maximización
    ya que para todo \(i\):
    \begin{equation*}
      y_i
        =   \sum_j (-a_{i j} x_j + b_i)
        =   -a_{i s} x_s + b_i
        \ge 0
    \end{equation*}
    Este vector da el valor \(\sum c_j x_j = c_s x_s\),
    que podemos aumentar a gusto a través de \(r_s\).

    Esta regla de pivote es conveniente
    por las siguientes.
    \begin{proposition}
      \label{prop:pivot-1-1}
      Si \(\mathbf{b} \ge \mathbf{0}\) antes de aplicar la regla,
      entonces se cumple después de pivotear.
    \end{proposition}
    \begin{proof}
      Marcamos los valores luego del pivote por circunflejos.
      Debemos demostrar \(\widehat{b}_i \ge 0\) para todo \(i\).
      Para \(i = s\)
      es \(\widehat{b}_s = b_s / a_{r s}\),
      no negativo ya que \(b_s \ge 0\) y \(a_{r s} > 0\).
      Para \(i \ne s\) resulta:
      \begin{equation*}
        \widehat{b}_i
          = b_i - \frac{a_{i s}}{b_s}{a_{r s}}
      \end{equation*}
      Si \(a_{i s} < 0\),
      estamos restando un valor negativo,
      \(\widehat{b}_i > b_i\);
      si \(a_{i s} > 0\) por la regla para elegir el pivote
      \(a_{i s} \le a_{r s}\),
      estamos restando una fracción de \(b_i\) que se mantiene no negativo.
    \end{proof}
    \begin{proposition}
      El valor del nuevo \emph{\foreignlanguage{french}{tableau}}
      nunca es menor que el original.
    \end{proposition}
    \begin{proof}
      Tenemos que \(-c_s < 0\), \(a_{r s} > 0\) y \(b_r \ge 0\),
      por lo que:
      \begin{equation*}
        \widehat{v}
          =   v - (-c_s) \frac{b_r}{a_{r s}}
          \ge v
      \end{equation*}
    \end{proof}
    Con estas dos propiedades,
    como nos movemos de un vértice a uno vecino,
    si en cada paso aumenta el valor
    hallaremos el óptimo en un número finito de pasos
    (el politopo tiene un número finito de vértices)
    o concluiremos que es factible no acotado.
  \item[Caso 2:]
    Si hay \(b_i\) negativos,
    elija el primero de ellos,
    llamémosle \(b_k\)
    (con esta elección tenemos \(b_1 \ge 0, b_2 \ge 0, \dotsc, b_{k - 1} \ge 0\)).
    Elija alguna entrada negativa en la fila \(k\),
    digamos \(a_{k s} < 0\).
    Esta será la columna pivote.
    Compare \(b_k / a_{k s}\) con los \(b_i / a_{i s}\)
    con \(b_i > 0\) y \(a_{i s} > 0\),
    sea \(r\) el índice que da la razón más pequeña
    (podría ser \(r = k\)),
    eligiendo cualquiera en caso de empate.
    Pivotee alrededor de \((r, s)\).

    Si esta regla no es aplicable,
    el problema máximo no tiene soluciones factibles.
    La fila \(k\) dice:
    \begin{equation*}
      - y_k
        = \sum_j a_{k j} x_j - b_k
    \end{equation*}
    Para vectores factibles
    \(\mathbf{x} \ge \mathbf{0}\),
    \(\mathbf{y} \ge \mathbf{0}\)
    el lado izquierdo es negativo o cero,
    el lado derecho es positivo.

    Lo que buscamos es aumentar ese \(b_k\),
    de manera de llegar al caso~1 que sabemos manejar.
    Esto por las siguientes.
    \begin{proposition}
      \label{prop:caso-2-1}
      Los \(b_i\) no negativos se mantienen no negativos al pivotear.
    \end{proposition}
    \begin{proof}
      Suponga que \(b_i \ge 0\),
      o sea \(i \ne k\).
      Si \(i = r\),
      entonces \(\widehat{b}_r = b_r / a_{r s} \ge 0\).
      Si \(i \ne r\),
      entonces:
      \begin{equation*}
        \widehat{b}_i
          = b_i - \frac{a_{i s}}{a_{r s}} b_r
      \end{equation*}
      Pero \(b_r / a_{r s} \ge 0\).
      Si \(a_{i s} < 0\)
      entonces \(\widehat{b}_i \ge b_i \ge 0\);
      mientras que si \(a_{i s} > 0\)
      entonces \(b_s / a_{r s} \le b_i / a_{i s}\),
      con lo que \(\widehat{b}_i \ge b_i - b_i = 0\).
    \end{proof}
    \begin{proposition}
      Al aplicar la regla,
      \(b_k\) negativo no disminuye.
    \end{proposition}
    \begin{proof}
      Si \(k = r\),
      entonces \(\widehat{b}_k = b_k / a_{r s} > 0\)
      (ambos son negativos).
      Si \(k \ne r\),
      entonces:
      \begin{equation*}
        \widehat{b}_k
          =   b_k - \frac{a_{k s}}{a_{r s}} b_r
          \ge b_k
      \end{equation*}
    \end{proof}
    Si aplicamos la regla del caso~2,
    y en cada paso aumenta \(b_k\),
    en algún momento se hará positivo y eliminamos un \(b_k\) negativo.
  \end{description}

\subsection{Ciclos}
\label{sec:ciclos}

  Lamentablemente,
  nuestras reglas no garantizan mejoras.
  Existe la posibilidad de entrar en un ciclo,
  como ilustra el siguiente ejemplo,
  maximizar \(3 x_1 - 5 x_2 + x_3 - 2 x_4\) sujeto a \(x_i \ge 0\) y:
  \begin{equation}
    \label{eq:simplex-cycle}
    \sysdelim..
    \systeme{
        x_1 - 2 x_2 - x_3 + 2 x_4 \leq 0,
      2 x_1 - 3 x_2 - x_3 +   x_4 \leq 0,
                      x_3	  \leq 1
    }
  \end{equation}
  Si se pivotea en el orden \((x_1, y_1)\),
  \((x_2, y_2)\),
  \((x_3, x_1)\),
  \((x_4, x_2)\),
  \((y_1, x_3)\),
  \((y_2, x_4\))
  volvemos al \emph{\foreignlanguage{french}{tableau}} original.

  El problema de ciclos en la práctica es muy raro.
  Pero resulta que hay una regla simple y eficiente,
  debida a Bland~%
    \cite{bland77:_new_finite_pivot_rules_simplex_method},
  que evita ciclos.
  Esta regla,
  llamada \emph{mínimo subíndice},
  dice que en caso de empate de fila o columna pivote,
  se seleccione la fila (o columna)
  en la cual la variable \(x\) tenga el mínimo subíndice;
  si no hay variables \(x\) elija la \(y\) con el mismo criterio.

  Finalmente tenemos una demostración constructiva para el teorema de dualidad:
  Use el método Simplex
  para obtener la solución al problema de máximo acotado;
  automáticamente obtenemos la solución al problema dual.

\subsection{Comentarios finales}
\label{sec:comentarios-simplex}

  Esto en realidad es una familia de algoritmos.
  No hemos especificado cómo elegir la columna pivote
  si hay varios coeficientes negativos en la última fila.
  Una regla simple,
  propuesta por Dantzig,
  es usar aquella columna con el coeficiente más negativo.
  Hay varias otras opciones,
  algunas de ellas discute Eigen~%
    \cite{eigen11:_pivot_rules_simplex_method},
  una comúnmente usada y generalmente efectiva
  es la del programa Devex,
  de Harris~%
    \cite{harris73:_pivot_selec_metod_devex_lp_code}.
  Lo que buscan es la arista que comienza el camino más corto
  al óptimo,
  pero hallarla es más difícil que resolver el problema,
  se emplean diversas heurísticas.
  Tampoco hay una regla para elegir en caso de empate
  (aunque hemos dado la regla de Bland).

  Un vértice de un politopo en \(n\) dimensiones
  (si hay \(n\) variables)
  está definido por \(n - 1\) hiperplanos,
  si hay \(m\) restricciones esto está acotado por \(\binom{m}{n - 1}\),
  lo que es más que cualquier polinomio en \(m n\)
  (el número de coeficientes que definen el modelo).
  Klee y Minty~%
    \cite{klee69:_how_good_simplex_algo}
  hallaron una familia infinita de modelos en los cuales Simplex
  usando la regla de pivote de Dantzig toma tiempo exponencial.
  En la práctica,
  el método es notablemente eficiente,
  particularmente con reglas de pivote elegidas cuidadosamente.

\section*{Ejercicios}
\label{sec:ex-17a}

  \begin{enumerate}
  \item
    Dé y muestre gráficamente problemas de programación lineal
    que no tienen soluciones factibles,
    que tienen soluciones factibles acotadas
    y soluciones factibles no acotadas.
    ¿Es posible tener áreas no acotadas con soluciones acotadas?
  \item
    Considere el problema de programación lineal:
    \begin{equation*}
      \sysdelim..
      \systeme{
          x_1 + 2 x_2 - x_3 + 3 x_4 \leq 7,
        2 x_1 - 3 x_2 - x_3 +	x_4 \leq 3,
                        x_3	    \leq 1,
          x_1 - x_2   + x_3 -	x_4 =	 3
      }
    \end{equation*}
    \begin{enumerate}
    \item
      Resuélvalo usando nuestra técnica inicial
      (mencionando explícitamente las ecuaciones completas),
      sin utilizar la ecuación para reducir el problema.
    \item
      Describa las modificaciones necesarias a la técnica
      usando \emph{\foreignlanguage{french}{tableaux}}
      para representar esto.
    \end{enumerate}
  \item
    \label{ex:-17a-cycle}
    Considere el problema
    de maximizar \(3 x_1 - 5 x_2 + x_3 - 2 x_4\) sujeto a \(x_i \ge 0\) y:
    \begin{equation*}
      \sysdelim..
      \systeme{
          x_1 - 2 x_2 - x_3 + 2 x_4 \leq 0,
        2 x_1 - 3 x_2 - x_3 +	x_4 \leq 0,
                        x_3	    \leq 1
      }
    \end{equation*}
    \begin{enumerate}
    \item
      Escriba el \emph{\foreignlanguage{french}{tableau}} inicial
      para este problema.
    \item
      Verifique que
      pivotear en el orden \((x_1, y_1)\),
      \((x_2, y_2)\),
      \((x_3, x_1)\),
      \((x_4, x_2)\),
      \((y_1, x_3)\),
      \((y_2, x_4\))
      respeta las reglas planteadas,
      pero volvemos a un \emph{\foreignlanguage{french}{tableau}}
      equivalente al original.
    \end{enumerate}
  \item
    Repita el ejercicio~\ref{sec:ex-17a},
    pero esta vez usando la regla de Bland.
  \end{enumerate}

\bibliography{../referencias}

%%% Local Variables:
%%% mode: latex
%%% TeX-master: "../INF-221_notas.tex"
%%% ispell-local-dictionary: "spanish"
%%% End:

% LocalWords:  recubridor subsecuencia eq std max problem lp feasible
% LocalWords:  maximización Simplex equilibrium carbohidratos simplex
% LocalWords:  hiperplanos english slack subespacio pivotear french
% LocalWords:  tableau tableaux cc Pivoteando Pivotee pivotea

\bibliographystyle{babplain-fl}

\chapter{Métodos simples de ordenamiento}
\label{cha:metodos-simples-ordenamiento}

  Como una ilustración de técnicas de análisis de algoritmos más avanzadas,
  analizaremos los métodos de ordenamiento más simples.
  Tenemos los métodos
  de la burbuja
  (listado~\ref{lst:bubblesort},
   ver también la discusión histórica de Astrachan~%
     \cite{astrachan03:_bubble_sort}),
  selección,
  (listado~\ref{lst:selection}),
  e inserción
  (listado~\ref{lst:insertion}).
  Es fácil ver que todos ellos tienen tiempo de ejecución \(O(n^2)\).
  El método de selección tiene mejor caso \(O(n^2)\),
  los otros dos tienen mejor caso \(O(n)\).
  \lstinputlisting[float,
                   language = C,
                   firstline = 8,
                   caption = {Método de la burbuja},
                   label = lst:bubblesort]
                  {code/bubblesort.c}
  \lstinputlisting[float,
                   language = C,
                   firstline = 8,
                   caption = {Método de selección},
                   label = lst:selection]
                  {code/selection.c}
  \lstinputlisting[float,
                   language = C,
                   firstline = 8,
                   caption = {Método de inserción},
                   label = lst:insertion]
                  {code/insertion.c}

\section{Rendimiento de métodos simples de ordenamiento}
\label{sec:rendimiento-metodos-simples}

  Nos interesa obtener información más detallada que esta
  sobre el rendimiento de estos algoritmos.
  En particular,
  interesa el tiempo promedio de ejecución.
  Para ello debemos considerar una distribución de los datos de entrada
  (valores repetidos,
   orden original del arreglo).
  Para simplificar,
  supondremos que no hay datos repetidos,
  y como los algoritmos únicamente comparan elementos
  podemos asumir que la entrada es una permutación de \(1, \dotsc, n\).
  O sea,
  requerimos la distribución de las permutaciones dadas al algoritmo.
  Esto en general es imposible de conseguir
  (y engorroso de tratar),
  así que supondremos que todas las permutaciones son igualmente probables.
  Esto reduce el análisis detallado a derivar propiedades promedio
  de las permutaciones.
  \begin{definition}
    Una \emph{inversión} de la permutación \(\pi\) de \(1, \dotsc, n\)
    es un par de índices \(i\), \(j\) tales que \(i < j\)
    y \(\pi(i) > \pi(j)\).
  \end{definition}
  El número mínimo de inversiones es \num{0}
  (el arreglo ordenado no tiene inversiones),
  el máximo es \(n (n - 1) / 2\)
  (en el arreglo ordenado de mayor a menor
   el elemento en la posición~\(i\)
   participa en \(i - 1\) inversiones con elementos previos,
   sumando para \(1 \le i \le n\) se tiene el valor citado).

  Es claro que en el método de la burbuja cada intercambio
  elimina exactamente una inversión.
  Vale decir,
  el número de asignaciones de elementos
  es tres veces el número de inversiones.

  El método de inserción funciona esencialmente como el de la burbuja,
  solo que en vez de intercambiar en cada paso
  deja un espacio libre en la posición original,
  copia cada elemento una posición hacia arriba
  si es mayor que el elemento bajo consideración,
  moviendo la posición libre hacia abajo;
  finalmente ubica el elemento en su posición
  (la que queda libre después de los malabares anteriores).
  Compare con la figura~\ref{fig:insercion}.
  \begin{figure}[ht]
    \centering
    \begin{tikzpicture}[scale = 0.75]
        \draw[thick] (0, 0) rectangle (13, 1);
        \draw[thick] (2, 0) rectangle (3, 1);
        \draw[thick, fill = blue!10] (3, 0) rectangle (9, 1);
        \draw[thick] (8, 0) rectangle (9, 1);
        \draw[thick] (8, 3) rectangle (9, 4);
          \node at (9.1, 3.5) [right] {tmp};

        \draw[-latex'] (8.5, 1.1) -- (8.5, 2.9);
        \draw (8.6, 2) node[circle, inner sep = 0.5pt, draw, right] {\num{1}};

        \draw[-latex'] (4, 0.5) -- (7, 0.5);
          \draw (5.5, 1.15) node[circle, inner sep = 0.5pt, draw, above]
             {\num{2}};

        \draw[-latex'] (7.9, 3.5) to [in = 75, out = 180] (2.5, 1.1);
           \draw (4, 3.3) node[circle, inner sep = 0.5pt, draw] {\num{3}};
      \end{tikzpicture}
    \caption{Operación del método de inserción}
    \label{fig:insercion}
  \end{figure}
  Nuevamente,
  si suponemos que el espacio vacío
  eventualmente será ocupado por el valor temporal,
  cada asignación elimina una inversión.
  Esto se resume en dos asignaciones para cada elemento,
  y una asignación adicional para cada inversión.

\section{Funciones generatrices cumulativas}
\label{sec:func-gener-cumul}

  Comparar los métodos
  es entonces esencialmente obtener el número medio de inversiones
  en las permutaciones de \(1, \dotsc, n\).
  Para ello recurrimos a nuestra técnica preferida,
  funciones generatrices.
  De manera muy similar
  a como contabilizamos las estructuras de un tamaño dado
  mediante funciones generatrices
  podemos representar el total de alguna característica.
  Dividiendo por el número de estructuras del tamaño respectivo
  tenemos el promedio del valor de interés.
  Véase el apéndice~\ref{apx:symbolic-method-dummies}
  o el apunte de Fundamentos de Informática~%
    \cite{brand17:_fundamentos_informatica}.

\section{Análisis de burbuja e inserción}
\label{sec:analisis-burbuja-insercion}

  Nuestros objetos de interés son permutaciones,
  objetos rotulados.
  Corresponde usar funciones generatrices exponenciales.

  Anotemos \(\iota(\pi)\) para el número de inversiones
  de la permutación \(\pi\),
  y definamos la función generatriz cumulativa:
  \begin{equation}
    \label{eq:I-def}
    I(z)
      = \sum_{\pi \in \mathscr{P}}
          \iota(\pi) \frac{z^{\lvert \pi \rvert}}{\lvert \pi \rvert !}
  \end{equation}
  En particular,
  nos interesa el número promedio de inversiones
  para permutaciones de tamaño~\(n\).

  Podemos describir permutaciones mediante la expresión simbólica:
  \begin{equation}
    \label{eq:P-class}
    \mathscr{P}
      = \mathscr{E} + \mathscr{P} \star \mathscr{Z}
  \end{equation}
  Vale decir,
  una permutación es vacía
  o es una permutación combinada con un elemento adicional.
  Dada la permutación \(\pi\)
  construimos permutaciones de largo \(\lvert \pi \rvert + 1\)
  añadiendo un nuevo elemento vía la operación \(\star\).
  Si elegimos \(j\) como nuevo último elemento,
  habrán \(j - 1\) elementos menores que él antes,
  o sea serán \(j - 1\) inversiones adicionales.
  Estamos creando \(\lvert \pi \rvert + 1\) nuevas permutaciones,
  cada una de las cuales conserva las inversiones que tiene,
  y agrega entre \num{0} y \(\lvert \pi \rvert\) nuevas inversiones
  dependiendo del valor elegido como último.
  El total de inversiones en el conjunto de permutaciones así creado
  a partir de \(\pi\) es:
  \begin{equation}
    \label{eq:iota-decomposed}
    (\lvert \pi \rvert + 1) \iota(\pi)
      + \sum_{0 \le k \le \lvert \pi \rvert} k
      = (\lvert \pi \rvert + 1) \iota(\pi)
          +  \frac{\lvert \pi \rvert ( \lvert \pi \rvert + 1)}{2}
  \end{equation}
  Con esto tenemos la descomposición para la función generatriz cumulativa
  (la permutación de cero elementos no tiene inversiones):
  \begin{align}
    I(z)
      &= \iota(\varepsilon)
           + \sum_{\pi \in \mathscr{P}}
               \left(
                 (\lvert \pi \rvert + 1) \iota(\pi)
                     +	\frac{\lvert \pi \rvert ( \lvert \pi \rvert + 1)}{2}
               \right)
               \frac{z^{\lvert \pi \rvert + 1}}{(\lvert \pi \rvert + 1)!}
                   \label{eq:Inv-decomposed} \\
      &= \sum_{\pi \in \mathscr{P}}
           \iota(\pi) \frac{z^{\lvert \pi \rvert + 1}}{\lvert \pi \rvert !}
           + \frac{1}{2}
               \sum_{\pi \in \mathscr{P}}
                 \frac{z^{\lvert \pi \rvert + 1}}{\lvert \pi \rvert !}
                 \lvert \pi \rvert \notag \\
   \intertext{Como hay \(k!\) permutaciones de tamaño \(k\),
              sumando sobre tamaños resulta:}
      &= z I(z) + \frac{1}{2} z \sum_{k \ge 0} k z^k \notag \\
      &= z I(z) + \frac{z^2}{2 (1 - z)^2}
  \end{align}
  Despejando:
  \begin{equation}
    \label{eq:Inv-explicit}
    I(z)
      = \frac{1}{2} \frac{z^2}{(1 - z)^3}
  \end{equation}
  Obtenemos el número promedio de inversiones directamente,
  ya que hay \(n!\) permutaciones de tamaño \(n\),
  y el promedio casualmente es el coeficiente de \(z^n\)
  en la función generatriz exponencial:
  \begin{align}
    \Exp_n[\iota]
      &= [z^n] I(z)
           \label{En-iota} \\
      &= \frac{1}{2} [z^n] \frac{z^2}{(1 - z)^3} \notag \\
      &= \frac{1}{2} [z^{n - 2}] (1 - z)^{-3} \notag \\
      &= \frac{1}{2} \binom{-3}{n - 2} \notag \\
      &= \frac{1}{2} \binom{n - 2 + (3 - 1)}{3 - 1} \notag \\
      &= \frac{1}{2} \binom{n}{2}
           \notag \\
      &= \frac{n (n - 1)}{4}
          \label{En-iota-explicit}
  \end{align}
  Podemos resumir las anteriores
  como número asintótico de asignaciones al ordenar \(n\) elementos
  en el cuadro~\ref{tab:bubble-insertion}.
  \begin{table}[ht]
    \centering
    \begin{tabular}{l*{3}{|>{\(}c<{\)}}}
      \multicolumn{1}{c|}{\textbf{Algoritmo}} &
        \multicolumn{1}{c|}{\textbf{Min}} &
        \multicolumn{1}{c|}{\textbf{Prom}} &
        \multicolumn{1}{c}{\textbf{Máx}} \\
       \hline
       Burbuja	 & 0   & 3 n^2 / 4 & 3 n^2 / 2 \\
       Inserción & 2 n & n^2 / 4   &   n^2 / 2
      \end{tabular}
    \caption{Comparación entre métodos de burbuja e inserción}
    \label{tab:bubble-insertion}
  \end{table}
  Queda claro
  (salvo para optimistas incurables)
  que el método de inserción es mejor.

\section{Análisis de selección}
\label{sec:analisis-seleccion}

  El método de selección tiene sentido
  si queremos minimizar el número de copias
  (podemos simplemente copiar y comparar claves,
   y copiar elementos solo para ubicarlos en su lugar).
  En tal caso interesa fundamentalmente el número de asignaciones.

  Para el método de selección el número de asignaciones
  está dado por el número de veces que hallamos un elemento mayor,
  ver el listado~\ref{lst:maximo}.
  \lstinputlisting[float,
                   language=C,
                   firstline = 6,
                   xleftmargin=3em, numbers=left,
                   caption={Hallar el máximo},
                   label=lst:maximo]
                   {code/maximum.c}
  o sea,
  el número de máximos de izquierda a derecha en la permutación.
  Todas las operaciones se efectúan \(n\)~veces,
  salvo las actualizaciones a la variable~\lstinline[language = C]!max!.
  Es evidente que el número de veces que se actualiza
  \lstinline[language = C]!max! es \(O(n)\),
  pero interesa una respuesta más precisa.
  Para obtener este valor,
  podemos optar por funciones generatrices cumulativas
  o bivariadas.
  Exploraremos ambas opciones como ejemplos en lo que sigue.

  Si suponemos que todos los valores son diferentes,
  y que todas las maneras de ordenarlos son igualmente probables,
  estamos buscando el número promedio de máximos de izquierda a derecha
  de permutaciones.
  Podemos describir la clase de permutaciones simbólicamente
  como:
  \begin{equation}
    \label{eq:P-class-again}
    \mathscr{P}
      = \mathscr{E} + \mathscr{P} \star \mathscr{Z}
  \end{equation}
  Llamaremos \(\chi(\pi)\) al número de máximos de izquierda a derecha
  en la permutación \(\pi\).

\subsection{Función generatriz cumulativa}
\label{sec:max-fg-cumulativa}

  Definimos la función generatriz cumulativa exponencial:
  \begin{equation}
    \label{eq:max-fg-cum}
    \widehat{C}(z)
      = \sum_{\pi \in \mathscr{P}}
          \chi(\pi) \frac{z^{\lvert \pi \rvert}}{\lvert \pi \rvert !}
  \end{equation}
  Como el último elemento de la permutación
  es un máximo de izquierda a derecha
  si es el máximo de todos ellos
  (y los demás máximos de izquierda a derecha se mantienen al rerotular),
  usando la convención de Iverson
  podemos expresar el número de máximos de izquierda a derecha
  en la permutación resultante de \(\pi \star (1)\)
  si se asigna el rótulo \(j\) al elemento nuevo como:
  \begin{equation}
    \label{eq:app:chi+1}
    \chi(\pi) + [j = \lvert \pi \rvert + 1]
  \end{equation}
  con lo que en total
  para el conjunto de permutaciones \(\pi \star (1)\) es:
  \begin{equation*}
    \sum_{1 \le j \le \lvert \pi \rvert + 1}
      (\chi(\pi) + [j = \lvert \pi \rvert + 1])
      = (\lvert \pi \rvert + 1 ) \chi(\pi)
          + 1
  \end{equation*}
  con lo que de~\eqref{eq:P-class-again} resulta:
  \begin{align*}
    \widehat{C}(z)
      &= \chi(\varepsilon) \frac{z^0}{0!}
           + \sum_{\pi \in \mathscr{P}}
               ((\lvert \pi \rvert + 1) \chi(\pi) + 1)
                 \frac{z^{\lvert \pi \rvert + 1}}{(\lvert \pi \rvert + 1)!} \\
      &= 0
           + z \sum_{\pi \in \mathscr{P}}
                 \chi(\pi) \frac{z^{\lvert \pi \rvert}}{\lvert \pi \rvert !}
           + \sum_{\pi \in \mathscr{P}}
               \frac{z^{\lvert \pi \rvert + 1}}{(\lvert \pi \rvert + 1)!} \\
  \intertext{Como hay \(n!\) permutaciones de largo \(n\),
             la última suma se simplifica:}
      &= z \widehat{C}(z)
           + \sum_{n \ge 0} n! \frac{z^{n + 1}}{(n + 1)!} \\
             \notag \\
      &= z \widehat{C}(z)
           + \sum_{n \ge 0} \frac{z^{n + 1}}{n + 1} \\
      &= z \widehat{C}(z) + \ln \frac{1}{1 - z}
  \end{align*}
  Despejamos:
  \begin{equation}
    \label{eq:max-fg-cum-explicit}
    \widehat{C}(z)
      = \frac{1}{1 - z} \ln \frac{1}{1 - z}
  \end{equation}
  Nos interesa el coeficiente de \(z^n\) de~\eqref{eq:max-fg-cum-explicit},
  que casualmente es directamente el valor promedio que buscamos
  (ver por ejemplo el apunte de Fundamentos de Informática~%
     \cite{brand17:_fundamentos_informatica}
   para desarrollo de las funciones generatrices empleadas):
  \begin{equation}
    \label{eq:1}
    [z^n] \widehat{C}(z)
      = H_n
  \end{equation}
  En promedio a \lstinline[language = C]!max! se asigna un nuevo valor
  \(H_n = \ln n + \gamma + O(1/n)\) veces
  al buscar el máximo de \(n\) valores.

\subsection{Función generatriz bivariada}
\label{sec:max-fg-bivariada}

  La función generatriz de probabilidad de que una permutación de tamaño \(n\)
  tenga \(k\) máximos de izquierda a derecha es:
  \begin{equation}
    \label{eq:M-pgf}
    M(z, u)
      = \sum_{\pi \in \mathscr{P}}
          \frac{z^{\lvert \pi \rvert}}{\lvert \pi \rvert !}
            u^{\chi(\pi)}
  \end{equation}
  Esto casualmente es la función generatriz exponencial bivariada
  correspondiente a la clase~\eqref{eq:P-class-again}.

  Razonando como en la sección~\ref{sec:max-fg-cumulativa}
  obtenemos:
  \begin{align}
    M(z, u)
      &= \frac{z^{\lvert \varepsilon \rvert}}{\lvert \varepsilon \rvert !}
           u^{\chi(\varepsilon)}
           + \sum_{\pi \in \mathscr{P}}
               \sum_{1 \le j \le \lvert \pi \rvert + 1}
                 \frac{z^{\lvert \pi \rvert + 1}}
                      {(\lvert \pi \rvert + 1)!}
                   u^{\chi(\pi) + [j = \lvert \pi \rvert + 1]}
                      \label{eq:M-decomposed} \\
      &= \frac{z^0}{0!} u^0
           + \sum_{\pi \in \mathscr{P}}
               \frac{z^{\lvert \pi \rvert + 1}}{(\lvert \pi \rvert + 1)!}
                  u^{\chi(\pi)}
               \sum_{1 \le j \le \lvert \pi \rvert + 1}
                 u^{[j = \lvert \pi \rvert + 1]}
                      \notag  \\
      &= 1
           + \sum_{\pi \in \mathscr{P}}
               \frac{z^{\lvert \pi \rvert + 1}}{(\lvert \pi \rvert + 1)!}
                  u^{\chi(\pi)} (\lvert \pi \rvert + u)
                      \label{eq:M-decomposed-result}
  \end{align}
  Derivando respecto de \(z\)
  (indicamos derivadas por subíndices para simplificar notación):
  \begin{align*}
    M_z(z, u)
      &= \sum_{\pi \in \mathscr{P}}
           \frac{z^{\lvert \pi \rvert}}{\lvert \pi \rvert !}
           u^{\chi(\pi)}
           (\lvert \pi \rvert + u) \\
      &= z M_z(z, u) + u M(z, u)
  \end{align*}
  Vale decir:
  \begin{equation}
    \label{eq:M-pde}
    (1 - z) M_z(z, u) - u M(z, u)
      = 0
  \end{equation}
  En~\eqref{eq:M-pde} la variable~\(u\) interviene como parámetro,
  esta es una ecuación diferencial ordinaria.
  Como \(M(0, u) = 1\),
  la solución es:
  \begin{equation*}
    \label{eq:M-solution}
    M(z, u)
      = \left( \frac{1}{1 - z} \right)^u
  \end{equation*}
  Las derivadas de interés son:
  \begin{align*}
    M_u(z, u)
      &= \frac{1}{(1 - z)^u} \ln \frac{1}{1 - z} \\
    [z^n] M_u(z, 1)
      &= H_n \\
    M_{u u}(z, u)
      &= \frac{1}{(1 - z)^u} \ln^2 \frac{1}{1 - z} \\
    [z^n] M_{u u}(z, 1)
      &= \sum_{1 \le k \le n} H_k \notag \\
      &= (n + 1) H_n - n
  \end{align*}
  Obtenemos:
  \begin{align}
    \Exp[ \chi_n ]
      &= [z^n] M_u (z, 1) \\
      &= H_n \\
    \var[ \chi_n ]
      &= M_{u u} (z, 1) - \left( M_u (z, 1) \right)^2 \notag \\
      &= ((n + 1) H_n - n) + H_n - H_n^2 \notag \\
      &= (n + 2) H_n - n - H_n^2
  \end{align}

\subsection{Ordenamiento por selección}
\label{sec:ordenamiento-seleccion}

  Volvamos a nuestro algoritmo de ordenamiento.
  Llamemos \(X_i\) al número de asignaciones a \lstinline[language = C]!max!
  en las distintas rondas del algoritmo
  (ciclo externo sobre \lstinline[language = C]!i!).
  Nos interesan:
  \begin{align}
    \Exp\left[ \sum_{1 \le i \le n} X_i \right]
      &= \sum_{1 \le i \le n} \Exp[ X_i ] \notag \\
      &= \sum_{1 \le i \le n} H_i \notag \\
      &= (n + 1) H_n - n
  \end{align}
  Vemos que los \(X_i\) son variables independientes,
  así que:
  \begin{align*}
    \var\left[ \sum_{1 \le i \le n} X_i \right]
      &= \sum_{1 \le i \le n} \var[ X_i ] \\
      &= \sum_{1 \le i \le n} ((i + 2) H_i - i - H_i^2) \\
      &= \sum_{1 \le i \le n} i H_i
           + 2 \sum_{1 \le i \le n} H_i
           - \frac{n (n + 1)}{2}
           - \sum_{1 \le i \le n} H_i^2
      &= \sum_{1 \le i \le n} i H_i
           + 2 ((n + 1) H_n - n)
           - \frac{n (n + 1)}{2}
           - \sum_{1 \le i \le n} H_i^2
  \end{align*}
  Atacamos las distintas sumas:
  \begin{align*}
    \sum_{1 \le i \le n} i H_i
      &= [z^n]
           \frac{z}{1 - z} \mathrm{D} \frac{1}{1 - z} \ln \frac{1}{1 - z} \\
      &= [z^n] \left(
                 \frac{z}{(1 - z)^3} \ln \frac{1}{1 - z}
                   + \frac{z}{(1 - z)^3}
               \right) \\
      &= [z^{n - 1}] \frac{1}{(1 - z)^3} \ln \frac{1}{1 - z}
          + \binom{-3}{n - 1} \\
      &= \binom{n - 1 + 2}{2} (H_{n - 1 + 2} - H_2)
           + \binom{n - 1 + 3 - 1}{3 - 1} \\
      &= \frac{n (n + 1)}{2} \left( H_{n + 1} - \frac{3}{2} \right)
           + \frac{(n + 1) (n + 2)}{2} \\
      &= \frac{n (n + 1)}{2} \left( H_{n + 1} - \frac{1}{2} \right) \\
      &= \frac{n (n + 1)}{2} H_n - \frac{n (n - 1)}{4}
  \end{align*}
  La otra suma se puede resolver sumando por partes:
    % MSE 933656
  \begin{align*}
    \sum_{1 \le i \le n} H_i^2
      &= ((n + 1) H_n - n) H_n
           - \sum_{1 \le i \le n - 1} \frac{(i + 1) H_i - i}{i + 1} \\
      &= (n + 1) H_n^2 - n H_n
          - (n + 1) H_n + n + H_n + (n - 1) - H_n + 1 \\
      &= (n + 1) H_n^2 - (2 n + 1) H_n + 2 n
  \end{align*}
  Uniendo las piezas y simplificando:
  \begin{equation}
    \var\left[ \sum_{1 \le i \le n} X_i \right]
      = \frac{n^2 + 9 n + 6}{2} H_n
          - (n + 1) H_n^2
          - \frac{n (3 n + 17)}{4}
  \end{equation}
  O sea,
  asintóticamente el número de asignaciones
  (recuerde que un intercambio son tres asignaciones)
  en el método de selección
  es mínimo \num{0}, máximo \(3 n\), promedio \(3 n \ln n\),
  varianza \(9 n^2 \ln n / 2\).

% To do: heapsort

\bibliography{../referencias}

%%% Local Variables:
%%% mode: latex
%%% TeX-master: "../INF-221_notas"
%%% ispell-local-dictionary: "spanish"
%%% End:

% LocalWords:  cumulativas cumulativa Min Prom Máx bivariadas eq max
% LocalWords:  rerotular class again fg cum explicit bivariada pde
% LocalWords:  asintóticamente

\bibliographystyle{babplain-fl}

% To do:
% - Check programs

\chapter{Quicksort}
\label{cha:quicksort}

  Quicksort,
  debido a Hoare~\cite{hoare62:_quicksort},
  es otro algoritmo basado en dividir y conquistar,
  pero en este caso la división no es fija.
  Referencia obligatoria es el detallado análisis de Sedgewick~%
    \cite{sedgewick77:_analysis_quicksort}.
  Dado un rango de elementos de un arreglo a ser ordenado,
  se elige un elemento \emph{pivote} de entre ellos
  y se reorganizan los elementos en el rango
  de forma que todos los elementos menores que el pivote
  queden antes de los mayores que éste.
  \begin{figure}[htbp]
    \centering
    \begin{tikzpicture}
      \draw (0, 0) rectangle (6, 0.5) node [pos = 0.5] {\({} \le p\)};
      \draw (6, 0) rectangle (10, 0.5) node [pos = 0.5] {\({} \ge p\)};
    \end{tikzpicture}
    \caption{Idea de Quicksort}
    \label{fig:qsort:idea}
  \end{figure}
  Con esto bastará ordenar recursivamente cada uno
  de los dos nuevos rangos generados
  para completar el trabajo.
  \begin{figure}[htbp]
    \centering
    \begin{tikzpicture}
      \draw ( 0,  0) rectangle ( 3,   0.5) node [pos = 0.5] {\({} < p\)};
      \draw ( 3,   0) rectangle ( 7,   0.5) node [pos = 0.5] {\({} \ge p\)};
      \draw ( 7,   0) rectangle ( 9.5, 0.5);
      \draw [fill = lightgray, lightgray] (7.1, 0.1) rectangle (9.4, 0.4);
      \draw ( 9.5, 0) rectangle (10,   0.5) node [pos = 0.5] {\(p\)};
      \node at (3.1, 0) [below] {\(j\)};
      \node at (6.9, 0) [below] {\(i\)};
    \end{tikzpicture}
    \caption{Particionamiento de Lomuto}
    \label{fig:qsort:Lomuto}
  \end{figure}
  La figura~\ref{fig:qsort:Lomuto}
  esquematiza una manera popular de explicar esta \emph{partición}:
  se elige un pivote de forma aleatoria
  y el pivote elegido se intercambia con el último elemento del rango
  (para sacarlo de en medio).
  En un ciclo se revisa el rango,
  elementos menores que el pivote
  se intercambian con el primero que se sabe mayor.
  \lstinputlisting[float,
                   language=C,
                   firstline = 23, lastline = 37,
                   caption={Versión simple de Quicksort
                            (particionamiento de Lomuto)},
                   label=lst:quicksort-Lomuto]
                   {code/quicksort-0.c}
% LaTeX bug: Leaving out the "float, ..." line gives a crash
  El ciclo de particionamiento mantiene el invariante
  que los elementos en el rango \(m\) a~\(j\) (exclusive)
  son menores que el pivote,
  los entre \(j\) (inclusive) a \(i\) son mayores o iguales al pivote,
  y el resto aún sin clasificar.
  Este algoritmo se atribuye a Lomuto,
  fue popularizado por Bentley~%
    \cite{bentley99:_programming_pearls}.
  El listado~\ref{lst:quicksort-Lomuto}
  muestra una versión simple del programa,
  que elige siempre el último elemento del rango como pivote.

  El esquema original de Hoare es mucho más eficiente,
  pero también notoriamente sutil
  (Bentley~%
    \cite{bentley07:_most_beaut_code_i_never_wrote}
   reconoce que nunca lo entendió del todo,
   a pesar de usarlo por años).
  Elige un pivote en el rango
  (en nuestra versión simple elegimos el elemento del medio),
  luego busca un elemento mayor al pivote desde la izquierda
  y uno menor desde la derecha.
  Si los índices no se han cruzado,
  están en el tramo equivocado y se intercambian.
  No es obvio que con el código indicado
  los índices nunca salen del rango dado.
  Tampoco conocemos la posición final del pivote,
  solo sabemos que los elementos hasta el índice \((j\) (inclusive)
  son menores o iguales que éste.
  El listado~\ref{lst:quicksort-Hoare}
  da detalles.
  \lstinputlisting[float,
                   language=C,
                   firstline = 23, lastline = 38,
                   caption={Versión simple de Quicksort
                            (particionamiento de Hoare)},
                   label=lst:quicksort-Hoare]
                   {code/quicksort-1.c}

\section{Análisis del promedio}
\label{sec:promedio-qsort}

  Evaluaremos el tiempo promedio de ejecución del algoritmo.
  Supondremos~\(n\) elementos todos diferentes,
  que las \(n!\)~permutaciones de los \(n\)~elementos
  son igualmente probables,
  y que el pivote se elige al azar en cada etapa.
  En este caso está claro que el método de particionamiento planteado
  no altera el orden de los elementos en las particiones
  respecto del orden que tenían originalmente.
  Luego,
  los elementos en cada partición también son una permutación al azar.

  Para efectos del análisis del algoritmo
  tomaremos como medida de costo el número promedio de comparaciones
  que efectúa Quicksort al ordenar un arreglo de \(n\)~elementos.
  El trabajo adicional que se hace en cada partición
  será aproximadamente proporcional a esto,
  por lo que esta es una buena vara de medida.
  Al particionar,
  cada uno de los \(n - 1\) elementos fuera del pivote
  se comparan con este exactamente una vez en el método planteado,
  y además es obvio que este es el mínimo número de comparaciones necesario
  para hacer este trabajo.
  Si llamamos \(k\) a la posición final del pivote,
  el costo de las llamadas recursivas que completan el ordenamiento
  será \(C(k - 1) + C(n - k)\).
  Si elegimos el pivote al azar
  la probabilidad de que \(k\) tenga un valor cualquiera entre~\num{1}
  y~\(n\) es la misma.
  Cuando el rango es vacío no se efectúan comparaciones.
  Estas consideraciones llevan a la recurrencia:
  \begin{equation*}
    C(n)
       =  n - 1 +
           \frac{1}{n} \, \sum_{1 \le k \le n}
              \left(C(k - 1) + C(n - k)\right) \quad C(0)  = 0
  \end{equation*}
  Por simetría podemos simplificar la suma,
  dado que estamos sumando los mismos términos
  en orden creciente y decreciente.
  Extendiendo el rango de la suma y multiplicando por \(n\) queda:
  \begin{equation*}
    n C(n)
      = n (n - 1) + 2 \sum_{0 \le k \le n - 1} C(k)
  \end{equation*}
  Ajustando los índices:
  \begin{equation*}
    (n + 1) C(n + 1)
      = n (n + 1) + 2 \sum_{0 \le k \le n} C(k) \quad C(0) = 0
  \end{equation*}

  Definimos la función generatriz ordinaria:
  \begin{equation*}
    c(z)
      = \sum_{n \ge 0} C(n) z^n
  \end{equation*}
  Aplicando las propiedades de funciones generatrices ordinarias
  a la recurrencia
  queda la ecuación diferencial:
  \begin{align*}
    \left( z \mathrm{D} + 1 \right) \, \frac{c(z)}{z}
      &= \left( (z \mathrm{D})^2 + z D \right) \, \frac{1}{1 - z}
           + \frac{2 c(z)}{1 - z}
           \qquad c(0) = 0 \\
    c'(z)
      &= \frac{2 c(z)}{1 - z} + \frac{2 z}{(1 - z)^3}
  \end{align*}
  La solución a esta ecuación es:
  \begin{equation*}
    c(z)
      = - 2 \, \frac{\ln (1 - z)}{(1 - z)^2}
            - \frac{2 z}{(1 - z)^2}
  \end{equation*}
  El primer término corresponde
  a la suma parcial de números harmónicos
  (se deriva su función generatriz
   en~\cite[capítulo 19]{brand17:_fundamentos_informatica}),
  el segundo término da un coeficiente binomial:
  \begin{align*}
    C(n)
      &= 2 \sum_{0 \le k \le n} H_k - 2 \, \binom{n}{1} \\
      &= 2 \sum_{0 \le k \le n} H_k - 2 n
  \end{align*}
  Para el primer término,
  consideremos cuántas veces aparece \(1/k\) en la suma:
  \begin{align}
    \sum_{0 \le k \le n} H_k
      &= \sum_{0 \le k \le n} \sum_{1 \le r \le k} \frac{1}{r} \notag \\
      &= \sum_{1 \le r \le n} \frac{n - r + 1}{r} \notag \\
      &= (n + 1) \sum_{1 \le r \le n} \frac{1}{r} - n
            \label{eq:Hn-sum}
  \end{align}
  Otra opción
  es sumar por partes
  (ver el apunte de Fundamentos de Informática,
   ecuación~(1.34)).
  Esto da finalmente:
  \begin{equation}
    \label{eq:quicksort-comparisons}
    C(n)
      = 2 (n + 1) H_n - 4 n
  \end{equation}
  Sabemos que
  (ver~\cite[capítulo 18]{brand17:_fundamentos_informatica})
  que \(H_n = \ln n + \gamma + O(1 / n)\),
  donde \(\gamma = 0,57721\,56649\,01532\,65120\)
  con lo que \(C(n) = 2 n \ln n + O(n)\).

\section{Análisis del peor y mejor caso}
\label{sec:peor-mejor-qsort}

  Pero podemos hacer más.
  En el peor caso,
  al particionar en cada paso elegimos uno de los elementos extremos,
  con lo que las particiones son de largo 0 y \(n - 1\),
  lo que da lugar a la recurrencia:
  \begin{equation*}
    C_{\text{peor}}(n)
      = n - 1 + C_{\text{peor}}(n - 1) \quad C_{\text{peor}}(0) = 0
  \end{equation*}
  Las técnicas estándar dan como solución:
  \begin{align*}
    C_{\text{peor}}(n)
      &= \frac{n (n - 1)}{2} \\
      &= \frac{1}{2} n^2 + O(n)
  \end{align*}
  El mejor caso es cuando en cada paso la división es equitativa,
  lo que lleva casi a la situación de dividir y conquistar
  analizada antes
  con \(a = 2\), \(b = 2\) y \(d = 1\),
  cuya solución sabemos es \(C_{\text{mejor}}(n) = O(n \log n)\).
  Un análisis más detallado,
  restringido al caso en que \(n = 2^k - 1\)
  de manera que los dos rangos siempre resulten del mismo largo,
  es como sigue.
  La recurrencia original se reduce a:
  \begin{equation*}
    C_{\text{mejor}}(n)
      = n - 1 + 2 C_{\text{mejor}}((n - 1) / 2)
      \quad C_{\text{mejor}}(0) = 0
  \end{equation*}
  Con el cambio de variables:
  \begin{equation*}
    n = 2^k - 1 \quad F(k) = C_{\text{mejor}}(2^k - 1)
  \end{equation*}
  esto se transforma en:
  \begin{equation*}
    F(k)
      = 2^k - 2 + 2 F(k - 1) \quad F(0) = 0
  \end{equation*}
  cuya solución es:
  \begin{align*}
    F(k)
      &= k 2^k + 2^{k + 1} + 2 \\
    C_{\text{mejor}}(n)
      &= (n + 1) \log_2 (n + 1) + 2 (n + 1) + 2 \\
      &= \frac{1}{\ln 2} \, n \ln n + O(n)
  \end{align*}
  La constante en este caso es aproximadamente \num{1,443},
  el mejor caso no es demasiado mejor que el promedio;
  pero el peor caso es mucho peor.

\section{Consideraciones prácticas}
\label{sec:consideraciones-practicas}

  Una variante común
  es usar un método de ordenamiento simple para rangos chicos,
  dado que Quicksort es más costoso que métodos simples para rangos pequeños.
  Una opción es cortar la recursión
  no cuando el rango se reduce a un único elemento
  sino cuando cae bajo un cierto margen;
  y luego se ordena todo mediante inserción,
  que funciona muy bien si los datos vienen \textquote{casi ordenados},
  como resulta de lo anterior.
  Para analizar esto se requieren medidas más ajustadas
  del costo de los métodos,
  y se cambian las condiciones de forma que para valores de \(n\)
  menor que el límite se usa el costo del método alternativo.
  Esto puede hacerse,
  pero es bastante engorroso y no lo veremos acá.

  Para evitar el peor caso
  (que se da cuando el pivote es uno de los elementos extremos)
  una opción es tomar una muestra de elementos y usar la mediana
  (el elemento del medio de la muestra)
  como pivote.
  La forma más simple de hacer esto es tomar tres elementos.
  Como además es frecuente que se invoque el procedimiento con un arreglo
  \textquote{casi ordenado}
  (o incluso ya ordenado),
  conviene tomar como muestra el primero,
  el último
  y un elemento del centro,
  de forma de elegir un buen pivote incluso en ese caso patológico.
  A esta idea se le conoce como \emph{mediana de tres}.
  Esta estrategia disminuye un tanto la constante
  por efecto de una división más equitativa.
  Tiene la ventaja adicional
  que tener elementos menor que el pivote al comienzo del rango
  y mayor al final
  no es necesario comparar índices para determinar
  si se llegó al borde del rango.
  El análisis detallado se encuentra por ejemplo en Sedgewick y Flajolet~%
    \cite{sedgewick13:_introd_anal_algor}.

  Por el otro lado,
  McIllroy~\cite{mcillroy99:_killer_adver_quicksort}
  muestra cómo lograr que siempre tome el máximo tiempo posible.
  Quicksort
  (haciendo honor a su nombre)
  es muy rápido
  ya que las operaciones en sus ciclos internos
  implican únicamente una comparación
  y un incremento o decremento de un índice.
  Es ampliamente usado,
  y como su peor caso es muy malo,
  vale la pena hacer un estudio detallado de la ingeniería del algoritmo,
  como hacen Bentley y McIllroy~%
    \cite{bentley93:_engin_sort_funct}.
  Debe tenerse cuidado con Quicksort por su peor caso,
  si un atacante puede determinar los datos
  puede hacer que el algoritmo consuma muchísimos recursos.
  Para evitar el peor caso se ha propuesto cambiar a Heapsort,
  debido a Williams~%
    \cite{williams64:_alg_heapsort}
  (garantizadamente \(O(n \log n)\),
   pero mucho más lento que Quicksort)
  si se detecta un caso malo,
  como propone Musser~\cite{musser97:_introsort}.

  Lo anterior supone que todos los valores son diferentes.
  Si hay muchos elementos repetidos
  (en el extremo,
   son todos iguales),
  resulta que si al particionar pasamos elementos iguales al pivote
  caemos en el peor caso
  (el pivote termina en un extremo).
  Conviene parar si hallamos un elemento igual al pivote,
  como se hace en el listado~\ref{lst:quicksort-Hoare}.

  Pero lo que realmente nos conviene es particionar en \emph{tres} tramos,
  como muestra la figura~\ref{fig:fat-partition}.
  \begin{figure}[htbp]
    \centering
    \begin{tikzpicture}
      \draw (0, 0) rectangle ( 5, 0.5) node [pos = 0.5] {\({} \le p\)};
      \draw (5, 0) rectangle ( 8, 0.5) node [pos = 0.5] {\({} = p\)};
      \draw (8, 0) rectangle (10, 0.5) node [pos = 0.5] {\({} \ge p\)};
    \end{tikzpicture}
    \caption{Particionamiento ancho}
    \label{fig:fat-partition}
  \end{figure}
  Esto es una variante del \emph{problema de la bandera holandesa}
  (\emph{\foreignlanguage{english}{Dutch national flag problem}},
   ver la figura~\ref{fig:dutch-flag})
  propuesto por Dijkstra~%
    \cite{dijkstra76:_discipline_programming}:
  dada una secuencia de canicas de colores rojo, blanco y azul,
  ordenarlas de manera de tener juntas las rojas, las blancas y las azules.
  \begin{figure}[ht]
    \centering
    \begin{tikzpicture}
      \draw (0, 0) [fill = blue] rectangle (4.5, 1);
      \draw (0, 1)		 rectangle (4.5, 2);
      \draw (0, 2) [fill = red]	 rectangle (4.5, 3);
    \end{tikzpicture}
    \caption{La bandera holandesa}
    \label{fig:dutch-flag}
  \end{figure}
  Una manera de efectuar esto
  es usar el invariante de la figura~\ref{fig:fat-partition-invariant}.
  \begin{figure}[ht]
    \centering
    \begin{tikzpicture}
      \draw (0, 0) rectangle ( 3, 0.5) node [pos = 0.5] {\({} < p\)};
      \draw (3, 0) rectangle ( 5, 0.5) node [pos = 0.5] {\({} = p\)};
      \draw (5, 0) rectangle ( 8, 0.5);
      \draw [fill = lightgray, lightgray] (5.1, 0.1) rectangle (7.9, 0.4);
      \draw (8, 0) rectangle (10, 0.5) node [pos = 0.5] {\({} > p\)};
      \node at (3.1, 0) [below]
         {\(\begin{array}{c} \uparrow \\ \text{lt} \end{array}\)};
      \node at (5.1, 0) [below]
         {\(\begin{array}{c} \uparrow \\ \text{i} \end{array}\)};
      \node at (7.9, 0) [below]
         {\(\begin{array}{c} \uparrow \\ \text{gt} \end{array}\)};
    \end{tikzpicture}
    \caption{Invariante para particionamiento ancho}
    \label{fig:fat-partition-invariant}
  \end{figure}
  Código es como indica el listado~\ref{lst:particion-2}.
  \lstinputlisting[float = ht,
                   language=C,
                   caption={Partición ancha},
                   label=lst:particion-2]
                   {code/partition-2.c}

  Durante mucho tiempo se consideró poco práctico
  Quicksort con más de un pivote,
  hasta que en 2009 se adoptó una variante con dos pivotes en Java~7,
  el algoritmo de Yaroslavskiy, Bentley y Bloch~%
    \cite{yaroslavskiy09:_dual_pivot_quick_algor}.
  Desde entonces se han analizando opciones con más de un pivote,
  una selección de referencias recientes es~%
    \cite{aumueller16:_how_good_multi_pivot_quicksort,
          kushagra14:_multi_pivot_quicksort}.
  Resulta que su mejor rendimiento se debe a efectos de memoria caché.
% To do: Show model, solution...

% To do:
% - Program(s) integrating all proposals

\bibliography{../referencias}

%%% Local Variables:
%%% mode: latex
%%% TeX-master: "../INF-221_notas"
%%% ispell-local-dictionary: "spanish"
%%% End:

% LocalWords:  Quicksort Particionamiento particionamiento Heapsort
% LocalWords:  particionar garantizadamente english Dutch national
% LocalWords:  flag problem

\bibliographystyle{babplain-fl}

\chapter{Análisis Amortizado}
\label{cha:analisis-amortizado}

  Discutiremos una forma útil de análisis,
  llamada \emph{análisis amortizado}.
  Al usar una estructura de datos
  (o sus algoritmos asociados)
  ciertamente interesa el costo de cada operación individual,
  pero en una perspectiva más amplia
  interesa acotar el costo total
  de la \emph{secuencia} de las operaciones efectuadas.
  Acotar el peor caso puede dar una cota exageradamente pesimista
  si los costos de las operaciones varían fuertemente,
  en particular si una operación \textquote{cara}
  solo es posible luego de una seguidilla de operaciones \textquote{baratas}.
  Otro caso es el que vimos recién,
  pagamos un costo ahora en la esperanza de ahorrar más adelante.
  La definición es simple:
  \begin{definition}
    \label{def:analisis-amortizado}
    El \emph{costo amortizado por operación}
    en una secuencia de \(n\) operaciones es el costo total de la secuencia
    dividido por \(n\).
  \end{definition}
  La definición es simple,
  la aplicación muchas veces requiere cuidado y creatividad.

  Note que esto difiere de análisis de caso promedio.
  Por ejemplo,
  para Quicksort derivamos un costo promedio de \(O(n \log n)\)
  para ordenar un arreglo de \(n\) elementos,
  pero el peor caso es \(O(n^2)\).
  Nada garantiza que en una secuencia de \(m\) ordenamientos
  el costo total esté acotado por \(O(m n \log n)\),
  perfectamente podemos caer casi siempre en el peor caso,
  dando \(O(m n^2)\).
  Acá buscamos acotar el peor caso de la \emph{secuencia} de operaciones,
  considerando las interacciones entre ellas.
  No entran en el análisis
  distribuciones de probabilidad de los datos de entrada
  (muchas veces impracticables de obtener,
   o al menos imposibles de tratar analíticamente,
   terminando en modelos bastante alejados de la realidad).
  Incluso la biblioteca estándar de plantillas de \cplusplus{}
  (STL,
   \emph{\foreignlanguage{english}{Standard Template Library}},
   revisar por ejemplo el clásico de Stroustrup~%
     \cite{stroustrup13:_C++_progr_lang},
   o la documentación de SGI~%
     \cite{SGI00:_STL,
           SGI00:_STL_complexity})
  ofrece una colección de estructuras de datos y algoritmos,
  especificando complejidades promedio
  y,
  donde aplicable,
  amortizadas para cada uno.

  Hay tres métodos principales
  (comparar con CLRS~%
     \cite[capítulo~17]{cormen09:_CLRS}):
  análisis agregado,
  método de contabilidad
  y funciones potenciales.
  El primero suele ser más simple;
  los dos últimos son equivalentes,
  en el sentido que los tres pueden aplicarse a los mismos problemas
  obteniendo los mismos resultados.
  La ventaja de los últimos dos métodos es que permiten análisis más detallado,
  asignando costos diferentes a operaciones distintas.
  Claro que dependiendo del problema
  uno puede resultar más natural que los otros.

\section{Arreglo dinámico}
\label{sec:arreglo-dinámico}

  Podemos representar un \emph{\foreignlanguage{english}{stack}}
  (\textquote{pila}, para los puristas del castellano)
  mediante un arreglo,
  extendiendo el arreglo si llega a llenarse.
  Las operaciones son bastante simples,
  ver el listado~\ref{lst:stack}.
  Hemos omitido el verificar
  si el \emph{\foreignlanguage{english}{stack}} contiene elementos
  o llenó el arreglo.
  \lstinputlisting[float,
                   caption = {Operaciones sobre
                              un \emph{\foreignlanguage{english}{stack}}},
                     label = lst:stack]
                  {code/stackops.c}
  La pregunta es qué hacer si el arreglo se llena.
  La biblioteca de C ofrece la opción de solicitar memoria,
  copiar el contenido de un área al comienzo de esta y liberar el original
  (ver \verb+realloc(3)+ en su Unix preferido
   o refiérase a Wikipedia~%
     \cite{wikipedia16:_c_standard_library}).
  lo que permite completar las anteriores.
  Supongamos que decidimos expandir el arreglo cuando se llene,
  la pregunta es en cuánto hacerlo para mantener rendimiento aceptable.
  Considerando el costo de un \lstinline!push! o \lstinline!pop!
  simplemente el costo de copiar un objeto,
  y similarmente que el costo de expandir el arreglo cuando tiene tamaño~\(n\)
  es \(n\)
  (en el fondo,
   el costo es solo el copiar los objetos).
  Si extendemos el arreglo en un elemento cada vez,
  para llegar a tamaño \(n\) hay que hacer \(n\) operaciones \lstinline!push!
  partiendo del \emph{\foreignlanguage{english}{stack}} vacío,
  cuando el \emph{\foreignlanguage{english}{stack}} tiene tamaño \(k\)
  el costo es \(k + 1\),
  para un costo total de \(n\) operaciones:
  \begin{equation*}
    \sum_{0 \le k \le n - 1} (k + 1)
      = \frac{n (n + 1)}{2}
  \end{equation*}
  dando un costo amortizado de \((n + 1) / 2\).
  Para operaciones tan simples esto es inaceptable.

  Si extendemos el arreglo duplicando su tamaño cada vez que se llena,
  lo que tenemos es que el costo de las duplicaciones en \(n\) operaciones
  está acotado por una suma de la forma:
  \begin{equation*}
    1 + (1 + 1) + (2 + 1) + 1 + (4 + 1) + 1 + 1 + 1
      + \dotsb
      + (2^k + 1) + 1
      + \dotsb
  \end{equation*}
  Acá estamos copiando un elemento
  cada vez que se efectúa un \lstinline!push!,
  y cada vez que pasamos de una potencia de dos
  además debemos copiar el arreglo actual.
  O sea,
  en total tenemos \(n\) copias por \lstinline!push!
  y copiamos \(2^k\) elementos cada vez que pasamos de esa potencia.
  El número total de copias es:
  \begin{align*}
    n + \sum_{0 \le k \le \lfloor \log_2 (n - 1) \rfloor} 2^k
      &=   n + 2^{\lfloor \log_2 (n - 1) \rfloor + 1} - 1 \\
      &\le n + 2^{\log_2 (n - 1) + 1} - 1 \\
      &=   n + 2 (n - 1) - 1 \\
      &=   3 n - 3 \\
      &<   3 n
  \end{align*}
  El costo amortizado es menor a \num{3}.
  Bastante más aceptable.

  Este es un ejemplo claro de \emph{análisis agregado},
  consideramos una secuencia de operaciones,
  calculamos el costo de ella y dividimos por el número de operaciones.

\section{Contador binario}
\label{sec:contador-binario}

  Imagine que debemos almacenar un contador binario grande
  en un arreglo \lstinline!a!,
  todas cuyas entradas se inician en \num{0}
  y en cada paso el contador se incrementa en \num{1}.
  Supongamos el modelo de costo
  que carga una unidad cada vez que un bit cambia.
  En una secuencia de \(n\) operaciones,
  el peor costo es \(\lfloor \log_2 n \rfloor\),
  el número máximo de bits que cambian de \num{1} a \num{0}.
  Pero el costo amortizado es menor a \num{2},
  cosa que demostraremos usando los distintos métodos.

\subsection{Método contable}
\label{sec:metodo-contable}

  La idea es mantener un saldo,
  cobramos por operaciones y ahorramos los sobrepagos,
  pagando por operaciones caras con el saldo.
  Hay que tener cuidado que el saldo nunca pueda hacerse negativo.

  \begin{proposition}
    El costo amortizado de la operación de incremento
    es a lo más \num{2} cambios de bit.
  \end{proposition}
  \begin{proof}
    Cargue \num{2} unidades a la operación de incremento.
    Si cambia \(0 \to 1\),
    gasta \num{1} en la operación y ahorra \num{1};
    si cambia \(1 \to 0\) usa lo ahorrado
    para pagar por los cambios adicionales.
    Una manera de ver que nunca terminamos con saldo negativo
    es considerar que cada bit tiene su propio saldo,
    cuando cambia de \num{0} a \num{1} se le abona \num{1},
    y ese se gasta al cambiar de \num{1} a \num{0}.
    Si tenemos una secuencia de \num{1} al final,
    e incrementamos,
    pagamos el \num{1} ahorrado en cada bit para cambiarlo a \num{0},
    pagamos \num{1} para cambiar el primer \num{0} a \num{1}
    y le dejamos \num{1} de ahorro.
    Hemos gastado \num{2}.

    Como de esta forma el saldo nunca es negativo,
    el costo de una secuencia de operaciones
    nunca sobrepasa \num{2}~cambios de bit por incremento.
  \end{proof}

\subsection{Método agregado}
\label{sec:metodo-agregado}

  Para contraste,
  una demostración usando el método agregado
  (calcular directamente el costo de la secuencia de operaciones)
  sería:
  \begin{proof}
    Consideremos una secuencia de incrementos,
    y consideremos cuántas veces cambia cada bit.
    Si el contador llega a \(n\)
    (\(n\) operaciones)
    el bit más alto es \(\lfloor \log_2 n \rfloor + 1\).
    Claramente,
    \lstinline!a[0]! cambia cada vez,
    \lstinline!a[1]! cambia cada dos incrementos,
    \ldots,
    \lstinline!a[k]! cambia cada \(2^k\) incrementos,
    y así sucesivamente.
    El costo total de los cambios a \lstinline!a[0]! es \(n\),
    los cambios a \lstinline!a[1]! tienen costo total el piso de \(n / 2\),
    \ldots,
    cambios de \lstinline!a[k]! cuestan el piso de \(n / 2^k\),
    y así sigue.

    En total,
    el costo es:
    \begin{align*}
      \lfloor n \rfloor
         + \left\lfloor \frac{n}{2} \right\rfloor
         + \dotsb
         + \left\lfloor
             \frac{n}{2^{\lfloor \log_2 n \rfloor}+ 1}
           \right\rfloor
        &=   \sum_{0 \le k \le \log_2 n + 1}
               \left\lfloor \frac{n}{2^k} \right\rfloor \\
        &\le n \sum_{0 \le k \le \log_2 n + 1} 2^{-k} \\
        &<   n \sum_{k \ge 0} 2^{-k} \\
        &=   2 n
    \end{align*}
    De aquí el costo amortizado es menor a \num{2}.
  \end{proof}

\subsection{Función potencial}
\label{sec:funcion-potencial}

  En las anteriores contabilizábamos costos por operaciones individuales.
  Una mirada alternativa es considerar que la estructura de datos
  tiene un \emph{potencial},
  una función \(\Phi \colon \mathscr{S} \to \mathbb{R}\)
  de estados \(\mathscr{S}\) de la estructura a los reales.
  Supongamos una secuencia de operaciones
  \(\sigma_1, \sigma_2, \dotsc, \sigma_n\),
  que llevan a la estructura del estado inicial \(s_0\)
  sucesivamente a \(s_1, \dotsc, s_n\).
  Sea \(c_i\) el costo real de la operación \(\sigma_i\),
  y defina el costo amortizado \(a_i\) de \(\sigma_i\) mediante:
  \begin{equation*}
    a_i
      = c_i + \Phi(s_i) - \Phi(s_{i - 1})
  \end{equation*}
  O sea:
  \begin{equation*}
    \text{(costo amortizado)}
      = \text{(costo real)}
          + \text{(cambio de potencial)}
  \end{equation*}
  Sumando sobre la secuencia de operaciones:
  \begin{align*}
    \sum_i a_i
      &= \sum_i (c_i + \Phi(s_i) - \Phi(s_{i - 1})) \\
      &= \sum_i c_i + \Phi(s_n) - \Phi(s_0)
  \end{align*}
  Reorganizando:
  \begin{equation*}
    \sum_i c_i
      = \sum_i a_i - (\Phi(s_n) - \Phi(s_0))
  \end{equation*}
  Si \(\Phi(s_n) \ge \Phi(s_0)\)
  (caso común),
  tenemos:
  \begin{equation*}
    \sum_i c_i
      \le \sum_i a_i
  \end{equation*}
  Acotando los costos amortizados \(a_i\),
  acotamos el costo de cualquier secuencia de operaciones.

  Es claro que esta visión es equivalente a las anteriores.

  Apliquemos este método a nuestro problema ahora.
  \begin{proof}
    Sea el potencial \(\Phi\) el número de bits \num{1} en el arreglo,
    y representemos el estado
    simplemente por el número representado por el contador.
    Vemos que \(\Phi(0) = 0\)
    y que \(\Phi(n) \ge 0\).
    Interesa acotar el costo amortizado de incrementos.

    Considere el \(k\)\nobreakdash-ésimo incremento,
    que cambia de \(k - 1\) a \(k\).
    Sea \(c\) el número de \emph{\foreignlanguage{english}{carries}}
    (número de reservas de dígitos)
    en este incremento,
    con lo que el costo de esta operación es \(c + 1\).
    El cambio de potencial que produce es \(- c + 1\)
    (hay \(c\) unos que cambian a ceros,
     y un cero que cambia a uno).
    El costo amortizado de la operación es:
    \begin{align*}
      a_k
        &= c + 1 + (-c + 1) \\
        &= 2
    \end{align*}
    Como el potencial final es mayor al inicial,
    tenemos que:
    \begin{align*}
      \sum_i c_i
        &< \sum_i a_i \\
        &= 2 n
    \end{align*}
  \end{proof}

  En este análisis,
  la selección de la función potencial es crítica.

\subsubsection{Diseño de una función potencial}
\label{sec:diseno-potencial}

  Esto claramente es la parte más difícil de esta técnica.
  En el caso del método contable,
  hay que definir un cobro adicional,
  que también es complejo de hacer correctamente,
  posiblemente más que lo que discutimos.
  La discusión presente es de Belleville~%
    \cite{belleville11:_amortized_analysis}.

  Si tenemos una estructura que maneja inserciones y eliminaciones,
  usualmente podemos construir un potencial adecuado
  considerando el costo adicional que significa el elemento
  una vez que se ha añadido.
  Por ejemplo,
  consideremos un \emph{\foreignlanguage{english}{stack}}
  con operaciones \verb+push+, \verb+pop+ y \verb+multipop+
  (la última elimina un número dado de objetos
   del \emph{\foreignlanguage{english}{stack}}).
  Una vez que un elemento se ha añadido
  al \emph{\foreignlanguage{english}{stack}},
  lo que queda por hacer con él es eliminarlo nuevamente,
  lo que sugiere usar como potencial
  el número de elementos actualmente presentes
  (asociar un valor \num{1} a cada elemento añadido).
  Esto sugiere:
  \begin{equation*}
    \Phi(D_i)
      = \text{número de elementos en \(D_i\)}
  \end{equation*}
  Es fácil demostrar con esto que con esto el costo amortizado por operación
  resulta constante.
  En el contador binario,
  nuestra función potencial es el número de unos,
  debemos ahorrar cada vez que un cero pasa a uno para volverlo a cero.

  En estos casos el potencial se distribuye entre los objetos de la estructura,
  pero hay casos en que no conviene tener una asociación demasiado estrecha,
  el potencial viene de la conformación de la estructura misma.

  Como ejemplo,
  consideremos una estructura simple
  que soporta operaciones de inserción y búsqueda.
  Si usamos un arreglo desordenado,
  si hay \(n\) elementos
  el costo de una inserción es constante
  y el costo de una búsqueda es \(O(n)\);
  si el arreglo se mantiene ordenado
  la búsqueda tiene costo \(O(\log n)\)
  mientras la inserción tiene costo \(O(n)\).
  Buscamos una estructura simple que de mejor rendimiento amortizado.

  Supongamos \(n\) elementos,
  y sea \(k = \lceil \log_2 (n + 1) \rceil\)
  (requerimos \(k\) bits para representar números hasta \(n\)),
  y sea \(\langle n_{k - 1} n_{k - 2} \ldots n_0 \rangle\)
  la representación de \(n\) en binario.
  Usaremos \(k\) arreglos ordenados
  \(A_0, A_1, \dotsc, A_{k - 1}\).
  El arreglo \(A_i\) o es vacío o contiene exactamente \(2^i\) elementos,
  dependiendo del valor de \(n_i\).
  Aunque cada arreglo está ordenado,
  no hay relación entre los elementos de los arreglos.

  Para buscar un objeto en esta estructura,
  buscamos en los \(k\) arreglos.
  Hay \(O(\log n)\) arreglos,
  la búsqueda en cada uno de ellos toma \(O(\log n)\),
  con lo que el costo total de la búsqueda es \(O(\log^2 n)\)
  (un análisis más fino da una cota algo mejor).

  Para insertar,
  creamos un nuevo arreglo con un único elemento,
  y vamos intercalando sucesivamente con \(A_0\),
  el resultado se intercala con \(A_1\) si no está vacío,
  y así sucesivamente
  hasta llegar a una posición vacía.
  El algoritmo~\ref{alg:insert} da el detalle.
  \begin{algorithm}[ht]
    \DontPrintSemicolon\Indp

    \Procedure{\(\mathrm{insert}(x)\)}{
      \(i \gets 0\) \;
      \(A \gets x\) \;
      \While{\(A_i\) not empty}{
        \(A \gets \mathrm{merge}(A_i, A)\) \;
        \(A_i \gets \text{empty}\) \;
        \(i \gets i + 1\) \;
      }
      \(A_i \gets A\) \;
    }
    \caption{Inserción en la estructura propuesta}
    \label{alg:insert}
  \end{algorithm}
  El peor caso de la operación \verb+insert+ es \(\Theta(n)\),
  por ejemplo si la estructura ya contiene \(n = 2^t - 1\) elementos.

  Para analizar el costo de una secuencia de \(n\) operaciones \verb+insert+
  Definimos la función potencial:
  \begin{equation*}
    \Phi(D_i)
      = \sum_{0 \le j \le k - 1} (k - j) n_j 2^j
  \end{equation*}
  donde \(k = \lceil \log_2 (n + 1) \rceil\)
  y \(n_j = [ A_j \text{not empty} ]\).
  La justificación es que los \(2^j\) elementos actualmente en el arreglo \(j\)
  potencialmente deben participar en operaciones \verb+merge+
  para migrar a los arreglos \(j + 1, j + 2, \dotsc, k\).

  Dividimos la operación en dos partes:
  crear el arreglo \(A\) con un elemento
  y las operaciones \verb+merge+
  hasta que haya un solo arreglo de tamaño \(2^j\) para cada \(j\).
  El costo de la primera parte es constante,
  y aumenta el potencial en \(k\).
  Su aporte al costo amortizado es \(k + 1\).

  Para las operaciones de intercalación,
  cada vez que se invoca sobre un par de arreglos de \(2^j\) elementos
  el costo es \(2^{j + 1}\)
  (el tiempo por elemento es constante).
  Como se mueven \(2^{j + 1}\) elementos de \(A_j\) a \(A_{j + 1}\),
  el potencial disminuye en \(2^{j + 1}\).
  El costo amortizado de cada \verb+merge+ \num{0}.

  En resumen,
  el costo amortizado de \(\mathrm{insert}\) es \(O(k + 1)\),
  que es \(O(\log n)\).

\section*{Ejercicios}
\label{sec:ejercicios-26}

\begin{enumerate}
\item
  Suponga una secuencia de operaciones numeradas \(1, 2, 3, \dotsc\)
  tal que la operación \(i\) tiene costo \num{1}
  si \(i\) no es una potencia de \num{2},
  mientras el costo es \(i\) si \(i\) es una potencia de \num{2}.
  ¿Cuál es una cota ajustada del costo amortizado de las operaciones?
\item
  Repita el análisis del arreglo extensible
  (\emph{\foreignlanguage{english}{stack}})
  usando el método potencial.
\item
  Consideremos una cola de prioridad con operaciones
  dadas en el algoritmo~\ref{alg:monotone-priority-queue}.
  Consideramos que inicia con todos los elementos de \num{1} a \(n\)
  en la cola,
  y representamos la presencia de un elemento mediante un booleano.
  \begin{algorithm}[ht]
    \DontPrintSemicolon\Indp

    \Procedure{\(\mathrm{Init}(n)\)}{
      \For{\(i \gets 1\) \KwTo \(n\)}{
        \(a[i] \gets \mathrm{true}\) \;
      }
    }
    \BlankLine
    \Procedure{\(\mathrm{Delete}(i)\)}{
      \(a[i] \gets \mathrm{false}\) \;
    }
    \BlankLine
    \Function{\(\mathrm{DeleteMin}()\)}{
      \(i \gets 1\) \;
      \While{\(\neg a[i]\)}{
        \(i \gets i + 1\) \;
      }
      \eIf{\(i \le n\)}{
        \(\mathrm{Delete}(i)\) \;
        \Return \(i\) \;
      }{
        \Return \num{0} \;
      }
    }
    \caption{Operaciones sobre la cola de prioridad}
    \label{alg:monotone-priority-queue}
  \end{algorithm}
  \begin{enumerate}
  \item
    Analice el tiempo de ejecución de cada procedimiento.
  \item
    Modifique \(\mathrm{DeleteMin}\)
    de manera que su tiempo de ejecución amortizado es \(O(1)\),
    manteniendo los órdenes de magnitud
    de los tiempos de las otras operaciones.
    Especifique la función potencial que emplea en su análisis.
  \item
    Dé una representación diferente
    con costos \(O(1)\) en el peor caso
    para \(\operatorname{Delete}\) y \(\operatorname{DeleteMin}\).
  \end{enumerate}
\item
  Un \emph{\foreignlanguage{english}{stack}}
  no se usa solo para operaciones \(\operatorname{push}\),
  puede también encogerse.
  Sea \(k\) el número actual de elementos,
  y \(L\) el largo del arreglo.
  Demuestre que si se recorta el arreglo a la mitad
  cuando se efectúa \(\mathrm{pop}\) con \(k = L / 4\),
  el costo amortizado de las operaciones es a lo más \num{3}.
  ¿Porqué no usar la cota más natural de ajustar el arreglo si \(k = L / 2\)?
\item
  Suponiendo la estructura de un contador binario como en el texto,
  describa una secuencia de \(n\) operaciones
  sobre un contador de \(k\) bits,
  inicialmente \num{0},
  de costo \(O(n k)\).

  Para manejar decrementos en forma eficiente,
  usamos \textquote{bits} que pueden tomar los valores \(-1, 0, 1\)
  (no solo \num{0, 1}).
  Almacenamos el contador en un arreglo \(a[k]\),
  y \(m\) es el último \textquote{bit} no cero
  (si todos son cero, definimos \(m = -1\)).
  El valor del contador es:
  \begin{equation*}
    \mathrm{val}(a, m)
      = \sum_{0 \le i \le m} a[i] \cdot 2^i
  \end{equation*}
  Note que \(\mathrm{val}(a, m) = 0\) si y solo si \(m = -1\).
  \begin{algorithm}[ht]
    \DontPrintSemicolon\Indp

    \Procedure{\(\mathrm{inc}\)(\(a, m\))}{
      \eIf{\(m = -1\)}{
        \(a[0] \gets 1\) \;
        \(m \gets 0\) \;
      }{
        \(i \gets 0\) \;
        \While{\(a[i] = 1\)}{
          \(a[i] \gets 0\) \;
          \(i \gets i + 1\) \;
        }
        \(a[i] \gets a[i] + 1\) \;
        \eIf{\(a[i] = 0 \wedge m = i\)}{
          \(m \gets - 1\) \;
        }{
          \(m \gets \mathrm{max}(m, i)\) \;
        }
      }
    }
    \caption{Incrementar el contador}
    \label{alg:counter-inc}
  \end{algorithm}
  \begin{algorithm}[ht]
    \DontPrintSemicolon\Indp

    \Procedure{\(\mathrm{dec}\)(\(a, m\))}{
      \eIf{\(m = -1\)}{
        \(a[0] \gets -1\) \;
        \(m \gets 0\) \;
      }{
        \(i \gets 0\) \;
        \While{\(a[i] = -1\)}{
          \(a[i] \gets 0\) \;
          \(i \gets i + 1\) \;
        }
        \(a[i] \gets a[i] - 1\) \;
        \eIf{\(a[i] = 0 \wedge m = i\)}{
          \(m \gets - 1\) \;
        }{
          \(m \gets \mathrm{max}(m, i)\) \;
        }
      }
    }
    \caption{Decrementar el contador}
    \label{alg:counter-dec}
  \end{algorithm}
  \begin{enumerate}
  \item
    Dé un ejemplo de dos representaciones diferentes de un número.
  \item
    Demuestre que los procedimientos de los algoritmos~%
    \ref{alg:counter-inc} y~\ref{alg:counter-dec}
    son correctos.
  \item
    Usando los procedimientos de los algoritmos~%
    \ref{alg:counter-inc} y~\ref{alg:counter-dec}
    para incrementar y decrementar
    (suponemos largo infinito, \(k = \infty\), para simplificar),
    demuestre que el costo amortizado de cada operación
    en una secuencia de \(n\) incrementos y decrementos
    sobre un contador inicialmente cero es \(O(1)\).
  \end{enumerate}
\end{enumerate}

\bibliography{../referencias}

%%% Local Variables:
%%% mode: latex
%%% TeX-master: "../INF-221_notas"
%%% ispell-local-dictionary: "spanish"
%%% End:

% LocalWords:  Quicksort STL english Standard Template Library SGI
% LocalWords:  CLRS stack realloc Wikipedia sobrepagos ésimo carries
% LocalWords:  push multipop not empty insert merge

\bibliographystyle{babplain-fl}

\chapter{Ejemplos de análisis amortizado}
\label{cha:ejemplos-amortizado}

  Trataremos en detalle algunos ejemplos adicionales
  de análisis amortizado para estructuras simples,
  de aplicación práctica.
  Parte de lo siguiente se adapta de Fiebrink~%
    \cite{fiebrink07:_amortized_analysis_explained}.

\section{Listas autoorganizantes}
\label{sec:list-autoorganizantes}

  Este es un ejemplo interesante
  en que el análisis compara respeto del algoritmo óptimo.
  El modelo no es demasiado realista,
  Es parte de la discusión de Tarjan~%
    \cite{sleator85:_amort_effic_list_updat_pagin_rules},
  quien resume resultados previos
  y discute varias otras situaciones afines,
  de aplicación directa.

  El modelo es de una lista,
  que se accede secuencialmente.
  En muchas aplicaciones,
  las referencias tienen localidad,
  un objeto accedido al instante \(t\)
  es más probable de ser accedido nuevamente poco después de \(t\).
  Para favorecer accesos futuros,
  el elemento buscado se mueve al principio de la lista
  (esta heurística se llama \emph{\foreignlanguage{english}{Move To Front}},
   abreviado MTF).

  Acceder al objeto en la posición \(i\) tiene costo \(i\),
  y un par de objetos vecinos pueden intercambiarse en tiempo constante.
  Con esto,
  el acceder al objeto en posición \(i\) tiene costo \(2 i - 1\)
  (\(i\) para hallarlo,
   luego \(i - 1\) intercambios para llevarlo al comienzo).

  Usamos análisis amortizado para demostrar que el rendimiento de MTF
  siempre está dentro de un factor de \num{4} de \emph{cualquier} algoritmo,
  incluso del óptimo,
  sin suposiciones sobre localidad de referencia.
  Sea \(A\) una lista ordenada,
  en que los elementos mantienen posiciones fijas,
  definimos el potencial de MTF al instante \(t\)
  como dos veces el número de objetos cuyo orden en la lista de MTF
  difiere del orden de la lista \(A\)
  (el número de inversiones).
  Por ejemplo,
  si la lista de \(A\) es \(\langle a, b, c, d, e \rangle\)
  y la de MTF es \(\langle a, d, b, c, e \rangle\),
  \(\Phi = 2 \cdot 2 = 4\)
  por \(b\) y \(c\) respecto de \(d\).
  Claramente,
  el potencial nunca es negativo,
  y a \(t = 0\) es cero,
  ambos comienzan con la misma secuencia.
  El costo amortizado es una cota superior
  al costo de una secuencia de operaciones.

  Considere el acceder al objeto \(x\),
  que está en la posición \(k\) de la lista de MTF
  y en la posición \(i\) en la de A.
  El costo en MTF es \(2 k - 1\),
  el costo en A es \(i\).
  Mover \(x\) al comienzo invierte el orden de los \(k - 1\) pares
  de elementos en posiciones \num{1} a \(k - 1\) con \(x\),
  todos los demás pares se mantienen.
  En la lista de A hay \(i - 1\) objetos antes de \(x\),
  todos ellos terminan después de \(x\) luego de promover a \(x\) en MTF.
  Se añaden a lo más \(\min \{ k - 1, i - 1 \}\) inversiones entre las listas.
  Las demás reorganizaciones
  (a lo menos \(k - 1 - \min \{ k - 1, i - 1 \}\))
  resultan en eliminar inversiones
  (las posiciones ahora concuerdan entre A y MTF).
  En consecuencia,
  el cambio de potencial en este acceso está acotado por arriba por:
  \begin{equation*}
    2 ( \min \{ k - 1, i - 1 \} - (k - 1 - \min \{ k - 1, i - 1 \}) )
      = 4 \min \{ k - 1, i - 1 \} - 2 (k - 1)
  \end{equation*}
  En consecuencia,
  el costo amortizado de esta operación es:
  \begin{align*}
    a
      &=   c + \Delta \Phi \\
      &\le 2 k - 1 + 4 \min \{ k - 1, i - 1 \} - 2 (k - 1) \\
      &\le 4 \min \{ k - 1, i - 1 \} \\
      & \le 4 i
  \end{align*}
  O sea,
  el costo de acceso amortizado de MTF está acotado por \num{4} veces el de A.

  Pero lo anterior no considera reorganizaciones que hace A.
  Sea que A intercambia dos objetos.
  Esto no introduce costo adicional para MTF,
  pero el potencial aumenta o disminuye en \num{2}
  (dos objetos cambian de posición),
  y aumenta el costo de acceso en A en \num{1}.
  La cota se mantiene,
  el costo amortizado aumenta a lo más en 2
  y la cota aumenta en \num{4}.
  Esto vale independiente del número de intercambios que hace A.

\section{Splay trees}
\label{sec:splay-trees}

  Supondremos que la terminología sobre árboles binarios de búsqueda
  junto con los algoritmos relevantes de búsqueda, inserción y eliminación
  son conocidos.
  Si no es así,
  revise sus apuntes de estructuras de datos.

  Recuerde que la \emph{profundidad} de un nodo en un árbol binario
  es su distancia a la raíz,
  y su \emph{altura} es la distancia a su hoja descendiente más lejana.
  La altura del árbol es simplemente la altura de su raíz.
  Llamaremos \emph{tamaño} de un nodo al número de nodos en su subárbol.
  El tamaño del árbol es el tamaño de su raíz.

  Un árbol de altura \(h\) tiene a lo más \(2^h\) hojas
  (un simple ejercicio de inducción),
  por lo que un árbol con \(n\) hojas
  tiene altura al menos \(\lceil \log_2 n \rceil\).
  En el peor caso,
  el tiempo para una búsqueda, inserción o eliminación
  es proporcional a la altura del árbol,
  por lo que nos interesa minimizar la altura.
  Lo mejor que podemos hacer
  es mantener \emph{árboles perfectamente balanceados},
  en los cuales cada subárbol
  (recursivamente)
  tiene lo más exactamente posible la mitad de los nodos.
  Esto asegura que la altura sea exactamente \(\lceil \log_2 n \rceil\)
  (otro ejercicio simple de inducción),
  con lo que el peor caso del tiempo de búsqueda es \(O(\log n)\).
  Sin embargo,
  si partimos con un árbol perfectamente balanceado
  una secuencia maliciosa de inserciones y eliminaciones
  puede hacerlo arbitrariamente desbalanceado,
  llevando los tiempos de búsqueda a \(\Theta(n)\).
  Para evitar esto,
  debemos modificar el árbol periódicamente para mantener el balance
  (al menos aproximadamente).
  Hay varios métodos para hacer esto,
  que llevan a árboles binarios con nombres diversos:
  árboles AVL,
  árboles rojo-negro,
  árboles balanceados en altura
  o en peso,
  y otros más.
  Plavec, Vranesic y Brown~%
    \cite{plavec07:_digital_search_trees}
  resumen evaluación experimental de estas
  y otras alternativas.
  El problema de todas ellas es que almacenan información adicional
  necesaria para mantener el balance
  (pueden reusarse bits en desuso para ello,
   pero eso lo hace poco portable),
  y operaciones engorrosas de programar.

  Una alternativa simple,
  que no requiere espacio adicional,
  relativamente simple de programar
  y con tiempo de ejecución amortizado \(O(\log n)\) para todas las operaciones
  son los \emph{\foreignlanguage{english}{splay trees}}
  propuestos por Sleator y Tarjan~%
    \cite{sleator85:_splay_trees}.
  La idea básica es promover a la raíz al nodo resultado de una búsqueda,
  reorganizando el árbol en el proceso.
  Resulta un árbol agradablemente balanceado,
  y en el caso común en que los accesos se aglomeran,
  los elementos buscados frecuentemente estarán cerca de la raíz.
  Eso sí que tienen la desventaja que incluso operaciones de \textquote{solo lectura}
  (búsquedas)
  reorganizan el árbol,
  lo que hace que esta estructura
  lamentablemente no pueda usarse en forma concurrente.

\section{Pairing Heaps}
\label{sec:pairing-heaps}

  En algunas aplicaciones de colas de prioridad
  es importante la operación de modificar la prioridad de un elemento
  (particularmente en algoritmos que trabajan sobre grafos,
   por ejemplo,
   el algoritmo de Dijkstra).
  Esta operación es provista en forma eficiente en el Fibonacci heap,
  de Fredman y Tarjan~%
    \cite{fredman87:_fibonacci_heaps}.
  Pero esta estructura es pesada y complejas de programar,
  además de lenta en la práctica.
  Fredman, Sedgewick, Sleator y Tarjan~%
    \cite{fredman86:_pairing_heap}
  proponen una versión simplificada
  que llaman \emph{\foreignlanguage{english}{pairing heaps}},
  que discutiremos acá.
  Hay varias variantes de la misma idea general,
  analizadas experimentalmente por Stasko y Vitter~%
    \cite{stasko87:_pairing_heaps},
  una estructura afín nueva
  es el \emph{\foreignlanguage{english}{rank-pairing heap}}
  de Haeupler, Sen y Tarjan~%
    \cite{haeupler11:_rank_pairing_heaps},
  quienes revisan las diversas estructuras propuestas;
  el resumen y análisis más reciente de estas y otras variantes
  es el de Iacono y Özkan~%
    \cite{iacono14:_why_some_heaps_support_const}.
  El consenso parece ser que las diferencias son menores,
  con \emph{\foreignlanguage{english}{pairing heaps}}
  la más simple y algo más eficiente en la práctica.

\subsection{Operaciones a soportar}
\label{sec:operaciones-heap}

  Una estructura \emph{\foreignlanguage{english}{heap}}
  (una ruma,
   en castellano;
   para computines es una cola de prioridad)
  es una estructura que contiene un número finito de objetos,
  cada uno con una \emph{clave}.
  Las operaciones a soportar por una cola de prioridad son:
  \begin{description}
  \item[\boldmath\(\mathrm{make\_heap}\)\unboldmath:]
    Retorna un nuevo \emph{\foreignlanguage{english}{heap}} vacío.
  \item[\boldmath\(\mathrm{empty}(H)\)\unboldmath:]
    Retorna si el \emph{\foreignlanguage{english}{heap}} \(H\) es vacío.
  \item[\boldmath\(\mathrm{insert}(H, x)\)\unboldmath:]
    Inserte el objeto \(x\)
    en el \emph{\foreignlanguage{english}{heap}} \(H\),
    que no contiene \(x\) previamente.
    La clave se supone que es parte de \(x\).
  \item[\boldmath\(\mathrm{find\_min}(H)\)\unboldmath:]
    Retorna un objeto con clave mínima en \(H\),
    sin cambiar este.
    Es un error invocar esta operación
    sobre un \emph{\foreignlanguage{english}{heap}} vacío.
  \item[\boldmath\(\mathrm{delete\_min}(H)\)\unboldmath:]
    Retorna un objeto con clave mínima en \(H\),
    eliminándolo del \emph{\foreignlanguage{english}{heap}}.
    Es un error invocar esta operación
    sobre un \emph{\foreignlanguage{english}{heap}} vacío.
  \end{description}
  Es claro que \(\mathrm{find\_min}\)
  corresponde a efectuar \(\mathrm{delete\_min}\)
  seguido por \(\mathrm{insert}\),
  pero puede ofrecerse una versión más eficiente de esta operación común.

  Operaciones menos comunes,
  pero importantes en algunas aplicaciones concretas,
  son las siguientes:
  \begin{description}
  \item[\boldmath\(\mathrm{meld}(H_1, H_2)\)\unboldmath:]
    Retorna un nuevo \emph{\foreignlanguage{english}{heap}},
    conteniendo los elementos
    de \(H_1\) y \(H_2\),
    que se destruyen.
    Es requisito que
    \(H_1\) y \(H_2\)
    no tengan objetos en común.
  \item[\boldmath\(\mathrm{decrease\_key}(H, x, \Delta)\)\unboldmath:]
    Disminuye la clave del elemento \(x\),
    miembro de \(H\),
    en \(\Delta\).
  \item[\boldmath\(\mathrm{delete}(H, x)\)\unboldmath:]
    Elimina el elemento \(x\)
    miembro de \(H\).
  \end{description}
  Sabemos que ordenar requiere \(\Omega(n \log n)\) comparaciones
  para ordenar \(n\) elementos
  si solo se permiten comparaciones entre elementos,
  con lo que el costo amortizado
  de \(n\) operaciones \(\mathrm{insert}\)
  y  \(\mathrm{delete\_min}\)
  es \(\Omega(\log n)\)
  (una manera de ordenar \(n\) elementos
   es insertarlos en un \emph{\foreignlanguage{english}{heap}}
   y luego extraerlos en orden).

\subsection{La estructura \emph{pairing heap}}
\label{sec:pairing-heap}

  La idea es representar el \emph{\foreignlanguage{english}{heap}}
  en forma de árbol,
  con cada nodo conteniendo un objeto con clave menor que sus descendientes,
  como muestra la figura~\ref{fig:pairing-heap}.
  \begin{figure}[ht]
    \centering
    \begin{tikzpicture}
      \node {5}
        child {node {17}
          child {node {70}}}
        child {node {8}}
        child {node {50}
          child {node {68}}
          child {node {90}}};
    \end{tikzpicture}
    \caption{Un ejemplo de \emph{pairing heap}}
    \label{fig:pairing-heap}
  \end{figure}
  Por ahora supondremos esta representación abstracta,
  más adelante discutiremos una estructura concreta.

  Obtener el mínimo
  (\(\mathrm{find\_min}(H)\))
  es acceder a la raíz,
  la operación \(\mathrm{meld}(H_1, H_2)\)
  es poner el árbol con raíz mayor debajo de la raíz del otro
  (elegidos arbitrariamente en caso de empate).
  La operación \(\mathrm{insert}(H, x)\)
  es crear un nuevo árbol con \(x\) de raíz,
  y luego unirlo con el existente.
  Todas tienen costo constante.

  Efectuar \(\mathrm{decrease\_key}(H, x, \Delta)\)
  podría hacerse ajustando la clave de \(x\)
  siempre que no resulte menor que la de su padre,
  o tomando el \emph{\foreignlanguage{english}{heap}} con \(x\) de raíz,
  eliminando \(x\) de él
  (esencialmente la operación \(\mathrm{delete\_min}(H)\)
   en ese subárbol)
  y uniéndolo con el resto.
  Luego insertamos el objeto \(x\) modificado.
  Por simplicidad,
  usaremos siempre la segunda opción
  (acceder al padre es costoso en la estructura que plantearemos más adelante).

  La operación central,
  \(\mathrm{delete\_min}(H)\) es más delicada.
  Al eliminar la raíz,
  quedan varios árboles huérfanos,
  que deben unirse.
  En el peor caso,
  todos los demás objetos son raíces,
  con lo que el costo de hacer esto en forma natural es \(O(n)\).
  La propuesta más simple,
  que llaman \emph{\foreignlanguage{english}{two pass pairing heaps}},
  es como sigue:
  al eliminar la raíz,
  quedan cero o más árboles.
  Estos los unimos a pares,
  partiendo desde el más nuevo
  (suponemos que se agregan árboles a la izquierda),
  y los pares se acumulan luego del más viejo al más nuevo
  (de derecha a izquierda,
   en nuestro orden).
  La figura~\ref{fig:ph-delete-min} muestra la operación.
  Debe considerarse que los nodos hijos a su vez son raíces de árboles,
  que se mantienen intactos.
  Note que el efecto es crear árboles con menos hijos,
  en el ejemplo la primera operación \(\mathrm{delete\_min}\) es cara,
  pero las siguientes serán baratas.
  \begin{figure}[ht]
    \centering
    \subfloat[Original]{
      \begin{tikzpicture}[sibling distance = 2em, level distance = 1cm]
        \node {1}
          child {node {13}}
          child {node {5}
            child {node{17}}}
          child {node {2}
            child {node {7}}}
          child {node {8}}
          child {node {3}
            child {node {6}}
            child {node {4}}};
      \end{tikzpicture}
      \label{subfig:ph-dm-original}
    }
    \hspace*{3.75em}
    \subfloat[Parear]{
      \begin{tikzpicture}[sibling distance = 2em, level distance = 1cm]
        \node (r1) {13};
        \node [right = of r1] (r2) {2}
          child {node {5}
            child {node {17}}}
          child {node {7}};
        \node [right = of r2] (r3) {3}
          child {node {8}}
          child {node {6}}
          child {node {4}};
      \end{tikzpicture}
      \label{subfig:ph-dm-parear}
    }
    \hspace*{3.75em}
    \subfloat[Acumular]{
      \begin{tikzpicture}[sibling distance = 2em, level distance = 1cm]
        \node {2}
          child {node {13}}
          child {node {5}
            child {node {17}}}
          child {node {7}}
          child {node {3}
            child {node {8}}
            child {node {6}}
            child {node {4}}};
      \end{tikzpicture}
      \label{subfig:ph-dm-acumular}
    }
    \caption{Operación \(\mathrm{delete\_min}\)}
    \label{fig:ph-delete-min}
  \end{figure}

\subsection{Estructura concreta}
\label{sec:ph-concreta}

  Dadas las operaciones que estamos efectuando con los árboles,
  es natural la representación enlazada mediante punteros al primer hijo
  y una lista doblemente enlazada de los hijos,
  con los punteros extremos de la lista apuntando al padre,
  junto con indicación si el nodo es primero o último en la lista.
  Esto permite agregar o eliminar elementos en tiempo \(O(1)\),
  y tener acceso al padre por ejemplo en caso que se quede sin hijos.
  En esta representación todas las operaciones toman tiempo \(O(1)\),
  salvo \(\operatorname{delete\_min}\),
  \(\operatorname{delete}\) y \(\operatorname{decrease\_key}\).

\subsection{Análisis amortizado}
\label{sec:ph-analisis-amortizado}

  Definimos el \emph{tamaño} de \(x\) en un un árbol,
  anotado \(s(x)\),
  como el número de nodos en el subárbol con \(x\) de raíz
  (incluyendo a \(x\));
  y su \emph{rango} como \(r(x) = \log_2 s(x)\).
  Usamos el método potencial,
  con función potencial de un conjunto de árboles:
  \begin{equation}
    \label{eq:ph-phi}
    \Phi(H)
      = \sum_{x \in H} r(x)
  \end{equation}
  El potencial de un conjunto vacío de árboles es \num{0},
  y el potencial nunca es negativo.

  Observamos que el rango de un nodo en un árbol de \(n\) nodos
  está entre \num{0} y \(\log_2 n\).
  Las operaciones \(\mathrm{make\_heap}\)
  y \(\mathrm{find\_min}\) no afectan a \(\Phi\),
  ya que no cambian el rango de ningún nodo;
  su costo amortizado es \(O(1)\).
  Si el número total de nodos es \(n\),
  las operaciones \(\operatorname{insert}\),
  \(\operatorname{meld}\) y \(\operatorname{decrease\_key}\)
  tienen costo amortizado \(O(\log n)\),
  cada una de ellas causa un aumento de potencial acotado por \(\log_2 n + 1\).
  Esto porque la raíz menor aumenta de rango,
  y en menos de \(\log_2 n\)
  (adquiere cuando más \(n - 1\) nuevos descendientes);
  y en caso de \(\mathrm{insert}\) estamos agregando un nuevo nodo,
  que aporta \num{1}.

  La operación \(\mathrm{delete\_min}(H)\) es más difícil de analizar.
  Sea \(H\) un \emph{\foreignlanguage{english}{heap}} de \(n\) nodos
  de raíz \(x\),
  y llamemos \(x_i\) el \(i\)\nobreakdash-ésimo hijo de \(x\).
  Buscamos acotar el número de operaciones
  al reconstruir un \emph{\foreignlanguage{english}{heap}}
  de los árboles con raíces \(x_1, \dotsc, x_k\).
  El tiempo de ejecución de esta operación es uno más
  de los enlaces de nodos efectuados.
  Los enlaces en el primer paso
  (parear)
  son al menos tantos como en el segundo paso
  (acumular).
  Cargaremos \num{2} por enlace en el primer paso.

\section*{Ejercicios}
\label{sec:ejercicios-26-seq2}

  \begin{enumerate}
  \item
    Escriba las operaciones de \emph{\foreignlanguage{english}{pairing heap}}
    como una clase por ejemplo en \cplusplus{} o Java.
  \end{enumerate}

\bibliography{../referencias}

%%% Local Variables:
%%% mode: latex
%%% TeX-master: "../INF-221_notas"
%%% ispell-local-dictionary: "spanish"
%%% End:

% LocalWords:  autoorganizantes english Move To Front MTF Splay trees
% LocalWords:  desbalanceado AVL portable splay Pairing Heaps heap
% LocalWords:  pairing heaps rank ruma computines two pass ésimo

\bibliographystyle{babplain-fl}

\chapter{Algoritmo de Kruskal, Union-Find}
\label{cha:algoritmo-de-kruskal}

  Un problema recurrente es hallar el árbol recubridor mínimo
  (\emph{\foreignlanguage{english}{Minimal Spanning Tree}} en inglés,
    abreviado MST)
  de un grafo rotulado conexo.
  En detalle,
  dado un grafo \(G = (V, E)\) conexo,
  con arcos rotulados por \(w \colon E \to \mathbb{R}^{+}\)
  (el rótulo representa costo del arco),
  hallar un árbol recubridor de costo total
  (suma de los costos de los arcos)
  mínimo.
  Una solución a este problema da el algoritmo de Kruskal~%
    \cite{kruskal56:_MST},
  (algoritmo~\ref{alg:Kruskal},
   que ya discutimos en el capítulo~\ref{cha:greedy-algorithm}).
  La idea es ir construyendo un bosque
  (conjunto de árboles),
  uniendo sucesivamente árboles
  mediante arcos de costo mínimo.
  Inicialmente el bosque es simplemente cada vértice por separado,
  al final es un árbol recubridor mínimo.
  \begin{algorithm}[ht]
    \DontPrintSemicolon\Indp

    Sort \(E\) by increeaing cost \;
    \(S \gets \varnothing\) \;
    \For{\(v \in V\)}{
      Add \(\{v\}\) to \(S\) \;
    }
    \(T \gets \varnothing\) \;
    \For{\(u v \in E\)}{
      \If{\(u\) and \(v\) don't belong to the same set of \(S\)}{
        Add \(u v\) to \(T\) \;
        Join the sets to which \(u\) and \(v\) belong in \(S\) \;
      }
    }
    \Return (V, T) \;

    \caption{Algoritmo de Kruskal}
    \label{alg:Kruskal}
  \end{algorithm}
  Este es un ejemplo clásico de algoritmo voraz.
  Nos interesa derivar un programa eficiente
  (y deducir su complejidad).
  Es claro que la manipulación del conjunto \(S\)
  es crucial.
  Parte de la discusión que sigue viene de Erickson~%
    \cite[clase~17]{erickson19:_algorithms}.

\section{Una estructura de datos
       para \emph{\foreignlanguage{english}{union-find}}}
\label{sec:union-find-estructura}

  Abstrayendo las operaciones empleadas,
  vemos que requerimos una estructura de datos
  que representa una partición de un conjunto universo \(V\),
  con operaciones de inicializar con elementos solitarios,
  hallar
  (\emph{\foreignlanguage{english}{find}})
  la partición a la que pertenece un elemento,
  y unir particiones disjuntas
  (\emph{\foreignlanguage{english}{union}}).
  Por ellas se le llama
  problema de \emph{\foreignlanguage{english}{union-find}}.
  \begin{figure}[ht]
    \centering
    \begin{tikzpicture}[edge from parent/.style = {draw, latex'-}]
      \node (a) at (0, 0) {\num{3}}
        child {node {\num{10}}
          child {node{\num{15}}}
          child {node{\num{17}}}};
      \path[-latex'] (a) edge [loop above] (a);

      \node (b) at (3, 0) {\num{11}}
        child {node {\num{4}}};
      \path[-latex'] (b) edge [loop above] (b);

      \node (c) at (6, 0) {\num{13}}
        child {node {\num{5}}}
        child {node {\num{6}}}
        child {node {\num{14}}
          child {node {\num{9}}
            child {node {\num{8}}}
            child {node {\num{16}}}
            child {node {\num{20}}
              child {node {\num{18}}}
              child {node {\num{19}}}}}};
      \path[-latex'] (c) edge [loop above] (c);
    \end{tikzpicture}
    \caption{Esquema de la estructura para
             \emph{\foreignlanguage{english}{union-find}}.}
    \label{fig:union-find-estructura}
  \end{figure}
  La manera de identificar a un subconjunto de \(V\) es irrelevante,
  podemos elegir un elemento cualquiera como representante.
  Para hallar el representante del conjunto al que pertenece \(v\)
  podemos hacer que cada elemento apunte a un elemento padre,
  siguiendo la lista hasta el final hallamos el representante.
  Unir dos conjuntos es hacer que el representante de uno
  quede como padre del representante del otro,
  ver la figura~\ref{fig:union-find-estructura}.
  Para simplificar algunos de los algoritmos,
  hacemos que las raíces de los árboles apunten a ellas mismas.
  En diagramas sucesivos omitiremos las flechas
  y los bucles en las raíces.

  La operación \(\mathrm{find}\) depende de la altura
  de los árboles que se construyan,
  interesa construir árboles bajos.
  Nos conviene hacer que el representante con el árbol menor
  dependa del representante con el árbol más alto,
  ya que esto no aumenta la altura.
  Si mantenemos un arreglo \(\mathrm{rank}\) con la altura del árbol
  de cada representante,
  basta poner de hijo al representante de altura menor.
  Solo en caso de empate la altura aumenta,
  elegimos uno de los dos como nuevo representante
  con \(\mathrm{rank}\) uno mayor.
  Inicialmente \(\mathrm{rank}\) es cero para todos los vértices.
  Note que solo cuando se unen dos árboles de la misma altura
  se ajusta \(\mathrm{rank}\) del nuevo representante.
  Es un simple ejercicio de inducción demostrar que de esta manera
  si \(v\) es representante de una clase,
  esta contiene al menos \(2^{\mathrm{rank}[v]}\) elementos.
  En consecuencia,
  el máximo camino posible de un vértice a su representante en la clase \(C\)
  es de largo \(\log_2 \lvert C \rvert\).
  El costo para cada operación es \(O(\log n)\).
  Esta estructura y su manipulación fueron propuestas por
  Galler y Fisher~%
    \cite{galler64:_improved_equiv_algorithm},
  aunque su análisis tomó años.
  La idea se basa en arreglos globales \(\mathrm{rank}\)
  (la altura del árbol con raíz en el vértice)
  y \(\mathrm{parent}\)
  (el padre del vértice,
   para la raíz es el mismo para simplificar el código).
  Vea los algoritmos
  para \(\mathrm{MakeSets}\)
  (crea la estructura inicial),
  \(\mathrm{union}\)
  (une las clases de \(u\) y \(v\))
  y \(\mathrm{find}\)
  (halla el representante para la clase de \(v\)),
  respectivamente~\ref{alg:union-find-MakeSets},
  \ref{alg:union-find-union},
  y~\ref{alg:union-find-find}.
  \begin{algorithm}[ht]
    \DontPrintSemicolon\Indp

    \Procedure{\(\operatorname{MakeSets}(V)\)}{
      \For{\(v \in V\)}{
        \(\mathrm{rank}[v] \gets 0\) \;
        \(\mathrm{parent}[v] \gets v\) \;
      }
    }
    \caption{Algoritmo para crear conjuntos}
    \label{alg:union-find-MakeSets}
  \end{algorithm}
  \begin{algorithm}[ht]
    \DontPrintSemicolon\Indp

    \Function{\(\operatorname{find}(v)\)}{
      \While{\(v \ne \mathrm{parent}[v]\)}{
        \(v \gets \mathrm{parent}[v]\) \;
      }
      \Return \(v\) \;
    }
    \caption{Algoritmo para encontrar representante}
    \label{alg:union-find-find}
  \end{algorithm}
  \begin{algorithm}[ht]
    \DontPrintSemicolon\Indp

    \Procedure{\(\operatorname{union}(u, v)\)}{
      \uIf{\(\mathrm{rank}[u] > \mathrm{rank}[v]\)}{
        \(\mathrm{parent}[v] \gets u\) \;
      }
      \Else{
        \(\mathrm{parent}[u] \gets v\) \;
        \If{\(\mathrm{rank}[u] = \mathrm{rank}[v]\)}{
          \(\mathrm{rank}[v] \gets \mathrm{rank}[v] + 1\) \;
        }
      }
    }
    \caption{Algoritmo para unir conjuntos dados representantes}
    \label{alg:union-find-union}
  \end{algorithm}

\bibliography{../referencias}

%%% Local Variables:
%%% mode: latex
%%% TeX-master: "../INF-221_notas"
%%% ispell-local-dictionary: "spanish"
%%% End:

% LocalWords:  Union Find recubridor english Spanning Tree MST Sort
% LocalWords:  by increeaing cost Add to and belong the same set of
% LocalWords:  Join sets which in union find

\bibliographystyle{babplain-fl}

\chapter{Análisis de Union-Find}
\label{cha:analisis-de-union-find}

  Vimos que la estructura union-find,
  como discutida el capítulo~\ref{cha:algoritmo-de-kruskal},
  es central en varios algoritmos,
  corresponde evaluar su rendimiento.
  Las técnicas empleadas son instructivas,
  las expandiremos en clases sucesivas.

\section{Análisis de la versión simple}
\label{sec:union-find-analisys-simple}

  Veamos algunas propiedades cruciales de nuestra estructura inicial
  bajo las operaciones mencionadas:
  \begin{enumerate}[label = {\textbf{Propiedad \arabic*:}},
                    ref = \arabic*,
                    leftmargin = *,
                    itemindent = 4em]
  \item
    \label{prop:ufs-1}
    \emph{Para todo \(v\) que no es raíz,
          \(\mathrm{rank}[v]
               < \mathrm{rank}[\mathrm{parent}[v]]\).}

    \begin{proof}
      Un nodo de rango \(k\) se crea al unir dos árboles de rango \(k - 1\),
      una vez que un nodo deja de ser raíz su rango no se modifica más.
    \end{proof}
  \item
    \label{prop:ufs-2}
    \emph{Una raíz de rango \(k\) tiene al menos \(2^k\) nodos en su árbol.}

    \begin{proof}
      Una simple inducción.
      Esto se aplica a nodos internos
      (no raíz)
      también:
      un nodo de rango \(k\) tiene al menos \(2^k\) descendientes,
      ya que alguna vez fue raíz
      y una vez que deja de serlo su rango y sus descendientes no cambian más.
    \end{proof}
  \item
    \label{prop:ufs-3}
    \emph{Si hay un total de \(n\) nodos,
          hay a lo más \(n / 2^k\) nodos de rango \(k\).}

    \begin{proof}
      Considere su valor favorito \(k\),
      y cada vez que \(\mathrm{rank}\)
      de un nodo \(x\)
      cambia de \(k - 1\) a \(k\) marque todos sus descendientes.
      Cada nodo puede ser marcado a lo más una vez,
      ya que el rango de la raíz solo aumenta.
      Un nodo con rango \(k\)
      representa al menos a \(2^k\) nodos,
      habiendo un total de \(n\) nodos,
      hay a lo más \(n / 2^k\) nodos de rango \(k\).
    \end{proof}
    Esta observación indica,
    crucialmente,
    que el rango máximo es \(\log_2 n\),
    con lo que todos los árboles tienen altura a lo más \(\log_2 n\),
    y esta es una cota superior
    al tiempo de ejecución de \(\operatorname{find}\)
    y de \(\operatorname{union}\).
  \end{enumerate}

  Resulta que la cota de altura \(\log_2 n\) es ajustada,
  el árbol binomial \(B_k\)
  (construido uniendo dos árboles binomiales \(B_{k - 1}\),
   poniendo la raíz de uno como hijo de la raíz del otro,
   ver figura~\ref{fig:binomial-trees})
  tiene \(2^k\) nodos y altura \(k\).
  \begin{figure}[ht]
    \centering
    \begin{tikzpicture}[every node/.style = {shape = circle, fill, draw},
                        scale = 0.75]
      \node (a0) at (0, 4) {};
      \node [draw = none, fill = none, below] at (0, -0.2) {\(B_0\)};

      \node (b0) at (1.5, 3) {};
      \node (b1) at (1.5, 4) {};
      \draw (b0) -- (b1);
      \node [draw = none, fill = none, below] at (1.5, -0.2) {\(B_1\)};

      \node (c0) at (3.0, 2) {};
      \node (c1) at (3.0, 3) {};
      \node (c2) at (4.0, 3) {};
      \node (c3) at (4.0, 4) {};
      \draw (c0) -- (c1)
            (c1) -- (c3)
            (c2) -- (c3);
      \node [draw = none, fill = none, below] at (3.5, -0.2) {\(B_2\)};

      \node (d0) at (5.5, 1) {};
      \node (d1) at (5.5, 2) {};
      \node (d2) at (6.5, 2) {};
      \node (d3) at (6.5, 3) {};
      \node (d4) at (7.5, 1) {};
      \node (d5) at (7.5, 2) {};
      \node (d6) at (8.5, 2) {};
      \node (d7) at (8.5, 4) {};
      \draw (d0) -- (d1)
            (d1) -- (d3)
            (d2) -- (d3)
            (d4) -- (d5)
            (d6) -- (d7)
            (d3) -- (d7)
            (d5) -- (d7);
      \node [draw = none, fill = none, below] at (7, -0.2) {\(B_3\)};

      \node (e0)  at (10, 0) {};
      \node (e1)  at (10, 1) {};
      \node (e2)  at (11, 1) {};
      \node (e3)  at (11, 2) {};
      \node (e4)  at (12, 1) {};
      \node (e5)  at (12, 2) {};
      \node (e6)  at (13, 2) {};
      \node (e7)  at (13, 3) {};
      \node (e8)  at (14, 1) {};
      \node (e9)  at (14, 2) {};
      \node (e10) at (15, 2) {};
      \node (e11) at (15, 3) {};
      \node (e12) at (16, 2) {};
      \node (e13) at (16, 3) {};
      \node (e14) at (17, 3) {};
      \node (e15) at (17, 4) {};
      \draw (e0) -- (e1)
            (e1) -- (e3)
            (e2) -- (e3)
            (e4) -- (e5)
            (e6) -- (e7)
            (e3) -- (e7)
            (e5) -- (e7)
            (e8) -- (e9)
            (e9) -- (e11)
            (e10) -- (e11)
            (e12) -- (e13)
            (e13) -- (e15)
            (e14) -- (e15)
            (e11) -- (e15)
            (e7) -- (e15);
      \node [draw = none, fill = none, below] at (13.5, -0.2) {\(B_4\)};
    \end{tikzpicture}
    \caption{Árboles binomiales}
    \label{fig:binomial-trees}
\end{figure}

\section{Compresión de caminos}
\label{sec:union-find-path-compression}

  Una mejora se obtiene de la observación
  que luego de una búsqueda podemos acortar los caminos al representante
  a un solo paso
  para todos los nodos que encontramos en el camino,
  la figura~\ref{fig:union-find-path-compression}
  ilustra el efecto de buscar \num{20} y comprimir caminos.
  Si tenemos un puntero al nodo del que buscamos representante,
  recorremos la lista de padres hasta llegar a la raíz;
  en una segunda pasada sobre la lista ajustamos los punteros a padres
  para que apunten directamente a la raíz.
  Los árboles de las ilustraciones
  definitivamente no son resultado de nuestros algoritmos,
  simplemente sirven para mostrar el efecto de las operaciones.
  \begin{figure}[ht]
    \centering
    \begin{tikzpicture}
      \node at (0, 0) {\num{13}}
        child {node {\num{5}}}
        child {node {\num{6}}}
        child {node {\num{14}}
          child {node {\num{9}}
            child {node {\num{8}}}
            child {node {\num{16}}}
            child {node {\num{20}}
              child {node {\num{18}}}
              child {node {\num{19}}}}}};

        \node at (7, 0) {\num{13}}
          child {node {\num{5}}}
          child {node {\num{6}}}
          child {node {\num{9}}
            child {node {\num{8}}}
            child {node {\num{16}}}}
          child {node {\num{14}}}
          child {node {\num{20}}
            child {node {\num{18}}}
            child {node {\num{19}}}};
    \end{tikzpicture}
    \caption{Acortar caminos
             (\emph{\foreignlanguage{english}{path compression}})
             al buscar \num{20}.}
    \label{fig:union-find-path-compression}
  \end{figure}
  El algoritmo resultante~\ref{alg:union-find-find-2}
  es la versión modificada de \(\operatorname{find}\).
  \begin{algorithm}[ht]
    \DontPrintSemicolon\Indp

    \Function{\(\operatorname{find}(v)\)}{
      \(u \gets v\) \;
      \While{\(v \ne \mathrm{parent}[v]\)}{
        \(v \gets \mathrm{parent}[v]\) \;
      }
      \While{\(u \ne v\)}{
        \(p \gets \mathrm{parent}[u]\) \;
        \(\mathrm{parent}[u] \gets v\) \;
        \(u \gets p\) \;
      }
      \Return \(v\) \;
    }
    \caption{Algoritmo modificado para encontrar representante}
    \label{alg:union-find-find-2}
  \end{algorithm}
  La idea es pagar un costo extra en las operaciones \(\operatorname{find}\)
  en la esperanza de ahorrar en operaciones futuras.
  Una variante más simple es cambiar abuelos por padres
  en el recorrido de la lista,
  ahorrando un recorrido separado,
  ver el algoritmo~\ref{alg:union-find-find-3},
  que difiere del algoritmo original~\ref{alg:union-find-find}
  en una única línea.
  \begin{algorithm}[ht]
    \DontPrintSemicolon\Indp

    \Function{\(\operatorname{find}(v)\)}{
      \While{\(v \ne \mathrm{parent}[v]\)}{
        \(\mathrm{parent}[v]
            \gets \mathrm{parent}[\mathrm{parent}[v]]\) \;
        \(v \gets \mathrm{parent}[v]\) \;
      }
      \Return \(v\) \;
    }
    \caption{Algoritmo para encontrar representante con compresión de abuelos}
    \label{alg:union-find-find-3}
  \end{algorithm}
  La figura~\ref{fig:union-find-path-compression-grandparent}
  ilustra el efecto al buscar \num{19} usando esta variante.
  El algoritmo~\ref{alg:union-find-find-3}
  aprovecha que la raíz es su propio padre.
  \begin{figure}[ht]
    \centering
    \begin{tikzpicture}
      \node at (0, 0) {\num{13}}
        child {node {\num{5}}}
        child {node {\num{6}}}
        child {node {\num{14}}
          child {node {\num{9}}
            child {node {\num{8}}}
            child {node {\num{16}}}
            child {node {\num{20}}
              child {node {\num{18}}}
              child {node {\num{19}}}}}};

      \node at (7, 0) {\num{13}}
        child {node {\num{5}}}
        child {node {\num{6}}}
        child {node {\num{9}}
          child {node {\num{8}}}
          child {node {\num{16}}}}
        child {node {\num{14}}
          child {node {\num{20}}
            child {node {\num{18}}}
            child {node {\num{19}}}}};
    \end{tikzpicture}
    \caption{Acortar caminos
             (\emph{\foreignlanguage{english}{path compression}})
             con abuelos desde \num{20}.}
    \label{fig:union-find-path-compression-grandparent}
  \end{figure}

\section{Análisis de compresión de caminos}
\label{sec:union-find-analysis-path-compression}

  Como estamos pagando un costo extra en ciertas operaciones
  en la esperanza de que produzca ahorros futuros,
  debemos analizar secuencias de operaciones,
  no operaciones individuales.

  Para \(g \colon \mathbb{R} \to \mathbb{R}\)
  tal que para \(x > 1\) siempre es \(g(x) < x\) definimos:
  \begin{equation}
    \label{eq:def-g*}
    g^*(x)
      = \begin{cases}
          0		& x \le 1 \\
          1 + g^*(g(x))	& x >	1
        \end{cases}
  \end{equation}
  En el fondo, \(g^*(x)\) es el número de veces
  que hay que aplicar \(g\) a \(x\)
  hasta obtener un valor \num{1} o menor.
  De acá definimos \(\log_2^* x\),
  donde el logaritmo es en base \num{2}
  (¡somos computines!).
  Es claro que \(\log_2^* n\) crece extremadamente lento:
  \begin{equation*}
    \log_2^* n
      = \begin{cases}
           0 & \phantom{1 < {}}
                        n \le 1	  \\
           1 & 1      < n \le 2	  \\
           2 & 2      < n \le 2^2 \\
           3 & 2^2    < n \le 2^4 \\
           4 & 2^4    < n \le 2^{16} \\
           5 & 2^{16} < n \le 2^{65\,536}
        \end{cases}
  \end{equation*}

  El análisis clásico de esta estructura
  (Hopcroft y Ullman~\cite{hopcroft73:_set_merging_algorithm},
   Tarjan~\cite{tarjan75:_union-find_analysis})
  es complejo.
  Acá seguimos la idea de Seidel y Sharir~%
    \cite{seidel05:_top_down_analysis_path_compr},
  que da un análisis sencillo
  (todo depende del cristal con que se mira\ldots).
  Primeramente,
  las tres propiedades enunciadas antes se siguen cumpliendo
  aún si se comprimen caminos.
  Basan su análisis en dos nuevas operaciones,
  \(\operatorname{compress}(u, v)\)
  que comprime un camino cualquiera en el bosque
  (no necesariamente llegando a una raíz)
  entre nodos \(u\) y \(v\),
  donde \(v\) es un ancestro de \(u\),
  y la operación \(\operatorname{shatter}(u, v)\),
  que hace una raíz de todo nodo en el camino.
  \begin{algorithm}[ht]
    \DontPrintSemicolon\Indp

    \Procedure{\(\operatorname{compress}(u, v)\)}{
      \emph{\(v\) must be ancestor of \(u\)} \;
      \If{\(u \ne v\)}{
        \(\operatorname{compress}(\mathrm{parent}[u], v)\) \;
        \(\mathrm{parent}[u] \gets \mathrm{parent}[v]\) \;
      }
    }
    \caption{Operación \(\operatorname{compress}\)}
    \label{alg:uf-compress}
  \end{algorithm}
  \begin{algorithm}[ht]
    \DontPrintSemicolon\Indp

    \Procedure{\(\operatorname{shatter}(u, v)\)}{
      \emph{\(v\) must be ancestor of \(u\)} \;
      \If{\(\mathrm{parent}[u] \ne v\)}{
        \(\operatorname{shatter}(\mathrm{parent}[u], v)\) \;
       \(\mathrm{parent}[u] \gets u\) \;
      }
    }
    \caption{Operación \(\operatorname{shatter}\)}
    \label{alg:uf-shatter}
  \end{algorithm}
  Cabe hacer notar que estas operaciones no son para uso en el programa,
  sirven para reordenar las acciones simplificando las demostraciones.
  En particular,
  si \(\operatorname{union}\) no reorganiza los árboles,
  solo manipula las raíces,
  una secuencia cualquiera
  de \(\operatorname{union}\) y \(\operatorname{find}\)
  puede efectuarse haciendo las \(\operatorname{union}\),
  seguidas por \(\operatorname{compress}\)
  sin cambiar el número de manipulaciones de punteros.
  El costo de \(\operatorname{union}\) es constante
  (\(O(1)\)),
  \(\operatorname{find}\) es básicamente \(\operatorname{compress}\),
  que es \(O(1)\)
  más un término proporcional al número de punteros manipulados.
  Fijaremos entonces el número de punteros manipulados
  como medida de costo.
  Sea \(T(m, n, r)\) el número de asignaciones de punteros en el peor caso
  en cualquier secuencia de \(m\) operaciones \(\operatorname{compress}\)
  sobre un bosque de a lo más \(n\) nodos,
  con \(\mathrm{rank}\) a lo más \(r\).

  La siguiente cota trivial sirve de base a nuestro argumento.
  \begin{lemma}
    \label{lem:T-trivial}
    \(T(m, n, r) \le n r\)
  \end{lemma}
  \begin{proof}
    Cada nodo puede cambiar padre a lo más \(r\) veces,
    ya que \(\mathrm{rank}\) siempre aumenta.
  \end{proof}
  Sea \(\mathscr{F}\) un bosque de \(n\) nodos
  con \(\mathrm{rank}\) máximo \(r\)
  y una secuencia \(C\) de \(m\) operaciones \(\operatorname{compress}\)
  sobre \(\mathscr{F}\),
  y sea \(T(\mathscr{F}, C)\)
  el número total de asignaciones de punteros ejecutados por esta secuencia.
  Divida el bosque en dos subbosques,
  un bosque \textquote{bajo} \(\mathscr{F}_-\)
  con los nodos de \(\mathrm{rank}[v] \le s\)
  y el bosque \textquote{alto} \(\mathscr{F}_+\)
  con los nodos de \(\mathrm{rank}[v] > s\).
  Como \(\mathrm{rank}\) aumenta al seguir punteros a padres,
  el padre de un nodo alto es otro nodo alto.
  Sean \(n_-\) y \(n_+\) el número de nodos bajos y altos,
  respectivamente.
  Ver la figura~\ref{fig:union-find-subforests}
  que muestra un bosque
  (un único árbol en el ejemplo)
  dividido en subbosques.
  \begin{figure}[ht]
    \centering
    \begin{tikzpicture}[scale = 1.75]
      \draw [fill = lightgray] (0, 0) -- (-1.154, -2) -- (1.154, -2) -- cycle;
      \node at (0, -1) {\(\mathscr{F}\)};

      \draw [dashed] (-2, -1) -- (2, -1);
      \node at (0.7, -1) [above right]
        {\scriptsize\(\mathrm{rank} \ge s\)};
      \node at (0.7, -1) [below right]
        {\scriptsize\(\mathrm{rank} < s\)};

      \draw [fill = lightgray]
         (4, 0) -- (4 - 0.577, -1) -- (4 + 0.577, -1) -- cycle;
      \node at (4, -0.577) {\(\mathscr{F}_+\)};

      \draw [draw = none, fill = lightgray, pattern = dots]
            (4 + 0 * 0.144 - 0.577, -1) -- (4 + 0 * 0.144 - 1.154, -2) --
            (4 + 8 * 0.144, -2) -- (4 + 8 * 0.144 - 0.577, -1) --
            cycle;

      \draw (4 + 0 * 0.144 - 0.577, -1) -- (4 + 0 * 0.144 - 1.154, -2)
            (4 + 0 * 0.144 - 0.577, -1) -- (4 + 0 * 0.144, -2);
      \draw (4 + 2 * 0.144 - 0.577, -1) -- (4 + 2 * 0.144 - 1.154, -2)
            (4 + 2 * 0.144 - 0.577, -1) -- (4 + 2 * 0.144, -2);
      \draw (4 + 4 * 0.144 - 0.577, -1) -- (4 + 4 * 0.144 - 1.154, -2)
            (4 + 4 * 0.144 - 0.577, -1) -- (4 + 4 * 0.144, -2);
      \draw (4 + 6 * 0.144 - 0.577, -1) -- (4 + 6 * 0.144 - 1.154, -2)
            (4 + 6 * 0.144 - 0.577, -1) -- (4 + 6 * 0.144, -2);
      \draw (4 + 8 * 0.144 - 0.577, -1) -- (4 + 8 * 0.144 - 1.154, -2)
            (4 + 8 * 0.144 - 0.577, -1) -- (4 + 8 * 0.144, -2);

      \node at (4, -1.5) {\(\mathscr{F}_-\)};
    \end{tikzpicture}
    \caption{Dividiendo el bosque según \(\mathrm{rank}\)}
    \label{fig:union-find-subforests}
  \end{figure}

  Cualquier secuencia de operaciones \(\operatorname{compress}\)
  sobre \(\mathscr{F}\)
  puede descomponerse
  en una secuencia de operaciones \(\operatorname{compress}\)
  sobre \(\mathscr{F}_+\)
  y una secuencia de operaciones \(\operatorname{compress}\)
  y \(\operatorname{shatter}\) sobre \(\mathscr{F}_-\)
  con el mismo costo.
  La modificación es prohibir a un nodo bajo tener un padre alto,
  haciendo que los nodos bajos se hagan raíces
  y dejando a los nodos altos de hijos de la raíz original,
  como muestra la figura~\ref{fig:union-find-compress-+-}
  para un camino simple a la raíz que cruza de nodos bajos a altos..
  \begin{figure}[ht]
    \centering
    \begin{tikzpicture}
      \node [draw, shape = circle]    at (0, 0)			 (0) {}
          node [draw, shape = circle] at (-0.2 *  1, -0.5 *  1)	 (1) {}
          node [draw, shape = circle] at (-0.2 *  2, -0.5 *  2)	 (2) {}
          node [draw, shape = circle] at (-0.2 *  3, -0.5 *  3)	 (3) {}
          node [draw, shape = circle] at (-0.2 *  4, -0.5 *  4)	 (4) {}
          node [draw, shape = circle] at (-0.2 *  5, -0.5 *  5)	 (5) {}
          node [draw, shape = circle] at (-0.2 *  6, -0.5 *  6)	 (6) {}
          node [draw, shape = circle] at (-0.2 *  7, -0.5 *  7)	 (7) {}
          node [draw, shape = circle] at (-0.2 *  8, -0.5 *  8)	 (8) {}
          node [draw, shape = circle] at (-0.2 *  9, -0.5 *  9)	 (9) {}
          node [draw, shape = circle] at (-0.2 * 10, -0.5 * 10) (10) {};

      \draw[gray] (-0.2 * 5.5 - 0.5, -0.5 * 5.5)
                    -- (0.2 * 5.5 + 0.5, -0.5 * 5.5);

      \path[-latex'] (10) edge node {} (9)
                 (9) edge node {} (8)
                 (8) edge node {} (7)
                 (7) edge node {} (6)
                 (6) edge node {} (5)
                 (5) edge node {} (4)
                 (4) edge node {} (3)
                 (3) edge node {} (2)
                 (2) edge node {} (1)
                 (1) edge node {} (0)
                 (0) edge [loop above] node {} (0);


      \node [draw, shape = circle]    at (5, 0)		      (0) {}
          node [draw, shape = circle] at (5 - 0.4 * 2, -0.7)  (1) {}
          node [draw, shape = circle] at (5 - 0.4 * 1, -0.7)  (2) {}
          node [draw, shape = circle] at (5 + 0.4 * 0, -0.7)  (3) {}
          node [draw, shape = circle] at (5 + 0.4 * 1, -0.7)  (4) {}
          node [draw, shape = circle] at (5 + 0.4 * 2, -0.7)  (5) {};

      \path[-latex'] (0) edge [loop above] node {} (0)
                (1) edge	      node {} (0)
                (2) edge	      node {} (0)
                (3) edge	      node {} (0)
                (4) edge	      node {} (0)
                (5) edge	      node {} (0);

      \draw[gray] (5 + -0.2 * 5.5 - 0.5, -0.5 * 5.5)
                    -- (5 + 0.2 * 5.5 + 0.5, -0.5 * 5.5);

      \node [draw, shape = circle]    at (5 - 0.4 * 2, -0.5 * 6)  (a) {}
          node [draw, shape = circle] at (5 - 0.4 * 1, -0.5 * 6)  (b) {}
          node [draw, shape = circle] at (5 + 0.4 * 0, -0.5 * 6)  (c) {}
          node [draw, shape = circle] at (5 + 0.4 * 1, -0.5 * 6)  (d) {}
          node [draw, shape = circle] at (5 + 0.4 * 2, -0.5 * 6)  (e) {};

      \path[-latex'] (a) edge [loop above] node {} (a)
                (b) edge [loop above] node {} (b)
                (c) edge [loop above] node {} (c)
                (d) edge [loop above] node {} (d)
                (e) edge [loop above] node {} (e);
    \end{tikzpicture}
    \caption{División de una operación \(\operatorname{compress}\)}
    \label{fig:union-find-compress-+-}
  \end{figure}
  El punto de hacer esto es descomponer la secuencia en secuencias menores.
  Al dividir el bosque en nodos \textquote{altos} y \textquote{bajos},
  queda un bosque alto muy pequeño
  (son pocos los nodos con \(\mathrm{rank}\) alto),
  con lo que bastarán cotas bastante burdas
  para el costo de operaciones en él.
  El resultado es una recurrencia
  que acota el costo de la secuencia de operaciones.
  \begin{algorithm}[ht]
    \DontPrintSemicolon\Indp

    \Procedure{\(\operatorname{compress-rank}(u, v)\)}{
      \uIf{\(\mathrm{rank}[u] > s\)}{
        \(\operatorname{compress}(u, v)\) \;
      }
      \uElseIf{\(\mathrm{rank}[v] \le s\)}{
        \(\operatorname{compress}(u, v)\) \;
      }
      \Else{
        \(z \gets u\) \;
        \While{
          \(\mathrm{rank}[\mathrm{parent}[z]]
              \le s\)}{
          \(z \gets \mathrm{parent}[z]\) \;
        }
        \(\operatorname{compress}(\mathrm{parent}[z], v)\) \;
        \(\operatorname{shatter}(u, z)\) \;
        \(\mathrm{parent}[z] \gets z\) \;
      }
    }
    \caption{Operación equivalente}
    \label{alg:uf-compress-shatter}
  \end{algorithm}

  La operación \(\operatorname{compress}\) adaptada a bosques divididos
  considera el caso en que el camino \(u\) a \(v\)
  es enteramente \textquote{alto}
  (caso \(\mathrm{rank}[u] > s\)),
  enteramente \textquote{bajo}
  (cuando \(\mathrm{rank}[v] \le s\)),
  o cruza el rango y hay que subdividir la operación.

  La última asignación del algoritmo~\ref{alg:uf-compress-shatter}
  parece superflua,
  pero es necesaria en el análisis para simular una operación
  \(\mathrm{parent}[z] \leftarrow w\),
  con \(z\) un nodo bajo,
  \(w\) un nodo alto y el padre de \(z\) era un nodo alto también.
  Estas asignaciones \textquote{redundantes} se ejecutan inmediatamente después
  de una operación \(\operatorname{compress}\) en el bosque superior,
  por lo que hay a lo más \(m_+\) de estas operaciones.

  Durante la secuencia de operaciones \(C\) cada nodo es tocado
  por a lo más una operación \(\operatorname{shatter}\),
  por lo que el número total de operaciones con punteros en ellas
  es a lo más \(n\).

  Al dividir el bosque hemos dividido
  la secuencia de operaciones \(\operatorname{compress}\)
  en subsecuencias \(C_-\) y \(C_+\)
  de operaciones \(\operatorname{compress}\),
  de largos respectivos \(m_-\) y \(m_+\)
  (es \(m = m_+ + m_-\)),
  además de operaciones \(\operatorname{shatter}\).
  En vista de las consideraciones anteriores se cumple la desigualdad:
  \begin{equation}
    \label{eq:desigualdad-T}
    T(\mathscr{F}, C)
      \le T(\mathscr{F}_-, C_-) + T(\mathscr{F}_+, C_+) + m_+ + n
  \end{equation}
  Como hay a lo más \(n / 2^i\) nodos de rango \(i\),
  tenemos que:
  \begin{equation*}
    n_+
      \le \sum_{i > s} \frac{n}{2^i}
      = \frac{n}{2^s}
  \end{equation*}
  Con esto la cota del lema~\ref{lem:T-trivial}
  implica:
  \begin{equation*}
    T(\mathscr{F}_+, C_+)
      \le \frac{r n}{2^s}
  \end{equation*}
  Fijemos \(s = \lfloor \log_2 r \rfloor\),
  de manera que \(T(\mathscr{F}_+, C_+) \le n\).
  El bosque \(\mathscr{F}_-\)
  tiene \(\mathrm{rank}\) máximo \(s = \lfloor \log_2 r \rfloor\),
  además es claro que \(\lvert C_- \rvert \le \lvert C \rvert = m\).
  Podemos simplificar nuestra recurrencia a:
  \begin{equation*}
    T(\mathscr{F}, C)
      \le T(\mathscr{F}_-, C_-) + m_+ + 2 n
  \end{equation*}
  lo que con las observaciones previas es lo mismo que:
  \begin{equation*}
    T(\mathscr{F}, C) - m
      \le T(\mathscr{F}_-, C_-) - m_- + 2 n
  \end{equation*}
  Como esto vale en \emph{cualquier} bosque \(\mathscr{F}\)
  y para toda secuencia \(C\),
  y como \(T(m, n, r)\) es creciente con sus dos primeros argumentos,
  hemos demostrado que
  para \(T'(m, n, r) = T(m, n, r) - m\):
  \begin{align*}
    T'(m, n, r)
      &\le T'(m_-, n, \lfloor \log_2 r \rfloor) + 2 n \\
      &\le T'(m, n, \lfloor \log_2 r \rfloor) + 2 n
  \end{align*}
  Como condición inicial,
  para \(r = 1\) todos los nodos son raíces o hijos de una raíz,
  no hay manipulación de \(\mathrm{parent}\),
  y \(T'(m, n, 1) \le 0\).
  La solución a esta recurrencia es:
  \begin{equation*}
    T'(m, n, r)
      \le 2 n \log_2^* r
  \end{equation*}

  Hemos demostrado:
  \begin{theorem}
    \label{theo:T-log*}
    \(T(m, n, r)
      \le m + 2 n \log_2^* r\)
  \end{theorem}

  El teorema~\ref{theo:T-log*} puede mejorarse.
  En la demostración usamos la cota del lema~\ref{lem:T-trivial},
  que nuestro teorema~\ref{theo:T-log*} mejora,
  y puede usarse recursivamente,
  aumentando cada vez algo la dependencia de \(m\)
  mientras disminuye la de \(r\).
  Erickson~%
    \cite[clase~17]{erickson19:_algorithms}
  completa el desarrollo.
  Se concluye lo siguiente:
  La función de Ackermann~%
    \cite{ackermann28:_zum_hilbert_aufbau_zahlen}
  (aunque usamos la forma de dos argumentos definida por Péter~%
     \cite{peter34:_konstruktion_nichtrekursiver_funktionen}
   y Robinson~%
     \cite{robinson48:_recursion_double_recursion})
  se define como:
  \begin{equation*}
    \label{eq:Ackermann}
    A(m, n)
      = \begin{cases}
          n + 1			 & m = 0 \\
          A(m - 1, 1)		 & m > 0 \wedge n = 0 \\
          A(m - 1, A(m, n - 1))	 & m > 0 \wedge n > 0
        \end{cases}
  \end{equation*}
  Para algunos ejemplos,
  vemos que:
  \begin{align*}
    A(0, n)
      &= n + 1 \\
    A(1, n)
      &= A(0, A(1, n - 1)) \\
      &= A(1, n - 1) + 1 \\
      &\vdots \\
      &= A(1, 0) + n \\
      &= A(0, 1) + n \\
      &= n + 2 \\
      &= 2 + (n + 3) - 3 \\
    A(2, n)
      &= A(1, A(2, n - 1)) \\
      &= A(1, A(1, A(2, n - 2))) \\
      &\vdots \\
      &= A(1, A(1, A(1, \dotsc, A(1, A(2, 0)))) \\
      &= A(1, A(1, A(1, \dotsc, A(1, 3)) \\
      &\vdots \\
      &= 2 n + 3 \\
      &= 2 (n + 3) - 3 \\
    A(3, n)
      &= A(2, A(3, n - 1)) \\
      &\vdots \\
      &= A(2, A(2, A(2, \dotsc, A(2, A(2, 0))) \\
      &= A(2, A(2, A(2, \dotsc, A(2, 3))) \\
      &\vdots \\
      &= 2^{n + 3} - 3
  \end{align*}
  Esencialmente,
  cada aumento de \(m\) significa
  \textquote{aplique la función anterior \(n + 3\) veces},
  \(A(1, n)\) es \textquote{\(n + 3\) veces sumar \num{1}},
  \(A(2, n)\) es \textquote{\(n + 3\) veces sumar \num{2}},
  \(A(3, n)\) es \textquote{\(n + 3\) veces multiplicar por \num{2}},
  \(A(4, n)\) es \textquote{\(n + 3\) veces elevar a la potencia \num{2}},
  lo que da:
  \begin{equation*}
    A(4, n)
      = \begin{matrix}
          \underbrace{{2^2}^{{\cdot}^{{\cdot}^{{\cdot}^2}}}}_{n + 3} - 3
        \end{matrix}
  \end{equation*}
  O sea,
  por ejemplo:
  \begin{equation*}
    A(4, 3)
      = 2^{2^{65536}} - 3
  \end{equation*}
  Se ve que esta función crece extremadamente rápido.

  Se definen dos funciones inversas,
  que crecen extremadamente lento:
  \begin{align}
    \alpha(n)
      &= A^{-1}(n, n)
            \label{eq:alpha-1} \\
    \alpha(m, n)
      &= \min \{ i \ge 1 \colon A(i, \lfloor m / n \rfloor) \ge \log_2 n \}
            \label{eq:alpha-2}
  \end{align}
  De los valores anteriores para \(A(m, n)\)
  vemos que para todos los valores imaginables de \(m\) y \(n\)
  es \(\alpha(m, n) \le 5\).
  Claro que en estricto rigor no está acotada por una constante.

  En estos términos,
  Tarjan~%
    \cite{tarjan75:_union-find_analysis}
  demuestra que
  toda secuencia intercalada
  de \(m \ge n\) operaciones \(\operatorname{find}\)
  y \(n - 1\) operaciones \(\operatorname{union}\)
  toma \(O(m \alpha(m, n))\) tiempo.

  Volvamos al algoritmo de Kruskal,
  llamemos simplemente \(V\) y \(E\) al número de vértices y arcos,
  respectivamente.
  Se hacen a lo más \(2 E\) operaciones \(\operatorname{find}\),
  y exactamente \(V - 1\) operaciones \(\operatorname{union}\).
  El paso inicial,
  ordenar los arcos,
  puede hacerse en tiempo \(O(E \log  E)\).
  El número de arcos está acotado por \(V (V - 1) / 2\),
  con lo que \(\log E = O(\log V)\).
  Sabemos que \(r \le \lfloor \log_2  V \rfloor\),
  como \(2^r \le V\),
  resulta \(\log_2^* r = \log_2^* V - 1\).
  Las operaciones con clases de equivalencia aportan
  \(O(V)\) para los \(\operatorname{union}\),
  los \(\operatorname{find}\) aportan
  \(O(2 E + 2 V \log_2^* V)\)
  para un costo total de:
  \begin{equation*}
    O(E \log E) + O(V + 2 E + 2 V \log_2^* V)
      = O(E \log E)
  \end{equation*}
  Esto es dominado por el costo de ordenar los arcos.
  Usar la cota de Tarjan no mejora esto,
  cambia el segundo término a \(O(2 E \alpha(2 E, V))\).

% To do:
% - Exercises, algorithms using the same ideas (Jeffe, ...)

\bibliography{../referencias}

%%% Local Variables:
%%% mode: latex
%%% TeX-master: "../INF-221_notas"
%%% ispell-local-dictionary: "spanish"
%%% End:

% LocalWords:  Union Find union find english path compression must of
% LocalWords:  computines ancestor subbosques subsecuencias

\bibliographystyle{babplain-fl}

\chapter{Hashing}
\label{cha:hashing}

  Una \emph{tabla \foreignlanguage{english}{hash}}
  es una estructura que almacena un conjunto de objetos,
  permitiendo determinar rápidamente si un objeto dado está o no presente.
  Generalmente se asocia una \emph{clave} que define el objeto,
  como es un nombre,
  un rol o un número de inventario.
  La idea central
  es elegir una \emph{función de hashing} \(h\) que mapea toda clave posible
  a un entero pequeño \(h(x)\).
  Almacenamos el objeto asociado a \(x\) en la posición \(h(x)\) de un arreglo,
  este arreglo es la \emph{tabla \foreignlanguage{english}{hash}}.
  La idea se le atribuye a Dumey~%
    \cite{dumey56:_indexing_rapid_random_access_memory_systems},
  quien discute ideas afines
  en sistemas de inventario usando tarjetas perforadas
  y las extiende a memorias de acceso aleatorio.

  La importancia de esta idea
  es que ofrece algoritmos que dan tiempos de búsqueda
  constantes,
  independientes del número de datos almacenados.
  Por esta razón es la técnica de búsqueda preferida.
  Lo malo es que el buen rendimiento es solo en valor esperado,
  los peores casos son \emph{muy} malos
  (pero extremadamente poco probables si se toman precauciones apropiadas).

  El análisis de los algoritmos discutidos
  requiere rudimentos de probabilidades,
  para notación y discusión de los conceptos usados
  revise el apéndice~\ref{apx:pizca-probabilidades}.
  Usaremos varias cotas simples,
  pero extraordinariamente útiles,
  que no se han tratado anteriormente.
  Las demostraciones se encuentran en el apéndice.

\section{Algunos resultados previos}
\label{sec:resultados-previos}

  Requeriremos algunos resultados auxiliares
  para analizar \emph{\foreignlanguage{english}{hashing}}.
  Primero,
  un límite clásico:
  \begin{lemma}
    \label{lem:classic-limit}
    Para todo \(x \in \mathbb{R}\) se tiene que:
    \begin{equation*}
      \lim_{n \to \infty} \left( 1 + \frac{x}{n} \right)^n
        = \mathrm{e}^x
    \end{equation*}
  \end{lemma}
  \begin{proof}
    Como \(\mathrm{e}^x\) es continua,
    podemos escribir:
    \begin{align*}
      \lim_{n \to \infty} \left( 1 + \frac{x}{n} \right)^n
        &= \exp\left(
                 \lim_{n \to \infty} n \ln \left( 1 + \frac{x}{n} \right)
               \right) \\
        &= \exp\left(
                  \lim_{n \to \infty} \frac{\ln (1 + x / n)}{1/n}
               \right) \\
        &= \exp\left(
                  \lim_{n \to \infty}
                   \frac{\frac{- x / n^2}{1 + x / n}}{- 1 / n^2}
               \right) \\
        &= \mathrm{e}^x
    \end{align*}
    Acá usamos l'Hôpital para calcular el límite interno.
  \end{proof}
  Luego un par de cotas extremadamente útiles.
  \begin{lemma}
    \label{lem:bounds-(1+x/n)^n}
    Para todo \(x \ge 0\):
    \begin{equation*}
      1 + x
        \le \left( 1 + \frac{x}{n} \right)^n
        \le \mathrm{e}^x
    \end{equation*}
  \end{lemma}
  \begin{proof}
    Para la primera desigualdad,
    del teorema del binomio:
    \begin{align*}
      \left( 1 + \frac{x}{n} \right)^n
        &=   \sum_k \binom{n}{k} \left( \frac{x}{n} \right)^k \\
        &=   1 + x
               + \sum_{k \ge 2} \binom{n}{k} \left( \frac{x}{n} \right)^k \\
        &\ge 1 + x
    \end{align*}

    Para la segunda desigualdad
    escribimos por el teorema del binomio:
    \begin{align*}
      \left( 1 + \frac{x}{n} \right)^n
        &= \sum_{k \ge 0} \binom{n}{k} \left( \frac{x}{n} \right)^k \\
        &= \sum_{k \ge 0} \frac{n^{\underline{k}}}{n^k} \frac{x^k}{k!}
    \end{align*}
    Comparando con la serie para \(\mathrm{e}^x\),
    ambas son series de términos positivos ya que \(x\) no es negativo,
    como \(n^{\underline{k}} / n^k \le 1\)
    los términos de la segunda acotan a los de la primera por arriba.
  \end{proof}

  Otro resultado que requeriremos es la cota siguiente:
  \begin{lemma}
    \label{lem:binomial}
    Para \(n > t > 0\) se cumple:
    \begin{equation*}
      \binom{n}{t}
        \le \frac{n^n}{t^t (n - t)^{n - t}}
    \end{equation*}
  \end{lemma}
  \begin{proof}
    Primeramente,
    sabemos que:
    \begin{equation*}
      \binom{n}{t}
        = \frac{n!}{t! (n - t)!}
    \end{equation*}
    Usando la fórmula de Stirling
    para factoriales:
    \begin{equation}
      \label{eq:Stirling}
      n!
        = \sqrt{ 2 \pi n}
             \left( \frac{n}{\mathrm{e}} \right)^n
             \mathrm{e}^{r(n)}
    \end{equation}
    donde usaremos las cotas de Robbins~%
      \cite{robbins55:_remark_Stirlings_formula}
    para \(r(n)\):
    \begin{equation}
      \label{eq:Robbins}
      \frac{1}{12 n + 1}
        < r(n)
        < \frac{1}{12 n}
    \end{equation}
    Con la fórmula de Stirling obtenemos:
    \begin{align*}
      \binom{n}{t}
        &= \frac{(n / \mathrm{e})^n \sqrt{2 \pi n} \mathrm{e}^{r(n)}}
                {(t / \mathrm{e})^t \sqrt{2 \pi t} \mathrm{e}^{r(t)}
                 ((n - t) / \mathrm{e})^{n - t}
                   \sqrt{2 \pi (n - t)} \mathrm{e}^{r(n - t)}} \\
        &= \frac{n^n}{t^t (n - t)^{n - t}}
             \cdot \frac{\sqrt{n} \mathrm{e}^{r(n) - r(t) - r(n - t)}}
                        {\sqrt{2 \pi} \sqrt{t (n - t)}}
    \end{align*}
    Nos interesa demostrar que el segundo factor es menor a \num{1}.
    De partida,
    vemos que el mínimo de \(t (n - 1)\)
    en el rango de interés \(1 \le t \le n - 1\)
    se da para \(t = 1\) o \(t = n - 1\),
    aporta un factor:
    \begin{align*}
      \frac{\sqrt{n}}{\sqrt{t (n - t)}}
        &\le \frac{\sqrt{n}}{\sqrt{n - 1}} \\
        &\le \sqrt{2}
    \end{align*}
    Con las cotas de Robbins para \(r(n)\) podemos escribir:
    \begin{equation*}
      \frac{1}{12 n + 1}
        - \frac{1}{12 (n - t)}
        - \frac{1}{12 t}
        \le r(n) - r(t) - r(n - t)
        \le \frac{1}{12 n}
               - \frac{1}{12 t + 1}
               - \frac{1}{12 (n - t) + 1}
    \end{equation*}
    El mínimo de la cota izquierda se da para \(n = 2\), \(t = 1\),
    donde vale \(-19/150\);
    la cota derecha es menor a \num{0}
    (cada uno de los términos que se restan a \(1 / 12 n\)
     es mayor a este).
    Juntando todas las piezas,
    vemos que:
    \begin{align*}
      \binom{n}{t}
        &\le \frac{n^n}{t^t (n - t)^{n - t}}
               \cdot \frac{\sqrt{2} \mathrm{e}^{0}}{\sqrt{2 \pi}} \\
        &=   \frac{1}{\sqrt{\pi}} \cdot \frac{n^n}{t^t (n - t)^{n - t}} \\
        &<   \frac{n^n}{t^t (n - t)^{n - t}}
    \end{align*}
    dado que el primer factor definitivamente es menor a \num{1}.
  \end{proof}

\section{La escena de hashing}
\label{sec:escena-hashing}

  Seamos más específicos.
  Nos interesa almacenar \(n\) objetos,
  que pertenecen a un universo \(\mathscr{U}\).
  La tabla \emph{\foreignlanguage{english}{hash}}
  es un arreglo \(T[0..m - 1]\),
  donde \(m\) es el tamaño de la tabla.
  En estos términos:
  \begin{equation*}
    h \colon \mathscr{U} \to [0, m - 1]
  \end{equation*}
  Obviamente,
  si \(\lvert \mathscr{U} \rvert = m\),
  podemos usar simplemente \(x\) como índice de la tabla.
  El caso de interés nuestro es que \(n\)
  (el número de objetos a almacenar)
  es muchísimo menor que el total de objetos posibles;
  buscamos que \(m\) no sea mucho mayor que \(n\) para ahorrar espacio.
  Pero en caso que \(m < \lvert \mathscr{U} \rvert\)
  necesariamente debemos manejar \emph{colisiones},
  situaciones en que queremos almacenar dos objetos \(x \ne y\)
  tales que \(h(x) = h(y)\).
  Para esto hay tres alternativas de solución:
  \begin{description}
  \item[Direccionamiento cerrado:]
    Los objetos que colisionan se almacenan en una estructura secundaria,
    como una lista o un árbol binario de búsqueda.
  \item[Direccionamiento abierto:]
    Si se produce una colisión,
    almacenamos uno de los objetos en alguna otra ubicación libre de la tabla.
    Si al agregar elementos se llena demasiado la tabla,
    podemos usar la idea básica de arreglos dinámicos
    (sección~\ref{sec:arreglo-dinámico}).
  \item[Hashing perfecto:]
    Si conocemos los objetos a almacenar de antemano,
    podemos elegir \(h\) de forma que no hayan colisiones.
    Pero las funciones de \emph{\foreignlanguage{english}{hashing}} perfectas
    (esencialmente permutaciones)
    son relativamente raras.
    Para \(n\) elementos hay \(n!\) permutaciones y \(n^n\) funciones en total,
    o sea,
    usando la fórmula de Stirling
    la proporción es aproximadamente:
    \begin{align}
      \frac{n!}{n^n}
        &\approx \frac{\sqrt{2 \pi n} (n / \mathrm{e})^n}{n^n} \\
        &=	 \sqrt{2 \pi n} \mathrm{e}^{-n}
                      \label{eq:fraction.perfect}
    \end{align}
    Fredman, Komlós y Szemerédi~%
      \cite{fredman84:_perfect_hashing}
    describen una técnica eficiente
    para construir funciones
    de \emph{\foreignlanguage{english}{hashing}} perfectas.
  \end{description}

\section{Importancia del azar}
\label{sec:azar-hash}

  Por el principio del palomar,
  sea cual sea la función de \emph{\foreignlanguage{english}{hashing}}
  elegida para una tabla de tamaño \(m\)
  hay al menos \(\lceil \lvert \mathscr{U} \rvert / m \rceil\) objetos
  que dan a la misma posición.
  Si en una aplicación aparecen muchos objetos que caen en la misma posición,
  ya sea por mala suerte o por elección maliciosa,
  el rendimiento será horrible.
  Este es un riesgo de seguridad importante
  cuando se procesan datos controlados
  por un potencial adversario
  (los llamados ataques de complejidad algorítmica,
   ver por ejemplo Crosby y Wallach~%
     \cite{crosby03:_DoS_algo_compl_attack}).
  La única solución práctica es elegir la función al azar
  entre una colección suficientemente grande.
  Específicamente,
  fijamos una colección de funciones \(\mathscr{H}\)
  de \(\mathscr{U}\) a \([0, m - 1]\),
  y elegimos \(h \in \mathscr{H}\) al azar según alguna distribución.
  Distintos conjuntos de funciones y distribuciones
  dan garantías teóricas diferentes.

  El análisis teórico es más simple
  si se suponen funciones \emph{\foreignlanguage{english}{hash}}
  \emph{ideales al azar},
  se elige \(h\) uniformemente al azar
  entre todas las funciones de \(\mathscr{U}\) a \([0, m - 1]\).
  Es un modelo simple y limpio,
  que da las máximas garantías posibles,
  pero requiere elegir el valor de \(h(x)\) al azar para cada \(x\),
  lo que significaría tener que registrar su valor\ldots{}
  y volvemos a nuestro problema inicial.
  En la práctica,
  nos restringimos a familias de funciones
  \textquote{lo suficientemente al azar}
  para dar buen rendimiento.

  Una propiedad que intuitivamente parece útil es \emph{uniformidad}.
  Se dice que la familia de funciones \(\mathscr{H}\) es uniforme
  si eligiendo una función \(h\) uniformemente al azar de \(\mathscr{H}\)
  cada valor es igualmente probable para cada objeto del universo:
  \begin{equation}
    \label{eq:hash-uniform}
    \Pr_{h \in \mathscr{H}}[ h(x) = i ]
      = \frac{1}{m}
      \quad\text{para todo \(x \in \mathscr{U}\)
                 y todo \(i \in [0, m - 1]\)}
  \end{equation}
  Sin embargo,
  esto no es particularmente relevante.
  Considere la familia \(\mathscr{K}\) de funciones constantes
  definidas mediante \(\mathrm{const}_a(x) = a\) para todo \(x\).
  La familia \(\mathscr{K}\) es perfectamente uniforme y totalmente inútil.

  Lo que buscamos realmente es minimizar colisiones.
  Se dice que la familia \(\mathscr{H}\) es \emph{universal}
  si la probabilidad de colisión entre dos objetos diferentes
  es la menor posible:
  \begin{equation}
    \label{eq:hash-universal}
    \Pr_{h \in \mathscr{H}}[ h(x) = h(y) ]
      \le \frac{1}{m}
      \quad\text{para todo \(x \ne y\)}
  \end{equation}

  La mayor parte de los análisis elementales
  requieren una versión menos exigente,
  se dice que la familia \(\mathscr{H}\)
  es \emph{cercana a universal}
  si la probabilidad de colisión es cercana a la ideal:
  \begin{equation}
    \label{eq:hash-near-universal}
    \Pr_{h \in \mathscr{H}}[ h(x) = h(y) ]
      \le \frac{2}{m}
      \quad\text{para todo \(x \ne y\)}
  \end{equation}
  No hay nada especial en la constante \num{2},
  toda constante explícita mayor a \num{1} sirve.

  Análisis más detallado
  requiere considerar colisiones en grupos mayores de objetos.
  Para cada entero \(k\) se dice que la familia \(\mathscr{H}\)
  es \emph{fuertemente \(k\)\nobreakdash-universal}
  o \emph{\(k\)\nobreakdash-uniforme}
  si para toda colección de \(k\) objetos y de \(k\) índices:
  \begin{equation}
    \label{eq:hash-k-uniform}
    \Pr_{h \in \mathscr{H}}
       \left[
         \bigwedge_{1 \le j \le k} h(x_j) = i_j
       \right]
       = \frac{1}{m^k}
      \quad\text{para todos \(x_1, \dotsc x_k\) distintos
                 y todos \(i_1, \dotsc i_k\)}
  \end{equation}
  Familias de funciones \emph{\foreignlanguage{english}{hash}} ideales al azar
  son \(k\)\nobreakdash-uniformes para todo \(k\).

\section{Una familia de funciones hash universal}
\label{sec:familia-hash-universal}

  Hay varias construcciones de familias
  \emph{\foreignlanguage{english}{hash}} universales,
  presentamos una de ellas.
  Una discusión más detallada
  y ejemplos adicionales
  se encuentra por ejemplo en el apunte de Erickson~%
    \cite{erickson19:_algorithms}.
  \begin{theorem}
    \label{theo:universal-multiplicative-hash}
    Considere un primo \(p\).
    Para cualquier par de enteros \((a, b)\)
    con \(1 \le a < p\) y \(0 \le b < p\)
    y \(m \le p\)
    defina la función \(h_{a, b} \colon \mathscr{U} \to [0, m - 1]\)
    mediante:
    \begin{equation}
      h_{a, b}(x)
        = ((a x + b) \bmod p) \bmod m
    \end{equation}
    Esta familia es universal.
  \end{theorem}
  \begin{proof}
    Fije enteros \(r, s, x, y\)
    con \(x \not\equiv y \pmod{p}\) y \(r \not\equiv s \pmod{p}\).
    El sistema lineal:
    \begin{align*}
      a x + b
        \equiv r \pmod{p} \\
      a y + b
        \equiv s \pmod{p}
    \end{align*}
    por el teorema chino de los residuos
    tiene solución única \((a, b)\) módulo \(p\)
    si y solo si \(x \not\equiv y \pmod{p}\).
    Se sigue que:
    \begin{equation*}
      \Pr_{a, b}[(a x + b) \bmod p = r
                    \wedge (a y + b) \bmod p = s]
        = \frac{1}{p (p - 1)}
    \end{equation*}
    por lo que:
    \begin{equation*}
      \Pr_{a, b}[h_{a, b}(x) = h_{a, b}(y)]
        = \frac{N}{p (p - 1)}
    \end{equation*}
    donde \(N\) es el número de pares ordenados \((r, s) \in \mathbb{Z}_p^2\)
    tales que \(r \ne s\) pero \(r \equiv s \pmod{m}\).
    Para cada \(r \in \mathbb{Z}_p\) fijo,
    hay a lo más \(\lfloor p / m \rfloor\) enteros \(s \in \mathbb{Z}_p\)
    tales que \(r \ne s\) pero \(r \bmod m = s \bmod m\).
    Como \(p\) es primo,
    \(\lfloor p / m \rfloor \le (p - 1) / m\),
    con lo que \(N \le p (p - 1) / m\).
    Concluimos
    \begin{equation*}
      \Pr_{a, b}[h_{a, b}(x) = h_{a, b}(y)]
        \le \frac{1}{m}
    \end{equation*}
    como queríamos demostrar.
  \end{proof}

  Para aplicar esto en la práctica,
  al momento de crear la tabla se eligen \(a\) y \(b\),
  uniformemente al azar y se usan durante su existencia.

% Sedgewick cites Knuth for number of probes as follows (a = n/m):
%
%				    Unsuccessful	     Successful
% Closed addressing (lists)	      1 + a/2		     (a + 1)/2
% Open addressing, linear probing  1/2 + 1/(2 (1 - a)^2)  1/2 + 1 / (2 (1 - a))
% Open addressing, double hashing    1 / (1 - a)	    -ln(1 - a)/a
%
% Unsuccessful for closed addressing is cut in half if lists are kept sorted
%
% Check/derive

\section{Direccionamiento cerrado}
\label{sec:hashing-cerrado}

  Esta situación es la más sencilla de analizar matemáticamente,
  usando herramientas simples de probabilidades.
  Nuestro desarrollo sigue al de Bogart, Stein y Drysdale~%
    \cite{bogart10:_discr_math_comp_sci}.
  Suponemos \emph{\foreignlanguage{english}{hashing}} ideal.

\subsection{Posiciones libres y ocupadas}
\label{sec:posiciones-libres-ocupadas}

  Sea \(X_i\) el número de objetos en la posición \(i\) de la tabla.
  Es claro que:
  \begin{equation*}
    \sum_i X_i
      = n
  \end{equation*}
  Por linealidad sabemos entonces:
  \begin{equation*}
    \sum_i \Exp[ X_i ]
      = n
  \end{equation*}
  Por simetría,
  \(\Exp[ X_i ]\) no depende de \(i\):
  \begin{equation*}
    \Exp[ X_i ]
      = \frac{n}{m}
  \end{equation*}
  Hemos demostrado:
  \begin{theorem}
    \label{theo:hash-expected}
    Con \emph{\foreignlanguage{english}{hashing}} ideal,
    al almacenar \(n\) objetos a una tabla de tamaño \(m\),
    el número esperado de objetos en cada posición es \(n/m\).
  \end{theorem}

  Después de agregar un objeto a la tabla,
  la probabilidad que la posición \(i\) esté vacía es \(1 - 1/m\).
  Después de \(n\) objetos agregados a la tabla,
  la probabilidad que siga vacío es \((1 - 1/m)^n\)
  (son \(n\) intentos independientes).
  Consideremos la misma secuencia de objetos,
  y sea \(X_i = 1\) si la posición \(i\) está libre,
  \(X_i = 0\) en caso contrario.
  El número de posiciones libres es:
  \begin{equation*}
    \sum_i X_i
  \end{equation*}
  Nuevamente por linealidad,
  como \(\Exp[ X_i ] = (1 - 1/m)^n\),
  tenemos:
  \begin{align*}
    \Exp\left[ \sum_i X_i \right]
      &= \sum_i \Exp[X_i] \\
      &= \sum_i \left( 1 - \frac{1}{m} \right)^n \\
      &= m \left( 1 - \frac{1}{m} \right)^n
  \end{align*}
  Hemos demostrado:
  \begin{theorem}
    \label{theo:hash-expected-free}
    Al hashear \(n\) objetos a una tabla de \(m\) ubicaciones,
    el número esperado de ubicaciones vacías es \(m (1 - 1/m)^n\).
  \end{theorem}
  Incidentalmente,
  si \(n = m\),
  la fracción esperada de espacios libres es \((1 - 1/m)^m\).
  Cuando \(n \to \infty\),
  por el límite clásico del lema~\ref{lem:classic-limit}
  cuando \(n = m\) y \(m \to \infty\),
  la fracción de espacios libres es \(\mathrm{e}^{-1}\),
  esperamos \(m/\mathrm{e}\) espacios libres.

\subsection{Problema del coleccionista de cupones}
\label{sec:coupon-collector}

  Es de interés calcular el número de objetos requeridos
  para llenar la tabla de tamaño \(m\).
  Esto se conoce como el \emph{problema del coleccionista de cupones}
  (\emph{\foreignlanguage{english}{coupon collector problem}},
   alguien recibe cupones elegidos al azar entre \(m\)
   y busca armar la colección completa).
  Planteamos el problema como ir llenando sucesivamente \(k\) posiciones,
  estamos interesados en ir de \(k\) llenas a \(k + 1\).
  Como los cupones llegan al azar,
  si hay \(k\) posiciones ocupadas la probabilidad de tener éxito
  (llenar una libre)
  en cada intento será:
  \begin{equation*}
    p_k
      = \frac{m - k}{m}
  \end{equation*}
  por lo que el número esperado de pasos
  requeridos en esta etapa es:
  \begin{equation*}
    \frac{1}{p_k}
      = \frac{m}{m - k}
  \end{equation*}
  Por la linealidad del valor esperado
  el valor esperado del número de pasos total \(T\) es:
  \begin{align*}
    \Exp[T]
      &=    \sum_{0 \le k \le m - 1} \frac{1}{p_k} \\
      &=    m \sum_{0 \le k \le m - 1} \frac{1}{m - k} \\
      &=    m \sum_{1 \le k \le m} \frac{1}{k} \\
      &=    m H_m \\
      &\sim m \ln m
  \end{align*}
  Acá hemos usado la definición de números harmónicos:
  \begin{equation*}
    H_n
      = \sum_{1 \le k \le n} \frac{1}{k}
  \end{equation*}
  Se sabe
  (ver por ejemplo el apunte de Fundamentos de Informática~%
    \cite[capítulo~18]{brand17:_fundamentos_informatica})
  que:
  \begin{equation*}
    H_n
      = \ln n + \gamma + O\left( \frac{1}{n} \right)
  \end{equation*}
  donde \(\gamma \approx 0,57721\) es la constante de Euler.

  Para la varianza,
  como el número de cupones recibidos en cada paso
  son variables aleatorias independientes:
  \begin{align*}
    \var[T]
      &=   \var[t_1] + \var[t_2] + \dotsb + \var[t_{m - 1}] \\
      &=   \frac{1 - p_1}{p_1^2}
             + \frac{1 - p_2}{p_2^2}
             + \dotsb
             + \frac{1 - p_{m - 1}}{p_{m - 1}^2} \\
      &\le \frac{m^2}{m^2}
             + \frac{m^2}{(m - 1)^2}
             + \dotsb
             + \frac{m^2}{1^2} \\
      &=   m^2 \left(
                 \frac{1}{1^2} + \frac{1}{2^2} + \dotsb + \frac{1}{m^2}
               \right) \\
      &<   \frac{\pi^2 m^2}{6}
  \end{align*}
  Esto último por la solución al famoso problema de Basilea
  (ver por ejemplo el apunte de Fundamentos de Informática~%
    \cite[capítulo~18]{brand17:_fundamentos_informatica}):
  \begin{equation*}
    \sum_{k \ge 1} \frac{1}{k^2}
      = \frac{\pi^2}{6}
  \end{equation*}

\subsection{Número esperado de colisiones}
\label{sec:numero-colisiones}

  El número de posiciones de la tabla que contienen más de un objeto
  es el número total \(n\) de objetos menos el número de ubicaciones ocupadas.
  Por los teoremas~\ref{theo:hash-expected} y~\ref{theo:hash-expected-free}
  tenemos:
  \begin{align*}
    \Exp[ \text{colisiones} ]
      &= n - \Exp[ \text{posiciones ocupadas} ] \\
      &= n - m + \Exp[ \text{posiciones libres} ]
  \end{align*}
  y tenemos otro teorema:
  \begin{theorem}
    \label{theo:hash-expected-collisions}
    Usando \emph{\foreignlanguage{english}{hashing}} ideal,
    al agregar \(n\) objetos a una tabla de tamaño \(m\),
    el número esperado de colisiones es:
    \begin{equation*}
      n - m + m \left( 1 - \frac{1}{m} \right)^n
    \end{equation*}
  \end{theorem}
  Esto va contra la intuición:
  en promedio,
  al agregar \(m\) objetos a una tabla de tamaño \(m\) cuando \(m\) es grande,
  esperamos un objeto en cada posición,
  pero una fracción de \(\mathrm{e}^{-1} = 0,3679\) de posiciones queda libre,
  y hay \(m \mathrm{e}^{-1}\) colisiones.

\subsection{Largo esperado de la lista más larga}
\label{sec:lista-mas-larga}

  Suponiendo que los elementos que dan en una posición dada de la tabla
  se almacenan en una lista,
  tiene sentido preguntarse por el peor caso de esta.
  Es claro que lo peor que puede ocurrir es que todos den en la misma posición,
  con lo que el peor largo es directamente \(n\).
  Pero esto es extremadamente poco probable,
  no es una medida realista.
  Usar listas es una buena opción,
  en promedio
  para tablas no demasiado llenas las listas serán cortas
  e importa que sean simples de administrar.
  Otras posibilidades son usar
  a su vez tablas \emph{\foreignlanguage{english}{hash}}
  con direccionamiento cerrado,
  que se hacen crecer cuando se llenen demasiado.

  Nos interesa acotar el valor esperado del tamaño de la lista más larga,
  cuando se ingresan \(n\) objetos a una tabla de tamaño \(n\).
  Usaremos la fórmula de Stirling~\eqref{eq:Stirling}:
  \begin{equation*}
    x!
      = \left( \frac{x}{\mathrm{e}} \right)^x
          \sqrt{2 \pi x}
          \mathrm{e}^{r(x)}
  \end{equation*}
  donde las cotas de Robbins~\eqref{eq:Robbins} son:
  \begin{equation*}
    \frac{1}{12 x + 1}
      < r(x)
      < \frac{1}{12 x}
  \end{equation*}
  Nos dice que,
  en términos gruesos,
  \((x/\mathrm{e})^x\) es una buena aproximación para \(x!\).
  Por lo demás,
  el factor de error es casi uno,
  podemos omitirlo.

  Partiremos por una cantidad que sabemos cómo calcular:
  sea \(H_{i t}\) el evento que \(t\) claves dan en la posición \(i\).
  Entonces \(\Pr[ H_{i t} ]\) es la probabilidad de exactamente \(t\) éxitos
  en \(n\) intentos independientes con probabilidad de éxito \(1/n\),
  o sea:
  \begin{equation}
    \label{eq:Pr[Hit]}
    \Pr[ H_{i t} ]
      = \binom{n}{t}
         \left( \frac{1}{n} \right)^t \left( 1 - \frac{1}{n} \right)^{n - t}
  \end{equation}
  Sea \(M_t\) el evento que la lista más larga sea de largo \(t\).
  Tenemos:
  \begin{lemma}
    \label{lem:Mt}
    Sea \(M_t\) el evento que la cola más larga al hashear \(n\) objetos
    en una tabla de tamaño \(n\) sea de largo \(t\),
    y sea \(H_{i t}\) el evento que exactamente \(t\) claves
    dan en la posición \(i\).
    Entonces:
    \begin{equation*}
      \Pr[ M_t ]
        \le n \Pr[H_{1 t} ]
    \end{equation*}
  \end{lemma}
  \begin{proof}
    Analizaremos el caso de la cola más larga en cualquier posición \(i\);
    por simetría la posición da lo mismo,
    podemos plantear el resultado para la posición \num{1},
    como hace el enunciado.

    Sea \(M_{i t}\) el evento que el largo máximo sea \(t\)
    y que está en la posición \(i\).
    Es claro que \(M_{i t} \subseteq H_{i t}\),
    por lo que \(\Pr[M_{i t}] \le \Pr[H_{i t}]\).
    Además:
    \begin{equation*}
      M_t
        = M_{1 t} \cup M_{2 t} \cup \dotsb \cup M_{n t}
    \end{equation*}
    Por simetría,
    \(\Pr[H_{i t}]\) son todas iguales,
    y por la cota de la unión
    resulta lo prometido.
  \end{proof}
  Podemos usar~\eqref{eq:Pr[Hit]} y el lema~\ref{lem:binomial}
  para acotar:
  \begin{align*}
    \Pr[H_{i t}]
      &=   \binom{n}{t}
             \left( \frac{1}{n} \right)^t
             \left(1 - \frac{1}{n} \right)^{n - t} \\
      &\le \frac{n^n}{t^t (n - t)^{n - t}}
             \left( \frac{1}{n} \right)^t
             \left(1 - \frac{1}{n} \right)^{n - t} \\
    \intertext{Simplificando,
               como \((1 - 1/n)^{n - t} < 1\):}
    \Pr[H_{i t}]
      &\le \frac{n^n}{t^t (n - t)^{n - t}}
              \left( \frac{1}{n} \right)^t \\
      &=   \left( \frac{n}{n - t} \right)^{n - t} \frac{1}{t^t} \\
    \intertext{Como \(\ln(1 + u) < u\) si \(u > 0\),
               exponenciando es \((1 + 1/x)^x < \mathrm{e}\):}
      &=   \left(
              \left( 1 + \frac{t}{n - t} \right)^{\frac{n - t}{t}}
            \right)^t \frac{1}{t^t} \\
      &\le \frac{\mathrm{e}^t}{t^t}
  \end{align*}
  Juntando los anteriores resultados:
  \begin{lemma}
    \label{lem:Pr[Mt]}
    \begin{equation*}
      \Pr[ M_t ]
        \le \frac{n \mathrm{e}^t}{t^t}
    \end{equation*}
  \end{lemma}
  Tenemos una cota,
  que podemos usar para acotar la cantidad de interés,
  el largo promedio de la lista más larga:
  \begin{equation}
    \label{eq:hash-longest-length-1}
    \sum_{0 \le t \le n} t \Pr[M_t]
  \end{equation}
  Pero hay un problema:
  la cota del lema~\ref{lem:Pr[Mt]}
  da valores absurdos para \(t\) pequeño.
  Por ejemplo,
  para \(t = 1\) da \(\Pr[M_1] \le n \mathrm{e}\),
  lo que está totalmente fuera de proporción.
  Para valores grandes de \(t\),
  el lema indica que \(\Pr[M_t]\) es pequeño,
  y la cota resulta ajustada.
  El truco para estimar la suma~\eqref{eq:hash-longest-length-1}
  es dividir en una suma sobre valores \textquote{chicos} y \textquote{grandes},
  y estimar las sumas por separado.
  Suponiendo que \(n \ge 3\)
  (para evitar logaritmos de números negativos)
  cortamos en \(t^* = \lfloor 5 \ln n / \ln \ln n \rfloor\):
  \begin{equation}
    \label{eq:hash-longest-length-cut}
    \sum_{0 \le t \le n} t \Pr[M_t]
      = \sum_{0 \le t \le t^*} t \Pr[M_t]
          + \sum_{t^* < t \le n} t \Pr[M_t]
  \end{equation}

  Para la primera suma,
  observamos que \(t \le t^*\),
  de manera que:
  \begin{align}
    \sum_{0 \le t \le t^*} t \Pr[M_t]
      &\le \sum_{0 \le t \le t^*} t^* \Pr[M_t] \notag \\
      &=   t^* \sum_{0 \le t \le t^*} \Pr[M_t] \notag \\
      &\le t^*
           \label{eq:hash-longest-length-small}
  \end{align}
  Lo último resulta simplemente porque las probabilidades de eventos disjuntos
  suman a lo más a uno.

  Para la segunda usamos nuestro lema~\ref{lem:Pr[Mt]}.
  Dijimos que esperamos que \(\Pr[M_t]\) acá sea pequeño,
  y la cota disminuye.
  Por el lema,
  para \(t \ge t^*\) después de algún álgebra tediosa vemos que:
  \begin{equation*}
    \Pr[M_t]
      \le \frac{1}{n^2}
  \end{equation*}
  Podemos entonces acotar la suma para \(t\) mayores:
  \begin{align}
    \sum_{t^* < t \le n} t \Pr[M_t]
      &\le \sum_{t^* < t \le n} n \frac{1}{n^2} \notag \\
      &=   \sum_{t^* < t \le n} \frac{1}{n} \notag \\
      &\le 1
           \label{eq:hash-longest-length-large}
  \end{align}
  Combinando~\eqref{eq:hash-longest-length-small}
  con~\eqref{eq:hash-longest-length-large}
  obtenemos:
  \begin{equation}
    \label{eq:hash-longest-length-estimate}
    \sum_{0 \le t \le n} t \Pr[M_t]
      \le 5 \ln n / \ln \ln n + 1
      = O( \ln n / \ln \ln n )
  \end{equation}

  La elección de \(t^*\) para cortar la suma parece magia.
  Nos pusimos una cota para \(\Pr[M_t]\) cercana a \(1/n^2\),
  que da la cómoda cota~\eqref{eq:hash-longest-length-large}.
  Esto da la ecuación:
  \begin{equation}
    \label{eq:relacion-base-corte}
    \frac{n \mathrm{e}^t}{t^t}
      = \frac{1}{n^2}
  \end{equation}
  Esto no puede resolverse para \(t\),
  pero es más o menos similar a
  (tome logaritmos y omita constantes molestas):
  \begin{align}
    t^t
      &= n \notag \\
    t
      &= \frac{\ln n}{\ln t} \label{eq:equacion-corte}
  \end{align}
  Tomando una aproximación inicial \(t = \ln n\),
  iterar una vez nos da:
  \begin{equation}
    \label{eq:aproximacion-corte}
    t
      \approx \frac{\ln n}{\ln \ln n}
  \end{equation}
  Sospechamos que la solución
  de nuestra ecuación~\eqref{eq:relacion-base-corte} es algo como:
  \begin{equation}
    \label{eq:corte-aproximado}
    t
      = c \frac{\ln n}{\ln \ln n}
  \end{equation}
  y un poco de experimentación muestra que \(c = 5\)
  da una división manejable en~\eqref{eq:hash-longest-length-cut}.

\subsection{Usar más de una función de hashing}
\label{sec:two-choices}

  Una idea sorprendentemente efectiva
  es usar varias funciones
  de \emph{\foreignlanguage{english}{hashing}} independientes
  y elegir aquella posición en la cual la lista es más corta.
  Mitzenmacher~%
   \cite{mitzenmacher96:_power_two_choic_random_load_balan}
  demostró que al usar dos el largo de la lista más larga
  es \(\Theta(\log \log n)\) con alta probabilidad.
  Esto es una reducción exponencial
  frente a la cota~\eqref{eq:hash-longest-length-estimate}.
  El desarrollo es complejo,
  no lo repetiremos acá.
  Una revisión reciente de este resultado y afines,
  junto a una discusión de aplicaciones,
  dan Mitzenmacher, Richa y Sitaraman~%
   \cite{mitzenmacher01:_power_two_random_choic}.
  Es de interés práctico porque disminuye el tiempo de búsqueda,
  además que las búsquedas en ambas listas pueden paralelizarse fácilmente.

\subsection{Análisis de direccionamiento cerrado}
\label{sec:analisis-cerrado}

  Consideremos los programas para direccionamiento cerrado,
  listado~\ref{lst:hashing-closed}.
  Usamos listas sin ordenar para cada casillero,
  por ser lo más simple.
  Arbitrariamente usamos 10267
  (un número primo)
  como número de casilleros.
  \lstinputlisting[float,
                   firstline = 4,
                   caption = {Hashing cerrado},
                   label = lst:hashing-closed]
                  {code/hashing-closed.c}
  Usaremos como medida de costo los nodos considerados en cada caso.
  Como antes,
  sea \(m\) el tamaño de la tabla y \(n\) el número de elementos que contiene,
  y suponemos \emph{\foreignlanguage{english}{hashing}} ideal.
  Abreviaremos
  el \emph{factor de carga} de la tabla por:
  \begin{equation}
    \label{eq:def-alpha}
    \alpha
      = \frac{n}{m}
  \end{equation}

\subsubsection{Inserción}
\label{sec:hashing-cerrado-insercion}

  Es claro
  (comparar con el listado~\ref{lst:hashing-closed})
  que el número de posiciones consideradas es constante
  (\(O(1)\)),
  no verificamos que el elemento a insertar no esté ya presente.

\subsubsection{Búsqueda exitosa}
\label{sec:hashing-cerrado-busqueda-exitosa}

  Sea la secuencia en que se ingresaron los objetos
  \(\langle x_i \rangle\),
  que suponemos todos diferentes entre sí.
  Buscamos hallar el costo promedio de búsqueda de los objetos,
  suponiendo que cada uno es buscado con la misma frecuencia.
  Como insertamos nuevos objetos a cada lista al comienzo,
  debemos comparar y pasar sobre los insertados después.

  Sean \(X_{i j}\) variables indicadoras,
  \(X_{i j} = [h(x_i) = h(x_j)]\).
  Por la suposición sobre la función \(h\)
  para todo \(i \ne j\) tenemos:
  \begin{equation*}
    \Exp[X_{i j}]
      = \frac{1}{m}
  \end{equation*}
  El valor esperado del número de posiciones revisadas
  es:
  \begin{align}
    \Exp\left[
          \frac{1}{n} \sum_{1 \le i \le n}
                        \left(
                          1 + \sum_{i + 1 \le j \le n} X_{i j}
                        \right)
          \right]
      &=	   \frac{1}{n} \sum_{1 \le i \le n}
                          \left(
                            1 + \sum_{i + 1 \le j \le n} \Exp[X_{i j}]
                          \right) \notag \\
      &=	   \frac{1}{n} \sum_{1 \le i \le n}
                          \left(
                            1 + \frac{1}{m} (n - i)
                          \right) \notag \\
      &=	   1 + \frac{1}{m n}
                  \left(
                    \sum_{1 \le i \le n} n - \sum_{1 \le i \le n} i
                  \right) \notag \\
      &=	   1 + \frac{n}{2 m} - \frac{1}{2 m} \\
      &=	   1 + \frac{\alpha}{2} - \frac{\alpha}{2 n} \notag \\
      &\sim 1 + \frac{\alpha}{2}
         \label{eq:hashing-closed-success}
  \end{align}

\subsubsection{Búsqueda fallida}
\label{sec:hashing-cerrado-busqueda-fallida}

  La situación es la misma anterior,
  solo que en este caso debemos revisar la lista completa:
  \begin{align}
    \Exp\left[
          \frac{1}{n} \sum_{1 \le i \le n}
                        \left(
                          1 + \sum_{1 \le j \le n} X_{i j}
                        \right)
        \right]
      &=	   \frac{1}{n} \sum_{1 \le i \le n}
                          \left(
                            1 + \sum_{1 \le j \le n} \Exp[X_{i j}]
                          \right) \notag \\
      &=	   \frac{1}{n} \sum_{1 \le i \le n}
                          \left(
                            1 + \frac{1}{m} n
                          \right) \notag \\
      &=	   1 + \frac{n}{m} \notag \\
      &=	   1 + \alpha
         \label{eq:hashing-closed-failure}
  \end{align}

\subsection{Variantes de interés}
\label{sec:hash-closed-variants}

  Algunas variantes que vale la pena considerar es manejar las listas ordenadas
  (esto encarece la inserción de nuevos elementos,
   pero disminuye el costo de búsquedas fallidas),
  o utilizar estructuras más sofisticadas en cada casillero,
  como árboles binarios
  o derechamente nuevas tablas \emph{\foreignlanguage{english}{hash}}.
  Claro que por lo discutido antes,
  estas generalmente solo tienen sentido
  en caso que \(n\) sea substancialmente mayor a \(m\).

\section{Direccionamiento abierto}
\label{sec:hash-open}

  Una alternativa es usar posiciones libres de la tabla
  cuando ocurren colisiones.
  El análisis de Knuth~%
    \cite{knuth63:_notes_open_addressing}
  de direccionamiento abierto con prueba lineal es considerado por muchos
  como el nacimiento del análisis de algoritmos.
  Una discusión detallada de estas técnicas ofrecen Sedgewick y Flajolet~%
    \cite[capítulo~8]{sedgewick13:_introd_anal_algor}.

  El algoritmo es~\ref{alg:hash-open},
  \begin{algorithm}[ht]
    \DontPrintSemicolon\Indp

    \For{\(i \gets 0\) \KwTo \(m - 1\)}{
      \uIf{\(T[h_i(x)] = x\)}{
        \Return \(h_i(x)\) \;
      }
      \ElseIf{\(T[h_i(x)] = \varnothing\)}{
        \Return Absent \;
      }
    }
    \caption{Direccionamiento abierto}
    \label{alg:hash-open}
  \end{algorithm}
  supone una secuencia de funciones \emph{\foreignlanguage{english}{hash}}
  \(\langle h_0, h_1, \dotsc, h_{m - 1} \rangle\)
  tales que para todo \(x \in \mathscr{U}\)
  tenemos que
  \(\langle h_0(x), h_1(x), \dotsc, h_{m - 1}(x) \rangle\)
  es una permutación de \(\{ 0, 1, 2, \dotsc, m - 1 \}\).
  En otras palabras,
  para \(1 \le i \le m - 1\)
  las funciones \(h_i\) mapean \(x\) a posiciones diferentes de la tabla.

  Un modelo simple
  (pero un tanto irreal)
  es el que planteó Peterson~%
    \cite{peterson57:_addressing_random_access_storage}:
  suponiendo que las \(n\)~posiciones ocupadas están distribuidas al azar
  entre las \(m\)~casillas del arreglo,
  la probabilidad de que se requieran \(r\)~accesos
  al insertar el siguiente elemento es simplemente:
  \begin{equation*}
    p_r
      = \binom{m - r}{n - r + 1} \bigg/ \binom{m}{n}
  \end{equation*}
  (fije \(r\)~posiciones a ser revisadas,
   \(r - 1\)~ocupadas y \num{1}~libre;
   entre ellas elegimos las demás \(n - (r - 1)\)~posiciones ocupadas).
  El número esperado de posiciones revisadas al insertar un elemento nuevo es:
  \begin{align}
    \Exp[\text{inserción}]
      &= \sum_{1 \le r \le m} r p_r \notag \\
      &= \sum_{1 \le r \le m} r \binom{m - n}{n - r + 1} \bigg/ \binom{m}{n} \notag \\
      &= m + 1
           - \sum_{1 \le r \le m}
               (m + 1 - r) \binom{m - r}{n - r + 1} \bigg/ \binom{m}{n} \notag \\
      &= m + 1
           - \sum_{1 \le r \le m}
               (m + 1 - r) \binom{m - r}{n - r + 1} \bigg/ \binom{m}{n} \notag \\
      &= m + 1
           - \sum_{1 \le r \le m}
               (m - n) \binom{m + 1 - r}{m - n} \bigg/ \binom{m}{n} \notag \\
      &= m + 1
           - (m - n) \binom{m + 1}{m - n} \bigg/ \binom{m}{n} \notag \\
      &= m + 1
           - (m - n) \frac{m + 1}{m - n + 1} \notag \\
      &= \frac{m + 1}{m - n + 1} \notag \\
      &= \frac{1 + 1 / m}{1 - \alpha + 1 / m} \notag \\
      &\approx \frac{1}{1 - \alpha}
            \label{eq:open-hashing:insert}
  \end{align}
  Exactamente las mismas posiciones se revisan en una búsqueda fallida.

  Una búsqueda exitosa revisa las mismas posiciones
  consideradas al insertar el elemento del caso:
  \begin{align*}
    \Exp[\text{exitosa}]
      &= \frac{1}{n}
           \sum_{0 \le k \le n - 1} \frac{m + 1}{m - k + 1} \notag \\
      &= \frac{m + 1}{n}
            \left(
              \sum_{1 \le k \le m + 1} \frac{1}{k}
                - \sum_{1 \le k \le m - n + 1} \frac{1}{k}
            \right) \notag \\
      &= \frac{m + 1}{n} (H_{m + 1} - H_{m - n + 1}) \notag \\
      &\approx \frac{1}{\alpha} \ln \frac{1}{1 - \alpha}
            \label{eq:open-hashing:success}
  \end{align*}

  El supuesto de secuencias de intentos independientes es irreal,
  en la práctica debemos usar secuencias más manejables.
  Tal vez la más sencilla es \emph{prueba lineal},
  usar una única función \(h\)
  e intentar las posiciones \((h(x) + i) \bmod m\).
  Además de ser simple,
  tiene la ventaja que intenta posiciones consecutivas,
  lo que funciona bien en arquitecturas de memoria con caché.
  Por otro lado,
  esto tiene el efecto de apiñar las posiciones ocupadas
  (una secuencia de \(k\) posiciones ocupadas tiene una probabilidad
   de \(k / m\) de alargarse
   porque el nuevo elemento cae en alguna de sus posiciones,
   lo que en realidad es aún peor ya que puede llegar a llenar posiciones
   entre secuencias),
  alargando las búsquedas al aumentar \(\alpha\).
  Propuestas para evitar este efecto son \emph{doble hashing},
  que calcula una nueva función de hashing independiente \(h_1(x)\)
  y revisa las posiciones \((h(x) + i h_1(x)) \bmod m\);
  o usar \emph{prueba cuadrática},
  que intenta \((h(x) + i^2) \bmod m\).
  Esto igual presenta apiñamiento
  si hay claves del mismo valor de \(h\)
  (le llaman secundario).
  Claro que esto solo revisa la mitad de las posiciones del arreglo
  si su tamaño es primo
  (recuerde que \(\mathbb{Z}_p^\times\) es cíclico,
   solo la mitad de sus elementos
   --los que son potencias pares del generador--
   son cuadrados).
  Otra opción es \emph{prueba binaria},
  que usa \(h_i(x) = h(x) \oplus i\),
  donde \(\oplus\) es la operación de o exclusivo bit a bit.
  Tiene la ventaja de revisar un rango de entradas contiguas
  antes de pasar a otra
  (o sea,
   recorre líneas de caché completas),
  pero no las recorre secuencialmente
  (evitando apiñamiento).
  Detalles da Erickson~%
    \cite{erickson19:_algorithms}.

\subsection{Análisis de direccionamiento abierto}
\label{sec:analisis-hashing-abierto}

  Como antes,
  sea \(m\) el tamaño de la tabla y \(n\) el número de elementos que contiene.
  Nuevamente la medida de costo es el número de posiciones revisadas,
  y promediamos sobre todos los elementos con probabilidad uniforme.
  Debemos hallar un espacio vacío para el elemento a insertar
  en caso que \(h(x)\) ya esté ocupado.
  La suposición más simple es que cada intento sucesivo
  considera las posiciones aún no consideradas en forma uniforme,
  esencialmente \emph{\foreignlanguage{english}{hashing}} ideal.

\subsubsection{Búsqueda fallida}
\label{sec:hashing-abierto-busqueda-fallida}

  Sea \(X\) el número de intentos al buscar \(x\),
  y sea \(A_i\) el evento que hay \(i\) intentos
  y la posición intentada en el intento \(i\) está ocupada.
  Entonces:
  \begin{align*}
    \Pr[X \ge i]
      &= \Pr\left[ \bigcap_{1 \le j \le i - 1} A_j \right] \\
      &= \Pr[A_1]
           \cdot \Pr[A_2 \mid A_1]
           \cdot \Pr[A_3 \mid A_1 \cap A_2]
           \cdot \dotsm
           \cdot \Pr[A_{i - 1} \mid A_1 \cap A_2 \cap \dotsb \cap A_{i - 2}]
  \end{align*}
  En el intento \(i\) ya descartamos \(i - 1\) posiciones a revisar,
  ocupadas por los \(i - 1\) elementos que están en ellas,
  quedan \(m - i + 1\) posiciones de las cuales hay \(n - i + 1\) ocupadas,
  y estamos eligiendo uniformemente al azar entre ellas:
  \begin{equation*}
    \Pr[A_i]
      = \frac{n - i + 1}{m - i + 1}
  \end{equation*}
  Como son intentos independientes:
  \begin{align*}
    \Pr[X \ge i]
      &=   \prod_{0 \le j \le i - 2} \frac{n - j}{m - j} \\
      &\le \left( \frac{n}{m} \right)^{i - 1} \\
      &=   \alpha^{i - 1}
  \end{align*}
  Por lo tanto:
  \begin{align}
    \Exp[X]
      &=   \sum_{i \ge 1} \Pr[X \ge i] \notag \\
      &\le \sum_{i \ge 1} \alpha^{i - 1} \notag \\
      &=   \sum_{i \ge 0} \alpha^i \notag \\
      &=   \frac{1}{1 - \alpha}
         \label{eq:hashing-open-failure}
  \end{align}

  Insertar un nuevo elemento en la tabla es una búsqueda fallida
  para hallar su lugar.

\subsubsection{Búsqueda exitosa}
\label{sec:hashing-abierto-busqueda-exitosa}

  Las posiciones revisadas al insertar \(x_{i + 1}\)
  (y revisadas al buscarlo luego)
  son exactamente las mismas
  que las búsquedas fallidas que llevaron a insertar ese elemento,
  cuando la tabla tenía \(i\) elementos,
  o sea,
  se revisaron a lo más \(1 / (1 - i / m) = m / (m - i)\)~posiciones.
  El promedio cumple:
  \begin{align}
    \frac{1}{n} \sum_{0 \le i \le n - 1} \frac{m}{m - i}
      &=   \frac{m}{n} \sum_{0 \le i \le n - 1} \frac{1}{m - i} \notag \\
      &=   \frac{1}{\alpha} (H_m - H_{m - n}) \notag \\
      &\le \frac{1}{\alpha} \int_{m - n}^m \frac{\mathrm{d} x}{x} \notag \\
      &=   \frac{1}{\alpha} \ln \frac{m}{m - n} \notag \\
      &=   \frac{1}{\alpha} \ln \frac{1}{1 - \alpha}
         \label{eq:hashing-open-success}
  \end{align}

\subsection{Resumen del análisis}
\label{sec:resumen-del-analisis}

  El cuadro~\ref{tab:hashing-analysis-summary}
  resume los resultados anteriores.
  El detalle se basa en el desarrollo de CLRS~%
    \cite{cormen09:_CLRS}.
  Sedgewick y Flajolet~%
    \cite{sedgewick13:_introd_anal_algor}
  analizan estas mismas situaciones usando el método simbólico,
  obteniendo resultados más ajustados y también varianzas.
  \begin{table}[ht]
    \centering
    \begin{tabular}{l*{2}{>{\rule{0pt}{2em}\(\displaystyle{}}c<{\)}}}
      \multicolumn{1}{c}{\textbf{Técnica}}
        & \multicolumn{1}{c}{\textbf{Exitosa}}
        & \multicolumn{1}{c}{\textbf{Fallida}} \\
      \hline
      Direccionamiento cerrado
        & 1 + \frac{\alpha}{2} & 1 + \alpha \\
      Direccionamiento abierto
        & \frac{1}{\alpha} \ln \frac{1}{1 - \alpha} & \frac{1}{1 - \alpha}
      \end{tabular}
    \caption{Resumen de posiciones revisadas en hashing}
    \label{tab:hashing-analysis-summary}
  \end{table}

% To do: Cucoo hashing...

\section{Otras aplicaciones}
\label{sec:otras-aplicaciones}

  La misma idea tiene otras aplicaciones.
  Exploraremos un par de ellas.

\subsection{Filtro de Bloom}
\label{sec:filtro-de-bloom}

  Hay situaciones en las que interesa
  representar un conjunto grande en forma compacta.
  Si aceptamos la posibilidad de errores,
  en particular \emph{falso positivo}
  (respondemos \textquote{sí},
   pero la respuesta correcta es \textquote{no}),
  una opción es un filtro de Bloom~%
   \cite{bloom70:_space_time_trade_hash_codin_allow_error}.
  Una discusión detallada,
  particularmente aplicaciones a sistemas distribuidos,
  dan Broder y Mitzenmacher~%
    \cite{broder04:_network_appl_bloom_filters}.
  La idea es representar el conjunto mediante un arreglo de \(m\) bits,
  para agregar el elemento \(x\)
  aplicarle \(k\) funciones \emph{\foreignlanguage{english}{hash}}
  independientes \(h_i\),
  poniendo en \num{1} los bits respectivos.
  Ver si \(x\) pertenece al conjunto es revisar los bits respectivos,
  si todos son \num{1}
  probablemente pertenece;
  si alguno es \num{0},
  definitivamente no pertenece.

  Interesa derivar la probabilidad de error si hay \(n\) elementos del conjunto
  representado con \(m\) bits
  y \(k\) funciones \emph{\foreignlanguage{english}{hash}}.
  Para ello primero derivaremos la probabilidad
  que luego de insertar \(n\) elementos,
  un bit dado siga en \num{0}.
  Esto corresponde a \(n k\) intentos fallidos de apuntarle a ese bit,
  con probabilidad de éxito \(1 / m\) en cada intento,
  o sea:
  \begin{align*}
    \Pr[\text{bit \(i\) en cero luego de agregar \(n\)}]
      &=	   \left( 1 - \frac{1}{m} \right)^{n k} \\
      &=	   \left( \left( 1 - \frac{1}{m} \right)^m \right)^{n k / m} \\
      &\approx \mathrm{e}^{- n k / m}
  \end{align*}

  Un falso positivo,
  por otro lado,
  corresponde a tener \(k\) éxitos
  (encontrar bit en \num{1})
  en \(k\) intentos independientes.
  La probabilidad de éxito en cada intento viene del punto anterior:
  \begin{align*}
    \Pr[\text{falso positivo}]
      &=	    \prod_{1 \le i \le k}
                \Pr[\text{bit \(h_i(x)\) es \num{1}}] \\
      &=	    \left(
                  1 - \left( 1 - \frac{1}{m} \right)^{n k}
                \right)^k \\
      &\approx \left( 1 - \mathrm{e}^{-n k / m} \right)^k
  \end{align*}

  Los filtros de Bloom balancean varios efectos contrarios:
  queremos usar poca memoria
  (\(m\) pequeño,
   pero eso hace más probables las colisiones,
   falsos positivos),
  poco tiempo de cómputo
  (\(k\) pequeño,
   usa menos cómputo pero hace más probables las colisiones).
  Un análisis aproximado es como sigue.
  Sea \(f(k)\) la función que nos interesa,
  queremos minimizar
  la probabilidad de falso positivo para \(m\) y \(n\) fijos,
  donde suponemos \(m\) lo suficientemente grande
  para poder aplicar la aproximación por la exponencial.
  Minimizar \(f(k)\) es minimizar \(\ln f(k)\):
  \begin{align*}
    \frac{\mathrm{d}}{\mathrm{d} k} \ln f(k)
      &= \frac{\mathrm{d}}{\mathrm{d} k}
           k \ln \left( 1 - \mathrm{e}^{- n k / m} \right) \\
      &= \ln \left( 1 - \mathrm{e}^{- n k / m} \right)
           + \frac{n k}{m}
               \cdot \frac{\mathrm{e}^{- n k / m}}
                          {1 - \mathrm{e}^{- n k / m}}
  \end{align*}
  Igualando a cero la derivada,
  hallamos que el mínimo se da con \(k = \frac{m}{n} \ln 2\),
  y este mínimo es global.
  En el mínimo,
  la probabilidad de falso positivo es:
  \begin{equation*}
    \left( \frac{1}{2} \right)^k
      = 0,6185^{m/n}
  \end{equation*}
  Conforme \(m\) crece respecto de \(n\),
  disminuye la tasa de falsos positivos.

  Un problema de los filtros de Bloom
  es que podemos \emph{agregar} elementos al conjunto,
  pero no \emph{eliminarlos}.
  Una posibilidad es no usar un simple bit,
  sino un contador.
  Este es el caso que discuten Fan et al~%
    \cite{fan00:_summary_cache}.

\subsection{Contar elementos distintos}
\label{sec:contar-distintos}

  En muchas aplicaciones hay un gran flujo de elementos,
  e interesa estimar cuántos elementos distintos hay.
  Por ejemplo,
  en páginas web interesa saber el número de visitantes diferentes.
  Una posibilidad es registrarlos y procesar la lista.
  Acá discutiremos una posibilidad más simple
  (aunque menos precisa).

  La idea básica es calcular una función \emph{\foreignlanguage{english}{hash}}
  de cada elemento e ir recordando el mínimo visto.
  Resulta que el mínimo tiene una relación simple
  con el número de elementos diferentes.

  \begin{theorem}
    \label{theo:minimum-iid}
    Si \(X_1, X_2, \dotsc, X_n\) son variables aleatorias
    independientes idénticamente distribuidas
    con función de distribución cumulativa \(F\),
    la función cumulativa del mínimo de los \(X_i\)
    es:
    \begin{equation*}
      F_{\text{min}}(y)
        = 1 - (1 - F(y))^n
    \end{equation*}
  \end{theorem}
  \begin{proof}
    Nos interesa:
    \begin{align*}
      F_{\text{min}}(y)
        &= \Pr[ \min\{ X_1, \dotsc, X_n \} \le y ] \\
        &= 1 - \Pr[ \min\{ X_1, \dotsc, X_n \} > y ] \\
        &= 1 - \Pr[ X_1 > y \wedge \dotsm \wedge X_n > y ] \\
        &= 1 - \Pr[ X_1 > y ] \cdot \dotsm \cdot \Pr[ X_n > y ] \\
        &= 1 - ( \Pr[ X_1 > y ] )^n \\
        &= 1 - (1 - \Pr[ X_1 \le y ] )^n \\
        &= 1 - (1 - F(y))^n
    \end{align*}
    \qedhere
  \end{proof}
  En particular,
  si \(X_i \sim \mathrm{\boldsymbol{\mathsf{U}}}(0, 1)\)
  (distribución uniforme continua)
  la función cumulativa es simplemente \(F(y) = y\) para \(0 \le y \le 1\)
  y resulta:
  \begin{equation*}
    F_{\text{min}}(y)
      = 1 - (1 - y)^n
  \end{equation*}
  De acá la función de densidad de probabilidad es la derivada:
  \begin{equation*}
    f_{\text{min}}(y)
      = n (1 - y)^{n - 1}
  \end{equation*}
  Es rutina calcular:
  \begin{align*}
    \Exp[\min\{ X_1, \dotsc, X_n \}]
      &= \int_0^1 y f_{\text{min}}(y) \, \mathrm{d} y \\
      &= \frac{1}{n + 1} \\
    \var[\min\{ X_1, \dotsc, X_n \}]
      &= \int_0^1
           \left( y - \frac{1}{n + 1} \right)^2
             f_{\text{min}}(y) \, \mathrm{d} y \\
      &= \frac{n}{(n + 1)^2 (n + 2)}
  \end{align*}
  O sea,
  registrando el mínimo de \(h(x)\)
  (normalizado a \([0, 1]\))
  para los elementos conforme llegan,
  obtenemos una estimación de \(n\),
  el número de elementos distintos observados.
  Lamentablemente la desviación es bastante grande,
  pero podemos usar el promedio de varias funciones para mejorar la estimación.

\section*{Ejercicios}
\label{sec:exercises-27-previa}

  \begin{enumerate}
  \item
    Las direcciones IPv4 se escriben tradicionalmente
    en la forma \(a.b.c.d\),
    donde \(a, b, c, d\) están entre \num{0} y \num{255}.
    Se propone
    la siguiente familia de funciones \emph{\foreignlanguage{english}{hash}}:
    se elige un primo \(p\),
    y se eligen al azar \(\alpha, \beta, \gamma, \delta \in \mathbb{Z}_p\),
    y:
    \begin{equation*}
      h(a, b, c, d)
        = (\alpha a + \beta b + \gamma c + \delta d) \bmod p
    \end{equation*}
    \begin{enumerate}
    \item
      Discuta porqué algunos miembros de la familia son malos.
      En particular,
      vea los casos \(\alpha = 1, \beta = \gamma = \delta = 0\)
      y	 \(\alpha = \beta = \gamma = 0, \delta = 1\).
    \item
      ¿Es universal la familia completa?
    \item
      ¿Qué pasa con versiones restringidas,
      como \(\alpha, \beta, \gamma, \delta \ne 0\)?
      ¿Al menos un coeficiente no cero?
      ¿Exactamente un coeficiente no cero?
    \end{enumerate}
  \item
    Considere una variante de filtros de Bloom,
    en la cual los \(m\) bits se dividen en \(k\) grupos de \(m / k\),
    cada uno reservado
    a una de las funciones \emph{\foreignlanguage{english}{hash}}.
    Esto tiene la ventaja que pueden calcularse en paralelo,
    incluso con memorias separadas.
    Compare la tasa de falsos positivos con la derivada en el texto.
  \end{enumerate}
\bibliography{../referencias}

%%% Local Variables:
%%% mode: latex
%%% TeX-master: "../INF-221_notas"
%%% ispell-local-dictionary: "spanish"
%%% End:

% LocalWords:  Hashing english hash hashing hashear coupon collector
% LocalWords:  problem Basilea eq Pr Hit longest length small large
% LocalWords:  relacion cut estimate Absent CLRS et cumulativa min
% LocalWords:  IPv

\bibliographystyle{babplain-fl}

\chapter{Algoritmos Aleatorizados}
\label{cha:randomized-algorithms}

  Un área activa de investigación reciente
  son los \emph{algoritmos aleatorizados}~%
    \cite{karp91:_intro_randomized_algorithms,
          motwani96:_randomized_algor}
  (fea traducción de \emph{\foreignlanguage{english}{randomized algorithms}}
   a la que nos obliga la RAE).
  En la visión tradicional de algoritmos deterministas
  nos interesan algoritmos que resuelven el problema \emph{correctamente}
  (siempre)
  y \emph{rápidamente}
  (típicamente,
   esperamos una solución en plazo polinomial en el tamaño de la entrada).
  Un algoritmo aleatorizado toma,
  además de la entrada,
  una secuencia de números aleatorios
  que usa para tomar decisiones durante la ejecución.
  Note que el comportamiento puede ser diferente
  incluso con la misma entrada.
  El interés es que en muchos casos un algoritmo aleatorizado
  es más simple
  y usa menos recursos
  (en promedio)
  que alternativas deterministas.
  Un uso importante es algoritmos aleatorizados
  para aproximar soluciones
  a problemas \NP\nobreakdash-completos.
  Véase también el capítulo~\ref{cha:algoritmos-aproximados}.
  Textos en el área son el clásico de Motwani y Raghavan~%
    \cite{motwani95:_randomized_algorithms}
  y el más accesible de Hromkovič~%
    \cite{hromkovic05:_design_anal_randomized_algor}.

\section{Clasificación de algoritmos aleatorizados}
\label{sec:clasificacion-aleatorizados}

  Los algoritmos aleatorizados se clasifican en dos grandes ramas,
  algoritmos de \emph{Monte Carlo} y \emph{Las Vegas}
  (por las famosas ciudades de juegos,
   nombres propuestos por Lázló Babai~%
     \cite{babai79:_monte_carlo_algo_graph_isomorph_test},
   en analogía a los métodos de Monte Carlo usados en análisis numérico,
   física estadística y simulación,
   aplicados ya en el proyecto Manhattan).
  Algunos hablan también de algoritmos de \emph{Atlantic City},
  otra ciudad famosa por sus casinos.

\subsection{Algoritmos de Monte Carlo}
\label{sec:algoritmos-monte-carlo}

   Un algoritmo de Monte Carlo tiene un tiempo de ejecución fijo,
   puede dar una respuesta incorrecta
   (típicamente con baja probabilidad).
   Es importante que las probabilidades y valores esperados involucrados
   son sobre las elecciones aleatorias del algoritmo,
   independientes de la entrada.
   Repitiendo el algoritmo suficientes veces
   la probabilidad de error
   disminuye exponencialmente con el número de corridas.

   Términos relevantes son \emph{errores unilaterales}
   (\emph{\foreignlanguage{english}{one-sided error}})
   y bilaterales
   (\emph{\foreignlanguage{english}{two-sided error}}).
   Un algoritmo para resolver un problema de decisión tiene error unilateral
   si siempre que se equivoca es en el mismo sentido:
   \emph{sesgo falso}
   (\emph{\foreignlanguage{english}{false biased}})
   si siempre está en lo correcto si retorna falso,
   puede equivocarse si retorna verdadero;
   \emph{sesgo verdadero}
   (\emph{\foreignlanguage{english}{true biased}})
   si siempre está en lo correcto si retorna verdadero,
   puede equivocarse si retorna falso.
   En el caso de error bilateral,
   puede equivocarse en ambas direcciones.

   Un algoritmo de Monte Carlo con sesgo falso,
   por ejemplo,
   puede usarse para determinar con alta probabilidad
   que la respuesta es \textquote{verdadero}:
   podemos correr el algoritmo suficientes veces.
   Si alguna vez responde \textquote{falso},
   sabemos que esa es la respuesta;
   si en \(n\) corridas independientes nunca responde \textquote{falso}
   sabemos con alta probabilidad que la respuesta es \textquote{verdadero}.
   En caso de un algoritmo con error bilateral,
   lo ejecutamos múltiples veces
   y quedamos con la respuesta mayoritaria.

\subsection{Algoritmos de Las Vegas}
\label{sec:algoritmos-las-vegas}

   Un algoritmo de Las Vegas
   siempre da el resultado correcto,
   pero no hay plazo definido.
   Usualmente se pide además
   que el valor esperado del tiempo de ejecución
   (dependiente de las elecciones aleatorias dentro del algoritmo)
   sea finito.

   Al analizar el tiempo de ejecución de estos algoritmos,
   este es una variable aleatoria
   (dependiente de los valores elegidos).
   Usaremos las siguientes definiciones para discutirlos.
   \begin{definition}
     \label{def:RT}
     Llamaremos \(\mathrm{RT}(u)\) al tiempo de ejecución del algoritmo
     con entrada \(u\).
   \end{definition}
   \begin{definition}
     \label{def:T-aleatorizado}
     El tiempo esperado de ejecución del algoritmo
     para entradas de tamaño \(n\) es:
     \begin{equation*}
       T(n)
         = \max_{\lvert u \rvert = n} \Exp[ \mathrm{RT}(u) ]
     \end{equation*}
   \end{definition}
   Cuidado,
   un algoritmo aleatorio puede tener tiempo máximo de ejecución
   exponencial en \(n\),
   o incluso ilimitado,
   aún si tiene buen tiempo de ejecución promedio.
   Por ejemplo,
   el tiempo de ejecución del algoritmo~\ref{alg:perder-tiempo}
   es una variable con distribución geométrica de probabilidad \(1/2\),
   con lo que,
   medido en número de invocaciones de \(\mathrm{RandBit}()\),
   es \(\Exp[\mathrm{RT}] = 2\).
   Sin embargo,
   en el peor caso nunca termina
   (si \(\mathrm{RandBit}()\) siempre retorna \num{0}).
   \begin{algorithm}[ht]
     \DontPrintSemicolon\Indp

     \While{\(\mathrm{RandBit}() = 1\)}{
       \emph{Nothing}
     }
     \caption{Perder el tiempo}
     \label{alg:perder-tiempo}
   \end{algorithm}

   Otra definición relevante es:
   \begin{definition}
     \label{def:high-probability}
     Decimos que el tiempo de ejecución del algoritmo \(A\)
     es \(O(f(n))\) \emph{con alta probabilidad}
     si hay constantes \(c > 0\) y \(d \ge 1\) tales que:
     \begin{equation*}
       \Pr[ \mathrm{RT}(A) \ge c \cdot f(n) ]
         \le \frac{1}{n^d}
     \end{equation*}
     Requeriremos también que:
     \begin{equation*}
       \Exp[ \mathrm{RT}(A) ]
         = O(f(n))
     \end{equation*}
   \end{definition}
   Note que hay varias definiciones alternativas,
   que difieren en ciertos detalles.
   Adoptaremos esta,
   mientras no haya consenso.

\subsection{Algoritmos de Atlantic City}
\label{sec:atlantic-city}

   Este término fue introducido por Finn~%
     \cite{finn82:_compar_probab_tests_primal},
   para referirse a algoritmos que dan la respuesta correcta al menos \(75\%\)
   de las veces
   (algunas otras versiones de la definición
    especifican alguna otra cota mayor a \(50\%\)).

\subsection{Primeros ejemplos}
\label{sec:randomized-examples}

  Nos interesan algoritmos más rápidos para nuestros problemas,
  posiblemente más rápidos a costa de precisión.
  Muchas veces podemos disminuir la posibilidad de error
  a tan bajo como queramos,
  aún acelerando el algoritmo.
  Los ejemplos iniciales vienen de~%
    \cite{OpenDSA16:_senior_algorithms}.

  Sabemos
  (capítulo~\ref{cha:discrete-algorithms})
  que la cota inferior para hallar el máximo
  de \(n\)~elementos no ordenados
  es \(\Omega(n)\).
  Esto es el mínimo tiempo necesario
  para asegurar que hemos identificado el máximo.
  Exploraremos cómo podemos relajar el \textquote{asegurar},
  que tiene varios aspectos.

  Hay varias garantías exigibles a un algoritmo
  que entrega \(X\) como valor máximo,
  siendo que el valor correcto es \(Y\).
  Hemos exigido \(X\) sea \(Y\).
  Esto es un algoritmo \emph{exacto} o \emph{determinista}.
  Podemos pedir solo que la posición de \(X\) en orden
  (su \emph{rango})
  es \textquote{cercano} al de \(Y\)
  (tal vez una distancia fija,
   o un porcentaje).
  Esto es un \emph{algoritmo aproximado}.
  Podemos solicitar que \(X\) \textquote{usualmente} sea \(Y\).
  Este es un \emph{algoritmo probabilístico} o \emph{aleatorizado}.
  Finalmente,
  podemos pedir
  que el rango de \(X\) \textquote{usualmente} sea \textquote{cercano}
  al rango de \(Y\).
  Esto suele llamarse \emph{algoritmo heurístico}.

  Veamos un ejemplo de algoritmo que entrega un valor grande
  dejando de lado la exigencia de obtener el mejor valor
  a cambio de mejor tiempo de ejecución.
  Es un algoritmo aleatorizado,
  en que usa el azar en su ejecución.
  Elija \(m\) elementos,
  entregue el mejor de estos como respuesta.
  Esto tiene un costo de \(m - 1\)~comparaciones
  (hay que hallar el máximo de \(m\)~elementos).
  No sabemos lo que obtendremos,
  pero podemos estimar que su rango será aproximadamente \(m n / (m + 1)\).
  Por ejemplo,
  con \(n = 1\,000\,000\) y \(m = \log n = 20\),
  esperamos que el máximo reportado pertenezca al \(5\%\)~mayor.

  Una ligera modificación es elegir un elemento en la mitad superior.
  Podemos elegir \((n + 1) / 2\) elementos y tomar el máximo de ellos,
  con un costo de \(n / 2\) comparaciones.
  Sabemos que si queremos garantía no puede hacerse mejor.
  Si estamos dispuestos a aceptar casi certeza,
  podemos hacerlo más rápido.
  Si elegimos dos elementos,
  el mayor de ellos estará en la mitad superior con probabilidad~\(3/4\)
  (no está en la mitad superior solo si ambos están en la mitad inferior).
  Si la probabilidad~\(3/4\) no es suficiente,
  ¡tome más pares!
  Para \(k\) pares,
  el máximo está en la mitad superior con probabilidad \(1 - 2^{-k}\),
  independiente del número de elementos~\(n\),
  siempre que \(n\) sea mucho mayor que~\(k\)
  (si no,
   es aún mejor).
  Con \num{10}~números,
  la probabilidad de fallar es una en~\num{1024},
  con \num{100}~números,
  la probabilidad de error es menor a la de cualquier catástrofe imaginable
  que interrumpa el proceso.

\section{Ámbitos de aplicación}
\label{sec:ambitos-aleatorizados}

% motwani96:_randomized_algor da una visión de aplicaciones más amplia

  Algoritmos aleatorizados son usados en el ámbito de teoría de números
  (el método de Miller-Rabin~%
    \cite{miller76:_Riemann_hypot_tests_primality,
          rabin80:_probab_algor_test_primality},
   un algoritmo de Monte Carlo
   para verificar primalidad,
   es el más usado actualmente).

   Una variante de Quicksort
   elige el pivote al azar,
   para hacer poco probable el peor caso
   (e incidentalmente complicarle la vida a un adversario
    que quiera forzar el peor caso).
   Estructuras de datos interesantes son
   \emph{\foreignlanguage{english}{skip lists}}~%
     \cite{pugh90:_skip_lists}
   y \emph{\foreignlanguage{english}{treaps}}~%
     \cite{aragon89:_random_search_tree,
           seidel96:_random_search_trees}
   (una mezcla de \emph{\foreignlanguage{english}{tree}}
    con \emph{\foreignlanguage{english}{heap}}),
   usan elecciones aleatorias para obtener buen rendimiento en promedio.
   Estos son todos ejemplos de algoritmos Las Vegas.

   Aplicaciones teóricas incluyen demostraciones probabilísticas de existencia:
   demostrar que un objeto con alguna característica especial
   debe aparecer con probabilidad no nula
   al elegir al azar entre una población adecuada
   demuestra que el objeto debe existir.
   Esta es la base del método probabilístico,
   que fue popularizado y aplicado ampliamente por Paul Erdős.
   Texto clásico es el de Alon y Spencer~%
     \cite{alon15:_probabilistic_method}.

   Una técnica para diseñar algoritmos deterministas
   es tomar un algoritmo aleatorizado y \textquote{desaleatorizarlo}.
   Esto es relevante en la práctica,
   pero su importancia principal
   está en desentrañar la relación
   entre las clases de complejidad correspondientes.

\section{Paradigmas de aplicación}
\label{sec:paradigmas-aleatorizado}

  Algunas de las razones que hacen útil un algoritmo aleatorizado
  son las siguientes.

\paragraph{Frustrar a un adversario}

  Si un adversario puede aprovechar el comportamiento
  de un algoritmo determinista,
  haciendo que el algoritmo tome decisiones al azar
  dificulta ataques.
  Es una posible defensa frente a \emph{ataques algorítmicos}
  (ver por ejemplo Crosby y Wallach~%
    \cite{crosby03:_DoS_algo_compl_attack}
   y McIllroy y Douglas~%
    \cite{mcillroy99:_killer_adver_quicksort}).

\paragraph{Muestreo}

  En muchos casos,
  extraemos una pequeña muestra de una gran población
  para inferir propiedades de la población.
  Computación con pequeñas muestras es barato,
  sus propiedades pueden guiar el cálculo de propiedades de la población.

\paragraph{Cualquiera sirve}

  Hay situaciones en que requerimos un objeto de características especiales,
  para los que no tenemos algoritmos eficientes que lo construyan.
  Si son comunes y simples de detectar,
  elegir al azar es una opción.

\paragraph{Abundancia de testigos}

  Muchos problemas computacionales de traducen en hallar un \emph{testigo}
  (o un certificado)
  que permita verificar una hipótesis eficientemente.
  Por ejemplo,
  para demostrar que un número es compuesto,
  basta exhibir un factor no trivial.
  Para muchos problemas,
  los testigos son parte
  de una población demasiado grande para ser revisada sistemáticamente.
  Si este espacio contiene un número relativamente grande de testigos,
  un elemento elegido al azar es probable que sea un testigo.
  Aún más,
  muestreando repetidas veces y no hallar un testigo
  disminuye exponencialmente la probabilidad de que haya un testigo,
  o sea que la hipótesis se cumple.

\paragraph{Huellas digitales}

  Una \emph{huella digital}
  (en inglés,
   \emph{\foreignlanguage{english}{fingerprint}})
  es la imagen de un elemento de un gran universo en uno mucho menor.
  Huellas digitales obtenidas mediante mapas al azar
  tienen muchas propiedades útiles.
  Veremos un par de ejemplos.

\paragraph{Reordenar al azar}

  Muchos problemas tienen la propiedad que un algoritmo bastante ingenuo
  se comporta extremadamente bien en promedio
  bajo el supuesto de datos entregados ordenados al azar.
  Aún si el algoritmo tiene mal peor caso,
  reordenar hace que el peor caso sea muy improbable.

\paragraph{Balance de carga}

  Cuando hay que elegir entre diversos recursos,
  asignar al azar
  puede usarse para \textquote{repartir} la carga en forma pareja.
  Esto resulta particularmente interesante
  cuando las decisiones deben tomarse en sistemas distribuidos,
  en forma local sin conocimiento global.

\paragraph{Aislar y quebrar simetría}

  En computación distribuida,
  es común necesitar romper un \emph{\foreignlanguage{english}{deadlock}}
  o simetría,
  o elegir un valor común
  (acordar un ordenamiento al azar de las soluciones,
   y luego buscar la primera solución por separado).

\section{Teoría de números}
\label{sec:number-theory}

  Entre las primeras aplicaciones de algoritmos probabilísticos
  se cuentan algoritmos en teoría de números.
  La teoría de números se ha hecho muy importante
  por sus aplicaciones criptográficas:
  el sistema de cifrado público RSA~%
    \cite{rivest78:_RSA}
  requiere números primos grandes,
  al igual que el intercambio de claves Diffie-Hellman~%
    \cite{diffie76:_new_directions_cryptography}.

\subsection{Preliminares teóricos}
\label{sec:nt-preliminaries}

  Si \(p\) es primo,
  el conjunto de residuos módulo \(p\) forma un campo,
  \(\mathbb{Z}_p\).
  Sabemos también que si \(x^2 = 1\) en un campo debe ser \(x = \pm 1\)
  (en nuestro caso,
   si \(p \mid x^2 - 1\) es que \(p \mid (x - 1) (x + 1)\),
   por el lema de Euclides
   \(p \mid x - 1\) o \(p \mid x + 1\)).

  El grupo de unidades \(\mathbb{Z}_p^\times\)
  (los elementos relativamente primos a \(p\),
   residuos \num{1} a \(p - 1\),
   con la multiplicación módulo \(p\))
  es cíclico,
  vale decir,
  hay un generador \(g\)
  (llamado \emph{raíz primitiva de \(p\)})
  tal que todo elemento \(a \in \mathbb{Z}_p^\times\) puede representarse como:
  \begin{equation}
    \label{eq:units-powers}
    a
      \equiv g^r
  \end{equation}
  Las raíces primitivas son relativamente frecuentes,
  es simple ver que si \(\gcd(r, p - 1) = 1\)
  entonces \(g^r\) es raíz primitiva;
  hay \(\phi(p - 1)\) raíces primitivas módulo \(p\).

  Sabemos del pequeño teorema de Fermat que para \(a \in \mathbb{Z}_p^\times\)
  es:
  \begin{equation}
    \label{eq:Fermat}
    a^{p - 1}
      \equiv 1 \pmod{p}
  \end{equation}

\subsection{Hallar números primos}
\label{sec:nt-prime-numbers}

  Si \(p\) es impar,
  \(p - 1 = q \cdot 2^s\) con \(q\) impar y \(s \ge 1\).
  Vale decir,
  \(a^{q \cdot 2^{s - 1}}\) es raíz cuadrada de \num{1},
  si es \num{1} lo es su raíz cuadrada \(a^{q \cdot 2^{s - 2}}\),
  \ldots
  Descendiendo de esta forma llegaremos a que \(a^q \equiv 1 \pmod{p}\)
  o que para algún \(0 \le k < s\) es \(a^{q \cdot 2^k} \equiv -1 \pmod{p}\).
  El contrapositivo de la observación precedente
  es que si hallamos \(a\) y \(0 \le k < s\)
  tales que \(a^{q \cdot 2^k} \equiv 1 \pmod{p}\)
  pero \(a^{q \cdot 2^{k - 1}} \not\equiv \pm 1 \pmod{p}\)
  entonces \(p\) es compuesto.
  Resulta que en tal caso a lo más una cuarta parte de los \(a\)
  mienten,
  en el sentido de dar como probable primo a un número compuesto.

  El algoritmo~\ref{alg:miller-rabin}
  es la modificación aleatorizada de Rabin~%
     \cite{rabin80:_probab_algor_test_primality}
  al algoritmo determinista de Miller~%
     \cite{miller76:_Riemann_hypot_tests_primality}.
  Supone \(p > 3\) impar.
  Generalmente se repite la prueba cierto número de veces
  para obtener mayor confianza en el resultado.
  Esta es la prueba de primalidad en más extenso uso hoy día,
  la provee por ejemplo la biblioteca LibTomMath~%
    \cite{teamtom19:_libtommath_1.2.0},
  descrita en detalle por St~Denis y Rose~%
    \cite{st_denis06:_bignum_math}.
  \begin{algorithm}
    \selectlanguage{english}
    \DontPrintSemicolon\Indp

    \Function{\(\mathrm{is\_prime}(p)\)}{
      Write \(p = q \cdot 2^s + 1\) with \(q\) odd \;
      Pick \(a\) at random from \([2, n - 2]\) \;
      \(x \gets a^q \bmod p\) \;
      \eIf{\(x = 1 \vee x = p - 1\)} {
        \Return Probably prime \;
      }{
        \For{\(k \gets 1 \KwTo s - 1\)}{
          \(x \gets x^2 \bmod p\) \;
          \If{\(x = p - 1\)}{
            \Return Probably prime \;
          }
          \Return Composite \;
        }
      }
    }
    \selectlanguage{spanish}
    \caption{Prueba de Miller-Rabin}
    \label{alg:miller-rabin}
  \end{algorithm}

\subsection{Hallar raíz primitiva módulo \(p\)}
\label{sec:nt-primitive-root}

  El intercambio de claves Diffie-Hellman~%
    \cite{diffie76:_new_directions_cryptography}
  requiere conocer una raíz primitiva \(g\) para el primo \(p\)
  para construir claves.
  No se conocen métodos eficientes para obtener una raíz primitiva
  de un primo grande,
  pero son bastante frecuentes:
  si \(g\) es raíz primitiva módulo \(p\),
  también lo es \(g^r\) siempre que \(\gcd(r, p - 1) = 1\).
  Ahora bien,
  \(g\) es raíz primitiva módulo \(p\)
  si y solo si para todos los primos \(q \mid p - 1\) es:
  \begin{equation*}
    g^{(p - 1) / q}
      \not\equiv 1 \pmod{p}
  \end{equation*}
  Un algoritmo aleatorizado para hallar un \(g\) apropiado es inmediato.

\subsection{Raíces cuadradas módulo \(p\)}
\label{sec:sqrt-mod-p}

  Es claro que si y solo si \(a \equiv g^{2 k}\)
  con \(g\) una raíz primitiva módulo \(p\)
  es un cuadrado
  (les llaman \emph{residuos cuadráticos}),
  exactamente la mitad de los elementos de \(\mathbb{Z}_p^\times\)
  es un cuadrado.
  Resulta el \emph{criterio de Euler}:
  \begin{equation}
    \label{eq:Euler-criterion}
    a^{(p - 1) / 2}
      \equiv \begin{cases}
                1 & \text{\(a\) es residuo cuadrático} \\
               -1 & \text{\(a\) es no-residuo cuadrático}
             \end{cases}
  \end{equation}
  Para referencia futura,
  al multiplicar residuos cuadráticos y no\nobreakdash-residuos cuadráticos
  da un residuo cuadrático
  si y solo si hay un número par de no\nobreakdash-residuos cuadráticos
  en el producto.

  Si \(a = 0\),
  su raíz cuadrada es \num{0}.

  En caso contrario,
  sea \(a\) un residuo cuadrático módulo \(p\),
  buscamos resolver \(x^2 \equiv a\).

  Debemos considerar varios casos.

\subsubsection{Caso \(p = 2\)}
\label{sec:p=2}

  Este caso es trivial,
  simplemente es \(\mathrm{sqrt}(a, 2) = a\).

\subsubsection{Caso \(p \equiv 3 \pmod{4}\)}
\label{sec:p-equiv-3}

  En este caso \((p - 1) / 2\) es impar,
  con lo que por el criterio de Euler:
  \begin{align*}
    a^{(p - 1) / 2}
      &\equiv 1 \\
    a \cdot a^{(p - 1) / 2}
      &\equiv a^{(p + 1) / 2} \\
      &\equiv a
  \end{align*}
  Como \(4 \mid (p + 1)\) en este caso,
  vemos que una raíz cuadrada es \(a^{(p + 1) / 4}\).

\subsubsection{Caso \(p \equiv 1 \pmod{4}\)}
\label{sec:p-equiv-1}

  Ahora \((p - 1) / 2\) es par.
  Escribamos \(p - 1 = q \cdot 2^s\) con \(q\) impar.
  Note que:
  \begin{equation*}
    \left( a^{(q + 1) / 2} \right)^2
      \equiv a^q \cdot a
  \end{equation*}
  Si \(a^q \equiv 1\),
  estamos listos.
  En caso contrario,
  tenemos \(r_0\) y \(t_0\) que satisfacen:
  \begin{equation}
    \label{eq:rt}
    r_0^2
      \equiv a t_0
  \end{equation}
  donde \(t_0 \equiv a^q\) es una raíz \(2^{s - 1}\) de \num{1},
  ya que \(t_0^{2^{s - 1}} \equiv a^{q \cdot 2^{s - 1}} \equiv a^{(p - 1) / 2}\).
  Si dados \(r_k\), \(s_k\) y \(t_k\) como en~\eqref{eq:rt}
  podemos construir un nuevo juego \(r_{k + 1}, s_{k + 1}, t_{k + 1}\)
  pero para \(s_{k + 1} = s_k - 1\),
  podemos ir reduciendo el exponente \(s\) hasta \num{0} y estamos listos.

  Si \(t_k^{2^{s_k - 2}} \equiv 1\),
  no hay nada que hacer,
  los valores anteriores sirven:
  \(r_{k + 1} = r_k\) y \(t_{k + 1} = t_k\).
  Si no,
  debe ser \(t_k^{2^{s_k - 2}} \equiv -1\),
  ya que su cuadrado es congruente con \num{1}.
  Para hallar un nuevo juego \(r_{k + 1}, t_{k + 1}\)
  multiplicamos \(r_k\) por un valor \(u\) a determinar
  tal que:
  \begin{align}
    (r_k u)^2
      &\equiv a t_k u^2 \\
      &\equiv 1 \label{eq:(ru)^2=1}
  \end{align}
  O sea:
  \begin{align*}
    r_{k + 1}
      &\equiv u r_k \\
    t_{k + 1}
      &\equiv u^2 t_k
  \end{align*}
  De~\eqref{eq:(ru)^2=1}
  vemos que \(u^2\) debe ser una raíz \(2^{s_k - 2}\) de \(-1\).
  El truco es usar un no\nobreakdash-residuo cuadrático \(b\),
  del que por el criterio de Euler sabemos:
  \begin{equation*}
    b^{q \cdot 2^{s_0 - 1}}
      \equiv -1
  \end{equation*}
  con lo que elevando \(b^q\) al cuadrado repetidas veces
  tenemos una secuencia de raíces \(2^k\) de \(-1\),
  elegimos la correcta.

  El algoritmo~\ref{alg:sqrt-mod-p} resultante es de Tonelli~%
     \cite{tonelli91:_bemerkung_aufloesung_congruenzen},
  redescubierto y mejorado casi un siglo después por Shanks~%
    \cite{shanks72:_five_number_theoretic_algorithms}.
  Es lo que provee por ejemplo la biblioteca LibTomMath~%
    \cite{teamtom19:_libtommath_1.2.0},
  descrita en detalle por St~Denis y Rose~%
    \cite{st_denis06:_bignum_math}.
  \begin{algorithm}
    \selectlanguage{english}
    \DontPrintSemicolon\Indp

    \Function{\(\mathrm{sqrt}(a, p)\)}{
      \eIf{\(p \equiv 3 \pmod{4}\)}{
        \Return \(a^{(p + 1) / 4} \bmod p\) \;
      }{
        \Repeat{\(b^{p - 1) / 2} \equiv -1 \pmod{p}\)}{
          Select \(b\) at random from \([2, (p - 1) / 2]\) \;
        }
        \tcc{\(b\) is a quadratic non-residue}
        \(q \gets (p - 1) / 4\) \;
        \(s \gets 2\) \;
        \While{\(\neg \mathrm{odd}(q)\)}{
          \(q \gets q / 2\) \;
          \(s \gets s + 1\) \;
        }
        \tcc{Now \(p - 1 = q \cdot 2^s\) with \(q\) odd}
        \(c \gets b^q \bmod p\) \;
        \(t \gets a^q \bmod p\) \;
        \(r \gets a^{(q + 1) / 2} \bmod p\) \;
        \While{\(t \not\equiv 1 \pmod{p}\)}{
          \(i \gets 1\) \;
          \(u \gets t\) \;
          \While{\(u \not\equiv 1 \pmod{p}\)}{
            \(u \gets u^2 \bmod p\) \;
            \(i \gets i + 1\) \;
          }
          \(b \gets c^{2^{s - i - 1}} \bmod p\) \;
          \(s \gets i\) \;
          \(c \gets b^2 \bmod p\) \;
          \(t \gets t b^2 \bmod p\) \;
          \(r \gets r b \bmod p\) \;
        }
        \Return \(r\) \;
      }
    }
    \selectlanguage{spanish}
    \caption{Raíz cuadrada módulo \(p\)
             (casos no triviales)}
    \label{alg:sqrt-mod-p}
  \end{algorithm}

\section{Balance de carga}
\label{sec:balance-de-carga}

  Supongamos un nuevo sitio social,
  \emph{MalaLeche}.
  Agrupa a gente dada a reclamar por todo,
  y pelearse por los temas más triviales.
  Como el tráfico es alto,
  se ha determinado que se requieren varios procesadores.
  Si alguna de las máquinas se ve sobrepasada,
  el rendimiento sufre
  (con los consiguientes reclamos de los usuarios).
  Un experimento fue asignar tareas por las primeras letras de los mensajes,
  pero peleas sobre \textquote{\textbf{pr}eferencia de editor, emacs o vi}
  y \textquote{\textbf{pr}oyecto hidroeléctrico} produjeron serios problemas.
  Si se conociera el detalle de las tareas de antemano,
  asignarlas de forma óptima entre máquinas
  es una variante del problema \textsc{Bin Packing},
  que se sabe \NP\nobreakdash-completo.
  Hay soluciones aproximadas,
  pero si no se conoce de antemano el detalle de las tareas,
  esto no tiene caso tampoco.
  Los desarrolladores abandonaron,
  y asignaron tareas al azar a las máquinas.
  Para su sorpresa,
  el sistema funciona sin problemas.

  Resulta que asignación al azar no solo balancea la carga razonablemente bien,
  también permite dar garantías de rendimiento.
  En general,
  conviene considerar un esquema aleatorizado
  si un sistema determinista es demasiado difícil
  o requiere información que simplemente no está disponible.

  Específicamente,
  MalaLeche recibe \num[group-digits = true]{24\,000} peticiones
  en cada período de \num[group-digits = true]{10} minutos.
  Se ha determinado que las peticiones
  toman a lo más \SI{1}{\second} de procesamiento;
  aunque la mayoría son triviales
  (quejarse de la ortografía del mensaje precedente
   y similares),
  siendo el tiempo promedio de ejecución \SI{0,25}{\second}.
  Midiendo el trabajo en unidades de \SI{1}{\second} de procesamiento,
  si a alguno de los servidores
  se le asignan más de \num[group-digits = true]{600} unidades de trabajo
  en \SI{10}{\minute},
  se cae y produce problemas.
  La carga total de MalaLeche
  de \(\num[group-digits = true]{24\,000}
           \cdot \num[group-digits = true]{0,25}
          = \num[group-digits = true]{6\,000}\)~unidades de trabajo
  cada \num[group-digits = true]{10} minutos
  indica que se requieren \num[group-digits = true]{10} máquinas
  trabajando al \SI{100}{\percent} con balance de carga perfecto.
  Necesitaremos más de \num[group-digits = true]{10}
  para acomodar fluctuaciones en la carga
  y balance imperfecto de carga,
  la pregunta es cuántos se requieren.

  Específicamente,
  nos interesa el número \(m\) de servidores que hace muy poco probable
  que alguno se vea sobrecargado al asignarle
  más de \num[group-digits = true]{600}~unidades de trabajo
  en un período de \num[group-digits = true]{10} minutos.

  Primero,
  acotemos la probabilidad de que el primer servidor se sobrecargue
  en un período dado.
  Sea \(T\) las unidades de trabajo asignadas a ese servidor,
  buscamos una cota superior a \(\Pr[T \ge 600]\).
  Sea \(t_i\) el tiempo que la primera máquina dedica a la tarea \(i\),
  con lo que \(t_i = 0\) si se asigna a otra máquina.
  Así,
  con \(n = \num[group-digits = true]{24\,000}\):
  \begin{equation*}
    T
      = \sum_{1 \le i \le n} t_i
  \end{equation*}
  Podemos usar las cotas de Chernoff
  (ver el apéndice~\ref{apx:pizca-probabilidades})
  si las variables son independientes
  y en el rango \([0, 1]\).
  La primera condición se cumple
  si la asignación de tareas a los servidores
  no depende de su tiempo de ejecución,
  la segunda se da porque ninguna tarea toma más de una unidad.

  Hay \num[group-digits = true]{24\,000} tareas,
  cada una de tiempo de procesamiento esperado de \(0,25\,[s]\).
  Asignado tareas al azar a los servidores,
  la carga esperada para el primer servidor es:
  \begin{align*}
    \Exp[T]
      &= \frac{\num[group-digits = true]{24\,000}
                 \cdot \num[group-digits = true]{0,25}}{m} \\
      &= \frac{\num[group-digits = true]{6\,000}}{m}
  \end{align*}
  Como vimos,
  con \(m < 10\),
  esperamos que se sobrecargue,
  con \(m = 10\) está al \SI{100}{\percent} de capacidad.

  Nos interesa el límite:
  \begin{equation*}
    600 = c \Exp[T]
  \end{equation*}
  con lo que \(c = m / 10\).
  La cota de Chernoff es:
  \begin{align*}
    \Pr[T \ge 600]
      &= \Pr\left[ T \ge \frac{m}{10} \cdot \Exp[T] \right] \\
      &\le \mathrm{e}^{-\beta(m / 10) \cdot \num[group-digits = true]{6\,000} / m}
  \end{align*}
  donde:
  \begin{equation*}
    \beta(c)
      = c \, \ln c - c + 1
  \end{equation*}
  Por la cota de la unión,
  la probabilidad de que \emph{alguna} de las máquinas
  se sobrecargue en una hora cualquiera es:
  \begin{align*}
    \Pr[\text{alguna máquina se sobrecarga}]
      &\le \sum_{1 \le i \le m} \Pr[\text{el servidor \(i\) se sobrecarga}] \\
      &=   m \Pr[\text{el servidor \num{1} se sobrecarga}] \\
      &\le m \mathrm{e}^{-\beta(m / 10) \cdot \num[group-digits = true]{6\,000} / m}
  \end{align*}
  Algunos valores se tabulan a continuación:
  \begin{center}
    \begin{tabular}[ht]{>{\(}c<{\):}>{\(}l<{\)}}
      m = 11 & 0,784\ldots \\
      m = 12 & 0,000999\ldots \\
      m = 13 & 0,0000000760\ldots
    \end{tabular}
  \end{center}
  O sea,
  con \num{11} máquinas alguna puede caerse casi inmediatamente,
  \num{12} debieran durar unos días,
  y \num{13} dan para un siglo o dos.

  Un resultado relevante es que si se revisan al azar dos de las máquinas,
  y se elige aquella con menos carga,
  la carga máxima esperada disminuye muy substancialmente
  (ver por ejemplo Mitzenmacher, Richa y Sitaraman~%
    \cite{mitzenmacher01:_power_two_random_choic}).
  Esto es importante porque no es necesario
  imponer la carga extra de revisar todas las máquinas,
  además que es fácil de hacer las consultas en paralelo.

  El resultado clásico
  (ver Gonnet~\cite{gonnet81:_expected_len_longest_probe_seq})
  es que si se distribuyen \(m\) bolas
  en \(m\) casilleros,
  con alta probabilidad el casillero más lleno contiene:
  \begin{equation*}
    (1 + o(1)) \frac{\ln m}{\ln \ln m}
  \end{equation*}
  Azar, Broder, Karlin y Upfal~%
    \cite{azar00:_balanced_alloc}
  demuestran que si hay \(m\) casilleros
  y se distribuyen \(n\)
  (\(n > m\))
  bolas entre ellos,
  si se eligen \(d\) casilleros al azar cada vez
  y se pone la bola en el más vacío
  donde \(d \ge 2\),
  el largo de la cola más larga con alta probabilidad es:
  \begin{equation*}
    (1 + o(1)) \frac{\ln \ln m}{\ln d} + \Theta(n/m)
  \end{equation*}
  La derivación es compleja,
  no la repetiremos acá.
  Vale decir,
  de \(d = 1\) a \(d = 2\) hay una mejora exponencial;
  para \(d \ge 2\) el cambio es solo en un factor constante moderado.

\section{Cotas inferiores a números de Ramsey}
\label{sec:numeros-Ramsey}

  Una aplicación del método probabilístico
  es la demostración de Erdős~%
    \cite{erdos47:_some_remarks_theo_graphs}
  de una cota inferior para el número de Ramsey \(R(r, r)\).

  El teorema de Ramsey~%
    \cite{ramsey30:_probl_formal_logic}
  en la forma que nos interesa dice que todo grafo de tamaño \(R(r, s)\)
  contiene una \emph{\foreignlanguage{english}{clique}} de tamaño \(r\)
  (un \(K_r\))
  o un conjunto independiente
  (vértices que no están unidos por arcos)
  de tamaño \(s\).
  Esto generalmente se expresa en términos de un grafo \(K_n\)
  cuyos arcos se colorean de rojo y azul,
  la pregunta es el mínimo \(n\)
  para el cual todo coloreo de arcos de \(K_n\) con rojo y azul
  tiene \(K_r\) rojo o \(K_s\) azul.
  El punto del teorema de Ramsey es que \(R(r, s)\) es finito.
  Obtener sus valores ha demostrado ser extraordinariamente difícil.

  Por ejemplo,
  tenemos el siguiente resultado:
  \begin{theorem}
    \label{theo:R(3,3)=6}
    \(R(3, 3) = 6\)
  \end{theorem}
  \begin{proof}
    Consideremos \(K_6\),
    con sus arcos coloreados de rojo o azul.
    Elija \(v\) entre sus vértices.
    Entre los \num{5} vértices restantes,
    hay al menos \num{3} unidos con arcos del mismo color a \(v\),
    digamos que es rojo.
    Si un par de estos tres vértices están conectados por un arco rojo,
    con \(v\) forman un \(K_3\) rojo.
    Si no,
    están unidos por arcos azules entre sí,
    y forman un \(K_3\) azul.
    Esto demuestra que \(R(3, 3) \le 6\).

    Por otro lado,
    la figura~\ref{fig:K5-dos-colores} muestra \(K_5\)
    con arcos coloreados de rojo y azul
    que no contiene \(K_3\) del mismo color,
    mostrando que \(R(3, 3) > 5\).
    \begin{figure}[ht]
      \centering
      % Idea for drawing filched from
      %	   http://www.texample.net/tikz/examples/combinatorial-graphs
      \begin{tikzpicture}[every node/.style
                            = {shape = circle, fill = black!75, draw = black}]
        \draw[thick, blue] \foreach \x in {18, 90, ..., 378}
        {
          (\x:1.5) -- (\x+72:1.5)
        };
        \draw[thick, red] \foreach \x in {18, 162, ..., 738}
        {
          (\x:1.5) node{} -- (\x+144:1.5)
        };
      \end{tikzpicture}
      \caption{\(K_5\) sin \(K_3\) monocromático}
      \label{fig:K5-dos-colores}
    \end{figure}
  \end{proof}

  Supongamos que tenemos un grafo completo \(K_n\),
  y coloreamos sus arcos de rojo y azul.
  Queremos demostrar que para valores suficientemente pequeños de \(n\)
  no hay \(r\) vértices los arcos entre los cuales son todos rojos o azules.

  Coloreemos el grafo al azar,
  asignándole el color rojo o azul a cada arco independientemente
  con la misma probabilidad.
  Calculamos el número esperado de grafos monocromáticos de \(r\) vértices
  como sigue:
  Para un conjunto \(S\) de vértices,
  sea \(X(S) = 1\)
  si todos los arcos entre vértices en \(S\) son del mismo color,
  \(X(S) = 0\) en caso contrario.
  El número de \(K_r\) monocromáticos es simplemente la suma de \(X(S)\)
  sobre todos los subconjuntos de \(r\) vértices.

  Consideremos un conjunto cualquiera \(S\) de \(r\) vértices.
  La probabilidad de que todos los arcos entre ellos
  sean del mismo color es simplemente:
  \begin{equation*}
    2 \cdot 2^{- \binom{r}{2}}
  \end{equation*}
  (el factor \num{2} es por los dos colores).
  La suma de los valores esperados de \(X(S)\) sobre todos los \(S\) es:
  \begin{equation*}
    \sum_S \Exp[X(S)]
      = \binom{n}{r} 2^{1 - \binom{r}{2}}
  \end{equation*}
  Por la linearidad del valor esperado,
  esto es:
  \begin{equation*}
    \Exp\left[ \sum_S X(S) \right]
      = \binom{n}{r} 2^{1 - \binom{r}{2}}
  \end{equation*}
  Pero esto es exactamente
  el número esperado de \(r\)\nobreakdash-subgrafos monocromáticos.

  Si este valor es menor a \num{1},
  como el número de \(K_r\) monocromáticos es un entero,
  debe haber al menos un coloreo en que es menor a \num{1}.
  Pero el único entero menor a \num{1} es \num{0}.
  O sea,
  si:
  \begin{equation*}
    \binom{n}{r}
      < 2^{\binom{r}{2} - 1}
  \end{equation*}
  hay un coloreo de los arcos de \(K_n\)
  tal que no contiene \(K_r\) rojos ni azules.

\section{Verificar producto de matrices}
\label{sec:verificar-producto-matrices}

  Supongamos que debemos verificar el producto de matrices
  \(\mathbf{A} \cdot \mathbf{B} = \mathbf{C}\).
  Usando el algoritmo tradicional con matrices de \(n \times n\)
  esto toma \(O(n^3)\) operaciones,
  el mejor algoritmo teórico da \(O(n^{2,38})\).
  El algoritmo de Freivalds~%
    \cite{freivalds77:_matrix_product_check}
  reduce esto a \(O(n^2)\).

  Elegimos \(\mathbf{a} \in \{0, 1\}^n\) uniformemente al azar,
  y calculamos \(\mathbf{A} ( \mathbf{B} \mathbf{a} )\),
  comparando con \(\mathbf{C} \mathbf{a}\).
  Esto son tres multiplicaciones de una matriz de \(n \times n\)
  por un vector de largo \(n\),
  todas demandan \(O(n^2)\) operaciones,
  y una comparación de dos vectores de largo \(n\).

  Si \(\mathbf{A} \cdot \mathbf{B} = \mathbf{C}\),
  lo anterior resulta siempre en igualdad.
  Si \(\mathbf{D} = \mathbf{A} \cdot \mathbf{B} - \mathbf{C} \ne \mathbf{0}\),
  tiene algún elemento no cero,
  digamos \(d_{i j} \ne 0\).
  Este se multiplica por \(a_i\)
  para dar
    \(\mathbf{D} \mathbf{a}
        = \mathbf{A} (\mathbf{B} \mathbf{a}) - \mathbf{C} \mathbf{a}\),
  por lo que a lo más una de las dos alternativas resultantes puede ser cero.
  O sea,
  a lo más la mitad
  de las elecciones de \(\mathbf{a}\) dice \textquote{iguales} erróneamente.
  Repitiendo \(k\) veces,
  respondiendo \textquote{distinto} si alguna vez resultan diferentes
  y \textquote{probablemente iguales} si siempre resultan iguales
  la probabilidad de error es menor a \(1 / 2^k\).
  El resultado tiene costo \(O(k n^2)\);
  y multiplicar por \(\mathbf{a}\)  no requiere multiplicaciones,
  solo sumar o no los coeficientes.

\section{Quicksort -- análisis aleatorizado}
\label{sec:quicksort-aleatorizado}

  Supongamos el siguiente modelo de Quicksort:
  estamos ordenando los valores \([1, n]\),
  antes de ordenar el arreglo los barajamos,
  y cada vez elegimos el primer elemento como pivote.
  Es un elemento al azar,
  no altera nuestro modelo.
  Definamos variables indicadoras \(X_{i j}\) para todos los pares \(i < j\)
  como \num{1} si \(i\) se compara con \(j\) durante la ejecución del algoritmo
  y \num{0} en caso contrario.
  Sorprendentemente,
  podemos calcular la probabilidad de este evento
  (y en consecuencia \(\Exp[X_{i j}]\)):
  se comparan solo si uno de los dos valores es elegido como pivote
  antes de terminar en particiones diferentes.
  Terminan en particiones diferentes si algún valor \(k\),
  con \(i < k < j\),
  se elige como pivote,
  que es exactamente si \(k\) aparece antes de \(i\) y de \(j\) en el arreglo
  al comenzar el proceso.
  Otros elementos no afectan el proceso,
  basta concentrarse en analizar permutaciones del rango \([i, j]\).
  Tenemos éxito
  (\(i\) se compara con \(j\))
  solo si la permutación de este rango comienza en \(i\) o \(j\),
  lo que ocurre con probabilidad \(2 / (j - i + 1)\),
  el valor esperado es:
  \begin{equation*}
    \Exp[X_{i j}]
       = \frac{2}{j - i + 1}
  \end{equation*}
  Sumando sobre los pares relevantes
  tenemos el número promedio de comparaciones:
  \begin{align*}
    \Exp\left[ \sum_{i < j} X_{i j} \right]
      &= \sum_{i < j} \Exp[X_{i j}] \\
      &= \sum_{1 \le i < j \le n} \frac{2}{j - i + 1} \\
      &= \sum_{1 \le i \le n - 1} \sum_{i < j \le n} \frac{2}{j - i + 1} \\
      &= 2 \sum_{1 \le i \le n - 1}
             \sum_{2 \le k \le n - i + 1} \frac{1}{k} \\
      &= 2 \sum_{1 \le i \le n - 1}
             \left( H_{n - i + 1} - 1 \right) \\
      &= 2 \sum_{1 \le i \le n - 1} H_{n - i + 1} - 2 (n - 1) \\
      &= 2 \sum_{2 \le k \le n} H_k - 2 (n - 1) \\
      &= 2 \sum_{1 \le k \le n} H_k - 2 n \\
      &= 2 \left( (n + 1) H_n - n \right) - 2 n \\
      &= 2 (n + 1) H_n - 4 n
  \end{align*}
  Igual que lo que obtuvimos antes~\eqref{eq:quicksort-comparisons}.

  Pero podemos ir más allá.
  Supongamos que corremos Quicksort aleatorizado
  sobre un arreglo de \(n\) elementos
  y paramos la recursión a profundidad \(c \ln n\) para una constante \(c\).
  La pregunta es cuál es la probabilidad
  que haya una hoja del árbol de llamadas
  al que no le corresponde ordenar un rango de un único elemento.
  Llame una división
  (inducida por un pivote)
  \emph{buena} si divide el conjunto \(S\) en pedazos \(S_1\) y \(S_2\)
  tales que:
  \begin{equation*}
    \min\{ \lvert S_1 \rvert, \lvert S_2 \rvert \}
      \ge \frac{1}{3} \lvert S \rvert
  \end{equation*}
  Si no es así,
  la llamamos \emph{mala}.
  Con nuestra suposición que todos los elementos son diferentes,
  vemos que la probabilidad de una buena división es \(1/3\).

  Cada buena división reduce el tamaño de la partición
  por un factor de al menos \(2/3\).
  Para llegar a un rango de un único elemento requerimos:
  \begin{align*}
    k
      &= \frac{\ln n}{\ln (3/2)} \\
      &= a \ln n
  \end{align*}
  buenas divisiones.
  Interesa ahora qué tan grande debe ser \(c\)
  para que la probabilidad de menos de \(a \ln n\) buenas divisiones
  sea pequeña.

  Consideremos el camino de la raíz a una hoja del árbol de recursión.
  Divisiones sucesivas son buenas o malas independientemente,
  podemos usar la cota de Chernoff:
  \begin{align*}
    \Pr\left[
         \text{número de buenas divisiones}
            < \frac{1}{3} a \ln n
        \right]
      &\le \mathrm{e}^{- \beta(1 / c) \cdot \frac{1}{3} c a \ln n} \\
      &= n^{- \beta(1 / c) c a / 3}
  \end{align*}
  Podemos elegir \(c\) de manera que \(\beta(1 / c) c a / 3 \ge 2\)
  (con \(c = 6\) tenemos de sobra).

  Por lo anterior,
  la probabilidad que en \emph{una} rama
  hayan muy pocas divisiones buenas es a lo más \(n^{-2}\);
  como hay a lo más \(n\) ramas,
  por la cota de la unión
  esto significa que la probabilidad
  que en \emph{alguna} rama hayan pocas divisiones buenas
  es a lo más \(n^{-1}\).
  Pero esto es la probabilidad que tome más de \(O(n \log n)\) comparaciones,
  y Quicksort aleatorizado ejecuta \(O(n \log n)\) comparaciones
  con alta probabilidad.

\section{Comparar por igualdad}
\label{sec:comparar}

  Supongamos que tenemos un gran archivo,
  por ejemplo medios de instalación de su distribución favorita,
  y quiere verificar que no contiene errores,
  vale decir,
  coincide con la versión original.
  Obviamente,
  queremos hacer esto sin demasiado cómputo adicional
  (somos impacientes)
  ni usando demasiado tráfico de red
  (somos amarretes).
  Omitiendo consideraciones criptográficas,
  el problema es calcular alguna forma
  de \emph{\foreignlanguage{english}{checksum}},
  tarea para la cual hay soluciones estándar,
  como CRC
  (\emph{\foreignlanguage{english}{Cyclic Redundancy Check}},
   inventado por Peterson~%
     \cite{peterson61:_CRC};
   un rápido resumen da por ejemplo Williams~%
     \cite{williams96:_painless_guide_crc}).

  Concretamente,
  supongamos que el archivo de marras es de \(n\)~bits,
  el original es \(\langle a_1, a_2, \dotsc, a_n \rangle\),
  la copia local es \(\langle b_1, b_2, \dotsc, b_n \rangle\).
  Algoritmos de \emph{\foreignlanguage{english}{checksum}} estándar
  garantizan
  que para la \textquote{mayoría} de los vectores \(\mathbf{a}\) y \(\mathbf{b}\)
  van a detectar si no son iguales.
  Acá la garantía
  es sobre una distribución de \(\mathbf{a}\) y \(\mathbf{b}\).
  Para los algoritmos comunes hay técnicas para \textquote{falsificar} CRC
  (ver por ejemplo Stigge y otros~%
     \cite{stigge06:_reversing_crc}),
  con lo que esto no es suficiente.
  Nos interesa únicamente acotar el tráfico de datos
  entre el origen y nosotros,
  el costo de cómputo es secundario.

  Nos interesa analizar el peor caso,
  interesa una garantía de la forma:
  para \emph{todo} par de vectores \(\mathbf{a}\) y \(\mathbf{b}\),
  elegiremos algunos valores al azar,
  y para la mayoría de los valores elegidos el algoritmo
  detectará si hay diferencias.
  Esta garantía no depende de \textquote{buenos} o \textquote{malos} vectores
  \(\mathbf{a}\) y \(\mathbf{b}\),
  solo de posiblemente \textquote{malos} valores elegidos.

  Nuestro algoritmo se basa en considerar los vectores
  como coeficientes de polinomios sobre un campo finito \(\mathbb{F}_p\).
  Recordemos el siguiente teorema
  (ver el apunte de Fundamentos de Informática~%
    \cite[sección~9.3]{brand17:_fundamentos_informatica}):
  \begin{theorem}
    Sea \(f(x)\) un polinomio no-cero de grado a lo más \(d\)
    sobre un campo.
    Entonces \(f\) tiene a lo más \(d\) ceros
    (hay a lo más \(d\) valores de \(x\) en el campo tales que \(f(x) = 0\)).
  \end{theorem}
  El primer paso es elegir un primo \(p \in [2 n, 4 n]\)
  (el postulado de Bertrand~%
     \cite{erdos30:_beweis_satz_tschebyschef}
   asegura que entre \(m\) y \(2 m\) siempre hay un primo,
   podemos buscar en el rango hasta hallar uno;
   los primos son relativamente numerosos,
   esto no es demasiado costoso).
  Enseguida,
  construya los polinomios sobre \(\mathbb{F}_p\):
  \begin{align*}
    a(x)
      &= \sum_{1 \le k \le n} a_k x^k \\
    b(x)
      &= \sum_{1 \le k \le n} b_k x^k
  \end{align*}
  Sea \(g(x) = a(x) - b(x)\).
  Note que \(g(x)\) es el polinomio cero
  si y solo si \(\mathbf{a} = \mathbf{b}\);
  si \(\mathbf{a} \ne \mathbf{b}\),
  \(g(x)\) es un polinomio de grado a lo más \(n\),
  que tiene a lo más \(n\) ceros.
  Si seleccionamos \(x \in \mathbb{F}_p\) uniformemente al azar,
  la probabilidad que \(g(x) = 0\) es a lo más:
  \begin{equation*}
    \frac{n}{\lvert \mathbb{F}_p \rvert}
      = \frac{n}{p}
      \le \frac{1}{2}
  \end{equation*}

  Esto sugiere el algoritmo~\ref{alg:comparar-archivos}.
  \begin{algorithm}[ht]
    \DontPrintSemicolon\Indp

    Agree on prime \(p\) with the origin \;
    \(k \gets 0\) \;
    \For{\(k \gets 1\) \KwTo \(m\)}{
      Origin selects \(x \in \mathbb{F}_p\) uniformly at random
        and computes \(a(x)\) \;
      Origin sends \(x\) and \(a(x)\) \;
      Destination computes \(b(x)\) \;
      \If{\(a(x) \ne b(x)\)}{
        \Return Different \;
      }
    }
    \Return Probably equal \;
    \caption{Comparar archivos remotos}
    \label{alg:comparar-archivos}
  \end{algorithm}

  Vemos que este algoritmo nunca dice \textquote{diferentes} por equivocación,
  se equivoca cada vez a lo más \(1/2\) de las veces,
  en \(m\) iteraciones la probabilidad de error es a lo más \(2^{-m}\).
  Intercambiamos \(m\) números en \([2 n, 4 n]\),
  el tráfico total es \(O(m \log n)\).

  Otra opción es elegir \(p \in [ r n, 2 r n]\),
  lo que con una iteración da probabilidad de falla a lo más \(1/r\),
  si esto es \(2^{-m}\) es \(r = 2^m\)
  y los bits intercambiados
  son \(O(\log p) = O(\log n + \log r) = O(\log n + m)\),
  mejor que lo anterior.

  Lo que discutimos acá es un ejemplo de lo que Karp~%
    \cite{karp91:_intro_randomized_algorithms}
  llama \emph{\foreignlanguage{english}{fingerprinting}},
  representar una estructura grande y compleja
  por una huella digital pequeña.
  Si dos estructuras tienen la misma huella digital,
  es fuerte evidencia de que en realidad son iguales.
  Una huella elegida al azar dificulta ataques o coincidencias,
  y puede repetirse para aumentar la confianza.

\section{Patrón en una palabra}
\label{sec:patron}

  Una tarea común es determinar si un patrón \(\sigma\)
  aparece en una palabra \(\omega\).
  El algoritmo obvio compara el patrón en cada posición de \(\omega\),
  dando un algoritmo determinista cuadrático
  (\(O(\lvert \sigma \rvert \cdot \lvert \omega \rvert)\)).
  Hay algoritmos deterministas lineales
  (\(O(\lvert \sigma \rvert + \lvert \omega \rvert)\)),
  como el de Knuth-Morris-Pratt~%
    \cite{knuth77:_fast_pattern_match_strings}
  y el de Boyer-Moore~%
    \cite{boyer77:_fast_string_search_algor}
  con la modificación de Galil~%
    \cite{galil79:_improving_worst_case_running_time},
  pero son muy complicados.

  Discutiremos el algoritmo de Karp y Rabin~%
    \cite{karp87:_efficient_random_pattern_match_algor}.
  Sigue la idea de fuerza bruta de ubicar el patrón en cada posición,
  pero en vez de comparar el patrón con la palabra compara una huella digital,
  fácil de calcular y de actualizar.
  Para simplificar lo que viene,
  sean \(n = \lvert \sigma \rvert\)
  y \(m = \lvert \omega \rvert\).
  Sea también \(b \ge \lvert \Sigma \rvert\) una base conveniente.
  Usaremos por ejemplo \(\omega_i\)
  para representar el \(i\)\nobreakdash-ésimo símbolo de \(\omega\).
  Para abreviar,
  anotaremos además \(\omega_{[i, j]}\)
  para referirnos al rango \(\omega_i \omega_{i + 1} \dots \omega_j\).
  Sea \(p\) un primo,
  elegido al azar en el rango \([1, n m^2]\).
  Usamos la huella digital
  (nuestras operaciones son en \(\mathbb{Z}_p\)):
  \begin{equation*}
    h(\sigma)
      = \sigma_1 \cdot b^{n - 1}
          + \sigma_2 \cdot b^{n - 2}
          + \dotsb
          + \sigma_n
  \end{equation*}
  Lo crítico
  es que es fácil actualizar \(h\) al eliminar el primer símbolo
  y agregar uno nuevo:
  \begin{equation*}
    h(\omega_{[i + 1, i + n]})
      = (h(\omega_{[i , i + n - 1]}) - \omega_i b^{n - 1})
           \cdot b + \omega_{i + n}
  \end{equation*}
  Esto da lugar al algoritmo~\ref{alg:Karp-Rabin}.
  \begin{algorithm}[ht]
    \DontPrintSemicolon\Indp

    Select \(p\) at random as indicated \;
    \BlankLine
    \(h \gets 0\) \;
    \For{\(k \gets 1\) \KwTo \(n\)}{
      \(h \gets h \cdot b + \sigma_k\) \;
    }
    \BlankLine
    \(s \gets 0\) \;
    \For{\(k \gets 1\) \KwTo \(n\)}{
      \(s \gets s \cdot b + \omega_k\) \;
    }
    \BlankLine
    \(k \gets n\) \;
    \While{\((s \ne h) \wedge (k < m)\)}{
      \(k \gets k + 1\) \;
      \(s \gets (s - \omega_{k - n} \cdot b^{n - 1}) \cdot b
                       + \omega_k\) \;
    }
    \eIf{\(s = h\)}{
      \Return Probable match at \(k - n\) \;
    }{
      \Return No match \;
    }
    \caption{El algoritmo de Karp-Rabin para calces de patrones}
    \label{alg:Karp-Rabin}
  \end{algorithm}
  Podemos verificar probables calces comparando símbolo a símbolo.

  Si \(h(\alpha) \ne h(\beta)\),
  es claro que \(\alpha \ne \beta\).
  Nos interesa el caso en que \(\alpha \ne \beta\),
  pero \(h(\alpha) = h(\beta)\).
  Un adversario que conoce el funcionamiento de nuestro algoritmo y \(p\)
  podría elegir \(\beta\) para forzar esto en muchas posiciones,
  haciendo que debamos recurrir a comparaciones inútiles
  (llegando a tiempo \(O(n m)\)).
  % Completar el análisis... ver el paper de Karp-Rabin
  %   karp87:_efficient_random_pattern_match_algor

% To do:
% Binary Planar Partition, autopartition ~~-> presentación de Raghavan
%  raghavan:_randomized_algorithms
% More algorithms (Karger, skip lists, max cut)
% Exercises!

\bibliography{../referencias}

%%% Local Variables:
%%% mode: latex
%%% TeX-master: "../INF-221_notas"
%%% ispell-local-dictionary: "spanish"
%%% End:

% LocalWords:  Aleatorizados aleatorizados english randomized Carlo
% LocalWords:  algorithms polinomial aleatorizado Lázló Babai City em
% LocalWords:  Manhattan Atlantic one sided two biased ht heurístico
% LocalWords:  probabilístico primalidad Quicksort skip lists treaps
% LocalWords:  tree heap probabilísticas Paul Alon Spencer Crosby RSA
% LocalWords:  desaleatorizarlo muestreando fingerprint Reordenar St
% LocalWords:  reordenar deadlock probabilísticos contrapositivo with
% LocalWords:  LibTomMath Write odd Pick at random from Probably Now
% LocalWords:  Composite redescubierto Select quadratic residue pr RT
% LocalWords:  MalaLeche eferencia emacs oyecto Bin Packing clique eq
% LocalWords:  linearidad subgrafos permutaciones permutación CRC rt
% LocalWords:  recursión amarretes checksum Cyclic Redundancy Check
% LocalWords:  fingerprinting ésimo false true Nothing aleatorizada
% LocalWords:  ru is quicksort comparisons Agree on the origin and
% LocalWords:  selects uniformly sends Destination Different equal
% LocalWords:  indicated match

\bibliographystyle{babplain-fl}

\chapter{Algoritmos aproximados}
\label{cha:algoritmos-aproximados}

  Vimos que muchos problemas de búsqueda importantes
  son \NP\nobreakdash-completos,
  con lo que la esperanza de obtener soluciones exactas es muy remota
  para instancias grandes.
  Nos contentaremos entonces con aproximaciones,
  y nos interesa saber qué tan buenas son.
  La presente clase se adapta de Dasgupta, Papadimitou y Vazirani~%
    \cite{dasgupta06:_algorithms}
  y de Mount~%
    \cite{mount03:_cmsc451}.
  Un texto que cubre el área en detalle es el de Vazirani~%
    \cite{vazirani03:_approx_algor},
  un borrador previo es~\cite{vazirani03:_approx_algor}.

  Siendo importantes problemas prácticos,
  simplemente abandonar no es opción.
  Nuestras alternativas generales son:
  \begin{description}
  \item[Usar fuerza bruta:]
    O al menos algún mecanismo de búsqueda exhaustiva,
    como \emph{\foreignlanguage{english}{branch and bound}} o \(A^*\).
    Hasta con los computadores paralelos más grandes,
    esto suele ser viable solo con instancias pequeñas del problema.
  \item[Algoritmos especializados:]
    Para problemas de gran importancia práctica
    se han desarrollado una variedad de algoritmos especializados,
    exactos o aproximados.
  \item[Heurísticas:]
    Una \emph{heurística} es una técnica que construye una solución válida,
    sin garantía de qué tan cerca está del óptimo.
    La heurística se basa en algún criterio
    que sugiere que la solución es buena.
    Se aplica a falta de otras opciones,
    o si basta una solución válida
    y no es muy relevante qué tan cerca del óptimo está.
  \item[Usar un algoritmo aleatorizado:]
    Esto se discute en el capítulo~\ref{cha:randomized-algorithms}.
    Generalmente entregan una solución aproximada con costo moderado,
    una opción es correr el algoritmo varias veces
    y retener la mejor solución.
  \item[Algoritmo aproximado:]
    Es un algoritmo que se ejecuta en tiempo polinomial
    (idealmente),
    y entrega una solución
    que está dentro de un factor garantizado del óptimo.
  \end{description}

\section{Cotas de rendimiento}
\label{sec:cotas-de-rendimiento}

  Por razones teóricas,
  planteamos los problemas \NP\nobreakdash-completos
  como problemas de decisión
  (¿Hay una \emph{\foreignlanguage{english}{clique}} de tamaño \(k\)
   en el grafo \(G\)?,
   ¿Es satisfacible la fórmula \(\phi\)?),
  pero vimos que el problema de búsqueda relacionado
  (Dé una \emph{\foreignlanguage{english}{clique}} de tamaño \(k\)
   del grafo \(G\),
   si la hay.
   Dé una asignación de valores a las variables de \(\phi\)
   que la hacen verdadera,
   si existe.)
  o de optimización
  (Dé una \emph{\foreignlanguage{english}{clique}} de máximo tamaño
   del grafo \(G\).
   Indique el ciclo de mínimo costo en un grafo dirigido
   con arcos rotulados con costos.)
  son \textquote{equivalentes} para problemas \NP\nobreakdash-completos.
  No nos extenderemos en esto,
  detalles dan Bellare y Goldwasser~%
    \cite{bellare94:_complexity_decision_search},
  un resumen accesible es el de Bellare~%
    \cite{bellare10:_decision_search}.
  Note que a veces maximizamos y otras minimizamos.
  Un algoritmo aproximado entrega una solución válida,
  no necesariamente óptima.

  ¿Cómo medir qué tan buena es la aproximación?
  Dada una instancia \(I\) del problema,
  llamamos \(C(I)\) al costo de la solución provista por el algoritmo,
  y \(C^*(I)\) al costo óptimo respectivo.
  Supondremos que los costos son estrictamente positivos,
  con lo que buscamos \(C(I) / C^*(I)\) pequeño si es minimización,
  para maximización \(C^*(I) / C(I)\) pequeño.
  Para datos de tamaño \(n\),
  decimos que el algoritmo logra \emph{cota de razón} \(\rho(n)\)
  si para todas las instancias \(I\):
  \begin{equation*}
    \max_{\lvert I \rvert = n}
      \left\{
        \frac{C(I)}{C^*(I)},
        \frac{C^*(I)}{C(I)}
      \right\}
      \le \rho(n)
  \end{equation*}
  Note que \(\rho(n)\) siempre es mayor o igual a \num{1},
  siendo \num{1} solo si la solución aproximada es el óptimo.

  Algunos problemas \NP\nobreakdash-completos
  pueden aproximarse arbitrariamente.
  Para ellos hay un algoritmo
  que toma una instancia \(I\) y un real \(\varepsilon > 0\),
  se ejecuta en tiempo polinomial en \(n = \lvert I \rvert\)
  y retorna una solución cuya cota de razón es a lo más \(1 + \varepsilon\).
  A este tipo de algoritmo se le llama
  \emph{\foreignlanguage{english}{polynomial time approximation scheme}}
  o PTAS.
  El tiempo de ejecución depende de \(n\) y \(\varepsilon\),
  en el caso de problemas \NP\nobreakdash-completos
  en muchos casos al disminuir \(\varepsilon\)
  el tiempo de ejecución crece más allá de polinomial,
  como es \(O(n^{1/\varepsilon})\).
  Si el tiempo de ejecución es polinomial en \(n\) y \(1/\varepsilon\),
  se habla de
    \emph{\foreignlanguage{english}
                          {fully polynomial time approximation scheme}}
  o FPTAS.
  Ejemplo es tiempo de ejecución \(O((1/\varepsilon)^2 n^3)\),
  mientras \(O(n^{1/\varepsilon})\) y \(O(2^{1/\varepsilon} n^2)\) no lo son.

  Aunque los problemas \NP\nobreakdash-completos son equivalentes
  respecto de si sus peores casos
  pueden resolverse exactamente en tiempo polinomial,
  su aproximabilidad varía considerablemente.
  \begin{itemize}
  \item
    Para algunos problemas,
    es muy poco probable que haya algoritmos aproximados.
    Por ejemplo,
    en el caso general del problema del vendedor viajero
    (\textsc{TSP}),
    si hay un algoritmo aproximado con una cota de razón menor que \(\infty\),
    entonces \(\cP = \NP\).
  \item
    Algunos problemas pueden aproximarse,
    pero la cota es una función que crece lentamente con \(n\).
    Por ejemplo
    \textsc{Set~Cover}
    puede aproximarse dentro de un factor de \(\ln n\).
  \item
    Hay problemas que se pueden aproximar dentro de un factor fijo,
    veremos un par de ejemplos más adelante.
  \item
    Hay problemas que tienen PTAS o FPTAS.
  \end{itemize}
  De hecho,
  similar al caso de los problemas \NP\nobreakdash-completos,
  hay colecciones de problemas que \textquote{se cree} que son equivalentes
  en que son difíciles de aproximar,
  y que si uno puede aproximarse en tiempo polinomial
  todos ellos pueden serlo.
  Pero el estudio de algoritmos aproximados
  sirve para llenar varios cursos\ldots

\section{Algunos algoritmos aproximados}
\label{sec:algoritmos-aproximados}

  Veremos algunos ejemplos de algoritmos aproximados,
  y las demostraciones de su rendimiento.
  Volveremos sobre algunos de los ejemplos ya tratados para ello.

\subsection{El problema \textsc{Vertex Cover}}
\label{sec:aproximar-VC}

  El problema  de optimización \textsc{Vertex~Cover}
  toma un grafo \(G = (V, E)\)
  y solicita un subconjunto mínimo \(C \subseteq V\)
  tal que todos los arcos de \(G\) contienen al menos un vértice en \(C\).

  Sabemos que el problema de decisión correspondiente
  (¿tiene \(G\) un \emph{\foreignlanguage{english}{vertex cover}}
   de tamaño \(k\)?)
  es \NP\nobreakdash-completo.
  Demostraremos que hay un algoritmo con cota de razón \num{2},
  o sea,
  obtiene un conjunto que a lo más tiene el doble tamaño del mínimo.

  Una manera de diseñar un algoritmo aproximado es tomar una heurística,
  alguna \textquote{estrategia razonable},
  técnica que en muchos casos da buenos resultados.
  Acá tenemos un algoritmo muy simple,
  que se basa en una observación obvia:
  si consideramos el arco \(u v \in E\),
  al menos uno de los dos vértices pertenece al conjunto,
  no sabemos cuál.
  Agregamos ambos a nuestro conjunto
  (más tonto,
   imposible),
  luego eliminamos los arcos que contienen los vértices \(u, v\),
  y repetimos el proceso con el resto.
  Llamemos \(\mathrm{ApproxVC}\) al algoritmo~\ref{alg:ApproxVC}.
  Note que es un algoritmo aleatorizado
  (elige un arco cualquiera en cada paso).
  \begin{algorithm}[ht]
    \DontPrintSemicolon\Indp

    \Function{\(\mathrm{ApproxVC}(G)\)}{
      \(C \gets \varnothing\) \;
      \While{\(E \ne \varnothing\)}{
        Let \(u v \in E\) be any edge \;
        \(C \gets C \cup \{u, v\}\) \;
        Delete all edges incident on \(u\) or \(v\) \;
      }
      \Return \(C\) \;
    }
    \caption{El algoritmo \(\mathrm{ApproxVC}\)}
    \label{alg:ApproxVC}
  \end{algorithm}
  \begin{proposition}
    \label{prop:ApproxVC}
    El algoritmo \(\mathrm{ApproxVC}\) tiene cota de razón \num{2}
    para \textsc{Vertex~Cover}.
  \end{proposition}
  \begin{proof}
    Sea \(C\) el conjunto que entrega \(\mathrm{ApproxVC}\),
    y sea \(C^*\) el mínimo.
    Sea \(A\) el conjunto de arcos
    seleccionados por \(\mathrm{ApproxVC}\),
    vemos que \(\lvert C \rvert = 2 \lvert A \rvert\)
    ya que cada arco seleccionado aporta \num{2} vértices a \(C\).
    Pero el \emph{\foreignlanguage{english}{vertex cover}} mínimo
    tiene que contener al menos uno de esos dos vértices,
    con lo que \(\lvert A \rvert \le \lvert C^* \rvert\).
    Tenemos:
    \begin{equation*}
      \frac{\lvert C \rvert}{2}
        = \lvert A \rvert
        \le \lvert C^* \rvert
    \end{equation*}
    vale decir:
    \begin{equation*}
      \frac{\lvert C \rvert}{\lvert C^* \rvert}
        \le 2
    \end{equation*}
    \qedhere
  \end{proof}
  Este ejemplo es típico:
  necesitamos hallar alguna cota para la solución óptima
  en términos de lo que entrega o hace el algoritmo.

  Hay una familia infinita de grafos
  para los cuales \(\mathrm{ApproxVC}\) da cota de razón
  exactamente \num{2}:
  en el grafo bipartito completo \(K_{n, n}\)
  el algoritmo \(\mathrm{ApproxVC}\) elige todos los vértices,
  para un total de \(2 n\);
  el óptimo es uno de los conjuntos de vértices,
  con \(n\) vértices.
  A esto se le llama \emph{ejemplo ajustado}
  (\emph{\foreignlanguage{english}{tight example}} en inglés)
  para el algoritmo.

  Otra idea es una estrategia voraz:
  ¿porqué no concentrarse en vértices que cubren el máximo número de arcos?
  O sea,
  ir agregando vértices de grado máximo.
  Esto da el algoritmo~\ref{alg:GreedyVC}.
  \begin{algorithm}[ht]
    \DontPrintSemicolon\Indp

    \Function{\(\mathrm{GreedyVC}(G)\)}{
      \(C \gets \varnothing\) \;
      \While{\(E \ne \varnothing\)}{
        Let \(u\) be the vertex of maximum degree in \(G\) \;
        \(C \gets C \cup \{ u \}\) \;
        Delete all edges containing \(u\) from \(E\) \;
      }
      \Return \(C\) \;
    }
    \caption{El algoritmo \(\mathrm{GreedyVC}\)}
    \label{alg:GreedyVC}
  \end{algorithm}
  Este algoritmo no siempre da una solución mejor que la estrategia estúpida
  de dos-por-uno.
  Sorprendentemente,
  actualmente \(\mathrm{ApproxVC}\)
  es el algoritmo aproximado que garantiza la mejor cota de razón
  para este problema.
  Es un ejercicio moderadamente difícil demostrar
  que para \(\mathrm{GreedyVC}(G)\)
  la cota de razón crece como \(\Theta(\log n)\),
  donde \(n\) es el número de vértices del grafo,
  ni siquiera es una constante.
  Vale decir,
  puede comportarse arbitrariamente peor que \(\mathrm{ApproxVC}\).
  Por otro lado,
  en \textquote{grafos típicos}
  suele dar mejores resultados que \(\mathrm{ApproxVC}\),
  vale la pena correr ambos
  (y ya que \(\mathrm{ApproxVC}\) es aleatorizado,
   correr este varias veces)
  y quedarse con el conjunto más pequeño.
  O combinar:
  usar \(\mathrm{ApproxVC}\),
  pero elegir siempre el arco \(u v\) con máximo \(\delta(u) + \delta(v)\),
  con la idea de eliminar los más arcos posibles.

  La moraleja es que no siempre la heurística \textquote{obvia}
  da el mejor resultado,
  puede ser necesario considerar varias alternativas.

\subsection{El problema del vendedor viajero}
\label{sec:vendedor-viajero}

  El problema del vendedor viajero
  (\emph{\foreignlanguage{english}{Travelling Salesman Problem}},
   o su versión \textquote{sexistamente correcta}
   \emph{\foreignlanguage{english}{Travelling Salesperson Problem}},
   abreviado \textsc{TSP})
  es el niño símbolo de lo \NP\nobreakdash-completo.
  Tenemos un grafo \(G = (V, E)\)
  con costos de los arcos,
  \(c \colon E \to \mathbb{R}^+\).
  Buscamos un ciclo de costo mínimo
  (sumando los costos de los arcos)
  que visite cada vértice exactamente una vez.

\subsubsection{No hay algoritmo aproximado para \textsc{TSP}}
\label{sec:TSP-no-aproximable}

  El problema \textsc{TSC} es similar al problema de ciclo Hamiltoniano
  (abreviado \textsc{Ham}),
  que dado un grafo \(G = (V, E)\)
  pregunta si existe un ciclo que pasa por cada vértice exactamente una vez.
  Usaremos esta similitud para demostrar
  que es imposible aproximar \textsc{TSP}.

  \begin{theorem}
    \label{theo:TSP-no-aproximable}
    No hay un algoritmo aproximado polinomial para \textsc{TSP}
    que dé una cota de razón menor que \(\infty\)
    a menos que \(\cP = \NP\).
  \end{theorem}
  \begin{proof}
    Esto lo haremos reduciendo el problema \textsc{Ham}
    a obtener una aproximación a \textsc{TSP}.
    Sea \(G = (V, E)\) una instancia de \textsc{Ham},
    y sea \(n = \lvert V \rvert\),
    elegimos una constante \(C\) y creamos una instancia de \textsc{TSP}
    con el grafo \(G' = K_n\) sobre los vértices \(V\),
    con:
    \begin{equation*}
      c(u, v)
        = \begin{cases}
            1 & u v \in E \\
            C & u v \notin E
          \end{cases}
    \end{equation*}
    Es claro que nuestra instancia de \textsc{TSP}
    tiene solución de costo total \(n\)
    si y solo si la instancia de \textsc{Ham} tiene solución.
    Si \textsc{Ham} no tiene solución,
    el costo de la travesía es a lo menos \(n - 1 + C\).
    Si hubiese un algoritmo polinomial
    que dé \(\rho(n) \le (C + n - 1) / n\),
    tendríamos un algoritmo polinomial para \textsc{Ham}
    (si da una solución de costo menor a \(C + n - 1\),
     tiene costo \(n\),
     quiere decir que \textsc{Ham} tiene solución).
    Pero \(C\) es arbitrario
    y \textsc{Ham} es \NP\nobreakdash-completo.
  \end{proof}

\subsubsection{Vendedor viajero con desigualdad triangular}
\label{sec:TSP-triangle}

  Una variante de \textsc{TSP},
  que llamaremos \(\textsc{TSP}_\Delta\),
  se da si para todos los vértices
  los costos cumplen la desigualdad triangular:
  \begin{equation*}
    c(u, v)
      \le c(u, x) + c(x, v)
  \end{equation*}
  Un caso especial de esto es cuando los vértices son puntos en el plano
  y los costos son las distancias entre ellos
  (\emph{vendedor viajero euclidiano},
   aplicable por ejemplo a movimientos de la pluma de un \emph{plotter}).

  \begin{theorem}
    \label{theo:TSC-triangle-approx}
    Para \(\textsc{TSP}_\Delta\)
    hay un algoritmo polinomial que asegura cota de razón \num{2}.
  \end{theorem}
  \begin{proof}
    Primeramente,
    sea \(C^*\) el costo del ciclo óptimo para la instancia de \textsc{TSP}.
    Si eliminamos uno de los arcos del ciclo
    obtenemos un árbol recubridor del grafo,
    cuyo costo es a lo menos el del árbol recubridor mínimo del grafo,
    llamémosle \(T^*\).
    Por otro lado,
    podemos construir un circuito que visita cada vértice dos veces
    partiendo de un árbol recubridor mínimo:
    recorrer el árbol \textquote{por fuera} da un circuito
    cuyo costo es \(2 T^*\);
    si en el recorrido
    (básicamente un recorrido en preorden)
    evitamos los vértices ya visitados,
    obtenemos un ciclo,
    como estamos evitando vértices,
    por la desigualdad triangular el costo es menor.
    Resumiendo,
    obtenemos un recorrido de costo \(C\)
    que cumple:
    \begin{equation*}
      T^*
        <	  C^*
        \le C
        \le 2 T^*
        <	  2 C^*
    \end{equation*}
    de donde tenemos:
    \begin{equation*}
      \frac{C}{C^*}
        < 2
    \end{equation*}
    \qedhere
  \end{proof}

\subsection{El problema \textsc{Set Cover}}
\label{sec:SetCover}

  Otro problema \NP\nobreakdash-completo famoso
  es \textsc{Set~Cover}:
  Dado un conjunto universo finito \(\mathscr{U}\)
  y una familia finita de subconjuntos \(\mathscr{S}_i\),
  \(1 \le i \le n\),
  tal que \(\bigcup_i \mathscr{S}_i = \mathscr{U}\)
  (los \(\mathscr{S}_i\) \emph{no} son necesariamente disjuntos),
  se busca la colección mínima de los \(\mathscr{S}_i\)
  tal que:
  \begin{equation*}
    \bigcup_{i \in \mathscr{C}} \mathscr{S}_i
      = \mathscr{U}
  \end{equation*}

  El problema \textsc{Vertex~Cover} es un caso particular,
  los arcos de \(G\) son subconjuntos de tamaño \num{2} de los vértices,
  y nos interesa la colección mínima de arcos
  que incluyen a todos los vértices.
  La estrategia que llevó a \(\mathrm{ApproxVC}\) acá no es aplicable,
  los conjuntos \(\mathscr{S}_i\) no son necesariamente del mismo tamaño.

  Una heurística plausible
  es la estrategia voraz de elegir en cada paso
  aquel subconjunto que más elementos aún no considerados incluye.
  Esta heurística puede ser engañada,
  como ilustra la figura~\ref{fig:SetCover-fools-greedy}.
  Dado un conjunto de \num{14} elementos,
  lo dividimos en dos subconjuntos de \num{7};
  también en conjuntos de \num{2}, \num{4} y \num{8} disjuntos,
  pero que intersectan de mitades con los primeros conjuntos.
  \begin{figure}[ht]
    \centering
    \begin{tikzpicture}
      \tikzstyle{dot} = [draw, shape = circle, fill];
      \tikzstyle{set} = [draw, shape = rounded rectangle]
      \path (0, 1) node[dot] (a0) {}
            (1, 1) node[dot] (a1) {}
            (2, 1) node[dot] (a2) {}
            (3, 1) node[dot] (a3) {}
            (4, 1) node[dot] (a4) {}
            (5, 1) node[dot] (a5) {}
            (6, 1) node[dot] (a6) {};
      \path (0, 0) node[dot] (b0) {}
            (1, 0) node[dot] (b1) {}
            (2, 0) node[dot] (b2) {}
            (3, 0) node[dot] (b3) {}
            (4, 0) node[dot] (b4) {}
            (5, 0) node[dot] (b5) {}
            (6, 0) node[dot] (b6) {};
      \node[set, fit = (a0) (a6)] {};
      \node[set, fit = (b0) (b6)] {};
      \node[set, inner sep = 5pt, fit = (a0) (b0)] {};
      \node[set, inner sep = 5pt, fit = (a1) (b2)] {};
      \node[set, inner sep = 5pt, fit = (a3) (b6)] {};
    \end{tikzpicture}
    \caption{Engañando a la heurística para \textsc{Set Cover}}
    \label{fig:SetCover-fools-greedy}
  \end{figure}
  La estrategia voraz elegiría el conjunto de \num{8} elementos,
  luego el de \num{4},
  y finalmente el de \num{2};
  la solución óptima es elegir simplemente ambos conjuntos de \num{7}.
  Este ejemplo puede extenderse a dos conjuntos de \(2^n - 1\)
  para cualquier \(n\)
  y los respectivos conjuntos de \(2 \cdot 2^k\)
  que intersectan a los conjuntos grandes en mitades,
  similar a la figura.
  Igual que en el ejemplo,
  la estrategia voraz elegiría los \(n\)~conjuntos pequeños,
  siendo que la solución óptima es elegir los dos conjuntos grandes.
  No hay límite a la cota de razón del algoritmo voraz.

  Antes de ir al resultado,
  recordamos una cota importante.
  \begin{lemma}
    \label{lem:desigualdad-importante}
    Para todo \(c > 0\):
    \begin{equation*}
      \left( 1 - \frac{1}{c} \right)^c
        \le \frac{1}{\mathrm{e}}
    \end{equation*}
  \end{lemma}
  \begin{proof}
    Usamos que para todo \(x\) es \(1 + x \le \mathrm{e}^x\)
    (son iguales para \(x = 0\) únicamente).
    Substituyendo \(x = - 1 / c\) tenemos \(1 - 1/c \le \mathrm{e}^{-1/c}\),
    elevando a la potencia \(c\) tenemos lo prometido.
  \end{proof}
  Tenemos:
  \begin{theorem}
    \label{theo:SetCover-greedy}
    El algoritmo voraz para \textsc{Set~Cover} tiene una cota de razón
    de a lo más \(\ln m\),
    donde \(m = \lvert \mathscr{U} \rvert\).
  \end{theorem}
  \begin{proof}
    Sea \(c\) el tamaño del \emph{\foreignlanguage{english}{set cover}} óptimo,
    y sea \(g\) el tamaño del \emph{\foreignlanguage{english}{set cover}}
    entregado por el algoritmo voraz menos uno.
    Demostraremos que \(g/c \le \ln m\),
    lo que no es exactamente lo prometido,
    pero está a \num{1} de distancia.

    Al iniciar el algoritmo voraz,
    tenemos \(m_0 = m\) elementos a cubrir.
    Sabemos que hay una cobertura de tamaño \(c\)
    (la óptima),
    con lo que por el principio del palomar
    hay a lo menos un conjunto de \(m_0 / c\) elementos
    (si todos fueran menores a \(m_0 / c\),
     en total \(c\) conjuntos no cubrirían \(m_0\) elementos).
    Al elegir el conjunto más grande,
    el algoritmo voraz elige un conjunto con al menos \(m_0 / c\) elementos,
    quedando \(m_1 \le m_0 - m_0 / c = m_0( 1 - 1/c)\) elementos a cubrir.
    Aplicando el mismo argumento,
    hay forma de cubrir los \(m_1\) elementos restantes
    con \(c - 1\) conjuntos,
    quedan \(m_2 \le m_1 (1 - 1 / (c - 1)) \le m_0 (1 - 1 / c)^2\),
    y así sucesivamente.
    Aplicando este argumento \(g\) veces,
    cada vez cubrimos una fracción de al menos \(1 - 1/c\) de los elementos,
    los elementos que restan al final del algoritmo voraz es a lo más
    \(m (1 - 1/c)^g\).
    El valor máximo de \(g\)
    para el cual queda al menos un elemento sin cubrir es:
    \begin{align*}
      1
        &\le m \left( 1 - \frac{1}{c} \right)^g \\
        &=   m \left( \left( 1 - \frac{1}{c} \right)^c \right)^{g/c} \\
    \intertext{Por el lema~\ref{lem:desigualdad-importante}:}
      1
        &\le m \mathrm{e}^{-g/c} \\
    \intertext{Multiplicando por \(\mathrm{e}^{g/c}\) y tomando logaritmos:}
      \frac{g}{c}
        &\le \ln m
    \end{align*}
    \qedhere
  \end{proof}
  En la práctica,
  la cota del teorema~\ref{theo:SetCover-greedy}
  resulta ser demasiado pesimista.

\subsection{El problema de la mochila}
\label{sec:mochila-aproximado}

  El problema de la mochila
  (\textsc{Knapsack})
  es otro problema \NP\nobreakdash-completo importante.
  Tenemos una mochila de capacidad \(M\)
  y \(n\) objetos,
  con el objeto \(i\) de peso \(p_i\) y valor \(v_i\).
  Buscamos el conjunto de objetos que da el valor máximo
  que no sobrepasa la capacidad.
  Este problema podemos plantearlo como:
  \begin{align*}
    &\max \sum_i x_i v_i \\
    &\text{tal que} \sum_i x_i p_i \le M \\
    &x_i \in \{0, 1\}
  \end{align*}
  Claramente objetos de peso mayor que \(M\)
  nunca pueden formar parte de la solución,
  podemos suponer sin pérdida de generalidad que \(p_i < M\).
  Suponemos además que tanto \(M\) como los \(p_i\) son enteros.
  Nuestra discusión sigue a Nelson~%
   \cite{nelson14:_advanced_algorithms}.

\subsubsection{Algoritmos voraces}
\label{sec:knapsack-greedy-approx}

  Sabemos
  (sección~\ref{sec:fractional-knapsack})
  que si se permiten fracciones de objetos,
  el algoritmo voraz de incluir los objetos
  en orden de \(v_i/p_i\) decreciente,
  agregando todo lo que se puede del último es óptimo.
  Sin embargo,
  en este caso esto no funciona.
  Por ejemplo,
  con \(p_1 = 1, v_1 = 1 + \varepsilon, p_2 = M, v_2 = M\)
  elige solo el objeto \num{1},
  para un valor \(1 + \varepsilon\);
  siendo que eligiendo \num{2} obtiene un valor \(M\).
  La aproximación no es mejor que \(M\),
  y \(M\) es arbitrario.

\subsubsection{Un FPTAS para \textsc{Knapsack}}
\label{sec:FPTAS-Knapsack}

  La idea central es usar redondeo.
  Si \(V = \sum_i v_i\),
  sabemos que hay un algoritmo de programación dinámica
  que resuelve el problema en tiempo \(O(n V)\).
  Esto no es polinomial en el largo de los datos de entrada
  (dados en una representación eficiente,
   como es números binarios,
   lo que daría la representación del largo de \(V\)
   como \(O(\log V)\)).
  Para obtener un FPTAS,
  defina:
  \begin{equation*}
    v_i'
      = \left\lfloor
          \frac{n}{\varepsilon} \cdot \frac{v_i}{v_{\mathrm{max}}}
        \right\rfloor
  \end{equation*}
  Ejecute el algoritmo de programación dinámica
  para obtener la solución óptima \(\mathrm{OPT}'\)
  al problema modificado,
  con costo \(O(n V')\).
  Sabemos:
  \begin{equation*}
    V' \le n \max_i v_i' \le \frac{n^2}{\varepsilon}
  \end{equation*}
  Es claro que \(\mathrm{OPT}' \ge v'_{\mathrm{max}} \ge n / \varepsilon\).
  Después de redondear,
  todo conjunto de \(k\) objetos pierde a lo más \(k < n\) en valor,
  lo que
  (por lo anterior)
  es a lo más \(\varepsilon \mathrm{OPT}'\).

  \begin{proposition}
    Para \(\varepsilon > 0\),
    la solución óptima de la instancia redondeada
    tiene valor al menos \((1 - \varepsilon) \mathrm{OPT}\).
  \end{proposition}
  \begin{proof}
    Sea \(v(S)\) el valor del conjunto \(S\) de objetos,
    \(v'(S)\) el valor del conjunto con valores redondeados.
    Para todo conjunto \(S\):
    \begin{equation*}
      \alpha v(S) - \lvert S \rvert
        \le v'(S)
        \le \alpha v(S)
    \end{equation*}
    donde \(\alpha = n / (\varepsilon v_{\mathrm{max}})\).
    Sea \(A^*\) algún conjunto óptimo para el problema original sin redondear
    y \(A\) un conjunto óptimo del problema redondeado,
    entonces:
    \begin{equation*}
      v(A)
        \ge \frac{1}{\alpha} v'(A)
        \ge \frac{1}{\alpha} v'(A^*)
        \ge \frac{1}{\alpha} v(A^*) - \frac{1}{\alpha} \lvert A^* \rvert
        \ge \mathrm{OPT} - \frac{n}{\alpha}
        \ge \mathrm{OPT} - \varepsilon v_{\mathrm{max}}
        \ge \mathrm{OPT} - \varepsilon \mathrm{OPT}
    \end{equation*}
    \qedhere
  \end{proof}
  Dado \(\varepsilon > 0\),
  el esquema de redondeo nos da un algoritmo
  que entrega una solución de valor \((1 - \varepsilon) \mathrm{OPT}\)
  en tiempo \(O(n^3 / \varepsilon)\),
  es un FPTAS.

  Lawler~%
    \cite{lawler79:_fast_approx_algor_knaps_probl}
  da un FPTAS de tiempo \(O(n \log (1/\varepsilon) + 1/\varepsilon^4)\),
  una versión preliminar es~\cite{lawler77:_fast_approx_algor_knaps_probl}.

\section*{Ejercicios}
\label{sec:ejercicios-approximados}

  \begin{enumerate}
  \item
    Usando los grafos construidos de la siguiente forma:
    tenemos un nivel base de \(m\) vértices,
    un segundo nivel de \(m/2\) vértices,
    un tercer nivel de \(m/3\) vértices,
    \ldots,
    un último nivel de \num{1} vértice.
    Todos los arcos terminan en el nivel base;
    cada vértice del nivel base está conectado
    a un vértice de cada nivel siguiente,
    distribuidos de forma lo más equitativa posible.

    Demuestre que en estos grafos la cota de razón
    para el algoritmo \(\mathrm{GreedyVC}(G)\)
    cumple \(\Theta(\log n)\),
    donde \(n\) es el número de vértices del grafo.
  \item
    Muestre cómo extender el ejemplo
    de la figura~\ref{fig:SetCover-fools-greedy}
    para construir un ejemplo ajustado para la heurística voraz
    para \textsc{Set~Cover}.
  \end{enumerate}

\bibliography{../referencias}

%%% Local Variables:
%%% mode: latex
%%% TeX-master: "../INF-221_notas"
%%% ispell-local-dictionary: "spanish"
%%% End:

% LocalWords:  english branch and bound aleatorizado clique scheme on
% LocalWords:  satisfacible maximización polynomial approximation TSP
% LocalWords:  PTAS fully FPTAS aproximabilidad Set Cover Vertex Let
% LocalWords:  vertex cover any edge Delete all edges incident or the
% LocalWords:  tight example of maximum degree in containing from TSC
% LocalWords:  Travelling Salesman Problem sexistamente Salesperson
% LocalWords:  Hamiltoniano Ham plotter recubridor preorden set max
% LocalWords:  intersectan Knapsack OPT

\bibliographystyle{babplain-fl}

\chapter{Optimización de funciones submodulares}
\label{cha:optimizacion-submodulares}

  Vimos que muchos problemas de búsqueda importantes
  son \NP\nobreakdash-completos,
  con lo que la esperanza de obtener soluciones exactas es muy remota
  para instancias grandes.
  Una tentación obvia es usar un algoritmo voraz
  para obtener soluciones aproximadas.
  Sorprendentemente,
  para muchas situaciones naturales el algoritmo voraz
  garantiza una aproximación bastante buena.
  La exposición siguiente sigue a Horel~%
    \cite{horel15:_notes_greedy_algo_submodular},
  y toma de Kun~%
    \cite{kun14:_greedy_algo_good_enough}.
  El punto es que las funciones submodulares,
  que estudiaremos a continuación,
  son comunes como objetivos de optimización
  (por su descripción como retornos decrecientes).
  \begin{definition}
    \label{def:submodular}
    Sea \(\mathscr{U}\) un conjunto
    (el \emph{conjunto base}),
    y una función \(f \colon 2^{\mathscr{U}} \to \mathbb{R}\).
    Se dice que \(f\) es \emph{submodular} si para todos los conjuntos
    \(\mathscr{A} \subseteq \mathscr{B} \subseteq \mathscr{U}\)
    y todo elemento \(e \in \mathscr{U} \smallsetminus \mathscr{B}\):
    \begin{equation}
      \label{eq:submodular-def}
      f(\mathscr{A} \cup \{ e \}) - f(\mathscr{A})
        \ge f(\mathscr{B} \cup \{ e \}) - f(\mathscr{B})
    \end{equation}
  \end{definition}
  Esta definición dice esencialmente
  que el aporte de un elemento dado al valor de \(f\)
  no aumenta al crecer el conjunto.
  Es lo que suelen llamar \textquote{ley de retornos decrecientes},
  lo que se puede ganar disminuye conforme aumenta lo que ya tenemos.
  Kun da un ejemplo:
  suponga que tiene un programa radial,
  hay una colección de estaciones radiales que pueden emitirlo.
  Cuenta con una estimación precisa de cuántos oyentes alcanza
  para cada combinación de radios contratadas.
  Nuestro conjunto base son las emisoras de radio.
  Agregar una radio dada a un conjunto de radios contratadas
  a lo más aporta el total de sus oyentes,
  como sus oyentes pueden traslapar con otras radios,
  agregarla a un conjunto mayor no puede hacer más aporte que en uno menor.
  \begin{definition}
    \label{def:submodular-monotone}
    Sea \(f\) una función submodular sobre el conjunto base \(\mathscr{U}\).
    Se dice que \(f\) es \emph{monótona}
    si para todo \(\mathscr{A} \subseteq \mathscr{B} \subseteq \mathscr{U}\)
    es \(f(\mathscr{A}) \le f(\mathscr{B})\).
  \end{definition}
  Vale decir,
  \(f\) nunca disminuye al agregar elementos.

  Un ejemplo simple de función submodular monótona
  sobre \(\mathscr{U} = \{1, 2, 3, 4, 5, 6\}\) es el máximo de un subconjunto.
  Por ejemplo,
  si \(\mathscr{A} = \{1, 2, 4\}\) y \(\mathscr{B} = \{1, 2, 4, 5\}\),
  para \(e \in \mathscr{U} \smallsetminus \mathscr{B} = \{3, 6\}\) tenemos:
  \begin{alignat*}{4}
    f(\mathscr{A} \cup \{3\}) - f(\mathscr{A})
      &= 4 - 4 = 0
      &\qquad&
      f(\mathscr{B} \cup \{3\}) - f(\mathscr{B})
      &= 5 - 5 = 0 \\
    f(\mathscr{A} \cup \{6\}) - f(\mathscr{A})
      &= 6 - 4 = 2
      &\qquad&
      f(\mathscr{B} \cup \{6\}) - f(\mathscr{B})
      &= 6 - 5 = 1
  \end{alignat*}
  Demostrar que en general para \(\mathscr{U} = [1, n]\) la función
  \(\max \colon 2^{\mathscr{U}} \to \mathbb{R}\) es submodular monótona
  es simple,
  y queda de ejercicio.
  \begin{definition}
    \label{def:marginal-utility}
    Sea \(f\) una función submodular sobre \(\mathscr{U}\),
    sea \(\mathscr{A} \subseteq \mathscr{U}\)
    y sea \(e \in \mathscr{U}\).
    Definimos la \emph{ganancia marginal} de \(e\) para \(\mathscr{A}\)
    como:
    \begin{equation*}
      \Delta(e \mid \mathscr{A})
        = f(\mathscr{A} \cup \{e\}) - f(\mathscr{A})
    \end{equation*}
    Extendemos lo anterior a conjuntos \(\mathscr{X} \subseteq \mathscr{U}\):
    \begin{equation*}
      \Delta(\mathscr{X} \mid \mathscr{A})
        = f(\mathscr{A} \cup \mathscr{X}) - f(\mathscr{A})
    \end{equation*}
  \end{definition}
  Nuestros primeros resultados describen funciones submodulares
  en formas distintas.
  \begin{theorem}
    \label{theo:submodular-intersection-union}
    La función \(f\) sobre \(\mathscr{U}\) es submodular
    si y solo si para todo \(\mathscr{S}, \mathscr{T} \subseteq \mathscr{U}\):
    \begin{equation}
      \label{eq:submodular-intersection-union}
      f(\mathscr{S} \cap \mathscr{T}) + f(\mathscr{S} \cup \mathscr{T})
       \le f(\mathscr{S}) + f(\mathscr{T})
    \end{equation}
  \end{theorem}
  \begin{proof}
    Demostramos implicancia en ambas direcciones.

    Primero,
    supongamos que \(f\) es submodular.
    Como \(\mathscr{S} \cap \mathscr{T} \subseteq \mathscr{T}\),
    aplicando sucesivamente la definición~\eqref{eq:submodular-def}
    a cada elemento de \(\mathscr{S} \smallsetminus \mathscr{T}\)
    vemos que:
    \begin{equation*}
      f(\mathscr{S} \cup \mathscr{T}) - f(\mathscr{T})
        \le f(\mathscr{S}) - f(\mathscr{S} \cap \mathscr{T})
    \end{equation*}
    Reorganizando obtenemos~\eqref{eq:submodular-intersection-union}.

    Por otro lado,
    si se cumple~\eqref{eq:submodular-intersection-union},
    consideremos conjuntos
      \(\mathscr{A} \subseteq \mathscr{B} \subseteq \mathscr{U}\)
    y un elemento \(e \in \mathscr{U} \smallsetminus \mathscr{B}\).
    Tomando \(\mathscr{S} = \mathscr{A} \cup \{e\}\)
    y \(\mathscr{T} = \mathscr{B}\)
    queda:
    \begin{align*}
      f(\mathscr{S} \cap \mathscr{T}) + f(\mathscr{S} \cup \mathscr{T})
        &= f(\mathscr{A}) + f(\mathscr{B} \cup \{e\}) \\
      f(\mathscr{S}) + f(\mathscr{T})
        &= f(\mathscr{A} \cup \{e\}) + f(\mathscr{B})
    \end{align*}
    Substituyendo esto en~\eqref{eq:submodular-intersection-union}
    y reorganizando:
    \begin{align*}
      f(\mathscr{A}) + f(\mathscr{B} \cup \{e\})
        &\le f(\mathscr{A} \cup \{e\}) + f(\mathscr{B}) \\
      f(\mathscr{A} \cup \{e\}) - f(\mathscr{A})
        &\ge f(\mathscr{B} \cup \{e\}) - f(\mathscr{B})
    \end{align*}
    que es exactamente~\eqref{eq:submodular-def}.
  \end{proof}
  Otra propiedad interesante es la siguiente:
  \begin{definition}
    \label{def:subaditive}
    Una función \(f \colon \mathscr{A} \to \mathscr{B}\)
    se dice \emph{subaditiva} si para todo \(x, y \in \mathscr{A}\):
    \begin{equation*}
      f(x + y)
        \le f(x) + f(y)
    \end{equation*}
  \end{definition}
  Resulta,
  si interpretamos la suma de conjuntos como unión:
  \begin{corollary}
    \label{cor:submodular-subadditive}
    Si \(f\) es una función submodular no negativa,
    \(f\) es subaditiva.
  \end{corollary}
  \begin{proof}
    Si \(f\) es submodular sobre \(\mathscr{U}\),
    no negativa,
    por el teorema~\ref{theo:submodular-intersection-union}
    para todo \(\mathscr{A}, \mathscr{B} \subseteq \mathscr{U}\):
    \begin{align*}
      f(\mathscr{A} \cup \mathscr{B}) + f(\mathscr{A} \cap \mathscr{B})
        &\le f(\mathscr{A}) + f(\mathscr{B}) \\
      f(\mathscr{A} \cup \mathscr{B})
        &\le f(\mathscr{A}) + f(\mathscr{B})
    \end{align*}
    Exactamente lo prometido.
  \end{proof}
  Caracterizaciones alternativas de funciones submodulares
  ofrece:
  \begin{theorem}
    \label{theo:submodular-equivalent}
    Son equivalentes las siguientes:
    \begin{enumerate}[label = \roman*, ref = (\roman*)]
    \item
      \label{item:f-submodular}
      La función \(f\) es submodular sobre \(\mathscr{U}\).
    \item
      \label{item:marginal-submodular}
      Para todo \(\mathscr{S} \subseteq \mathscr{U}\),
      la función \(\Delta(e \mid \mathscr{S})\) es submodular.
    \item
      \label{item:marginal-subadditive}
      Para todo \(\mathscr{S} \subseteq \mathscr{U}\),
      la función \(\Delta(\mathscr{X} \mid \mathscr{S})\) es subaditiva.
    \end{enumerate}
  \end{theorem}
  \begin{proof}
    Demostraremos un ciclo de implicancias.
    \begin{description}
    \item[\boldmath\(
            \ref{item:f-submodular}
              \Rightarrow \ref{item:marginal-submodular}
          \)\unboldmath:]
      Supongamos \(f\) submodular,
      sean conjuntos cualquiera \(\mathscr{S} \subseteq \mathscr{U}\),
      \(\mathscr{A} \subseteq \mathscr{B} \subseteq \mathscr{U}\).
      Por definición:
      \begin{align*}
        \Delta(\mathscr{A} \cup \{e\} \mid \mathscr{S})
          - \Delta(\mathscr{A} \cup \mathscr{S})
          &= f(\mathscr{A} \cup \{e\} \cup \mathscr{S})
                 - f(\mathscr{S})
               -(f(\mathscr{A} \cup \mathscr{S})
                 - f(\mathscr{S})) \\
          &= f(\mathscr{A} \cup \{e\} \cup \mathscr{S})
               - f(\mathscr{A} \cup \mathscr{S}) \\
        \Delta(\mathscr{B} \cup \{e\} \mid \mathscr{S})
          - \Delta(\mathscr{B} \cup \mathscr{S})
          &= f(\mathscr{B} \cup \{e\} \cup \mathscr{S})
               - f(\mathscr{B} \cup \mathscr{S})
      \end{align*}
      Si \(e \in \mathscr{S}\),
      ambas se anulan y la desigualad se cumple.
      Si \(e \notin \mathscr{S}\),
      la desigualdad pedida resulta por submodularidad de \(f\)
      y \(\mathscr{A} \cup \mathscr{S}
            \subseteq \mathscr{B} \cup \mathscr{S}\).
    \item[\boldmath\(
            \ref{item:marginal-submodular}
               \Rightarrow \ref{item:marginal-subadditive}
          \)\unboldmath:]
      Si \(\Delta(e \mid \mathscr{S})\) es submodular,
      por el corolario~\ref{cor:submodular-subadditive} es subaditiva.
    \item[\boldmath\(
            \ref{item:marginal-subadditive}
              \Rightarrow \ref{item:f-submodular}
          \)\unboldmath:]
      Sea \(f\) una función que cumple~\ref{item:marginal-subadditive},
      y sean \(\mathscr{A}, \mathscr{B} \subseteq \mathscr{U}\),
      definamos
        \(\mathscr{X} = \mathscr{A} \smallsetminus \mathscr{B}\),
        \(\mathscr{Y} = \mathscr{B} \smallsetminus \mathscr{A}\),
        \(\mathscr{S} = \mathscr{A} \cap \mathscr{B}\).
      Como \(\Delta(\mathscr{X} \mid \mathscr{S})\)
      es subaditiva:
      \begin{align*}
        \Delta(\mathscr{X} \cup \mathscr{Y} \mid \mathscr{S})
          &\le \Delta(\mathscr{X} \mid \mathscr{S})
                 + \Delta(\mathscr{X} \mid \mathscr{S}) \\
        f(\mathscr{X} \cup \mathscr{Y} \cup \mathscr{S})
          - f(\mathscr{S})
          &\le f(\mathscr{X} \cup \mathscr{S}) - f(\mathscr{S})
                 + f(\mathscr{Y} \cup \mathscr{S}) - f(\mathscr{S}) \\
        f(\mathscr{X} \cup \mathscr{Y} \cup \mathscr{S})
          + f(\mathscr{S})
          &\le f(\mathscr{X} \cup \mathscr{S})
                 + f(\mathscr{Y} \cup \mathscr{S}) \\
        f(\mathscr{A} \cup \mathscr{B}) + f(\mathscr{A} \cap \mathscr{B})
          &\le f(\mathscr{A}) + f(\mathscr{B})
      \end{align*}
      Por el teorema~(\ref{theo:submodular-intersection-union})
      vemos que \(f\) es submodular.
    \end{description}
    Cerramos el ciclo,
    las tres aseveraciones son equivalentes.
  \end{proof}

\section{Restricción de cardinalidad}
\label{sec:submocular-cardinality}

  Un caso típico de optimización es hallar un conjunto de cardinalidad \(k\)
  que maximice una función dada.
  En su máxima generalidad,
  esto claramente es \NP\nobreakdash-duro
  (por ejemplo,
   dado un grafo hallar un conjunto de vértices de tamaño \(k\)
   que maximice el número de arcos entre ellos
   es equivalente a \textsc{Clique}).
  Si la función es submodular,
  obtenemos una cota bastante buena para el algoritmo voraz obvio~%
    \ref{alg:submodular-cardinality}.

  \begin{algorithm}
    \DontPrintSemicolon\Indp

    \Function{\(\mathrm{GreedyMaximize}(f)\)}{
      \(\mathscr{S} \gets \varnothing\) \;
      \For{\(i \gets 1 \; \KwTo \; k\)}{
        \(\displaystyle
            \mathscr{S}
              \gets \mathscr{S}
                           \cup \{ \argmax_{x \notin \mathscr{S}}
                                     \{ f(\mathscr{S} \cup \{x\}) \}\}\) \;
      }
      \Return \(\mathscr{S}\) \;
    }
    \caption{Algoritmo voraz de maximización}
    \label{alg:submodular-cardinality}
  \end{algorithm}

  \begin{theorem}
    \label{theo:submodular-cardinality-approximation}
    Sea \(f\) una función submodular monótona no negativa.
    Entonces el algoritmo~\ref{alg:submodular-cardinality}
    da un factor de aproximación de \((1 - 1 / \mathrm{e})\).
  \end{theorem}
  \begin{proof}
    Sea \(\mathscr{S}_i\) el conjunto en el algoritmo en el paso \(i\)
    con \(\mathscr{S}_0 = \varnothing\),
    sea \(\mathscr{S}^* = \{ x_1^*, \dotsc, x_k^* \}\)
    un conjunto óptimo
    (el orden en que listamos los elementos es irrelevante).
    Lo siguiente no es más que una manera extraña de reescribir:
    \begin{equation}
      \label{eq:sma-1}
      f(\mathscr{S}^* \cup \mathscr{S}_i)
        = f(\mathscr{S}_i)
           + \sum_{1 \le j \le k}
               \left(
                 f(\mathscr{S}_i \cup \{ x_1^*, \dotsc, x_j^* \})
                   - f(\mathscr{S}_i \cup \{ x_1^*, \dotsc, x_{j - 1}^* \})
               \right)
    \end{equation}
    Como suponemos \(f\) monótona,
    \(f(\mathscr{S}^*) \le f(\mathscr{S}^* \cup \mathscr{S}_i)\).
    Como \(f\) es submodular,
    por definición tenemos:
    \begin{equation*}
      f(\mathscr{S}_i \cup \{ x_1^*, \dotsc, x_j^* \})
           - f(\mathscr{S}_i \cup \{ x_1^*, \dotsc, x_{j - 1}^* \})
        \le f(\mathscr{S}_i \cup \{  x_j^* \})
              - f(\mathscr{S}_i)
    \end{equation*}
    A su vez,
    elegimos el elemento a agregar a \(\mathscr{S}_i\)
    vorazmente,
    por lo que para cada \(j\) tenemos:
    \begin{equation*}
      f(\mathscr{S}_i \cup \{  x_j^* \})
         - f(\mathscr{S}_i)
        \le f(\mathscr{S}_{i + 1})
              - f(\mathscr{S}_i)
    \end{equation*}
    Esto nos da una cota:
    \begin{equation*}
      f(\mathscr{S}^*)
        \le f(\mathscr{S}_i)
              + k \left(
                    f(\mathscr{S}_{i + 1})
                      - f(\mathscr{S}_i)
                  \right)
    \end{equation*}
    Si definimos \(a_i = f(\mathscr{S}^*) - f(\mathscr{S}_i)\),
    la cota se traduce en:
    \begin{align*}
      a_i
        &\le k (a_i - a_{i + 1}) \\
      a_{i + 1}
        &\le \left(1 - \frac{1}{k}\right) a_i
    \end{align*}
    Por inducción,
    junto con la cota \(1 - x \le \mathrm{e}^{-x}\),
    esto nos da:
    \begin{align*}
      a_i
        &\le \left(1 - \frac{1}{k}\right)^i a_0 \\
        &\le \mathrm{e}^{- i/k} a_0 \\
      f(\mathscr{S}^*) - f(\mathscr{S}_i)
        &\le \mathrm{e}^{- i/k} f(\mathscr{S}^*) \\
      f(\mathscr{S}_i)
        &\ge \left( 1 - \mathrm{e}^{- i/k} \right) f(\mathscr{S}^*)
    \end{align*}
    Con \(i = k\) obtenemos lo prometido.
  \end{proof}
  Note que la demostración nos da una estimación de lo que podríamos ganar
  al relajar la condición de número de elementos a elegir.

  La demostración parece dejar mucho espacio para mejora,
  pero Feige~%
    \cite{feige98:_threshold_approximating_set_cover}
  demostró que si \(\cP \ne \NP\)
  para el problema \textsc{Set~Cover}
  (en este caso,
   hallar el máximo conjunto que puede cubrirse con \(k\)~conjuntos)
  solo puede lograrse una aproximación \(1 - 1/\mathrm{e}\) eficientemente.
  Ver también la exposición de Filmus~%
    \cite{filmus11:_hardness_approximating_set_cover}
  que resume este resultado.

  Minoux~%
    \cite{minoux78:_accel_greed_algor_maxim_submod_set_funct}
  describe una manera simple de acelerar el algoritmo voraz
  para funciones submodulares.
  Observa que si llamamos \(\mathscr{S}_i\) al conjunto en la iteración \(i\),
  claramente \(\mathscr{S}_i \subset \mathscr{S}_j\) si \(i < j\),
  con lo que para cualquier elemento \(e \notin \mathscr{S}_j\)
  por submodularidad
  es \(\Delta(e \mid \mathscr{S}_i) \ge \Delta(e \mid \mathscr{S}_j)\).
  Vale decir,
  solo se requiere calcular \(\Delta(e \mid \mathscr{S})\)
  para aquellos elementos para los cuales antes hemos obtenido valores grandes.
  El detalle es el algoritmo~\ref{alg:Minoux}.
  Supone una cola de prioridad \(Q\).
  \begin{algorithm}[htbp]
     \DontPrintSemicolon\Indp

    \Function{\(\mathrm{GreedyMaxAD}(f, \mathscr{U})\)}{
      \ForEach{\(e \in \mathscr{U}\)}{
        \(\mathrm{Insert}(Q, e, \Delta(e \mid \varnothing))\) \;
      }
      \((e, \delta)
           \gets \mathrm{DeleteMax}(Q)\) \;
      \(\mathscr{S} \gets \{e\}\) \;
      \For{\(j \gets 2 \; \KwTo \; k\)}{
        \((e_{\text{max}}, \delta_{\text{max}})
             \gets \mathrm{DeleteMax}(Q)\) \;
        \Loop{
          \((e, \delta)
             \gets \mathrm{DeleteMax}(Q)\) \;
          \If{\(\delta < \delta_{\text{max}}\)}{
            \Break \;
          }
          \(\delta \gets \Delta(e \mid \mathscr{S})\) \;
          \eIf{\(\delta > \delta_{\text{max}}\)}{
            \(\mathrm{Insert}(Q, e_{\text{max}},
                                    \delta_{\text{max}})\) \;
            \((e_{\text{max}}, \delta_{\text{max}})
                \gets (e, \delta)\) \;
          }
          {
            \(\mathrm{Insert}(Q, e, \delta)\) \;
          }
        }
        \(\mathscr{S} \gets \mathscr{S} \cup \{e_{\text{max}}\}\) \;
      }
      \Return \(\mathscr{S}\) \;
    }
    \caption{El algoritmo voraz adaptativo de Minoux}
    \label{alg:Minoux}
  \end{algorithm}
  Esto no cambia el peor caso,
  pero en la práctica mejora los tiempos de ejecución
  en varios órdenes de magnitud.
  Minoux reporta que en casos de prueba los tiempos de ejecución
  disminuyen de horas a minutos.

\section{Restricción de presupuesto}
\label{sec:submodular-knapsack}

  Otra variante de interés es una restricción de mochila
  (\emph{\foreignlanguage{english}{knapsack restriction}} en inglés):
  cada ítem tiene un costo,
  y hay un presupuesto finito a gastar en elementos a agregar.
  Cabría esperar que la extensión obvia
  de agregar cada vez el elemento que maximiza el aumento del valor de \(f\)
  por unidad de costo
  (dentro del presupuesto)
  da buenos resultados,
  pero este algoritmo puede dar resultados arbitrariamente malos,
  como vimos en la sección~\ref{sec:mochila-aproximado}.
  Sin embargo,
  Leskovec et al~%
    \cite{leskovec07:_cost_effective_outbreak_detection_networks}
  demuestran que si se ejecuta el algoritmo así extendido
  y el algoritmo que ignora los costos,
  al menos uno de los dos da una aproximación de al menos 30\%.

  Extendemos la función de costos a conjuntos de forma obvia:
  \begin{equation*}
    c(\mathscr{S})
      = \sum_{e \in \mathscr{S}} c(e)
  \end{equation*}
  \begin{algorithm}[htbp]
    \DontPrintSemicolon\Indp

    \Function{\(\mathrm{GreedyKnapsack}(\mathscr{U}, B, f, c)\)}{
      \(\mathscr{S} \gets \varnothing\) \;
      \While{\(\mathscr{U} \ne \varnothing\)}{
        \(e^*
           \gets \argmax_{e \in \mathscr{U}}
                        \frac{\Delta(e \mid \mathscr{S})}{c(e)}\) \;
        \If{\(c(\mathscr{S}) + c(e) \le	 B\)}{
          \(\mathscr{S} \gets \mathscr{S} \cup \{e\}\) \;
        }
        \(\mathscr{U} \gets \mathscr{U} \smallsetminus \{e\}\) \;
      }
      \Return \(\mathscr{S}\) \;
    }
    \caption{El algoritmo voraz para restricción de mochila}
    \label{alg:GreedyKnapsack}
  \end{algorithm}
  Tenemos el siguiente resultado:
  \begin{lemma}
    \label{lem:greedy-knapsack}
    Cuando en el algoritmo~\ref{alg:GreedyKnapsack}
    la condición \(c(\mathscr{S}) + c(e) \le  B\) es falsa
    es \(f(\mathscr{S} \cup \{e\}) \ge f(\mathscr{S}^*)\).
  \end{lemma}
  \begin{proof}
    Sea \(\mathscr{S}_i = \{e_1, e_2, \dotsc, e_i\}\)
    el conjunto \(\mathscr{S}\) cuando contiene \(i\) elementos,
    listados en el orden que se agregaron.
    Por submodularidad y el hecho que el algoritmo es voraz:
    \begin{align*}
     f(\mathscr{S}^*)
       &\le f(\mathscr{S}_i)
               + \sum_{e \in \mathscr{S}^* \smallsetminus \mathscr{S}_{i - 1}}
                   \Delta(e \mid \mathscr{S}_{i - 1}) \\
       &\le f(\mathscr{S}_i)
               + \sum_{e \in \mathscr{S}^* \smallsetminus \mathscr{S}_{i - 1}}
                   c(e) \frac{\Delta(e \mid \mathscr{S}_{i - 1})}{c(e)} \\
       &\le f(\mathscr{S}_i)
               + \frac{f(\mathscr{S}_i) - f(\mathscr{S}_{i - i})}{c(e_i)}
                   \cdot \sum_{e \in \mathscr{S}^*
                               \smallsetminus \mathscr{S}_{i - 1}} c(e) \\
       &\le f(\mathscr{S}_i)
               + \frac{B}{c(e_i)} (f(\mathscr{S}_i) - f(\mathscr{S}_{i - i})
    \end{align*}
    Al final usamos que \(c(S^*) \le B\).
    Restando \(B/c(e_i)\) y reorganizando se llega a la recurrencia:
    \begin{equation*}
      f(\mathscr{S}_i) - f(\mathscr{S}^*)
        \ge \left(1 - \frac{c(e_i)}{B}\right)
             (f(\mathscr{S}_{i - 1}) - f(\mathscr{S}^*))
    \end{equation*}
    cuya solución es:
    \begin{equation*}
      f(\mathscr{S}_i)
        \ge \left(
               1 - \prod_{1 \le k \le i}
                      \left(
                         1 - \frac{c(e_i)}{B}
                      \right)
             \right) f(\mathscr{S}^*)
    \end{equation*}
    Usando la desigualdad \(1 - x \le \mathrm{e}^{-x}\):
    \begin{equation*}
      f(\mathscr{S}_i)
        \ge \left(
               1 - \exp\left(
                         - \frac{c(S_i)}{B}
                       \right)
             \right) f(\mathscr{S}^*)
    \end{equation*}
    Podemos aplicar esto a \(e\) elegido por el algoritmo voraz
    antes de considerar la restricción de presupuesto:
    \begin{equation*}
      f(\mathscr{S} \cup \{e\})
        \ge \left(
               1 - \exp\left(
                         - \frac{c(S) - c(e)}{B}
                       \right)
             \right) f(\mathscr{S}^*)
    \end{equation*}
    Suponemos que el elemento elegido sobrepasa el presupuesto,
    con lo que \(c(\mathscr{S}) + c(e) > B\),
    y resulta lo prometido.
  \end{proof}
  Podemos aprovechar esto
  para construir un algoritmo con razón de aproximación garantizada:
  \begin{algorithm}[htbp]
    \DontPrintSemicolon\Indp

    \Function{\(\mathrm{GreedyKnapsackA}(\mathscr{U}, B, f, c)\)}{
      \(e^*
          \gets \argmax_{e \in \mathscr{U},
                              c(e) \le B} f(e)\) \;
      \(\mathscr{S}
          \gets \mathrm{GreedyKnapsack}(\mathscr{U}, B, f, c)\) \;
      \Return \(\argmax\{f(\mathscr{S}), f(\{e^*\})\}\) \;
    }
    \caption{El algoritmo voraz para restricción de mochila,
             corrección simple}
    \label{alg:GreedyKnapsack-A}
  \end{algorithm}
  \begin{theorem}
    \label{theo:greedy-knapsack-A}
    Sea \(\mathscr{S}\) el conjunto
    retornado por el algoritmo~\ref{alg:GreedyKnapsack-A}.
    Entonces:
    \begin{equation*}
      f(\mathscr{S})
        \ge \frac{1}{2} \left(1 - \frac{1}{\mathrm{e}}\right) f(\mathscr{S}^*)
    \end{equation*}
  \end{theorem}
  \begin{proof}
    Sea \(\mathscr{S}_G\) el conjunto
    que retorna el algoritmo~\ref{alg:GreedyKnapsack}
    y \(e\) el elemento siguiente
    (el primero rechazado después de completar \(\mathscr{S}_G\)):
    \begin{align*}
      2 f(\mathscr{S})
        &\ge f(\mathscr{S}_G) + f(\{e^*\}) \\
        &\ge f(\mathscr{S}_G) + f(\{e\}) \\
        &\ge f(\mathscr{S}_G \cup \{e\}) \\
        &\ge \left(
               1 - \frac{1}{\mathrm{e}}
             \right) f(\mathscr{S}^*)
    \end{align*}
    La primera desigualdad es de la definición de \(\mathscr{S}\),
    la segunda viene de la definición de \(e^*\),
    la tercera es por ser \(f\) subaditiva,
    la última usa el lema~\ref{lem:greedy-knapsack}.
  \end{proof}
  Khuller, Moss y~Naor~%
    \cite{khuller99:_budget_max_cover_problem}
  refinan el análisis del teorema~\ref{theo:greedy-knapsack-A}
  y mejoran el factor a \(1 - 1 / \sqrt{\mathrm{e}}\).

  Pero podemos hacer mejor.
  En vez de iniciar el algoritmo voraz con el conjunto vacío,
  iniciamos con todos los conjuntos de tamaño \(d\)
  (fijaremos \(d\) más adelante)
  y elegimos el mejor de los resultados.
  Llamaremos
    \(\mathrm{GreedyKnapsack}'(\mathscr{U}, \mathscr{S}, B, f, c)\)
  a la variante del algoritmo~\ref{alg:GreedyKnapsack}
  que inicia con \(\mathscr{S}\).
  El algoritmo~\ref{alg:GreedyKnapsack-B} da detalles.
  \begin{algorithm}[htbp]
    \DontPrintSemicolon\Indp

    \Function{\(\mathrm{GreedyKnapsackB}(\mathscr{U}, B, f, c)\)}{
      \(\mathscr{S}_1
          \gets \argmax_{\mathscr{S} \subseteq \mathscr{U},
                              c(\mathscr{S}) \le B,
                              \lvert \mathscr{S} \rvert < d}
                         f(\mathscr{S})\) \;
      \(\mathscr{S_2} \gets \varnothing\) \;
      \ForAll{\(\mathscr{S} \subseteq \mathscr{U},
                \lvert \mathscr{S} \rvert = d,
                c(\mathscr{S}) \le B\)}{
         \(\mathscr{S}_G
             \gets \mathrm{GreedyKnapsack}'
                (\mathscr{U} \smallsetminus \mathscr{S}_G, \mathscr{S}_G,
                 B, f, c)\) \;
         \If{\(f(\mathscr{S}_G) > f(\mathscr{S}_2)\)}{
           \(\mathscr{S}_2 \gets \mathscr{S}_G\) \;
         }
      }
      \Return \(\argmax\{f(\mathscr{S}_1), f(\mathscr{S}_2)\}\) \;
    }
    \caption{El algoritmo voraz para restricción de mochila,
             enumeración parcial}
    \label{alg:GreedyKnapsack-B}
  \end{algorithm}
  Resulta:
  \begin{theorem}
    \label{theo:greedy-knapsack-B}
    Para \(d = 3\) el conjunto retornado
    por el algoritmo~\ref{alg:GreedyKnapsack-B} cumple:
    \begin{equation*}
      f(\mathscr{S})
        \ge \left(1 - \frac{1}{\mathrm{e}}\right) f(\mathscr{S}^*)
    \end{equation*}
  \end{theorem}
  \begin{proof}
    Sin pérdida de generalidad,
    \(\lvert \mathscr{S}^* \rvert > d\),
    ya que en caso contrario obtenemos el óptimo.
    Escribimos \(\mathscr{S}^* = \{e_1^*, e_2^*, \dotsc, e_n^*\}\)
    y \(\mathscr{S}_i^* = \{e_1^*, e_2^*, \dotsc, e_i^*\}\),
    donde:
    \begin{equation*}
      e_i^*
        = \argmax_{e \in \mathscr{S}^* \smallsetminus \mathscr{S}_{i - 1}^*}
             \Delta(e \mid \mathscr{S}_{i - 1}^* )
    \end{equation*}
    Consideremos la iteración que considera \(\mathscr{S}_d^*\).
    La llamada a \(\mathrm{GreedyKnapsack}'\)
    es equivalente a llamarla con la función \(\Delta(e \mid \mathscr{S}_d^*)\)
    sobre el conjunto \(\mathscr{U} \smallsetminus \mathscr{S}_d^*\).
    Consideremos la primera vez
    que la condición en el algoritmo~\ref{alg:GreedyKnapsack} es falsa
    para \(x^* \in \mathscr{S}^* \smallsetminus \mathscr{S}^*_d\).
    Por el lema~\ref{lem:greedy-knapsack}:
    \begin{equation}
      \label{eq:gk-B-1}
      f(\mathscr{S}_G \cup \{x^*\}) - f(\mathscr{S}_d^*
        \ge \left(1 - \frac{1}{\mathrm{e}}\right)
              (f(\mathscr{S}^*) - f(\mathscr{S}_d^*))
    \end{equation}
    Por submodularidad y el ordenamiento de \(\mathscr{S}_d^*\):
    \begin{equation*}
      \Delta(x^* \mid \mathscr{S}_G)
        \le \Delta(x^* \mid \mathscr{S}_i^*)
        \le F(\mathscr{S}_i^*) - f(\mathscr{S}_{i - 1}^*)
    \end{equation*}
    Sumando para \(1 \le i \le d\):
    \begin{equation}
      \label{eq:gk-B-2}
      f(\mathscr{S}_G \cup \{x^*\})
        \le f(\mathscr{S}_G) + \frac{1}{d} f(\mathscr{S}_d^*)
    \end{equation}
    Combinando~\eqref{eq:gk-B-1} con~\eqref{eq:gk-B-2} da:
    \begin{equation*}
      f(\mathscr{S}_G)
        \ge \left(1 - \frac{1}{\mathrm{e}}\right) f(\mathscr{S}^*)
              + \left(\frac{1}{\mathrm{e}} - \frac{1}{d}\right)
                  f(\mathscr{S}_d^*)
    \end{equation*}
    Con \(d \ge 3\) el segundo término es positivo,
    completando la demostración.
  \end{proof}

\section{Otras restricciones}
\label{sec:submodular-other}

  Hay otras condiciones naturales a considerar,
  como es hallar el menor conjunto \(\mathscr{S}\)
  tal que \(f(\mathscr{S}) \ge q\).
  Wolsey~%
    \cite{wolsey82:_analy_greed_algor_submod_set_cover_probl}
  muestra que una variante simple del algoritmo voraz
  logra un conjunto un factor \(1 + \ln \max_x \{f(\{x\})\}\)~mayor
  que el óptimo.

  Krause y Golovin~%
    \cite{krause14:_submodular_survey}
  discuten distintos ámbitos en que aparecen funciones submodulares
  y algoritmos de maximización aproximados.

\section*{Ejercicios}
\label{sec:ejercicios-submodular}

  \begin{enumerate}
  \item
    Demuestre que en general para \(\mathscr{U} = [1, n]\) la función
    \(\max \colon 2^{\mathscr{U}} \to \mathbb{R}\) es submodular monótona.
  \item
    Demuestre que para \(\mathscr{U}\) la función
    \(\mathrm{card} \colon \mathscr{U} \to \mathbb{R}\)
    tal que \(\mathrm{card}(\mathscr{A}) = \lvert \mathscr{A} \rvert\)
    (cardinalidad)
    es submodular monótona.
  \item
    Sea \(w \colon \mathscr{S} \to \mathbb{R}\)
    una función de peso no negativa.
    Defina:
    \begin{equation*}
      f(\mathscr{S})
        = \sum_{e \in \mathscr{S}} w(e)
    \end{equation*}
    Demuestre que \(f\) es submodular monótona.
    Muchos ejemplos de optimización
    (como el caso del conjunto óptimo de radios planteado por Kun)
    son variaciones sobre esto.
  \item
    Demuestre que las siguientes funciones son submodulares:
    \begin{enumerate}
    \item
      \(f(\mathscr{S}) = \sum_{x \in \mathscr{S}} g(x)\),
      si \(g \colon \mathscr{U} \to \mathbb{R}\) es una función cualquiera.
    \item
      \(f(\mathscr{S})
           = \lvert \mathscr{S} \rvert (\lvert \mathscr{S} \rvert - 1) / 2
               + \sum_{x \in \mathscr{S}} g(x)\),
      si \(g \colon \mathscr{U} \to \mathbb{R}\) es una función cualquiera.
    \end{enumerate}
  \item
    Considere un grafo conexo \(G = (V, E)\),
    sea \(w \colon E \to \mathbf{R}\) con \(w(e) > 0\) para todo \(e \in E\)
    el costo de los arcos.
    Defina para \(\mathscr{S} \subseteq E\):
    \begin{equation*}
      w(\mathscr{S})
        = \sum_{e \in \mathscr{S}} w(e)
    \end{equation*}
    Defina además:
    \begin{equation*}
      f(\mathscr{S})
        = \begin{cases}
            - w(E \smallsetminus \mathscr{S}) &
                \text{si \(G_{\mathscr{S}}
                             = (V, E \smallsetminus \mathscr{S})\)
                      es conexo} \\
            - \infty			      &
                \text{caso contrario}
          \end{cases}
    \end{equation*}
    Demuestre que \(f\) es submodular.
  \item
    Demuestre que si \(f_1, f_2, \dotsc, f_n\)
    son funciones submodulares sobre \(\mathscr{U}\),
    y \(\alpha_1, \alpha_2, \dotsc, \alpha_n\) son constantes positivas,
    es submodular
    \(f(\mathscr{X})
        = \alpha_1 f_1(\mathscr{X}) + \alpha_2 f_2(\mathscr{X})
            + \dotsb
            + \alpha_n f_n(\mathscr{X})\).
  \item
    Se define el \emph{truncamiento} (en \(c\)) de la función \(f\)
    mediante:
    \begin{equation*}
      f_T(x)
        = \min \{ f(x), c \}
    \end{equation*}
    Demuestre que si \(f\) es submodular monótona y \(c\) es una constante,
    su truncamiento es submodular.
  \item
    Exhiba una función submodular que no es monótona.
  \item
    Demuestre que la función \(f(x) = x^\alpha\)
    para \(x > 0\) y \(0 \le \alpha \le 1\) es subaditiva.
  \item
    ¿Cuántas veces se invoca la función \(f\)
    en los algoritmos~\ref{alg:GreedyKnapsack-A}
    y~\ref{alg:GreedyKnapsack-B}?
  \end{enumerate}

\bibliography{../referencias}

%%% Local Variables:
%%% mode: latex
%%% TeX-master: "../INF-221_notas"
%%% ispell-local-dictionary: "spanish"
%%% End:

% LocalWords:  submodulares submodular eq def intersection union Set
% LocalWords:  subaditiva submodularidad Clique maximización Cover et
% LocalWords:  english knapsack restriction gk


\part*{Apéndices}
\thispagestyle{empty}

\appendix

%\bibliographystyle{babplain-fl}

\chapter{Algunos resultados del análisis}
\label{apx:analysis}

  En derivaciones en análisis numérico
  usamos varios resultados conocidos
  (y no tan conocidos)
  del análisis.
  Resumimos los más importantes,
  con sus demostraciones.

  Primeramente,
  algunas definiciones previas.
  \begin{definition}
    El conjunto de funciones continuas sobre el intervalo cerrado \([a, b]\)
    se anota \(C[a, b]\).
    Usamos \(C^n[a, b]\) para denotar el conjunto de funciones reales
    con \(n\)\nobreakdash-ésima derivada continua
    sobre el intervalo \([a, b]\).
    Notaciones correspondientes se usan para intervalos abiertos
    en alguno de sus extremos,
    como \(C(a, b)\) para funciones continuas
    en el intervalo abierto \((a. b)\).
  \end{definition}

  Un \textquote{resultado obvio}
  (cuya demostración en realidad es bastante compleja,
   y omitiremos)
  es el siguiente:
  \begin{theorem}[Valores extremos]
    \label{theo:extreme value}
    Sea \(f\) una función continua sobre el intervalo \([a, b]\).
    Sean el mínimo y el máximo de \(f\) sobre el intervalo,
    \(m = \min_{x \in [a, b]} \{ f(x) \}\)
    y \(M = \max_{x \in [a, b]} \{ f(x) \}\).
    Entonces para todo \(v \in [m, M]\)
    hay \(\zeta \in [a, b]\) tal que \(f(\zeta) = v\).
  \end{theorem}
  O sea,
  \(f\) toma todos los valores entre \(m\) y \(M\).

  \begin{theorem}[Rolle]
    \label{theo:Rolle}
    Sea \(f \colon [a, b] \to \mathbb{R}\) continua,
    diferenciable en \((a, b)\),
    tal que \(f(a) = f(b)\).
    Entonces hay \(\zeta \in (a, b)\) tal que \(f'(\zeta) = 0\).
  \end{theorem}
  En particular,
  considerando el caso \(f(a) = f(b) = 0\)
  dice que entre cada par de ceros de \(f\) hay al menos un cero de \(f'\).
  \begin{proof}
    Como \(f\) es continua en \([a, b]\),
    alcanza sus valores máximo y mínimo en el intervalo.
    Si máximo y mínimo coinciden en ambos extremos,
    la función es constante,
    su derivada se anula en \((a, b)\)
    y podemos elegir \(\zeta \in (a, b)\) arbitrariamente.

    Supongamos entonces que el máximo no ocurre en los extremos
    (la situación con el mínimo es similar),
    llamemos \(\zeta\) al punto máximo.
    Para \(\zeta + h \in (a, b)\) consideremos la expresión:
    \begin{equation*}
      \frac{f(\zeta + h) - f(\zeta)}{h}
    \end{equation*}
    Para \(h\) positivo esta expresión es negativa
    (numerador negativo,
     denominador positivo)
    para \(h\) negativo es positiva
    (numerador negativo,
     denominador negativo).
    Concluimos que:
    \begin{align*}
      \lim_{h \to 0} \frac{f(\zeta + h) - f(\zeta)}{h}
        &= f'(\zeta) \\
        &= 0
    \end{align*}
    El límite existe ya que supusimos que \(f\) es diferenciable.
    Esto es lo que queríamos demostrar.
  \end{proof}
  Un resultado que usamos es el siguiente:
  \begin{theorem}[Rolle extendido]
    \label{theo:Rolle-extended}
    Sea \(f \colon [a, b] \to \mathbb{R}\) continua,
    diferenciable \(n\) veces en \((a, b)\),
    tal que \(f(a) = f(b) = 0\)
    y con \(n + 1\)~ceros distintos en \([a, b]\).
    Entonces hay \(\zeta \in (a, b)\) tal que \(f^{(n)}(\zeta) = 0\).
  \end{theorem}
  \begin{proof}
    La demostración es por inducción.
    \begin{description}
    \item[Base:]
      Para \(n = 0\) tenemos el teorema de Rolle,
      teorema~\ref{theo:Rolle}.
    \item[Inducción:]
      Supongamos que vale para \(n\),
      y consideremos la función \(f'(x)\).
      Por hipótesis \(f'\) es continua en \((a, b)\).
      Por el teorema de Rolle aplicado
      a los intervalos
        \([x_0, x_1], [x_1, x_2], \dotsc, [x_n, x_{n + 1}]\),
      cada uno de estos contiene un cero de \(f'(x)\),
      llamémosles \(\zeta_0, \zeta_1, \dotsc, \zeta_n\).
      Por inducción,
      el intervalo \([\zeta_0, \zeta_n] \subset [a, b]\)
      contiene un cero \(\zeta\)
      de la \(n\)\nobreakdash-ésima derivada de \(f'(x)\),
      que es \(f^{(n + 1)}(x)\).
    \end{description}
    Por inducción vale para todo \(n \in \mathbb{N}_0\).
  \end{proof}
  \begin{theorem}[Valor medio]
    \label{theo:mean-value}
    Sea \(f \colon [a, b] \to \mathbb{R}\) continua en \([a, b]\)
    y derivable en \((a, b)\).
    Entonces hay \(\zeta \in (a, b)\) tal que:
    \begin{equation*}
      f'(\zeta)
        = \frac{f(b) - f(a)}{b - a}
    \end{equation*}
  \end{theorem}
  \begin{proof}
    Consideremos la función:
    \begin{equation*}
      g(x)
        = f(x) - \frac{f(b) - f(a)}{b - a} x
    \end{equation*}
    Entonces \(g\) cumple las hipótesis del teorema de Rolle,
    \ref{theo:Rolle}.
    Vale decir,
    hay un punto \(\zeta \in (a, b)\) tal que:
    \begin{align*}
      g'(\zeta)
        &= 0 \\
        &= f'(\zeta) - \frac{f(b) - f(a)}{b - a}
    \end{align*}
    El resultado sigue.
  \end{proof}
  El siguiente se conoce como teorema del valor medio extendido,
  o teorema de valor medio de Cauchy.
  \begin{theorem}[Valor medio extendido]
    \label{theo:mean-value-extended}
    Sean \(f, g\) funciones continuas en el intervalo \([a, b]\),
    derivables en \((a, b)\).
    Entonces hay \(\zeta \in (a, b)\) tal que:
    \begin{equation*}
      (f(b) - f(a)) g'(\zeta)
        = (g(b) - g(a)) f'(\zeta)
    \end{equation*}
  \end{theorem}
  Obviamente,
  si \(g(a) \ne g(b)\) y \(g'(\zeta) \ne 0\) esto equivale a:
  \begin{equation*}
    \frac{f'(\zeta)}{g'(\zeta)}
      = \frac{f(b) - f(a)}{g(b) - g(a)}
  \end{equation*}
  \begin{proof}
    Si \(g(a) = g(b)\),
    por el teorema de Rolle
    (teorema~\ref{theo:Rolle})
    hay \(\zeta \in (a, b)\) tal que \(g'(\zeta) = 0\),
    para este el teorema se cumple trivialmente.

    Suponga ahora \(g(a) \ne g(b)\).
    Defina:
    \begin{equation*}
      h(x)
        = f(x) - \frac{f(b) - f(a)}{g(b) - g(a)} g(x)
    \end{equation*}
    Como \(f, g\) son continuas en \([a, b]\) y derivables en \((a, b)\),
    esto se cumple para \(h\),
    al satisfacer las hipótesis del teorema de Rolle
    hay \(\zeta \in (a, b)\) tal que:
    \begin{align*}
      h'(\zeta)
        &= 0 \\
        &= f'(\zeta) - \frac{f(b) - f(a)}{g(b) - g(a)} g'(\zeta)
    \end{align*}
    El resultado es inmediato.
  \end{proof}
  El teorema del valor medio,
  teorema~\ref{theo:mean-value},
  es el caso particular \(g(x) = x\).

  Finalmente:
  \begin{theorem}
    \label{theo:mean-value-integral}
    Sea \(f \colon [a, b] \to \mathrm{R}\) continua,
    y sea \(g \colon [a, b] \to \mathbb{R}\)
    integrable que no cambia de signo en \([a, b]\).
    Entonces hay \(\zeta \in [a, b]\) tal que:
    \begin{equation*}
      \int_a^b f(x) g(x) \mathrm{d} x
        = f(\zeta) \int_a^b g(x) \mathrm{d} x
    \end{equation*}
  \end{theorem}
  El caso especial \(g(x) = 1\) da el resultado conocido:
  \begin{equation*}
    \int_a^b f(x) \mathrm{d} x
      = f(\zeta) (b - a)
  \end{equation*}
  \begin{proof}
    Supongamos \(g\) es no negativa,
    el caso de \(g\) no positiva es similar.
    Sabemos que hay valores mínimo y máximo de \(f\) sobre \([a, b]\),
    llamémosles \(m, M\),
    tales que para \(x \in [a, b]\):
    \begin{equation*}
      m \le f(x) \le M
    \end{equation*}
    Como \(g\) no es negativa:
    \begin{equation*}
      m \int_a^b g(x) \mathrm{d} x
        \le \int_a^b f(x) g(x) \mathrm{d} x
        \le M \int_a^b g(x) \mathrm{d} x
    \end{equation*}
    Si ahora:
    \begin{equation*}
      \int_a^b g(x) \mathrm{d} x
        = 0
    \end{equation*}
    podemos elegir \(\zeta \in [a, b]\) arbitrario.
    Si la última integral no se anula,
    por el teorema de los valores extremos,
    teorema~\ref{theo:extreme value},
    hay \(\zeta \in [a, b]\) con el que se cumple lo prometido.
  \end{proof}
  Un resultado importantísimo es:
  \begin{theorem}[Taylor]
    \label{theo:taylor}
    Sea \(f \in C^{n + 1}(a, x)\) con \(f^{(n + 1)} \in C[a, x]\).
    Entonces hay \(\xi \in [a, x]\) tal que:
    \begin{equation*}
      f(x)
        = \sum_{0 \le k \le n} \frac{f^{(k)}(a)}{k!} (x - a)^k
            + \frac{f^{(n + 1)}(\xi)}{(n + 1)!} (x - a)^{n + 1}
    \end{equation*}
  \end{theorem}
  A esta se le conoce como \emph{la forma de Lagrange del residuo},
  es la más útil para nuestros fines.
  \begin{proof}
    Sean:
    \begin{align*}
      F(t)
        &= \sum_{0 \le k \le n} \frac{f^{(k)}(t)}{k!} (x - t)^k \\
      G(t)
        &= (t - x)^{n + 1}
    \end{align*}
    Por el teorema~\ref{theo:mean-value-extended},
    tenemos que hay \(\xi \in (a, x)\) tal que:
    \begin{equation*}
      \frac{F'(\xi)}{G'(\xi)}
        = \frac{F(x) - F(a)}{G(x) - G(a)}
    \end{equation*}
    Vemos que:
    \begin{align*}
      F'(t)
        &= \frac{f^{(n + 1)}(t)}{n!} (x - t)^n \\
      G'(t)
        &= (n + 1) (t - x)^n
    \end{align*}
    Substituyendo y simplificando queda:
    \begin{equation*}
      f(x)
        = \sum_{0 \le k \le n} \frac{f^{(k)}(a)}{k!} (x - a)^k
            + \frac{f^{(n + 1)}(\xi)}{(n + 1)!} (x - a)^{k + 1}
    \end{equation*}
    Esto es lo que había que demostrar.
  \end{proof}

%\bibliography{../referencias}

%%% Local Variables:
%%% mode: latex
%%% TeX-master: "../INF-221_notas"
%%% ispell-local-dictionary: "spanish"
%%% End:

% LocalWords:  ésima diferenciable

\bibliographystyle{babplain-fl}

\chapter{Píldoras de funciones generatrices}
\label{apx:funciones-generatrices}

  La idea de funciones generatrices es usar una serie
  para representar una secuencia.
  Resulta que en muchos casos manipular la serie es mucho más fácil
  que trabajar con la secuencia,
  y se pueden obtener resultados sorprendentes en forma muy sencilla.
  Véase el apéndice~\ref{apx:symbolic-method-dummies} para ejemplos.
  Mucho más detalle se da en los apuntes de Fundamentos de Informática~%
    \cite{brand17:_fundamentos_informatica},
  en los textos de Wilf~%
    \cite{wilf06:_gfology}
  y de Flajolet y Sedgewick~%
    \cite{flajolet09:_analy_combin},
  y muy particularmente aplicado a nuestro tema en el de
  Sedgewick y Flajolet~%
    \cite{sedgewick13:_introd_anal_algor}.

  Wilf~\cite{wilf06:_gfology} expresa
  que la función generatriz es una línea de ropa
  de la cual se cuelgan los coeficientes para exhibición.
  Entendemos el exponente de \(z\) como un contador,
  índice del coeficiente correspondiente.
  Como veremos,
  operaciones sobre la función generatriz
  corresponden a actuar sobre la secuencia,
  en muchos casos resulta más sencillo manipular la serie
  que trabajar con la secuencia.
  Para nuestros efectos,
  en general basta manipularlos como si fueran \textquote{polinomios infinitos}.
  Si tenemos la suerte que la serie converge para algún rango
  alrededor de \(z = 0\)
  (como en nuestros ejemplos),
  podremos aplicar las herramientas del cálculo.

\section{Ejemplos combinatorios}
\label{sec:gf-ejemplos-combinatorios}

  Acá nos interesa analizar las sumas resultantes
  al lanzar combinaciones de los dados no tradicionales.
  Un dado puede representarse por una tupla
  que indica el número de caras con cada valor,
  agregando el valor cero para completar.
  Los dados tradicionales quedan representados por:
  \begin{align*}
    \mathrm{D}
      \text{\ (caras \(\{1, 2, 3, 4, 5, 6\}\))}
      &\colon \langle 0, 1, 1, 1, 1, 1, 1 \rangle
  \end{align*}
  Los dados no transitivos que consideraremos quedan descritos por:
  \begin{align*}
    \mathrm{A}
      \text{\ (caras \(\{1, 1, 3, 5, 5, 6\}\))}
      &\colon \langle 0, 2, 0, 1, 0, 2, 1 \rangle \\
    \mathrm{B}
      \text{\ (caras \(\{2, 3, 3, 4, 4, 5\}\))}
      &\colon \langle 0, 0, 1, 2, 2, 1, 0 \rangle \\
    \mathrm{C}
      \text{\ (caras \(\{1, 2, 2, 4, 6, 6\}\))}
      &\colon \langle 0, 1, 2, 0, 1, 0, 2 \rangle
  \end{align*}
  Para lanzar
  la suma \num{7} con dos dados
  tenemos las opciones
  desde que el primero aporte \num{1} y el segundo \num{6}
  hasta aportes respectivos de \num{6} y \num{1}.
  Debemos considerar además el número de caras del valor considerado
  en cada dado.
  Para los dados \(\mathrm{A}\) y \(\mathrm{C}\)
  esto se traduce en:
  \begin{equation*}
    2\cdot 2 + 0 \cdot 0 + 1 \cdot 1 + 0 \cdot 0 + 2 \cdot 2 + 1 \cdot 1
      = 10
  \end{equation*}
  posibilidades.
  Esta es exactamente la manera en que se calcula
  el coeficiente de \(z^7\) al multiplicar polinomios
  con el coeficiente de \(z^n\)
  dando el número de caras con valor \(n\) para cada dado:
  \begin{align*}
    A(z)
      &= 2 z		   +   z^3	   + 2 z^5 +   z^6 \\
    B(z)
      &=	 z^2	   + 2 z^3 + 2 z^4 +   z^5 \\
    C(z)
      &=   z + 2 z^2	   + z^4		   + 2 z^6
  \end{align*}
  El coeficiente de \(z^7\)
  en el producto \(A(z) \cdot C(z)\) nos da el valor buscado:
  \begin{equation*}
    A(z) \cdot C(z)
      = 2 z^2 + 4 z^3 +	  z^4 + 4 z^5	 + 2 z^6    + 10 z^7
              + 2 z^8 + 4 z^9 +	  z^{10} + 4 z^{11} + 2 z^{12}
  \end{equation*}
  Una sencilla operación
  considera todas las combinaciones posibles.
  Note que los polinomios del caso nos interesan
  puramente por sus propiedades algebraicas,
  los valores de los polinomios para diversos valores de \(z\)
  no interesan en lo más mínimo.

  Al lanzar dos dados tradicionales
  las sumas \num{2} y \num{12} se pueden obtener de una única manera,
  mientras para \num{4} hay tres (\(1 + 3 = 2 + 2 = 3 + 1\)).
  Representamos un dado mediante el polinomio:
  \begin{equation}
    \label{eq:gf-dado}
    D(z)
      = z + z^2 + z^3 + z^4 + z^5 + z^6
  \end{equation}
  con lo cual para lanzamientos de dos dados:
  \begin{equation}
    \label{eq:dos-dados}
    D^2(z)
      = z^2 + 2 z^3 + 3 z^4 + 4 z^5 + 5 z^6 + 6 z^7
          + 5 z^8 + 4 z^9 + 3 z^{10} + 2 z^{11} + z^{12}
  \end{equation}
  Interesa hallar dados marcados en forma diferente
  que den la misma distribución de las sumas
  (\textquote{dados locos}).%
    \index{dados!locos|see{Sicherman, dados de}}
  Para construirlos
  buscamos polinomios \(D_1(z)\) y \(D_2(z)\)
  que den el producto~\eqref{eq:dos-dados}.
  Queremos además que ambas representen dados,
  o sea tengan \num{6} caras,
  y que cada cara debe estar marcada por al menos un punto.
  El número de caras
  es simplemente el valor de la función en \(z = 1\);
  que cada cara esté marcada con al menos un punto
  se traduce en que la función generatriz sea divisible por \(z\)
  (el número de caras marcadas con cero,
   o sea el coeficiente de \(z^0\),
   debe ser cero),
  O sea:
  \begin{equation}
    \label{eq:dados-locos-caras}
    D_1(1)
      = D_2(1)
      = 6
  \end{equation}
  Factorizamos~\eqref{eq:gf-dado}:
  \begin{equation}
    \label{eq:gf-dado-factorizada}
    D(z)
      = z (z + 1) (z^2 - z + 1) (z^2 + z + 1)
  \end{equation}
  Los factores \(z\) y \(z^2 - z + 1\) tienen valor \num{1} para \(z = 1\),
  \(z + 1\) da \num{2} y \(z^2 + z + 1\) da \num{3}.
  Tanto \(D_1(z)\) como \(D_2(z)\)
  deben tener los factores \(z\),
  \(z + 1\) y \(z^2 + z + 1\);
  solo quedan por redistribuir los dos factores \(z^2 - z + 1\) en \(D^2(z)\):
  \begin{align}
    D_1(z)
      &= z (z + 1) (z^2 + z + 1) \notag \\
      &= z + 2 z^2 + 2 z^3 + z^4 \label{eq:Sicherman-1} \\
    D_2(z)
      &= z (z + 1) (z^2 - z + 1)^2 (z^2 + z + 1) \notag \\
      &= z + z^3 + z^4 + z^5 + z^6 + z^8 \label{eq:Sicherman-2}
  \end{align}
  Los dados marcados con \(\{ 1, 2, 2, 3, 3, 4 \}\)
  y \(\{ 1, 3, 4, 5, 6, 8 \}\)
  se conocen como \emph{dados de Sicherman}~%
    \cite{gardner78_2:_mathem_games}.

\section{Definiciones formales}
\label{sec:gf-definiciones-formales}

  Sea una secuencia
  \(\left\langle a_n \right\rangle_{n \ge 0}
     = \left\langle
         a_0, a_1, a_2, \dotsc, a_n, \dotsc
       \right\rangle\).
  La \emph{función generatriz} (ordinaria) de la secuencia es
  la serie de potencias:%
    \index{generatriz!ordinaria|textbfhy}
  \begin{equation*}
    A(z)=\sum_{0 \le n} a_n z^n
  \end{equation*}
  Anotaremos
  \(A(z)
     \ogf \left\langle a_n\right\rangle_{n \ge 0}\) en este caso
  (\emph{ogf} es por
     \emph{\foreignlanguage{english}
                           {Ordinary Generating Function}}).

  La \emph{función generatriz exponencial}%
    \index{generatriz!exponencial|textbfhy}
  de la secuencia es la serie:
  \begin{equation*}
    \widehat{A}(z)
      = \sum_{0 \le n} a_n \, \frac{z^n}{n!}
  \end{equation*}
  Anotaremos
  \(\widehat{A}(z)
     \egf \left\langle a_n\right\rangle_{n \ge 0}\) en este caso
  (\emph{egf} es por
     \emph{\foreignlanguage{english}
                           {Exponential Generating Function}}).

  Por comodidad,
  a veces escribiremos estas relaciones
  con la función generatriz al lado derecho.

\subsection{Reglas OGF}
\label{sec:reglas-OGF}
\index{generatriz!ordinaria!reglas}

  Las propiedades siguientes de funciones generatrices ordinarias
  son directamente las definiciones del caso
  o son muy simples de demostrar,
  sus justificaciones detalladas quedarán de ejercicios.

  \begin{description}
  \item[Linealidad:]
    Si \(A(z) \ogf \left\langle a_n \right\rangle_{n \ge 0}\)
    y \(B(z) \ogf \left\langle b_n \right\rangle_{n \ge 0}\),
    y \(\alpha\) y \(\beta\) son constantes,
    entonces:
    \begin{equation*}
      \alpha A(z) + \beta B(z)
         \ogf \left\langle
                \alpha a_n + \beta b_n
              \right\rangle_{n \ge 0}
    \end{equation*}
  \item[Secuencia desplazada a la izquierda:]
    Si
    \(A(z) \ogf \left\langle a_n \right\rangle_{n \ge 0}\),
    entonces:
    \begin{equation*}
      \frac{A(z) - a_0 - a_1 z - \dotsb - a_{k - 1} z^{k - 1}}{z^k}
        \ogf \left\langle a_{n + k}\right\rangle_{n \ge 0}
    \end{equation*}
  \item[Multiplicar por \(n\):]
    Consideremos:
    \begin{align*}
      A(z)
        &\ogf \left\langle a_n\right\rangle_{n \ge 0} \\
      z \, \frac{\mathrm{d}}{\mathrm{d} z} A(z)
        &\ogf \left\langle n a_n\right\rangle_{n \ge 0}
    \end{align*}
    Esta operación
    se expresa en términos del operador \(z \mathrm{D}\)
    (acá \(\mathrm{D}\) es por derivada,
     para abreviar).
    Además:
    \begin{equation*}
      (z \mathrm{D})^2 A(z)
        = z D (z D A(z))
        \ogf \left\langle n^2 a_n\right\rangle_{n \ge 0}
    \end{equation*}
    Note que
      \((z \mathrm{D})^2 = z \mathrm{D} + z^2 \mathrm{D}^2\)
    es diferente de \(z^2 \mathrm{D}^2\).
  \item[Multiplicar por un polinomio en \(n\):]
    Si \(p(n)\) es un polinomio,
    usando el operador \(z \mathrm{D}\) varias veces
    para potencias de \(n\)
    y por linealidad:
    \begin{align*}
      p(z \mathrm{D}) A(z)
        &\ogf \left\langle p(n) a_n \right\rangle_{n \ge 0}
    \end{align*}
  \item[Convolución:]
    Si \(A(z) \ogf \left\langle a_n \right\rangle_{n \ge 0}\)
    y \(B(z) \ogf \left\langle b_n \right\rangle_{n \ge 0}\)
    entonces:
    \begin{equation*}
      A(z) \cdot B(z)
        \ogf \left\langle
               \sum_{0 \le k \le n} a_k b_{n - k}
              \right\rangle_{n \ge 0}
    \end{equation*}
  \item
    Sea \(k\) un entero positivo
    y \(A(z) \ogf \left\langle a_n\right\rangle_{n \ge 0}\),
    entonces:
    \begin{equation*}
      (A(z))^k
        \ogf \left\langle \sum_{n_1 + n_2 + \dotsb + n_k = n}
               \left( a_{n_1} \cdot a_{n_2} \dotsm a_{n_k} \right)
             \right\rangle_{n \ge 0}
    \end{equation*}
    Vale la pena tener presente el caso especial:
    \begin{equation*}
      (A(z))^2
        \ogf \left\langle
               \sum_{0 \le i \le n} a_i a_{n - i}
             \right\rangle_{n \ge 0}
    \end{equation*}
  \item[Sumas parciales:]
    Supongamos:
    \begin{equation*}
      A(z) \ogf \left\langle a_n \right\rangle_{n \ge 0}
    \end{equation*}
    Podemos escribir:
    \begin{equation*}
      \sum_{0 \le k \le n} a_k
        = \sum_{0 \le k \le n} 1 \cdot a_k
    \end{equation*}
    Esto no es más que la convolución de las secuencias
    \(\left\langle 1 \right\rangle_{n \ge 0}\)
    y \(\left\langle a_n \right\rangle_{n \ge 0}\),
    y la función generatriz de la primera es nuestra vieja conocida,
    la serie geométrica,
    con lo que:
    \begin{equation}
      \label{eq:sumas-parciales}
      \frac{A(z)}{1 - z}
        \ogf \left\langle
               \sum_{0 \le k \le n} a_k
             \right\rangle_{n \ge 0}
    \end{equation}
  \end{description}

\subsection{Reglas EGF}
\label{sec:reglas-EGF}
\index{generatriz!exponencial!reglas}

  Las siguientes resumen propiedades
  de las funciones generatrices exponenciales.
  Son simples de demostrar,
  y las justificaciones que no se dan acá quedarán de ejercicios.
  \begin{description}
  \item[Linealidad:]
    Si \(\widehat{A}(z)
           \egf \left\langle a_n \right\rangle_{n \ge 0}\)
    y \(\widehat{B}(z)
          \egf \left\langle b_n \right\rangle_{n \ge 0}\),
    y \(\alpha\) y \(\beta\) son constantes,
    entonces:
    \begin{equation*}
      \alpha \widehat{A}(z) + \beta \widehat{B}(z)
        \egf \left\langle
               \alpha a_n + \beta b_n
             \right\rangle_{n \ge 0}
    \end{equation*}
  \item[Secuencia desplazada a la izquierda:]
    Si \(\widehat{A}(z)
           \egf \left\langle a_n \right\rangle_{n \ge 0}\),
    entonces,
    usando nuevamente \(\mathrm{D}\) para el operador derivada:
    \begin{equation*}
      \mathrm{D}^k \widehat{A}(z)
        \egf \left\langle a_{n + k} \right\rangle_{n \ge 0}
    \end{equation*}
  \item[Multiplicación por un polinomio en \(n\):]
    Si es \(\widehat{A}(z)
              \egf \left\langle a_n \right\rangle_{n \ge 0}\),
    y \(p\) es un polinomio,
    entonces:
    \begin{equation*}
      p(z \mathrm{D}) \widehat{A}(z)
        \egf \left\langle p(n) a_n \right\rangle_{n \ge 0}
    \end{equation*}
    Es la misma que en funciones generatrices ordinarias,
    ya que la operación \(z \mathrm{D}\)
    no altera el exponente en \(z^n\).
  \item[Convolución binomial:]
    Si \(\widehat{A}(z)
           \egf \left\langle a_n \right\rangle_{n \ge 0}\) y
    \(\widehat{B}(z)
        \egf \left\langle b_n \right\rangle_{n \ge 0}\)
    entonces:
    \begin{align*}
      \widehat{A}(z) \cdot \widehat{B}(z)
        &= \sum_{n \ge 0}\biggl( \,
                           \sum_{0 \le k \le n}
                           \frac{a_k}{k!} \, \frac{b_{n - k}}
                                                  {(n - k)!}
                         \biggr) z^n \\
        &= \sum_{n \ge 0} \biggl( \,
                            \sum_{0 \le k \le n}
                               \binom{n}{k} \, a_k b_{n - k}
                          \biggr)
               \frac{z^n}{n!}
    \end{align*}
    Vale decir:
    \begin{equation*}
      \widehat{A}(z) \cdot \widehat{B}(z)
        \egf \left\langle
               \sum_{0 \le k \le n} \binom{n}{k} \, a_k b_{n - k}
             \right\rangle_{n \ge 0}
    \end{equation*}
  \item[Términos individuales:]
    Es fácil ver que si
    \(\widehat{A}(z)
        \egf \left\langle a_n \right\rangle_{n \ge 0}\) entonces:
    \begin{equation*}
      a_n = \widehat{A}^{(n)}(0)
    \end{equation*}
    Esto en realidad no es más que el teorema de Maclaurin.%
      \index{Maclaurin, teorema de}
  \end{description}

\section{\protect\boldmath
           \texorpdfstring{El truco \(z \mathrm{D}\log\)}
                          {Derivada logarítmica}%
       \protect\unboldmath}
\index{derivada logaritmica@derivada logarítmica|textbfhy}

  Los logaritmos ayudan a simplificar expresiones con exponenciales
  y potencias.
  Pero terminamos con el logaritmo de una suma
  si el argumento es una serie,
  que es algo bastante feo de contemplar.
  Eliminar el logaritmo se logra derivando:
  \begin{equation*}
    \frac{\mathrm{d} \ln(A)}{\mathrm{d} z} = \frac{A'}{A}
  \end{equation*}
  Esto es mucho más decente.
  Multiplicamos por \(z\)
  para reponer la potencia \textquote{perdida} al derivar.

\section{Algunas series útiles}
\label{ref:series-utiles}
\index{serie de potencias!series utiles@series útiles}

  Las expansiones en serie que más aparecen son las siguientes.

\subsection{Serie geométrica}
\label{sec:serie-geometrica}
\index{serie de potencias!geometrica@geométrica}

  Es la serie más común en aplicaciones.
  Si \(\lvert z \rvert < 1\),
  se cumple:
  \begin{equation}
    \label{eq:serie-geometrica-b}
    \sum_{n \ge 0} z^n
      = \frac{1}{1 - z}
  \end{equation}
  Una variante importante es la siguiente,
  expansión válida para \(\lvert a z \rvert < 1\)
  (con la convención \(0^0 = 1\)):
  \begin{equation}
    \label{eq:serie-geometrica-c}
    \sum_{n \ge 0} a^n z^n
      = \frac{1}{1 - a z}
  \end{equation}

\subsection{Teorema del binomio}
\label{sec:teorema-binomio}
\index{serie de potencias!binomio}

  Una de las series más importantes
  es la expansión de la potencia de un binomio:
  \begin{equation}
    \label{eq:serie-binomio}
    \sum_{n \ge 0} \binom{\alpha}{n} \, z^n
       = (1 + z)^\alpha
  \end{equation}
  Siempre que \(\lvert z \rvert < 1\)
  esto vale incluso para \(\alpha\) complejos,
  si definimos:
  \begin{equation}
    \label{eq:coeficiente-binomial}
    \binom{\alpha}{k}
       = \frac{\alpha}{1} \cdot \frac{\alpha - 1}{2}
            \cdot \frac{\alpha - 2}{3}
            \cdot \dots
            \cdot \frac{\alpha - k + 1}{k}
       = \frac{\alpha^{\underline{k}}}{k!}
  \end{equation}
  y (consistente con la convención que productos vacíos son \num{1})
  siempre es:
  \begin{equation}
    \label{eq:binomial(alpha,0)}
    \binom{\alpha}{0}
      = 1
  \end{equation}
  A los coeficientes~\eqref{eq:coeficiente-binomial}
  se les llama \emph{coeficientes binomiales}
  por su conexión con la potencia de un binomio.
  La expansión~\eqref{eq:serie-binomio}
  (también conocida como \emph{fórmula de Newton}
   si \(\alpha\) no es un natural)
  es fácil de demostrar por el teorema de Maclaurin.
  Resulta que~\eqref{eq:serie-geometrica-b} es
  un caso particular de~\eqref{eq:serie-binomio}.

  Si \(\alpha\) es un entero positivo,
  la serie~\eqref{eq:serie-binomio} se reduce a un polinomio
  (el factor \(\alpha^{\underline{k}}\) se anula si \(k > \alpha\))
  y la relación es válida para todo \(z\).
  Además,
  en caso que \(n\) sea natural podemos escribir:
  \begin{equation}
    \label{eq:coeficiente-binomial-factorial}
    \binom{n}{k}
       = \frac{n!}{k! (n - k)!}
  \end{equation}
  Es claro que:
  \begin{equation}
    \label{eq:coeficiente-binomial-contorno}
    \binom{n}{k}
      = 0 \text{\ si \(k < 0\) o \(k > n\)}
  \end{equation}
  Esto con~\eqref{eq:coeficiente-binomial-factorial}
  sugiere la convención:
  \begin{equation}
    \label{eq:1/k!-convention}
    \frac{1}{k!}
      = 0 \quad \text{si \(k < 0\)}
  \end{equation}
  Note la simetría:
  \begin{equation}
    \label{eq:coeficiente-binomial-simetria}
    \binom{n}{k}
      = \binom{n}{n - k}
  \end{equation}

  Casos especiales notables de coeficientes binomiales
  para \(\alpha \notin \mathbb{N}\) son los siguientes:
  \begin{description}
  \item[\boldmath Caso \(\alpha = 1 / 2\):\unboldmath]
    Tenemos,
    como siempre:
    \begin{equation}
      \label{eq:binomial(1/2,0)}
      \binom{1/2}{0}
        = 1
    \end{equation}
    Cuando \(k \ge 1\):
    \begin{align}
      \binom{1/2}{k}
         &= \frac{\frac{1}{2} \cdot (\frac{1}{2}-1)
               \dotsm (\frac{1}{2} - k + 1)}{k!} \notag \\
         &= \frac{1}{2^k}
               \cdot \frac{1 \cdot (1 - 2) \cdot (1 - 4)
                             \dotsm (1 - 2 k + 2)}{k!} \notag \\
         &= \frac{(-1)^{k - 1}}{2^k k!}
               \cdot (1 \cdot 3 \dotsm (2 k - 3)) \notag \\
         &= \frac{(-1)^{k - 1}}{2^k k!}
               \cdot \frac{1 \cdot 2 \cdot 3 \cdot 4
                              \cdot \dotsm
                              \cdot (2 k - 3) \cdot (2 k - 2)}
                          {2 \cdot 4 \cdot 6 \dotsm (2 k - 2)}
                                \notag \\
         &= \frac{(-1)^{k - 1}}{2^k k!}
               \cdot \frac{(2 k - 2)!}{2^{k - 1} (k - 1)!}
                  \notag \\
         &= \frac{(-1)^{k - 1}}{2^{2 k - 1} \cdot k}
               \cdot \frac{(2 k - 2)!}{(k - 1)! \, (k - 1)!}
                  \notag \\
         &= \frac{(-1)^{k - 1}}{2^{2 k - 1} \cdot k}
               \cdot \binom{2 k - 2}{k - 1}
            \label{eq:binomial(1/2,k)}
    \end{align}
    Hay que tener cuidado,
    la última fórmula no cubre el caso \(k = 0\).

  \item[\boldmath Caso \(\alpha = -1/2\):\unboldmath]
    Mucha de la derivación es similar a la del caso anterior.
    Tenemos,
    para \(k > 0\):
    \begin{align}
      \binom{-1/2}{k}
        &= \frac{(-1/2) \cdot (-1/2 - 1) \cdot \dotsm
                   \cdot (-1/2 - k + 1)}
                {k!} \notag \\
        &= (-1)^k \frac{1}{2^k}
             \cdot \frac{1 \cdot 3 \dotsm (2 k - 1)}{k!} \notag \\
        &= (-1)^k \frac{1}{2^k}
             \cdot \frac{(2 k)!}{k! \, 2^k \, k!} \notag \\
        &= (-1)^k \frac{1}{2^{2 k}} \, \binom{2 k}{k}
            \label{eq:binomial(-1/2,k)}
    \end{align}
    Esta fórmula con \(k = 0\) da:
    \begin{equation*}
      \binom{-1/2}{0} = 1
    \end{equation*}
    así no se necesita hacer un caso especial acá.
  \item[\boldmath Caso \(\alpha = -n\):\unboldmath]
    Cuando \(\alpha\) es un entero negativo,
    podemos escribir:
    \begin{equation}
      \label{eq:binomial(-n,k)}
      \binom{-n}{k}
        = \frac{(-n)^{\underline{k}}}{k!}
        = (-1)^k \, \frac{n^{\overline{k}}}{k!}
        = (-1)^k \, \frac{(n + k - 1)^{\underline{k}}}{k!}
        = (-1)^k \, \binom{k + n - 1}{n - 1}
    \end{equation}
    Una fórmula cómoda es:
    \begin{equation}
      \label{eq:serie-binomio-negativo}
      \frac{1}{(1 - z)^{n + 1}}
        = \sum_{k \ge 0} \binom{n + k}{n} \, z^k
    \end{equation}
  \end{description}
  Interesantes resultan las series al variar \(n\), no \(k\).
  Sabemos que \(\binom{n}{k} = 0\) si no es que \(0 \le k \le n\),
  podremos ahorrarnos los límites de las sumas para simplificar.
  El índice \(n\) o \(n + k\) para \(k\) fijo
  da lo mismo al sumar sobre todo \(n \in \mathbb{Z}\):
  \begin{align}
    \sum_n \binom{n}{k} \, z^n
      &= \sum_n \binom{n + k}{k} \, z^{n + k} \notag \\
      &= z^k \sum_n \binom{n + k}{n} \, z^n \notag \\
      &= \frac{z^k}{(1 - z)^{k + 1}}
            \label{eq:serie-binomio-n}
  \end{align}
  Al final usamos~\eqref{eq:serie-binomio-negativo}.
  Omitir los rangos de los índices ahorró interminables ajustes.
\subsection{Otras series}
\label{sec:otras-series}

  Una serie común es la exponencial:%
    \index{serie de potencias!exponencial}
  \begin{equation}
    \label{eq:exponencial}
    \mathrm{e}^z
      = \sum_{n \ge 0} \frac{z^n}{n!}
  \end{equation}
  A veces aparecen funciones trigonométricas:%
    \index{serie de potencias!seno}%
    \index{serie de potencias!coseno}
  \begin{equation*}
    \sin z
      = \sum_{n \ge 0} (-1)^n \frac{z^{2 n + 1}}{(2 n + 1)!} \qquad
    \cos z
      = \sum_{n \ge 0} (-1)^n \frac{z^{2 n}}{(2 n)!}
  \end{equation*}
  o hiperbólicas:%
    \index{serie de potencias!seno hiperbolico@seno hiperbólico}%
    \index{serie de potencias!coseno hiperbolico@coseno hiperbólico}
  \begin{equation*}
    \sinh z
      = \sum_{n \ge 0} \frac{z^{2 n + 1}}{(2 n + 1)!} \qquad
    \cosh z
      = \sum_{n \ge 0} \frac{z^{2 n}}{(2 n)!}
  \end{equation*}
  Una relación útil es la fórmula de Euler:%
    \index{Euler, formula de (exponencial complejo)@Euler, fórmula de (exponencial complejo)}
  \begin{equation}
    \label{eq:formula-Euler-exponencial}
    \mathrm{e}^{u + \mathrm{i} v}
      = \mathrm{e}^u (\cos v + \mathrm{i} \sin v)
  \end{equation}

  Es frecuente la serie para el logaritmo,%
    \index{serie de potencias!logaritmo}
  que podemos hallar integrando término a término:
  \begin{align}
    \frac{\mathrm{d}}{\mathrm{d} z} \, \ln (1 - z)
      &= - \frac{1}{1 - z}
       = - \sum_{n \ge 0} z^n \notag \\
    \ln (1 - z)
      &= - \sum_{n \ge 1} \frac{z^n}{n}
           \label{eq:ln(1-z)}
  \end{align}
  Muchos ejemplos adicionales de series útiles
  se hallan en el texto de Wilf~\cite{wilf06:_gfology}.

\section{Notación para coeficientes}
\label{sec:funciones-generatrices:notacion}
\index{serie de potencias!extraer coeficiente}

  Comúnmente extraeremos el coeficiente de un término de una serie.
  Generalmente no hay términos con potencias negativas de \(z\),
  tales coeficientes serán cero.
  Para esto,
  dadas las series:
  \begin{equation*}
    A(z)
      = \sum_{n \ge 0} a_n z^n
    \hspace{3em}
    B(z)
      = \sum_{n \ge 0} b_n z^n
  \end{equation*}
  usaremos la notación:
  \begin{equation*}
     \left[ z^n \right] A(z) = a_n
  \end{equation*}
  Tenemos algunas propiedades simples:
  \begin{align*}
    \left[ z^n \right] z^k A(z)
      &= \begin{cases}
           0				  & n < k \\
           \left[ z^{n - k} \right] A(z)  & n \ge k
         \end{cases} \\
    \left[ z^n \right] (\alpha A(z) + \beta B(z))
      &= \alpha \left[ z^n \right] A(z)
           + \beta \left[ z^n \right] B(z)
  \end{align*}

\section{Aceite de serpiente}
\label{sec:snake-oil}
\index{generatriz!aceite de serpiente}

  La manera tradicional de simplificar sumatorias
  (particularmente las que involucran coeficientes binomiales)
  es aplicar identidades u otras manipulaciones de los índices,
  como magistralmente exponen Knuth~%
    \cite{knuth97:_fundam_algor}
  y Graham, Knuth y~Patashnik~%
    \cite{graham94:_concr_mathem}.
  Acá mostramos un método alternativo,
  que no requiere saber y aplicar
  una enorme variedad de identidades.
  Wilf~%
    \cite{wilf06:_gfology}
  le llama \emph{\foreignlanguage{english}{Snake Oil Method}},
  por la cura milagrosa que se ve en las películas del viejo oeste.
  La técnica es bastante simple:
  \begin{enumerate}
  \item
    Identificar la variable libre,
    llamémosle \(n\),
    de la que depende la suma.
    Sea \(f(n)\) nuestra suma.
  \item
    Sea \(F(z)\) la función generatriz ordinaria
    de la secuencia \(\langle f(n) \rangle_{n \ge 0}\).
  \item
    Multiplique la suma por \(z^n\) y sume sobre \(n\).
    Tenemos \(F(z)\) expresado como una doble suma,
    sobre \(n\) y la variable de la suma original.
  \item
    Intercambie el orden de las sumas,
    y exprese la suma interna en forma simple y cerrada.
  \item
    Encuentre los coeficientes,
    son los valores de \(f(n)\) buscados.
  \end{enumerate}
  Sorprende la alta tasa de éxitos de la técnica.
  Tiene la ventaja de que no requiere mayor creatividad;
  resulta claro cuándo funciona
  y es obvio cuando falla.

  Usaremos la convención que toda suma sin restricciones
  es sobre el rango \(-\infty\) a \(\infty\).
  Como los coeficientes binomiales \(\binom{n}{k}\)
  que usaremos en los ejemplos se anulan cuando
  \(k\) no está en el rango \([0, n]\),
  esto evita interminables ajustes de índices.

  Nuestro ejemplo viene de Riordan~%
    \cite{riordan68:_combin_ident},
  donde se resuelve mediante delicadas maniobras.
  Nuestro desarrollo sigue a Dobrushkin~%
    \cite{dobrushkin10:_method_algor_analysis}.

  Evaluar:
  \begin{equation*}
    h_n
      = \sum_{0 \le k \le n}
          (-1)^{n - k} \, 4^k \, \binom{n + k + 1}{2 k + 1}
  \end{equation*}

  Definimos \(H(z)\) como la función generatriz de los \(h_n\);
  multiplicamos por \(z^n\),
  sumamos para \(n \ge 0\)
  e intercambiamos orden de suma:
  \begin{align*}
    H(z)
      &= \sum_{n \ge 0} z^n
           \sum_{0 \le k \le n}
             (-1)^{n - k} \, 4^k \, \binom{n + k + 1}{2 k + 1} \\
      &= \sum_{n \ge 0}
           \sum_{0 \le k \le n}
             (-4)^k \, (-z)^n \, \binom{n + k + 1}{2 k + 1} \\
      &= \sum_{k \ge 0}
           (-4)^k \,
           \sum_{n \ge k}
             \binom{n + k + 1}{2 k + 1} \, (-z)^n
  \end{align*}
  Para completar el trabajo necesitamos la suma interna.
  Haciendo el cambio de variable \(r = n - k\):
  \begin{equation*}
    \sum_{n \ge k} \binom{n + k + 1}{2 k + 1} \, (-z)^n
      = (-z)^k \, \sum_{r \ge 0}
                    \binom{r + 2 k + 1}{2 k + 1} \, (-z)^r
      = \frac{(-z)^k}{(1 + z)^{2 k + 2}}
  \end{equation*}
  Substituyendo en lo anterior:
  \begin{equation*}
    H(z)
      = \sum_{k \ge 0} \frac{(4 z)^k}{(1 + z)^{2 k + 2}}
      = \frac{1}{(1 + z)^2}
          \cdot \frac{1}{1 - \frac{4 z}{(1 + z)^2}}
      = \frac{1}{(1 - z)^2}
  \end{equation*}
  Resta extraer los coeficientes,
  lo que da:
  \begin{equation*}
    h_n
      = (-1)^n \, \binom{-2}{n}
      = \binom{n + 1}{1}
      = n + 1
  \end{equation*}

\section{La fórmula de inversión de Lagrange}
\label{sec:Lagrange-inversion}

  Es común encontrarse con ecuaciones de la forma
  \begin{equation*}
    u = t \phi(u)
  \end{equation*}
  donde \(\phi\) es una función dada de \(u\).
  Esta relación define \(u\) en función de \(t\),
  y \textquote{estamos despejando \(u\) en términos de \(t\)}.
  Fue demostrada por Lagrange%
    \index{Lagrange, inversion de@Lagrange, inversión de|textbfhy}%
    \index{Lagrange-Burmann, inversion de@Lagrange-Bürmann, inversión de|see{Lagrange, inversión de}}
  y casi simultáneamente por Bürmann,%
    \index{Burmann, Hans Heinrich@Bürmann, Hans Heinrich}
  la forma dada acá es la de Bürmann.
  \begin{theorem}[Fórmula de inversión de Lagrange]
    \label{theo:LIF}
    Sean \(f(u)\) y \(\phi(u)\) series de potencias en \(u\),
    con \(\phi(0) = 1\).
    Entonces hay una única serie \(u = u(t)\) que cumple:
    \begin{equation*}
      u = t \phi(u)
    \end{equation*}
    Además,
    el valor \(f(u(t))\) de \(f\) en el cero \(u = u(t)\)
    cumple:
    \begin{equation*}
      \left[ t^n \right] \left\{ f(u(t)) \right\}
         = \frac{1}{n} \, \left[ u^{n - 1} \right] \,
                            \left\{ f'(u) \phi(u)^n \right\}
    \end{equation*}
  \end{theorem}
  Dadas \(f\) y \(\phi\),
  esta fórmula da los coeficientes de \(f(u(t))\) en bandeja.
  No demostraremos este resultado,
  ya que nos llevaría demasiado fuera del rango de este ramo.
  La demostración del teorema puede encontrarse en el texto de Wilf~%
    \cite{wilf06:_gfology}.

\bibliography{../referencias}

%%% Local Variables:
%%% mode: latex
%%% TeX-master: "../INF-221_notas"
%%% ispell-local-dictionary: "spanish"
%%% End:

% LocalWords:  see eq Factorizamos gf textbfhy ogf english Ordinary
% LocalWords:  Generating Function egf Exponential geometrica Snake
% LocalWords:  Oil Method inversion Burmann Bürmann Hans Heinrich

\bibliographystyle{babplain-fl}

\chapter{Recurrencias}
\label{apx:recurrencias}

  Una \emph{recurrencia} es una descripción recursiva de una función,
  la describe en términos de ella misma.
  Como toda definición recursiva,
  hay algunos \emph{casos base}
  y \emph{casos recursivos}.
  Cada caso es una igualdad o desigualdad,
  donde los casos base dan valores explícitos
  y los casos recursivos
  relacionan el valor de la función con valores anteriores.

  Decimos que una función \emph{satisface} la recurrencia,
  o es \emph{solución} de la recurrencia,
  si para ella cada uno de los casos se cumple.
  En la mayoría de los casos que encontramos en la práctica
  hay una solución
  -- podemos escribir una función recursiva de la recurrencia,
  y la función que esta compute es la solución.
  Aún más,
  en casos de interés,
  si cada uno de los casos es una igualdad
  la solución es única.

  Por sí misma la recurrencia
  no es una descripción satisfactoria de la función.
  Buscamos una \emph{forma cerrada},
  una descripción no recursiva de la función.
  Pero una forma cerrada exacta puede no existir,
  o ser demasiado complicada para uso práctico.
  Muchas veces nos conformamos con una solución asintótica
  del tipo \(f(n) = \Theta(g(n))\),
  aunque Sedgewick y Flajolet~%
    \cite{sedgewick13:_introd_anal_algor}
  arguyen que debiéramos ir más allá
  y dar asintóticas de la forma \(f(n) \sim g(n)\).

  Para desigualdades recursivas preferimos una solución ajustada,
  que se cumple incluso si las desigualdades
  se reemplazan por las igualdades correspondientes.
  Nuevamente,
  soluciones ajustadas exactas pueden no existir o ser demasiado complicadas,
  y nos conformamos con \(f(n) = \Omega(g(n))\) o \(f(n) = O(g(n))\),
  según sea el caso.
  Vea también el apéndice~\ref{apx:asymptotics}.

\section{El método definitivo: adivinar y verificar}
\label{sec:adivinar-verificar}

  Siempre podemos aplicar la idea de adivinar la solución
  y verificar que cumple la recurrencia.
  Esto es esencialmente una demostración por inducción:
  verificamos que cumple los casos base
  y los casos recursivos
  (inducción).
  Claro que esto traslada el problema a adivinar la solución.
  Algunas de las secciones siguientes
  presentan guías que ayudan a adivinar correctamente.

\subsection{Calcular valores}
\label{sec:calcular-valores}

  En el caso de las torres de Hanoi,
  tenemos la recurrencia:
  \begin{equation*}
    H(n)
      = 2 H(n - 1) + 1
      \quad H(0) = 0
  \end{equation*}
  Viendo la recurrencia,
  sospechamos que la solución es algo como \(H(n) = 2^n\),
  probando algunos valores obtenemos \(0, 1, 3, 7, 15, \dotsc\),
  parece que \(H(n) = 2^n - 1\).
  Substituyendo,
  vemos que cumple la recurrencia.

  Acá es relevante la forma en que se dejan los elementos.
  Si se simplifican demasiado,
  puede ocultar su forma.
  No simplificar lo suficiente también puede ser contraproducente.

\subsection{Desenrollar la recurrencia}
\label{sec:desenrollar-recurrencia}

  Otra alternativa es desenrollar la recurrencia,
  substituyéndola en sí misma:
  \begin{align*}
    H(n)
      &= 2 H(n - 1) + 1 \\
      &= 2 (2 H(n - 2) + 1) + 1
       = 4 H(n - 2) + 3 \\
      &= 4 (2 H(n - 3) + 1) + 3
       = 8 H(n - 3) + 7 \\
      &= \dotsb
  \end{align*}
  Da la impresión que al desenrollar
  resulta \(H(n) = 2^k H(n - k) + 2^k - 1\).
  Esto podemos demostrarlo por inducción,
  y finalmente deducir \(H(n) = 2^n H(0) + 2^n - 1 = 2^n - 1\).
  Incidentalmente,
  nos da la dependencia de \(H(n)\) del valor inicial.

  Los números de Fibonacci dan otro ejemplo:
  \begin{equation*}
    F_{n + 2}
      = F_{n + 1} + F_n
      \quad F_0 = 0, F_1 = 1
  \end{equation*}
  Los primeros valores
  \(0, 1, 1, 2, 3, 5, \dotsc\) no muestran un patrón claro,
  pero la forma hace sospechar algo de la forma \(c \alpha^n\).
  Intentemos demostrar por inducción que \(F_n \le c \alpha^n\)
  y ver dónde llegamos:
  \begin{align*}
    F_{n + 2}
      &=   F_{n + 1} + F_n \\
      &\le c \alpha^{n + 1} + c \alpha^n
  \end{align*}
  Por otro lado,
  queremos que:
  \begin{align*}
    F_{n + 2}
      &\le c \alpha^{n + 2}
  \end{align*}
  Una forma de lograr que ambas se cumplan es:
  \begin{align*}
    c \alpha^{n + 2}
      &\ge c \alpha^{n + 1} + c \alpha^n \\
    \alpha^2
      &\ge \alpha + 1
  \end{align*}
  El mínimo \(\alpha\) que funciona es \(\alpha = \tau = (1 + \sqrt{5}) / 2\)
  (el otro cero de la ecuación es menor,
   y podemos ignorarlo).
  Faltan los casos base,
  que determinan la constante \(c\).
  Resultan:
  \begin{align*}
    F_0
      &= 0 \le c \tau^0 \\
    F_1
      &= 1 \le c \tau^1
  \end{align*}
  Ambas se cumplen si elegimos \(c = 1 / \tau\),
  hemos demostrado
  \begin{equation*}
    F_n
      \le \tau^{n - 1}
  \end{equation*}
  Vamos por una cota inferior ahora.
  Podemos usar la misma estrategia con \(F_n \ge d \tau^n\),
  pero el único valor que funciona en todos los casos es el trivial \(d = 0\).
  Pero nos interesa \emph{que funcione para \(n\) grande},
  podemos omitir el caso \(n = 0\),
  que nos lleva a concluir que para \(n \ge 1\) funciona \(d = 1 / \tau^2\),
  y:
  \begin{equation*}
    \tau^{n - 2}
      \le F_n
      \le \tau^{n - 1}
  \end{equation*}
  Uniendo ambas cotas y ajustando la constante es:
  \begin{equation*}
    F_n
      = \Theta(\tau^n)
  \end{equation*}

  Veamos la recurrencia:
  \begin{equation*}
    T(n)
      = \sqrt{n} T(\sqrt(n)) + n
  \end{equation*}
  Es claro que \(T(n) > n\),
  pero no mucho.
  Si intentamos \(T(n) \le a n \log_2 n\),
  la demostración funciona bien:
  \begin{align*}
    T(n)
      &=   \sqrt{n} T(\sqrt{n}) + n \\
      &\le \sqrt{n} \cdot a \sqrt{n} \log_2 \sqrt{n} + n \\
      &=   \frac{a}{2} n \log_2 n + n \\
      &\le a n \log_2 n
  \end{align*}
  siempre que \(1 \le a/2 \log_2 n\),
  o \(n \ge 2^{2/a}\).
  Pero esto funciona para cualquier \(a\),
  no importa qué tan chico.
  Esto indica que nuestro intento es muy grande.
  Buscar \(b\) tal que \(T(n) \ge b n \log_2 n\) falla,
  lo que confirma que estamos mal.
  Debemos buscar alguna función entre \(n\) y \(n \log_2 n\).
  Intentemos \(n \log_2 \log_2 n\):
  \begin{align*}
    T(n)
      &\le \sqrt{n} \cdot a \sqrt{n} \log_2 \log_2 \sqrt{n} + n \\
      &=   a n \log_2 \log_2 n - a n + n \\
      &\le a n \log_2 \log_2 n
  \end{align*}
  Lo último siempre y cuando \(a \ge 1\),
  tener una cota inferior para \(a\)
  es fuerte indicación que vamos por el camino correcto.
  Intentando \(T(n) \ge b n \log_2 \log_2 n\)
  tenemos éxito siempre que \(b \le 1\),
  concluimos que \(T(n) = \Theta(n \log \log n)\).

\subsection{Coeficientes indeterminados}
\label{sec:coef-indet}

  La idea es combinar soluciones parciales,
  dependiendo de parámetros a determinar.
  Volvemos a los números de Fibonacci,
  que llevaron a sospechar soluciones de las formas
  \(c_1 \tau^n\) y \(c_2 \phi^n\),
  donde \(\tau\) y \(\phi\) son los ceros de \(x^2 = x + 1\):
  \begin{align*}
    \tau
      &= \frac{1 + \sqrt{5}}{2} \\
    \phi
      &= \frac{1 - \sqrt{5}}{2}
  \end{align*}
  Substituyendo el intento \(F_n = c_1 \tau^n + c_2 \phi^n\)
  en la recurrencia vemos que cumple el caso recursivo
  para todo \(c_1\) y \(c_2\),
  los casos base dan dos ecuaciones que permiten determinar las constantes:
  \begin{align*}
    F_n
      &= \frac{\tau^n - \phi^n}{\tau - \phi} \\
      &= \frac{(1 + \sqrt{5})^n - (1 - \sqrt{5})^n}{2^n \sqrt{5}}
  \end{align*}
  Es sorprendente que los números de Fibonacci,
  claramente enteros,
  se expresen en términos de números irracionales.

  Una variante es lo que Knuth llama \emph{método del repertorio}.
  La idea es armar un repertorio de secuencias \(f(n)\)
  con sus correspondientes resultados de aplicar los casos recursivos,
  para construir la solución que nos interesa
  mediante una combinación lineal.
  Un ejemplo tomado de Rossmanith~%
    \cite{rossmanith12:_analysis_algorithms}
  es hallar \(a_n\) tal que:
  \begin{equation*}
    a_n
      = n a_{n - 1} + n^2 a_{n - 2}
         - n^4 - 3 n^2 + 5
      \quad a_0 = 5, a_1 = 9
  \end{equation*}
  Es claro que si \(a_n\) es un polinomio en \(n\),
  lo es \(a_n - n a_{n - 1} - n^2 a_{n - 2}\).
  No tiene sentido ir más allá de \(n^2\),
  daría potencias mayores a \(n^4\).
  Armamos el cuadro~\ref{tab:operador} con algunos polinomios simples.
  \begin{table}[ht]
    \centering
    \begin{tabular}{*{2}{>{\(}l<{\)}}}
      \multicolumn{1}{c}{\boldmath\(a_n\)\unboldmath} &
        \multicolumn{1}{c}
            {\boldmath\(a_n - n a_{n - 1} - n^2 a_{n - 2}\)\unboldmath} \\
      \hline
      1	  & - n^2 - n + 1 \\
      n	  & - n^3 + n^2 + 2 n \\
      n^2 & - n^4 + 3 n^3 - n^2 - n
      \end{tabular}
    \caption{Valores del operador \(G = a_n - n a_{n - 1} - n^2 a_{n - 2}\)}
    \label{tab:operador}
  \end{table}
  Buscamos entonces constantes \(\alpha, \beta, \gamma\)
  tales que:
  \begin{equation*}
    \alpha G(1) + \beta G(n) + \gamma G(n^2)
      = - n^4 - 3 n^2 + 5
  \end{equation*}
  Resultan \(\alpha = 5\), \(\beta = 3\), \(\gamma = 1\).
  Tenemos la suerte que \(5 + 3 n + n^2\) cumple las condiciones iniciales,
  y la respuesta es:
  \begin{equation*}
    a_n
      = n^2 + 3 n + 5
  \end{equation*}

\section{Recurrencia lineal de primer orden}
\label{sec:lineal-recurrence-1st}

  Un caso especial importante es la recurrencia lineal de primer orden:
  \begin{equation*}
    a_{n + 1}
      = u_n a_n + f_n
  \end{equation*}
  Vemos que si dividimos por el \emph{factor sumador}:
  \begin{equation*}
    s_n
      = \prod_{0 \le k \le n} u_k
  \end{equation*}
  (lo que presupone que \(u_k \ne 0\) en el rango de interés)
  queda:
  \begin{equation*}
    \frac{a_{n + 1}}{s_n} - \frac{a_n}{s_{n - 1}}
      = \frac{f_n}{s_n}
  \end{equation*}
  Sumando ambos lados obtenemos la solución.

\section{Transformaciones}
\label{sec:transformaciones}

  En algunos casos las otras técnicas no pueden aplicarse directamente,
  pero una transformación de la recurrencia la lleva a una forma familiar.
  Las más importantes son las siguientes:
  \begin{description}
  \item[Transformación de dominio:]
    Defina una nueva función \(S(n) = T(f(n))\) para alguna función \(f\)
    que dé una recurrencia simple para \(S\).

    Por ejemplo,
    para mergesort si \(M(n)\) cuenta el número de comparaciones
    la recurrencia exacta es:
    \begin{equation*}
      M(n)
        = M(\lfloor n / 2 \rfloor) + M(\lceil n / 2 \rceil) + n - 1
        \quad M(0) = 0
    \end{equation*}
    El cambio de variables \(n = 2^k\), \(m(k) = M(2^k)\)
    y la observación \(m(0) = M(1) = 0\)
    lleva esto a la forma manejable
    de una recurrencia lineal de primer orden::
    \begin{equation*}
      m(k)
        = 2 m(k - 1) + 2^k - 1
        \quad m(0) = 0
    \end{equation*}
  \item[Transformación de rango:]
    Defina una nueva función \(S(n) = f(T(n))\) para alguna función \(f\)
    que dé una recurrencia simple para \(S\).

    Un ejemplo es la solución de Brand~%
      \cite{brand55:_seq_def_difference_eq}
    de la recurrencia de Ricatti:
    \begin{equation*}
      w_{n + 1}
        = \frac{a w_n + b}{c w_n + d}
        \quad w_0 \text{\ dado}
    \end{equation*}
    donde \(c \ne 0\)
    y \(a d - b c \ne 0\)
    (si \(c = 0\) es una recurrencia lineal de primer orden;
    si \(a d = b c\) se reduce a \(w_{n + 1} = \text{constante}\)).
    Si substituimos \(y_n \mapsto c w_n + d\),
    queda:
    \begin{equation*}
      y_n
        = \alpha - \frac{\beta}{y_{n - 1}}
    \end{equation*}
    donde:
    \begin{align*}
      \alpha
        &= a + d \\
      \beta
        &= a d - b c
    \end{align*}
    Claramente eso solo vale si \(a d - b c \ne 0\).
    Substituyendo ahora \(y_n \mapsto x_{n + 1} / x_n\)
    resulta:
    \begin{equation*}
      x_{n + 2} - \alpha x_{n + 1} + \beta x_n
        = 0
    \end{equation*}
    Necesitamos dos valores iniciales para resolverla,
    podemos elegir bastante arbitrariamente \(x_0 = 1\),
    dando \(x_1 = y_0\),
    que a su vez podemos obtener de la condición inicial original.
  \item[Diferencias:]
    Si aparecen sumas en los casos recursivos
    puede ser útil aplicar diferencias
    y buscar una recurrencia para \(S(n) = \Delta T(n) = T(n + 1) - T(n)\).

    Hay otras situaciones en las que diferencias ayudan a simplificar
    la recurrencia.
    Un ejemplo es la solución exacta a la recurrencia de mergesort
    discutida en el capítulo~\ref{cha:dividir-conquistar-solucion}.
  \end{description}

\section{Funciones generatrices}
\label{sec:recurrencias-funciones-generatrices}

  Podemos usar las herramientas de funciones generatrices.
  Dada la recurrencia,
  definimos una función generatriz adecuada,
  y aplicamos las propiedades respectivas para obtener una ecuación
  de la recurrencia.
  Resuelta la ecuación,
  extraemos coeficientes para obtener la solución.

  Por ejemplo,
  los números de Catalan \(C_n\) cumplen la recurrencia:
  \begin{equation*}
    C_{n + 1}
      = \sum_{0 \le k \le n} C_k C_{n - k}
      \quad C_0 = 1
  \end{equation*}
  Definiendo la función generatriz ordinaria:
  \begin{equation*}
    c(z)
      = \sum_{n \ge 0} C_n z^n
  \end{equation*}
  de las propiedades vemos que cumple:
  \begin{equation*}
    \frac{c(z) - C_0}{z}
      = c^2(z)
  \end{equation*}
  Despejando queda:
  \begin{equation*}
    z c^2(z) - c(z) + 1
      = 0
  \end{equation*}
  Obtenemos:
  \begin{equation*}
    c(z)
      = \frac{1 \pm \sqrt{1 - 4 z}}{2 z}
  \end{equation*}
  Sabemos que debe ser \(c(0) = C_0 = 1\),
  lo que descarta el signo positivo:
  \begin{equation*}
    c(z)
      = \frac{1 - \sqrt{1 - 4 z}}{2 z}
  \end{equation*}
  Usando la expansión de\((1 - 4 z)^{1/2}\)
  por el teorema del binomio y simplificando
  se obtiene finalmente:
  \begin{equation*}
    C_n
      = \frac{1}{n + 1} \binom{2 n}{n}
  \end{equation*}
  Alternativamente,
  podemos definir:
  \begin{equation*}
    u(z)
      = c(z) - 1
  \end{equation*}
  con lo que la ecuación toma la forma:
  \begin{equation*}
    u(z)
      = z (u(z) + 1)^2
  \end{equation*}
  Usando la fórmula de inversión de Lagrange con \(\phi(u) = (u + 1)^2\)
  y \(f(u) = u\) se obtiene directamente para \(n \ge 1\):
  \begin{align*}
    C_n
      &= [z^n] u(z) \\
      &= \frac{1}{n} [u^{n - 1}] (1 + u)^{2 n} \\
      &= \frac{1}{n} \binom{2 n}{n - 1} \\
      &= \frac{1}{n + 1} \binom{2 n}{n}
  \end{align*}
  Casualmente coincide \(C_0 = 1\) con esta fórmula.

  El número \(D_n\) de desarreglos
  (permutaciones sin puntos fijos)
  de \(n\) elementos cumple la recurrencia:
  \begin{equation*}
    D_{n + 1}
      = n D_n + n D_{n - 1}
      \quad D_0 = 1, D_1 = 0
  \end{equation*}
  Definimos la función generatriz exponencial:
  \begin{equation*}
    \widehat{d}(z)
      = \sum_{n \ge 0} D_n \frac{z^n}{n!}
  \end{equation*}
  Las propiedades dan:
  \begin{align*}
    \widehat{d}'(z)
      &\egf \left\langle D_{n + 1} \right\rangle_{n \ge 0} \\
    z \widehat{d}'(z)
      &\egf \left\langle n D_n \right\rangle_{n \ge 0}
  \end{align*}
  Falta el segundo término del lado derecho:
  \begin{equation*}
    \sum_{n \ge 1} n D_{n - 1} \, \frac{z^n}{n!}
      = z \sum_{n \ge 1} D_{n - 1} \, \frac{z^{n - 1}}{(n - 1)!}
      = z \widehat{d}(z)
  \end{equation*}
    Combinando las anteriores:
  \begin{align*}
    \widehat{d}'(z)
      &= z \widehat{d}'(z) + z \widehat{d}(z)
           \qquad \widehat{d}(0) = D_0 = 1 \\
    \frac{\widehat{d}'(z)}{\widehat{d}(z)}
      &= \frac{z}{1 - z}
  \end{align*}
  La solución de esta ecuación diferencial es:
  \begin{equation*}
    \widehat{d}(z)
      = \frac{\mathrm{e}^{-z}}{1 - z}
  \end{equation*}
  Por las propiedades,
  reconocemos la función generatriz ordinaria
  de las sumas parciales de la serie exponencial:
  \begin{align*}
    D_n
      &= n! [z^n] \widehat{d}(z) \\
      &= n! \sum_{0 \le k \le n} \frac{(-1)^k}{k!}
  \end{align*}

\section{Perturbación}
\label{sec:perturbacion}

  Una técnica aplicable para aproximar o acotar
  la solución a algunas recurrencias es el método de perturbación,
  que ilustramos mediante un ejemplo tomado de Sedgewick y Flajolet~%
    \cite{sedgewick13:_introd_anal_algor}.
  Sea la recurrencia:
  \begin{equation*}
    a_{n + 1}
      = 2 a_n + \frac{a_{n - 1}}{n^2}
      \quad a_0 = 1, a_1 = 2
  \end{equation*}
  Para valores grandes de \(n\)
  el último término tendrá poco efecto.
  Consideremos entonces la recurrencia aproximada:
  \begin{equation*}
    b_{n + 1}
      = 2 b_n
      \quad b_0 = 1
  \end{equation*}
  que tiene la solución obvia \(b_n = 2^n\),
  y comparemos los valores aproximados y exactos:
  \begin{equation*}
    p_n
      = \frac{a_n}{b_n}
      = 2^{-n} a_n
  \end{equation*}
  Substituyendo en la recurrencia exacta:
  \begin{equation*}
    2^{n + 1} p_{n + 1}
      = 2 \cdot 2^n p_n + \frac{2^{n - 1} p_{n - 1}}{n^2}
  \end{equation*}
  Esto se simplifica a:
  \begin{equation*}
    p_{n + 1}
      = p_n + \frac{p_{n - 1}}{4 n^2}
      \quad p_0 = p_1 = 1
  \end{equation*}
  Es claro que los \(p_n\) aumentan,
  por lo que:
  \begin{align*}
    p_{n + 1}
      &\le p_n \left( 1 + \frac{1}{4 n^2} \right) \\
      &\le \prod_{1 \le k \le n} \left( 1 + \frac{1}{4 k^2} \right) \\
      &<   \prod_{k \ge 1} \left( 1 + \frac{1}{4 k^2} \right)
  \end{align*}
  Este producto infinito
  puede expresarse usando la fórmula de producto de Euler para el seno:
  \begin{equation*}
    \sin \pi z
      = \pi z \prod_{k \ge 1} \left( 1 - \frac{z^2}{k^2} \right)
  \end{equation*}
  Con la fórmula para el seno en términos de exponenciales:
  \begin{equation*}
    \sin z
      = \frac{\mathrm{e^{\mathrm{i} z}} - \mathrm{e^{- \mathit{i} z}}}
             {2 \mathrm{i}}
  \end{equation*}
  vemos que nuestro producto es:
  \begin{align*}
    \prod_{k \ge 1} \left( 1 + \frac{1}{4 k^2} \right)
      &= \frac{\sin\left( \frac{\pi \mathrm{i}}{2} \right)}
              {\frac{\pi \mathrm{i}}{2}} \\
      &= \frac{\mathrm{e}^{\pi / 2} - \mathrm{e}^{- \pi / 2}}{\pi} \\
      &= 1,46505
  \end{align*}
  con lo que obtenemos:
  \begin{align*}
    a_n
      \sim \alpha 2^n
  \end{align*}
  donde sabemos que \(1 < \alpha < 1,46505\).

\section{Asintóticas}
\label{sec:asymptotics-recurrence}

  Como ejemplo de una técnica útil para obtener una estimación asintótica
  de la solución de una recurrencia,
  partiremos con la recurrencia planteada por Narayana Pandita,
  un matemático indio del siglo XIV.
  Plantea que al inicio tiene un ternero,
  que a los tres años se transforma en vaca.
  Cada vaca anualmente produce un nuevo ternero,
  y pregunta cuántos animales tiene al año \num{17}.
  Esto lleva a la recurrencia:
  \begin{equation}
    \label{eq:Narayana-cows-recurrence}
    a_{n + 3}
      = a_{n + 2} + a_n
      \qquad a_0 = a_1 = a_2 = 1
  \end{equation}
  La danza ritual da la función generatriz:
  \begin{align}
    A(z)
      &= \sum_{n \ge 0} a_n z^n
            \notag \\
      &= \frac{1}{1 - z - z^3}
            \label{eq:Narayana-cows-gf}
  \end{align}
  Lamentablemente,
  los ceros del polinomio característico \(p(\rho) = \rho^3 - \rho^2 - 1\)
  son expresiones muy complicadas,
  lo que se entorpece nuestra técnica de dividir en fracciones parciales
  y leer los coeficientes de allí.
  Tomaremos un camino indirecto.

  Primeramente,
  por la regla de signos de Descartes
  (ver por ejemplo~%
      \cite{levin20:_descartes_rule_signs})
  hay un solo cero real positivo
  (hay un único cambio de signo en la secuencia de coeficientes de \(p\))
  y ninguno negativo
  (\(p(-\rho)\) no tiene cambios de signo en su secuencia de coeficientes).
  Hay dos ceros complejos conjugados.
  Por las fórmulas de Vieta
  sabemos que el producto de los ceros es \(-1\)
  con lo que si \(\rho\) es el cero positivo,
  los ceros complejos tienen magnitud \(\rho^{-1/2}\).
  Una rápida aplicación del método de Newton
  (ver capítulo~\ref{cha:ceros-funciones})
  nos dice que \(\rho = 1,465571\),
  con lo que los ceros complejos tienen magnitud \num{0,8260314}.
  La solución será dominada por el cero real.

  De la expansión en fracciones parciales:
  \begin{align*}
    F(z)
      &= \frac{A}{1 - \rho z}
          + \frac{B_+}{1 - c_+ z} + \frac{B_-}{1 - c_- z} \\
    A
      &= \lim_{z \to 1 / \rho} F(z) (1 - \rho z)
  \end{align*}
  Podemos usar l'Hôpital para calcular el límite
  (por construcción,
   es de la forma \(0 / 0\)):
  \begin{align*}
    A
      &= \lim_{z \to 1 / \rho}	\frac{1 - \rho z}{1 - z - z^3} \\
      &= \lim_{z \to 1 / \rho} \frac{- \rho}{-1 - 3 z^2} \\
      &= \frac{\rho^3}{\rho^2 + 3} \\
      &= 0,6114918
  \end{align*}
  Sabemos entonces que:
  \begin{equation*}
    a_n
      = \frac{\rho^3}{\rho^2 + 3} \cdot \rho^n
        + O(\rho^{-n/2}) \\
  \end{equation*}
  Como una rápida confirmación,
  calculamos \(a_{10} = 28\),
  la fórmula aproximada da \num{27,95}.
  Resulta que aproximar al entero más cercano
  ya da el valor correcto para \(n = 0\).

  Un programita nos indica que \(a_{17} = 406\),
  nuestra aproximación es \num{405,98}.

%%% Local Variables:
%%% mode: latex
%%% TeX-master: "../INF-221_notas"
%%% ispell-local-dictionary: "spanish"
%%% End:

% LocalWords:  XIV programita


\bibliography{../referencias}

%%% Local Variables:
%%% mode: latex
%%% TeX-master: "../INF-221_notas"
%%% ispell-local-dictionary: "spanish"
%%% End:

% LocalWords:  mergesort

\documentclass[english, spanish, fleqn%
, hyperref = {colorlinks, urlcolor = blue}%
% , handout%
]{beamer}

\usepackage{beamerthemesplit}

\usepackage{fourier}
\usepackage{amsmath}
\usepackage{amsthm}
\usepackage{babel}
\usepackage{pgfplots}
\pgfplotsset{compat = newest}

\title{Asintóticas}
\subtitle{INF-221}

\author[Horst H. von Brand]{Horst H. von Brand\\
  \href{mailto:vonbrand@inf.utfsm.cl}{vonbrand@inf.utfsm.cl}}

\institute[DI UTFSM]{Departamento de Informática\\
                     Universidad Técnica Federico Santa María}
\date{}

\begin{document}
\beamerdefaultoverlayspecification{<+->}
\frame{\maketitle}

\begin{frame}
  \setcounter{beamerpauses}{2}
  \frametitle{Funciones ejemplo}

  \uncover<+->{
    Consideremos las funciones:
    \begin{align*}
      f(x)
        &= 3 x \cos 20 x \\
      g(x)
        &= \sin 5 x
    \end{align*}
  }
  \uncover<+->{
    Es simple demostrar que alrededor de \(0\)
    (en realidad,
     valen en todo \(\mathbb{R}\)):
    \begin{align*}
      f(x)
        &= O(x)	 \\
      g(x)
        &= O(x) \\
      f(x) + g(x)
        &= O(x) \\
      f(x) - g(x)
        &= O(x)
    \end{align*}
  }
\end{frame}

\begin{frame}
  \setcounter{beamerpauses}{2}
  \frametitle{Gráficos}
  \framesubtitle{La función \(f(x)\) y cotas \(\pm x\)}

  \begin{center}
    \begin{tikzpicture}
      \begin{axis}[axis lines = middle,
        xtick = \empty,
        ytick = \empty]
        \addplot [domain = -3.5:3.5, samples = 200, smooth, blue]
           {x * cos(deg(10 * x))};
        \addplot [domain = -3.5:3.5, red]
           {x};
        \addplot [domain = -3.5:3.5, red]
           {-x};
      \end{axis}
    \end{tikzpicture}
  \end{center}
\end{frame}

\begin{frame}
  \setcounter{beamerpauses}{2}
  \frametitle{Gráficos}
  \framesubtitle{La función \(g(x)\) y cotas \(\pm 4 x\)}

  \begin{center}
    \begin{tikzpicture}
      \begin{axis}[axis lines = middle,
        xtick = \empty,
        ytick = \empty]
        \addplot [domain = -3.5:3.5, samples = 200, smooth, blue]
           {sin(deg(4 * x))};
        \addplot [domain = -0.875:0.875, red]
           {4 * x};
        \addplot [domain = -0.875:0.875, red]
           {-4 * x};
      \end{axis}
    \end{tikzpicture}
  \end{center}
\end{frame}

\begin{frame}
  \setcounter{beamerpauses}{2}
  \frametitle{Gráficos}
  \framesubtitle{Las funciones \(f(x) \pm \: g(x)\) y cotas \(\pm 5 x\)}

  \begin{center}
    \begin{tikzpicture}
      \begin{axis}[axis lines = middle,
        xtick = \empty,
        ytick = \empty]
        \addplot [domain = -3.5:3.5, samples = 200, smooth, blue]
           {x * cos(deg(20 * x)) + sin(deg(4 * x))};
        \addplot [domain = -3.5:3.5, samples = 200, smooth, green]
           {x * cos(deg(20 * x)) - sin(deg(4 * x))};
        \addplot [domain = -0.8:0.8, red]
           {5 * x};
        \addplot [domain = -0.8:0.8, red]
           {-5 * x};
      \end{axis}
    \end{tikzpicture}
  \end{center}
\end{frame}

\begin{frame}
  \setcounter{beamerpauses}{2}
  \frametitle{Sumas y restas}

  \uncover<+->{
    Este último ejemplo muestra que una suma o resta de funciones
    debe considerar la suma de las cotas
    (en valor absoluto).
  }
\end{frame}

\begin{frame}
  \setcounter{beamerpauses}{2}
  \frametitle{Varios \(O(\cdot)\)}

  \uncover<+->{
    Note que al decir:
    \begin{align*}
      f(x) - g(x)
      &= O(x) + O(x) \\
      &= O(x)
    \end{align*}
    cada una de las veces que aparece \(O(\cdot)\)
    es una función \emph{diferente} la representada.
  }
\end{frame}

\begin{frame}
  \setcounter{beamerpauses}{2}
  \frametitle{Simplificando}

  \uncover<+->{
    Consideremos la expresión:
    \begin{equation*}
      \frac{1}{1 + O(x)}
    \end{equation*}
    Esto representa alguna expresión de la forma:
    \begin{equation*}
      \frac{1}{1 + h(x)} \qquad h(x) = O(x)
    \end{equation*}
  }
\end{frame}

\begin{frame}
  \setcounter{beamerpauses}{2}
  \frametitle{Simplificando}

  \uncover<+->{
    Sabemos que la  fracción puede expandirse,
    siempre que \(\lvert h(x) \rvert < 1\):
  }
  \begin{align*}
    \uncover<+->{
      \frac{1}{1 + h(x)}
        &=   \sum_{n \ge 0} (-1)^k h^k(x) \\
    }
    \uncover<+->{
        &=   1 - h(x) + \sum_{k \ge 2} (-1)^{k - 2} h^k(x) \\
    }
    \uncover<+->{
        &=   1 - h(x) + h^2(x) \sum_{k \ge 2} (-1)^{k - 2} h^{k - 2}(x) \\
    }
    \uncover<+->{
        &=   1 - h(x) + h^2(x) \sum_{k \ge 0}(-1)^k h^k(x) \\
    }
    \uncover<+->{
        &\le 1 - h(x) + h^2(x) \sum_{k \ge 0} \lvert h(x) \rvert^k
    }
  \end{align*}
\end{frame}

\begin{frame}
  \setcounter{beamerpauses}{2}
  \frametitle{Simplificando}

  \uncover<+->{
    Fijemos algún rango \(\lvert x \rvert \le \delta\)
    tal que en él \(\lvert h(x) \rvert \le 1 / 2\).
  }
  \uncover<+->{
    Con esto tenemos:
  }
  \begin{align*}
    \uncover<+->{
      \frac{1}{1 + h(x)}
        &\le 1 - h(x) + h^2(x) \sum_{k \ge 0} \lvert h(x) \rvert^k \\
    }
    \uncover<+->{
        &\le 1 - h(x) + h^2(x) \sum_{k \ge 0} 2^{-k} \\
    }
    \uncover<+->{
        &=   1 - h(x) + h^2(x) \frac{1}{1 - 1/2} \\
    }
    \uncover<+->{
        &=   1 - h(x) + 2 h^2(x) \\
    }
    \uncover<+->{
        &=   1 - h(x) + O(h^2(x))
    }
  \end{align*}
  \uncover<+->{
    Como tenemos \(h(x) = O(x)\),
    por la advertencia anterior concluimos:
    \begin{align*}
      \uncover<.->{
        \frac{1}{1 + O(x)}
          = 1 + O(x) + O(x^2)
      }
      \uncover<+->{
          = 1 + O(x)
      }
    \end{align*}
  }
\end{frame}

\begin{frame}
  \setcounter{beamerpauses}{2}
  \frametitle{Simplificando}

  \uncover<+->{
    Si buscamos simplificar una expresión más compleja,
    como ser:
    \begin{equation*}
      \frac{1 + O(x)}{1 + O(x)}
    \end{equation*}
  }
  \uncover<+->{
    Nuevamente,
    debe tenerse presente que ambos \(O(\cdot)\)
    representan funciones posiblemente diferentes.
  }
  \begin{align*}
    \uncover<+->{
      \frac{1 + O(x)}{1 + O(x)}
         &= (1 + O(x)) \cdot (1 + O(x))
             && \text{Por lo anterior} \\
    }
    \uncover<+->{
         &= 1 + (O(x) + O(x)) + O(x) \cdot O(x)
             && \text{Multiplicar expresiones} \\
    }
    \uncover<+->{
         &= 1 + O(x) + O(x^2)
             && \text{Agrupar, simplificar} \\
    }
    \uncover<+->{
         &= 1 + O(x)
             && \text{Absorber}
    }
  \end{align*}
\end{frame}

\begin{frame}
  \setcounter{beamerpauses}{2}
  \frametitle{Simplificando}

  \uncover<+->{
    Un ejemplo algo más complicado es:
  }
  \begin{align*}
    \uncover<.->{
      \frac{1 + 3 x + O(x^2)}{1 + 5 x + O(x^3)}
         &= (1 + 3 x + O(x^2)) \cdot {} \\
         &\qquad (1 - (5 x + O(x^3)) + (5 x + O(x^3)^2 - \dotsm) \\
    }
    \uncover<+->{
         &= (1 + 3 x + O(x^2)) \cdot {} \\
         &\qquad (1 - 5 x + O(x^3) + 25 x^2 + 10 x O(x^3) + O(x^6) - \dotsm) \\
    }
    \uncover<+->{
         &= (1 + 3 x + O(x^2)) \cdot {} \\
         &\qquad (1 - 5 x + 25 x^2 + O(x^3) + O(x^4) + O(x^6) - \dotsm) \\
    }
    \uncover<+->{
         &= (1 + 3 x + O(x^2))
               (1 - 5 x + 25 x^2 + O(x^3)) \\
    }
    \uncover<+->{
         &= 1 - 2 x + 10 x^2 + O(x^2) + \dotsm \\
    }
    \uncover<+->{
         &= 1 - 2 x + O(x^2)
    }
  \end{align*}
\end{frame}

\begin{frame}
  \setcounter{beamerpauses}{2}
  \frametitle{Simplificando}

  \uncover<+->{
    Esto último ilustra que en general no tiene sentido
    darse el trabajo de combinar aproximaciones de orden distinto,
    términos extra de la expresión más precisa terminan absorbidos.
  }
\end{frame}
\end{document}

%%% Local Variables:
%%% mode: latex
%%% TeX-master: t
%%% End:

\bibliographystyle{babplain-fl}

\chapter{Rendimiento de programas}
\label{apx:rendimiento-programas}

  Formalizaremos algunas técnicas que hemos usado
  para evaluar rendimiento de programas.
  Detalle exhaustivo de la idea da Knuth~%
    \cite{knuth97:_fundam_algor},
  una discusión más accesible y sistematizada es la de Rossmanith~%
    \cite{rossmanith12:_analysis_algorithms}.

\section{Número de veces que se ejecuta cada línea}
\label{sec:ejecucion-lineas}

  El primer paso para una evaluación rigurosa de los recursos gastados
  es determinar cuántas veces se ejecuta cada línea del algoritmo.
  Contando con costos precisos de cada una de ellas
  puede calcularse el costo total.
  Dividimos el programa en bloques básicos
  (designados por letras mayúsculas, \(A, B, \dotsc\),
  secuencias de instrucciones que siempre se ejecutan juntas,
  y las unimos por arcos dirigidos \(e_i\) entre bloques
  siempre que existe la posibilidad de que el control pase de un bloque a otro.
  Habrá un bloque en el que comienza el programa,
  y bloques en los cuales termina.
  Los conectamos por arcos ficticios.
  Llamaremos \(E_i\) al número de veces
  que el control pasa por el arco \(e_i\).
  La técnica se basa en lo siguiente,
  resultado bastante evidente si se considera que el control
  nunca se ve atrapado en un ciclo infinito:
  \begin{theorem}[Ley de Kirchhoff]
    \label{theo:Kirchhoff}
    Sea \(X\) un bloque básico,
    y sean \(\mathscr{I}\) el conjunto de \(i\) tales que el arco \(e_i\)
    entra a \(X\)
    y \(\mathscr{O}\) el conjunto de \(i\) tales que el arco \(e_i\)
    sale de \(X\).
    Entonces:
    \begin{equation*}
      \sum_{i \in \mathscr{I}} E_i
        = \sum_{i \in \mathscr{O}} E_i
    \end{equation*}
    La suma es el número de veces que se ejecutan las instrucciones en \(X\).
  \end{theorem}
  El primer paso es descomponer en bloques básicos,
  e identificar los arcos \(e_i\).
  Luego se halla un árbol recubridor del grafo,
  obviando que los arcos son dirigidos.
  Arcos que no pertenecen al árbol recubridor crean ciclos,
  que llamaremos \emph{ciclos fundamentales}.
  Describimos los ciclos eligiendo un arco perteneciente al árbol recubridor
  como positivo y recorremos el ciclo en esa dirección,
  asignando signos positivo o negativo según si el arco
  va en la dirección en que seguimos el ciclo o no.
  Lo interesante es que cada ciclo da una solución a la ley de Kirchhoff,
  si asignamos \(E_i = 0\) a los arcos que no pertenecen al ciclo
  y \(E_i = 1\) a los que pertenecen.
  Como las ecuaciones son lineales,
  combinaciones lineales de soluciones nuevamente son soluciones.
  Podemos expresar los flujos a través del ciclo \(C_i\) como \(\lambda_i\),
  planteando un sistema de ecuaciones.
  Esto nos permite expresar todos los flujos \(E_i\)
  en términos de un subconjunto de ellos.

\subsection{Descomposición en bloques básicos}
\label{sec:bloques-basicos}

  El primer paso es descomponer el programa en \emph{bloques básicos},
  secuencias de instrucciones que se ejecutan siempre en orden.
  Comienzan en el destino de control de flujo
  y terminan con alguna instrucción que
  (condicionalmente o no)
  cambia el flujo de control.
  Consideremos por ejemplo el algoritmo de Prim,
  algoritmo~\ref{alg:apx:Prim},
  para un árbol recubridor mínimo de grafo.
  Datos de entrada son un grafo conexo \(G = (V, E)\),
  una función de peso \(w \colon E \to \mathbb{R}\)
  y un vértice inicial \(s \in V\)
  (arbitrario).
  Anotaremos \(N(v)\) para el conjunto de vértices vecinos a \(v\) en \(G\).

  Representamos la solución mediante arreglos,
  el arreglo \(\pi[v]\) contendrá el padre en el árbol recubridor mínimo
  con \(s\) de raíz,
  \(\mathrm{key}[v]\) es el costo del camino más corto conocido
  entre el árbol actual y ese vértice.
  Suponemos la operación \(\mathrm{min-from}(M)\)
  que extrae un vértice del conjunto de vértices \(M\)
  con mínimo valor de \(\mathrm{key}[v]\).
  \begin{algorithm}[ht]
    \DontPrintSemicolon\Indp

    \ForEach{\(u \in V\)}{
      \(\mathrm{key}[u] \gets \infty\) \;
      \(\pi[u] \gets \mathrm{\mathbf{nil}}\) \;
    }
    \(\mathrm{key}[s] \gets 0\) \;
    \(M \gets V\) \;
    \While{\(M \ne \varnothing\)}{
      \(u \gets \mathrm{min-from}(M)\) \;
      \ForEach{\(v \in N(u)\)}{
        \If{\((v \in M) \wedge (w(u, v) < \mathrm{key}[v])\)}{
          \(\pi[v] \gets u\) \;
          \(\mathrm{key}[v] \gets w(u, v)\) \;
        }
      }
    }
    \caption{Algoritmo de Prim}
    \label{alg:apx:Prim}
  \end{algorithm}
  Este programa se representa en forma de diagrama de flujo
  (grafo de bloques básicos)
  en la figura~\ref{fig:Prim}.
  La convención es que flujos \textquote{hacia abajo}
  representan salida con condición al final del bloque verdadera,
  flujos \textquote{hacia el lado} representan condiciones falsas.
  \begin{figure}[ht]
    \centering
    \begin{tikzpicture}[block/.style = {rectangle, rounded corners, draw,
                                        minimum width  = 10em,
                                        minimum height =  2em,
                                        text width     =  9em},
                        flow/.style = {-latex'},
                        stflow/.style ={thick, -latex'}]
      \node	      (0) {};
      \node[block, below of = 0, yshift = 1.5em, label = north west:{\(A\)}]
                      (A) {\(M \leftarrow V\)};
      \coordinate[below of = A, yshift = 2.5em]
                      (AB) {};
      \node[block, below of = AB, yshift = 2.5em, label = north west:{\(B\)}]
                      (B) {\(M \ne \varnothing\)};
      \node[block, below of = B, label = north west:{\(C\)}]
                      (C) {\(u \leftarrow \mathrm{any-from}(M)\) \\
                           \(\mathrm{key}[u] \leftarrow \infty\) \\
                           \(\pi[u] \leftarrow \mathrm{\mathbf{nil}}\)};
      \coordinate[below of = C, yshift = 2.5em]
                      (CD) {};
      \node[block, below of = CD, yshift = 2.5em, label = north west:{\(D\)}]
                      (D) {\(\mathrm{key}[s] \leftarrow 0\) \\
                           \(M \leftarrow V\)};
      \node[below of = D, yshift = 2.5em]
                      (DE) {};
      \node[block, below of = DE, yshift = 2.5em, label = north west:{\(E\)}]
                      (E) {\(M \ne \varnothing\)};

      \node[block, below of = E, yshift = 0.5em, label = north west:{\(F\)}]
                      (F) {\(u \leftarrow \mathrm{min-from}(M)\) \\
                           \(N \leftarrow N(u)\)};
      \node[below of = F, yshift = 2.5em]
                      (FG) {};
      \node[block, below of = FG, yshift = 2.5em, label = north west:{\(G\)}]
                      (G) {\(N \ne \varnothing\)};
      \node[block, below of = G, yshift = 0.5em, label = north west:{\(H\)}]
                      (H) {\(v \leftarrow \mathrm{any-from}(N)\) \\
                           \(v \in M\)};
      \node[block, below of = H, yshift = 0.5em, label = north west:{\(I\)}]
                      (I) {\(w(u, v) < \mathrm{key}[v]\)};
      \node[block, below of = I, yshift = 0.5em, label = north west:{\(J\)}]
                      (J) {\(\pi[v] \leftarrow v\) \\
                           \(\mathrm{key}[v] \leftarrow w(u, v)\)};
      \coordinate[below of = J, yshift = 2.5em]
                      (N) {};

      \coordinate[right of = AB, node distance =    8em] (ABR)	{};
      \coordinate[right of = AB, node distance =  2.3em] (BA)	{};
      \coordinate[left of  = B,	 node distance =    8em] (BL)	{};
      \coordinate[left of  = CD, node distance =  2.2em] (DA)	{};
      \coordinate[left of  = CD, node distance =    8em] (CDL)	{};
      \coordinate[right of = CD, node distance =    8em] (CDR)	{};
      \coordinate[right of = DE, node distance =  2.3em] (EA)	{};
      \coordinate[right of = DE, node distance =    8em] (DER)	{};
      \coordinate[left of  = E,	 node distance =   10em] (EL)	{};
      \coordinate[left of  = FG, node distance =  2.3em] (GA)	{};
      \coordinate[left of  = FG, node distance =    8em] (FGL1) {};
      \coordinate[left of  = FG, node distance =   10em] (FGL2) {};
      \coordinate[left of  = FG, node distance =   12em] (FGL3) {};
      \coordinate[right of = G,	 node distance =    8em] (GR)	{};
      \coordinate[left of  = H,	 node distance =    8em] (HL)	{};
      \coordinate[left of  = I,	 node distance =   10em] (IL)	{};
      \coordinate[left of  = N,	 node distance =   12em] (NL)	{};

      \draw[flow]   (0)	   -- node [left]  {\(e_0\)}	(A);
      \draw[stflow] (A)	   -- node [left]  {\(e_1\)}	(B);
      \draw[stflow] (B)	   -- node [left]  {\(e_2\)}	(C);
      \draw[flow]   (C)	   -- (CD) -- (CDR) -- node [right]  {\(e_3\)} (ABR)
                           -- (BA) -- (BA|-B.north);
      \draw[stflow] (B)	   -- (BL) -- node [left] {\(e_4\)} (CDL)
                           -- (DA) -- (DA|-D.north);
      \draw[stflow] (D)	   -- node [left]  {\(e_5\)}	(E);
      \draw[stflow] (E)	   -- node [left]  {\(e_6\)}	(F);
      \draw[stflow] (F)	   -- node [right] {\(e_7\)}	(G);
      \draw[stflow] (G)	   -- node [left]  {\(e_8\)}	(H);
      \draw[stflow] (H)	   -- node [left]  {\(e_9\)}	(I);
      \draw[stflow] (I)	   -- node [left]  {\(e_{10}\)} (J);
      \draw[flow]   (I)	   -- node [above] {\(e_{11}\)} (IL)
                           -- (FGL2);
      \draw[flow]   (J)	   -- (N) -- node [above, near end] {\(e_{12}\)} (NL)
                           -- (FGL3) -- (GA) -- (GA|-G.north);
      \draw[flow]   (H)	   -- (HL) -- node [right] {\(e_{13}\)} (FGL1);
      \draw[flow]   (G)	   -- (GR)  -- node [right]  {\(e_{14}\)} (DER)
                           -- (EA) -- (EA|-E.north);
      \draw[flow]   (E)	   -- node [above, near end] {\(e_{15}\)} (EL);
    \end{tikzpicture}
    \caption{Diagrama de flujo del algoritmo de Prim}
    \label{fig:Prim}
  \end{figure}
  Note que hemos separado la inicialización de los ciclos
  de sus cuerpos,
  se ejecutan aparte.
  Para representar el elegir un vértice cualquiera
  estamos usando la operación
  \(\mathrm{any-from}(M)\),
  que elige un elemento cualquiera del conjunto \(M\),
  lo elimina de este y lo retorna.

\subsection{Aplicar ley de Kirchhoff}
\label{sec:aplicar-Kirchhoff}

  En la figura~\ref{fig:Prim} hemos marcado los arcos
  \(e_1, e_2, e_4, e_5, e_6, e_7, e_8, e_9\) y \(e_{10}\)
  (estamos identificando \(e_0 = e_{15}\),
   un flujo ficticio a través del \textquote{exterior})
  del árbol recubridor elegido en negrita.
  Esto da los ciclos fundamentales representados como:
  \begin{align*}
    C_0
      &= e_0 + e_1 + e_4 + e_5 \\
    C_1
      &= e_2 + e_3 \\
    C_2
      &= e_6 + e_7 + e_{14} \\
    C_3
      &= e_8 + e_{13} \\
    C_4
      &= e_8 + e_9 + e_{11} \\
    C_5
      &= e_8 + e_9 + e_{10} + e_{12}
  \end{align*}
  Por la ley de Kirchhoff podemos representar los flujos como:
  \begin{align*}
    \begin{pmatrix}
      E_0 \\
      E_1 \\
      E_2 \\
      E_3 \\
      E_4 \\
      E_5 \\
      E_6 \\
      E_7 \\
      E_8 \\
      E_9 \\
      E_{10} \\
      E_{11} \\
      E_{12} \\
      E_{13} \\
      E_{14}
    \end{pmatrix}
      &= \lambda_0
           \begin{pmatrix}
             1 \\
             1 \\
             0 \\
             0 \\
             1 \\
             1 \\
             0 \\
             0 \\
             0 \\
             0 \\
             0 \\
             0 \\
             0 \\
             0 \\
             0
           \end{pmatrix}
         + \lambda_1
           \begin{pmatrix}
             0 \\
             0 \\
             1 \\
             1 \\
             0 \\
             0 \\
             0 \\
             0 \\
             0 \\
             0 \\
             0 \\
             0 \\
             0 \\
             0 \\
             0
           \end{pmatrix}
         + \lambda_2
           \begin{pmatrix}
             0 \\
             0 \\
             0 \\
             0 \\
             0 \\
             0 \\
             1 \\
             1 \\
             0 \\
             0 \\
             0 \\
             0 \\
             0 \\
             0 \\
             1
           \end{pmatrix}
         + \lambda_3
           \begin{pmatrix}
             0 \\
             0 \\
             0 \\
             0 \\
             0 \\
             0 \\
             0 \\
             0 \\
             1 \\
             0 \\
             0 \\
             0 \\
             0 \\
             1 \\
             0
           \end{pmatrix}
         + \lambda_4
           \begin{pmatrix}
             0 \\
             0 \\
             0 \\
             0 \\
             0 \\
             0 \\
             0 \\
             0 \\
             1 \\
             1 \\
             0 \\
             1 \\
             0 \\
             0 \\
             0
           \end{pmatrix}
         + \lambda_5
           \begin{pmatrix}
             0 \\
             0 \\
             0 \\
             0 \\
             0 \\
             0 \\
             0 \\
             0 \\
             1 \\
             1 \\
             1 \\
             0 \\
             1 \\
             0 \\
             0
           \end{pmatrix} \\
      &= \begin{pmatrix}
           1 & 0 & 0 & 0 & 0 & 0 \\
           1 & 0 & 0 & 0 & 0 & 0 \\
           0 & 1 & 0 & 0 & 0 & 0 \\
           0 & 1 & 0 & 0 & 0 & 0 \\
           1 & 0 & 0 & 0 & 0 & 0 \\
           1 & 0 & 0 & 0 & 0 & 0 \\
           0 & 0 & 1 & 0 & 0 & 0 \\
           0 & 0 & 1 & 0 & 0 & 0 \\
           0 & 0 & 0 & 1 & 1 & 1 \\
           0 & 0 & 0 & 0 & 1 & 1 \\
           0 & 0 & 0 & 0 & 0 & 1 \\
           0 & 0 & 0 & 0 & 1 & 0 \\
           0 & 0 & 0 & 0 & 0 & 1 \\
           0 & 0 & 0 & 1 & 0 & 0 \\
           0 & 0 & 1 & 0 & 0 & 0
         \end{pmatrix}
           \cdot \begin{pmatrix}
                   \lambda_0 \\
                   \lambda_1 \\
                   \lambda_2 \\
                   \lambda_3 \\
                   \lambda_4 \\
                   \lambda_5
                 \end{pmatrix}
  \end{align*}
  Interesa identificar un conjunto de seis flujos cuyas filas en la matriz
  sean linealmente independientes,
  y expresar las demás en términos de ellas.
  Notamos que \(E_0 = E_1 = E_4 = E_5\),
  \(E_2 = E_3\),
  también \(E_6 = E_7 = E_{14}\)
  y finalmente \(E_{10} = E_{12}\)
  ya que sus filas son iguales.
  Una mirada al diagrama de flujo~\ref{fig:Prim}
  confirma estas igualdades.

  Buscamos ahora expresar los demás flujos
  en términos de un conjunto de flujos linealmente independientes.
  Tenemos varias igualdades ya,
  completando con las expresiones para \(E_8\) y \(E_9\)
  todos quedan expresados en términos de
  \(E_0, E_2, E_6, E_{10}, E_{11}\) y~\(E_{13}\):
  \begin{align*}
    E_1
      &= E_0 \\
    E_3
      &= E_2 \\
    E_4
      &= E_0 \\
    E_5
      &= E_0 \\
    E_7
      &= E_6 \\
    E_8
      &= E_9 + E_{13} \\
    E_9
      &= E_{10} + E_{11} \\
    E_{12}
      &= E_{10} \\
    E_{14}
      &= E_6
  \end{align*}

  Sabemos que el programa se invoca una única vez,
  por lo que \(E_0 = 1\).
  Los números de veces que se repiten los demás ciclos
  deberemos determinarlos.
  Conociendo los flujos,
  podemos calcular cuántas veces se ejecuta cada vértice
  (bloque básico).
  Llamando como el bloque en el diagrama de flujo
  al número de veces que se le ejecuta resulta:
  \begin{align*}
    A
      &= E_0 \\
    B
      &= E_0 + E_3 \\
      &= E_0 + E_2 \\
    C
      &= E_2 \\
    D
      &= E_4 \\
      &= E_0 \\
    E
      &= E_5 + E_{14} \\
      &= E_0 + E_6 \\
    F
      &= E_6 \\
    G
      &= E_7 + E_{11} + E_{12} + E_{13} \\
      &= E_6 + E_{10} + E_{11} + E_{13} \\
    H
      &= E_8 \\
      &= E_9 + E_{13} \\
    I
      &= E_9 \\
    J
      &= E_{10}
  \end{align*}
  Del programa vemos que:
  \begin{align*}
    E_0
      &= 1 \\
    E_2
      &= E_6 = \lvert V \rvert
  \end{align*}
  Los demás valores dependen del grafo,
  del orden en que se eligen nodos
  y de la función \(w\).

\subsection{Llamadas a funciones}
\label{sec:llamadas-funciones}

  En el caso que el programa llame a una función,
  simplemente el costo de esa línea será el costo de esa función
  para esos argumentos específicos.
  En caso que sea una llamada recursiva,
  esto da lugar a una recurrencia.
  Por ejemplo,
  considere el programa~\ref{lst:apx-snapl}.
  \lstinputlisting[language = C++,
                   firstline = 3,
                   xleftmargin = 3em,
                   numbers = none,
                   caption = {El programa \texttt{snapl}},
                   label = lst:apx-snapl]
                  {code/snapl.cc}
  Nos interesa el número \(a_n\) de números
  que escribe la llamada \lstinline[language = C++]!snapl(n)!.

  Resulta la recurrencia:
  \begin{equation*}
    a_n
      = 3 a_{n - 2} + 2 a_{n - 3}
      \quad a_0 = 0, a_1 = 1, a_2 = 0
  \end{equation*}
  Técnicas estándar de solución de recurrencias dan:
  \begin{equation*}
    a_n
      = \frac{2^{n + 1} - (-1)^n (3 n + 2)}{9}
  \end{equation*}

\section{Análisis probabilístico}
\label{sec:analisis-probabilistico}

  Analicemos el programa~\ref{lst:apx-maximo}.
  \lstinputlisting[language = C,
                   firstline = 6,
                   xleftmargin = 3em,
                   numbers = left, stepnumber = 1, numberstyle = {\small},
                   caption = {Hallar el máximo},
                   label = lst:apx-maximo]
                   {code/maximum.c}
  Es claro que\ldots{}

  Nos interesa específicamente el número de asignaciones
  a la variable \lstinline[language = C]!max!.
  Al efecto,
  supondremos que los elementos son todos distintos,
  son una permutación elegida uniformemente al azar de \(n\) elementos.

  El mínimo es \num{1}
  (si \lstinline[language = C]!a[0]! es el máximo).
  La probabilidad de este caso es \(1/n\)
  (hay \(n\) valores posibles de \lstinline[language = C]!a[0]!,
   todos igualmente probables,
   solo uno de ellos corresponde al evento de interés).

  El máximo es \(n\)
  (si vienen en orden).
  La probabilidad de este caso es \(1/n!\)
  (hay \(n!\) permutaciones,
   solo una de ellas con los elementos en orden).

  Para el promedio,
  consideremos el caso que para \(i\) se ejecuta la línea~9.
  Podemos describirlo diciendo que elegimos \(i + 1\) elementos para
  \lstinline[language = C]!a[0]! a \lstinline[language = C]!a[i]!
  entre los \(n\),
  uno de ellos es el máximo
  (en \lstinline[language = C]!a[i]!).
  Se permutan los demás,
  \(i\) antes de \lstinline[language = C]!a[i]!
  y \(n - i - 1\) después.
  Como en total hay \(n!\) permutaciones
  la probabilidad de este evento es:
  \begin{align*}
    \frac{\binom{n}{i + 1} i! (n - i - 1)!}{n!}
      &= \frac{n!}{(i + 1)! (n - i - 1)!} \cdot \frac{i! (n - i - 1)!}{n!} \\
      &= \frac{1}{i + 1}
  \end{align*}
  Por la linealidad del valor esperado,
  el número esperado de veces que se ejecuta la línea~9 es:
  \begin{align*}
    \sum_{1 \le i < n} \Pr[\text{línea 9 se ejecuta para \(i\)}]
      &= \sum_{1 \le i < n} \frac{1}{i + 1} \\
      &= \sum_{2 \le i \le n} \frac{1}{i} \\
      &= H_n - 1
  \end{align*}
  A las asignaciones de la línea~9
  debemos sumar la asignación de la línea~4,
  dando que el valor esperado del número total
  de asignaciones a \lstinline[language = C]!max!
  es \(H_n \sim \ln n\).

  En resumen,
  las líneas 6 y 10 se ejecutan siempre \num{1} vez,
  las líneas 7 y 8 siempre se ejecutan \(n - 1\) veces,
  mientras la línea 9 se ejecuta
  mínimo \num{0}~veces,
  máximo~\(n - 1\) veces,
  en promedio \(H_n - 1\) veces.
  Asignaciones en total son así
  mínimo \num{1}~vez,
  máximo \(n\)~veces,,
  promedio \(H_n \sim \ln n\)~veces.

\bibliography{../referencias}

%%% Local Variables:
%%% mode: latex
%%% TeX-master: "../INF-221_notas"
%%% ispell-local-dictionary: "spanish"
%%% End:

% LocalWords:  recubridor

\bibliographystyle{babplain-fl}

\chapter{Espacios normados}
\label{apx:espacios-normados}

  Nos interesan espacios vectoriales
  a los cuales les adjuntamos un concepto \emph{producto interno},
  y también de \emph{norma}.
  Partiremos con las definiciones del caso,
  para luego mostrar algunas consecuencias simples que usamos en el texto.

\section{Espacio vectorial}
\label{sec:espacio-vectorial}

  \begin{definition}
    \label{def:espacio-vectorial}
    Un \emph{espacio vectorial} \(V\)
    sobre un campo \(\mathbb{F}\)
    (para nosotros casi siempre \(\mathbb{R}\),
     ocasionalmente \(\mathbb{C}\))
    es un conjunto de elementos,
    \(\mathbf{x}, \mathbf{y}, \dotsc \in V\) con una operación de \emph{suma}
    \(\mathbf{x} + \mathbf{y}\) que entrega un nuevo vector,
    y una operación de \emph{multiplicación escalar}
    que dados \(\alpha \in \mathbb{F}\) y \(\mathbf{x} \in V\)
    retorna un vector \(\alpha \mathbf{x}\),
    que cumplen los siguientes axiomas:
    \begin{enumerate}
    \item
      La suma es \emph{asociativa}:
      para todo \(\mathbf{x}, \mathbf{y}, \mathbf{z} \in V\)
      es \((\mathbf{x} + \mathbf{y}) + \mathbf{z}
              = \mathbf{x} + (\mathbf{y} + \mathbf{z})\).
    \item
      La suma es \emph{conmutativa}:
      para todo \(\mathbf{x}, \mathbf{y} \in V\)
      es \(\mathbf{x} + \mathbf{y} = \mathbf{y} + \mathbf{x}\).
    \item
      Hay un \emph{elemento neutro} \(\mathbf{0} \in V\)
      tal que para todo \(\mathbf{x} \in V\)
      es \(\mathbf{x} + \mathbf{0} = \mathbf{0} + \mathbf{x} = \mathbf{x}\).
    \item
      Para cada \(\mathbf{x} \in V\) existe un \emph{inverso aditivo}
      \(- \mathbf{x} \in V\)
      tal que \(\mathbf{x} + (- \mathbf{x}) = \mathbf{0}\).
    \item
      La multiplicación escalar es \emph{compatible}:
      para todos \(\alpha, \beta \in \mathbb{F}\)
      y todo \(\mathbf{x} \in V\)
      es \(\alpha (\beta \mathbf{x}) = (\alpha \beta) \mathbf{x}\).
    \item
      Elemento \emph{identidad} para multiplicación escalar:
      si \(1 \in \mathbb{F}\) es la identidad multiplicativa,
      entonces para todo \(\mathbf{x} \in V\) es \(1 \mathbf{x} = \mathbf{x}\).
    \item
      La multiplicación escalar distribuye sobre la suma vectorial:
      para todo \(\alpha \in \mathbb{F}\)
      y todos \(\mathbf{x}, \mathbf{y} \in V\)
      es \(\alpha (\mathbf{x} + \mathbf{y})
             = \alpha \mathbf{x} + \alpha \mathbf{y}\).
    \item
      La multiplicación escalar distribuye sobre la suma escalar:
      para todo \(\alpha, \beta \in \mathbb{F}\)
      y todo \(\mathbf{x} \in V\)
      es \((\alpha + \beta) \mathbf{x}
              = \alpha \mathbf{x} + \beta \mathbf{x}\).
    \end{enumerate}
  \end{definition}
  Comúnmente
  se abrevia \(\mathbf{x} + (- \mathbf{y}) = \mathbf{x} - \mathbf{y}\),
  y omitimos paréntesis dadas la asociatividad y la compatibilidad.

  De las anteriores pueden demostrarse consecuencias simples,
  como las siguientes:
  \begin{enumerate}
  \item
    El elemento neutro \(\mathbf{0} \in V\) es único.
  \item
    Para \(\mathbf{x} \in V\),
    el inverso aditivo \(-\mathbf{x}\) es único.
  \item
    Para todo \(\mathbf{x} \in V\),
    \(0 \mathbf{x} = \mathbf{0}\).
  \item
    Si \(\alpha \mathbf{x} = \mathbf{0}\),
    es \(\alpha = 0\) o \(\mathbf{x} = \mathbf{0}\).
  \end{enumerate}

\section{Espacios normados}
\label{sec:espacios-normados}

  \begin{definition}
    \label{def:norma}
    Sea \(V\) un espacio vectorial sobre el campo \(\mathbb{F}\)
    (reales o complejos).
    Una \emph{norma} en \(V\) es una función
    \(\lVert \cdot \rVert \colon V \to \mathbb{R}\) que cumple los axiomas:
    \begin{enumerate}
    \item
      Homogeneidad:
      para todo \(\alpha \in \mathbb{F}\) y todo \(\mathbf{x} \in V\)
      es \(\lVert \alpha \mathbf{x} \rVert
             = \lvert \alpha \rvert \lVert \mathbf{x} \rVert\).
    \item
      Desigualdad triangular:
      para todo \(\mathbf{x}, \mathbf{y} \in V\)
      se cumple
      \(\lVert \mathbf{x} + \mathbf{y} \rVert
          \le \lVert \mathbf{x} \rVert + \lVert \mathbf{y} \rVert\)
    \item
      Si \(\lVert \mathbf{x} \rVert = 0\) entonces \(\mathbf{x} = \mathbf{0}\).
    \end{enumerate}
  \end{definition}
  De acá se demuestra:
  \begin{enumerate}
  \item
    Identidad triangular reversa:
    para todo \(\mathbf{x}, \mathbf{y} \in V\) es
    \(\lvert \lVert \mathbf{x} \rVert - \lVert \mathbf{y} \rVert \rvert
        \le \lVert \mathbf{x} - \mathbf{y} \rVert\)
  \end{enumerate}
  Intuitivamente,
  interpretamos \(\lVert \mathbf{x} - \mathbf{y} \rVert\)
  como la \emph{distancia} entre \(\mathbf{x}\) e \(\mathbf{y}\).

\section{Producto interno}
\label{sec:producto-interno}

  Nos interesan espacios vectoriales sobre el campo escalar \(\mathbb{F}\)
  (reales o complejos)
  con la operación de \emph{producto interno} entre vectores.
  \begin{definition}
    \label{def:producto-interno-apx}
    Sea \(V\) un espacio vectorial sobre el campo \(\mathbb{F}\)
    (reales o complejos).
    Un \emph{producto interno} en \(V\) es una operación
    \(\langle \cdot, \cdot \rangle \colon V \times V \to \mathbb{F}\)
    que cumple los siguientes axiomas:
    \begin{enumerate}
    \item
      Simetría conjugada:
      para todo \(\mathbf{x}, \mathbf{y} \in V\) es
      \(\langle \mathbf{x}, \mathbf{y} \rangle
          = \overline{\langle \mathbf{y}, \mathbf{x} \rangle}\)
    \item
      Lineal en el primer argumento:
      para todo \(\mathbf{x}, \mathbf{y}, \mathbf{z} \in V\)
      y \(\alpha, \beta \in \mathbb{F}\) es
      \(\langle \alpha \mathbf{x} + \beta \mathbf{y}, \mathbf{z} \rangle
         = \alpha \langle \mathbf{x}, \mathbf{z} \rangle
             + \beta \langle \mathbf{y}, \mathbf{z} \rangle\)
    \item
      Positivo definido:
      para todo \(\mathbf{x} \in V\) es
      \(\langle \mathbf{x}, \mathbf{x} \rangle \ge 0\)
      (por simetría conjugada es siempre real)
      y \(\langle \mathbf{x}, \mathbf{x} \rangle = 0\)
      si y solo si \(\mathbf{x} = 0\).
    \end{enumerate}
  \end{definition}
  Consecuencias simples son:
  \begin{enumerate}
  \item
    \(\langle \mathbf{x}, \mathbf{y} + \mathbf{z} \rangle
        = \langle \mathbf{x}, \mathbf{y} \rangle
            + \langle \mathbf{x}, \mathbf{z} \rangle\)
  \end{enumerate}

  Vemos que \(\lVert \mathbf{x} \rVert
                 = \langle \mathbf{x}, \mathbf{x} \rangle^{1/2}\)
  es una norma en \(V\).

  Podemos interpretar \(\langle \mathbf{x}, \mathbf{y} \rangle\)
  como indicativo del \textquote{ángulo} entre \(\mathbf{x}\) e \(\mathbf{y}\),
  si \(\langle \mathbf{x}, \mathbf{y} \rangle = 0\) son ortogonales.

  El teorema siguiente relaciona normas con productos internos.
  \begin{theorem}[Desigualdad de Cauchy-Schwartz]
    \label{theo:Cauchy-Schwartz}
    \begin{equation*}
      \lvert \langle \mathbf{x}, \mathbf{y} \rangle \lvert
        \le \lVert \mathbf{x} \rVert \cdot \lVert \mathbf{y} \rVert
    \end{equation*}
  \end{theorem}
  \begin{proof}
    La demostración en sí es simple,
    el razonamiento previo
    es para justificar el \(t\) misterioso empleado en ella.

    Si \(\mathbf{x}\) o \(\mathbf{y}\) son cero,
    el resultado es trivial.
    Supongamos \(\mathbf{y} \ne 0\).
    Por las propiedades del producto interno,
    para todo escalar \(t\):
    \begin{align*}
      0
        &\le \lVert \mathbf{x} - t \mathbf{y} \rVert^2 \\
        &=   \langle \mathbf{x} - t \mathbf{y},
                     \mathbf{x} - t \mathbf{y} \rangle \\
        &=   \langle \mathbf{x}, \mathbf{x} - t \mathbf{y} \rangle
                - t \langle \mathbf{y}, \mathbf{x} - t \mathbf{y} \rangle \\
        &=   \lVert \mathbf{x} \rVert^2
                - t \left(
                      \langle \mathbf{x}, \mathbf{y} \rangle
                        + \overline{\langle \mathbf{x}, \mathbf{y} \rangle}
                    \right)
                + t^2 \lVert \mathbf{y} \rVert^2 \\
        &=   \lVert \mathbf{x} \rVert^2
                - 2 t \Re \langle \mathbf{x}, \mathbf{y} \rangle
                + t^2 \lVert \mathbf{y} \rVert^2 \\
    \end{align*}
    Se obtiene lo prometido al substituir el valor de \(t\)
    que minimiza la expresión indicada:
    \begin{align*}
      t
        =   \frac{\Re \langle \mathbf{x}, \mathbf{y} \rangle}
                 {\lVert \mathbf{y} \rVert^2} \\
        \le \frac{\lvert \langle \mathbf{x}, \mathbf{y} \rangle \rvert}
                 {\lVert \mathbf{y} \rVert^2}
    \end{align*}
    \qedhere
  \end{proof}
  Tenemos el importante corolario:
  \begin{corollary}[Desigualdad triangular]
    \label{cor:desigualdad-triangular}
    \(\lVert \mathbf{x} + \mathbf{y} \rVert
        \le \lVert \mathbf{x} \rVert + \lVert \mathbf{y} \rVert\)
  \end{corollary}
  \begin{proof}
    \begin{align*}
      \lVert \mathbf{x} + \mathbf{y} \rVert^2
        &=   \langle
               \mathbf{x} + \mathbf{y},
               \mathbf{x} + \mathbf{y}
             \rangle \\
        &=   \lVert \mathbf{x} \rVert^2
               + \langle \mathbf{x}, \mathbf{y} \rangle
               + \langle \mathbf{y}, \mathbf{x} \rangle
               + \lVert \mathbf{y} \rVert^2 \\
        &\le \lVert \mathbf{x} \rVert^2
               + 2 \lVert \mathbf{x} \rVert \cdot \lVert \mathbf{y} \rVert
               + \lVert \mathbf{x} \rVert^2 \\
        &= ( \lVert \mathbf{x} \rVert + \lVert \mathbf{y} \rVert )^2
    \end{align*}
    \qedhere
  \end{proof}

\section{Construir conjunto de vectores ortogonales}
\label{sec:construir-vectores-ortogonales}

  Es común querer construir un conjunto de vectores ortogonales
  dado un conjunto de vectores linealmente independientes.
  Esto se logra por el proceso de Gram-Schmidt
  (ver cualquier texto de álgebra lineal,
   como~%
     \cite{treil17:_linear_algeb_done_wrong}).

  Supongamos entonces un conjunto de vectores linealmente independientes
  \(\mathbf{x}_1, \mathbf{x}_2, \mathbf{x}_3, \dotsc\),
  y queremos construir un conjunto de vectores ortogonales
  \(\mathbf{y}_1, \mathbf{y}_2, \mathbf{y}_3, \dotsc\).
  La construcción se basa en la siguiente observación:
  sean \(\mathbf{x}_1\) y \(\mathbf{x}_2\) linealmente independientes,
  buscamos un par de vectores ortogonales \(\mathbf{y}_1\) y \(\mathbf{y}_2\)
  que abarca el espacio definido por \(\mathbf{x}_1\) y \(\mathbf{x}_2\).
  Vale decir,
  \(\mathbf{y}_1\) y \(\mathbf{y}_2\)
  se pueden describir como combinaciones lineales
  de \(\mathbf{x}_1\) y \(\mathbf{x}_2\).
  Podemos arbitrariamente elegir \(\mathbf{y}_1 = \mathbf{x}_1\),
  queda por hallar un posible \(\mathbf{y}_2\).
  Por ejemplo,
  buscamos expresar:
  \begin{equation*}
    \mathbf{y}_2
      = \mathbf{x}_2 + \alpha \mathbf{y}_1
  \end{equation*}
  tal que:
  \begin{equation*}
    \langle \mathbf{y}_1, \mathbf{y}_2 \rangle
      = 0
  \end{equation*}
  Vale decir:
  \begin{align*}
    \langle \mathbf{y}_1, \mathbf{y}_2 \rangle
      &= \langle \mathbf{y}_1, \mathbf{x}_2 + \alpha_2 \mathbf{y}_1 \rangle \\
      &= \langle \mathbf{y}_1, \mathbf{x}_2 \rangle
           + \langle \mathbf{y}_1, \alpha_2 \mathbf{y}_1 \rangle \\
      &= \langle \mathbf{y}_1, \mathbf{x}_2 \rangle
           + \alpha_2 \langle \mathbf{y}_1, \mathbf{y}_1 \rangle
  \end{align*}
  Igualando a cero,
  despejamos:
  \begin{equation*}
    \alpha_2
      = - \frac{\langle \mathbf{y}_1, \mathbf{x}_2 \rangle}
               {\langle \mathbf{y}_1, \mathbf{y}_1 \rangle}
  \end{equation*}
  Así:
  \begin{equation*}
    \mathbf{y}_2
      = \mathbf{x}_2
          - \frac{\langle \mathbf{y}_1, \mathbf{x}_2 \rangle}
                 {\langle \mathbf{y}_1, \mathbf{y}_1 \rangle} \mathbf{y}_1
  \end{equation*}
  Podemos usar esta misma técnica
  para eliminar los componentes a lo largo de los anteriores \(\mathbf{y}_k\)
  de los demás vectores.
  En resumen,
  calculamos sucesivamente para \(i = 1, 2, \dotsc\):
  \begin{equation*}
    \mathbf{y}_i
      = \mathbf{x}_i
          - \sum_{1 \le k < i}
              \frac{\langle \mathbf{y}_k, \mathbf{x}_i \rangle}
                   {\langle \mathbf{y}_k, \mathbf{y}_k \rangle} \mathbf{y}_k
  \end{equation*}

\bibliography{../referencias}

%%% Local Variables:
%%% mode: latex
%%% TeX-master: "../INF-221_notas"
%%% ispell-local-dictionary: "spanish"
%%% End:

\bibliographystyle{babplain-fl}

\chapter{Symbolic Method for Dummies}
\label{apx:symbolic-method-dummies}

  La idea básica es usar funciones generatrices en forma sistemática
  en combinatoria.
  Lo que sigue es un condensado del apunte de Fundamentos de Informática~%
    \cite[capítulo~21]{brand17:_fundamentos_informatica}.
  Ver también Lumbroso y~Morcrette~%
    \cite{lumbroso12:_gentle_intro_analy_combin}.
  Mucho más detalle dan Flajolet y Sedgewick~%
    \cite{flajolet09:_analy_combin},
  Sedgewick y Flajolet~%
    \cite{sedgewick13:_introd_anal_algor}
  se concentran en aplicaciones al análisis de algoritmos.

  La idea es tener una \emph{clase} de objetos,
  que anotaremos mediante letras caligráficas,
  como \(\mathscr{A}\).
  La clase \(\mathscr{A}\) consta de \emph{objetos},
  \(\alpha \in \mathscr{A}\).
  Para el objeto \(\alpha\) hay una noción de \emph{tamaño},
  que anotamos \(\lvert \alpha \rvert\)
  (un número natural,
   generalmente el número de \emph{átomos} que componen \(\alpha\)).
  Al número de objetos de tamaño \(n\) lo anotaremos \(a_n\).
  Usaremos \(\mathscr{A}_n\)
  para referirnos al conjunto de objetos de la clase \(\mathscr{A}\)
  de tamaño \(n\),
  con lo que \(a_n = \lvert \mathscr{A}_n \rvert\).
  Condición adicional es
  que el número de objetos de cada tamaño sea finito.

  A las funciones generatrices ordinaria y exponencial
  correspondientes
  les llamaremos \(A(z)\) y \(\widehat{A}(z)\),
  respectivamente:
  \begin{align*}
    A(z)
      &= \sum_{\alpha \in \mathscr{A}} z^{\lvert \alpha \rvert}
       = \sum_{n \ge 0} a_n z^n \\
    \widehat{A}(z)
      &= \sum_{\alpha \in \mathscr{A}}
           \frac{z^{\lvert \alpha \rvert}}{\lvert \alpha \rvert !}
       = \sum_{n \ge 0} a_n \, \frac{z^n}{n!}
  \end{align*}

  Nuestro siguiente objetivo es construir nuevas clases
  a partir de las que ya tenemos.
  Debe tenerse presente que como lo que nos interesa
  es contar el número de objetos de cada tamaño,
  basta construir objetos con distribución de tamaños adecuada
  (o sea,
   relacionados con lo que deseamos contar por una biyección).
  Comúnmente el tamaño de los objetos es el número de alguna clase de átomos
  que lo componen.
  Al combinar objetos para crear objetos mayores
  los tamaños simplemente se suman.

  Las clases más elementales son \(\varnothing\),
  la clase que no contiene objetos;
  \(\mathscr{E} = \{\epsilon\}\),
  la clase que contiene únicamente el objeto vacío \(\epsilon\)
  (de tamaño nulo);
  y la clase que comúnmente llamaremos \(\mathscr{Z}\),
  conteniendo un único objeto de tamaño uno
  (que llamaremos \(\zeta\) por consistencia).
  Luego definimos operaciones que combinan
  las clases \(\mathscr{A}\) y \(\mathscr{B}\)
  mediante \emph{unión combinatoria} \(\mathscr{A} + \mathscr{B}\),
  en que aparecen los \(\alpha\) y los \(\beta\) con sus tamaños
  (los objetos individuales se \textquote{decoran} con su proveniencia,
   de forma que \(\mathscr{A}\) y  \(\mathscr{B}\)
   no necesitan ser disjuntos;
   pero generalmente nos preocuparemos
   que \(\mathscr{A}\) y \(\mathscr{B}\)
   sean disjuntos,
   o podemos usar el principio de inclusión y exclusión
   para contar los conjuntos de interés).
  Ocasionalmente restaremos objetos de una clase,
  lo que debe interpretarse sin decoraciones
  (estamos dejando fuera ciertos elementos,
   simplemente).
  Usaremos \emph{producto cartesiano}
  \(\mathscr{A} \times \mathscr{B}\),
  cuyos elementos son pares \((\alpha, \beta)\)
  y el tamaño del par
  es \(\lvert \alpha \rvert + \lvert \beta \rvert\).
  Otras operaciones son formar \emph{secuencias}
  de elementos de \(\mathscr{A}\)
  (se anota \(\Seq(\mathscr{A})\)),
  formar \emph{conjuntos}
  \(\Set(\mathscr{A})\)
  y \emph{multiconjuntos}
  \(\MSet(\mathscr{A})\)
  de elementos de \(\mathscr{A}\).

  De incluir objetos de tamaño cero
  en estas construcciones pueden crearse infinitos objetos de un tamaño dado,
  lo que no es una clase según nuestra definición.
  Por ello estas construcciones son aplicables
  solo si \(\mathscr{A}_0 = \varnothing\).

  Es importante recalcar las relaciones y diferencias
  entre las estructuras.
  En una secuencia es central
  el orden de las piezas que la componen.
  Ejemplo son las palabras,
  interesa el orden exacto de las letras
  (y estas pueden repetirse).
  En un conjunto solo interesa si el elemento está presente o no,
  no hay orden.
  En un conjunto un elemento en particular está o no presente,
  a un multiconjunto puede pertenecer varias veces.

  Hay dos grandes opciones:
  Objetos rotulados y no rotulados.
  Consideramos que el objeto \(\alpha\) es \emph{rotulado}
  si sus átomos componentes tienen identidad,
  cosa que se representa rotulándolos de \num{1} a \(\lvert \alpha \rvert\).
  Si los átomos son libremente intercambiables,
  son objetos \emph{no rotulados}.
  Un punto que produce particular confusión
  es que tiene perfecto sentido
  hablar de secuencias de elementos sin rotular.
  La secuencia impone un orden,
  pero elementos iguales se consideran indistinguibles
  (en una palabra interesa el orden de las letras,
   pero al intercambiar dos letras iguales
   la palabra sigue siendo la misma).
  Estos dos casos requieren tratamiento separado,
  y en algunos casos operaciones especializadas.

\section{Objetos no rotulados}
\label{sec:objetos-no-rotulados}

  Nuestro primer teorema
  relaciona las funciones generatrices ordinarias
  respectivas para algunas de las operaciones entre clases
  definidas antes.
  Las funciones generatrices de las clases \(\varnothing\),
  \(\mathscr{E}\) y \(\mathscr{Z}\)
  son,
  respectivamente,
  \num{0}, \num{1} y \(z\).
  En las derivaciones
  de las transferencias de ecuaciones simbólicas
  a ecuaciones para las funciones generatrices
  lo que nos interesa es contar los objetos entre manos,
  recurriremos a biyecciones para ello en algunos de los casos.
  \begin{theorem}[Método simbólico, OGF]
    \label{theo:ms-OGF}
    Sean \(\mathscr{A}\) y \(\mathscr{B}\) clases de objetos,
    con funciones generatrices ordinarias
    respectivamente \(A(z)\) y \(B(z)\).
    Entonces tenemos
    las siguientes funciones generatrices ordinarias:
    \begin{enumerate}
    \item
      Para enumerar \(\mathscr{A} + \mathscr{B}\):
      \begin{equation*}
        A(z) + B(z)
      \end{equation*}
    \item
      Para enumerar \(\mathscr{A} \times \mathscr{B}\):
      \begin{equation*}
        A(z) \cdot B(z)
      \end{equation*}
    \item
      Para enumerar \(\Seq(\mathscr{A})\):
      \begin{equation*}
        \frac{1}{1 - A(z)}
      \end{equation*}
    \item
      Para enumerar \(\Set(\mathscr{A})\):
      \begin{equation*}
        \prod_{\alpha \in \mathscr{A}}
           \left( 1 + z^{\lvert \alpha \rvert} \right)
          = \prod_{n \ge 1} (1 + z^n)^{a_n}
          = \exp \left(
                   \sum_{k \ge 1} \frac{(-1)^{k + 1}}{k} \, A(z^k)
                 \right)
      \end{equation*}
    \item
      Para enumerar \(\MSet(\mathscr{A})\):
      \begin{equation*}
        \prod_{\alpha \in \mathscr{A}}
           \frac{1}{1 - z^{\lvert \alpha \rvert}}
          = \prod_{n \ge 1} \frac{1}{(1 - z^n)^{a_n}}
          = \exp\left(
                   \sum_{k \ge 1} \frac{A(z^k)}{k}
                \right)
      \end{equation*}
    \item
      Para enumerar \(\Cyc(\mathscr{A})\):
      \begin{equation*}
        \sum_{n \ge 1} \frac{\phi(n)}{n} \, \ln \frac{1}{1 - A(z^n)}
      \end{equation*}
    \end{enumerate}
  \end{theorem}
  \begin{proof}
    Usamos libremente resultados sobre funciones generatrices,
    ver~%
      \cite[capítulo~14]{brand17:_fundamentos_informatica},
    en las demostraciones de cada caso.
    Usaremos casos ya demostrados en las demostraciones sucesivas.
    \begin{enumerate}
    \item % A + B
      Si hay \(a_n\) elementos de \(\mathscr{A}\) de tamaño \(n\)
      y \(b_n\) elementos de \(\mathscr{B}\) de tamaño \(n\),
      habrán \(a_n + b_n\) elementos
      de \(\mathscr{A} + \mathscr{B}\)
      de tamaño \(n\).

      Alternativamente,
      usando la notación de Iverson
      (ver~%
        \cite[sección~1.4]{brand17:_fundamentos_informatica}):
      \begin{equation*}
        \sum_{\mathclap{\gamma \in \mathscr{A} + \mathscr{B}}}
          z^{\lvert \gamma \rvert}
          = \; \sum_{\mathclap{\gamma \in \mathscr{A}
                                            + \mathscr{B}}}
                 \left(
                   [\gamma \in \mathscr{A}]
                     z^{\lvert \gamma \rvert}
                      + [\gamma \in \mathscr{B}]
                          z^{\lvert \gamma \rvert}
                 \right)
          = \sum_{\alpha \in \mathscr{A}} z^{\lvert \alpha \rvert}
              + \sum_{\beta \in \mathscr{B}}
                  z^{\lvert \beta \rvert}
          = A(z) + B(z)
      \end{equation*}
    \item % A x B
      Hay:
      \begin{equation*}
        \sum_{0 \le k \le n} a_k b_{n - k}
      \end{equation*}
      maneras de combinar elementos de \(\mathscr{A}\)
      con elementos de \(\mathscr{B}\) cuyos tamaños sumen \(n\),
      y este es precisamente
      el coeficiente de \(z^n\) en \(A(z) \cdot B(z)\).

      Alternativamente:
      \begin{align*}
        \sum_{\mathclap{\gamma \in \mathscr{A} \times \mathscr{B}}}
             z^{\lvert \gamma \rvert}
          = \sum_{\mathclap{\substack{
                              \alpha \in \mathscr{A} \\
                              \beta \in \mathscr{B}
                 }}} z^{\lvert (\alpha, \beta) \rvert}
          = \sum_{\mathclap{\substack{
                              \alpha \in \mathscr{A} \\
                              \beta \in \mathscr{B}
                 }}} z^{\lvert \alpha \rvert + \lvert \beta \rvert}
          = \left(
               \sum_{\alpha \in \mathscr{A}}
                 z^{\lvert \alpha \rvert}
             \right)
               \cdot \left(
                        \sum_{\beta \in \mathscr{B}}
                          z^{\lvert \beta \rvert}
                     \right)
          = A(z) \cdot B(z)
      \end{align*}
    \item % Seq(A)
      Hay una manera de obtener la secuencia de largo 0
      (aporta el objeto vacío \(\epsilon\)),
      las secuencias de largo \num{1}
      son simplemente los elementos de \(\mathscr{A}\),
      las secuencias de largo \num{2}
      son elementos de \(\mathscr{A} \times \mathscr{A}\),
      y así sucesivamente.
      O sea,
      las secuencias se representan mediante:
      \begin{equation*}
        \mathscr{E}
          + \mathscr{A}
          + \mathscr{A} \times \mathscr{A}
          + \mathscr{A} \times \mathscr{A} \times \mathscr{A}
          + \dotsb
      \end{equation*}
      Por la segunda parte
      y la serie geométrica,
      la función generatriz correspondiente es:
      \begin{equation*}
        1 + A(z) + A^2(z) + A^3(z) + \dotsb
          = \frac{1}{1 - A(z)}
      \end{equation*}
    \item % Set(A)
      La clase de los subconjuntos finitos de \(\mathscr{A}\)
      queda representada por el producto simbólico:
      \begin{equation*}
        \prod_{\alpha \in \mathscr{A}} (\mathscr{E} + \{\alpha\})
      \end{equation*}
      ya que
      al distribuir los productos de todas las formas posibles
      aparecen todos los subconjuntos de \(\mathscr{A}\).
      Directamente obtenemos entonces:
      \begin{equation*}
        \prod_{\alpha \in \mathscr{A}}
            \left( 1 + z^{\lvert \alpha \rvert} \right)
          = \prod_{n \ge 0} (1 + z^n)^{a_n}
      \end{equation*}
      Otra forma de verlo es que cada objeto de tamaño \(n\)
      aporta un factor \(1 + z^n\),
      si hay \(a_n\) de estos
      el aporte total es \((1 + z^n)^{a_n}\).
      Esta es la primera parte de lo aseverado.
      Aplicando logaritmo:
      \begin{align*}
        \sum_{\alpha \in \mathscr{A}}
            \ln \left(1 + z^{\lvert \alpha \rvert} \right)
          &= -\sum_{\alpha \in \mathscr{A}}
                \sum_{k \ge 1}
                  \frac{(-1)^k z^{\lvert \alpha \rvert k}}{k}  \\
          &= -\sum_{k \ge 1} \frac{(-1)^k}{k} \,
                \sum_{\alpha \in \mathscr{A}}
                  z^{\lvert \alpha \rvert k} \\
          &= \sum_{k \ge 1} \frac{(-1)^{k + 1} \, A(z^k)}{k}
      \end{align*}
      Exponenciando lo último
      resulta equivalente a la segunda parte.
    \item % MSet(A)
      Podemos considerar un multiconjunto finito
      como la combinación de una secuencia
      para cada tipo de elemento:
      \begin{equation*}
        \prod_{\alpha \in \mathscr{A}} \Seq(\{ \alpha \})
      \end{equation*}
      La función generatriz buscada es:
      \begin{equation*}
        \prod_{\alpha \in \mathscr{A}}
          \frac{1}{1 - z^{\lvert \alpha \rvert}}
          = \prod_{n \ge 0} \frac{1}{(1 - z^n)^{a_n}}
      \end{equation*}
      Esto provee la primera parte.
      Nuevamente aplicamos logaritmo para simplificar:
      \begin{align*}
        \ln \prod_{\alpha \in \mathscr{A}}
               \frac{1}{1 - z^{\lvert \alpha \rvert}}
          &= - \sum_{\alpha \in \mathscr{A}}
                 \ln \left( 1 - z^{\lvert \alpha \rvert} \right) \\
          &= \sum_{\alpha \in \mathscr{A}}
               \sum_{k \ge 1}
                 \frac{z^{k \lvert \alpha \rvert}}{k} \\
          &= \sum_{k \ge 1}
               \frac{1}{k} \,
                 \sum_{\alpha \in \mathscr{A}}
                   z^{k \lvert \alpha \rvert} \\
          &= \sum_{k \ge 1}
               \frac{A(z^k)}{k}
      \end{align*}
    \item % Cyc(A)
      Esta situación es más compleja de tratar,
      vea la discusión en~%
        \cite[sección~21.2.3]{brand17:_fundamentos_informatica}.
      \qedhere
    \end{enumerate}
  \end{proof}

\subsection{Algunas aplicaciones}
\label{sec:ms-ogf-aplicaciones}

  La clase de los árboles binarios \(\mathscr{B}\)
  es por definición es la unión disjunta del árbol vacío
  y la clase de tuplas de un nodo (la raíz)
  y dos árboles binarios.
  O sea:
  \begin{equation*}
    \mathscr{B}
      = \mathscr{E}
          + \mathscr{Z} \times \mathscr{B} \times \mathscr{B}
  \end{equation*}
  de donde directamente obtenemos:
  \begin{equation*}
    B(z)
      = 1 + z B^2(z)
  \end{equation*}
  Con el cambio de variable \(u(z) = B(z) - 1\) queda:
  \begin{equation*}
    u(z)
      = z (1 + u(z))^2
  \end{equation*}
  Es aplicable la fórmula de inversión de Lagrange~%
    \cite[teorema~17.8]{brand17:_fundamentos_informatica}
  con \(\phi(u) = (u + 1)^2\) y \(f(u) = u\):
  \begin{align*}
    \left[ z^n \right] u(z)
      &= \frac{1}{n} \, \left[ u^{n - 1} \right] \, \phi(u)^n \\
      &= \frac{1}{n} \, \left[ u^{n - 1} \right] \, (u + 1)^{2 n} \\
      &= \frac{1}{n} \, \binom{2 n}{n - 1} \\
      &= \frac{1}{n + 1} \, \binom{2 n}{n}
  \end{align*}
  Tenemos,
  como \(u(z) = B(z) - 1\)
  (sabemos que \([z^0] u(z) = 0\))
  y de la condición inicial \(b_0 = 1\):
  \begin{equation*}
    b_n =
    \begin{cases}
      \displaystyle
        \frac{1}{n + 1} \, \binom{2 n}{n}
               & \text{si \(n \ge 1\)} \\
      1
               & \text{si \(n = 0\)}
    \end{cases}
  \end{equation*}
  Casualmente la expresión simplificada para \(n \ge 1\)
  da el valor correcto \(b_0 = 1\).
  Estos son los números de Catalan,
  es \(b_n = C_n\).

  Sea ahora \(\mathscr{A}\)
  la clase de \emph{árboles con raíz ordenados},
  formados por un nodo raíz
  conectado a las raíces de una secuencia de árboles ordenados.
  La idea es que la raíz tiene hijos en un cierto orden.
  Simbólicamente:
  \begin{equation*}
    \mathscr{A}
      = \mathscr{Z} \times \Seq(\mathscr{A})
  \end{equation*}
  El método simbólico entrega directamente la ecuación:
  \begin{equation*}
    A(z)
      = \frac{z}{1 - A(z)}
  \end{equation*}
  Nuevamente es aplicable la fórmula de inversión de Lagrange,
  con \(\phi(A) = (1 - A)^{-1}\) y \(f(A) = A\):
  \begin{align*}
    \left[ z^n \right] A(z)
      &= \frac{1}{n} \, \left[ A^{n - 1} \right] \, \phi(A)^n \\
      &= \frac{1}{n} \, \left[ A^{n - 1} \right] \, (1 - A)^{-n} \\
      &= \frac{1}{n} \, \binom{2 n - 2}{n - 1} \\
      &= C_{n - 1}
  \end{align*}
  Otra vez números de Catalan.

  La manera obvia de representar \(\mathbb{N}_0\)
  es por secuencias de marcas,
  como \(||||\) para 4;
  simbólicamente \(\mathbb{N}_0 = \Seq(\mathscr{Z})\).
  Para calcular el número de multiconjuntos de \(k\) elementos
  tomados entre \(n\),
  un multiconjunto queda representado
  por las cuentas de cada uno los \(n\) elementos de que se compone,
  y eso corresponde a:
  \begin{equation*}
    \mathbb{N}_0 \times \dotsb \times \mathbb{N}_0
      = (\Seq(\mathscr{Z}))^n
  \end{equation*}
  Para obtener el número que nos interesa:
  \begin{align*}
    \multiset{n}{k}
      &= \left[ z^k \right] (1 - z)^{-n} \\
      &= (-1)^n \binom{-n}{k} \\
      &= \binom{n + k - 1}{n}
  \end{align*}
  Una \emph{combinación} de \(n\) es expresarlo como una suma.
  Por ejemplo,
  hay \num{8} combinaciones de \num{4}:
  \begin{equation*}
      4
        = 3 + 1
        = 2 + 2
        = 2 + 1 + 1
        = 1 + 3
        = 1 + 2 + 1
        = 1 + 1 + 2
        = 1 + 1 + 1 + 1
  \end{equation*}
  Sea \(c(n)\) el número de combinaciones de \(n\).
  Como antes,
  tenemos \(\mathbb{N} = \mathscr{Z} \times \Seq(\mathscr{Z})\),
  que da:
  \begin{equation*}
    N(z)
      = \frac{z}{1 - z}
  \end{equation*}
  A su vez,
  una combinación no es más que una secuencia de naturales
  (separados por \(+\)):
  \begin{equation*}
    \mathscr{C}
      = \Seq(\mathbb{N})
  \end{equation*}
  Directamente resulta:
  \begin{align*}
    C(z)
      &= \sum_{n \ge 0} c(n) z^n \\
      &= \frac{1}{1 - N(z)} \\
      &= \frac{1}{2} + \frac{1}{2} \cdot \frac{1}{1 - 2 z} \\
    c(n)
      &= [z^n] C(z) \\
      &= \frac{1}{2} \, [n = 0] + \frac{1}{2} \cdot 2^n \\
      &= \frac{1}{2} \, [n = 0] + 2^{n - 1}
  \end{align*}
  Esto es consistente con \(c(4) = 8\) obtenido arriba.

\section{Objetos rotulados}
\label{sec:objetos-rotulados}

  En la discusión previa solo interesaba el tamaño de los objetos,
  no su disposición particular.
  Consideraremos ahora objetos rotulados,
  donde importa cómo se compone el objeto de sus partes
  (los átomos tienen identidades,
   posiblemente porque se ubican en orden).

  El objeto más simple con partes rotuladas son las permutaciones
  (biyecciones \(\sigma \colon [n] \rightarrow [n]\),
   podemos considerarlas secuencias de átomos numerados).
  Para la función generatriz exponencial tenemos,
  ya que hay \(n!\) permutaciones de \(n\) elementos:
  \begin{equation*}
    \sum_{\sigma}
        \frac{z^{\lvert \sigma \rvert}}{\lvert \sigma \rvert !}
      = \sum_{n \ge 0} n! \, \frac{z^n}{n!}
      = \frac{1}{1 - z}
  \end{equation*}

  Lo siguiente más simple de considerar
  es colecciones de ciclos rotulados.
  Por ejemplo,
  escribimos \((1\;3\;2)\) para el objeto
  en que viene \num{3} luego de \num{1},
  \num{2} sigue a \num{3},
  y a su vez \num{1} sigue a \num{2}.
  Así \((2\;1\;3)\) es solo otra forma de anotar el ciclo anterior,
  que no es lo mismo que \((3\;1\;2)\).
  Interesa definir formas consistentes
  de combinar objetos rotulados.
  Por ejemplo,
  al combinar el ciclo \((1\;2)\) con el ciclo \((1\;3\;2)\)
  resultará un objeto con \num{5} rótulos,
  y debemos ver cómo los distribuimos entre las partes.
  El cuadro~\ref{tab:ciclo+ciclo}
  reseña las posibilidades al respetar
  el orden de los elementos asignados a cada parte.
  \begin{table}[htbp]
    \centering
    \begin{tabular}{*{4}{>{\(}l<{\)}}}
      (1\;2) (3\;5\;4) & (2\;3) (1\;5\;4) & (3\;4) (1\;5\;2) &
          (4\;5) (1\;3\;2) \\
      (1\;3) (2\;5\;4) & (2\;4) (1\;5\;3) & (3\;5) (1\;4\;2) \\
      (1\;4) (2\;5\;3) & (2\;5) (1\;4\;3) \\
      (1\;5) (2\;4\;3)
    \end{tabular}
    \caption{Combinando los ciclos \((1\;2)\) y \((1\;3\;2)\)}
    \label{tab:ciclo+ciclo}
  \end{table}
  Es claro que lo que estamos haciendo es elegir
  un subconjunto de \num{2} rótulos
  de entre los \num{5} para asignárselos al primer ciclo.
  El combinar
  dos clases de objetos \(\mathscr{A}\) y \(\mathscr{B}\)
  de esta forma lo anotaremos \(\mathscr{A} \star \mathscr{B}\).

  Otra operación común es la \emph{composición},
  anotada \(\mathscr{A} \circ \mathscr{B}\).
  La idea es elegir un elemento \(\alpha \in \mathscr{A}\),
  luego elegir \(\lvert \alpha \rvert\) elementos
  de \(\mathscr{B}\),
  y reemplazar los \(\mathscr{B}\) por las partes de \(\alpha\),
  en el orden que están rotuladas;
  para finalmente asignar rótulos a los átomos
  que conforman la.estructura completa
  respetando el orden de los rótulos
  al interior de los \(\mathscr{B}\)
  (igual como lo hicimos para \(\star\)).

  Ocasionalmente es útil \emph{marcar}
  uno de los componentes del objeto,
  operación que anotaremos \(\mathscr{A}^\bullet\).
  Usaremos también la construcción \(\MSet(\mathscr{A})\),
  que podemos considerar como una secuencia de elementos numerados
  obviando el orden.
  Cuidado,
  muchos textos le llaman \(\Set()\) a esta operación.

  Tenemos el siguiente teorema:
  \begin{theorem}[Método simbólico, EGF]
    \label{theo:ms-EGF}
    Sean \(\mathscr{A}\) y \(\mathscr{B}\) clases de objetos rotulados,
    con funciones generatrices exponenciales
    \(\widehat{A}(z)\) y \(\widehat{B}(z)\),
    respectivamente.
    Entonces tenemos
    las siguientes funciones generatrices exponenciales:
    \begin{enumerate}
    \item
      Para enumerar \(\mathscr{A}^\bullet\):
      \begin{equation*}
        z \mathrm{D} \widehat{A}(z)
      \end{equation*}
    \item
      Para enumerar \(\mathscr{A} + \mathscr{B}\):
      \begin{equation*}
        \widehat{A}(z) + \widehat{B}(z)
      \end{equation*}
    \item
      Para enumerar \(\mathscr{A} \star \mathscr{B}\):
      \begin{equation*}
        \widehat{A}(z) \cdot \widehat{B}(z)
      \end{equation*}
    \item
      Para enumerar \(\mathscr{A} \circ \mathscr{B}\):
      \begin{equation*}
        \widehat{A}(\widehat{B}(z))
      \end{equation*}
    \item
      Para enumerar \(\Seq(\mathscr{A})\):
      \begin{equation*}
        \frac{1}{1 - \widehat{A}(z)}
      \end{equation*}
    \item
      Para enumerar \(\MSet(\mathscr{A})\):
      \begin{equation*}
        \mathrm{e}^{\widehat{A}(z)}
      \end{equation*}
    \item
      Para enumerar \(\Cyc(\mathscr{A})\):
      \begin{equation*}
        -\ln(1 - \widehat{A}(z))
      \end{equation*}
    \end{enumerate}
  \end{theorem}
  \begin{proof}
    Usaremos casos ya demostrados en las demostraciones sucesivas.
    \begin{enumerate}
    \item % mark A
      El objeto \(\alpha \in \mathscr{A}\)
      da lugar a \(\lvert \alpha \rvert\) objetos
      al marcar cada uno de sus átomos,
      lo que da la función generatriz exponencial:
      \begin{equation*}
        \sum_{\alpha \in \mathscr{A}}
          \lvert \alpha \rvert
            \frac{z^{\lvert \alpha \rvert}}{\lvert \alpha \rvert !}
      \end{equation*}
      Esto es lo indicado.
    \item % A + B
      Nuevamente trivial.
    \item % A x B
      El número de objetos \(\gamma\) que se obtienen
      al combinar \(\alpha \in \mathscr{A}\)
      con \(\beta \in \mathscr{B}\) es:
      \begin{equation*}
        \binom{\lvert \alpha \rvert + \lvert \beta \rvert}
              {\lvert \alpha \rvert}
      \end{equation*}
      y tenemos la función generatriz exponencial:
      \begin{equation*}
        \sum_{\gamma \in \mathscr{A} \star \mathscr{B}}
            \frac{z^{\lvert \gamma \rvert}}{\lvert \gamma \rvert !}
          = \sum_{\substack{
                     \alpha \in \mathscr{A} \\
                     \beta \in \mathscr{B}
                 }}
               \binom{\lvert \alpha \rvert + \lvert \beta \rvert}
                     {\lvert \alpha \rvert}
                  \frac{z^{\lvert \alpha \rvert
                            + \lvert \beta \rvert}}
                       {(\lvert \alpha \rvert
                            + \lvert \beta \rvert)!}
          = \left(
              \sum_{\alpha \in \mathscr{A}}
                \frac{z^{\lvert \alpha \rvert}}
                     {\lvert \alpha \rvert !}
            \right)
              \cdot \left(
                \sum_{\beta \in \mathscr{B}}
                  \frac{z^{\lvert \beta \rvert}}
                       {\lvert \beta \rvert !}
                    \right)
          = \widehat{A}(z) \cdot \widehat{B}(z)
      \end{equation*}
    \item % A circ B
      Tomemos \(\alpha \in \mathscr{A}\),
      de tamaño \(n = \lvert \alpha \rvert\),
      y \(n\) elementos de \(\mathscr{B}\) en orden
      a ser reemplazados por las partes de \(\alpha\).
      Esa secuencia de \(\mathscr{B}\) es representada por:
      \begin{equation*}
        \mathscr{B}
          \star \mathscr{B}
          \star \dotsb
          \star \mathscr{B}
      \end{equation*}
      con función generatriz exponencial:
      \begin{equation*}
        \widehat{B}^n (z)
      \end{equation*}
      Sumando sobre las contribuciones:
      \begin{equation*}
        \sum_{\alpha \in \mathscr{A}}
           \frac{\widehat{B}^{\lvert \alpha \rvert}(z)}
                {\lvert \alpha \rvert \, !}
      \end{equation*}
      Esto es lo prometido.
    \item % Seq(A)
      Primeramente,
      podemos describir:
      \begin{equation*}
        \Seq(\mathscr{Z})
          = \mathscr{E} + \mathscr{Z} \star \Seq(\mathscr{Z})
      \end{equation*}
      que lleva a ecuación:
      \begin{equation*}
        \widehat{S}(z)
          = 1 + z \widehat{S}(z)
      \end{equation*}
      de donde:
      \begin{equation*}
        \widehat{S}(z)
          = \frac{1}{1 - z}
      \end{equation*}
      Aplicando composición se obtiene lo indicado.
    \item % MSet(A)
      Hay un único multiconjunto de \(n\) elementos rotulados
      (se rotulan simplemente de 1 a  \(n\)),
      con lo que \(\MSet(\mathscr{Z})\) corresponde a:
      \begin{equation*}
        \sum_{n \ge 0} \frac{z^n}{n!}
          = \exp(z)
      \end{equation*}
      Al aplicar composición resulta lo anunciado.

      Otra demostración es considerar el multiconjunto de \(\mathscr{A}\),
      descrito por \(\mathscr{M} = \MSet(\mathscr{A})\).
      Si marcamos uno de los átomos de \(\mathscr{M}\)
      estamos marcando uno de los \(\mathscr{A}\),
      el resto sigue formando un multiconjunto de \(\mathscr{A}\):
      \begin{equation*}
        \mathscr{M}^\bullet
          = \mathscr{A}^\bullet \star \mathscr{M}
      \end{equation*}
      Por lo anterior:
      \begin{equation*}
        z \widehat{M}'(z)
          = z \widehat{A}'(z) \widehat{M}(z)
      \end{equation*}
      Hay un único multiconjunto de tamaño \num{0},
      o sea \(\widehat{M}(0) = 1\);
      y hemos impuesto la condición
      que no hay objetos de tamaño \num{0} en \(\mathscr{A}\),
      vale decir,
      \(\widehat{A}(0) = 0\).
      Así la solución a la ecuación diferencial es:
      \begin{equation*}
        \widehat{M}(z)
          = \exp(\widehat{A}(z))
      \end{equation*}
    \item % Cyc(A)
      Consideremos un ciclo de \(\mathscr{A}\),
      o sea \(\mathscr{C} = \Cyc(\mathscr{A})\).
      Si marcamos los \(\mathscr{C}\),
      estamos marcando uno de los \(\mathscr{A}\),
      y el resto es una secuencia:
      \begin{equation*}
        \mathscr{C}^\bullet
          = \mathscr{A}^\bullet \star \Seq(\mathscr{A})
      \end{equation*}
      Esto se traduce en la ecuación diferencial:
      \begin{equation*}
        z \widehat{C}'(z)
          = z \widehat{A}'(z) \frac{1}{1 - \widehat{A}(z)}
      \end{equation*}
      Integrando bajo el entendido \(\widehat{C}(0) = 0\)
      con \(\widehat{A}(0) = 0\)
      se obtiene lo indicado.
    \end{enumerate}
  \end{proof}

\subsection{Algunas aplicaciones}
\label{sec:ms-egf-aplicaciones}

  Un ejemplo simple es el caso de permutaciones,
  que son simplemente secuencias de elementos rotulados:
  \begin{align*}
    \mathscr{P}
      &= \Seq(\mathscr{Z}) \\
    \widehat{P}(z)
      &= \frac{1}{1 - z} \\
    P_n
      &= n! [z^n] \widehat{P}(z) \\
      &= n!
  \end{align*}

  Consideremos colecciones de ciclos:
  \begin{equation*}
    \MSet(\Cyc(\mathscr{Z}))
  \end{equation*}
  Vemos que esto corresponde a:
  \begin{equation*}
    \exp\left( \ln \frac{1}{1 - z} \right)
      = \frac{1}{1 - z}
  \end{equation*}
  Hay tantas permutaciones de tamaño \(n\) como colecciones de ciclos.
  Una biyección se da ordenando los ciclos
  de manera que se inicien con su mayor elemento,
  y listar los ciclos en orden de máximo elemento creciente.
  Cualquier lista de elementos puede reinterpretarse como ciclos
  de una única manera de esta forma.
  En el fondo,
  podemos representar permutaciones
  como los ciclos ordenados para comenzar con sus elementos máximos.

  Podemos describir permutaciones
  como un conjunto de elementos que quedan fijos
  combinado con otros elementos que están fuera de orden
  (un \emph{desarreglo},
   clase \(\mathscr{D}\)).
  O sea:
  \begin{align*}
    \mathscr{P}
      &= \MSet(\mathscr{Z}) \star \mathscr{D} \\
    \frac{1}{1 - z}
      &= \mathrm{e}^z \cdot \widehat{D}(z) \\
    \widehat{D}(z)
      &= \frac{\mathrm{e}^{-z}}{1 - z}
  \end{align*}
  De acá,
  por propiedades de las funciones generatrices vemos que:
  \begin{equation*}
    [z^n] \widehat{D}(z)
      = \sum_{0 \le k \le n} \frac{(-1)^k}{k!}
  \end{equation*}
  y tenemos,
  usando la notación común para el número de desarreglos de tamaño \(n\):
  \begin{equation*}
    D_n
      = n! \sum_{0 \le k \le n} \frac{(-1)^k}{k!}
  \end{equation*}

  Una \emph{involución} es una permutación \(\pi\)
  tal que \(\pi \circ \pi\) es la identidad.
  Es claro que una involución
  es una colección de ciclos de largos \num{1} y \num{2},
  o sea:
  \begin{equation*}
    \mathscr{I}
      = \MSet(\Cyc_{\le 2}(\mathscr{Z}))
  \end{equation*}
  Revisando la derivación,
  vemos que \(\Cyc_{\le 2}(\mathscr{Z})\) corresponde a:
  \begin{equation*}
    \frac{z}{1} + \frac{z^2}{2}
  \end{equation*}
  y tenemos para la función generatriz:
  \begin{equation*}
    \widehat{I}(z)
      = \exp\left( \frac{z}{1} + \frac{z^2}{2} \right)
  \end{equation*}
  Un paquete de álgebra simbólica da:
  \begin{equation*}
    \widehat{I}(z)
      = 1 +    1 \cdot \frac{z}{1!}
          +    2 \cdot \frac{z^2}{2!}
          +    4 \cdot \frac{z^3}{3!}
          +   10 \cdot \frac{z^4}{4!}
          +   26 \cdot \frac{z^5}{5!}
          +   76 \cdot \frac{z^6}{6!}
          +  232 \cdot \frac{z^7}{7!}
          +  764 \cdot \frac{z^8}{8!}
          + 2620 \cdot \frac{z^9}{9!}
%	  + 9496 \cdot \frac{z^{10}}{10!}
          + \dotsm
  \end{equation*}

  Un \emph{desarreglo} es una permutación sin puntos fijos,
  vale decir,
  sin ciclos de largo \num{1}.
  O sea:
  \begin{equation*}
    \mathscr{D}
      = \MSet(Cyc_{> 1}(\mathscr{Z}))
  \end{equation*}
  Revisando las derivaciones para las operaciones,
  esto corresponde a:
  \begin{align*}
    \widehat{D}(z)
      &= \exp \left( \sum_{k \ge 2} \frac{z^k}{k} \right) \\
      &= \exp \left( \sum_{k \ge 1} \frac{z^k}{k} - z \right) \\
      &= \exp \left( \ln \frac{1}{1 - z} - z \right) \\
      &= \frac{1}{1 - z} \cdot e^{-z}
  \end{align*}
  Igual que obtuvimos antes.

\section{Funciones generatrices cumulativas}
\label{sec:gf-cumulativa}

  Para precisar,
  consideremos una clase de objetos \(\mathscr{A}\).
  Como siempre el número de objetos de tamaño \(n\)
  lo anotaremos \(a_n\),
  con función generatriz:
  \begin{align}
    A(z)
      &= \sum_{\alpha \in \mathscr{A}} z^{\lvert \alpha \rvert}
            \label{eq:A-def} \\
      &= \sum_{n \ge 0} a_n z^n
            \label{eq:A-an}
  \end{align}
  Consideremos no solo el número de objetos,
  sino alguna característica,
  cuyo valor para el objeto \(\alpha\) anotaremos \(\chi(\alpha)\).
  Es natural definir la \emph{función generatriz cumulativa}:
  \begin{equation}
    \label{eq:cogf-def}
    C(z)
      = \sum_{\alpha \in \mathscr{A}} \chi(\alpha) z^{\lvert \alpha \rvert}
  \end{equation}
  Vale decir,
  los coeficientes son la suma de la medida \(\chi\)
  para un tamaño dado:
  \begin{equation}
    \label{eq:cogf-coefficient}
    [z^n] C(z)
      = \sum_{\lvert \alpha \rvert = n} \chi(\alpha)
  \end{equation}
  Así tenemos el valor promedio para objetos de tamaño \(n\):
  \begin{equation}
    \label{eq:chi-expected}
    \Exp_n[\chi]
      = \frac{[z^n] C(z)}{[z^n] A(z)}
  \end{equation}

  La discusión precedente es aplicable
  si tenemos objetos no rotulados entre manos.
  Si corresponden objetos rotulados,
  podemos definir las respectivas funciones generatrices exponenciales:
  \begin{align}
    \widehat{A}(z)
      &= \sum_{\alpha \in \mathscr{A}}
           \frac{z^{\lvert \alpha \rvert}}{\lvert \alpha \rvert !}
                \label{eq:Ahat-def} \\
      &= \sum_{n \ge 0} a_n \frac{z^n}{n!}
                \label{eq:Ahat-an} \\
    \widehat{C}(z)
      &= \sum_{\alpha \in \mathscr{A}}
           \chi(\alpha) \frac{z^{\lvert \alpha \rvert}}{\lvert \alpha \rvert !}
                \label{eq:Chat-def}
  \end{align}
  Nuevamente,
  como los factoriales en los coeficientes se cancelan:
  \begin{equation}
    \label{eq:chi-expected-hat}
    \Exp_n[\chi]
      = \frac{[z^n] \widehat{C}(z)}{[z^n] \widehat{A}(z)}
  \end{equation}

\section{Funciones generatrices bivariadas}
\label{sec:gf-bivariada}

  Consideremos una clase \(\mathscr{A}\),
  con objetos \(\alpha \in \mathscr{A}\) de tamaño \(\lvert \alpha \rvert\);
  y a su vez un parámetro,
  cuyo valor para \(\alpha\) es \(\chi(\alpha)\).
  Si los átomos que componen \(\alpha\) son indistinguibles,
  es natural definir la función generatriz bivariada ordinaria:
  \begin{equation}
    \label{eq:obgf-def}
    A(z, u)
      = \sum_{\alpha \in \mathscr{A}} z^{\lvert \alpha \rvert} u^{\chi(\alpha)}
  \end{equation}
  De la misma forma,
  si los átomos son distinguibles
  es apropiada la función generatriz exponencial:
  \begin{equation}
    \label{eq:ebgf-def}
    \widehat{A}(z, u)
      = \sum_{\alpha \in \mathscr{A}}
          \frac{z^{\lvert \alpha \rvert}}{\lvert \alpha \rvert !}
             u^{\chi(\alpha)}
  \end{equation}
  Es común que nos interese el valor promedio de \(\chi(\alpha)\)
  para objetos de tamaño dado.
  Note que:
  \begin{equation}
    \label{eq:obgf-partial}
    \frac{\partial A}{\partial u}
      = \sum_{\alpha \in \mathscr{A}}
          \chi(\alpha) u^{\chi(\alpha) - 1} z^{\lvert \alpha \rvert}
  \end{equation}
  Así podemos calcular los valores promedios a partir de
  los coeficientes de las siguientes sumas:
  \begin{align}
    \sum_{\alpha \in \mathscr{A}} z^{\lvert \alpha \rvert}
      &= A(z, 1)
          \label{eq:obgf|u=1} \\
    \sum_{\alpha \in \mathscr{A}} \chi(\alpha) z^{\lvert \alpha \rvert}
      &= \left. \frac{\partial A}{\partial u} \right\rvert_{u = 1}
          \label{eq:obgf_u|u=1}
  \end{align}
  Vemos que~\eqref{eq:obgf|u=1}
  no es más que la función generatriz del número de objetos,
  mientras~\eqref{eq:obgf_u|u=1}
  es la función generatriz cumulativa.

\bibliography{../referencias}

%%% Local Variables:
%%% mode: latex
%%% TeX-master: "../INF-221_notas"
%%% ispell-local-dictionary: "spanish"
%%% End:

% LocalWords:  Symbolic Method for Dummies proveniencia multiconjunto
% LocalWords:  multiconjuntos biyecciones Exponenciando Cyc bivariada
% LocalWords:  reinterpretarse cumulativas cumulativa bivariadas eq
% LocalWords:  obgf

\bibliographystyle{babplain-fl}

\chapter{Una pizca de probabilidades}
\label{apx:pizca-probabilidades}

  Requeriremos una pizca de probabilidades discretas,
  ofrecemos un rápido repaso de los resultados principales
  con sus derivaciones.
  Incluimos un par de resultados simples que generalmente no se tratan
  en ramos de probabilidades,
  pero que son de utilidad para análisis de algoritmos.
  La discusión general se adapta de Ash~%
    \cite{ash08:_basic_probab_theo}.
  Para mucho más detalle,
  en particular de los temas relevantes a algoritmos aleatorizados
  vea por ejemplo Aspnes~%
    \cite{aspnes20:_notes_randomized_algorithms}.

\section{Definiciones básicas}
\label{sec:definiciones-probabilidades}

  El origen de la teoría de probabilidades viene de juegos de azar.
  Por ejemplo,
  se vio que al lanzar una moneda muchas veces,
  muy aproximadamente la mitad de las veces sale cara.
  De la misma forma,
  si se toma un mazo de cartas inglés
  (sin comodines),
  se baraja y se extrae una carta,
  y se repite el ejercicio muchas veces,
  un cuarto de las veces la carta es trébol,
  y una en trece es un as.

  En el experimento con cartas tenemos \num{52} posibles resultados,
  y el \textquote{principio de razón insuficiente} o \textquote{mínima sorpresa}
  nos dice que esperamos cualquiera de los resultados,
  sin preferencias.
  Los pioneros del área definieron probabilidad
  como el número de casos favorables dividido por el número total de casos,
  con la justificación de que eran todos igualmente probables.
  En el caso de la pinta de la carta,
   \(13 / 52\) o \(1 / 4\).
  Esta definición es restrictiva
  (supone un número finito de posibilidades),
  pero,
  mucho más grave,
  es circular:
  estamos definiendo \textquote{probabilidad}
  en términos de \textquote{igualmente probable}.
  Necesitamos una base más sólida.

\subsection{Formalizando probabilidades}
\label{sec:formalizar-probabilidad}

  El primer ingrediente en una teoría matemática de las probabilidades
  es el \emph{espacio muestral},
  que anotaremos \(\Omega\),
  el conjunto de posibles resultados de un experimento al azar.
  El requisito esencial es que una ejecución del experimento
  entrega exactamente un resultado de \(\Omega\).
  El segundo ingrediente es el \emph{evento},
  una pregunta sobre el resultado de un experimento
  que tiene respuesta \textquote{si} o \textquote{no}
  (por ejemplo,
   si la carta elegida es una figura).
  Un evento no es más que un subconjunto del espacio muestral.
  Por convención,
  los eventos se anotan con letras romanas mayúsculas
  del comienzo del alfabeto \(A, B\), y así sucesivamente.
  Los resultados en que se cumple el evento \(A\)
  se llaman \emph{favorables} (para \(A\)).
  El evento \(\Omega\) se dice \emph{seguro}
  (siempre ocurre),
  el evento \(\varnothing\) se dice \emph{imposible}
  (jamás ocurre).
  Siendo conjuntos los eventos,
  se aplican las operaciones conocidas de conjuntos.

  Buscamos asignar probabilidades a eventos.
  Aparecen problemas técnicos,
  no siempre es posible asignar probabilidades en forma consistente
  a subconjuntos de \(\Omega\).
  Requeriremos que la clase de eventos \(\mathscr{F}\)
  forme lo que se conoce como un \emph{campo sigma},
  o sea,
  cumple los siguientes tres requisitos:
  \begin{align}
    &\Omega
       \in \mathscr{F}
           \label{eq:campo-sigma-Omega} \\
    &A_1, A_2, \dotsc
       \in \mathscr{F}
      \text{\ implica\ } \bigcup_n A_n \in \mathscr{F}
           \label{eq:campo-sigma-union} \\
    &A \in \mathscr{F}
      \text{\ implica\ } \overline{A} \in \mathscr{F}
           \label{eq:campo-sigma-complemento}
  \end{align}
  O sea,
  es cerrado respecto de unión finita
  o infinita numerable~\eqref{eq:campo-sigma-union}
  y complemento~\eqref{eq:campo-sigma-complemento}.
  Por~\eqref{eq:campo-sigma-Omega} y~\eqref{eq:campo-sigma-complemento}
  concluimos que \(\varnothing \in \mathscr{F}\).
  Por de~Morgan,
  de~\eqref{eq:campo-sigma-union} y~\eqref{eq:campo-sigma-complemento}
  también es cerrado respecto de intersección finita o infinita numerable.
  O sea,
  si la pregunta \textquote{¿Ocurrió \(A_i\)?}
  tiene respuesta definitiva para \(i = 1, 2, \dotsc\)
  también tiene respuesta definitiva
  \textquote{¿Ocurrió alguno de los \(A_i\)?}
  al igual que \textquote{¿Ocurrieron todos los \(A_i\)?}

  En muchos casos podremos tomar \(\mathscr{F}\)
  como el conjunto de todos los subconjuntos de \(\Omega\),
  situaciones en las cuales se requieren las sutilezas indicadas
  se dan cuando \(\Omega\) son conjuntos no numerables,
  como \(\mathbb{R}\).

  Estamos en condiciones de definir probabilidades de eventos.
  Ponemos los siguientes requisitos:
  \begin{align}
    &0 \le \Pr[A] \le 1
         \label{eq:Pr-0-1} \\
    &\Pr[\Omega] = 1
         \label{eq:Pr-1} \\
    &\Pr[A \cup B]
       = \Pr[A] + \Pr[B]
      && \text{si \(A \cap B = \varnothing\)}
         \label{eq:Pr-union} \\
    &\Pr\left[ \bigcup_i A_i \right]
       = \sum_i \Pr[A_i]
      && \text{para colecciones contables de \(A_i\) disjuntos}
         \label{eq:Pr-union-contable} \\
  \end{align}
  \begin{definition}
    Un \emph{espacio de probabilidades} es un trío
    \((\Omega, \mathscr{F}, \Pr)\),
    con \(\Omega\) es un conjunto,
    \(\mathscr{F}\) es un campo sigma de subconjuntos de \(\Omega\),
    y \(\Pr\) es una medida de probabilidad en \(\mathscr{F}\).
  \end{definition}
  Algunas consecuencias simples son:
  \begin{enumerate}
  \item
    \(\Pr[\varnothing] = 0\)

    Como \(A \cup \varnothing = A\) y \(A \cap \varnothing = \varnothing\),
    tenemos \(\Pr[A] = \Pr[A \cup \varnothing] = \Pr[A] + \Pr[\varnothing]\),
    y concluimos \(\Pr[\varnothing] = 0\).
  \item
    \(\Pr[A \cup B] = \Pr[A] + \Pr[B] - \Pr[A \cap B]\)

    Este es simplemente un caso especial
    del principio de inclusión y exclusión.
    En particular,
    como \(\Pr[A \cap B] \ge 0\),
    tenemos la \emph{cota de unión} \(\Pr[A \cup B] \le \Pr[A] + \Pr[B]\)
  \item
    Si \(A \subseteq B\),
    entonces \(\Pr[A] \le \Pr[B]\),
    específicamente
      \(\Pr[B \smallsetminus A] = \Pr[B] - \Pr[A]\).
  \item
    Para una colección numerable de eventos \(A_i\)
    se cumple
    \(\Pr\left[ \bigcup_i A_i \right] \le \sum_i \Pr[A_i]\)
  \end{enumerate}
  Las demostraciones de estas aseveraciones quedan de ejercicios.

  Se dice que los eventos \(A\) y \(B\) son \emph{mutuamente exclusivos}
  si \(A \cap B = \varnothing\)
  (no pueden ocurrir ambos a la vez).
  El evento \(\varnothing\) se dice \emph{imposible}
  (jamás ocurre).

\subsection{Probabilidades condicionales}
\label{sec:probabilidades-condicionales}

  Dados dos eventos \(A\) y \(B\)
  con \(\Pr[B] > 0\),
  la \emph{probabilidad condicional} de \(A\) dado \(B\) se define como:
  \begin{equation}
    \label{eq:probabilidad-condicional}
    \Pr[A \mid B]
      = \frac{\Pr[A \cap B]}{\Pr[B]}
  \end{equation}
  Estamos limitando nuestro universo al evento \(B\).

\subsection{Variables aleatorias}
\label{sec:variable-aleatoria}

  Una \emph{variable aleatoria} es el resultado de un experimento.
  Generalmente nos interesará el caso de variables aleatorias numéricas.
  Usaremos letras mayúsculas hacia el final del alfabeto
  (\(X, Y, \dotsc\))
  para representar variables,
  y las correspondientes letras minúsculas para sus valores.
  Los eventos en tales casos pueden describirse como por ejemplo \(X = a\)
  o \(a \le X \le b\),
  para valores \(a, b\) dados.
  Para ellas podemos definir el \emph{valor esperado} de la variable \(X\)
  como:
  \begin{equation}
    \label{eq:E}
    \Exp[X]
      = \sum_x x \Pr[X = x]
  \end{equation}
  y una medida importante de la dispersión es la \emph{varianza}:
  \begin{equation}
    \label{eq:var}
    \var[X]
      = \Exp[(X - \Exp[X])^2]
  \end{equation}
  Comúnmente se usan las notaciones \(\mu = \Exp[X]\)
  y \(\sigma^2 = \var[X]\).
  En el caso que las variables no tomen valores discretos
  (como acá),
  en vez de sumas aparecen integrales.

  Si se supone que los distintos resultados son igualmente probables
  se habla de \emph{distribución uniforme}.
  Así se habla comúnmente de \emph{elegir uniformemente al azar}
  para indicar que hay un conjunto de posibles resultados,
  de los que se elige uno al azar
  siendo igualmente probable que el elemento elegido
  sea cualquiera del conjunto.

  Un resultado extremadamente importante
  es:
  \begin{theorem}[Linearidad del valor esperado]
    \label{theo:E-lineal}
    Sean \(X_1\), \(X_2\) variables aleatorias,
    \(\alpha\) y \(\beta\) constantes arbitrarias.
    Entonces:
    \begin{align*}
      \Exp[\alpha X_1 + \beta X_2]
        = \alpha \Exp[X_1] + \beta \Exp[X_2]
    \end{align*}
  \end{theorem}
  \begin{proof}
    En el caso de variables discretas tenemos:
    \begin{align*}
      \Exp[ \alpha X_1 + \beta X_2 ]
        &= \sum_x
             (\alpha x \Pr[X_1 = x] + \beta x \Pr[X_2 = x]) \\
        &= \alpha \sum_x x \Pr[X_1 = x]
             + \beta \sum_x x \Pr[X_2 = x] \\
        &= \alpha \Exp[X_1] + \beta \Exp[X_2]
    \end{align*}
    Las sumas se extienden sobre posibles valores de \(X_1\) y \(X_2\).
    Esencialmente la misma demostración vale para variables continuas,
    con integrales en vez de sumas.
  \end{proof}
  Note que no hemos supuesto nada sobre las variables involucradas.
  Por inducción,
  esto se extiende a un número finito de variables.

\subsection{Independencia}
\label{sec:independencia}

  Decimos que dos variables \(X\) e \(Y\) son \emph{independientes}
  si para todo par de valores \(x\) e \(y\):
  \begin{equation}
    \label{eq:XY-independientes}
    \Pr[(X = x) \wedge (Y = y)]
      = \Pr[X = x] \cdot \Pr[Y = y]
  \end{equation}
  Una colección de variables es \emph{independiente a pares}
  si para todo \(i \ne j\) y todo par de valores \(x_i, x_j\):
  \begin{equation}
    \label{eq:eq:pares-independientes}
    \Pr[(X_i = x_i)
          \wedge (X_j = x_j)
      = \Pr[X_i = x_i]
          \cdot \Pr[X_j = x_j]
  \end{equation}
  Una colección de variables es \emph{mutuamente independiente}
  si para todo subconjunto \(S \subseteq N\)
  y toda colección de valores \(x_i\):
  \begin{equation}
    \label{eq:eq:mutuamente-independientes}
    \Pr[(X_{i_1} = x_{i_1})
          \wedge (X_{i_2} = x_{i_2})
          \wedge \dotsb
          \wedge (X_{i_s} = x_{i_s})]
      = \Pr[X_{i_1} = x_{i_1}]
          \cdot \Pr[X_{i_2} = x_{i_2}]
          \dotsm
          \Pr[X_{i_s} = x_{i_s}]
  \end{equation}

  Un teorema importante sobre variables independientes es:
  \begin{theorem}
    \label{theo:E-independientes}
    Si \(X_1, X_2\) son independientes:
    \begin{align*}
      \Exp[X_1 X_2]
        &= \Exp[X_1] \cdot \Exp[X_2] \\
      \Exp[f(X_1) f(X_2)]
        &= \Exp[f(X_1)] \cdot \Exp[f(X_2)]
    \end{align*}
  \end{theorem}
  \begin{proof}
    Si las variables \(X_1\) y \(X_2\) son independientes,
    entonces:
    \begin{align*}
      \Exp[ X_1 X_2 ]
        &= \sum_{x_1, x_2} x_1 x_2 \Pr[ X_1 = x_1 \wedge X_2 = x_2 ] \\
        &= \sum_{x_1, x_2} x_1 x_2 \Pr[ X_1 = x_1 ] \Pr[X_2 = x_2 ] \\
        &= \sum_{x_1} x_1 \Pr[ X_1 = x_1 ]
             \cdot \sum_{x_2} x_2 \Pr[X_2 = x_2 ]
    \end{align*}
    La misma demostración,
    con integrales en vez de sumas,
    vale para variables continuas.
  \end{proof}
  De la misma forma obtenemos:
  \begin{theorem}
    \label{theo:varianza-suma}
    Si \(X_1\) y \(X_2\) son variables aleatorias independientes,
    entonces:
    \begin{equation*}
      \var[X_1 + X_2]
        = \var[X_1] + \var[X_2].
    \end{equation*}
  \end{theorem}
  \begin{proof}
    Abreviamos \(\mu_1 = \Exp[X_1]\) y \(\mu_2 = \Exp[X_2]\),
    y sabemos \(\Exp[X_1 + X_2] = \mu_1 + \mu_2\):
    \begin{align*}
      \var[X_1 + X_2]
        &= \Exp[(X_1 + X_2 - (\mu_1 + \mu_2))^2] \\
        &= \Exp[(X_1 - \mu_1)^2
                   + 2 (X_1 - \mu_1) (X_2 - \mu_2)
                   + (X_2 - \mu_2)^2] \\
       &= \Exp[(X_1 - \mu_1)^2]
            + 2 \Exp[(X_1 - \mu_1)(X_2 - \mu_2)]
            + \Exp[(X_2 - \mu_2)^2] \\
       &= \Exp[(X_1 - \mu_1)^2]
            + 2 \Exp[X_1 - \mu_1] \Exp[X_2 - \mu_2]
            + \Exp[(X_2 - \mu_2)^2] \\
       &= \var[X_1] + \var[X_2]
    \end{align*}
    El término mixto se anula ya que \(X_1\) y \(X_2\) son independientes.
  \end{proof}

  Un resultado interesante es el siguiente:
  \begin{theorem}
     \label{theo:repeat}
     Considere un experimento con probabilidad de éxito \(p\).
     El número de repeticiones independientes hasta tener éxito
     tiene valor esperado y varianza:
     \begin{align*}
       \Exp[X]
         &= \frac{1}{p} \\
       \var[X]
         &= \frac{1 - p}{p^2}
     \end{align*}
  \end{theorem}
  \begin{proof}
    La variable \(X\) toma valores naturales.
    Si tiene éxito en la \(x\)\nobreakdash-ésima repetición
    quiere decir que falló \(x - 1\) veces y tuvo éxito una vez.
    O sea,
    la probabilidad de este evento es:
    \begin{equation*}
      (1 - p)^{x - 1} p
    \end{equation*}
    La función generatriz de probabilidad relevante es:
    \begin{align*}
      G(z)
        &= \sum_{x \ge 1} (1 - p)^{x - 1} p z^x \\
        &= p z \sum_{x \ge 0} ((1 - p) z)^x \\
        &= \frac{p z}{1 - (1 - p) z}
    \end{align*}
    Podemos calcular las estadísticas del caso:
    \begin{align*}
      \Exp[X]
        &= G'(1) \\
        &= \frac{1}{p} \\
      \var[X]
        &= G''(1) + G'(1) - \left( G'(1) \right)^2 \\
        &= \frac{1 - p}{p^2}
    \end{align*}
  \end{proof}

\section{Relaciones elementales}
\label{sec:elementales-probabilidades}

  Suele ser útil considerar si las funciones entre manos
  son \emph{convexas}
  (o \emph{cóncavas}).
  \begin{definition}
    La función \(f\) se dice \emph{convexa}
    si para \(0 \le \alpha \le 1\):
    \begin{equation*}
      \alpha f(x) + (1 - \alpha) f(y)
        \ge f(\alpha x + (1 - \alpha) y)
    \end{equation*}
  \end{definition}
  Vale decir,
  la gráfica de \(f\) está por debajo de una cuerda que une dos puntos,
  ver la figura~\ref{fig:convexitud}.
  \begin{figure}[ht]
    \centering
    \begin{tikzpicture}
      \begin{axis}[axis lines = left, xtick = \empty, ytick = \empty]
        \addplot[domain = 1.5:5.5, samples = 100]{x^2};
        \addplot[domain = 1.7:5.3, color = red]
            {(5^2 - 2^2) * (x - 2) / (5 - 2) + 2^2};
        \addplot[color = gray] coordinates {(2, 1.5)(2,	 4)};
        \addplot[color = gray] coordinates {(5, 1.5)(5, 25)};
      \end{axis}
    \end{tikzpicture}
    \caption{Una función convexa}
    \label{fig:convexitud}
  \end{figure}
  Similar es:
  \begin{definition}
    La función \(f\) se dice \emph{cóncava}
    si para \(0 \le \alpha \le 1\):
    \begin{equation*}
      \alpha f(x) + (1 - \alpha) f(y)
        \le f(\alpha x + (1 - \alpha) y)
    \end{equation*}
  \end{definition}
  En este caso la función está debajo de la secante.

  Por inducción,
  si \(\alpha_i \ge 0\) con \(\sum_i \alpha_i = 1\),
  para una función convexa es:
  \begin{equation*}
    \sum_i \alpha_i f(x_i)
      \ge f\left( \sum_i \alpha_i x_i \right)
  \end{equation*}
  Aplicando esto a una variable aleatoria finita,
  con los \(\alpha_i\) las probabilidades de los valores de \(X\),
  obtenemos:
  \begin{theorem}[Desigualdad de Jensen]
    Si la función \(f\) es convexa:
    \begin{equation}
      \label{eq:Jensen}
      f\left( \Exp[X] \right)
        \le \Exp[f(X)]
    \end{equation}
  \end{theorem}

\section{Desigualdad de Markov}
\label{sec:desigualdad-Markov}

  Sea \(X\) una variable aleatoria discreta,
  no negativa,
  y sea \(c > 0\) una constante.
  Interesa derivar la probabilidad de que \(X\) sea mayor a \(c\).
  O sea,
  interesa \(\Pr[ X \ge c ]\).
  Como \(X\) es discreta,
  podemos escribir:
  \begin{align*}
    \mu
      &=   \Exp[X] \\
      &=   \sum_{x} x \Pr[X = x] \\
      &=   \sum_{0 < x < c} x \Pr[X = x] + \sum_{x \ge c} x \Pr[X = x] \\
      &\ge \sum_{x \ge c} x \Pr[X = x] \\
      &\ge \sum_{x \ge c} c \Pr[X = x] \\
      &\ge c \sum_{x \ge c} \Pr[X = x] \\
      &=   c \Pr[X \ge c]
  \end{align*}
  de donde tenemos:
  \begin{theorem}[Desigualdad de Markov]
    \label{theo:Markov-inequality}
    \begin{equation}
      \label{eq:Markov-inequality}
      \Pr[X \ge c]
      \le \frac{\mu}{c}
    \end{equation}
  \end{theorem}
  Es claro que exactamente lo mismo puede hacerse si \(X\)
  es una variable continua.

  Esta desigualdad es útil por sí misma,
  pero aún más porque es instrumental
  en la derivación de desigualdades más ajustadas.

\section{Desigualdad de Chebyshev}
\label{sec:desigualdad-Chebyshev}

  Para una variable aleatoria general \(X\)
  nos interesa acotar la probabilidad
  \(\Pr[ \lvert X - \mu \rvert > a ]\),
  donde \(\mu = \Exp[X]\).
  Notamos que este es el mismo evento
  \((X - \mu)^2 > a^2\),
  como \((X - \mu)^2\) es una variable no negativa,
  podemos aplicarle la desigualdad de Markov.
  Recordando que \(\Exp[ (X - \mu)^2 ] = \var[X] = \sigma^2\):
  \begin{align*}
    \Pr[ \lvert X - \mu \rvert > a ]
      &=   \Pr[ (X - \mu)^2 > a^2 ] \\
      &\le \frac{\Exp[ (X - \mu)^2 ]}{a^2} \\
      &=   \frac{\sigma^2}{a^2}
  \end{align*}
  En particular,
  substituyendo \(a = c \sigma\):
  \begin{theorem}[Desigualdad de Chebyshev]
    \label{theo:Chebyshev}
    \begin{equation}
      \label{eq:Chebyshev}
      \Pr[ \lvert X - \mu \rvert \ge c \sigma ]
      \le \frac{1}{c^2}
    \end{equation}
  \end{theorem}

\section{Momentos superiores}
\label{sec:momentos-superiores}

  Hemos definido la varianza de una variable aleatoria \(X\) con media \(\mu\)
  como:
  \begin{equation*}
    \sigma^2
      = \Exp[ (X - \mu)^2 ]
  \end{equation*}
  Podemos extender esto a otras potencias,
  obteniendo los \emph{momentos centrales}:
  \begin{equation*}
    \mu_k
      = \Exp[ (X - \mu)^k ]
  \end{equation*}
  Note que:
  \begin{align*}
    \mu_1
      &= \Exp[ (X - \mu)^1 ]
        = 0 \\
    \mu_2
      &= \Exp[ (X - \mu)^2 ]
        = \sigma^2
  \end{align*}
  Los momentos centrales
  dan una medida de cuánto se dispersa la distribución de la media.
  Mayores \(k\) dan mayor importancia a valores más alejados.

  Podemos usar el mismo truco
  que usamos para deducir la desigualdad de Chebyshev
  para obtener una desigualdad involucrando \(\mu_4\).

  Sea \(X\) una variable aleatoria con media \(\mu\)
  y cuarto momento central \(\mu_4\).
  Entonces:
  \begin{align*}
    \Pr[ \lvert X - \mu \rvert \ge c \sqrt[4]{\mu_4} ]
      &=   \Pr[ \lvert X - \mu \rvert^4 \ge c^4 \mu_4 ] \\
    \intertext{Aplicando la desigualdad de Markov a \((X - \mu)^4\),
              sabiendo que \(\Exp[ (X - \mu)^4 ] = \mu_4\):}
      &=   \Pr[(X - \mu)^4 \ge c^4 E[(X - \mu)^4]] \\
      &\le \frac{1}{c^4}
  \end{align*}

  Un desarrollo afín es posible siempre que \(k\) es par.

\section{Cotas de Chernoff}
\label{sec:Chernoff}

  Para obtener la cota de Chebyshev elevamos al cuadrado,
  ahora exponenciamos.
  Esto da toda una familia de desigualdades,
  dependiendo de la distribución supuesta
  y el detalle de las cotas empleadas para simplificar el resultado.
  La importancia radica en que da cotas ajustadas
  usando información mínima sobre las variables.
  La idea básica es de Chernoff~%
   \cite{chernoff52:_bound}.
  Desarrollaremos una versión general en detalle,
  basándonos en Lehman, Leighton y Meyer~%
    \cite[sección~20.6.2]{lehman18:_mathem_comput_scien}.
  Si se revisan los detalles de la demostración,
  es claro que el mismo desarrollo se aplica
  si alguna de las variables es continua.

  \begin{theorem}[Cota superior de Chernoff]
    \label{theo:Chernoff-upper-tail}
    Sean \(X_1, \dotsc, X_n\) variables aleatorias discretas
    mutuamente independientes
    con \(0 \le X_i \le 1\) para todo \(i\).
    Sea \(X = X_1 + \dotsb + X_n\)
    y sea \(\mu = \Exp[X]\).
    Entonces para todo \(c \ge 1\):
    \begin{equation}
      \label{eq:Chernoff-upper-tail}
      \Pr[X \ge c \mu]
        \le \mathrm{e}^{- \beta(c) \mu}
    \end{equation}
    donde \(\beta(c) = c \ln c - c + 1\).
  \end{theorem}
  En aras de la claridad,
  partiremos con la demostración central,
  lo que mostrará la necesidad
  de acotar feos productos,
  cotas que demostraremos inmediatamente a continuación como lemas.

  Dejamos anotado para uso futuro
  que la función \(\beta(c)\) es convexa para \(c > 0\),
  con un mínimo en \(c = 1\) donde \(\beta(1) = 0\).
  \begin{proof}
    El punto clave es exponenciar ambos lados de la desigualdad
    \(X \ge c \mu\) y aplicar la desigualdad de Markov:
    \begin{align*}
      \Pr[X \ge c \mu]
        &=   \Pr[c^X \ge c^{c \mu}] \\
        &\le \frac{\Exp[c^X]}{c^{c \mu}}
                && \text{(cota de Markov)} \\
        &\le \frac{\mathrm{e^{(c - 1) \mu}}}{\mathrm{e}^{\mu c \ln c}}
               && \text{(ver lema~\ref{lem:Chernoff-X})} \\
        &=   \mathrm{e}^{- \beta(c) \mu}
    \end{align*}
  \end{proof}
  Hemos usado el siguiente resultado,
  expresado con las mismas variables anteriores:
  \begin{lemma}
    \label{lem:Chernoff-X}
    \begin{equation*}
      \Exp[c^X]
        \le \mathrm{e}^{(c - 1) \mu}
    \end{equation*}
  \end{lemma}
  \begin{proof}
    \begin{align*}
      \Exp\left[ c^X \right]
        &= \Exp\left[ c^{X_1 + \dotsb + X_n} \right]
               && \text{(definición de \(X\))} \\
        &=   \Exp\left[ c^{X_1} \dotsm c^{X_n} \right] \\
        &=   \Exp\left[ c^{X_1} \right] \dotsm \Exp\left[ c^{X_n} \right]
               && \text{(producto de valores independientes)} \\
        &\le \mathrm{e}^{(c - 1) \Exp[X_1]}
                \dotsm \mathrm{e}^{(c - 1) \Exp[X_n]}
               && \text{(ver lema~\ref{lem:Chernoff-Xi} más adelante)} \\
        &= \mathrm{e}^{(c - 1) (\Exp[X_1] + \dotsb + \Exp[X_n])} \\
        &= \mathrm{e}^{(c - 1) (\Exp[X_1 + \dotsb + X_n])}
               && \text{(linearidad del valor esperado)} \\
        &= \mathrm{e}^{(c - 1) \Exp[X]}
    \end{align*}
  \end{proof}
  Finalmente:
  \begin{lemma}
    \label{lem:Chernoff-Xi}
    \begin{equation*}
      \Exp\left[ c^{X_i} \right]
        \le \mathrm{e}^{(c - 1) \Exp[X_i]}
    \end{equation*}
  \end{lemma}
  \begin{proof}
    En lo que sigue,
    las sumas son sobre valores \(x\) tomados por la variable \(X_i\).
    Por la definición de \(X_i\),
    \(x \in [0, 1]\).
    \begin{align*}
      \Exp\left[ c^{X_i} \right]
        &=   \sum_x c^x \Pr[X_i = x]
                && \text{(definición de valor esperado)} \\
        &\le \sum_x (1 + (c - 1) x) \Pr[X_i = x]
                && \text{(convexidad -- ver abajo)} \\
        &=   \sum_x ( \Pr[X_i = x] + (1 - c) x \Pr[X_i = x] ) \\
        &=   1 + (c - 1) \Exp[X_i] \\
        &\le \mathrm{e}^{(c - 1) \Exp[X_i]}
                && \text{(porque \(1 + z \le \mathrm{e}^z\))}
    \end{align*}
    El segundo paso usa la desigualdad:
    \begin{equation*}
      c^x
        \le 1 + (c - 1) x
    \end{equation*}
    que vale para \(c \ge 1\) y \(0 \le x \le 1\)
    ya que la función convexa \(c^x\)
    está por debajo de la cuerda entre los puntos \(x = 0\) y \(x = 1\).
    Esta es la razón de restringir las variables \(X_i\) a \([0, 1]\),
    y el descomponer
    la demostración del lema~\ref{lem:Chernoff-X}.
  \end{proof}
  Ocasionalmente interesa una cota superior,
  provista por el siguiente teorema.
  \begin{theorem}
    \label{theo:Chernoff-lower-tail}
    Con las mismas suposiciones del teorema~\ref{theo:Chernoff-upper-tail}:
    \begin{equation}
      \label{eq:Chernoff-lower-tail}
      \Pr[X < \mu / c]
        \le \mathrm{e}^{- \beta(1 / c) \mu}
    \end{equation}
  \end{theorem}
  \begin{proof}
    La demostración es esencialmente igual
    a la del teorema~\ref{theo:Chernoff-upper-tail}.
    \begin{align*}
      \Pr[X < \mu / c]
        &=   \Pr\left[ c^{-X} > c^{- \mu / c} \right] \\
        &\le \frac{\Exp[c^{- X}]}{c^{- \mu / c}}
                && \text{(cota de Markov)} \\
        &\le \frac{\mathrm{e}^{-(1 - 1/c) \mu}}{\mathrm{e}^{- \mu \ln c / c}}
                && \text{(ver lema~\ref{lem:Chernoff-X-2})}
    \end{align*}
    Reorganizando esto obtenemos lo aseverado.
  \end{proof}
  Tenemos las variantes
  de los lemas~\ref{lem:Chernoff-X} y~\ref{lem:Chernoff-Xi}:
  \begin{lemma}
    \label{lem:Chernoff-X-2}
    \begin{equation*}
      \Exp[c^{-X}]
        \le \mathrm{e}^{- (1 - 1 / c) \mu}
    \end{equation*}
  \end{lemma}
  \begin{proof}
    \begin{align*}
      \Exp\left[ c^{-X} \right]
        &= \Exp\left[ c^{-X_1 - \dotsb - X_n} \right]
               && \text{(definición de \(X\))} \\
        &=   \Exp\left[ c^{-X_1} \dotsm c^{-X_n} \right] \\
        &=   \Exp\left[ c^{-X_1} \right] \dotsm \Exp\left[ c^{-X_n} \right]
               && \text{(producto de valores independientes)} \\
        &\le \mathrm{e}^{- (1 - 1 / c) \Exp[X_1]}
               \dotsm \mathrm{e}^{- (1 - 1 / c) \Exp[X_n]}
               && \text{(ver lema~\ref{lem:Chernoff-Xi-2} más adelante)} \\
        &= \mathrm{e}^{- (1 - 1 / c) (\Exp[X_1] + \dotsb + \Exp[X_n])} \\
        &= \mathrm{e}^{- (1 - 1 / c) (\Exp[X_1 + \dotsb + X_n])}
               && \text{(linearidad del valor esperado)} \\
        &= \mathrm{e}^{- (1 - 1 / c) \Exp[X]}
    \end{align*}
  \end{proof}
  \begin{lemma}
    \label{lem:Chernoff-Xi-2}
    \begin{equation*}
      \Exp\left[ c^{-X_i} \right]
        \le \mathrm{e}^{- (1 - 1 / c) \Exp[X_i]}
    \end{equation*}
  \end{lemma}
  \begin{proof}
    En lo que sigue,
    las sumas son sobre valores \(x\) tomados por la variable \(X_i\).
    Por la definición de \(X_i\),
    \(x \in [0, 1]\).
    \begin{align*}
      \Exp\left[ c^{-X_i} \right]
        &=   \sum c^{-x} \Pr[X_i = x]
                && \text{(definición de valor esperado)} \\
        &\le \sum (1 - (1 - 1 / c) x) \Pr[X_i = x]
                && \text{(convexidad -- ver abajo)} \\
        &=   \sum ( \Pr[X_i = x] - (1 - 1/ c) x \Pr[X_i = x] ) \\
        &=   1 - (1 - 1 / c) \Exp[X_i] \\
        &\le \mathrm{e}^{- (1 - 1 / c) \Exp[X_i]}
                && \text{(porque \(1 + z \le \mathrm{e}^z\))}
    \end{align*}
    El segundo paso usa la desigualdad:
    \begin{equation*}
      c^{-x}
        \le 1 - (1 - 1 / c) x
    \end{equation*}
    que vale para \(c \ge 1\) y \(1 \le x \le 1\)
    ya que la función convexa \(c^{-x}\)
    está por debajo de la cuerda entre los puntos \(x = 0\) y \(x = 1\).
    Esta es la razón de restringir las variables \(X_i\) a \([0, 1]\),
    y el descomponer
    la demostración del lema~\ref{lem:Chernoff-X-2}.
  \end{proof}

\section{Cotas de Høffding}
\label{sec:Hoeffding-inequality}

  Una desigualdad muy útil es la de Høffding
  (a veces transliterado Hoeffding).
  Se ha dicho que todos los teoremas de teoría de aprendizaje
  son aplicaciones de estas cotas\ldots

  Es similar a las de Chernoff~\ref{theo:Chernoff-upper-tail},
  pero no exige variables \([0, 1]\).
  Primero un lema.
  \begin{lemma}[Høffding]
    \label{lem:Hoeffding}
    Sea \(X\) una variable aleatoria real tal que \(Exp[X] = 0\)
    y tal que \(a \le X \le b\).
    Entonces,
    para todo \(\lambda \in \mathbb{R}\):
    \begin{equation}
      \label{eq:Hoeffding-lemma}
      \Exp\left[ \mathrm{e}^{\lambda X} \right]
        \le \exp\left( \frac{\lambda^2 (b - a)^2}{8} \right)
    \end{equation}
  \end{lemma}
  Note que por las suposiciones sobre \(X\) debe ser \(a < 0 < b\).
  \begin{proof}
    Como \(\mathrm{e}^{\lambda x}\) es una función convexa en \(\mathbb{R}\),
    para \(a \le x \le b\) podemos escribir
    (usando una forma un tanto curiosa de la línea que conecta los puntos
     \((a, \mathrm{e}^{\lambda a})\)
     y \((a, \mathrm{e}^{\lambda a})\)):
     \begin{equation*}
       \mathrm{e}^{\lambda x}
         \le \frac{b - x}{b - a} \mathrm{e}^{\lambda a}
               + \frac{x - a}{b - a} \mathrm{e}^{\lambda b}
     \end{equation*}
     Por linealidad del valor esperado:
     \begin{equation*}
       \Exp\left[ \mathrm{e}^{\lambda X} \right]
         \le \frac{b - \Exp[X]}{b - a} \mathrm{e}^{\lambda a}
               + \frac{\Exp[X] - a}{b - a} \mathrm{e}^{\lambda b}
     \end{equation*}
     Para simplificar,
     definamos:
     \begin{align*}
       h
         &= \lambda (b - a) \\
       p
         &= \frac{-a}{b - a} \\
       L(h)
         &= - h p + \ln(1 - p + p \mathrm{e}^h)
     \end{align*}
     Note que \(p \ge 0\) y \(1 - p \ge 0\),
     por lo que \(L\) está definido para todo \(\mathbb{R}\).
     De lo anterior,
     como \(\Exp[X] = 0\) vemos que:
     \begin{align*}
       \frac{b - \Exp[X]}{b - a} \mathrm{e}^{\lambda a}
         + \frac{\Exp[X] - a}{b - a} \mathrm{e}^{\lambda b}
         = \mathrm{e}^{L(h)}
     \end{align*}
     Derivando \(L(h)\) obtenemos:
     \begin{align*}
       L(0)
         &= L'(0)
          = 0 \\
       L''(h)
         &= \frac{p \mathrm{e}^h}{p \mathrm{e}^h + 1 - p}
              - \frac{p \mathrm{e}^{2 h}}{(p \mathrm{e}^h + 1 - p)^2}
     \end{align*}
     Nos interesa acotar \(L(h)\).
     Veamos el máximo de \(L(h)\),
     que podemos escribir en términos de \(u = \mathrm{e}^h\):
     \begin{align*}
       \frac{\mathrm{d}}{\mathrm{d} u}
          \left( \frac{p u}{p u + 1 - p}
                    - \frac{p^2 u^2}{(p u + 1 - p)^2} \right)
         &= \frac{p}{(p u + 1 - p}
              - \frac{3 p^2 u}{(p u + 1 - p)^2}
              + \frac{2 p^3 u^2}{(p u + 1 - p)^3} \\
         &= 0
     \end{align*}
     Esto entrega:
     \begin{align*}
       u
         &= \frac{1 - p}{p} \\
       L''\left( \ln \frac{1 - p}{p} \right)
         &= \frac{1}{4} \\
       L'''\left( \ln \frac{1 - p}{p} \right)
         &= - \frac{p^2}{8 (p - 1)^2}
          < 0
     \end{align*}
     Por el teorema de Taylor con el resto en la forma de Lagrange,
     sabemos que para todo \(h \in \mathbb{R}\)
     hay \(\zeta\) entre \num{0} y \(h\) tal que::
     \begin{align*}
       L(h)
         &=   L(0)
                + L'(0) h
                + \frac{1}{2!} L''(\zeta) h^2 \\
         &\le \frac{h^2}{8} \\
         &=   \frac{\lambda^2 (b - a)^2}{8}
     \end{align*}
     Reemplazado obtenemos la desigualdad~\eqref{eq:Hoeffding-lemma}.
     \qedhere
  \end{proof}
  Ahora la desigualdad principal:
  \begin{theorem}[Desigualdad de Høffding]
    \label{theo:Hoeffding}
    Sean \(X_i\) variables aleatorias tales que \(a_i \le X_i \le b_i\)
    para \(1 \le i \le n\).
    Defina:
    \begin{equation*}
      \overline{X}
        = \frac{1}{n} \sum_{1 \le i \le n} X_i
    \end{equation*}
    Entonces,
    para \(t > 0\):
    \begin{align}
      \Pr\left[ \overline{X} - \Exp[ \overline{X} ] \ge t \right]
        &\le \exp\left(
                    - \frac{2 n^2 t^2}{\sum_{1 \le i \le n} (b_i - a_i)^2}
                 \right)
           \label{eq:Hoeffding-1} \\
      \Pr\left[\left\lvert
                 \overline{X} - \Exp[ \overline{X} ] \ge t \right\rvert\right]
        &\le 2 \exp\left(
                      - \frac{2 n^2 t^2}{\sum_{1 \le i \le n} (b_i - a_i)^2}
                   \right)
           \label{eq:Hoeffding-2}
    \end{align}
  \end{theorem}
  Note que las ecuaciones~\eqref{eq:Hoeffding-1} y~\eqref{eq:Hoeffding-2}
  pueden también expresarse en términos de la suma:
  \begin{equation*}
    S
      = \sum_{1 \le i n} X_i
  \end{equation*}
  como:
  \begin{align}
    \Pr\left[ S - \Exp[ \overline{S} ] \ge t \right]
      &\le \exp\left(
                  - \frac{2 t^2}{\sum_{1 \le i \le n} (b_i - a_i)^2}
               \right)
         \label{eq:Hoeffding-sum-1} \\
    \Pr\left[\left\lvert
               S - \Exp[ \overline{S} ] \ge t \right\rvert\right]
      &\le 2 \exp\left(
                    - \frac{2 t^2}{\sum_{1 \le i \le n} (b_i - a_i)^2}
                 \right)
         \label{eq:Hoeffding-sum-2}
  \end{align}
  \begin{proof}
    Sea:
    \begin{equation*}
      S
        = \sum_{1 \le i \le n} X_i
    \end{equation*}
    Por la desigualdad de Markov,
    ecuación~\eqref{eq:Markov-inequality},
    y la independencia de los \(X_i\)
    para \(s, t \ge 0\) es:
    \begin{align*}
      \Pr[ S - \Exp[S] \ge t]
        &=   Pr\left[
                  \mathrm{e}^{s (S - \Exp[S])} \ge \mathrm{e}^{s t}
                \right] \\
        &\le \mathrm{e}^{-s t}
               \Exp\left[
                      \mathrm{e}^{s(S - \Exp[S])}
                   \right] \\
        &=   \mathrm{e}^{-s t}
               \prod_{1 \le i \le n}
                 \Exp\left[
                        \mathrm{e}^{s(X_i - \Exp[X_i])}
                     \right] \\
        &\le \mathrm{e}^{-s t}
               \prod_{1 \le i \le n}
                  \mathrm{e}^{s^2 (b_i - a_i)^2}{8} \\
        &=   \exp\left(
                    -st + \frac{1}{8} s^2 \sum_{1 \le i \le n} (b_i - a_i)^2
                 \right)
    \end{align*}
    Para obtener la mejor cota posible,
    hallamos el mínimo del lado derecho en términos de \(s\).
    Vemos que el exponente alcanza su mínimo para:
    \begin{equation*}
      s
        = \frac{4 t}{\sum_{1 \le i \le n} (b_i - a_i)^2}
    \end{equation*}
    Esto da~\eqref{eq:Hoeffding-1}.
    La desigualdad~\eqref{eq:Hoeffding-2}
    resulta de aplicar~~\eqref{eq:Hoeffding-1}
    a ambos lados.
  \end{proof}

\section*{Ejercicios}
\label{sec:ejercicios-27previa}

  \begin{enumerate}
  \item
    Complete las demostraciones de las propiedades de probabilidades de eventos
    de la sección~\ref{sec:formalizar-probabilidad}.
  \item
    Se da a entender en la sección~\ref{sec:momentos-superiores}
    que la técnica empleada no es aplicable si \(k\) es impar.
    Explique.
  \end{enumerate}

\bibliography{../referencias}

%%% Local Variables:
%%% mode: latex
%%% TeX-master: "../INF-221_notas"
%%% ispell-local-dictionary: "spanish"
%%% End:

% LocalWords:  aleatorizados muestral eq union ésima exponenciamos
% LocalWords:  exponenciar lemma inequality


\backmatter

%\printindex
\end{document}

%%% Local Variables:
%%% mode: latex
%%% TeX-master: t
%%% ispell-local-dictionary: "spanish"
%%% End:

% LocalWords:  lstlisting lol chapter listofalgorithms mdseries end H
% LocalWords:  Function function Procedure procedure Downto downto em
% LocalWords:  Loop loop Continue continue Break break KwStep step
% LocalWords:  INF Horst von Brand Federico Utopia Fourier GUTenberg
% LocalWords:  Aldo Berríos Claudio empty plain María
