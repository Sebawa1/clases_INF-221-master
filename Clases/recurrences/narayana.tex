\section{Asintóticas}
\label{sec:asymptotics-recurrence}

  Como ejemplo de una técnica útil para obtener una estimación asintótica
  de la solución de una recurrencia,
  partiremos con la recurrencia planteada por Narayana Pandita,
  un matemático indio del siglo XIV.
  Plantea que al inicio tiene un ternero,
  que a los tres años se transforma en vaca.
  Cada vaca anualmente produce un nuevo ternero,
  y pregunta cuántos animales tiene al año \num{17}.
  Esto lleva a la recurrencia:
  \begin{equation}
    \label{eq:Narayana-cows-recurrence}
    a_{n + 3}
      = a_{n + 2} + a_n
      \qquad a_0 = a_1 = a_2 = 1
  \end{equation}
  La danza ritual da la función generatriz:
  \begin{align}
    A(z)
      &= \sum_{n \ge 0} a_n z^n
            \notag \\
      &= \frac{1}{1 - z - z^3}
            \label{eq:Narayana-cows-gf}
  \end{align}
  Lamentablemente,
  los ceros del polinomio característico \(p(\rho) = \rho^3 - \rho^2 - 1\)
  son expresiones muy complicadas,
  lo que se entorpece nuestra técnica de dividir en fracciones parciales
  y leer los coeficientes de allí.
  Tomaremos un camino indirecto.

  Primeramente,
  por la regla de signos de Descartes
  (ver por ejemplo~%
      \cite{levin20:_descartes_rule_signs})
  hay un solo cero real positivo
  (hay un único cambio de signo en la secuencia de coeficientes de \(p\))
  y ninguno negativo
  (\(p(-\rho)\) no tiene cambios de signo en su secuencia de coeficientes).
  Hay dos ceros complejos conjugados.
  Por las fórmulas de Vieta
  sabemos que el producto de los ceros es \(-1\)
  con lo que si \(\rho\) es el cero positivo,
  los ceros complejos tienen magnitud \(\rho^{-1/2}\).
  Una rápida aplicación del método de Newton
  (ver capítulo~\ref{cha:ceros-funciones})
  nos dice que \(\rho = 1,465571\),
  con lo que los ceros complejos tienen magnitud \num{0,8260314}.
  La solución será dominada por el cero real.

  De la expansión en fracciones parciales:
  \begin{align*}
    F(z)
      &= \frac{A}{1 - \rho z}
          + \frac{B_+}{1 - c_+ z} + \frac{B_-}{1 - c_- z} \\
    A
      &= \lim_{z \to 1 / \rho} F(z) (1 - \rho z)
  \end{align*}
  Podemos usar l'Hôpital para calcular el límite
  (por construcción,
   es de la forma \(0 / 0\)):
  \begin{align*}
    A
      &= \lim_{z \to 1 / \rho}	\frac{1 - \rho z}{1 - z - z^3} \\
      &= \lim_{z \to 1 / \rho} \frac{- \rho}{-1 - 3 z^2} \\
      &= \frac{\rho^3}{\rho^2 + 3} \\
      &= 0,6114918
  \end{align*}
  Sabemos entonces que:
  \begin{equation*}
    a_n
      = \frac{\rho^3}{\rho^2 + 3} \cdot \rho^n
        + O(\rho^{-n/2}) \\
  \end{equation*}
  Como una rápida confirmación,
  calculamos \(a_{10} = 28\),
  la fórmula aproximada da \num{27,95}.
  Resulta que aproximar al entero más cercano
  ya da el valor correcto para \(n = 0\).

  Un programita nos indica que \(a_{17} = 406\),
  nuestra aproximación es \num{405,98}.

%%% Local Variables:
%%% mode: latex
%%% TeX-master: "../INF-221_notas"
%%% ispell-local-dictionary: "spanish"
%%% End:

% LocalWords:  XIV programita
