\bibliographystyle{babplain-fl}

\chapter{Recurrencias}
\label{apx:recurrencias}

  Una \emph{recurrencia} es una descripción recursiva de una función,
  la describe en términos de ella misma.
  Como toda definición recursiva,
  hay algunos \emph{casos base}
  y \emph{casos recursivos}.
  Cada caso es una igualdad o desigualdad,
  donde los casos base dan valores explícitos
  y los casos recursivos
  relacionan el valor de la función con valores anteriores.

  Decimos que una función \emph{satisface} la recurrencia,
  o es \emph{solución} de la recurrencia,
  si para ella cada uno de los casos se cumple.
  En la mayoría de los casos que encontramos en la práctica
  hay una solución
  -- podemos escribir una función recursiva de la recurrencia,
  y la función que esta compute es la solución.
  Aún más,
  en casos de interés,
  si cada uno de los casos es una igualdad
  la solución es única.

  Por sí misma la recurrencia
  no es una descripción satisfactoria de la función.
  Buscamos una \emph{forma cerrada},
  una descripción no recursiva de la función.
  Pero una forma cerrada exacta puede no existir,
  o ser demasiado complicada para uso práctico.
  Muchas veces nos conformamos con una solución asintótica
  del tipo \(f(n) = \Theta(g(n))\),
  aunque Sedgewick y Flajolet~%
    \cite{sedgewick13:_introd_anal_algor}
  arguyen que debiéramos ir más allá
  y dar asintóticas de la forma \(f(n) \sim g(n)\).

  Para desigualdades recursivas preferimos una solución ajustada,
  que se cumple incluso si las desigualdades
  se reemplazan por las igualdades correspondientes.
  Nuevamente,
  soluciones ajustadas exactas pueden no existir o ser demasiado complicadas,
  y nos conformamos con \(f(n) = \Omega(g(n))\) o \(f(n) = O(g(n))\),
  según sea el caso.
  Vea también el apéndice~\ref{apx:asymptotics}.

\section{El método definitivo: adivinar y verificar}
\label{sec:adivinar-verificar}

  Siempre podemos aplicar la idea de adivinar la solución
  y verificar que cumple la recurrencia.
  Esto es esencialmente una demostración por inducción:
  verificamos que cumple los casos base
  y los casos recursivos
  (inducción).
  Claro que esto traslada el problema a adivinar la solución.
  Algunas de las secciones siguientes
  presentan guías que ayudan a adivinar correctamente.

\subsection{Calcular valores}
\label{sec:calcular-valores}

  En el caso de las torres de Hanoi,
  tenemos la recurrencia:
  \begin{equation*}
    H(n)
      = 2 H(n - 1) + 1
      \quad H(0) = 0
  \end{equation*}
  Viendo la recurrencia,
  sospechamos que la solución es algo como \(H(n) = 2^n\),
  probando algunos valores obtenemos \(0, 1, 3, 7, 15, \dotsc\),
  parece que \(H(n) = 2^n - 1\).
  Substituyendo,
  vemos que cumple la recurrencia.

  Acá es relevante la forma en que se dejan los elementos.
  Si se simplifican demasiado,
  puede ocultar su forma.
  No simplificar lo suficiente también puede ser contraproducente.

\subsection{Desenrollar la recurrencia}
\label{sec:desenrollar-recurrencia}

  Otra alternativa es desenrollar la recurrencia,
  substituyéndola en sí misma:
  \begin{align*}
    H(n)
      &= 2 H(n - 1) + 1 \\
      &= 2 (2 H(n - 2) + 1) + 1
       = 4 H(n - 2) + 3 \\
      &= 4 (2 H(n - 3) + 1) + 3
       = 8 H(n - 3) + 7 \\
      &= \dotsb
  \end{align*}
  Da la impresión que al desenrollar
  resulta \(H(n) = 2^k H(n - k) + 2^k - 1\).
  Esto podemos demostrarlo por inducción,
  y finalmente deducir \(H(n) = 2^n H(0) + 2^n - 1 = 2^n - 1\).
  Incidentalmente,
  nos da la dependencia de \(H(n)\) del valor inicial.

  Los números de Fibonacci dan otro ejemplo:
  \begin{equation*}
    F_{n + 2}
      = F_{n + 1} + F_n
      \quad F_0 = 0, F_1 = 1
  \end{equation*}
  Los primeros valores
  \(0, 1, 1, 2, 3, 5, \dotsc\) no muestran un patrón claro,
  pero la forma hace sospechar algo de la forma \(c \alpha^n\).
  Intentemos demostrar por inducción que \(F_n \le c \alpha^n\)
  y ver dónde llegamos:
  \begin{align*}
    F_{n + 2}
      &=   F_{n + 1} + F_n \\
      &\le c \alpha^{n + 1} + c \alpha^n
  \end{align*}
  Por otro lado,
  queremos que:
  \begin{align*}
    F_{n + 2}
      &\le c \alpha^{n + 2}
  \end{align*}
  Una forma de lograr que ambas se cumplan es:
  \begin{align*}
    c \alpha^{n + 2}
      &\ge c \alpha^{n + 1} + c \alpha^n \\
    \alpha^2
      &\ge \alpha + 1
  \end{align*}
  El mínimo \(\alpha\) que funciona es \(\alpha = \tau = (1 + \sqrt{5}) / 2\)
  (el otro cero de la ecuación es menor,
   y podemos ignorarlo).
  Faltan los casos base,
  que determinan la constante \(c\).
  Resultan:
  \begin{align*}
    F_0
      &= 0 \le c \tau^0 \\
    F_1
      &= 1 \le c \tau^1
  \end{align*}
  Ambas se cumplen si elegimos \(c = 1 / \tau\),
  hemos demostrado
  \begin{equation*}
    F_n
      \le \tau^{n - 1}
  \end{equation*}
  Vamos por una cota inferior ahora.
  Podemos usar la misma estrategia con \(F_n \ge d \tau^n\),
  pero el único valor que funciona en todos los casos es el trivial \(d = 0\).
  Pero nos interesa \emph{que funcione para \(n\) grande},
  podemos omitir el caso \(n = 0\),
  que nos lleva a concluir que para \(n \ge 1\) funciona \(d = 1 / \tau^2\),
  y:
  \begin{equation*}
    \tau^{n - 2}
      \le F_n
      \le \tau^{n - 1}
  \end{equation*}
  Uniendo ambas cotas y ajustando la constante es:
  \begin{equation*}
    F_n
      = \Theta(\tau^n)
  \end{equation*}

  Veamos la recurrencia:
  \begin{equation*}
    T(n)
      = \sqrt{n} T(\sqrt(n)) + n
  \end{equation*}
  Es claro que \(T(n) > n\),
  pero no mucho.
  Si intentamos \(T(n) \le a n \log_2 n\),
  la demostración funciona bien:
  \begin{align*}
    T(n)
      &=   \sqrt{n} T(\sqrt{n}) + n \\
      &\le \sqrt{n} \cdot a \sqrt{n} \log_2 \sqrt{n} + n \\
      &=   \frac{a}{2} n \log_2 n + n \\
      &\le a n \log_2 n
  \end{align*}
  siempre que \(1 \le a/2 \log_2 n\),
  o \(n \ge 2^{2/a}\).
  Pero esto funciona para cualquier \(a\),
  no importa qué tan chico.
  Esto indica que nuestro intento es muy grande.
  Buscar \(b\) tal que \(T(n) \ge b n \log_2 n\) falla,
  lo que confirma que estamos mal.
  Debemos buscar alguna función entre \(n\) y \(n \log_2 n\).
  Intentemos \(n \log_2 \log_2 n\):
  \begin{align*}
    T(n)
      &\le \sqrt{n} \cdot a \sqrt{n} \log_2 \log_2 \sqrt{n} + n \\
      &=   a n \log_2 \log_2 n - a n + n \\
      &\le a n \log_2 \log_2 n
  \end{align*}
  Lo último siempre y cuando \(a \ge 1\),
  tener una cota inferior para \(a\)
  es fuerte indicación que vamos por el camino correcto.
  Intentando \(T(n) \ge b n \log_2 \log_2 n\)
  tenemos éxito siempre que \(b \le 1\),
  concluimos que \(T(n) = \Theta(n \log \log n)\).

\subsection{Coeficientes indeterminados}
\label{sec:coef-indet}

  La idea es combinar soluciones parciales,
  dependiendo de parámetros a determinar.
  Volvemos a los números de Fibonacci,
  que llevaron a sospechar soluciones de las formas
  \(c_1 \tau^n\) y \(c_2 \phi^n\),
  donde \(\tau\) y \(\phi\) son los ceros de \(x^2 = x + 1\):
  \begin{align*}
    \tau
      &= \frac{1 + \sqrt{5}}{2} \\
    \phi
      &= \frac{1 - \sqrt{5}}{2}
  \end{align*}
  Substituyendo el intento \(F_n = c_1 \tau^n + c_2 \phi^n\)
  en la recurrencia vemos que cumple el caso recursivo
  para todo \(c_1\) y \(c_2\),
  los casos base dan dos ecuaciones que permiten determinar las constantes:
  \begin{align*}
    F_n
      &= \frac{\tau^n - \phi^n}{\tau - \phi} \\
      &= \frac{(1 + \sqrt{5})^n - (1 - \sqrt{5})^n}{2^n \sqrt{5}}
  \end{align*}
  Es sorprendente que los números de Fibonacci,
  claramente enteros,
  se expresen en términos de números irracionales.

  Una variante es lo que Knuth llama \emph{método del repertorio}.
  La idea es armar un repertorio de secuencias \(f(n)\)
  con sus correspondientes resultados de aplicar los casos recursivos,
  para construir la solución que nos interesa
  mediante una combinación lineal.
  Un ejemplo tomado de Rossmanith~%
    \cite{rossmanith12:_analysis_algorithms}
  es hallar \(a_n\) tal que:
  \begin{equation*}
    a_n
      = n a_{n - 1} + n^2 a_{n - 2}
         - n^4 - 3 n^2 + 5
      \quad a_0 = 5, a_1 = 9
  \end{equation*}
  Es claro que si \(a_n\) es un polinomio en \(n\),
  lo es \(a_n - n a_{n - 1} - n^2 a_{n - 2}\).
  No tiene sentido ir más allá de \(n^2\),
  daría potencias mayores a \(n^4\).
  Armamos el cuadro~\ref{tab:operador} con algunos polinomios simples.
  \begin{table}[ht]
    \centering
    \begin{tabular}{*{2}{>{\(}l<{\)}}}
      \multicolumn{1}{c}{\boldmath\(a_n\)\unboldmath} &
        \multicolumn{1}{c}
            {\boldmath\(a_n - n a_{n - 1} - n^2 a_{n - 2}\)\unboldmath} \\
      \hline
      1	  & - n^2 - n + 1 \\
      n	  & - n^3 + n^2 + 2 n \\
      n^2 & - n^4 + 3 n^3 - n^2 - n
      \end{tabular}
    \caption{Valores del operador \(G = a_n - n a_{n - 1} - n^2 a_{n - 2}\)}
    \label{tab:operador}
  \end{table}
  Buscamos entonces constantes \(\alpha, \beta, \gamma\)
  tales que:
  \begin{equation*}
    \alpha G(1) + \beta G(n) + \gamma G(n^2)
      = - n^4 - 3 n^2 + 5
  \end{equation*}
  Resultan \(\alpha = 5\), \(\beta = 3\), \(\gamma = 1\).
  Tenemos la suerte que \(5 + 3 n + n^2\) cumple las condiciones iniciales,
  y la respuesta es:
  \begin{equation*}
    a_n
      = n^2 + 3 n + 5
  \end{equation*}

\section{Recurrencia lineal de primer orden}
\label{sec:lineal-recurrence-1st}

  Un caso especial importante es la recurrencia lineal de primer orden:
  \begin{equation*}
    a_{n + 1}
      = u_n a_n + f_n
  \end{equation*}
  Vemos que si dividimos por el \emph{factor sumador}:
  \begin{equation*}
    s_n
      = \prod_{0 \le k \le n} u_k
  \end{equation*}
  (lo que presupone que \(u_k \ne 0\) en el rango de interés)
  queda:
  \begin{equation*}
    \frac{a_{n + 1}}{s_n} - \frac{a_n}{s_{n - 1}}
      = \frac{f_n}{s_n}
  \end{equation*}
  Sumando ambos lados obtenemos la solución.

\section{Transformaciones}
\label{sec:transformaciones}

  En algunos casos las otras técnicas no pueden aplicarse directamente,
  pero una transformación de la recurrencia la lleva a una forma familiar.
  Las más importantes son las siguientes:
  \begin{description}
  \item[Transformación de dominio:]
    Defina una nueva función \(S(n) = T(f(n))\) para alguna función \(f\)
    que dé una recurrencia simple para \(S\).

    Por ejemplo,
    para mergesort si \(M(n)\) cuenta el número de comparaciones
    la recurrencia exacta es:
    \begin{equation*}
      M(n)
        = M(\lfloor n / 2 \rfloor) + M(\lceil n / 2 \rceil) + n - 1
        \quad M(0) = 0
    \end{equation*}
    El cambio de variables \(n = 2^k\), \(m(k) = M(2^k)\)
    y la observación \(m(0) = M(1) = 0\)
    lleva esto a la forma manejable
    de una recurrencia lineal de primer orden::
    \begin{equation*}
      m(k)
        = 2 m(k - 1) + 2^k - 1
        \quad m(0) = 0
    \end{equation*}
  \item[Transformación de rango:]
    Defina una nueva función \(S(n) = f(T(n))\) para alguna función \(f\)
    que dé una recurrencia simple para \(S\).

    Un ejemplo es la solución de Brand~%
      \cite{brand55:_seq_def_difference_eq}
    de la recurrencia de Ricatti:
    \begin{equation*}
      w_{n + 1}
        = \frac{a w_n + b}{c w_n + d}
        \quad w_0 \text{\ dado}
    \end{equation*}
    donde \(c \ne 0\)
    y \(a d - b c \ne 0\)
    (si \(c = 0\) es una recurrencia lineal de primer orden;
    si \(a d = b c\) se reduce a \(w_{n + 1} = \text{constante}\)).
    Si substituimos \(y_n \mapsto c w_n + d\),
    queda:
    \begin{equation*}
      y_n
        = \alpha - \frac{\beta}{y_{n - 1}}
    \end{equation*}
    donde:
    \begin{align*}
      \alpha
        &= a + d \\
      \beta
        &= a d - b c
    \end{align*}
    Claramente eso solo vale si \(a d - b c \ne 0\).
    Substituyendo ahora \(y_n \mapsto x_{n + 1} / x_n\)
    resulta:
    \begin{equation*}
      x_{n + 2} - \alpha x_{n + 1} + \beta x_n
        = 0
    \end{equation*}
    Necesitamos dos valores iniciales para resolverla,
    podemos elegir bastante arbitrariamente \(x_0 = 1\),
    dando \(x_1 = y_0\),
    que a su vez podemos obtener de la condición inicial original.
  \item[Diferencias:]
    Si aparecen sumas en los casos recursivos
    puede ser útil aplicar diferencias
    y buscar una recurrencia para \(S(n) = \Delta T(n) = T(n + 1) - T(n)\).

    Hay otras situaciones en las que diferencias ayudan a simplificar
    la recurrencia.
    Un ejemplo es la solución exacta a la recurrencia de mergesort
    discutida en el capítulo~\ref{cha:dividir-conquistar-solucion}.
  \end{description}

\section{Funciones generatrices}
\label{sec:recurrencias-funciones-generatrices}

  Podemos usar las herramientas de funciones generatrices.
  Dada la recurrencia,
  definimos una función generatriz adecuada,
  y aplicamos las propiedades respectivas para obtener una ecuación
  de la recurrencia.
  Resuelta la ecuación,
  extraemos coeficientes para obtener la solución.

  Por ejemplo,
  los números de Catalan \(C_n\) cumplen la recurrencia:
  \begin{equation*}
    C_{n + 1}
      = \sum_{0 \le k \le n} C_k C_{n - k}
      \quad C_0 = 1
  \end{equation*}
  Definiendo la función generatriz ordinaria:
  \begin{equation*}
    c(z)
      = \sum_{n \ge 0} C_n z^n
  \end{equation*}
  de las propiedades vemos que cumple:
  \begin{equation*}
    \frac{c(z) - C_0}{z}
      = c^2(z)
  \end{equation*}
  Despejando queda:
  \begin{equation*}
    z c^2(z) - c(z) + 1
      = 0
  \end{equation*}
  Obtenemos:
  \begin{equation*}
    c(z)
      = \frac{1 \pm \sqrt{1 - 4 z}}{2 z}
  \end{equation*}
  Sabemos que debe ser \(c(0) = C_0 = 1\),
  lo que descarta el signo positivo:
  \begin{equation*}
    c(z)
      = \frac{1 - \sqrt{1 - 4 z}}{2 z}
  \end{equation*}
  Usando la expansión de\((1 - 4 z)^{1/2}\)
  por el teorema del binomio y simplificando
  se obtiene finalmente:
  \begin{equation*}
    C_n
      = \frac{1}{n + 1} \binom{2 n}{n}
  \end{equation*}
  Alternativamente,
  podemos definir:
  \begin{equation*}
    u(z)
      = c(z) - 1
  \end{equation*}
  con lo que la ecuación toma la forma:
  \begin{equation*}
    u(z)
      = z (u(z) + 1)^2
  \end{equation*}
  Usando la fórmula de inversión de Lagrange con \(\phi(u) = (u + 1)^2\)
  y \(f(u) = u\) se obtiene directamente para \(n \ge 1\):
  \begin{align*}
    C_n
      &= [z^n] u(z) \\
      &= \frac{1}{n} [u^{n - 1}] (1 + u)^{2 n} \\
      &= \frac{1}{n} \binom{2 n}{n - 1} \\
      &= \frac{1}{n + 1} \binom{2 n}{n}
  \end{align*}
  Casualmente coincide \(C_0 = 1\) con esta fórmula.

  El número \(D_n\) de desarreglos
  (permutaciones sin puntos fijos)
  de \(n\) elementos cumple la recurrencia:
  \begin{equation*}
    D_{n + 1}
      = n D_n + n D_{n - 1}
      \quad D_0 = 1, D_1 = 0
  \end{equation*}
  Definimos la función generatriz exponencial:
  \begin{equation*}
    \widehat{d}(z)
      = \sum_{n \ge 0} D_n \frac{z^n}{n!}
  \end{equation*}
  Las propiedades dan:
  \begin{align*}
    \widehat{d}'(z)
      &\egf \left\langle D_{n + 1} \right\rangle_{n \ge 0} \\
    z \widehat{d}'(z)
      &\egf \left\langle n D_n \right\rangle_{n \ge 0}
  \end{align*}
  Falta el segundo término del lado derecho:
  \begin{equation*}
    \sum_{n \ge 1} n D_{n - 1} \, \frac{z^n}{n!}
      = z \sum_{n \ge 1} D_{n - 1} \, \frac{z^{n - 1}}{(n - 1)!}
      = z \widehat{d}(z)
  \end{equation*}
    Combinando las anteriores:
  \begin{align*}
    \widehat{d}'(z)
      &= z \widehat{d}'(z) + z \widehat{d}(z)
           \qquad \widehat{d}(0) = D_0 = 1 \\
    \frac{\widehat{d}'(z)}{\widehat{d}(z)}
      &= \frac{z}{1 - z}
  \end{align*}
  La solución de esta ecuación diferencial es:
  \begin{equation*}
    \widehat{d}(z)
      = \frac{\mathrm{e}^{-z}}{1 - z}
  \end{equation*}
  Por las propiedades,
  reconocemos la función generatriz ordinaria
  de las sumas parciales de la serie exponencial:
  \begin{align*}
    D_n
      &= n! [z^n] \widehat{d}(z) \\
      &= n! \sum_{0 \le k \le n} \frac{(-1)^k}{k!}
  \end{align*}

\section{Perturbación}
\label{sec:perturbacion}

  Una técnica aplicable para aproximar o acotar
  la solución a algunas recurrencias es el método de perturbación,
  que ilustramos mediante un ejemplo tomado de Sedgewick y Flajolet~%
    \cite{sedgewick13:_introd_anal_algor}.
  Sea la recurrencia:
  \begin{equation*}
    a_{n + 1}
      = 2 a_n + \frac{a_{n - 1}}{n^2}
      \quad a_0 = 1, a_1 = 2
  \end{equation*}
  Para valores grandes de \(n\)
  el último término tendrá poco efecto.
  Consideremos entonces la recurrencia aproximada:
  \begin{equation*}
    b_{n + 1}
      = 2 b_n
      \quad b_0 = 1
  \end{equation*}
  que tiene la solución obvia \(b_n = 2^n\),
  y comparemos los valores aproximados y exactos:
  \begin{equation*}
    p_n
      = \frac{a_n}{b_n}
      = 2^{-n} a_n
  \end{equation*}
  Substituyendo en la recurrencia exacta:
  \begin{equation*}
    2^{n + 1} p_{n + 1}
      = 2 \cdot 2^n p_n + \frac{2^{n - 1} p_{n - 1}}{n^2}
  \end{equation*}
  Esto se simplifica a:
  \begin{equation*}
    p_{n + 1}
      = p_n + \frac{p_{n - 1}}{4 n^2}
      \quad p_0 = p_1 = 1
  \end{equation*}
  Es claro que los \(p_n\) aumentan,
  por lo que:
  \begin{align*}
    p_{n + 1}
      &\le p_n \left( 1 + \frac{1}{4 n^2} \right) \\
      &\le \prod_{1 \le k \le n} \left( 1 + \frac{1}{4 k^2} \right) \\
      &<   \prod_{k \ge 1} \left( 1 + \frac{1}{4 k^2} \right)
  \end{align*}
  Este producto infinito
  puede expresarse usando la fórmula de producto de Euler para el seno:
  \begin{equation*}
    \sin \pi z
      = \pi z \prod_{k \ge 1} \left( 1 - \frac{z^2}{k^2} \right)
  \end{equation*}
  Con la fórmula para el seno en términos de exponenciales:
  \begin{equation*}
    \sin z
      = \frac{\mathrm{e^{\mathrm{i} z}} - \mathrm{e^{- \mathit{i} z}}}
             {2 \mathrm{i}}
  \end{equation*}
  vemos que nuestro producto es:
  \begin{align*}
    \prod_{k \ge 1} \left( 1 + \frac{1}{4 k^2} \right)
      &= \frac{\sin\left( \frac{\pi \mathrm{i}}{2} \right)}
              {\frac{\pi \mathrm{i}}{2}} \\
      &= \frac{\mathrm{e}^{\pi / 2} - \mathrm{e}^{- \pi / 2}}{\pi} \\
      &= 1,46505
  \end{align*}
  con lo que obtenemos:
  \begin{align*}
    a_n
      \sim \alpha 2^n
  \end{align*}
  donde sabemos que \(1 < \alpha < 1,46505\).

\section{Asintóticas}
\label{sec:asymptotics-recurrence}

  Como ejemplo de una técnica útil para obtener una estimación asintótica
  de la solución de una recurrencia,
  partiremos con la recurrencia planteada por Narayana Pandita,
  un matemático indio del siglo XIV.
  Plantea que al inicio tiene un ternero,
  que a los tres años se transforma en vaca.
  Cada vaca anualmente produce un nuevo ternero,
  y pregunta cuántos animales tiene al año \num{17}.
  Esto lleva a la recurrencia:
  \begin{equation}
    \label{eq:Narayana-cows-recurrence}
    a_{n + 3}
      = a_{n + 2} + a_n
      \qquad a_0 = a_1 = a_2 = 1
  \end{equation}
  La danza ritual da la función generatriz:
  \begin{align}
    A(z)
      &= \sum_{n \ge 0} a_n z^n
            \notag \\
      &= \frac{1}{1 - z - z^3}
            \label{eq:Narayana-cows-gf}
  \end{align}
  Lamentablemente,
  los ceros del polinomio característico \(p(\rho) = \rho^3 - \rho^2 - 1\)
  son expresiones muy complicadas,
  lo que se entorpece nuestra técnica de dividir en fracciones parciales
  y leer los coeficientes de allí.
  Tomaremos un camino indirecto.

  Primeramente,
  por la regla de signos de Descartes
  (ver por ejemplo~%
      \cite{levin20:_descartes_rule_signs})
  hay un solo cero real positivo
  (hay un único cambio de signo en la secuencia de coeficientes de \(p\))
  y ninguno negativo
  (\(p(-\rho)\) no tiene cambios de signo en su secuencia de coeficientes).
  Hay dos ceros complejos conjugados.
  Por las fórmulas de Vieta
  sabemos que el producto de los ceros es \(-1\)
  con lo que si \(\rho\) es el cero positivo,
  los ceros complejos tienen magnitud \(\rho^{-1/2}\).
  Una rápida aplicación del método de Newton
  (ver capítulo~\ref{cha:ceros-funciones})
  nos dice que \(\rho = 1,465571\),
  con lo que los ceros complejos tienen magnitud \num{0,8260314}.
  La solución será dominada por el cero real.

  De la expansión en fracciones parciales:
  \begin{align*}
    F(z)
      &= \frac{A}{1 - \rho z}
          + \frac{B_+}{1 - c_+ z} + \frac{B_-}{1 - c_- z} \\
    A
      &= \lim_{z \to 1 / \rho} F(z) (1 - \rho z)
  \end{align*}
  Podemos usar l'Hôpital para calcular el límite
  (por construcción,
   es de la forma \(0 / 0\)):
  \begin{align*}
    A
      &= \lim_{z \to 1 / \rho}	\frac{1 - \rho z}{1 - z - z^3} \\
      &= \lim_{z \to 1 / \rho} \frac{- \rho}{-1 - 3 z^2} \\
      &= \frac{\rho^3}{\rho^2 + 3} \\
      &= 0,6114918
  \end{align*}
  Sabemos entonces que:
  \begin{equation*}
    a_n
      = \frac{\rho^3}{\rho^2 + 3} \cdot \rho^n
        + O(\rho^{-n/2}) \\
  \end{equation*}
  Como una rápida confirmación,
  calculamos \(a_{10} = 28\),
  la fórmula aproximada da \num{27,95}.
  Resulta que aproximar al entero más cercano
  ya da el valor correcto para \(n = 0\).

  Un programita nos indica que \(a_{17} = 406\),
  nuestra aproximación es \num{405,98}.

%%% Local Variables:
%%% mode: latex
%%% TeX-master: "../INF-221_notas"
%%% ispell-local-dictionary: "spanish"
%%% End:

% LocalWords:  XIV programita


\bibliography{../referencias}

%%% Local Variables:
%%% mode: latex
%%% TeX-master: "../INF-221_notas"
%%% ispell-local-dictionary: "spanish"
%%% End:

% LocalWords:  mergesort
